\documentclass{uicthesi}

\usepackage{booktabs} % For formal tables

\usepackage{framed}
\usepackage{hyperref}
\usepackage{balance}
\usepackage[dvips]{graphics,color}

\usepackage{epsfig}
\usepackage{color}
\usepackage{subfigure}
\usepackage{multirow,tabularx}
\usepackage{placeins}
%\usepackage{miniltx}
\usepackage{mathtools}
\usepackage{graphicx}
\usepackage{epstopdf}
\usepackage{bm}

\usepackage{csquotes}
\usepackage{array}
\usepackage[yyyymmdd,hhmmss]{datetime}
\usepackage[subject={Todo}]{pdfcomment}
\usepackage[textsize=scriptsize,bordercolor=black!20]{todonotes}
\usepackage{xcolor,colortbl}
\usepackage{amssymb}% http://ctan.org/pkg/amssymb
\usepackage{pifont}% http://ctan.org/pkg/pifont
\usepackage[algo2e,titlenumbered,ruled]{algorithm2e} 
\usepackage{lipsum,environ,amsmath}
\usepackage{slashbox,booktabs,amsmath}
%\usepackage{todonotes}
\usepackage{cite}


\usepackage{dsfont}
\usepackage{amsfonts}
\usepackage{comment}

\usepackage{xspace}


\hyphenation{PageRank Convex dictatorship Dic-ta-tor-ship}

\newcounter{Lcount}
\newcommand{\numsquishlist}{
   \begin{list}{\arabic{Lcount}. }
    { \usecounter{Lcount}
 \setlength{\itemsep}{-.1ex}      \setlength{\parsep}{0ex}
      \setlength{\topsep}{0ex}       \setlength{\partopsep}{0ex}
      \setlength{\leftmargin}{1em} \setlength{\labelwidth}{1em}
      \setlength{\labelsep}{0.1em} } }
\newcommand{\numsquishend}{\end{list}}

\newcommand{\squishlist}{
   \begin{list}{$\bullet$}
    { \setlength{\itemsep}{-.1ex}      \setlength{\parsep}{0ex}
      \setlength{\topsep}{0ex}       \setlength{\partopsep}{0ex}
      \setlength{\leftmargin}{.8em} \setlength{\labelwidth}{1em}
      \setlength{\labelsep}{0.5em} } }
\newcommand{\squishend}{\end{list}}



\makeatletter
\DeclareOldFontCommand{\rm}{\normalfont\rmfamily}{\mathrm}
\DeclareOldFontCommand{\sf}{\normalfont\sffamily}{\mathsf}
\DeclareOldFontCommand{\tt}{\normalfont\ttfamily}{\mathtt}
\DeclareOldFontCommand{\bf}{\normalfont\bfseries}{\mathbf}
\DeclareOldFontCommand{\it}{\normalfont\itshape}{\mathit}
\DeclareOldFontCommand{\sl}{\normalfont\slshape}{\@nomath\sl}
\DeclareOldFontCommand{\sc}{\normalfont\scshape}{\@nomath\sc}
\makeatother


\newcounter{problem}
\newenvironment{problem}[1][htb]
  {\renewcommand{\algorithmcfname}{Problem}% Update algorithm name
   \begin{algorithm2e}[#1]%
   \SetAlFnt{\small}
    \SetAlCapFnt{\small}
    \SetAlCapNameFnt{\small}
    \SetAlCapHSkip{0pt}
  }{\end{algorithm2e}}
  
  \newenvironment{alprocedure}[1][htb]
  {\renewcommand{\algorithmcfname}{Algorithm}% Update algorithm name
   \begin{algorithm2e}[#1]%
    \SetAlFnt{\small}
\SetAlCapFnt{\small}
\SetAlCapNameFnt{\small}
\SetAlCapHSkip{0pt}
\IncMargin{-\parindent}
   
  }{\end{algorithm2e}}

% T: Addditions to the template
% For darkheaded arrows
\usepackage{wasysym}


\newtheorem{definition}{Definition}
\newtheorem{lemma}{Lemma}

% Shorthand for notations
\newcommand{\R}{\mathbb{R}}
\newcommand{\calX}{\mathcal{X}}
\newcommand{\calY}{\mathcal{Y}}
\newcommand{\calZ}{\mathcal{Z}}


\begin{document}

% Declarations for Front Matter

\title{Title of Thesis is here}
\author{Thiruvenkadam Sivaprakasam Radhakrishnan}
\pdegrees{}
\degree{Master's in Computer Science}
\committee{Prof. Ian Kash, Chair and Advisor \\ Prof. Anastasios Sidiropoulos  \\ Prof. Ugo Buy}
\maketitle


% \dedication
% {\null\vfil
% {\large
% \begin{center}
% To myself,\\\vspace{12pt}
% Perry H. Disdainful,\\\vspace{12pt}
% the only person worthy of my company.
% \end{center}}
% \vfil\null}


\acknowledgements
{The thesis has been completed... (INSERT YOUR TEXTS)\\

  \begin{flushright}YOUR INITIAL\end{flushright}}
% \acknowledgements
% {I want to ``thank'' my committee, without whose ridiculous demands, I
% would have graduated so, so, very much faster.}

% \preface
% This preface is purely optional at UIC.

\tableofcontents
\listoftables
\listoffigures
\listofabbreviations
\begin{list}
  {}
  {\setlength
    {\labelwidth}{1in}
    \setlength{\leftmargin}{1.5in}
    \setlength{\labelsep}{.5in}
    \setlength{\rightmargin}{\leftmargin}}
  \item[$\theta$\hfill] Parameters of a function approximator.
\end{list}

\summary
Put your summary of thesis here.

\chapter{Introduction}

In this work, we study two mirror-descent based reinforcement learning algorithms and propose novel
improvements to them.

\chapter{Background}
This work mainly discusses algorithms that are present in the intersection of three subject areas
namely, Reinforcement learning, Game Theory, and Online Learning.
We provide some brief background to relevant concepts required to follow the ideas discussed in the following sections 
and point to more comprehensive resources when applicable.

\section{Reinforcement
  Learning}

% What is it?
Reinforcement learning (RL) is sub-domain of machine learning that deals with designing interactive
agents that learn to maximize a reward signal in an environment.
The reward signal encodes information about the goal that the designer wants the agent to learn to
achieve without informing anything about how that goal should be achevied.
Reinforcement learning has been shown to be
effective in many application domains to learn and solve an arbitrary problem as long as it can be
encapsulated into an interactive environment with a well-defined reward signal.
\blue{ - cite RL applications}

% Motivation to study and design RL algorithms
\textbf{Markov Decision Process.} Reinforcement learning problems are formally modeled as Markov Decision Processes (MDPs), that represented 
as a 
tuple $(\calS, \calA, P, R, \gamma, \mu)$ where: $\calS$ is the state space, $\calA$ is the action space, 
$P(s'|s,a): \calS \times \calA \times \calS \mapsto [0, 1]$ is the transition or the dynamics function, 
$R(s,a) \subset \R$ is the reward function,  
$\gamma \in [0,1]$ is the discount factor and $\mu$ is the initial state distribution.

The objective is to maximize the expected discounted reward:
$$G_t = \sum_{k=0}^{\infty} \gamma^k R(s_{t+k}, a_{t+k})$$

In learning to take actions that maximize this objective, algorithms usually involve learning value functions 
and policies. A policy is a mapping from a state to an action distribution $\pi: \calS \mapsto \calA$, and a value 
function $V_{\pi}(s)$ estimates the expected reward that can be achieved from the current state by following a 
given policy: $V_{\pi}(s) = \E_{\pi}[G_t | S_t = s]$.

The objective can then be reformulated as to find a policy that maximizes the value of the initial state under 
the starting distribution $\mu$:
$$\max_\pi \E_{s_0 \sim \mu}[V_{\pi}(s_0)]$$

Action-value or Q-value  function estimates the expected reward of taking a specific action $a$ at a given 
state $s$ and then following the policy $\pi$: $ Q_{\pi}(s, a) = \E_{\pi}[G_t | S_t = s, A_t = a] $.
The difference between Q and V functions is referred to as the advantage function $A_\pi(s,a) = Q_\pi(s,a) - V_\pi(s)$. 
This is the advantage of taking a particular action over following the average policy.
The \textbf{optimal policy} $\pi^{\ast}$ is one that maximizes the value function. There maybe more than one optimal policy, but 
they all share the same value function $V_{\pi^{\ast}} = \max_{\pi} V_\pi(s), \forall s \in \calS$.

\textbf{Generalized Policy Iteration (GPI).} For small, finite state spaces optimal policies can be learnt through tabular methods and dynamic programming. 
Policy iteration is an iterative method that employs policy evaluation, and improvement steps to learn the optimal policy. 
The policy evaluation step learns the value function for the current policy using the following iterative update 
until it converges.
$$V_{k+1}(s) = \E_{\pi} [R_{t+1} + \gamma V_k(S_t+1 | S_t = s)]$$

Policy improvement step then uses the value function to greedily improve the current policy, $\pi_{k+1}(s) = \arg \max_a Q_{k}(s,a)$.
Policy improvement always yields a strictly better policy except when the policy is already optimal~\cite{suttonReinforcement2018}.
Exact convergence of policy evaluation might make the learning process slower. 
This step can be truncated to only evaluate the value for the immediate states (i.e. perform only one step of policy evaluation).
This is known as \textbf{value iteration}.
In \textit{Generalized policy iteration (GPI)}, these steps are run independently, and asynchronously until convergence.

For very large state spaces that is common in many practical applications, there is a need to learn approximate 
value functions and parameterized policies. This is typically done through function approximation. 

\blue{
	\begin{itemize}
		\item Bellman eqn
		\item 
	\end{itemize}
}

Fixed points of the Bellman quation are optimal policies and optimal value functions.
For small MDPs that strictly satisfy the Markov property, Bellman equiation can be explicity
solved.
In settings where the transitions and rewards can be represented explicitly in a tabular manner,
dynamic programming is one approach to solve the Bellman equation.
However, for most practical applications it is common to use parameterized policies and approximate
value functions.

In formulating Reinforcement learning algorithms, there are two major approaches, and their
derivatives.
% Most of the tabular, and iterative methods require an
% explicit model of an environment with access to the transition functions.
% In the absence of a model or transition functions, one can use Monet Carlo methods to approximate
% these probabilities through sampling.
% This method of learning from experience by iteracting with a blackbox enivornment or a simulator is
% the setting in most of the recent efforts in machine learning.
% This is because it is difficult to explictly design a model that accurately captures all the
% properties of a real-world artifact, and exhaustively model the effects of taking all the actions
% that are available to the agent.
% It is also sometimes impossible or unknown.
% However, it is possible to have access to such a real-world artifact, and we only need to provide
% an interface for the agent to the artifact and expose available actions to the agent.
% We can use the real-world artifact to evaluate these actions and their effects and convert them
% into rewards to provide a signal for the agent to learn from.

\label{sec:spg} \subsection{Policy gradient methods}\label{sec:pg}

Policy gradient method involve directly learning a parameterized policy that enables action selection 
without the use of a value function.
These are generally called Policy gradient methods, and are a major area of study within Reinforcement learning.
Policy gradient methods sometimes have the advantage that the policy space could be simpler to
learn compared to the value function space.
In this case, the policy is parameterized by $\theta \in \Theta$ and $\pi(s, a) = P(s, a; \theta)$.

Given some objective $J(\theta)$, these methods seek to learn the parameters that maximize this objective through 
gradient ascent.


One key challenge in Policy gradient methods is that the evaluation of performance of a policy
depends on the state distribution which could be unknown.
However, the Policy Gradient Theorem establishes that the gradient is independent of underlying
environment's state distribution as long as it is stationary conditioned on the current policy.

The most popular policy gradient method is Reinforce, which is a Monte Carlo policy gradient
method.

A value function may still be used in guiding the policy learning, and these are called actor-critic methods.
Here the actor refers to the policy, and the critic refers to the value function that evaluates the
action taking by the policy guiding the policy learning.

\textbf{Reinforce}: 
\blue{
	- Widely popular PG algorithm, many insantiations possible including variants with baselines to
	reduce variance.
	- Convergence is proven using PG theorem.
}

\textbf{Softmax Policy Gradients}
\blue{
	- Parametrizing policies
	- PG with softmax parameterization is referred to as Softmax Policy gradients.
	- General significance of softmax
}


\subsection{PPO}

- Trust region methods
- PPO an approximation of TRPO with hueristic objective

\subsection{Multi-agent Reinforcement Learning (MARL)}\label{sec:marl}

There are a few key challenges in extending Reinforcement learning algorithms to multi-agent
settings.
From a theoretical perspective, many of the function approximation based algorithms assume that the
state distribution $\mu(s)$ remains stationary for a given policy.
However, in multi-agent settings the state distribution can become non-stationary due to the
changes in the behavior of the other agents.
This makes it difficult in adapting the algorithms disucssed above directly to multi-agent
settings.

From an algorithm design perspective, the action space explodes exponentially in multi-agent
settings making it computationally challenging to apply reinforcement algorithms without
decomposing the problem into more managable sub-problems first.

\section{Game Theory}

Game theory is the mathematical study of interaction between agents to produce outcomes while
trying to uphold certain individual or group preferences.
It has a wide range of applications including economics, biology, and computer science.
In overcoming the challenges mentioned above, many algorithms have adopted game-theoretic
constrtucts and ideas when designing Reinforcement learning algorithms for multi-agent settings.
It is also a common practice to use game theoretic constructs in evaluating the performance of
multi-agent algorithms and provide theoretical guarantees.
In this work, we mainly focus on a branch of game theory called non-cooperative game theory in
which each agent has their own indivudual preference.

\subsection{Problem Representations}
\label{subsec:reps}

In discussing agent interactions, and preferences, we need a formal notion of how agents act, and
how agent preferences can be defined.
In game theory, agent preferences are formalized using Utility theory, where each agent has a
utility function that maps the agents preferences over outcomes to a real value.

The primary way of modeling problems in game theory is through a \textit{game} that encodes
information about the agents, possible actions agents can take in different situations, their
preferences, and the outcome of an interaction.
There are many types of such representations, a few relevant of which we introduce below.
Before we introduce such representations, we first cover some preliminary concepts that will be
helpful in formally defining those representations and agent preferences.

- what are utilities, and strategies?

\textbf{Normal-Form Games:} Normal-Form games are a popular way of representing situations in which all agents act
simultaneously and the outcome is revealed after each agent has taken their action.
A few popular games that can be represented in this form are rock-paper-scissors, matching pennies,
prisoner's dilemma etc. A more formal definition of a normal-form game is as follows:

\begin{definition}[Normal-form games]

	A $(N, A, u)$ tuple
	is a n-player normal form game, where $N$ is the set of players, $A = A_1 \times A_2 \ldots \times
		A_n$, with $A_i$ being the set of actions available to player $i$, and $u = (u_i \forall i \in N)$
	is the set of utility functions that map an action profile to a real utility value for each agent,
	$u_i: A \mapsto \R$.

\end{definition}

Normal-form games are typically represented using a n-dimensional tensor, where each dimension
represents the possible actions available to each agent, and every entry represents an outcome.
The actual entries of the

An NFG that is widely popular in the literature is the \textbf{Biased/Perturbed RPS} problem. \blue{add details.}

\subsubsection*{Sequential Games}

Although normal-form games provide a neat representation, many real-world scenarios necessitate
agents act sequentially which is difficult to represent as a matrix.
These problems require a tree-like representation where each node is an agent's turn to make a
choice, and each edge is a possible action.
There are a few ways to represent such scenarios, one being normal-form games themselves.
A downside is that the size of the normal-form representation for sequential games explode
exponentially in the size of the game tree.
Other possible representations include the Extensive-form, and Sequence-form.

\begin{definition}[Extensive-form games]
\end{definition}

\begin{definition}[Sequence form games]
\end{definition}

\subsection{Solution Concepts}

Now that we have - what are solution concepts?
- what are the common solution concepts?
Nash equilibrium, Quantal response equilibrium.
- what are the relevant information related to solution conepts for this work?
Existence of a nash equilibrium Uniqueness of QRE

\section{Online Learning and Mirror Descent}

Online learning is the study of designing
algorithms that use historical knowledge in making predictions for future rounds while trying to
minimize some loss function in an adaptive (possibly adversarial) setting.

% Why is it relevant?
The problem of training agents in a potentially non-stationary setting can be cast into an online
learning problem.
Hence, it is useful to study online learning algorithms from the perspective of designing
reinforcement learning algorithms as they provide a framework for the analysis of the algorithms
and deriving theoretical guarantees.
% In this work, we also study algorithms that can be derived from an online-learning perspective, and
% also can be shown equivalent to popular online learning algorithms.
The brief background provided in this section closely follows the details as presented
in~\cite{shalev-shwartzOnline2012}.
For a more in-depth introduction into Online learning and Online Convex Optimization, please refer
to the above work.

% Definition
In Online Learning, a learner is tasked with predicting the answer to a set of questions over a
sequence of consecutive rounds.
We now define an Online learning problem more formally as follows:

\begin{definition}
	\label{def:olearning} For each round $t$, given an instance $x_t \in \calX$, and
	a prediction $p_t \in \calY$, a loss function $l(p_t, y_t) \mapsto \R$
\end{definition}

At each round t, a question $x_t$ is taken from an instance domain $\calX$, and
the learner is required to predict an answer, $p_t$ to this question.
After the prediction is made, the correct answer $y_t$, from a target domain $\calY$ is revealed
and the learner suffers a loss $l(p_t, y_t)$.
The prediction $p_t$ could belong to $\calY$ or a larger set, $\mathcal{D}$.

% Assumptions around the problem definition
% It is known that without any further assumptions about the learning problem, a learner's regret can
% be unbounded in an adversarial setting~\cite{coverBehavior1965}.

% A popular assumption in Online learning settings is called Realizability, where we assume the
% target mapping arises from a fixed hypothesis class $\calH$.
The main aim of an online learning algorithm $A$, is to minimize the cumulative regret of a learner
with respect to the best competing hypothesis $h^*$ from the assumed hypothesis class $\calH$.

\begin{equation}
	\label{eqn:regret}
	Regret_T(h^*) = \sum_{t=1}^T l(p_t, y_t) - \sum_{t=1}^T l(h^*(x_t), y_t),
\end{equation}

The regret of $A$ with $\calH$ is,

\begin{equation}
	\label{eqn:regret_all} Regret_T(\calH) = \max_{h^* \in \calH} Regret_T(h^*)
\end{equation}

A popular framework for studying and designing Online learning algorithms is
through Convex optimization.
The assumptions made around the framework provides properties that are useful in deriving
convergence guarantees.
We define a few terms below that are used in this section and the following ones.

% Convex sets and Convex functions

\begin{definition}[Strongly Smooth]~\label{def:strsmooth}
	Given a convex set $\calX \in \R^n$, a convex function $f: \calX \mapsto \R$ is $\sigma$-strongly
	smooth with respect to a norm $\|.
		\|$, if $\|\nabla f(x) - \nabla f(y) \|_{\ast} \leq \sigma \|x - y \|, \forall x, y \in \calX$.
	For a given constant $L$, this is also referred to as $L-smooth$.
\end{definition}

% Strongly smooth/L-smooth

% Monotonicity/Strong monotonicity

The typical structure of online learning expressed as an online convex optimization problem is as
follows:

\begin{alprocedure}[H] \algorithmcfname{OCO}~\label{alg:oco}

	input: a convex set S for t = 1, 2, 
	$\ldots$

	predict a vector $w_t \in S$

	receive a convex loss function $f_t: S \mapsto \mathbb{R}$
\end{alprocedure}

Reframing~\ref{eqn:regret} in terms of convex optimization, we refer to a competing hypothesis here
as some vector $u$ from the convex set $S$.

\begin{equation}
	Regret_T(u) = \sum_{t=1}^T f_t(w_t) - \sum_{t=1}^T f_t(u)
\end{equation}

and
similarly, the regret with respect to a set of competing vectors $U$ is, \begin{equation}
	Regret_T(U) = \max_{u \in U} Regret_T(u)
\end{equation}

The set $U$ can be same
as $S$ or different in other cases.
Here we assume $U=S$ and $S=\mathbb{R^d}$ unless specified otherwise.

\subsection{FoReL}\label{sec:forel}

Follow-the-Regularized-leader (FoReL) is a classic online learning algorithm that acts as a base in
deriving various regret mimization algorithms.
The idea of FoReL is to include a regularization term to stabilize the updates in each iteration
leading to better convergence behaviors.

The learning rule can be written as,

$$\forall t, w_t = argmin_{w \in S}
	\sum_{i=1}^{t-1} f_i(w) + R(w).
$$

where $R(w)$ is the regularization term.
The choice of the regularization function lead to different algorithms with varying regret bounds.

\subsection{Gradient Descent}

In the case of linear loss functions with respect to some $z_t$, i.e., $f_t(w) = \langle w, z_t
	\rangle$, and $S=\mathbb{R}^d$, if FoReL is run with $l_2$-norm regularization $R(w) = \frac{1}{2
		\eta} \|w\|_2^2$, then the learning rule can be written as,

\begin{equation}
	w_{t+1} = -\eta \sum_{i=1}^t z_i = w_t - \eta z_t
\end{equation}

Since, $\nabla
	f_t(w_t) = z_t$, this can also be written as, $w_{t+1} = w_t - \eta \nabla f_t(w_t)$.
This update rule is also commonly known as Online Gradient Descent.
The regret of FoReL run on Online linear optimization with a euclidean-norm regularizer is:

$$Regret_T(U) \leq BL \sqrt {2T}.
$$

where $U = {u : \|u\| \leq B}$ and $\frac{1}{T} \sum_{t=1}^T \|z_t\|_2^2 \leq L^2$ with $\eta = \frac{B}{L\sqrt{2T}}$.

Beyond Euclidean regularization, FoReL can also be run with other regularization functions and
yield similar regret bounds given that the regularization functions are strongly convex.

\begin{definition}
	For any $\sigma$-strongly-convex function $f: S \mapsto \mathbb{R}$ with respect to a norm $\|.
		\|$, for any $w \in S$,
	\begin{equation}
		\forall z \in \partial f(w), \forall u \in S, f(u) \geq f(w) + \langle z, u - w\rangle + \frac{\sigma}{2}\| u - w \|^2.
	\end{equation}
\end{definition}

\begin{lemma}
	\label{lem:forelrb}
	For a FoReL algorithm producing a sequence of vectors $w_1, \ldots, w_T$ with a sequence of loss
	functions $f_1, \ldots, f_T$, for all $u \in S$, $$\sum_{t=1}^T (f_t(w_t) - f_t(u)) \leq R(u) -
		R(w_1) + \sum_{t=1}^T (f_t(w_t) - f_t(w_{t+1}))$$
\end{lemma}

\subsection{Hedge} - what is hedge?

Gradient descent as a FTRL variant


\section{Mirror Descent}

In this section we discuss about Mirror Descent, and two mirror descent-based reinforcement
learning algorithms (MDPO, and MMD).
Mirror descent is a popular first order optimization algorithm that has seen wide applications in
machine learning, and reinforcement learning.

There are different views of arriving at Mirror Descent as an optimization algorithm, here we
present the Mirror Descent framework through the lens of mirror maps.

\subsection{Mirror Maps}

% Define Bregman divergence
\begin{definition}
	\label{def:bregman}
	Given a convex set $\calX \subset \R^n$, and a differentiable convex function $f: \calX \mapsto
		\R$, the Bregman Divergence associated with the function $f$ is defined as,

	$$
		D_f(x, y) = f(x) - f(y) - \nabla f(y)^T (x-y).
	$$
\end{definition}

For an alternate view of deriving Mirror Descent as an improvement of FoReL, please refer
to~\cite{shalev-shwartzOnline2012}[Section 2.6].
% A disadvantage of FoReL~\ref{sec:forel} in solving online learning problems is that, there is a
% minimization that must be performed at every step.
% Mirror descent overcomes this by using a recursive update rule that does not required us to perform
% a minimization at every step.
\chapter{Mirror Descent in Reinforcement Learning}

Given the generality of the mirror descent algorithm as a first order optimization method, there
has been continued efforts to incorporate it as a single/multi-agent reinforcement learning
algorithm.
Extensive studies of mirror descent under various settings and assumptions help in deriving much
needed theoretical guarantees for mirror descent based reinforcement learning algorithms in terms
of sample complexity, and convergence rates.
In this work, we study two such algorithms, namely MDPO (Mirror Descent Policy Optimization), and
MMD (Magnetic Mirror Descent).
We first begin with a discussion of Softmax Policy Gradients to set a base framework for the rest
of the chapter before moving on to the above algorithms.
% For the following discussion, and the experiments, we consider a class of parameterized stochastic policies,
% i.e., $\Pi = \{\pi_{\theta}, \theta \in \Theta\}$.
% We also only consider on-policy learning.

\section[SPG]{Softmax Policy Gradients}
\label{sec:spg}
In case of a parameterized policy $\pi_\theta$, where $\theta \in \Theta$ represent the policy
parameter space, the aim of policy gradient methods is to maximize some objective $J(\theta)$.
Similar to the value-based approximation methods, the policy can also be directly learnt using
gradient ascent.
A prototypical performance measure is simply the value of the initial state under the current
policy: $J(\theta) = V_{\pi_\theta}(s_0)$.
The Policy Gradient Theorem~\cite[Chapter 13.2]{suttonReinforcement2018} establishes that the
gradient of this objective function can be estimated without knowledge of the environment's state
distribution as long as it is stationary conditioned on the current policy.
% One key challenge in Policy gradient methods is that the evaluation of performance of a policy% depends on the state distribution which could be unknown.

$$ \nabla_\theta J(\theta) \propto \sum_s \mu(s) \sum_a q_\pi(s,a) \nabla_\theta \pi_theta(a|s) $$

\textbf{Softmax Policy Gradients}
For discrete action settings, it is common to use a parameterized function $y_\theta(s, a)$ to
represent action preferences or \textit{logits}, and the policy is then extracted using a softmax
operator on top: $\pi(a|s, \theta) \doteq \frac{e^{y_\theta(s, a)}}{\sum_b e^{y_\theta(s,b)}}$.
Softmax parameterization is the most popular form of policy representation in RL under function
approximation settings.
Policy gradient method with a softmax parameterization is typically referred to as Softmax Policy
Gradients (SPG).

\textbf{Reinforce}:
The most fundamental policy gradient method \textit{REINFORCE}, uses monte-carlo estimations to approximate
the performance measure.
\begin{equation}
	\label{eqn:reinforce}
	\nabla_\theta J(\theta_t)_{|\theta=\theta_t} \propto \E_\pi \big[ G_t \nabla_\theta \log \pi_{\theta_t}(S_t, A_t) \big]
\end{equation}
Although Monte-carlo estimates of the returns are unbiased, they can be of high variance.
A baseline that is independent of the action can be used to reduce this variance.
A popular choice for such a baseline an approximate value function such as the one from value-based
approximation methods ($V(s;w)$).

\textbf{Actor-Critic Methods:}
Apart from using this approximate value function as a baseline, we can also use them to better
estimate the peformance measure used in the objective.
This leads to \textit{actor-critic} methods, that learn a parameterized policy (called the actor),
guided by an approximate value function is referred to as the critic.

\subsection{Trust-region methods}

As discussed in the previous section, Policy gradient methods typically use the following objective
to perform gradient updates.
Kakade et.~al~\cite{kakadeApproximately2002a} note that this gradient update is heavily down
weighted for unlikely states and actions, requiring very high sample complexity to make policy
improvements for these states.
% However, vanilla Policy gradient method cannot reuse data collected for more than one update, and
% hence requires a higher sample complexity to learn a good policy. 
Also, it is not clear how to decide on the stepsize $\alpha$, as too large of a step can move the
policy away from a good space, and too low of a step size can take a long time to converge.
They instead propose Conservative Policy Iteration (CPI), to instead optimize the objective with
respect to an alternate objective measure that guarantees improvement at every state when learning
from an on-policy distribution.

Trust region policy optimization (TRPO)~\cite{schulmanTrust2015} is a policy gradient method that
expands on the idea of CPI using constrained policy updates instead of using mixture policies.
TRPO guarantees monotonic improvement with every update and can be extended to function
approximation settings.
The TRPO algorithm uses a conjugate gradient algorithm and performs a line search to update the
policy in a way that satisfies the KL-constraint.
Although TRPO shows good performance in complex continuous control tasks, it is difficult to adapt
this idea when working with RNNs, and parameter sharing.

Proximal Policy Optimiation (PPO)~\cite{schulmanProximal2017} instead approximates this
KL-constrained objective by using a hueristic objective and uses a simple first-order optimization
algorithm (SGD) instead of the conjugate gradient algorithm.
The simple nature of the clipped objective makes it easier to extend this to other settings like
RNNs, and parameter sharing where it was difficult to apply TRPO.

\section[MDPO]{Mirror Descent Policy Optimization}
 (\revdone{For trust region methods: This seems like it mostly belongs in 3.1.2 and then you need to rework this introduction;})

 (\revdone{For MDPO citation: Make sure you include the citation at least the first time you reference them by name so that the reader can get evrything synced up})

 (\revdone{for tabular mirror descent guarantees: Which are?
  need to establish some background either in 2 or here})

 (\revdone{On-policy only: Needs an explanation why}).

(\revdone{Needs founddations somewhere}).

Now, we move on to the first mirror descent-based algorithm, Mirror Descent Policy
Optimization~\cite{tomarMirror2022} (MDPO).
Although the trust-region methods discussed in the previous section approach the problem of
performing conservative policy updates, they can also be treated as instances of mirror descent.
Shani~et.~al~\cite{shaniAdaptive2020} show sample-based TRPO as an instance of Mirror Descent with
negative entropy regularization along with an adaptive scaling for the proximal regularization.
A similar study of trust-region methods in the case of policies represented using neural networks
was done in~\cite{liuNeural2019}.

Following these theoretical results, Tomar et.~al~\cite{tomarMirror2022} present MDPO as a
practical implementation of extending mirror descent as a deep reinforcement learning algorithm,
that takes a mirror descent step at each iteration to update the policy:
\begin{equation}
	\label{eqn:mdrl} \pi_{k+1}(.
	|s) \leftarrow \arg
	\max_{\pi \in \Pi}
	\E_{a \sim \pi} [A_{\pi_k}(s,a)] -
	\frac{1}{t_k}
	KL(s; \pi, \pi_k)
\end{equation} This form of update is analagous to the trust-region methods
discussed above, however is obtained from an optimization perspective instead of tackling the
problem of performing conservative policy updates.
For a parameterized policy $\pi_{\theta}$, the on-policy MDPO update is given by: $$ \theta_{k+1}
	\leftarrow \arg \max_{\theta} J(\theta, \theta_k)$$ $$ \text{where}, J(\theta, \theta_k) = \E_{s
		\sim \rho_{\theta_k}} [ \E_{a \sim \pi_\theta}[A_{\theta_k}(s,a)] - \frac{1}{t_k} KL(s; \pi_\theta,
		\pi_{\theta_k}]$$

Instead of solving this objective exactly at each iteration,
MDPO approximates it using stochastic gradient updates.
As noted in~\cite{tomarMirror2022}, the gradient of the KL component in~\ref{eqn:mdrl} for one step
is zero, and hence it is necessary to take multiple gradient steps every iteration to properly
approximate the mirror descent objective.
For $m$ steps MDPO uses the following gradient to update the policy parameters,
\begin{equation}
	\label{eqn:mdpograd} \nabla_{\theta} J(\theta, \theta_k)|_{\theta = \theta_k^{(i)}} = \E_{s \sim
		\rho_{\theta_k}, a \sim \theta_k} \left[ \frac{ \pi_{\theta_k}^{(i)}}{ \pi_{\theta_k}}
		\nabla_{\theta} \log \pi_{\theta_k}^{(i)} (a|s) A_{\theta_k}(s,a) \right] - \frac{1}{t_k} \E_{s
		\sim \rho_{\theta_k}} \left[\nabla_\theta KL(s; \pi_\theta, \pi_{\theta_k})_{|\theta=
				\theta_k^{(i)}} \right]
\end{equation} where $i=0,1,\ldots,m-1$.

Where, $\frac{1}{\pi_{\theta_k}}$ is the importance sampling factor that adjusts the updates for
the trajectories sampled using $\pi_{\theta_k}$.

MDPO also accomodates off-policy learning, and we refer the reader to~\cite{tomarMirror2022} for
details on the off-policy algorithm, and a discussion of MDPO's connection to popular RL
algorithms.
% \textbf{Gradient of KL-Divergence}% Let us consider a reference policy $Q$ and $P_{\theta}$, a policy that is being optimized. % The reverse KL-Divergence between the two is given by% $$% KL(P_{\theta} \| Q) = \sum_{a \in A} P_{\theta}(a) \log \frac{P_{\theta}(a)}{Q(a)}% $$% The gradient of KL-Divergence with respect to the parameters theta is given by:% \begin{equation*}% 	\begin{split}% \nabla_{\theta} KL &= \sum_{a \in A} \nabla_{\theta} [P_{\theta}(a) \log P_{\theta}(a)] - % 						\nabla_{\theta} [P_{\theta}(a) \log Q(a)] \\% 						&= \sum_{a \in A} [\nabla_{\theta} P_{\theta}(a) \log P_{\theta}(a) + \nabla_{\theta} P_{\theta}(a)] -  % 						 \nabla_{\theta} P_{\theta}(a) \log Q(a) \\% 						&= \sum_{a \in A} \nabla_{\theta} P_{\theta}(a) (\log P_{\theta}(a) - \log Q(a) + 1)% 	\end{split}% \end{equation*}
% MDPO has been shown to have better empirical performance than PPO, and TRPO in continuous control tasks, and Atari environments.

\subsection{Tabular MDPO}
In this section, we formulate a tabular version of MDPO with exact policy parameterization, where
the parameters are the logits, or action preferences associated with each action.
The parameters are converted to a distribution over the action space using a softmax operator.
We also assume that exact action-values are available for the entire action space at each state.
In normal-form games, these action-values correspond to the expected payoffs associated with each
action given a fixed opponent policy.
This is also referred to as the all-actions setting (also known as \textit{full-feedback} or
first-order information setting).
In this setting, the gradient computation in~\ref{eqn:mdpograd} becomes: \[ \nabla
	\theta_{|\theta=\theta_{k}^{(i)}} = \nabla_{\theta} \sum_{a \in A} [ \pi_{\theta_k}^{(i)} (a)
		A_{\theta_k}(a)] - \frac{1}{t_k} [\nabla_\theta KL(\pi_\theta, \pi_{\theta_k})] \]

We use this update for the tabular experiments and the original on-policy
formulation for the neural experiments.

\section[MMD]{Magnetic Mirror Descent}
 (\revdone{It isn't immediately clear where the entropy and regularization are in
	 (3.4)})
 (\revdone{For (3.7): This is an equation, not al algorithm, because it doesn't tell you how to compute the
	 minimumum})
 (\revdone{For discussion on MMD vs CFR performance: Cite?
	 Also, have you introduced CFR at some point?
 })

The second extension of mirror descent that we study is Magnetic Mirror Descent (MMD).
Sokota~et.~al~\cite{sokotaUnified2023}, study relation between equilibrium solving and Variational
Inequalities with composite structures.
They use this connection to propose strategy learning algorithms for two-player zero-sum games.
Moreover, the proposed algorithm also extends well as a reinforcement learning algorithm that is
performant in single, and multiagent settings.
In this section, first we outline the connection between Varational inequalities and equlibrium
solving as presented in~\cite{sokotaUnified2023}.
Then we outline the Magnetic Mirror Descent algorithm and its convergence properties.

\subsection{Connection between Variational Inequalities and QREs}
A Variational Inequality (VI) problem, written as $VI(Z, F)$ is generally defined as follows:
\begin{definition}
	\label{def:vi} Given $\calZ \subseteq \R^n$ and mapping $F: \calZ \rightarrow
		\R^n$, the variational inequality problem VI($\calZ, F$) is to find $z_{\ast} \in \calZ$ such that,
	\[ \langle F(z_{\ast}), z - z_{\ast} \rangle \geq 0 \quad \forall z \in \calZ.
	\]
\end{definition}

The VI problem described above is very general, and as such a wide range of problems can be cast
into this framework~\cite{facchineiFiniteDimensional2004}.
We mainly focus on the relation between VI problems with a similar structure, and QREs.

Finding the QRE of a two-player zero-sum game can be represented as the regularized saddle point
problem.
Given $\calX \subseteq \R^n$, $\calY \subseteq \R^m$, and $g_1: \R^n \mapsto \R$, $g_2: \R^m
	\mapsto \R$, find:
\begin{equation}
	\label{eqn:saddle} \min_{x \in \mathcal{X}} \max_{y \in
		\mathcal{Y}} \alpha g_1(x) + f(x, y) + \alpha g_2(y),
\end{equation} where $g_1$, and $g_2$, are
strongly-convex functions.
For a two-player zero-sum game, $f(x, y)$ represents the payoff matrix, and $g_1$, $g_2$ represent
entropy-regularization of the strategies of the two players.
The solution $(x_{\ast}, y_{\ast})$ to~ \ref{eqn:saddle} has the following first-order optimality
conditions:
\begin{equation}
	\label{eqn:optcon}
	\begin{split}
		\langle \alpha \nabla g_1(x_*) +
		\nabla_{x_*} f(x_*, y_*), x - x_* \rangle \geq 0, \forall x \in \calX.
		\\
		\langle \alpha \nabla g_2(y_*) +
		\nabla_{y_*} f(x_*, y_*),
		y - y_* \rangle \geq 0, \forall y \in \calY.
	\end{split}
\end{equation}

The optimality conditions~\ref{eqn:optcon} are equivalent to the optimality conditions of a
VI$(\calZ, G)$ with the following composite objective $G = F + \alpha \nabla g$, where $\calZ =
	\mathcal{X} \times \mathcal{Y}$, $F(z) = [\nabla_x f(x,y) - \nabla_y f(x,y)]$, and $\nabla g =
	[\nabla_y g_1(x), \nabla_x g_2(y)]$.
Hence, the solution to the VI: $z^* = (x^*, y^*)$ is also the solution to the saddle point problem
stated in~\ref{eqn:saddle}.
So, the reguarlized saddle point problem~\ref{eqn:saddle} of finding QREs can be cast as a VI
problem allowing us to tap into the vast array of existing techniques for solving Variational
inequalities.

\subsection{MMD Algorithm}
Various algorithms have been proposed to solve the VI problem~\ref{def:vi}.
In particular, the proximal point method has linear last iterate convergence for Variational
inequality problems with a strongly monotone operator~\cite{rockafellarMonotone1976}.
This algorithm was extended to composite objectives~\cite{tsenglinear1995}, and to non-euclidean
spaces with Bergman divergence as a proximity measure, that allows for non-euclidean proximal
regularization~\cite{tsengApproximation2010}.

The non-euclidean proximal gradient algorithm, that is more generally applicable to any VI problem
with a monotone operator, performs the following update at each iteration:

\begin{equation}
	\label{eqn:proxgrad} z_{t+1} = \arg \min_{z \in \calZ} \eta (\langle F(z_t),
	z\rangle + \alpha g(z)) + B_{\psi} (z; z_t).
\end{equation}

where $z_1 \in \text{int dom } \psi \cap \calZ$, and $\psi$ is a strongly convex function with respect to $\|.\|$ over $\calZ$.

The algorithm that is termed Magnetic Mirror Descent (MMD) uses the same update
as~\ref{eqn:proxgrad} with $g$ taken to be either $\psi$, or $B_{\psi}(.
	;z')$.
In the former, $\psi$ is as a strongly convex regularizer that makes the objective smoother and
encouranges exploration.
In the latter form, $B_{\psi}$ is another proximity term that forces the iterates ($z_{t+1}$) to
stay close to some \textit{magnet} ($z'$).
For all our discussion, we only consider the former update rule which is more widely applicable.
In this case, the main MMD algorithm~\cite[(Algorithm 3.6)]{sokotaUnified2023} becomes,

\begin{equation}
	\label{eqn:mmd} z_{t+1} = \arg \min_{z \in \calZ} \eta
	(\langle F(z_t), z\rangle + \alpha \psi(z)) + B_{\psi} (z; z_t).
\end{equation}

The MMD algorithm using the above update rule has the following linear convergence guarantee.
\begin{theorem}
	\label{thm:mmdconv}
	\cite[Theorem 3.4]{sokotaUnified2023}
	Assuming that the solution $z_{\ast}$ to the problem VI ($\calZ, F + \alpha \nabla g$) lies in the
	int dom $\psi$, then

	\[ B_{\psi} (z_{\ast}; z_{t + 1}) \leq { \left(\frac{1}{1
				+ \eta \alpha}\right)}^t B_{\psi} (z_{\ast}; z_1), \]

	if $\alpha > 0$, and
	$\eta \leq \frac{\alpha}{L^2}$.
\end{theorem}

\subsection{Behavioral form MMD}

The update rule~\ref{eqn:mmd} admits closed form in some instances, while requires approximation
through gradient updates in other cases.
For a parameterized policy $\pi_\theta$, when $\psi$ is negative entropy it can be restated in RL
terms as follows:
\begin{equation}
	\label{eqn:mmdbf} \pi_{ \theta_{k+1}}(s,a) = \arg \max_{\theta}
	\E_{s \sim \rho_{\theta_k}} \left[ \E_{a \sim \pi_{\theta_k}} [Q_{\theta_k}(s, a)] + \alpha
		H(\pi_\theta) - \frac{1}{\eta} KL(\pi_\theta, \pi_{\theta_k}) \right],
\end{equation}

where $H(\pi_\theta)$ is the entropy of the policy being optimized.

This behavioral form update can also be approximated using gradient updates similar to MDPO.
For $m$ steps MMD uses the following gradient to update the policy parameters,
\begin{equation}
	\label{eqn:mmdgrad} \nabla_{\theta} J(\theta, \theta_k)|_{\theta = \theta_k^{(i)}} = \E_{s \sim
		\rho_{\theta_k} \\ a \sim \theta_k} [ \frac{ \pi_{\theta_k}^{(i)}}{ \pi_{\theta_k}} \nabla_{\theta}
		\log \pi_{\theta_k}^{(i)} (a|s) A_{\theta_k}(s,a)] + \alpha H( \pi_{\theta_k}^{(i)}) -
	\frac{1}{\eta} \E_{s \sim \rho_{\theta_k}} [\nabla_\theta KL(s; \pi_\theta,
		\pi_{\theta_k})_{|\theta= \theta_k^{(i)}}]
\end{equation} where $i=0,1,\ldots,m-1$.

\textbf{Tabular MMD: }
Similar to tabular MDPO, the gradient computation for behavioral-form MMD in single-state
all-action setting becomes:

\begin{equation}
	\label{eqn:tabmmd} \nabla
	\theta_{|\theta=\theta_{k}^{(i)}} = \nabla_{\theta} \sum_{a \in A} [ \pi_{\theta_k}^{(i)} (a)
		A_{\theta_k}(a)] + \alpha \nabla_{\theta} H(\pi_{\theta_k}) - \frac{1}{\eta} [\nabla_\theta
		KL(\pi_\theta, \pi_{\theta_k})]
\end{equation}

\subsection{Closed-form vs
	Behavioral-form} MMD with a negative entropy mirror map~ \cite[equation (12)]{sokotaUnified2023}
has the following closed-form:

\begin{equation}
	\label{eqn:mmdcf} \pi_{k+1}
	\propto [\pi_t \rho^{\alpha \eta} e^{\eta Q_{\pi_k}}]^{ \frac{1}{1+\alpha \eta}}
\end{equation}

To examine the effect of the choice of $m$, we compare the performance of
tabular MMD~\ref{eqn:tabmmd} to the above closed-form.

In Fig~ \ref{fig:?
}, we plot the norm of the difference between the closed-form and behavioral form policies at each iteration.
It can be seen that the behavioral form approximates the closed form well as expected for most
choices of step size and $m$.
For large number of gradient steps, or large step sizes the updates are unstable.
For all our tabular experiments, we use a step-size of 0.1, and $m=10$.

\subsection{Comparison of MMD to other Mirror Decent based methods}

The non-euclidean proximal gradient method~\ref{eqn:proxgrad}, has strong connections to Mirror
Descent~\cite[Appendix D.3]{sokotaUnified2023}.
% \fillin{TBD: other works also discuss this relationship between mirror descent and proximal gradient methods applied to VIs}.
Consequently negative entropy based MMD is also equivalent to MDPO with an added entropy
regularization as detailed in~\cite[Appendix L]{sokotaUnified2023}.
Sokota et.~al~\cite{sokotaUnified2023} experimentally demonstrate MMD's strong performance in
single, and multi-agent settings by evaluating it against popular baselines.
In single agent settings MMD's performance is competitive with PPO in Atari and MuJoCo
environments.
MMD instantiated as a reinforcement learning algorithm performing behavioral form update at each
information state performs on the same level as CFR, but relatively worse compared to CFR+.

\chapter{Modified Updates for Mirror-descent based methods}

The main work of this thesis is exploring a few existing techniques that have been proven to be 
effective in multi-agent settings in the context of the above alogrithms.
These techniques are namely - Neural Replicator Dynamics~\cite{hennesNeural2020}, Extragradient methods~\cite{korpelevichextragradient1976}, and Optimistic updates \red{[?]}. 
We propose combining these techniques to the methods discussed in the previous section and
investigate their behavior through experiments in two-player zero-sum games.
We now discuss these techniques and how they have generally been applied in the literature. 
While discussing each technique, we also outline how they fit into the algorithms discussed 
in the previous section and hypothesize about the potential effects these modifications.

\section{Neural Replicator Dynamics (NeuRD)}
Replicator Dynamics is an idea from Evolutionary game theory (EGT) that defines operators to update
the dynamics of a population in order to maximize some pay-off defined by a fitness function.
Neural Replicator Dynamics (NeuRD) is an extension of Replicator Dynamics 
to function approximation settings.

The single-population replicator dynamics is defined by the following system of differntial
equations:

\begin{equation}
	\label{eqn:rd} \dot{\pi}(a) = \pi(a)[u(a, \pi) -
		\bar{u}(\pi)], \forall a \in \mathcal{A}
\end{equation}

Hennes et.al,~\cite{hennesNeural2020} show equivalence between Softmax Policy gradients~\ref{sec:pg} and
continuous-time Replicator Dynamics~\cite[THEOREM 1, on p5]{hennesNeural2020}\label{thm:spgrd}.

With knowledge of this equivalence they detail how NeuRD can be implemented as a one line change to SPG.
This can be viewed as a fix to the regular SPG update to make it more suited to multi-agent settings 
by making the policy updates more responsive to changes in dynamics.

Replicator dynamics also have a strong connection to no-regret algorithms, and in this work, the authors also establish 
the equivalence between single-state all actions tabular NeuRD, Hedge, and discrete time RD~\cite[Statement 1, p5]{hennesNeural2020}.
Due to this equivalence, NeuRD also inherits the no-regret properties of algorithms like Hedge.
While NeuRD has average iterate convergence, the authors also induce last iterate convergence in imperfect 
information settings for NeuRD using reward transformation based regularization~\cite{perolatPoincare2021}.

e As seen in Fig.., tabular SPG does not converge to the Nash Equilibrium, 
whereas the average policy of tabular NeuRD converges to the Nash Equlibrium.

\subsection{Alternating vs Simultaneous updates}

We also observe that while alternating updates shows average-iterate convergence, simultaneous updates do not converge.
\b{More background about this}

\subsection{NeuRD fix in MMD, and MDPO}
The NeuRD fix can be applied to MMD, and MDPO in a similar way to SPG.


\section{Extragradient updates}

The Extragradient method (EG) was first introduced by
G.M.Korpelevich~\cite{korpelevichextragradient1976} as a modification of gradient descent methods
in solving saddle point problems.
EG is a classical method for solving smooth and strongly convex-concave bilinear saddle
point problems with a linear rate of convergence.
Extragradient and Optimistic Gradient Descent Ascent methods have been shown to be approximations
of proximal-point method for solving saddle point methods~\cite{mokhtariUnified2020}.

EG also has linear last-iterate convergence guarantees for variational inequality problems with a 
strongly monotone operator. This 
\cite{tsenglinear1995} - Last iterate linear convergence using extragradient method for VI problems.

\subsection{MMD-EG}

\subsection{MDPO-EG}

\section{Optimism}




\newtheorem{theorem}{Theorem}


\chapter{Derivations}

\section{Online Learning}


\subsection{FoReL}


\section{Online Mirror Descent}


The FoReL update rule is,

\begin{align*}
    w_{t+1} &= argmin_w R(w) + \sum_{i=1}^t \langle w, z_t \rangle \\
            &= argmin_w R(w) + \langle w, z_{1:t}\rangle \\
            &= argmax_w \langle w, -z_{1:t} \rangle - R(w)    
\end{align*}

Let $g(\theta) = argmax_w \langle w, \theta \rangle - R(w)$. Then the FoReL update rule can 
be written as,

\begin{align*}
    \theta_{t+1} = \theta_t - z_t
    w_{t+1} = g(\theta_{t+1})
\end{align*}

where $g(\theta)$ is a link function that projects the predictions back to the convex set $S$.

Using different regularization functions yield different algorithms that have different regret bounds.

\begin{theorem}
    If R is a $(\frac{1}{\eta})$-strongly-convex function over $S$ with respect to some norm $\|.\|$, and OMD 
    is run on a sequence with the following link function

    $$g(\theta) = argmax_w (\langle w, \theta \rangle - R(w))$$

    then,

    $$\forall u \in S, Regret_T(u) \leq R(u) - min_{v \in S} R(v) + \eta \sum_{t=1}^T \|z\|_*^2$$

    where $\|.\|_*$ is the dual norm.
\end{theorem}


\subsection{Hedge}


Hedge or normalized Exponentiated Gradient is OMD with entropic regularization. The link function here is

\begin{equation}
    g_i(\theta) = \frac{e^{\eta \theta[i]}}{\sum_j e^{\eta \theta[j]}}.
\end{equation}

Fitting this into the OMD framework yields the following update rule,

\begin{align*}
    w_{t+1}[i] = \frac{w_t[i] e^{-\eta z_t[i]}}{\sum_j w_t[j] e^{-\eta z_t[j]}}
\end{align*}

We can analyze the regret bounds of Hedge with $R(w) = \frac{1}{\eta} \sum_i w[i] log(w[i])$. 


It is also useful to analyze OMD with the language of duality. The framework utilizing duality makes it easier 
in deriving new algorithms and also in proving tighter regret bounds.

\subsection{Fenchel Conjugacy}

The Fenchel conjugate of a function $f$ is defined as,

$$f^*(\theta) = max_u \langle u, \theta \rangle - f(u)$$

Fenchel conjugate by definition implies the Fenchel-Young inequality:

$$\forall u, f^*(\theta) \geq \langle u, \theta \rangle - f(u)$$.

If $u$ is a sub-gradient of $f^*$ at $\theta$ and if $f^*$ is differentiable, then the equality 
condition holds when $u = \nabla f^*(\theta)$. 


\subsection{Bergman Divergences}

For a differentiable function $R$, the Bergman divergence between two vectors is defined as,

\begin{equation}
    D_R(w \| u) = R(w) - R(u) + (\langle R(u), w-u \rangle)
\end{equation}

Bergman divergence is asymmetric and is always non-negative if R is convex.

\subsection{Online Mirror Descent in terms of Duality}

The link function in the OMD framework is defined as,

$$g(\theta) = argmax_w (\langle w, \theta \rangle - R(w)).$$

This can be also rewritten in terms of the conjugate of $R$ as,

$$g(\theta) = \nabla R^*(\theta)$$

With this, we can obtain different algorithms by using different regularization functions and deriving 
the update rules by using their conjugate.

\subsection{KL-Divergence and its Fenchel Conjugate}

KL-Divergence is a distance metric between two probability distributions and is defined as,

$$D_{KL}(p \| q) = \sum_i p[i] log \frac{p[i]}{q[i]}$$


% TODO: Derivation of the fenchel conjugate
The Fenchel Conjugate of KL-Divergence is given by,

$$f^*_q(x) = \log (\sum_i q_i e^{x_i}).$$


\section{MDPO}

The on-policy MDPO update rule is written as,

$$\theta_{k+1} \leftarrow argmax_{\theta \in \Theta} \Psi(\theta, \theta_k)$$

where,

$$\Psi(\theta, \theta_k) = \mathbb{E}_{s \sim \rho_{\theta_{k}}} [\mathbb{E}_{a \sim \pi_{\theta}}[A^{\theta_k}(s, a)] - \frac{1}{t_k} KL(s; \pi_{\theta}, \pi_{\theta_k})]$$


The gradient of the above update rule is as follows:

\begin{align*}
    \nabla_{\theta} \Psi(\theta, \theta_k) |_{\theta=\theta_k}
                                    &= \mathbb{E}_{s \sim \rho_{\theta_k}} [\sum_a \nabla_{\theta} \pi_{\theta} (a|s) A^{\theta_k}(s, a)] \\
                                    &= \mathbb{E}_{s \sim \rho_{\theta_k}} [\sum_a \pi_{\theta_k}(a|s) \frac{\nabla_{\theta} \pi_{\theta}(a|s)}{\pi_{\theta_k}(a|s)}  A^{\theta_k}(s, a)] \\
                                    &= \mathbb{E}_{s \sim \rho_{\theta_k}, a \sim \pi_{\theta_k}} [\nabla \log \pi_{\theta_k} (a|s) A^{\theta_k}(s,a)]
\end{align*}

For one-step MDPO, the gradient of the KL-Divergence term becomes 0. Hence it is proposed that the policy update at each iteration $k$ is done through $m$ steps of SGD.

$${\theta_k^{(0)} = \theta_k},$$

$$\theta_k^{(i+1)} \leftarrow \theta_k^{(i)} + \eta \nabla_{\theta} \Psi(\theta, \theta_k)|_{\theta=\theta_k^{(i)}}$$

and, $\theta_{k+1} = \theta_k^{(m)}$.


Then the gradient of the objective function evaluated at each step of the SGD update is,

\begin{align*}
    \nabla_{\theta} \Psi(\theta, \theta_k)|_{\theta = \theta_k^{(i)}} &=
                    \mathbb{E}_{s \sim \rho_{\theta_k}, a \sim \pi_{\theta_k}}[\frac{\pi_{\theta_k}^{(i)}}{\pi_{\theta_k}} \nabla \log \pi_{\theta_k^{(i)}} (a|s) A^{\theta_k}(s,a)] \\
                    &- \frac{1}{t_k} \mathbb{E}_{s \sim \rho_{\theta_k}}[\nabla_{\theta} KL(s; \pi_{\theta}, \pi_{\theta_k})|_{\theta = \theta_k^{(i)}}].
\end{align*}


$$KL(s; \pi_{\theta}, \pi_{\theta_k}) = \sum_{a \in \mathcal{A}} \pi_{\theta_k^{(i)}}(a|s) \log \frac{\pi_{\theta_k^{(i)}}(a|s)}{\pi_{\theta_k}(a|s)}$$

The gradient of the KL-Divergence term is given by,

\begin{align*}
    \nabla_{\theta} KL(s; \pi_{\theta}, \pi_{\theta_{k}})|_{\theta = \theta_{k}^{(i)}} 
    &= \sum_{a \in \mathcal{A}} [\nabla_{\theta_{k}^{(i)}} (\pi_{\theta_k^{(i)}}(a|s) \log \pi_{\theta_k}^{(i)}(a|s)) - \nabla_{\theta_k^{(i)}} (\pi_{\theta_k^{(i)}}(a|s) \log \pi_{\theta_k}(a|s))] \\
    &= \log \pi_{\theta_k^{(i)}}(a|s) \nabla_{\theta_{k}^{(i)}} \pi_{\theta_k^{(i)}}(a|s) + \nabla_{\theta_{k}^{(i)}} \pi_{\theta_k^{(i)}}(a|s) - \log \pi_{\theta_{k}}(a|s) \nabla_{\theta_{k}^{(i)}} \pi_{\theta_{k}^{(i)}}(a|s)\\
    &= \sum_{a \in \mathcal{A}} [(\log \pi_{\theta_{k}}^{(i)}(a|s) + 1 - \log \pi_{\theta_k}(a|s)) \nabla_{\theta_{k}^{(i)}}{\pi_{\theta_k^{(i)}}}(a|s)].
\end{align*}



As for the first term of the gradient, it can be seen that the gradient includes a term to account for the fact that the action $a$ was sampled from the policy $\pi_{\theta_k}$

\begin{align*}
    \nabla_{\theta} \Psi(\theta, \theta_k)|_{\theta = \theta_k^{(i)}}
        &= \mathbb{E}_{s \sim \rho_{\theta_k}} [\sum_a \nabla_{\theta_k^{(i)}} \pi_{\theta_k^{(i)}}(a|s) A^{\theta_k}(s, a)] \\
        &= \mathbb{E}_{s \sim \rho_{\theta_k}} [\sum_a \pi_{\theta_k}(a|s) \frac{\pi_{\theta_k^{(i)}}(a|s)}{\pi_{\theta_k}(a|s)} \frac{\nabla_{\theta_k^{(i)}} \pi_{\theta_k^{(i)}}(a | s)}{\pi_{\theta_k^{(i)}}(a|s)} A^{\theta_k}(s, a)] \\
        &= \mathbb{E}_{s \sim \rho_{\theta_k}, a \sim \pi_{\theta_k}} [\frac{\pi_{\theta_k^{(i)}}(a|s)}{\pi_{\theta_k}(a|s)} \nabla_{\theta_k^{(i)}} \log \pi_{\theta_k^{(i)}}(a | s) A^{\theta_k}(s, a)]
\end{align*}


































































%\documentclass{article}

%\usepackage{dsfont}
%\usepackage{amsfonts}

%\begin{document}

%\title{Derivations for the main thesis}

\newtheorem{definition}{Definition}
\newtheorem{lemma}{Lemma}

\chapter{Online Learning and Online Convex Optimization}

\section{Online Learning}

Online Learning is a sub-domain of machine learning that has important theoretical and practical applications. 
In Online Learning, a learner is tasked with predicting the answer to a set of questions over a sequence of consecutive rounds.
At each round t, a question $x_t$ is taken from an instance domain $\mathcal{X}$, and the learner is required to predict 
an answer, $p_t$ to this question. After the prediction is made, the correct answer $y_t$, from a target domain $\mathcal{Y}$ 
is revealed and the learner suffers a loss $l(p_t, y_t)$. The prediction $p_t$ could belong to $\mathcal{Y}$ or a larger set, 
$\mathcal{D}$.

There are many special cases of Online learning that translate to popular Online learning problems. Some common ones are,

Online Classification: $\mathcal{Y}=\mathcal{D}=\{0,1\}$, and typically the loss function is the 0-1 loss: $l(p_t, y_t)=|p_t - y_t|$.

Online Regression:

Expert's case:


The goal of an Online learning algorithm is to minimize the cumulative loss across all the rounds it has been through so far.
The learner uses the information from the previous rounds to improve its prediction on present and future rounds.

The sequence of questions can be deterministic, stochastic or even adversarial. This means, for any online learning algorithm 
an adversary can make the cumulative loss unbounded, by simply providing an opposing answer to the algorithm's answer as the correct 
answer. To make learning possible, certain restrictions are imposed on the structure of the problem.

Realizability: It is assumed that the answers are generated by a target mapping $h^*: \mathcal{X} \rightarrow \mathcal{Y}$, and that $h^*$ is 
taken from a fixed set, $\mathcal{H}$ called the hypothesis class. Now, for any Online learning algorithm, A, $M_A(\mathcal{H})$ is the number 
of mistakes $A$ makes on a sequence of questions, labelled by some $h^* \in \mathcal{H}$. $M_A(\mathcal{H})$ is called the $\textit{mistake-bound}$ 
of $A$.

A relaxation from realizable assumption is that the answers are not generated by some fixed mapping $h^*$, but the learner is still only required 
to be competitive with the best fixed predictor from $\mathcal{H}$. This is the regret of an Online learning algorithm for not having followed a 
fixed hypothesis $h^* \in \mathcal{H}$.

\begin{equation}\label{eqn_regretdef}
    Regret_T(h^*) = \sum_{t=1}^T l(p_t, y_t) - \sum_{t=1}^T l(h^*(x_t), y_t),
\end{equation}

The regret of $A$ with $\mathcal{H}$ is,

\begin{equation}\label{eqn_regretdef}
    Regret_T(\mathcal{H}) = max_{h^* \in \mathcal{H}} Regret_T(h^*)
\end{equation}


\section{Online Convex Optimization}

An established approach to design efficient online learning algorithm has been using convex optimization. This typically frames online learning as an 
online convex optimization problem as follows:

input: a convex set S
for t = 1, 2. $\ldots$
predict a vector $w_t \in S$
receive a convex loss function $f_t: S \mapsto \mathbb{R}$

Reframing \ref{eqn_regretdef} in terms of convex optimization, we refer to a competing hypothesis here as some vector $u$ from the convex set $S$.

\begin{equation}
    Regret_T(u) = \sum_{t=1}^T f_t(w_t) - \sum_{t=1}^T f_t(u)
\end{equation}

and similarly, the regret with respect to a set of competing vectors $U$ is,
\begin{equation}
    Regret_T(U) = max_{u \in U} Regret_T(u)
\end{equation}

As stated in the case of online learning, the set $U$ can be same as $S$ or different in other cases. In this work, the default setting is $U=S$ and 
$S=\mathbb{R^d}$ unless specified otherwise.

% Convexification can be added here briefly

\subsection{FoReL}

Follow-the-Regularized-leader (FoReL) is a classic learning algorithm for online convex optimization, where the algorithm tries to minimize the loss on 
all past rounds along with a regularization term. The regularization term is used to stabilize the solution and prevent it from oscillating too much every 
round preventing converging to a solution.

The learning rule can be written as,

$$\forall t, w_t = argmin_{w \in S} \sum_{i=1}^{t-1} f_i(w) + R(w).$$

where $R(w)$ is the regularization term. Different regularization functions lead to different algorithms with varying regret bounds.


In the case of linear loss functions with respect to some $z_t$, i.e., $f_t(w) = \langle w, z_t \rangle$, and $S=\mathbb{R}^d$,  if FoReL is run with 
$l_2$-norm regularization $R(w) = \frac{1}{2 \eta} \|w\|_2^2$ , then the learning rule can be written as,

\begin{equation}
    w_{t+1} = -\eta \sum_{i=1}^t z_i = w_t - \eta z_t
\end{equation}

Since, $\nabla f_t(w_t) = z_t$, this can also be written as, $w_{t+1} = w_t - \eta \nabla f_t(w_t)$. This update rule is also commonly known as Online Gradient Descent.
The regret of FoReL run on Online linear optimization with a euclidean-norm regularizer is:

$$Regret_T(U) \leq BL \sqrt {2T}.$$

where $U = {u : \|u\| \leq B}$ and $\frac{1}{T} \sum_{t=1}^T \|z_t\|_2^2 \leq L^2$ with $\eta = \frac{B}{L\sqrt{2T}}$.

This can also be generalized to Convex Functions in general through linearization using the property of convex functions. For a convex set S, a convex function $f: S \mapsto \mathbb{R}$ is convex iff $\forall w \in S, \exists z$ such that,

\begin{equation}
    \forall u \in S, f(u) \leq f(w) + \langle u-w, z \rangle  
\end{equation}

Following this, in Online Convex Optimization for each round $t$, there exists a $z_t$ such that for all competing hypothesis $u$, 

$$f_t(w_t) - f_t(u) \leq \langle w_t - u, z_t \rangle.$$

where $z_t \in \partial f_t(w_t)$ is a sub-gradient of $f_t$ at $w_t$.

Then, for a sequence of convex loss functions $f_1, \ldots, f_T$ and vectors $w_1, \ldots, w_T$ and if for all $t$, $z_t \in \partial f_t(w_t)$,

\begin{equation}
    \sum_{t=1}^T (f_t(w_t) - f_t(u)) \leq \sum_{t=1}^T (\langle w_t, z_t\rangle - \langle u, z_t \rangle)
\end{equation}

This implies, the regret of an algorithm for Online Convex Optimization is upper bounded by the regret with respect to the linearization of the 
sequence of convex functions.

% Add details about theorem 2.4 and discuss about the requirement of the norm of z_t to be bounded by L, and how it relates to the lipschitzness of the loss function

Beyond Euclidean regularization, FoReL can also be run with other regularization functions and yield similar regret bounds given that the regularization functions are 
strongly convex.

\begin{definition}
    For any $\sigma$-strongly-convex function $f: S \mapsto \mathbb{R}$ with respect to a norm $\|.\|$, for any $w \in S$,
    \begin{equation}
        \forall z \in \partial f(w), \forall u \in S, f(u) \geq f(w) + \langle z, u - w\rangle + \frac{\sigma}{2}\| u - w \|^2.
    \end{equation}
\end{definition}


% Lemma 2.3 to be shifted above

\begin{lemma}\label{lem:forelrb}
    For a FoReL algorithm producing a sequence of vectors $w_1, \ldots, w_T$ with a sequence of loss functions $f_1, \ldots, f_T$, for all $u \in S$, 
    $$\sum_{t=1}^T (f_t(w_t) - f_t(u)) \leq R(u) - R(w_1) + \sum_{t=1}^T (f_t(w_t) - f_t(w_{t+1}))$$
\end{lemma}

\subsection{FoReL with Strongly Convex Regularizers}
From Lemma \ref{lem:forelrb}, the regret bound is given by,

$$\sum_{t=1}^T (f_t(w_t) - f_t(u)) \leq R(u) - R(w_1) + \sum_{t=1}^T (f_t(w_t) - f_t(w_{t+1}))$$

If $f_t$ is $L$-Lipschitz with respect to some norm $\|.\|$ then,

$$f_t(w_t) - f_t(u) \leq L \| w_t - w_{t+1} \|$$

If $\| w_t - w_{t+1} \|$ is small that leads to a better regret bound. It can be shown that if the regularization function $R(w)$ is strongly convex with 
respect to the same norm $\|.\|$ then $\|w_t - w_{t+1}\|$ is also bounded.

For a sequence of predictions $w_1, w_2, \ldots$ of the FoReL algorithm, with a regularizer $R: S \mapsto \mathbb{R}$,

$$f_t(w_t) - f_t(w_{t+1}) \leq L_t \|w-t - w_{t+1} \| \leq \frac{L_t^2}{\sigma}.$$

if $f_t$ is $L$-Lipschitz with respect to $\|.\|$ and $R$ is $\sigma$-strongly-convex.

\begin{theorem}\label{thm:forelregret}
    FoReL run on a sequence of convex functions $f_1, \ldots, f_T$ such that $f_t$ is $L_t$-Lipschitz, with a $\sigma$-strongly-convex regularization function 
    has a regret bound given by, 

    $$Regret_T(u) \leq R(u) - min_{v \in S} R(v) + \frac{TL^2}{\sigma}$$

    where $\frac{1}{T} \sum_{t=1}^T L_t^2 \leq L^2$.
\end{theorem}

\textcolor{red}{To add: derived regret bounds for euclidean and entropic regularizers}

\section{Online Mirror Descent}




% 

\begin{comment}
Mirror Descent:

Mirror descent with entropy regularization

Mirror descent with KL Divergence regularizations 


Mirror Descent Policy optimization (MDPO)

The update rule for on-policy MDPO is given by,

$\theta_{k+1} \leftarrow argmax_{\theta \in \Theta \psi(\theta, \theta_k)}$

$\psi(\theta, \theta_k) = \mathds{E}_{s ~ \rho_{\theta_k}}[\mathds{E}_{a~\pi_{\theta}}[A^{\theta_k}(S, a)] - \frac{1}{t_k} \textrm{KL}(s; \pi_{\theta}, \pi_{\theta_k})]$

\end{comment}

%\end{document}

\appendices
\newpage
\appendix

\chapter{Some Ancillary Stuff}

Ancillary material should be put in appendices.

\chapter{Some More Ancillary Stuff}

% Here is yet another appendix! Wahoo!

\cite{Farine20162243}

%\nocite{*}
\bibformb
\bibliography{BibFile}
\newpage
% \vita
% This is where the vita goes.  Its organization is left as an exercise.

\end{document}