\documentclass{uicthesi}

\usepackage{booktabs} % For formal tables

\usepackage{framed}
\usepackage{hyperref}
\usepackage{balance}
\usepackage[dvips]{graphics,color}

\usepackage{epsfig}
\usepackage{color}
\usepackage{subfigure}
\usepackage{multirow,tabularx}
\usepackage{placeins}
%\usepackage{miniltx}
\usepackage{mathtools}
\usepackage{graphicx}
\usepackage{epstopdf}
\usepackage{bm}

\usepackage{csquotes}
\usepackage{array}
\usepackage[yyyymmdd,hhmmss]{datetime}
\usepackage[subject={Todo}]{pdfcomment}
\usepackage[textsize=scriptsize,bordercolor=black!20]{todonotes}
\usepackage{xcolor,colortbl}
\usepackage{amssymb}% http://ctan.org/pkg/amssymb
\usepackage{pifont}% http://ctan.org/pkg/pifont
\usepackage[algo2e,titlenumbered,ruled]{algorithm2e} 
\usepackage{lipsum,environ,amsmath}
\usepackage{slashbox,booktabs,amsmath}
%\usepackage{todonotes}
\usepackage{cite}


\usepackage{dsfont}
\usepackage{amsfonts}
\usepackage{comment}

\usepackage{xspace}


\hyphenation{PageRank Convex dictatorship Dic-ta-tor-ship}

\newcounter{Lcount}
\newcommand{\numsquishlist}{
   \begin{list}{\arabic{Lcount}. }
    { \usecounter{Lcount}
 \setlength{\itemsep}{-.1ex}      \setlength{\parsep}{0ex}
      \setlength{\topsep}{0ex}       \setlength{\partopsep}{0ex}
      \setlength{\leftmargin}{1em} \setlength{\labelwidth}{1em}
      \setlength{\labelsep}{0.1em} } }
\newcommand{\numsquishend}{\end{list}}

\newcommand{\squishlist}{
   \begin{list}{$\bullet$}
    { \setlength{\itemsep}{-.1ex}      \setlength{\parsep}{0ex}
      \setlength{\topsep}{0ex}       \setlength{\partopsep}{0ex}
      \setlength{\leftmargin}{.8em} \setlength{\labelwidth}{1em}
      \setlength{\labelsep}{0.5em} } }
\newcommand{\squishend}{\end{list}}



\makeatletter
\DeclareOldFontCommand{\rm}{\normalfont\rmfamily}{\mathrm}
\DeclareOldFontCommand{\sf}{\normalfont\sffamily}{\mathsf}
\DeclareOldFontCommand{\tt}{\normalfont\ttfamily}{\mathtt}
\DeclareOldFontCommand{\bf}{\normalfont\bfseries}{\mathbf}
\DeclareOldFontCommand{\it}{\normalfont\itshape}{\mathit}
\DeclareOldFontCommand{\sl}{\normalfont\slshape}{\@nomath\sl}
\DeclareOldFontCommand{\sc}{\normalfont\scshape}{\@nomath\sc}
\makeatother


\newcounter{problem}
\newenvironment{problem}[1][htb]
  {\renewcommand{\algorithmcfname}{Problem}% Update algorithm name
   \begin{algorithm2e}[#1]%
   \SetAlFnt{\small}
    \SetAlCapFnt{\small}
    \SetAlCapNameFnt{\small}
    \SetAlCapHSkip{0pt}
  }{\end{algorithm2e}}
  
  \newenvironment{alprocedure}[1][htb]
  {\renewcommand{\algorithmcfname}{Algorithm}% Update algorithm name
   \begin{algorithm2e}[#1]%
    \SetAlFnt{\small}
\SetAlCapFnt{\small}
\SetAlCapNameFnt{\small}
\SetAlCapHSkip{0pt}
\IncMargin{-\parindent}
   
  }{\end{algorithm2e}}

% T: Addditions to the template
% For darkheaded arrows
\usepackage{wasysym}


\newtheorem{definition}{Definition}
\newtheorem{lemma}{Lemma}

% Shorthand for notations
\newcommand{\R}{\mathbb{R}}
\newcommand{\calX}{\mathcal{X}}
\newcommand{\calY}{\mathcal{Y}}
\newcommand{\calZ}{\mathcal{Z}}


\begin{document}

% Declarations for Front Matter

\title{Title of Thesis is here}
\author{Thiruvenkadam Sivaprakasam Radhakrishnan}
\pdegrees{}
\degree{Master's in Computer Science}
\committee{Prof. Ian Kash, Chair and Advisor \\ Prof. Anastasios Sidiropoulos  \\ Prof. Ugo Buy}
\maketitle


% \dedication
% {\null\vfil
% {\large
% \begin{center}
% To myself,\\\vspace{12pt}
% Perry H. Disdainful,\\\vspace{12pt}
% the only person worthy of my company.
% \end{center}}
% \vfil\null}


\acknowledgements
{The thesis has been completed... (INSERT YOUR TEXTS)\\

  \begin{flushright}YOUR INITIAL\end{flushright}}
% \acknowledgements
% {I want to ``thank'' my committee, without whose ridiculous demands, I
% would have graduated so, so, very much faster.}

% \preface
% This preface is purely optional at UIC.

\tableofcontents
\listoftables
\listoffigures
\listofabbreviations
\begin{list}
  {}
  {\setlength
    {\labelwidth}{1in}
    \setlength{\leftmargin}{1.5in}
    \setlength{\labelsep}{.5in}
    \setlength{\rightmargin}{\leftmargin}}
  \item[$\theta$\hfill] Parameters of a function approximator.
\end{list}

\summary
Put your summary of thesis here.

\chapter{Introduction}

% Setup
Multi-agent Systems (MAS)~\cite{tuylsMultiagent2012} are frameworks of problems of distributed
nature with independent actors called agents that cooperate or compete to achieve some outcome.
Due to the complexity of such problems, designing efficient algorithms for them can be challenging.
Machine learning and Reinforcement learning present opportunities for creating learning algorithms
for agents in Multi-agent settings.
Multi-agent Reinforcement Learning (MARL) as a research area deals with designing efficient and
performant RL algorithms for general multi-agent problems.
Due to the inherent issues in multi-agent settings such as the violation of non-stationarity assumption that underlies most RL algorithms, their direct adaptation to the multi-agent setting
can be limiting.
Also, the multi-agent setting presents new complications in the form of large state and action
spaces, high-dimensional problem representations, credit assignment to name a few.
There have continued efforts to apply popular policy gradient, and value-based single-agent RL
methods to multi-agent setting through novel adaptions that account for some of the above
limitations.
On the other hand, various works have adopted ideas from areas like online learning, game theory,
and evolutionary biology to design novel RL algorithms that circumvent the above limitations by
providing new theoretical guarantees, and display strong empirical performances.
Of these, a vast majority of methods rely on regret minimization, and self-play learning where
theoretical convergence guarantees necessitate maintaining an average of past strategies that the
agents use (also called as the average policy in RL terms).
While this is easier in tabular settings, maintaining such averages past policies can become
burdening for large-scale problems especially when utilizing function approximation methods like
neural networks to represent the policy parameters.
Hence, there is a strong incentive in designing algorithms that have convergence guarantees for the
last-iterate policy as opposed to having to maintain an average policy.

In terms of the problem setting that concerns our work, we focus on the game-theoretic construct of
two-player zero sum games (2p0s).
There are various reasons for the interesting nature of two-player zero-sum games.
Firstly, within various game-theoretic problem formulations these games present a simple enough
structure while still retaining the multi-agent setting.
This presents an opportunity to study and design algorithms under the dynamics of two-player
zero-sum games, with an aim to extend these algorithms to general multi-agent settings.
Second, two-player zero-sum games are saddle-point problems, and provide an opportunity to connect
with optimization algorithms that have been studied extensively for solving saddle-point problems.
Finally, two-player zero-sum games also model other learning problems such as generative
adversarial networks (GANs)~\cite{goodfellowGenerative2014} and as such any improvements in methods
for the games can provide insights into designing algorithms for applications like GANs.

Our work primarily revolves around algorithms that were proposed or can be interpreted as
extensions of mirror descent as a reinforcement learning update rule.
Mirror Descent is a popular first-order optimization algorithm that has seen wide application in
various problem settings such as online learning, convex optimization, optimal transport etc,.
In this work, we study adaptations of mirror descent to reinforcement learning algorithm in single,
and multi-agent setting.
Specifically, we study Mirror Descent Policy Optimization (MDPO), and Magnetic Mirror Descent
(MMD), two recent algorithms that extend mirror descent as RL algorithms, and approximate
equilibrium solvers.
While there have been many works studying mirror descent with the aim of designing RL algorithms,
MDPO presents a practical deep RL algorithm and studies its similarities in structure, and
connections to existing work in the literature.
MDPO does not directly apply to two-player zero-sum games as it is a single-agent RL algorithm, and
as such has no convergence guarantees.
However, in our work we study its performance in this setting, and also propose some modifications
to the update that MDPO uses to make it more amenable to the multi-agent setting.
MMD, on the other hand approaches a different problem of entropy regularized equilibrium (QREs)
solving in two-player zero-sum games.
Finding such equilibrium is cast as variational inequality problems and an algorithm from the
variational inequality literature is applied to this problem setting under the name of MMD.
MMD enjoys novel linear last-iterate convergence by taking advantage of problem specific
assumptions in finding QREs.
Although there are no theoretical guarantees, MMD is also empirically studied as an algorithm for
finding approximate Nash equilibria by annealing the strength of the regularization.
MMD shows strong empirical performance as both tabular and deep reinforcement learning algorithm in
self-play to converge to the Nash equilibrium.
In this work, we use MMD as a baseline method, and empirically study the effect of the same
proposed modifications to understand how it improves MMD's performance.
Given the already strong baseline, any improvements observed can guide further algorithmic design
choices, and direct more study to understand the effect of the proposed modification in this
setting.

The modifications that we apply in this work are two-fold - one that modifies loss being used by
the algorithm, and the other that modifies the update structure of the algorithms, there by
affecting the problem's geometry in an effort to improve, or induce last-iterate convergence.
The first modification is the Neural Replicator Dynamics (NeuRD), an extension of the evolutionary
game-theoretic notion of Replicator dynamics to function approximation, and deep reinforcement
learning.
NeuRD can be interpreted as a modification of softmax policy gradients (SPGs) to handle the
plateauing effect of the non-linearity in adapting to the non-stationary rewards in multi-agent
settings.
Due to this structure, NeuRD can be incorporated into other algorithms that are extensions of SPGs.
The second proposed modification is the incorporation of Extragradient, and Optimistic updates to
the above algorithms.
The idea of using extra-gradients, or reusing past gradients is independent of the loss function
being used, and is therefore extendable to the algorithms discussed in this work.
However, there can be inherent limitations in applying these algorithms due to the theoretical
assumptions behind them when applied to specific problem-settings.
We only study the effect of the modifications empirically, and do not provide any theoretical
convergence guarantees for the proposed updates.
However, we take the potential theoretical limitations into consideration when discussing the
results of the empirical evaluations.

We evaluate the proposed algorithms on the classic game-theoretic normal-form game of biased
rock-paper-scissor to understand their empirical performance.
% Results
We show that in the presence of certain modifications for last-iterate convergence policy gradient
methods outperform the newer methods in this basic setting.
Through this comprehensive study, we then present these algorithms as extensions of policy gradient
methods with various design choices.
This presentation mainly aims at identifying the components of these extensions that contribute
more towards the performance and potential components that detract the performance.
Based on our findings, we proceed to implement and evaluate a subset of the modifications as deep
multi-agent reinforcement learning algorithms by training the players in self-play.
We use standard extensive-form games like Kuhn poker, and large-scale benchmarks like Abrupt Dark
Hex to test the applicability and scalability of the modifications in two-player zero-sum extensive
form games.
We summarize our findings regarding the effectiveness of mirror-descent-based reinforcement
learning algorithms in the multi-agent setting and the effect of the modifications we apply.
Through this study, we provide some recommendations in designing MARL algorithms and close with
some remarks about future research directions.

\chapter{Background}

We begin by providing some necessary background in Game Theory, Online Learning, and Reinforcement
learning to make the reader familiar with the concepts required to follow the ideas discussed in
the following sections.

\section{Game Theory}
- what is game theory?
- why is it useful?
- how is it related?

\subsection{Problem Domains and Representations}
- how are problems typically represented in game theory?
- what are the usual representations?
Normal Form games, Extensive Form games, Sequence Form.
- how is this relevant?

\subsection{Solution Concepts}
- what are solution concepts?
- what are the common solution concepts?
Nash equilibrium, Quantal response equilibrium.
- what are the relevant information related to solution conepts for this work?
Existence of a nash equilibrium Uniqueness of QRE

\section{Reinforcement
  Learning}

\subsection{Policy gradient methods}

\subsubsection{Softmax Policy Gradients}\label{sec:spg}

\subsection{PPO}

\section{Online Learning}

- what is online learning?

Online learning is the study of designing algorithms that use historical knowledge in predicting
actions for future rounds while trying to minimize some loss function in an adaptive (possibly
adversarial) setting.

- why is it useful?
- why is it relevant here?

\subsection{FoReL}\label{sec:forel}
- what is forel?
- relevant info?

\subsection{Hedge}
- what is hedge?
\chapter{Mirror Descent in Reinforcement Learning}

There have been a few well-established efforts in adapting mirror descent from a general
first-order optimization algorithm as described above into a reinforcement learning algorithm.
In this work, we mainly consider two such algorithms, namely MDPO (Mirror Descent Policy
Optimization), and MMD (Magnetic Mirror Descent).
We provide a high-level description of the approach taken in arriving at these algorithms and how
they are connected to Mirror Descent before applying modifications on top of these algorithms in an
effort to improve convergence behaviors of these algorithms in two-player zero-sum settings.

\subsection[MDPO]{Mirror Descent Policy Optimization}

The first of these algorithms Mirror Descent Policy Optimization~\cite{tomarMirror2022} addresses
the problem of trust-region solving to stabilize reinforcement learning updates.
Reinforcement learning algorithms, especially Policy gradient methods are notoriously known for
having high-variance updates causing troubles in converging to the optimal policy \red{cite}.
One approach to tackle this problem from an optimization perspective is to ensure that the policy
does not change drastically with each gradient update to stabilize the learning.
This is popularly known as trust region optimization.
PPO, TRPO are popular single agent reinforcement learning algorithms that incorporate trust region
optimization into \dots \red{outline PPO, and TRPO}.
\cite{schulmanProximal2017}

While MDPO differs from the above approaches in that they arrive at a similar algorithm from the
perspective of Mirror Descent, they also show strong connections to PPO, TRPO, and SAC.

\subsection{Mirror Descent as an RL algorithm}
\red{RL objective is not convex; can still use MD..
	\cite{shaniAdaptive2020}}

$$ \pi_{k+1} <- \arg \max_{\pi \in \Pi} \E_{s \sim \rho_{\pi_k}} $$


\textbf{Gradient of KL-Divergence}

Let's consider a reference policy $Q$ and $P_{\theta}$, a policy that is being optimized. 
The reverse KL-Divergence between the two is given by

$$
D_{KL}(P_{\theta} \| Q) = \sum_{a \in A} P_{\theta}(a) \log \frac{P_{\theta}(a)}{Q(a)}
$$

The gradient of KL-Divergence with respect to the parameters theta is given by:

\begin{equation*}
	\begin{split}
\nabla_{\theta} D_{KL} &= \sum_{a \in A} \nabla_{\theta} [P_{\theta}(a) \log P_{\theta}(a)] - 
						\nabla_{\theta} [P_{\theta}(a) \log Q(a)] \\
						&= \sum_{a \in A} [\nabla_{\theta} P_{\theta}(a) \log P_{\theta}(a) + \nabla_{\theta} P_{\theta}(a)] -  
						 \nabla_{\theta} P_{\theta}(a) \log Q(a) \\
						&= \sum_{a \in A} \nabla_{\theta} P_{\theta}(a) (\log P_{\theta}(a) - \log Q(a) + 1)
	\end{split}
\end{equation*}

\section[MMD]{Magnetic Mirror Descent}

Another extension of Mirror Descent to reinforcement learning is Magnetic Mirror Descent~\cite{sokotaUnified2023}, 
that attempts to create a unified algorithm that works well in both single and multiagent settings.
In this work, first an equivalence between solving regularized saddle point problems
and Variational Inequality problems with composite structures is established. 
Then this connection is used to propose an equilibrium solving algorithm that has linear last iterate convergence guarantees. 
Moreover, the proposed method is extended as a reinforcement learning algorithm in single agent setting, and 
also as a multiagent reinforcement algorithm through selfplay.

\subsection{Connection between Variational Inequalities and QREs}

Finding the QRE of a two-player zero-sum game can be represented as the following entropy-regularized saddle
point problem. Given $\calX \subseteq \R^n$, $calY \subseteq \R^m$, and $g_1: \R^n \mapsto \R$, $g_2: \R^m \mapsto \R$, 
find:
\begin{equation}
	\label{eqn:saddle} \min_{x \in \mathcal{X}}
	\max_{y \in \mathcal{Y}} \alpha g_1(x) + f(x, y) + \alpha g_2(y),
\end{equation}

The solution $(x_{\ast}, y_{\ast})$ to the saddle point problem~\ref{eqn:saddle} has the following first-order optimality conditions:
\begin{equation}
	\label{eqn:optcon} 
	\begin{split}
		\langle \alpha \nabla g_1(x_*) + \nabla_{x_*}
		f(x_*, y_*), x - x_* \rangle \geq 0, \forall x \in \calX. \\
		\langle \alpha \nabla g_2(y_*) + \nabla_{y_*} f(x_*, y_*),
		y - y_* \rangle \geq 0, \forall y \in \calY.
	\end{split}
\end{equation}

A Variational Inequality (VI) problem, written as $VI(Z, F)$ is generally defined as follows:
\begin{definition}
	\label{def:vi}
	Given $\calZ
		\subseteq \R^n$ and mapping $F: \calZ \rightarrow \R^n$, the variational inequality problem VI
	($\calZ, F$) is to find $z_{\ast} \in \calZ$ such that,
	\[ \langle F(z_{\ast}),
		z - z_{\ast} \rangle \geq 0 \quad \forall z \in \calZ.
	\]
\end{definition}

The reguarlized saddle point problem~\ref{eqn:saddle} of solving for QREs can be related to the VI problem 
with a composite objective. If $G = F + \alpha \nabla g$, and $\calZ = \mathcal{X} \times \mathcal{Y}$, 
where 
The optimality conditions~\ref{eqn:optcon} are equivalent to VI$(\calZ, G)$, where $G = F + \alpha \nabla g$,
, and $g: \calZ \rightarrow \mathbb{R}$.
Hence, the solution the the VI problem ($z_{\ast}= (x_{\ast}, y_{\ast})$), corresponds to the
solution of the saddle point problem stated in~\ref{eqn:saddle}.

\subsection{MMD Algorithm}

Various algorithms have been proposed to solve the VI problem~\ref{def:vi}.
In particular, the proximal point method has linear last iterate convergence for Variational 
inequality problems with a monotone operator~\cite{rockafellarMonotone1976}.
This algorithm was extended to composite objectives~\cite{tsenglinear1995}, and to non-euclidean 
spaces with Bergman divergence as a proximity measure, 
that allows for non-euclidean proximal regularization~\cite{tsengApproximation2010}.

This non-euclidean proximal gradient algorithm, that is more generally applicable for VI problems 
with monotone operators, performs the following update at each iteration:

\begin{equation}
	\label{eqn:proxgrad} z_{t+1} = \arg \min_{z \in \calZ} \eta (\langle F(z_t), z\rangle + \alpha
	g(z)) + B_{\psi} (z; z_t).
\end{equation}

where $\psi$ is a strongly convex function with respect to $\|.\|$ over $\calZ$.

The algorithm that is termed Magnetic Mirror Descent (MMD) uses the same update as~\ref{eqn:proxgrad}, with $g$ taken to be
$\psi$ as a proximal regularization operator or $B_{\psi}(.;z')$ enforcing $z_{t+1}$ to move close to some reference vector 
$z'$. In the latter case, $z'$ is referred to as the magnet, hence the name Magnetic Mirror Descent.

We now restate the main algorithm as stated in Sokota et.al,~\cite{sokotaUnified2023},

\begin{alprocedure}[H] \algorithmcfname{MMD~\cite[(Algorithm
			3.6)]{sokotaUnified2023}} \label{alg:mmd} Starting with $z_1 \in \text{int dom } \psi \cap \calZ$,
	at each iteration $t$ do $$ z_{t+1} = \arg \min_{z \in \calZ} \eta (\langle F(z_t), z\rangle +
		\alpha \psi(z)) + B_{\psi} (z; z_t).
	$$
	or,
	Given some $z'$, do $$ z_{t+1} = \arg \min_{z \in \calZ} \eta (\langle F(z_t), z\rangle + \alpha
		B_{\psi}(z; z')) + B_{\psi} (z, z_t).
	$$
\end{alprocedure}

\hfill \break
Algorithm~\ref{alg:mmd} provides the following convergence guarantees.

\begin{theorem}
	\label{thm:mmdconv}
	\cite[Theorem 3.4]{sokotaUnified2023}
	Assuming that the solution $z_{\ast}$ to the problem VI ($\calZ, F + \alpha \nabla g$) lies in the
	int dom $\psi$, then

	\[ B_{\psi} (z_{\ast}; z_{t + 1}) \leq {\left(\frac{1}{1 +
				\eta \alpha}\right)}^t B_{\psi} (z_{\ast}; z_1), \]

	if $\alpha > 0$, and $\eta
		\leq \frac{\alpha}{L^2}$.
\end{theorem}

\subsection{Behavioral form MMD}

Algorithm~\ref{alg:mmd} can be either implemented in closed form for a given set of parameters or
by performing stochastic gradient descent at each step to approximate the closed form.
This approximation is especially useful when it is not possible to compute the closed form such as
in function approximation settings.

In the two form of updates specified in Algorithm~\ref{alg:mmd}, the second form involves a $z'$,
term that is referred to as the magnet.
This can either be a reference policy such as the ones used in Imitation learning, or one that is
trailing the current policy to stabilize learning.
Alternatively, one can also use a uniform policy in which case the term reduces to entropy and acts
as a regularization term and encourages exploration.
For the rest of our discussion on MMD, we assume the magnet to always be a uniform policy.

With $\psi$ taken to be negative entropy, the behavioral form of MMD is to perform the following
update at each information state,

\begin{equation}
	\label{eqn:mmdbf} \pi_{t+1}
	= \arg \max_{\pi} \mathbb{E}_{A \sim \pi} q_t(A) + \alpha H(\pi) - \frac{1}{\eta} KL(\pi, \pi_t),
\end{equation}

where $\pi_t$ is the current policy, $q_t$ is a vector
containing the q-values following the policy $\pi_t$, and $H(\pi)$ is the entropy of the policy
being optimized.

In single-agent settings MMD's performance is competitive with PPO in Atari and MuJoCo
environments.
And, in the multi-agent setting the performance of tabular MMD is on par with CFR, but worse than
CFR+.

\subsection{Comparison of MMD to other Mirror Decent based methods}

The non-euclidean proximal gradient method~\ref{eqn:proxgrad}, has strong connections to Mirror Descent. 
\cite[Appendix D.3]{sokotaUnified2023} also details this relationship by showing MMD is equivalent to Mirror 
Descent with an added regularization term.
Since MDPO is a direct extension of Mirror Descent to reinforcement learning, negative entropy based MMD 
is also equivalent to MDPO with an added negative entropy regularization as detailed in~\cite[Appendix L]{sokotaUnified2023}.
\chapter{Faster MMD and MDPO}

\section{Faster-MMD}

We now propose a few modified versions of Magnetic Mirror Descent, and MDPO

\section{Neural Replicator Dynamics (NeuRD)} Neural Replicator Dynamics (NeuRD)
\cite{hennesNeural2020} is a model-free sample-based algorithm that applies function approximation
to Replicator Dynamics.
Replicator Dynamics is an idea from Evolutionary game theory (EGT) that defines operators to update
the dynamics of a population in ord to maximize some pay-off defined by a fitness function.

The single-population replicator dynamics is defined by the following system of differntial
equations:

\begin{equation}
	\label{eqn:rd} \dot{\pi}(a) = \pi(a)[u(a, \pi) -
		\bar{u}(\pi)], \forall a \in \mathcal{A}
\end{equation}

Hennes et.
al, \cite{hennesNeural2020} show equivalence between Softmax Policy gradients
\ref{sec:spg} and continuous-time Replicator Dynamics \cite[THEOREM 1, on p5]{hennesNeural2020}\label{thm:spgrd}.

NeuRD can be applied to reinforcement learning as a single line change to the Softmax Policy
gradient.
This can be seen as applying a fix to the update of the Softmax Policy gradient algorithm to make
it more responsive to changes in a non-stationary environment.
We refer to this as the ``NeuRD-fix''.

\subsection{FMMD-N}

The first modified version of MMD, and MDPO is obtained applying the NeuRD-fix to these algorithms.
Below, we derive the gradient of the behavioral form update rule for these algorithms, and show
where the fix applies in these updates.

\section{Extragradient updates}

The Extragradient method was first introduced by G.M.Korpelevich~\cite{korpelevichEG1976}.
EG is a classical method for solving smooth and strongly convex-concave bilinear saddle point
problems with a linear rate of convergence.
Extragradient and Optimistic Gradient Descent Ascent methods have been shown to be approximations
of proximal-point method for solving saddle point methods~\cite{mokhtariUnified2020}.

\subsection{FMMD-EG}
In this section, we outline a version of MMD with extragradient updates that we call
\textit{FMMD-EG}.

\section{Optimism}

\subsection{Optimistic Mirror Descent}

\subsection{OFMMD}


\newtheorem{theorem}{Theorem}


\chapter{Derivations}

\section{Online Learning}


\subsection{FoReL}


\section{Online Mirror Descent}


The FoReL update rule is,

\begin{align*}
    w_{t+1} &= argmin_w R(w) + \sum_{i=1}^t \langle w, z_t \rangle \\
            &= argmin_w R(w) + \langle w, z_{1:t}\rangle \\
            &= argmax_w \langle w, -z_{1:t} \rangle - R(w)    
\end{align*}

Let $g(\theta) = argmax_w \langle w, \theta \rangle - R(w)$. Then the FoReL update rule can 
be written as,

\begin{align*}
    \theta_{t+1} = \theta_t - z_t
    w_{t+1} = g(\theta_{t+1})
\end{align*}

where $g(\theta)$ is a link function that projects the predictions back to the convex set $S$.

Using different regularization functions yield different algorithms that have different regret bounds.

\begin{theorem}
    If R is a $(\frac{1}{\eta})$-strongly-convex function over $S$ with respect to some norm $\|.\|$, and OMD 
    is run on a sequence with the following link function

    $$g(\theta) = argmax_w (\langle w, \theta \rangle - R(w))$$

    then,

    $$\forall u \in S, Regret_T(u) \leq R(u) - min_{v \in S} R(v) + \eta \sum_{t=1}^T \|z\|_*^2$$

    where $\|.\|_*$ is the dual norm.
\end{theorem}


\subsection{Hedge}


Hedge or normalized Exponentiated Gradient is OMD with entropic regularization. The link function here is

\begin{equation}
    g_i(\theta) = \frac{e^{\eta \theta[i]}}{\sum_j e^{\eta \theta[j]}}.
\end{equation}

Fitting this into the OMD framework yields the following update rule,

\begin{align*}
    w_{t+1}[i] = \frac{w_t[i] e^{-\eta z_t[i]}}{\sum_j w_t[j] e^{-\eta z_t[j]}}
\end{align*}

We can analyze the regret bounds of Hedge with $R(w) = \frac{1}{\eta} \sum_i w[i] log(w[i])$. 


It is also useful to analyze OMD with the language of duality. The framework utilizing duality makes it easier 
in deriving new algorithms and also in proving tighter regret bounds.

\subsection{Fenchel Conjugacy}

The Fenchel conjugate of a function $f$ is defined as,

$$f^*(\theta) = max_u \langle u, \theta \rangle - f(u)$$

Fenchel conjugate by definition implies the Fenchel-Young inequality:

$$\forall u, f^*(\theta) \geq \langle u, \theta \rangle - f(u)$$.

If $u$ is a sub-gradient of $f^*$ at $\theta$ and if $f^*$ is differentiable, then the equality 
condition holds when $u = \nabla f^*(\theta)$. 


\subsection{Bergman Divergences}

For a differentiable function $R$, the Bergman divergence between two vectors is defined as,

\begin{equation}
    D_R(w \| u) = R(w) - R(u) + (\langle R(u), w-u \rangle)
\end{equation}

Bergman divergence is asymmetric and is always non-negative if R is convex.

\subsection{Online Mirror Descent in terms of Duality}

The link function in the OMD framework is defined as,

$$g(\theta) = argmax_w (\langle w, \theta \rangle - R(w)).$$

This can be also rewritten in terms of the conjugate of $R$ as,

$$g(\theta) = \nabla R^*(\theta)$$

With this, we can obtain different algorithms by using different regularization functions and deriving 
the update rules by using their conjugate.

\subsection{KL-Divergence and its Fenchel Conjugate}

KL-Divergence is a distance metric between two probability distributions and is defined as,

$$D_{KL}(p \| q) = \sum_i p[i] log \frac{p[i]}{q[i]}$$


% TODO: Derivation of the fenchel conjugate
The Fenchel Conjugate of KL-Divergence is given by,

$$f^*_q(x) = \log (\sum_i q_i e^{x_i}).$$


\section{MDPO}

The on-policy MDPO update rule is written as,

$$\theta_{k+1} \leftarrow argmax_{\theta \in \Theta} \Psi(\theta, \theta_k)$$

where,

$$\Psi(\theta, \theta_k) = \mathbb{E}_{s \sim \rho_{\theta_{k}}} [\mathbb{E}_{a \sim \pi_{\theta}}[A^{\theta_k}(s, a)] - \frac{1}{t_k} KL(s; \pi_{\theta}, \pi_{\theta_k})]$$


The gradient of the above update rule is as follows:

\begin{align*}
    \nabla_{\theta} \Psi(\theta, \theta_k) |_{\theta=\theta_k}
                                    &= \mathbb{E}_{s \sim \rho_{\theta_k}} [\sum_a \nabla_{\theta} \pi_{\theta} (a|s) A^{\theta_k}(s, a)] \\
                                    &= \mathbb{E}_{s \sim \rho_{\theta_k}} [\sum_a \pi_{\theta_k}(a|s) \frac{\nabla_{\theta} \pi_{\theta}(a|s)}{\pi_{\theta_k}(a|s)}  A^{\theta_k}(s, a)] \\
                                    &= \mathbb{E}_{s \sim \rho_{\theta_k}, a \sim \pi_{\theta_k}} [\nabla \log \pi_{\theta_k} (a|s) A^{\theta_k}(s,a)]
\end{align*}

For one-step MDPO, the gradient of the KL-Divergence term becomes 0. Hence it is proposed that the policy update at each iteration $k$ is done through $m$ steps of SGD.

$${\theta_k^{(0)} = \theta_k},$$

$$\theta_k^{(i+1)} \leftarrow \theta_k^{(i)} + \eta \nabla_{\theta} \Psi(\theta, \theta_k)|_{\theta=\theta_k^{(i)}}$$

and, $\theta_{k+1} = \theta_k^{(m)}$.


Then the gradient of the objective function evaluated at each step of the SGD update is,

\begin{align*}
    \nabla_{\theta} \Psi(\theta, \theta_k)|_{\theta = \theta_k^{(i)}} &=
                    \mathbb{E}_{s \sim \rho_{\theta_k}, a \sim \pi_{\theta_k}}[\frac{\pi_{\theta_k}^{(i)}}{\pi_{\theta_k}} \nabla \log \pi_{\theta_k^{(i)}} (a|s) A^{\theta_k}(s,a)] \\
                    &- \frac{1}{t_k} \mathbb{E}_{s \sim \rho_{\theta_k}}[\nabla_{\theta} KL(s; \pi_{\theta}, \pi_{\theta_k})|_{\theta = \theta_k^{(i)}}].
\end{align*}


$$KL(s; \pi_{\theta}, \pi_{\theta_k}) = \sum_{a \in \mathcal{A}} \pi_{\theta_k^{(i)}}(a|s) \log \frac{\pi_{\theta_k^{(i)}}(a|s)}{\pi_{\theta_k}(a|s)}$$

The gradient of the KL-Divergence term is given by,

\begin{align*}
    \nabla_{\theta} KL(s; \pi_{\theta}, \pi_{\theta_{k}})|_{\theta = \theta_{k}^{(i)}} 
    &= \sum_{a \in \mathcal{A}} [\nabla_{\theta_{k}^{(i)}} (\pi_{\theta_k^{(i)}}(a|s) \log \pi_{\theta_k}^{(i)}(a|s)) - \nabla_{\theta_k^{(i)}} (\pi_{\theta_k^{(i)}}(a|s) \log \pi_{\theta_k}(a|s))] \\
    &= \log \pi_{\theta_k^{(i)}}(a|s) \nabla_{\theta_{k}^{(i)}} \pi_{\theta_k^{(i)}}(a|s) + \nabla_{\theta_{k}^{(i)}} \pi_{\theta_k^{(i)}}(a|s) - \log \pi_{\theta_{k}}(a|s) \nabla_{\theta_{k}^{(i)}} \pi_{\theta_{k}^{(i)}}(a|s)\\
    &= \sum_{a \in \mathcal{A}} [(\log \pi_{\theta_{k}}^{(i)}(a|s) + 1 - \log \pi_{\theta_k}(a|s)) \nabla_{\theta_{k}^{(i)}}{\pi_{\theta_k^{(i)}}}(a|s)].
\end{align*}



As for the first term of the gradient, it can be seen that the gradient includes a term to account for the fact that the action $a$ was sampled from the policy $\pi_{\theta_k}$

\begin{align*}
    \nabla_{\theta} \Psi(\theta, \theta_k)|_{\theta = \theta_k^{(i)}}
        &= \mathbb{E}_{s \sim \rho_{\theta_k}} [\sum_a \nabla_{\theta_k^{(i)}} \pi_{\theta_k^{(i)}}(a|s) A^{\theta_k}(s, a)] \\
        &= \mathbb{E}_{s \sim \rho_{\theta_k}} [\sum_a \pi_{\theta_k}(a|s) \frac{\pi_{\theta_k^{(i)}}(a|s)}{\pi_{\theta_k}(a|s)} \frac{\nabla_{\theta_k^{(i)}} \pi_{\theta_k^{(i)}}(a | s)}{\pi_{\theta_k^{(i)}}(a|s)} A^{\theta_k}(s, a)] \\
        &= \mathbb{E}_{s \sim \rho_{\theta_k}, a \sim \pi_{\theta_k}} [\frac{\pi_{\theta_k^{(i)}}(a|s)}{\pi_{\theta_k}(a|s)} \nabla_{\theta_k^{(i)}} \log \pi_{\theta_k^{(i)}}(a | s) A^{\theta_k}(s, a)]
\end{align*}


































































%\documentclass{article}

%\usepackage{dsfont}
%\usepackage{amsfonts}

%\begin{document}

%\title{Derivations for the main thesis}

\chapter{Online Learning and Online Convex Optimization}

\section{Online Learning}

Online Learning is a sub-domain of machine learning that has important theoretical and practical applications. 
In Online Learning, a learner is tasked with predicting the answer to a set of questions over a sequence of consecutive rounds.
At each round t, a question $x_t$ is taken from an instance domain $\mathcal{X}$, and the learner is required to predict 
an answer, $p_t$ to this question. After the prediction is made, the correct answer $y_t$, from a target domain $\mathcal{Y}$ 
is revealed and the learner suffers a loss $l(p_t, y_t)$. The prediction $p_t$ could belong to $\mathcal{Y}$ or a larger set, 
$\mathcal{D}$.

There are many special cases of Online learning that translate to popular Online learning problems. Some common ones are,

Online Classification: $\mathcal{Y}=\mathcal{D}=\{0,1\}$, and typically the loss function is the 0-1 loss: $l(p_t, y_t)=|p_t - y_t|$.

Online Regression:

Expert's case:


The goal of an Online learning algorithm is to minimize the cumulative loss across all the rounds it has been through so far.
The learner uses the information from the previous rounds to improve its prediction on present and future rounds.

The sequence of questions can be deterministic, stochastic or even adversarial. This means, for any online learning algorithm 
an adversary can make the cumulative loss unbounded, by simply providing an opposing answer to the algorithm's answer as the correct 
answer. To make learning possible, certain restrictions are imposed on the structure of the problem.

Realizability: It is assumed that the answers are generated by a target mapping $h^*: \mathcal{X} \rightarrow \mathcal{Y}$, and that $h^*$ is 
taken from a fixed set, $\mathcal{H}$ called the hypothesis class. Now, for any Online learning algorithm, A, $M_A(\mathcal{H})$ is the number 
of mistakes $A$ makes on a sequence of questions, labelled by some $h^* \in \mathcal{H}$. $M_A(\mathcal{H})$ is called the $\textit{mistake-bound}$ 
of $A$.

A relaxation from realizable assumption is that the answers are not generated by some fixed mapping $h^*$, but the learner is still only required 
to be competitive with the best fixed predictor from $\mathcal{H}$. This is the regret of an Online learning algorithm for not having followed a 
fixed hypothesis $h^* \in \mathcal{H}$.

\begin{equation}\label{eqn_regretdef}
    Regret_T(h^*) = \sum_{t=1}^T l(p_t, y_t) - \sum_{t=1}^T l(h^*(x_t), y_t),
\end{equation}

The regret of $A$ with $\mathcal{H}$ is,

\begin{equation}\label{eqn_regretdef_all}
    Regret_T(\mathcal{H}) = max_{h^* \in \mathcal{H}} Regret_T(h^*)
\end{equation}


\section{Online Convex Optimization}

An established approach to design efficient online learning algorithm has been using convex optimization. This typically frames online learning as an 
online convex optimization problem as follows:

input: a convex set S
for t = 1, 2. $\ldots$
predict a vector $w_t \in S$
receive a convex loss function $f_t: S \mapsto \mathbb{R}$

Reframing \ref{eqn_regretdef} in terms of convex optimization, we refer to a competing hypothesis here as some vector $u$ from the convex set $S$.

\begin{equation}
    Regret_T(u) = \sum_{t=1}^T f_t(w_t) - \sum_{t=1}^T f_t(u)
\end{equation}

and similarly, the regret with respect to a set of competing vectors $U$ is,
\begin{equation}
    Regret_T(U) = max_{u \in U} Regret_T(u)
\end{equation}

As stated in the case of online learning, the set $U$ can be same as $S$ or different in other cases. In this work, the default setting is $U=S$ and 
$S=\mathbb{R^d}$ unless specified otherwise.

% Convexification can be added here briefly

\subsection{FoReL}

Follow-the-Regularized-leader (FoReL) is a classic learning algorithm for online convex optimization, where the algorithm tries to minimize the loss on 
all past rounds along with a regularization term. The regularization term is used to stabilize the solution and prevent it from oscillating too much every 
round preventing converging to a solution.

The learning rule can be written as,

$$\forall t, w_t = argmin_{w \in S} \sum_{i=1}^{t-1} f_i(w) + R(w).$$

where $R(w)$ is the regularization term. Different regularization functions lead to different algorithms with varying regret bounds.


In the case of linear loss functions with respect to some $z_t$, i.e., $f_t(w) = \langle w, z_t \rangle$, and $S=\mathbb{R}^d$,  if FoReL is run with 
$l_2$-norm regularization $R(w) = \frac{1}{2 \eta} \|w\|_2^2$ , then the learning rule can be written as,

\begin{equation}
    w_{t+1} = -\eta \sum_{i=1}^t z_i = w_t - \eta z_t
\end{equation}

Since, $\nabla f_t(w_t) = z_t$, this can also be written as, $w_{t+1} = w_t - \eta \nabla f_t(w_t)$. This update rule is also commonly known as Online Gradient Descent.
The regret of FoReL run on Online linear optimization with a euclidean-norm regularizer is:

$$Regret_T(U) \leq BL \sqrt {2T}.$$

where $U = {u : \|u\| \leq B}$ and $\frac{1}{T} \sum_{t=1}^T \|z_t\|_2^2 \leq L^2$ with $\eta = \frac{B}{L\sqrt{2T}}$.

This can also be generalized to Convex Functions in general through linearization using the property of convex functions. For a convex set S, a convex function $f: S \mapsto \mathbb{R}$ is convex iff $\forall w \in S, \exists z$ such that,

\begin{equation}
    \forall u \in S, f(u) \leq f(w) + \langle u-w, z \rangle  
\end{equation}

Following this, in Online Convex Optimization for each round $t$, there exists a $z_t$ such that for all competing hypothesis $u$, 

$$f_t(w_t) - f_t(u) \leq \langle w_t - u, z_t \rangle.$$

where $z_t \in \partial f_t(w_t)$ is a sub-gradient of $f_t$ at $w_t$.

Then, for a sequence of convex loss functions $f_1, \ldots, f_T$ and vectors $w_1, \ldots, w_T$ and if for all $t$, $z_t \in \partial f_t(w_t)$,

\begin{equation}
    \sum_{t=1}^T (f_t(w_t) - f_t(u)) \leq \sum_{t=1}^T (\langle w_t, z_t\rangle - \langle u, z_t \rangle)
\end{equation}

This implies, the regret of an algorithm for Online Convex Optimization is upper bounded by the regret with respect to the linearization of the 
sequence of convex functions.

% Add details about theorem 2.4 and discuss about the requirement of the norm of z_t to be bounded by L, and how it relates to the lipschitzness of the loss function

Beyond Euclidean regularization, FoReL can also be run with other regularization functions and yield similar regret bounds given that the regularization functions are 
strongly convex.

\begin{definition}
    For any $\sigma$-strongly-convex function $f: S \mapsto \mathbb{R}$ with respect to a norm $\|.\|$, for any $w \in S$,
    \begin{equation}
        \forall z \in \partial f(w), \forall u \in S, f(u) \geq f(w) + \langle z, u - w\rangle + \frac{\sigma}{2}\| u - w \|^2.
    \end{equation}
\end{definition}


% Lemma 2.3 to be shifted above

\begin{lemma}\label{lem:forelrb}
    For a FoReL algorithm producing a sequence of vectors $w_1, \ldots, w_T$ with a sequence of loss functions $f_1, \ldots, f_T$, for all $u \in S$, 
    $$\sum_{t=1}^T (f_t(w_t) - f_t(u)) \leq R(u) - R(w_1) + \sum_{t=1}^T (f_t(w_t) - f_t(w_{t+1}))$$
\end{lemma}

\subsection{FoReL with Strongly Convex Regularizers}
From Lemma \ref{lem:forelrb}, the regret bound is given by,

$$\sum_{t=1}^T (f_t(w_t) - f_t(u)) \leq R(u) - R(w_1) + \sum_{t=1}^T (f_t(w_t) - f_t(w_{t+1}))$$

If $f_t$ is $L$-Lipschitz with respect to some norm $\|.\|$ then,

$$f_t(w_t) - f_t(u) \leq L \| w_t - w_{t+1} \|$$

If $\| w_t - w_{t+1} \|$ is small that leads to a better regret bound. It can be shown that if the regularization function $R(w)$ is strongly convex with 
respect to the same norm $\|.\|$ then $\|w_t - w_{t+1}\|$ is also bounded.

For a sequence of predictions $w_1, w_2, \ldots$ of the FoReL algorithm, with a regularizer $R: S \mapsto \mathbb{R}$,

$$f_t(w_t) - f_t(w_{t+1}) \leq L_t \|w-t - w_{t+1} \| \leq \frac{L_t^2}{\sigma}.$$

if $f_t$ is $L$-Lipschitz with respect to $\|.\|$ and $R$ is $\sigma$-strongly-convex.

\begin{theorem}\label{thm:forelregret}
    FoReL run on a sequence of convex functions $f_1, \ldots, f_T$ such that $f_t$ is $L_t$-Lipschitz, with a $\sigma$-strongly-convex regularization function 
    has a regret bound given by, 

    $$Regret_T(u) \leq R(u) - min_{v \in S} R(v) + \frac{TL^2}{\sigma}$$

    where $\frac{1}{T} \sum_{t=1}^T L_t^2 \leq L^2$.
\end{theorem}

\textcolor{red}{To add: derived regret bounds for euclidean and entropic regularizers}

\section{Online Mirror Descent}
a


% 

\begin{comment}
Mirror Descent:

Mirror descent with entropy regularization

Mirror descent with KL Divergence regularizations 


Mirror Descent Policy optimization (MDPO)

The update rule for on-policy MDPO is given by,

$\theta_{k+1} \leftarrow argmax_{\theta \in \Theta \psi(\theta, \theta_k)}$

$\psi(\theta, \theta_k) = \mathds{E}_{s ~ \rho_{\theta_k}}[\mathds{E}_{a~\pi_{\theta}}[A^{\theta_k}(S, a)] - \frac{1}{t_k} \textrm{KL}(s; \pi_{\theta}, \pi_{\theta_k})]$

\end{comment}

%\end{document}

\appendices
\newpage
\appendix

\chapter{Some Ancillary Stuff}

Ancillary material should be put in appendices.

\chapter{Some More Ancillary Stuff}

% Here is yet another appendix! Wahoo!

\cite{Farine20162243}

%\nocite{*}
\bibformb
\bibliography{BibFile}
\newpage
% \vita
% This is where the vita goes.  Its organization is left as an exercise.

\end{document}