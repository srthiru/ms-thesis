\chapter{Background}
In this section, we lay out some ideas, and concepts from Game Theory, Reinforcement Learning, Online learning, and Optimization that are relevant to this work.
We only introduce fundamental notions within these areas that are useful in following the thesis, and we direct the reader to more comprehensive resources for an in-depth discussion wherever necessary.
The framework of multi-agent learning has been conventionally rooted in Game Theory, and so we begin by discussing preliminary constructs from game theory and detail the setting that is pertinent to this work, namely, two-player zero-sum games.
We then discuss foundational concepts within reinforcement learning that serves as a base for the algorithms discussed in this work. We also talk about the current limitations of Reinforcement learning methods in multi-agent settings and motivate the need for the development of better algorithms in this setting.
Finally, we also present some preliminary concepts from online learning and optimization algorithms that are useful in understanding the approaches we study in this work.

\section{Game Theory}
Game theory~\cite{osborneintroduction2004} is the study of modeling interactions between
decision-makers.
Problems are represented as a game in which decision-makers are players taking actions that adhere
to a set of rules in accordance with some real-world situation.
Games intuitively capture the nature of the interaction, while abstracting away irrelevant details
and complexity of real-world problems that do not affect the outcome of interactions.
Game theory is widely applied in modeling problems that are of economic, social, and political
importance.
% In overcoming the challenges mentioned above, many algorithms have adopted game-theoretic
% constructs and ideas when designing Reinforcement learning algorithms for multi-agent settings.
% It is also a common practice to use game theoretic constructs in evaluating the performance of
% multi-agent algorithms and provide theoretical guarantees.
In this section, we briefly introduce concepts from game theory that are relevant to this work.
For an in-depth and more rigorous discussion of game theory and how it relates to problems in multi-agent learning, we refer the reader
to~\cite{nisanAlgorithmic2007a, shohamMultiagent2008}.

\subsection{Problems Formulations in Game Theory}
\label{subsec:reps}
% In discussing agent interactions, and preferences, we need a formal notion of how agents act, and
% how agent preferences can be defined.
The primary components of modeling a problem using game theory are \textit{actions} available to
the players, and a specification of each player's preferences over outcomes.
Players can choose from a set of available actions, and the availability of actions only depends on
the situation, not on a player's preferences.
Also, there is no restriction placed upon a player's preferences, and as such they can be
risk-preferring, risk-averse, or even altruistic.
It is also assumed that players follow \textit{rational choice}, i.e., given the same information,
an agent consistently chooses the same action irrespective of the qualitative nature of that
action.
The set of actions of all the players in a specific situation is jointly referred to as the
\textit{action profile}.
Preferences are formalized through a \textit{payoff function} (or utility function) that maps
outcomes to scalar payoffs (or utilities).
Payoff functions only induce an ordering over outcomes and are not reflective of the relative
intensity or magnitude of such preferences.

A \textit{strategy} describes how the player acts when given some information, and strategies can
be deterministic (\textit{pure strategies}) or distributions over pure strategies (\textit{mixed
	strategies}).
The expected payoff that can be achieved by a player following a given strategy is conditioned on
the strategies of all the other players.
% Players aim to learn strategies that maximize their expected payoff as they are being affected by.
While each player aims to learn strategies to maximize their expected payoffs, the expected payoffs
are themselves affected by the changing strategies, giving rise to the complexity in learning good
strategies.

\subsection{Game Representations}

Two fundamental game-theoretic representations are the normal form and the extensive form games.
% \subsubsection*{Normal-form Games}
% Normal form is the most fundamental form of representing problems in game theory.
In a normal-form (or a strategic) game, all players act simultaneously to reveal an outcome
immediately after, and a payoff is assigned to each player.
A more formal definition of a normal-form game is as follows:

\begin{definition}[Normal-form games] A n-player, general-sum normal form game is given by the $(N,
		A, u)$ tuple, where:
	\begin{itemize}
		\item $N = \{1 \dots n \}$ is the set of players
		\item $A =
			      A_1 \times A_2 \ldots \times A_n$, with $A_i$ being the set of actions available to player $i$
		\item $u = {u_i, \forall i \in N}$ is the set of payoff functions that map an action profile to a
		      real payoff value for each agent, $u_i: A \mapsto \R$.
	\end{itemize}
\end{definition}

For a normal-form game, the set of mixed strategies available to player $i$ is given by $S_i =
	\Pi(A_i)$.
The set of mixed-strategy profiles of all the players in the game is called the mixed strategy
profile $S = S1 \times \dots \times S_n$.
Normal-form games are typically represented as an n-dimensional matrix, where each dimension
corresponds to the pure-strategy space of each player, and the entries correspond to the payoffs of
action profiles.
For this reason, normal-form games are also referred to as \textit{matrix games}.
Canonical examples of normal-form-games from game theory literature include Matching pennies,
Battle of the sexes, Prisoner's dilemma, and Rock-paper-scissors (RPS), etc.

Normal-form games do not explicitly model the temporal nature of decision-making that occurs in
many real-world problems.
% \subsubsection*{Extensive-form Games} 
As a result, while these problems can still be modeled as NFGs, this representation can be
computationally expensive to work with.
These problems are more naturally modeled as an Extensive-form game (EFG), a representation that
explicitly incorporates the sequential nature of interactions in the form of a game tree.

\begin{definition}[Extensive-form games]
	An extensive form game is specified by the tuple $G=(N, A, H, Z, \chi, \rho, \sigma, u)$, where:
	\begin{itemize}
		\item $N = \{1 \dots n \}$ is the set of players
		\item $A$ is the set of actions
		\item $H$ is the set of non-terminal decision nodes
		\item $Z$ is the set of terminal nodes; $H \cap Z = \emptyset$
		\item $\chi: H \mapsto 2^A$ is the action function that assigns each decision node a set of actions available to the player
		\item $\rho: H \mapsto N$, is the player function
		\item $\sigma: H \times A \mapsto H \cup Z$ is the successor function
		\item $u = (u_i,
			      \forall i \in N)$ is the set of payoff functions that map terminal nodes to a real payoff value
		      for each agent, $u_i: Z \mapsto \R$.
	\end{itemize}
\end{definition}

The above representation definition assumes that all the relevant information for decision-making is
available and accessible to all the players equally.
This is known as the \textit{perfect-information} setting, as opposed to the
\textit{imperfect-information} setting that occurs in many real-world problems where relevant
information is obscured to the players.

\begin{definition}[Imperfect-information Extensive-form games]
	An imperfect-information extensive-form game, $G=(N, A, H, Z, \chi, \rho, \sigma, u, I)$ corresponds to a
	perfect-information extensive-form game $G=(N, A, H, Z, \chi, \rho, \sigma, u$), where
	$I = \{I_i, \forall i \in N\}$, and $I_i = (I_{i,1}, \dots, I_{i,k_i})$ defines a partition over all non terminal-histories
	$\{h \in H : \rho(h) = i\}$, such that
	$\chi(h) = \chi(h')$, and $\rho(h)=\rho(h')$, whenever $\exists j$ for which $h,h' \in I_{i,j}$.
\end{definition}

A crucial assumption in such a setting is a \textit{perfect recall} of the sequence of actions taken
by a player to reach the current node of the game tree.
If all players possess perfect recall, then the game is said to be of perfect recall.
While mixed strategies in extensive-form games are distributions over the pure strategies of the
entire game tree, \textit{behavioral strategies} are distributions over actions at each decision
node.
Normal-form representation of an extensive-form game is referred to as a reduced-NFG.
Here the dimensions of the matrix represent pure strategies over the entire sequence of
interactions, leading to an exponential blowup of the action space for the reduced-form NFGs.
A more efficient \textit{sequence-form} representation makes use of the varying depth of the game
tree by defining the payoff matrix with respect to the subsets of histories instead of
pure-strategy profiles.
The payoffs achieved in the terminal histories are passed up the trees to non-terminal histories as
well.

Beyond the above representations, many variants have been developed to add an account for specific
characteristics of problems.
More generally, definitions of game representations are typically based on the number of agents
(Single-agent, Two-player, Multi-agent), the length of the game (single-shot, sequential), type of
interaction (simultaneous-move, turn-based), utility values of players (general-sum, constant-sum,
zero-sum), etc. In this work, we mainly deal with two-player zero-sum games (interchangeably referred
to in the text as 2p0s), a purely competitive setting that only involves interactions between two
agents whose payoff functions are a negation of each other.
\begin{definition}[Two-player Zero-sum games]
	A two-player normal-form game is zero-sum if for each strategy profile $a \in A_1 \times A_2$,
	$u_1(a) + u_2(a) = 0$.
\end{definition}
Two-player zero-sum games are an important class of problems that coincide with other important
domains including saddle point optimization, variational inequalities, and image generation in deep
learning.

\subsection{Solution Concepts}
% The above representations provide a well-defined framework to formalize problems in a game-theoretic manner.
Solution concepts are typically \textit{equilibrium} in the joint strategy space of the players
that define a notion of optimality of a player's strategies and allow us to analyze player
behaviors.

% \textbf{Best Response:}
The best response of player $i$ to the strategy profile $s_{-i}$ is a mixed strategy such that
$u_i(s_i^\ast, s_{-i}) \geq u_i(s_i, s_{-i}), \forall s_i \in S_i$.

\begin{definition}[Nash equlibrium]
	A strategy profile $s=(s_1, \dots, s_n), s \in S$ in an n-player normal-form game is a Nash-equilibrium if $s_i$ is a best response to $s_{-i}$, for all
	$i$.
\end{definition}

A major result in game theory ensures the existence of a Nash equilibrium for any finite n-person
normal-form game~\cite{nashEquilibrium1950}, and extensive-form games with perfect
recall~\cite{kuhnExtensive1950}.
While Nash equilibrium is the most prevalent solution concept, various other solution concepts with
interesting properties have also been proposed and studied.
Minmax and Maxmin strategies assure payoffs for players in the most aggressive (player $i$ playing
to minimize player $j$'s payoffs) and most conservative (player $-i$ playing to minimize player
$i$'s payoffs) scenarios.

While the solution concept of Nash equilibrium assumes perfect rationality, the Quantal response
equilibrium (QRE) models agents with bounded rationality allowing for some errors in action
predictions.
QRE was first proposed for normal-form games~\cite{mckelveyQuantal1995}, and then later extended to
extensive form games using the sequence form representation~\cite{mckelveyQuantal1998}.
An instance of QRE is the logit-QRE (which we simply refer to as QRE going forward) where the
errors are modeled using a Weibull distribution and added to the payoffs of each agent in a
normal-form game.
For extensive-form games, if the behavioral strategies of all agents are a QRE, then the joint
policy is said to be an Agent QRE (AQRE).
QRE predicts systematic deviations from Nash equilibrium when fit with experimental data of
game-theoretic problems.

\section{Reinforcement
  Learning}
Reinforcement learning (RL)~\cite{suttonReinforcement2018} deals with designing algorithms for
\textit{agents} that interactively learn to maximize a \textit{reward} signal in an
\textit{environment}.
The reward signal encodes information about the goal that the agent has to learn to achieve without
any specific directions about how that goal should be achieved.

\textbf{Markov Decision Process.}
Markov Decision Processes (MDPs) provide a framework to formalize reinforcement learning problems.
A $\gamma$-discounted MDP ($\mathcal{M}$) is given by the tuple, $\mathcal{M}=(\calS, \calA, P, R,
	\gamma)$ where: $\calS$ is the state space, $\calA$ is the action space,
$P(s',r|s,a):~\calS\times\calA\times\calS~\mapsto~[0,1]$ is the transition dynamics function,
$R(s,a) \subset \R$ is the reward function, $\gamma \in [0,1]$ is the discount factor.
In this work, we discuss finite MDPs where the sequence of interactions is characterized as
episodes that have a terminal state.
The expected return, $G_t$, at time step $t$ is the discounted sum of rewards accumulated by the
agent starting from $t$ till the end of the episode.
Formally, $G_t = \sum_{k=t+1}^{T} \gamma^{k-t-1} R_k$, where the discount rate $\gamma$ indicates
the importance placed upon future rewards over immediate rewards.
The objective of an agent is to learn to act in a way that maximizes its \textit{expected return}.
A \textit{Policy} is a mapping from a state to an action distribution $\pi: \calS \mapsto \calA$.
A \textit{Value function}, $V_{\pi}(s)$ estimates the expected reward that can be achieved from the
current state under the policy $\pi$: $V_{\pi}(s) = \E_{\pi}[G_t | S_t = s]$.
Analogously, the \textit{Q-value function} estimates the expected reward of taking a specific
action $a$ at a given state $s$ and then following the policy $\pi$: $ Q_{\pi}(s, a) = \E_{\pi}[G_t
		| S_t = s, A_t = a] $.
The difference between Q and V functions is referred to as the advantage function $A_\pi(s,a) =
	Q_\pi(s,a) - V_\pi(s)$.
This is the advantage of taking a particular action over following the average policy.

Value functions impose a ranking over policies in terms of the expected return achieved by an agent
following a given policy.
An \textit{optimal policy} ($\pi^\ast$) is one that is at least as better as any other policy ($\pi
	\in \Pi$) in terms of this ranking, i.e. $V_{\pi^\ast}(s) \geq V_\pi(s), \forall s \in \calS$.
While there may be more than one optimal policy, they all share the same unique \textit{optimal
	value function}: $V_{\pi^{\ast}} = \max_{\pi} V_\pi(s), \forall s \in \calS$.

\subsection{Tabular methods}
For a given policy, the value function satisfies the \textbf{Bellman equation}:
\begin{equation}
	\label{eqn:bellman} V_\pi(s) \doteq \E_\pi \left[r + \gamma V_\pi(s') | R_t = r, S_{t+1} = s'
		\right]
\end{equation} For small state spaces, it is possible to learn optimal value functions by
iterating using Dynamic programming.
Starting with an arbitrary policy, we can learn an optimal policy interleaving Policy evaluation,
and Policy improvement.
In the \textit{Policy Evaluation} step, we first estimate the value function until it
satisfies~\ref{eqn:bellman}, i.e., $v_{k+1}(s) \doteq \E_\pi \left[R_{t+1} + \gamma v_k(S_{t+1})
		|S_t =s\right]$.
Next, in the \textit{Policy Improvement} step, we update the policy greedily at every state, i.e.,
$\pi'(s) \doteq \arg \max_{a} Q_\pi(s,a)$.
The Policy improvement theorem guarantees that this new policy is strictly better than the old
policy, unless the policy has converged.
This method is known as \textit{Policy Iteration}, and if the policy evaluation is approximated
instead of exact convergence, then it is known as \textit{Value Iteration}.

Most of the tabular, and iterative methods require an explicit model of an environment with access
to the transition functions.
In the absence of a model or transition functions, one can use Monet Carlo methods to approximate
these probabilities through sampling.
Tabular methods are realizable in settings with small state spaces, where we have access to the
the perfect model of the environment in the form of MDPs.
Tabular methods are also useful in establishing theory and guarantees for various algorithms.
However, for most practical applications it is common to use parameterized policies and approximate
value functions.
Policy gradient methods allow for directly learning a parameterized policy that enables action
selection without the use of a value function.
Policy gradient methods have the advantage that in many problem settings the policy space could be
relatively simpler to approximate compared to the value function space.
We discuss Policy gradient methods in more detail in the next chapter as it serves as a base for
the rest of the thesis.
We refer the reader to~\cite{suttonReinforcement2018} a more detailed discussion on tabular
methods, and approximate value functions.

\section{Multi-agent Reinforcement Learning (MARL)}\label{sec:marl}
% While game theory provides a framework for defining problems that are multiagent in nature and
% optimality of solutions to those problems, reinforcement learning can be leveraged to learn
% such optimal strategies.
Single-agent RL assumes the world to be Markovian, and this assumption is still valid in many
real-world settings, particularly when the environment can be isolated from the outside world.
Advancements in theoretical understanding, development of algorithms for learning function
approximators and the availability of computational resources have led to a resonating success of
reinforcement learning in tasks with high-dimensional state spaces, and a diverse range of
applications including game playing, nuclear reactor control, video compression, assisting surgeons
with anesthesia control, and aligning models with human preferences~\fillin{citations}.
However, often in real-world problems, there are multiple agents with objectives that can be
aligned or indifferent, or adversarial to each other.
Studying and designing learning algorithms for agents in this problem setting is termed
\textit{Multi-agent learning (MAL)}~\cite{tuylsMultiagent2012}.
The area of Multi-agent reinforcement learning (MARL) aims to design reinforcement learning
algorithms that are performant in the multi-agent setting, with strong theoretical guarantees, and
empirical performance.

There are multiple approaches that have been explored in designing RL algorithms that are
performant in the multiagent setting.
Independent MARL treats every other agent as a part of the environment and tries to maximize the
reward for an agent.
This approach, although not grounded in theory, is a standard baseline approach for problem domains
without theoretically grounded or empirically proven algorithms.
Other approaches include designing fully-decentralized learning algorithms and centralized
algorithms that either share information only during the training, or both during training and
execution.
Among these, the Centralized Training Decentralized Execution (CDTE) regime has been widely
adopted especially in cooperative settings.
Another focus in cooperative or mixed-cooperative settings is to encourage agents to learn to
communicate information while satisfying some communication bottlenecks.
However, there is still a huge gap to fill in designing algorithms that support practical
implementation, strong performance in real-world applications, and well-rooted in theory.

\subsection{Challenges in MARL}
Beyond the problems that are prevalent in single-agent settings such as sample efficiency, and
variance of updates, there is a new range of problems that arise in the multiagent setting.
Many of the theoretical guarantees behind reinforcement learning, no longer hold in the multi-agent
setting as the environment can be non-Markovian due to the presence of other agents that affect the
reward an agent gets for taking the same set of actions.
Also, the joint action space explodes exponentially in the number of agents making it
computationally challenging to apply reinforcement algorithms.
In cooperative settings, as agents act toward a common goal it becomes harder to discern the
actions that led to some global reward, which is known as the credit assignment problem.
Hence, designing algorithms that tackle the above issues is critical for better performance in many
applications.
To this end, there have been many attempts at designing MARL algorithms that borrow from game
theory, online learning, evolutionary biology, and economics.

\section{Online Learning, and Convex Optimization}
As discussed in previous sections, problems in Game theory and Reinforcement learning typically
involve learning strategies or policies in an online manner through interaction between agents and
environment.
Online Learning~\cite{shalev-shwartzOnline2012} studies the problem of learning to make accurate
predictions in an online manner with an aim to minimize a potentially adversarial loss function.
Formally, an Online learning problem is stated as follows:

%\begin{noindent}
\begin{definition}[Online Learning] 
	\label{def:olearning} 
	At each round $t$, given an instance or a question $x_t \in \calX$, a learner makes a prediction $p_t \in \mathcal{D}$, 
	and is revealed the answer $y_t \in \calY$. The learner receives a scalar loss according to some loss function $l(p_t, y_t) \mapsto \R$. 
	Here, $\calX$ is the instance domain, $\calY$ is the target domain, and $\mathcal{D}$ can be $\calY$ or some superset of $\calY$.
\end{definition}
%\end{noindent}

The answer for each round can be chosen by some
adversary before observing the learner's prediction.
The goal of an online learning algorithm $A$ is to make predictions that are competitive against
the best-fixed hypothesis from some hypothesis class ($\calH$).
The relative performance of $A$ with respect to some fixed \textit{competing hypothesis} ($h^\ast$)
is called its \textit{regret}:
\begin{equation}
	\label{eqn:regret} Regret_T(h^*) = \sum_{t=1}^T
	l(p_t, y_t) - \sum_{t=1}^T l(h^*(x_t), y_t),
\end{equation} Then, the regret of $A$ with respect to
the hypothesis class $\calH$ is,
\begin{equation}
	\label{eqn:regret_all} Regret_T(\calH) =
	\max_{h^* \in \calH} Regret_T(h^*)
\end{equation}

If the loss of an Online
learning grows sublinearly with each round then, the learner's average regret goes to zero as $t$
tends to infinity.
Algorithms that possess this property are called \textit{no-regret} algorithms and are of great
interest in online learning, learning strategies in games, and other prediction problems.
If the learner is restricted to deterministic predictions at each round, it can be shown that there
problems where the learner's regret grows unbounded~\cite{coverBehavior1965}.
Hence, an important relaxation is to allow the learner to make randomized predictions and the adversary has to choose the answer $y_t$ without knowledge of the learner's randomization for that
round.

\subsection{Convex Optimization}

We begin this section with some relevant definitions for discussing convex optimization algorithms.

% Convex sets, convex functions, convexity?%
% Strongly Convex
\begin{definition}[$\mu$-Strongly-Convex]
	~\label{def:strconvex}
	A convex function $f: S \mapsto \mathbb{R}$ is $\mu$-strongly-convex with respect to a norm $\|.\|$ if for any $w \in S$
	\begin{equation}
		\forall z \in \partial f(w), \forall u \in S, f(u) \geq f(w) + \langle z, u - w\rangle + \frac{\mu}{2}\| u - w \|^2.
	\end{equation}
\end{definition}

% Strongly smooth/L-smooth
\begin{definition}[$\sigma$-Strongly-Smooth]~\label{def:strsmooth}
	Given a convex set $\calX \in \R^n$, a convex function $f: \calX \mapsto \R$ is
	$\sigma$-strongly-smooth with respect to a norm $\|.
		\|$, if $\|\nabla f(x) - \nabla f(y) \|_{\ast} \leq \sigma \|x - y \|, \forall x, y \in \calX$.
	For a given constant $L$, this is also referred to as $L-smooth$.
\end{definition}

% Monotonicity/Strong monotonicity

It is beneficial to use tools from the well-studied area of Convex Optimization in designing and
analyzing online learning algorithms.
In this case, predictions are vectors that belong to a convex set $w_t \in S$, and the learner's
loss at each round is given by some function, $f_t: S \mapsto \mathbb{R}$, is convex in the
$\text{dom}(S)$.
Similarly, the competing hypothesis is also a vector from a convex set $u \in U$, and the
Regret~(\ref{eqn:regret}) is redefined as follows:
\begin{equation}
	Regret_T(u) = \sum_{t=1}^T
	f_t(w_t) - \sum_{t=1}^T f_t(u)
\end{equation} And, the regret with respect to a set of all
competing vectors in $U$ is,
\begin{equation}
	Regret_T(U) = \max_{u \in U} Regret_T(u)
\end{equation}

A common assumption is that $U=S=\mathbb{R}^d$.
The convex optimization framework is useful in deriving performance guarantees for Online Learning
algorithms under these more restrictive assumptions.
Also, for a non-convex problem, there are various ways one can convexify the objective with the loss of
some accuracy in the solution to take advantage of algorithms with better performance guarantees.
A preliminary algorithm is to choose the prediction vector that minimizes the cumulative loss of
all previous rounds, this is called \textit{Follow-the-leader}(FTL).
$$ \forall t, w_t = \arg \min_{w \in S} \sum_{i=1}^{t-1} f_i(w)$$
As it turns out, this algorithm can have constant regret for some
adversarial loss functions due to its unstable predictions.

\subsection{FTRL}\label{sec:ftrl}
Follow-the-Regularized-leader (FTRL) extends FTL by including a regularization term to stabilize
the predictions at each round.
The learning rule can be written as, $$\forall t, w_t = \arg \min_{w \in S} \sum_{i=1}^{t-1} f_i(w)
	+ R(w).
$$
where $R: S \mapsto \R$ is the regularization function.
The choice of the regularization function leads to different algorithms with varying regret bounds.
For instance, with a euclidean norm regularization, $R(w) = \frac{1}{2 \eta} \|w\|_2^2$, the FTRL
update becomes:
\begin{equation}
	w_{t+1} = w_t - \eta \nabla f_t(w_t)
\end{equation} leading to the
familiar gradient descent udpate, referred to as the Online Gradient Descent.

For constrained problems (e.g., the predictions have to be a valid probability distribution), it is
necessary to use non-euclidean regularization functions as FTRL with Euclidean regularization (OGD)
is no longer applicable.
Mirror descent is a generalization of FTRL that relies on link functions or mirror maps to simplify
FTRL updates.
For the discussion of mirror descent, we move away from the framework of Online learning, and state
the ideas from an optimization perspective following-\cite{beckMirror2003a,beckFirstOrder2017}.

\subsection{Mirror Descent}
Consider the following convex optimization problem:
\begin{equation}
	\label{eqn:convopt} x^{\ast} =
	\min_{x \in X} f(x)
\end{equation} where $X \subseteq \R^n$ is closed and convex, and $f: X \mapsto
	\R$ is convex, $L$-Lipschitz with respect to some norm $\|.
	\|$.

Given some $u \in \R^n$, the Euclidean projection of $u$ to $X$ is given by:
\begin{equation}
	\label{eqn:eucproj} P_\calX(u) = \arg \min_{x \in \calX} \| u - x \|.
\end{equation}

A popular approach to solve~\ref{eqn:convopt} is using the \textbf{Projected subgradient method},
by iteratively applying the following update:
\begin{equation}
	\label{eqn:projsubg} x_{k+1} =
	P_{\calX}(x_k - t_k f'(x_k))
\end{equation} where $f'(x_k) \in \partial f(x_k)$ is a subgradient.
Mirror Descent is an extension of the projected subgradient method~\ref{eqn:projsubg} to
non-euclidean spaces that utilize Bergman divergences instead of the Euclidean norm.

\begin{definition}[Bergman Divergence]
	\label{def:bregman}
	Given a convex set $\calX \subset \R^n$, and a differentiable $\sigma$-strongly convex potential
	function $\phi: \calX \mapsto \R$, the Bregman Divergence associated $\phi$ is defined as, $$
		B_\phi(x, y) = \phi(x) - \phi(y) - \langle \nabla \phi(y), x-y \rangle.
	$$
\end{definition}

Bergman Divergences are not true norms, as they can be asymmetric, and need not satisfy the
triangle inequality.
Given a strongly-convex potential function $\phi$, the mirror descent update rule is as follows:
\begin{equation}
	x_{k+1} = \arg \min_{x \in \calX} \left \{ \left \langle t_k \nabla f(x_k), x
	\right \rangle + B_\phi(x, x_k) \right \}.
\end{equation}

%\end{noindent}
One can recover the original projected subgradient method by letting $\phi=\|.
	\|$.

\subsection{Proximal gradient methods}
Consider the following convex optimization problem with a composite objective:
\begin{equation*}
	x^{\ast} = \min_{x \in X} f(x) + g(x)
\end{equation*} where $f$ is $L$-smooth $g$ is closed and
convex.
Applying projected subgradient method to the above problem results in the following update:

$$ x_{k+1} = \arg \min_{x \in \calX} \left\{ t_k g(x) + \frac{1}{2} \| x - (x_k
	- t_k \nabla f(x_k)) \|^2 \right\}$$.

\begin{definition}[Proximal operator] The
	proximal operator or proximal mapping of a function $f$ is given by:
	$$\text{prox}_{t_k, g}(y_k) = \arg \min_{x \in \calX} \left\{ t_k g(x) + \frac{1}{2}\| x - y_k \|^2 \right\} $$.
\end{definition}

Simplifying the update rule using the proximal operator results in the \textbf{Proximal gradient
	method}:

$$ x_{k+1} = prox_{t_k, g}(x_k - t_k \nabla f(x_k))$$.

Analogous to the extension of the projected subgradient method to non-euclidean settings, the proximal
gradient method can also be extended to non-euclidean settings by substituting the Euclidean
distance in the proximal operator with a Bergman divergence resulting in the \textbf{Non-euclidean
	proximal gradient method}~\cite{tsengApproximation2010,beckFirstOrder2017}.

\subsection{Convergence analysis}
In discussing learning and optimization algorithms, it is of theoretical and practical interest to
understand the convergence guarantees of iterative methods to ascertain the efficiency and
computational demands in solving various problems.
Analysis of such methods typically aims to bound performance measure of a sequence produced by the
the algorithm at each iteration against some solution point.
Performance can be measured by the approximation accuracy ($\epsilon$) of the current iterate
against some solution point or by establishing the asymptotic rate of convergence as a function of
the number of iterations ($k$).
Convergence rates in general can be sublinear (e.g., $O(1/k)$ or $O(1/\epsilon)$), linear (e.g.,
$O(q^k)$ for some $q \in [0, 1]$ or $O(\log{(1/\epsilon)})$), superlinear, or quadratic.

\subsection{Regret Minimization and CFR}
Regret minimization is an important class of algorithms for problem learning in games.
This approach aims to derive algorithms that have no-regret properties.
\fillin{CFR}