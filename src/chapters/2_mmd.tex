\chapter{Magnetic Mirror Descent}

\subsection{Mirror Descent}




\subsection[MMD]{Magnetic Mirror Descent}

Magnetic mirror descent (MMD) \cite{sokotaUnified2023} is a last-iterate equilibrium approximation algorithm
for two-player zero-sum games that is an extension of mirror descent with entropy regularization.

The idea behind MMD begins with the observation that solving for QRE in two-player zero-sum games can
be reformulated as the solution to a negative entropy regularized saddle point problem as follows,

\begin{equation}\label{eqn:saddle}
    \min_{x \in \mathcal{X}} \max_{y \in \mathcal{Y}} \alpha g_1(x) + f(x, y) + \alpha g_2(y),
\end{equation}

where $\mathcal{X} \subset \mathbb{R}^n$, $\mathcal{Y} \subset \mathbb{R}^m$ are closed and convex, and
$g_1: \mathbb{R}^n \rightarrow \mathbb{R}$,
$g_2: \mathbb{R}^m \rightarrow \mathbb{R}$,
$f : \mathbb{R}^n \times \mathbb{R}^m \rightarrow \mathbb{R}$.

The solution $(x_{\ast}, y_{\ast})$ to \ref{eqn:saddle}, is a Nash equilibrium in the reguarlized game
with the following best response conditions,

\begin{equation}\label{eqn:bestrescon1}
    x_{\ast} \in \arg \min_{x \in \mathcal{X}} \alpha g_1(x) + f(x, y_{\ast})
\end{equation}

\begin{equation}\label{eqn:bestrescon2}
    y_{\ast} \in \arg \min_{y \in \mathcal{Y}} \alpha g_2(y) + f(x_{\ast}, y)
\end{equation}


\subsection*{Zero-sum games and Variational Inequalities}
MMD reframes the solution to QRE as a variational inequality problem.

\begin{definition}
    Given $\calZ \subseteq \R^n$ and mapping $G: \calZ \rightarrow \R^n$, the variational inequality problem
    VI($\calZ, G$) is to find $z_{\ast} \in \calZ$ such that,

    $$\langle G(z_{\ast}, z - z_{\ast}) \rangle \geq 0 \quad \forall z \in \calZ. $$
\end{definition}


The optimality conditions are equivalent to VI$(\calZ, G)$, where $G = F + \alpha \nabla g$,  $\calZ = \mathcal{X} \times \mathcal{Y}$, and
$g: \calZ \rightarrow \mathbb{R}$.

Now, the solution the the VI problem ($z_{\ast}= (x_{\ast}, y_{\ast})$), corresponds to the
solution of the saddle point problem stated in \ref{eqn:saddle}, and satisfies the best response conditions
\ref{eqn:bestrescon1} and \ref{eqn:bestrescon2}.

The algorithm that the authors propose is a non-Euclidean proximal gradient method to solve the VI($\calZ, F + \alpha \nabla g$) problem.

We now restate the main algorithm as stated in \cite{sokotaUnified2023}(3.1),

\begin{alprocedure}
    \algorithmcfname{MMD}\label{alg:proxgrad}
    Starting with $z_1  \in \text{int dom } \psi \cap \calZ$, at each iteration $t$ do
    $z_{t+1} = \arg \min_{z \in \calZ} \eta (\langle F(z_t), z\rangle + \alpha g(z)) + B_{\psi} (z, z_t).$
\end{alprocedure}

With the following assumptions, $z_{t+1}$ is well defined:

\begin{itemize}
    \item $\psi$ is 1-strongly convex with respect to $\|.\|$ over $\calZ$, and for
          any $l$, stepsize $\eta > 0$, $\alpha > 0$,
          $z_{t+1} = \arg \min_{z \in \calZ} \eta (\langle l, z \rangle + \alpha g(z)) + B_{\psi}(z;z_t) \in \text{int dom } \psi.$

    \item $F$ is monotone and $L$-smooth with respect to $\|.\|$ and $g$ is 1-strongly convex
          relative to $\psi$ over $\calZ$ with $g$ differentiable over int dom $\psi$.
\end{itemize}

Algorithm \ref{alg:proxgrad} provides the following convergence guarantees,

Assuming that the solution $z_{\ast}$ to the problem VI($\calZ, F + \alpha \nabla g$) lies in the int dom $\psi$, then
the algorithm \ref{alg:proxgrad} guarantees

$$ B_{\psi} (z_{\ast}; z_{t + 1}) \leq \left(\frac{1}{1 + \eta \alpha}\right)^t B_{\psi} (z_{\ast}; z_1),$$

if $\alpha > 0$, and $\eta \leq \frac{\alpha}{L^2}$.




MMD is also a first attempt in developing algorithms that are performant in both
Single-agent and Multi-agent settings.

In single-agent RL MMD's performance is competitive with that of PPO in Atari and MuJoCo environments
(though not evaluated extensively). And, in the multi-agent setting the performance of tabular MMD is on par
with that of CFR, but worse than that of CFR+.