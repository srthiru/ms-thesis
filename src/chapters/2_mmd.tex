\chapter{Mirror Descent in Reinforcement Learning}

Given the generality of the mirror descent algorithm as a first order optimization method, there
has been continued efforts to incorporate it as a single/multi-agent reinforcement learning
algorithm.
Extensive studies of mirror descent under various settings and assumptions help in deriving much
needed theoretical guarantees for mirror descent based reinforcement learning algorithms in terms
of sample complexity, and convergence rates.
In this work, we study two such algorithms, namely MDPO (Mirror Descent Policy Optimization), and
MMD (Magnetic Mirror Descent).
We first begin with a discussion of PG methods in RL to set a base for the rest of the chapter
before moving on to the above algorithms.
% For the following discussion, and the experiments, we consider a class of parameterized stochastic policies,
% i.e., $\Pi = \{\pi_{\theta}, \theta \in \Theta\}$.
% We also only consider on-policy learning.

% \blue{RL objective is not convex; can still use MD..
% 	\cite{shaniAdaptive2020}}

\section{Policy Gradients}

\label{sec:spg} \subsection{Policy gradient methods}\label{sec:pg}

Policy gradient method involve directly learning a parameterized policy that enables action
selection without the use of a value function.
These are generally called Policy gradient methods, and are a major area of study within
Reinforcement learning.
Policy gradient methods sometimes have the advantage that the policy space could be simpler to
learn compared to the value function space.
In this case, the policy is parameterized by $\theta \in \Theta$ and $\pi(s, a) = P(s, a; \theta)$.

Given some objective $J(\theta)$, these methods seek to learn the parameters that maximize this
objective through gradient ascent.

One key challenge in Policy gradient methods is that the evaluation of performance of a policy
depends on the state distribution which could be unknown.
However, the Policy Gradient Theorem establishes that the gradient is independent of underlying
environment's state distribution as long as it is stationary conditioned on the current policy.
The most popular policy gradient method is Reinforce uses monte-carlo estimations to approximate
the value estimation.
A value function may still be used in guiding the policy learning, and these are called
actor-critic methods.
Here the actor refers to the policy, and the critic refers to the value function that evaluates the
action taking by the policy guiding the policy learning.

\textbf{Reinforce}:
\blue{
	- Widely popular PG algorithm, many insantiations possible including variants with baselines to
	reduce variance.
	- Convergence is proven using PG theorem.
}

\textbf{Softmax Policy Gradients}
\blue{
	- Parametrizing policies
	- PG with softmax parameterization is referred to as Softmax Policy gradients.
	- General significance of softmax
}

\subsection{PPO}

- Trust region methods
- PPO an approximation of TRPO with hueristic objective

\section[MDPO]{Mirror Descent Policy Optimization}

The first method we discuss is the Mirror Descent Policy Optimization~ \cite{tomarMirror2022}
(MDPO).
MDPO tackles the problem developing a stable reinforcement learning algorithm through the framework
of trust-region optimization.
Trust-region optimization stabilizes learning by constraining the policy update at each iteration.
Instances of trust region optimization methods such as PPO~ \cite{schulmanProximal2017}, TRPO~
\cite{schulmanTrust2015} and their variants have found large success in single agent RL with
state-of-the-art performances in a wide array of tasks.

Due to the presence of a proximal regularization, mirror descent in itself can be interpreted as a
trust-region optimization method.
%cite reference?
Building on top of existing theoretical guarantees for mirror descent in tabular RL settings, MDPO
derives practical RL algorithms that is performant in function approximation settings.
Using the formulation of a parameterized policy from~ \ref{sec:pg}, MDPO performs the following
update at each iteration:
\begin{equation}
	\label{eqn:mdrl} \pi_{k+1}(.
	|s) \leftarrow \arg
	\max_{\pi \in \Pi}
	\E_{a \sim \pi} [A_{\pi_k}(s,a)] -
	\frac{1}{t_k}
	KL(s; \pi, \pi_k)
\end{equation}

This update is approximately solved at each
iteration by taking a few stochastic gradient steps.
The gradient of the KL component of the above objective for one step of SGD is zero, and hence it
is necessary to take multiple gradient steps every iteration to correctly approximate the mirror
descent objective.
Tomar et.al, derive on-policy and off-policy variants of MDPO, and in this work we mainly focus on
the on-policy algorithm.

For a parameterized policy $ \pi_{\theta}$ the on-policy MDPO update is given by: $$ \theta_{k+1}
	\leftarrow \arg \max_{\theta} J(\theta, \theta_k)$$ $$ \text{where}, J(\theta, \theta_k) = \E_{s
		\sim \rho_{\theta_k}} [ \E_{a \sim \pi_\theta}[A_{\theta_k}(s,a)] - \frac{1}{t_k} KL(s; \pi_\theta,
		\pi_{\theta_k}]$$

For $m$ steps MDPO uses the following gradient to update the
policy parameters,
\begin{equation}
	\label{eqn:mdpograd} \nabla_{\theta} J(\theta,
	\theta_k)|_{\theta = \theta_k^{(i)}} = \E_{s \sim \rho_{\theta_k} \\ a \sim \theta_k} [ \frac{
			\pi_{\theta_k}^{(i)}}{ \pi_{\theta_k}} \nabla_{\theta} \log \pi_{\theta_k}^{(i)} (a|s)
		A_{\theta_k}(s,a)] - \frac{1}{t_k} \E_{s \sim \rho_{\theta_k}} [\nabla_\theta KL(s; \pi_\theta,
		\pi_{\theta_k})_{|\theta= \theta_k^{(i)}}]
\end{equation} where $i=0,1,\ldots,m-1$.

% \textbf{Gradient of KL-Divergence}

% Let us consider a reference policy $Q$ and $P_{\theta}$, a policy that is being optimized. 
% The reverse KL-Divergence between the two is given by

% $$
% KL(P_{\theta} \| Q) = \sum_{a \in A} P_{\theta}(a) \log \frac{P_{\theta}(a)}{Q(a)}
% $$

% The gradient of KL-Divergence with respect to the parameters theta is given by:

% \begin{equation*}
% 	\begin{split}
% \nabla_{\theta} KL &= \sum_{a \in A} \nabla_{\theta} [P_{\theta}(a) \log P_{\theta}(a)] - 
% 						\nabla_{\theta} [P_{\theta}(a) \log Q(a)] \\
% 						&= \sum_{a \in A} [\nabla_{\theta} P_{\theta}(a) \log P_{\theta}(a) + \nabla_{\theta} P_{\theta}(a)] -  
% 						 \nabla_{\theta} P_{\theta}(a) \log Q(a) \\
% 						&= \sum_{a \in A} \nabla_{\theta} P_{\theta}(a) (\log P_{\theta}(a) - \log Q(a) + 1)
% 	\end{split}
% \end{equation*}

MDPO strong connections to the other constrained optimization RL algorithms and entropy regularized
algorithms.
We refer to~ \cite{tomarMirror2022} for a detailed discussion of connections between on-policy MDPO
to PPO and TRPO, and off-policy MDPO to SAC.
% MDPO has been shown to have better empirical performance than PPO, and TRPO in continuous control tasks, and Atari environments.

\subsection{MDPO in MARL}
Although MDPO exhibits strong empirical performance in single-agent RL, it is not directly
extendable to multi-agent settings due to the non-stationarity of the dynamics.

We demonstrate the behavior of MDPO using the Perturbed RPS problem discussed in Section~
\ref{subsec:reps}.
Normal form games allow for tabular policy representations with direct or softmax
parameterizations.
Algorithms can typically use exact value estimations for all possible actions (payoffs) given the
opponent policy, as opposed to a sample-based expected value estimations.
This is referred to as the all-actions setting (also known as \textit{full-feedback} or first-order
information setting).
In this setting, the gradient computation in~ \ref{eqn:mdpograd} becomes: \[ \nabla_{\theta}
	J(\theta, \theta_k)|_{\theta = \theta_k^{(i)}} = \nabla_{\theta} \sum_{a \in A} [
		\pi_{\theta_k}^{(i)} (a) A_{\theta_k}(a)] - \frac{1}{t_k} [\nabla_\theta KL(\pi_\theta,
		\pi_{\theta_k})] \]

From the trajectories in Fig \blue{add trajectory plots} we
can see that MDPO fails to converge to a Nash equilibrium (marked by a red dot).
In the next chapter we discuss ways to adapt MDPO for last-iterate convergence in two-player
zero-sum games.

\section[MMD]{Magnetic Mirror Descent}

Another extension of Mirror Descent to reinforcement learning is Magnetic Mirror Descent~
\cite{sokotaUnified2023}, that attempts to create a unified approach to work well in both single
and multiagent settings.
This work studies the relation between equilibrium solving and Variational inequalities with
composite structures.
Taking advantage of this connection, a equilibrium solving algorithm has linear last-iterate
convergence guarantees is proposed.
Moreover, this approach extends well as a reinforcement learning algorithm in single agent, and
multiagent settings.
In this section, first we outline the connection between Varational inequalities and equlibrium
solving as presented in~ \cite{sokotaUnified2023}.
Then we outline the Magnetic Mirror Descent algorithm and its convergence properties.

\subsection{Connection between Variational Inequalities and QREs}
A Variational Inequality (VI) problem, written as $VI(Z, F)$ is generally defined as follows:
\begin{definition}
	\label{def:vi} Given $\calZ \subseteq \R^n$ and mapping $F: \calZ \rightarrow
		\R^n$, the variational inequality problem VI($\calZ, F$) is to find $z_{\ast} \in \calZ$ such that,
	\[ \langle F(z_{\ast}), z - z_{\ast} \rangle \geq 0 \quad \forall z \in \calZ.
	\]
\end{definition}

The VI problem described above is very general, and as such a wide range of problems can be cast
into this framework~ \cite{facchineiFiniteDimensional2004}.
We mainly focus on the relation between VI problems with a similar structure, and QREs.

Finding the QRE of a two-player zero-sum game can be represented as the following
entropy-regularized saddle point problem.
Given $\calX \subseteq \R^n$, $calY \subseteq \R^m$, and $g_1: \R^n \mapsto \R$, $g_2: \R^m \mapsto
	\R$, find:
\begin{equation}
	\label{eqn:saddle} \min_{x \in \mathcal{X}} \max_{y \in \mathcal{Y}}
	\alpha g_1(x) + f(x, y) + \alpha g_2(y),
\end{equation}

The solution
$(x_{\ast}, y_{\ast})$ to the saddle point problem~ \ref{eqn:saddle} has the following first-order
optimality conditions:
\begin{equation}
	\label{eqn:optcon}
	\begin{split}
		\langle \alpha \nabla
		g_1(x_*) + \nabla_{x_*} f(x_*, y_*), x - x_* \rangle \geq 0, \forall x \in \calX.
		\\
		\langle \alpha \nabla g_2(y_*) +
		\nabla_{y_*} f(x_*, y_*),
		y - y_* \rangle \geq 0, \forall y \in \calY.
	\end{split}
\end{equation}

The reguarlized saddle point problem~ \ref{eqn:saddle} of solving for QREs is equivalent to the VI
problem with the following composite objective $G = F + \alpha \nabla g$, where $\calZ =
	\mathcal{X} \times \mathcal{Y}$, $F(z) = [\nabla_x f(x,y) - \nabla_y f(x,y)]$, and $\nabla g =
	[\nabla_y g_1(x), \nabla_x g_2(y)]$.
The optimality conditions~ \ref{eqn:optcon} are equivalent to the VI$(\calZ, G)$, and thus the
solution to the VI: $z^* = (x^*, y^*)$ is also the solution to the saddle point problem stated in~
\ref{eqn:saddle}.

\subsection{MMD Algorithm}

Various algorithms have been proposed to solve the VI problem~ \ref{def:vi}.
In particular, the proximal point method has linear last iterate convergence for Variational
inequality problems with a monotone operator~ \cite{rockafellarMonotone1976}.
This algorithm was extended to composite objectives~ \cite{tsenglinear1995}, and to non-euclidean
spaces with Bergman divergence as a proximity measure, that allows for non-euclidean proximal
regularization~ \cite{tsengApproximation2010}.

The non-euclidean proximal gradient algorithm, that is more generally applicable to any VI problem
with a monotone operator, performs the following update at each iteration:

\begin{equation}
	\label{eqn:proxgrad} z_{t+1} = \arg \min_{z \in \calZ} \eta (\langle F(z_t),
	z\rangle + \alpha g(z)) + B_{\psi} (z; z_t).
\end{equation}

where $\psi$ is a strongly convex function with respect to $\|.\|$ over $\calZ$.

The algorithm that is termed Magnetic Mirror Descent (MMD) uses the same update as~
\ref{eqn:proxgrad} with $g$ taken to be either $\psi$, or $B_{\psi}(.
	;z')$.
In the former, $\psi$ is as a strongly convex regularizer that makes the objective smoother and
encouranges exploration.
In the latter form, $B_{\psi}$ is another proximity term that forces the iterates ($z_{t+1}$) to
stay close to some \textit{magnet} ($z'$).
For all our discussion, we only consider the former update rule which is more widely applicable.

We now restate the main algorithm as stated in Sokota et.al,~ \cite{sokotaUnified2023},

\begin{alprocedure}[H] \algorithmcfname{MMD~ \cite[(Algorithm
			3.6)]{sokotaUnified2023}} \label{alg:mmd} Starting with $z_1 \in \text{int dom } \psi \cap \calZ$,
	at each iteration $t$ do $$ z_{t+1} = \arg \min_{z \in \calZ} \eta (\langle F(z_t), z\rangle +
		\alpha \psi(z)) + B_{\psi} (z; z_t).
	$$
\end{alprocedure}

\hfill \break
Algorithm~
\ref{alg:mmd} provides the following convergence guarantees.
\begin{theorem}
	\label{thm:mmdconv}
	\cite[Theorem 3.4]{sokotaUnified2023}
	Assuming that the solution $z_{\ast}$ to the problem VI ($\calZ, F + \alpha \nabla g$) lies in the
	int dom $\psi$, then

	\[ B_{\psi} (z_{\ast}; z_{t + 1}) \leq { \left(\frac{1}{1
				+ \eta \alpha}\right)}^t B_{\psi} (z_{\ast}; z_1), \]

	if $\alpha > 0$, and
	$\eta \leq \frac{\alpha}{L^2}$.
\end{theorem}

\subsection{Behavioral form MMD}

The update rule from Algorithm~ \ref{alg:mmd} admits closed form in some instances, while requires
approximation through gradient updates in other cases.
For a parameterized policy $pi_\theta$, when $\psi$ is negative entropy the update rule from
Algorithm~ \ref{alg:mmd} can be restated in RL terms as follows:
\begin{equation}
	\label{eqn:mmdbf}
	\pi_{ \theta_{k+1}}(s,a) = \arg \max_{\theta} \E_{s \sim \rho_{\theta_k}} \left[ \E_{a \sim
			\pi_{\theta_k}} [Q_{\theta_k}(s, a)] + \alpha H(\pi_\theta) - \frac{1}{\eta} KL(\pi_\theta,
		\pi_{\theta_k}) \right],
\end{equation}

where $H(\pi_\theta)$ is the entropy of
the policy being optimized.

This behavioral form update can also be approximated using gradient updates similar to MDPO.
For $m$ steps MMD uses the following gradient to update the policy parameters,
\begin{equation}
	\label{eqn:mmdgrad} \nabla_{\theta} J(\theta, \theta_k)|_{\theta = \theta_k^{(i)}} = \E_{s \sim
		\rho_{\theta_k} \\ a \sim \theta_k} [ \frac{ \pi_{\theta_k}^{(i)}}{ \pi_{\theta_k}} \nabla_{\theta}
		\log \pi_{\theta_k}^{(i)} (a|s) A_{\theta_k}(s,a)] + \alpha H( \pi_{\theta_k}^{(i)}) -
	\frac{1}{\eta} \E_{s \sim \rho_{\theta_k}} [\nabla_\theta KL(s; \pi_\theta,
		\pi_{\theta_k})_{|\theta= \theta_k^{(i)}}]
\end{equation} where $i=0,1,\ldots,m-1$.

To examine the effect of the choice of $m$, we compare the performance of the behavioral form
update from~ \ref{eqn:mmdbf} to the closed-form for MMD with a negative entropy mirror map~
\cite[equation (12)]{sokotaUnified2023}: $$ \pi_{k+1} \propto [\pi_t \rho^{\alpha \eta} e^{\eta
				Q_{\pi_k}}]^{ \frac{1}{1+\alpha \eta}}$$

For a two-player zero-sum normal form
game, the expected Q-values are the expected payoffs for the max and min players respectively.
As there is only one state, we can replace the expectations in the update with an all-action
Q-value computation that is only dependent upon the opponent's policy.

In Fig~ \ref{fig:?
}, we plot the norm of the difference between the closed-form and behavioral form policies at each iteration.
It can be seen that the behavioral form approximates the closed form well as expected for most
choices of step size and $m$.
For large number of gradient steps, or large step sizes the updates are unstable.
For all our tabular experiments, we use a step-size of 0.1, and $m=10$.

\subsection{Comparison of MMD to other Mirror Decent based methods}

The non-euclidean proximal gradient method~ \ref{eqn:proxgrad}, has strong connections to Mirror
Descent~ \cite[Appendix D.3]{sokotaUnified2023}.
\blue{TBD: other works also discuss this relationship between mirror descent and proximal gradient methods applied to VIs}.
Consequently negative entropy based MMD is also equivalent to MDPO with an added entropy
regularization as detailed in~ \cite[Appendix L]{sokotaUnified2023} and as can be seen from~
\ref{eqn:mdrl} and~ \ref{eqn:mmdbf}.

MMD as a reinforcement learning algorithm performs on par with CFR.
In single agent settings MMD's performance is competitive with PPO in Atari and MuJoCo
environments.
