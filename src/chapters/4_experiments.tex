\chapter{Tabular Experiments}

We now evaluate our proposed methods experimentally in both tabular and function approximation
settings as approximate equilibrium solvers.
Through the experiments, we aim to answer the following questions:
\begin{enumerate}
	\item {What is
	      the last-iterate and average-iterate convergence behavior of these algorithms?
	      }\label{qn1}
	\item {How does the addition of NeuRD-fix, Extragradient updates, and Optimistic upates
	      affect the convergence rate of these algorithms (alone and in combination)?}\label{qn2}
	\item {Do these performance improvements scale well with the size of the game?}\label{qn3}
\end{enumerate}

MMD has theoretical converegence guarantees only as a QRE solver, but shows a strong empirical
performance for finding approximate Nash Equilibriums.
In this work our main focus is convergence to the Nash Equilibrium, and as such we focus our main
experimental results for the same.
We provide some additional results for the performance of different algorithms as QRE solvers in
the appendix.

\section{Experimental Domains}
We evaluate the algorithms on two normal form games namely, Perturbed RPS and Matching Pennies.

\textbf{Perturbed RPS}

\textbf{Matching Pennies}

\section{Evaluation
  Metrics}
Our main focus being the convergence behaviors and rates, we need a notion of distance from the
equilibrium point to measure the performance of these algorithms.

\subsection{Divergence to the equilibrium}
In settings with a known unique equilibrium, we can compute the distance of the current policy to
the known equilibrium using a measure of distance in the policy space such as the KL-Divergence.
Given the policy at iteration $t$, $pi_t$, and an equilibrium policy $\pi_*$, the metric we measure
is: $KL(\pi_t || \pi_*) = \sum_{a \in A} \pi_t(a) \log \left( \frac{\pi_t(a)}{\pi_*(a)} \right)$.

\subsection{Exploitability}
In general, there might not be a unique Nash equilibrium and we might not know which equilibrium
point the current policy is converging towards.
This makes it tricky to use a direct measure of distance as the above metric.
Exploitability is another metric that is commonly used as a notion of optimality in game theory.
Exploitability measures the gain in value a player can achieve by deviating from the current
policy.
We measure the value that a worst-case opponent can achieve by keeping the current policy fixed by
computing a best response at every state.
The difference between the value that this best-response opponent can acheive and the game value is
the exploitability of the current policy.

\blue{ - Exploitability formal expression in terms of best responses, and value.}

\subsection{Experiment Setup}
The policies that are logit-parameterized with a softmax projection and all the algorithms use
full-feedback gradient updates as discussed in Chapter 3.
Being symmetric matrix games, both games have a unique Nash Equilibrium: Perturbed RPS
$(\frac{1}{7}, \frac{3}{7}, \frac{3}{7})$, Matching Pennies $(\frac{1}{2}, \frac{1}{2})$.
For all the algorithms, we train both the players for 5000 training steps with alternating updates
using the exact payoff vectors.
For the EG variants, we train the players for only 2500 steps as they use two gradient computation
per step for a fair comparison.
We use the $m=10$ gradient steps per iteration with a learning rate of 0.1 for all the runs.
For MMD and MDPO, we anneal the temperature (entropy coefficient) with the schedule $\alpha_t =
	1/\sqrt{t}$.
For the KL-coefficient, we use different schedules for MMD ($\eta_t = \max(1 / \sqrt{t}, 0.2)$),
and MDPO ($\eta_t = \max(1 - t/T, 0.2)$).
MDPO's schedule is motivated by mirror-descent theory, while MMD's schedule is closer to the type
of annealing schedule used for the original MMD experiments found through a hyperparameter sweep.
For both of these methods we cap the KL-coefficient at 0.2 because very low values destabilize
the updates, especially for the NeuRD version of the algorithms.

\section{Results} \ref{tab:tabres} summarizes the last and average
iterate convergences behaviors of 3 different algorithms - MMD, MDPO, SPG under the proposed
modifications.
% We present the results for NE, and QRE convergence (with 0.5 temperature).
Fig~\ref{fig:tabne} summarizes the results for convergence to the Nash Equilibrium.

% \begin{noindent}

\begin{table}[htbp]
	\centering
	\begin{tabular}{|c|c|c|c|c|c|c|c|c|}
	\hline
	\multirow{2}{*}{\textbf{Algorithm}} &
	% \multicolumn{4}{c|}{\textbf{Nash}} &
	% \multicolumn{2}{c|}{\textbf{QRE ($\alpha$=0.5)}} \\
	% \cline{2-7}
	% & 
	\multicolumn{2}{c|}{\textbf{PG}} & 
	\multicolumn{2}{c|}{\textbf{MDPO}} &
	\multicolumn{2}{c|}{\textbf{MMD}} \\
	\cline{2-7}
	& \textbf{Avg} & \textbf{Last} & \textbf{Avg} & \textbf{Last}
	& \textbf{Avg} & \textbf{Last} \\
	\hline
	Base	 	& \red{\texttimes} 	& \red{\texttimes} 	& \red{\texttimes} 	& \red{\texttimes}		& \checked 	& \checked \\ \hline
	NeuRD 		& \checked 			& \red{\texttimes} 	& \checked 			& \red{\texttimes}		& \checked 	& \checked \\ \hline
	EG 			& \checked 			& \checked 			& \checked 			& \checked 				& \checked 	& \checked \\ \hline
	OPT 		& \checked 			& \checked 			& \checked 			& \checked 				& \checked 	& \checked \\ \hline
	EG-OPT 		& \checked 			& \checked 			& \checked 			& \checked 				& \checked 	& \checked \\ \hline
	EG-N 		& \checked 			& \checked 			& \checked 			& \checked 				& \checked 	& \checked \\ \hline
	OPT-N 		& \checked 			& \checked 			& \checked 			& \checked 				& \checked 	& \checked \\ \hline
	EG-OPT-N	& \checked 			& \checked 			& \checked 			& \checked 				& \checked 	& \checked \\ \hline
	\end{tabular}
	\caption{Convergence in Perturbed RPS for last, and average iterates.}
	\label{tab:tabres}
\end{table}
% \end{noindent}

\begin{figure}[H]
	\centering
	\scalebox{0.7}[0.6]{%% Creator: Matplotlib, PGF backend
%%
%% To include the figure in your LaTeX document, write
%%   \input{<filename>.pgf}
%%
%% Make sure the required packages are loaded in your preamble
%%   \usepackage{pgf}
%%
%% Also ensure that all the required font packages are loaded; for instance,
%% the lmodern package is sometimes necessary when using math font.
%%   \usepackage{lmodern}
%%
%% Figures using additional raster images can only be included by \input if
%% they are in the same directory as the main LaTeX file. For loading figures
%% from other directories you can use the `import` package
%%   \usepackage{import}
%%
%% and then include the figures with
%%   \import{<path to file>}{<filename>.pgf}
%%
%% Matplotlib used the following preamble
%%   
%%   \makeatletter\@ifpackageloaded{underscore}{}{\usepackage[strings]{underscore}}\makeatother
%%
\begingroup%
\makeatletter%
\begin{pgfpicture}%
\pgfpathrectangle{\pgfpointorigin}{\pgfqpoint{6.400000in}{4.800000in}}%
\pgfusepath{use as bounding box, clip}%
\begin{pgfscope}%
\pgfsetbuttcap%
\pgfsetmiterjoin%
\definecolor{currentfill}{rgb}{1.000000,1.000000,1.000000}%
\pgfsetfillcolor{currentfill}%
\pgfsetlinewidth{0.000000pt}%
\definecolor{currentstroke}{rgb}{1.000000,1.000000,1.000000}%
\pgfsetstrokecolor{currentstroke}%
\pgfsetdash{}{0pt}%
\pgfpathmoveto{\pgfqpoint{0.000000in}{0.000000in}}%
\pgfpathlineto{\pgfqpoint{6.400000in}{0.000000in}}%
\pgfpathlineto{\pgfqpoint{6.400000in}{4.800000in}}%
\pgfpathlineto{\pgfqpoint{0.000000in}{4.800000in}}%
\pgfpathlineto{\pgfqpoint{0.000000in}{0.000000in}}%
\pgfpathclose%
\pgfusepath{fill}%
\end{pgfscope}%
\begin{pgfscope}%
\pgfsetbuttcap%
\pgfsetmiterjoin%
\definecolor{currentfill}{rgb}{1.000000,1.000000,1.000000}%
\pgfsetfillcolor{currentfill}%
\pgfsetlinewidth{0.000000pt}%
\definecolor{currentstroke}{rgb}{0.000000,0.000000,0.000000}%
\pgfsetstrokecolor{currentstroke}%
\pgfsetstrokeopacity{0.000000}%
\pgfsetdash{}{0pt}%
\pgfpathmoveto{\pgfqpoint{0.800000in}{2.544000in}}%
\pgfpathlineto{\pgfqpoint{3.054545in}{2.544000in}}%
\pgfpathlineto{\pgfqpoint{3.054545in}{4.224000in}}%
\pgfpathlineto{\pgfqpoint{0.800000in}{4.224000in}}%
\pgfpathlineto{\pgfqpoint{0.800000in}{2.544000in}}%
\pgfpathclose%
\pgfusepath{fill}%
\end{pgfscope}%
\begin{pgfscope}%
\pgfsetbuttcap%
\pgfsetroundjoin%
\definecolor{currentfill}{rgb}{0.000000,0.000000,0.000000}%
\pgfsetfillcolor{currentfill}%
\pgfsetlinewidth{0.803000pt}%
\definecolor{currentstroke}{rgb}{0.000000,0.000000,0.000000}%
\pgfsetstrokecolor{currentstroke}%
\pgfsetdash{}{0pt}%
\pgfsys@defobject{currentmarker}{\pgfqpoint{0.000000in}{-0.048611in}}{\pgfqpoint{0.000000in}{0.000000in}}{%
\pgfpathmoveto{\pgfqpoint{0.000000in}{0.000000in}}%
\pgfpathlineto{\pgfqpoint{0.000000in}{-0.048611in}}%
\pgfusepath{stroke,fill}%
}%
\begin{pgfscope}%
\pgfsys@transformshift{0.902479in}{2.544000in}%
\pgfsys@useobject{currentmarker}{}%
\end{pgfscope}%
\end{pgfscope}%
\begin{pgfscope}%
\definecolor{textcolor}{rgb}{0.000000,0.000000,0.000000}%
\pgfsetstrokecolor{textcolor}%
\pgfsetfillcolor{textcolor}%
\pgftext[x=0.902479in,y=2.446778in,,top]{\color{textcolor}\rmfamily\fontsize{10.000000}{12.000000}\selectfont \(\displaystyle {0}\)}%
\end{pgfscope}%
\begin{pgfscope}%
\pgfsetbuttcap%
\pgfsetroundjoin%
\definecolor{currentfill}{rgb}{0.000000,0.000000,0.000000}%
\pgfsetfillcolor{currentfill}%
\pgfsetlinewidth{0.803000pt}%
\definecolor{currentstroke}{rgb}{0.000000,0.000000,0.000000}%
\pgfsetstrokecolor{currentstroke}%
\pgfsetdash{}{0pt}%
\pgfsys@defobject{currentmarker}{\pgfqpoint{0.000000in}{-0.048611in}}{\pgfqpoint{0.000000in}{0.000000in}}{%
\pgfpathmoveto{\pgfqpoint{0.000000in}{0.000000in}}%
\pgfpathlineto{\pgfqpoint{0.000000in}{-0.048611in}}%
\pgfusepath{stroke,fill}%
}%
\begin{pgfscope}%
\pgfsys@transformshift{1.722478in}{2.544000in}%
\pgfsys@useobject{currentmarker}{}%
\end{pgfscope}%
\end{pgfscope}%
\begin{pgfscope}%
\definecolor{textcolor}{rgb}{0.000000,0.000000,0.000000}%
\pgfsetstrokecolor{textcolor}%
\pgfsetfillcolor{textcolor}%
\pgftext[x=1.722478in,y=2.446778in,,top]{\color{textcolor}\rmfamily\fontsize{10.000000}{12.000000}\selectfont \(\displaystyle {2000}\)}%
\end{pgfscope}%
\begin{pgfscope}%
\pgfsetbuttcap%
\pgfsetroundjoin%
\definecolor{currentfill}{rgb}{0.000000,0.000000,0.000000}%
\pgfsetfillcolor{currentfill}%
\pgfsetlinewidth{0.803000pt}%
\definecolor{currentstroke}{rgb}{0.000000,0.000000,0.000000}%
\pgfsetstrokecolor{currentstroke}%
\pgfsetdash{}{0pt}%
\pgfsys@defobject{currentmarker}{\pgfqpoint{0.000000in}{-0.048611in}}{\pgfqpoint{0.000000in}{0.000000in}}{%
\pgfpathmoveto{\pgfqpoint{0.000000in}{0.000000in}}%
\pgfpathlineto{\pgfqpoint{0.000000in}{-0.048611in}}%
\pgfusepath{stroke,fill}%
}%
\begin{pgfscope}%
\pgfsys@transformshift{2.542477in}{2.544000in}%
\pgfsys@useobject{currentmarker}{}%
\end{pgfscope}%
\end{pgfscope}%
\begin{pgfscope}%
\definecolor{textcolor}{rgb}{0.000000,0.000000,0.000000}%
\pgfsetstrokecolor{textcolor}%
\pgfsetfillcolor{textcolor}%
\pgftext[x=2.542477in,y=2.446778in,,top]{\color{textcolor}\rmfamily\fontsize{10.000000}{12.000000}\selectfont \(\displaystyle {4000}\)}%
\end{pgfscope}%
\begin{pgfscope}%
\pgfsetbuttcap%
\pgfsetroundjoin%
\definecolor{currentfill}{rgb}{0.000000,0.000000,0.000000}%
\pgfsetfillcolor{currentfill}%
\pgfsetlinewidth{0.803000pt}%
\definecolor{currentstroke}{rgb}{0.000000,0.000000,0.000000}%
\pgfsetstrokecolor{currentstroke}%
\pgfsetdash{}{0pt}%
\pgfsys@defobject{currentmarker}{\pgfqpoint{-0.048611in}{0.000000in}}{\pgfqpoint{-0.000000in}{0.000000in}}{%
\pgfpathmoveto{\pgfqpoint{-0.000000in}{0.000000in}}%
\pgfpathlineto{\pgfqpoint{-0.048611in}{0.000000in}}%
\pgfusepath{stroke,fill}%
}%
\begin{pgfscope}%
\pgfsys@transformshift{0.800000in}{2.820175in}%
\pgfsys@useobject{currentmarker}{}%
\end{pgfscope}%
\end{pgfscope}%
\begin{pgfscope}%
\definecolor{textcolor}{rgb}{0.000000,0.000000,0.000000}%
\pgfsetstrokecolor{textcolor}%
\pgfsetfillcolor{textcolor}%
\pgftext[x=0.455863in, y=2.771950in, left, base]{\color{textcolor}\rmfamily\fontsize{10.000000}{12.000000}\selectfont \(\displaystyle {0.00}\)}%
\end{pgfscope}%
\begin{pgfscope}%
\pgfsetbuttcap%
\pgfsetroundjoin%
\definecolor{currentfill}{rgb}{0.000000,0.000000,0.000000}%
\pgfsetfillcolor{currentfill}%
\pgfsetlinewidth{0.803000pt}%
\definecolor{currentstroke}{rgb}{0.000000,0.000000,0.000000}%
\pgfsetstrokecolor{currentstroke}%
\pgfsetdash{}{0pt}%
\pgfsys@defobject{currentmarker}{\pgfqpoint{-0.048611in}{0.000000in}}{\pgfqpoint{-0.000000in}{0.000000in}}{%
\pgfpathmoveto{\pgfqpoint{-0.000000in}{0.000000in}}%
\pgfpathlineto{\pgfqpoint{-0.048611in}{0.000000in}}%
\pgfusepath{stroke,fill}%
}%
\begin{pgfscope}%
\pgfsys@transformshift{0.800000in}{3.171131in}%
\pgfsys@useobject{currentmarker}{}%
\end{pgfscope}%
\end{pgfscope}%
\begin{pgfscope}%
\definecolor{textcolor}{rgb}{0.000000,0.000000,0.000000}%
\pgfsetstrokecolor{textcolor}%
\pgfsetfillcolor{textcolor}%
\pgftext[x=0.455863in, y=3.122906in, left, base]{\color{textcolor}\rmfamily\fontsize{10.000000}{12.000000}\selectfont \(\displaystyle {0.05}\)}%
\end{pgfscope}%
\begin{pgfscope}%
\pgfsetbuttcap%
\pgfsetroundjoin%
\definecolor{currentfill}{rgb}{0.000000,0.000000,0.000000}%
\pgfsetfillcolor{currentfill}%
\pgfsetlinewidth{0.803000pt}%
\definecolor{currentstroke}{rgb}{0.000000,0.000000,0.000000}%
\pgfsetstrokecolor{currentstroke}%
\pgfsetdash{}{0pt}%
\pgfsys@defobject{currentmarker}{\pgfqpoint{-0.048611in}{0.000000in}}{\pgfqpoint{-0.000000in}{0.000000in}}{%
\pgfpathmoveto{\pgfqpoint{-0.000000in}{0.000000in}}%
\pgfpathlineto{\pgfqpoint{-0.048611in}{0.000000in}}%
\pgfusepath{stroke,fill}%
}%
\begin{pgfscope}%
\pgfsys@transformshift{0.800000in}{3.522087in}%
\pgfsys@useobject{currentmarker}{}%
\end{pgfscope}%
\end{pgfscope}%
\begin{pgfscope}%
\definecolor{textcolor}{rgb}{0.000000,0.000000,0.000000}%
\pgfsetstrokecolor{textcolor}%
\pgfsetfillcolor{textcolor}%
\pgftext[x=0.455863in, y=3.473862in, left, base]{\color{textcolor}\rmfamily\fontsize{10.000000}{12.000000}\selectfont \(\displaystyle {0.10}\)}%
\end{pgfscope}%
\begin{pgfscope}%
\pgfsetbuttcap%
\pgfsetroundjoin%
\definecolor{currentfill}{rgb}{0.000000,0.000000,0.000000}%
\pgfsetfillcolor{currentfill}%
\pgfsetlinewidth{0.803000pt}%
\definecolor{currentstroke}{rgb}{0.000000,0.000000,0.000000}%
\pgfsetstrokecolor{currentstroke}%
\pgfsetdash{}{0pt}%
\pgfsys@defobject{currentmarker}{\pgfqpoint{-0.048611in}{0.000000in}}{\pgfqpoint{-0.000000in}{0.000000in}}{%
\pgfpathmoveto{\pgfqpoint{-0.000000in}{0.000000in}}%
\pgfpathlineto{\pgfqpoint{-0.048611in}{0.000000in}}%
\pgfusepath{stroke,fill}%
}%
\begin{pgfscope}%
\pgfsys@transformshift{0.800000in}{3.873044in}%
\pgfsys@useobject{currentmarker}{}%
\end{pgfscope}%
\end{pgfscope}%
\begin{pgfscope}%
\definecolor{textcolor}{rgb}{0.000000,0.000000,0.000000}%
\pgfsetstrokecolor{textcolor}%
\pgfsetfillcolor{textcolor}%
\pgftext[x=0.455863in, y=3.824818in, left, base]{\color{textcolor}\rmfamily\fontsize{10.000000}{12.000000}\selectfont \(\displaystyle {0.15}\)}%
\end{pgfscope}%
\begin{pgfscope}%
\pgfsetbuttcap%
\pgfsetroundjoin%
\definecolor{currentfill}{rgb}{0.000000,0.000000,0.000000}%
\pgfsetfillcolor{currentfill}%
\pgfsetlinewidth{0.803000pt}%
\definecolor{currentstroke}{rgb}{0.000000,0.000000,0.000000}%
\pgfsetstrokecolor{currentstroke}%
\pgfsetdash{}{0pt}%
\pgfsys@defobject{currentmarker}{\pgfqpoint{-0.048611in}{0.000000in}}{\pgfqpoint{-0.000000in}{0.000000in}}{%
\pgfpathmoveto{\pgfqpoint{-0.000000in}{0.000000in}}%
\pgfpathlineto{\pgfqpoint{-0.048611in}{0.000000in}}%
\pgfusepath{stroke,fill}%
}%
\begin{pgfscope}%
\pgfsys@transformshift{0.800000in}{4.224000in}%
\pgfsys@useobject{currentmarker}{}%
\end{pgfscope}%
\end{pgfscope}%
\begin{pgfscope}%
\definecolor{textcolor}{rgb}{0.000000,0.000000,0.000000}%
\pgfsetstrokecolor{textcolor}%
\pgfsetfillcolor{textcolor}%
\pgftext[x=0.455863in, y=4.175775in, left, base]{\color{textcolor}\rmfamily\fontsize{10.000000}{12.000000}\selectfont \(\displaystyle {0.20}\)}%
\end{pgfscope}%
\begin{pgfscope}%
\pgfpathrectangle{\pgfqpoint{0.800000in}{2.544000in}}{\pgfqpoint{2.254545in}{1.680000in}}%
\pgfusepath{clip}%
\pgfsetrectcap%
\pgfsetroundjoin%
\pgfsetlinewidth{1.505625pt}%
\definecolor{currentstroke}{rgb}{0.121569,0.466667,0.705882}%
\pgfsetstrokecolor{currentstroke}%
\pgfsetdash{}{0pt}%
\pgfpathmoveto{\pgfqpoint{0.902479in}{3.626600in}}%
\pgfpathlineto{\pgfqpoint{0.903299in}{3.614257in}}%
\pgfpathlineto{\pgfqpoint{0.905349in}{3.507940in}}%
\pgfpathlineto{\pgfqpoint{0.911909in}{3.145699in}}%
\pgfpathlineto{\pgfqpoint{0.917649in}{2.951908in}}%
\pgfpathlineto{\pgfqpoint{0.921339in}{2.886500in}}%
\pgfpathlineto{\pgfqpoint{0.924619in}{2.862523in}}%
\pgfpathlineto{\pgfqpoint{0.926669in}{2.858659in}}%
\pgfpathlineto{\pgfqpoint{0.928309in}{2.859272in}}%
\pgfpathlineto{\pgfqpoint{0.931999in}{2.865262in}}%
\pgfpathlineto{\pgfqpoint{0.935689in}{2.869192in}}%
\pgfpathlineto{\pgfqpoint{0.937739in}{2.868806in}}%
\pgfpathlineto{\pgfqpoint{0.940609in}{2.865411in}}%
\pgfpathlineto{\pgfqpoint{0.953319in}{2.845918in}}%
\pgfpathlineto{\pgfqpoint{0.957829in}{2.844999in}}%
\pgfpathlineto{\pgfqpoint{0.967259in}{2.843933in}}%
\pgfpathlineto{\pgfqpoint{0.987349in}{2.837279in}}%
\pgfpathlineto{\pgfqpoint{1.002929in}{2.834709in}}%
\pgfpathlineto{\pgfqpoint{1.018099in}{2.832915in}}%
\pgfpathlineto{\pgfqpoint{1.082469in}{2.828360in}}%
\pgfpathlineto{\pgfqpoint{1.154629in}{2.826030in}}%
\pgfpathlineto{\pgfqpoint{1.295259in}{2.823943in}}%
\pgfpathlineto{\pgfqpoint{1.575698in}{2.822378in}}%
\pgfpathlineto{\pgfqpoint{2.255067in}{2.821273in}}%
\pgfpathlineto{\pgfqpoint{2.952066in}{2.820900in}}%
\pgfpathlineto{\pgfqpoint{2.952066in}{2.820900in}}%
\pgfusepath{stroke}%
\end{pgfscope}%
\begin{pgfscope}%
\pgfpathrectangle{\pgfqpoint{0.800000in}{2.544000in}}{\pgfqpoint{2.254545in}{1.680000in}}%
\pgfusepath{clip}%
\pgfsetrectcap%
\pgfsetroundjoin%
\pgfsetlinewidth{1.505625pt}%
\definecolor{currentstroke}{rgb}{1.000000,0.498039,0.054902}%
\pgfsetstrokecolor{currentstroke}%
\pgfsetdash{}{0pt}%
\pgfpathmoveto{\pgfqpoint{0.902479in}{3.626600in}}%
\pgfpathlineto{\pgfqpoint{0.903709in}{3.618782in}}%
\pgfpathlineto{\pgfqpoint{0.906169in}{3.579928in}}%
\pgfpathlineto{\pgfqpoint{0.918879in}{3.335558in}}%
\pgfpathlineto{\pgfqpoint{0.934869in}{3.153776in}}%
\pgfpathlineto{\pgfqpoint{0.950859in}{2.924065in}}%
\pgfpathlineto{\pgfqpoint{0.956599in}{2.875458in}}%
\pgfpathlineto{\pgfqpoint{0.961519in}{2.850007in}}%
\pgfpathlineto{\pgfqpoint{0.966439in}{2.835307in}}%
\pgfpathlineto{\pgfqpoint{0.971359in}{2.827246in}}%
\pgfpathlineto{\pgfqpoint{0.976279in}{2.823157in}}%
\pgfpathlineto{\pgfqpoint{0.980789in}{2.822123in}}%
\pgfpathlineto{\pgfqpoint{0.984889in}{2.823526in}}%
\pgfpathlineto{\pgfqpoint{0.989399in}{2.827743in}}%
\pgfpathlineto{\pgfqpoint{0.995549in}{2.837148in}}%
\pgfpathlineto{\pgfqpoint{1.011129in}{2.862555in}}%
\pgfpathlineto{\pgfqpoint{1.016459in}{2.866622in}}%
\pgfpathlineto{\pgfqpoint{1.020559in}{2.867161in}}%
\pgfpathlineto{\pgfqpoint{1.024659in}{2.865387in}}%
\pgfpathlineto{\pgfqpoint{1.029579in}{2.860580in}}%
\pgfpathlineto{\pgfqpoint{1.037369in}{2.849265in}}%
\pgfpathlineto{\pgfqpoint{1.049669in}{2.831746in}}%
\pgfpathlineto{\pgfqpoint{1.057049in}{2.825145in}}%
\pgfpathlineto{\pgfqpoint{1.064019in}{2.821907in}}%
\pgfpathlineto{\pgfqpoint{1.070989in}{2.821113in}}%
\pgfpathlineto{\pgfqpoint{1.078369in}{2.822478in}}%
\pgfpathlineto{\pgfqpoint{1.087799in}{2.826650in}}%
\pgfpathlineto{\pgfqpoint{1.106249in}{2.835258in}}%
\pgfpathlineto{\pgfqpoint{1.114039in}{2.836032in}}%
\pgfpathlineto{\pgfqpoint{1.121829in}{2.834540in}}%
\pgfpathlineto{\pgfqpoint{1.133719in}{2.829626in}}%
\pgfpathlineto{\pgfqpoint{1.148479in}{2.824077in}}%
\pgfpathlineto{\pgfqpoint{1.159139in}{2.822465in}}%
\pgfpathlineto{\pgfqpoint{1.170619in}{2.823028in}}%
\pgfpathlineto{\pgfqpoint{1.189889in}{2.826766in}}%
\pgfpathlineto{\pgfqpoint{1.203829in}{2.828303in}}%
\pgfpathlineto{\pgfqpoint{1.216539in}{2.827391in}}%
\pgfpathlineto{\pgfqpoint{1.255899in}{2.822755in}}%
\pgfpathlineto{\pgfqpoint{1.276399in}{2.824201in}}%
\pgfpathlineto{\pgfqpoint{1.299769in}{2.825324in}}%
\pgfpathlineto{\pgfqpoint{1.323549in}{2.823698in}}%
\pgfpathlineto{\pgfqpoint{1.349379in}{2.822699in}}%
\pgfpathlineto{\pgfqpoint{1.426869in}{2.822641in}}%
\pgfpathlineto{\pgfqpoint{1.461308in}{2.822874in}}%
\pgfpathlineto{\pgfqpoint{1.502718in}{2.822732in}}%
\pgfpathlineto{\pgfqpoint{1.554378in}{2.822496in}}%
\pgfpathlineto{\pgfqpoint{1.615878in}{2.822111in}}%
\pgfpathlineto{\pgfqpoint{1.905338in}{2.821630in}}%
\pgfpathlineto{\pgfqpoint{2.952066in}{2.820901in}}%
\pgfpathlineto{\pgfqpoint{2.952066in}{2.820901in}}%
\pgfusepath{stroke}%
\end{pgfscope}%
\begin{pgfscope}%
\pgfpathrectangle{\pgfqpoint{0.800000in}{2.544000in}}{\pgfqpoint{2.254545in}{1.680000in}}%
\pgfusepath{clip}%
\pgfsetrectcap%
\pgfsetroundjoin%
\pgfsetlinewidth{1.505625pt}%
\definecolor{currentstroke}{rgb}{0.172549,0.627451,0.172549}%
\pgfsetstrokecolor{currentstroke}%
\pgfsetdash{}{0pt}%
\pgfpathmoveto{\pgfqpoint{0.902479in}{3.626600in}}%
\pgfpathlineto{\pgfqpoint{0.903709in}{3.611702in}}%
\pgfpathlineto{\pgfqpoint{0.906169in}{3.540380in}}%
\pgfpathlineto{\pgfqpoint{0.912319in}{3.359187in}}%
\pgfpathlineto{\pgfqpoint{0.917649in}{3.279699in}}%
\pgfpathlineto{\pgfqpoint{0.922979in}{3.192332in}}%
\pgfpathlineto{\pgfqpoint{0.939789in}{2.882540in}}%
\pgfpathlineto{\pgfqpoint{0.944299in}{2.847072in}}%
\pgfpathlineto{\pgfqpoint{0.948399in}{2.830376in}}%
\pgfpathlineto{\pgfqpoint{0.952089in}{2.823430in}}%
\pgfpathlineto{\pgfqpoint{0.955369in}{2.821459in}}%
\pgfpathlineto{\pgfqpoint{0.958239in}{2.822176in}}%
\pgfpathlineto{\pgfqpoint{0.961929in}{2.825863in}}%
\pgfpathlineto{\pgfqpoint{0.967259in}{2.835243in}}%
\pgfpathlineto{\pgfqpoint{0.978739in}{2.856661in}}%
\pgfpathlineto{\pgfqpoint{0.982839in}{2.859886in}}%
\pgfpathlineto{\pgfqpoint{0.986119in}{2.859948in}}%
\pgfpathlineto{\pgfqpoint{0.989809in}{2.857511in}}%
\pgfpathlineto{\pgfqpoint{0.995139in}{2.850613in}}%
\pgfpathlineto{\pgfqpoint{1.009079in}{2.831130in}}%
\pgfpathlineto{\pgfqpoint{1.014819in}{2.827244in}}%
\pgfpathlineto{\pgfqpoint{1.020559in}{2.826069in}}%
\pgfpathlineto{\pgfqpoint{1.026709in}{2.827142in}}%
\pgfpathlineto{\pgfqpoint{1.037369in}{2.831827in}}%
\pgfpathlineto{\pgfqpoint{1.046389in}{2.834767in}}%
\pgfpathlineto{\pgfqpoint{1.053359in}{2.834694in}}%
\pgfpathlineto{\pgfqpoint{1.061969in}{2.832120in}}%
\pgfpathlineto{\pgfqpoint{1.078779in}{2.826668in}}%
\pgfpathlineto{\pgfqpoint{1.088619in}{2.826177in}}%
\pgfpathlineto{\pgfqpoint{1.105429in}{2.828209in}}%
\pgfpathlineto{\pgfqpoint{1.118549in}{2.828539in}}%
\pgfpathlineto{\pgfqpoint{1.138639in}{2.826027in}}%
\pgfpathlineto{\pgfqpoint{1.155039in}{2.825364in}}%
\pgfpathlineto{\pgfqpoint{1.196859in}{2.825299in}}%
\pgfpathlineto{\pgfqpoint{1.224739in}{2.824613in}}%
\pgfpathlineto{\pgfqpoint{1.267789in}{2.824153in}}%
\pgfpathlineto{\pgfqpoint{1.325599in}{2.823706in}}%
\pgfpathlineto{\pgfqpoint{1.421949in}{2.823027in}}%
\pgfpathlineto{\pgfqpoint{2.952066in}{2.820900in}}%
\pgfpathlineto{\pgfqpoint{2.952066in}{2.820900in}}%
\pgfusepath{stroke}%
\end{pgfscope}%
\begin{pgfscope}%
\pgfpathrectangle{\pgfqpoint{0.800000in}{2.544000in}}{\pgfqpoint{2.254545in}{1.680000in}}%
\pgfusepath{clip}%
\pgfsetrectcap%
\pgfsetroundjoin%
\pgfsetlinewidth{1.505625pt}%
\definecolor{currentstroke}{rgb}{0.839216,0.152941,0.156863}%
\pgfsetstrokecolor{currentstroke}%
\pgfsetdash{}{0pt}%
\pgfpathmoveto{\pgfqpoint{0.902479in}{3.626600in}}%
\pgfpathlineto{\pgfqpoint{0.903299in}{3.603198in}}%
\pgfpathlineto{\pgfqpoint{0.904939in}{3.507261in}}%
\pgfpathlineto{\pgfqpoint{0.911089in}{3.120011in}}%
\pgfpathlineto{\pgfqpoint{0.913959in}{3.079897in}}%
\pgfpathlineto{\pgfqpoint{0.918059in}{3.038484in}}%
\pgfpathlineto{\pgfqpoint{0.921339in}{2.965811in}}%
\pgfpathlineto{\pgfqpoint{0.924209in}{2.914529in}}%
\pgfpathlineto{\pgfqpoint{0.924619in}{2.913064in}}%
\pgfpathlineto{\pgfqpoint{0.925029in}{2.913264in}}%
\pgfpathlineto{\pgfqpoint{0.926259in}{2.922222in}}%
\pgfpathlineto{\pgfqpoint{0.930359in}{2.960468in}}%
\pgfpathlineto{\pgfqpoint{0.931179in}{2.957751in}}%
\pgfpathlineto{\pgfqpoint{0.933229in}{2.937392in}}%
\pgfpathlineto{\pgfqpoint{0.936919in}{2.902002in}}%
\pgfpathlineto{\pgfqpoint{0.938969in}{2.897113in}}%
\pgfpathlineto{\pgfqpoint{0.941839in}{2.893999in}}%
\pgfpathlineto{\pgfqpoint{0.943889in}{2.885613in}}%
\pgfpathlineto{\pgfqpoint{0.950449in}{2.846093in}}%
\pgfpathlineto{\pgfqpoint{0.950859in}{2.846367in}}%
\pgfpathlineto{\pgfqpoint{0.952499in}{2.851518in}}%
\pgfpathlineto{\pgfqpoint{0.956599in}{2.866136in}}%
\pgfpathlineto{\pgfqpoint{0.957829in}{2.865031in}}%
\pgfpathlineto{\pgfqpoint{0.960699in}{2.856128in}}%
\pgfpathlineto{\pgfqpoint{0.963569in}{2.849981in}}%
\pgfpathlineto{\pgfqpoint{0.969719in}{2.843900in}}%
\pgfpathlineto{\pgfqpoint{0.976279in}{2.828485in}}%
\pgfpathlineto{\pgfqpoint{0.977919in}{2.830260in}}%
\pgfpathlineto{\pgfqpoint{0.983249in}{2.838201in}}%
\pgfpathlineto{\pgfqpoint{0.985299in}{2.836861in}}%
\pgfpathlineto{\pgfqpoint{0.990219in}{2.832908in}}%
\pgfpathlineto{\pgfqpoint{0.994729in}{2.830646in}}%
\pgfpathlineto{\pgfqpoint{1.002519in}{2.823060in}}%
\pgfpathlineto{\pgfqpoint{1.005389in}{2.825544in}}%
\pgfpathlineto{\pgfqpoint{1.009079in}{2.828108in}}%
\pgfpathlineto{\pgfqpoint{1.012359in}{2.827429in}}%
\pgfpathlineto{\pgfqpoint{1.029989in}{2.821791in}}%
\pgfpathlineto{\pgfqpoint{1.036549in}{2.824077in}}%
\pgfpathlineto{\pgfqpoint{1.045979in}{2.822519in}}%
\pgfpathlineto{\pgfqpoint{1.054589in}{2.820552in}}%
\pgfpathlineto{\pgfqpoint{1.066889in}{2.822078in}}%
\pgfpathlineto{\pgfqpoint{1.087389in}{2.821375in}}%
\pgfpathlineto{\pgfqpoint{1.097639in}{2.820872in}}%
\pgfpathlineto{\pgfqpoint{1.108299in}{2.820452in}}%
\pgfpathlineto{\pgfqpoint{1.120599in}{2.820835in}}%
\pgfpathlineto{\pgfqpoint{1.139459in}{2.820700in}}%
\pgfpathlineto{\pgfqpoint{1.171439in}{2.820538in}}%
\pgfpathlineto{\pgfqpoint{1.216539in}{2.820408in}}%
\pgfpathlineto{\pgfqpoint{1.298129in}{2.820331in}}%
\pgfpathlineto{\pgfqpoint{1.847938in}{2.820228in}}%
\pgfpathlineto{\pgfqpoint{2.952066in}{2.820199in}}%
\pgfpathlineto{\pgfqpoint{2.952066in}{2.820199in}}%
\pgfusepath{stroke}%
\end{pgfscope}%
\begin{pgfscope}%
\pgfpathrectangle{\pgfqpoint{0.800000in}{2.544000in}}{\pgfqpoint{2.254545in}{1.680000in}}%
\pgfusepath{clip}%
\pgfsetrectcap%
\pgfsetroundjoin%
\pgfsetlinewidth{1.505625pt}%
\definecolor{currentstroke}{rgb}{0.580392,0.403922,0.741176}%
\pgfsetstrokecolor{currentstroke}%
\pgfsetdash{}{0pt}%
\pgfpathmoveto{\pgfqpoint{0.902479in}{3.626600in}}%
\pgfpathlineto{\pgfqpoint{0.904529in}{3.453131in}}%
\pgfpathlineto{\pgfqpoint{0.908629in}{3.097885in}}%
\pgfpathlineto{\pgfqpoint{0.911909in}{2.974067in}}%
\pgfpathlineto{\pgfqpoint{0.917239in}{2.877169in}}%
\pgfpathlineto{\pgfqpoint{0.918469in}{2.868590in}}%
\pgfpathlineto{\pgfqpoint{0.918879in}{2.869092in}}%
\pgfpathlineto{\pgfqpoint{0.920519in}{2.882173in}}%
\pgfpathlineto{\pgfqpoint{0.922569in}{2.894264in}}%
\pgfpathlineto{\pgfqpoint{0.923389in}{2.892673in}}%
\pgfpathlineto{\pgfqpoint{0.926669in}{2.872927in}}%
\pgfpathlineto{\pgfqpoint{0.935279in}{2.829566in}}%
\pgfpathlineto{\pgfqpoint{0.936509in}{2.828452in}}%
\pgfpathlineto{\pgfqpoint{0.936919in}{2.828812in}}%
\pgfpathlineto{\pgfqpoint{0.938969in}{2.834117in}}%
\pgfpathlineto{\pgfqpoint{0.941429in}{2.838962in}}%
\pgfpathlineto{\pgfqpoint{0.943069in}{2.838484in}}%
\pgfpathlineto{\pgfqpoint{0.946759in}{2.833498in}}%
\pgfpathlineto{\pgfqpoint{0.954549in}{2.821693in}}%
\pgfpathlineto{\pgfqpoint{0.956599in}{2.823459in}}%
\pgfpathlineto{\pgfqpoint{0.960699in}{2.826987in}}%
\pgfpathlineto{\pgfqpoint{0.963979in}{2.826112in}}%
\pgfpathlineto{\pgfqpoint{0.969309in}{2.821710in}}%
\pgfpathlineto{\pgfqpoint{0.972589in}{2.820411in}}%
\pgfpathlineto{\pgfqpoint{0.976279in}{2.822185in}}%
\pgfpathlineto{\pgfqpoint{0.980379in}{2.823288in}}%
\pgfpathlineto{\pgfqpoint{0.985299in}{2.821831in}}%
\pgfpathlineto{\pgfqpoint{0.991039in}{2.820265in}}%
\pgfpathlineto{\pgfqpoint{1.004159in}{2.820946in}}%
\pgfpathlineto{\pgfqpoint{1.010719in}{2.820448in}}%
\pgfpathlineto{\pgfqpoint{1.020559in}{2.820954in}}%
\pgfpathlineto{\pgfqpoint{1.031219in}{2.820646in}}%
\pgfpathlineto{\pgfqpoint{1.041059in}{2.820543in}}%
\pgfpathlineto{\pgfqpoint{1.052949in}{2.820689in}}%
\pgfpathlineto{\pgfqpoint{1.147659in}{2.820392in}}%
\pgfpathlineto{\pgfqpoint{1.374389in}{2.820281in}}%
\pgfpathlineto{\pgfqpoint{2.952066in}{2.820199in}}%
\pgfpathlineto{\pgfqpoint{2.952066in}{2.820199in}}%
\pgfusepath{stroke}%
\end{pgfscope}%
\begin{pgfscope}%
\pgfpathrectangle{\pgfqpoint{0.800000in}{2.544000in}}{\pgfqpoint{2.254545in}{1.680000in}}%
\pgfusepath{clip}%
\pgfsetrectcap%
\pgfsetroundjoin%
\pgfsetlinewidth{1.505625pt}%
\definecolor{currentstroke}{rgb}{0.549020,0.337255,0.294118}%
\pgfsetstrokecolor{currentstroke}%
\pgfsetdash{}{0pt}%
\pgfpathmoveto{\pgfqpoint{0.902479in}{3.626600in}}%
\pgfpathlineto{\pgfqpoint{0.923389in}{3.625529in}}%
\pgfpathlineto{\pgfqpoint{0.936509in}{3.622746in}}%
\pgfpathlineto{\pgfqpoint{0.946349in}{3.618471in}}%
\pgfpathlineto{\pgfqpoint{0.954959in}{3.612384in}}%
\pgfpathlineto{\pgfqpoint{0.963159in}{3.603903in}}%
\pgfpathlineto{\pgfqpoint{0.971769in}{3.591594in}}%
\pgfpathlineto{\pgfqpoint{0.980789in}{3.574540in}}%
\pgfpathlineto{\pgfqpoint{0.991039in}{3.550078in}}%
\pgfpathlineto{\pgfqpoint{1.006209in}{3.507118in}}%
\pgfpathlineto{\pgfqpoint{1.020559in}{3.468461in}}%
\pgfpathlineto{\pgfqpoint{1.027939in}{3.454270in}}%
\pgfpathlineto{\pgfqpoint{1.033269in}{3.448156in}}%
\pgfpathlineto{\pgfqpoint{1.037369in}{3.446378in}}%
\pgfpathlineto{\pgfqpoint{1.041059in}{3.447229in}}%
\pgfpathlineto{\pgfqpoint{1.044749in}{3.450579in}}%
\pgfpathlineto{\pgfqpoint{1.049259in}{3.458234in}}%
\pgfpathlineto{\pgfqpoint{1.054589in}{3.472436in}}%
\pgfpathlineto{\pgfqpoint{1.060739in}{3.495621in}}%
\pgfpathlineto{\pgfqpoint{1.068529in}{3.534515in}}%
\pgfpathlineto{\pgfqpoint{1.078779in}{3.598414in}}%
\pgfpathlineto{\pgfqpoint{1.116909in}{3.849655in}}%
\pgfpathlineto{\pgfqpoint{1.124699in}{3.882303in}}%
\pgfpathlineto{\pgfqpoint{1.130849in}{3.899822in}}%
\pgfpathlineto{\pgfqpoint{1.135769in}{3.907957in}}%
\pgfpathlineto{\pgfqpoint{1.139459in}{3.910373in}}%
\pgfpathlineto{\pgfqpoint{1.142329in}{3.909964in}}%
\pgfpathlineto{\pgfqpoint{1.145609in}{3.906959in}}%
\pgfpathlineto{\pgfqpoint{1.149299in}{3.900242in}}%
\pgfpathlineto{\pgfqpoint{1.153809in}{3.887090in}}%
\pgfpathlineto{\pgfqpoint{1.159139in}{3.864339in}}%
\pgfpathlineto{\pgfqpoint{1.165699in}{3.825380in}}%
\pgfpathlineto{\pgfqpoint{1.173079in}{3.767023in}}%
\pgfpathlineto{\pgfqpoint{1.181689in}{3.680146in}}%
\pgfpathlineto{\pgfqpoint{1.192349in}{3.548186in}}%
\pgfpathlineto{\pgfqpoint{1.223509in}{3.142898in}}%
\pgfpathlineto{\pgfqpoint{1.229249in}{3.101747in}}%
\pgfpathlineto{\pgfqpoint{1.233349in}{3.085838in}}%
\pgfpathlineto{\pgfqpoint{1.235809in}{3.082311in}}%
\pgfpathlineto{\pgfqpoint{1.237449in}{3.082586in}}%
\pgfpathlineto{\pgfqpoint{1.239499in}{3.085942in}}%
\pgfpathlineto{\pgfqpoint{1.242369in}{3.096305in}}%
\pgfpathlineto{\pgfqpoint{1.246059in}{3.119223in}}%
\pgfpathlineto{\pgfqpoint{1.250979in}{3.165672in}}%
\pgfpathlineto{\pgfqpoint{1.257539in}{3.252000in}}%
\pgfpathlineto{\pgfqpoint{1.266969in}{3.409575in}}%
\pgfpathlineto{\pgfqpoint{1.288699in}{3.780541in}}%
\pgfpathlineto{\pgfqpoint{1.296079in}{3.868045in}}%
\pgfpathlineto{\pgfqpoint{1.301819in}{3.912142in}}%
\pgfpathlineto{\pgfqpoint{1.305919in}{3.928449in}}%
\pgfpathlineto{\pgfqpoint{1.308379in}{3.931527in}}%
\pgfpathlineto{\pgfqpoint{1.310019in}{3.930621in}}%
\pgfpathlineto{\pgfqpoint{1.312069in}{3.926035in}}%
\pgfpathlineto{\pgfqpoint{1.314939in}{3.912957in}}%
\pgfpathlineto{\pgfqpoint{1.318629in}{3.884399in}}%
\pgfpathlineto{\pgfqpoint{1.323549in}{3.825837in}}%
\pgfpathlineto{\pgfqpoint{1.330109in}{3.716590in}}%
\pgfpathlineto{\pgfqpoint{1.340769in}{3.537456in}}%
\pgfpathlineto{\pgfqpoint{1.343639in}{3.517613in}}%
\pgfpathlineto{\pgfqpoint{1.344869in}{3.516138in}}%
\pgfpathlineto{\pgfqpoint{1.346099in}{3.519358in}}%
\pgfpathlineto{\pgfqpoint{1.348149in}{3.535410in}}%
\pgfpathlineto{\pgfqpoint{1.351429in}{3.586296in}}%
\pgfpathlineto{\pgfqpoint{1.361269in}{3.763658in}}%
\pgfpathlineto{\pgfqpoint{1.362909in}{3.768343in}}%
\pgfpathlineto{\pgfqpoint{1.363729in}{3.766376in}}%
\pgfpathlineto{\pgfqpoint{1.365369in}{3.753965in}}%
\pgfpathlineto{\pgfqpoint{1.368239in}{3.708751in}}%
\pgfpathlineto{\pgfqpoint{1.378079in}{3.527272in}}%
\pgfpathlineto{\pgfqpoint{1.379719in}{3.522160in}}%
\pgfpathlineto{\pgfqpoint{1.380539in}{3.523552in}}%
\pgfpathlineto{\pgfqpoint{1.382179in}{3.533930in}}%
\pgfpathlineto{\pgfqpoint{1.385049in}{3.573532in}}%
\pgfpathlineto{\pgfqpoint{1.389969in}{3.683600in}}%
\pgfpathlineto{\pgfqpoint{1.402269in}{3.970911in}}%
\pgfpathlineto{\pgfqpoint{1.407189in}{4.037356in}}%
\pgfpathlineto{\pgfqpoint{1.410879in}{4.062288in}}%
\pgfpathlineto{\pgfqpoint{1.412929in}{4.066664in}}%
\pgfpathlineto{\pgfqpoint{1.414159in}{4.066036in}}%
\pgfpathlineto{\pgfqpoint{1.415799in}{4.061415in}}%
\pgfpathlineto{\pgfqpoint{1.418259in}{4.046442in}}%
\pgfpathlineto{\pgfqpoint{1.421539in}{4.011713in}}%
\pgfpathlineto{\pgfqpoint{1.426049in}{3.937398in}}%
\pgfpathlineto{\pgfqpoint{1.431789in}{3.802083in}}%
\pgfpathlineto{\pgfqpoint{1.440398in}{3.536362in}}%
\pgfpathlineto{\pgfqpoint{1.451468in}{3.203832in}}%
\pgfpathlineto{\pgfqpoint{1.455978in}{3.128118in}}%
\pgfpathlineto{\pgfqpoint{1.458848in}{3.110633in}}%
\pgfpathlineto{\pgfqpoint{1.459668in}{3.110463in}}%
\pgfpathlineto{\pgfqpoint{1.460898in}{3.114301in}}%
\pgfpathlineto{\pgfqpoint{1.462948in}{3.131492in}}%
\pgfpathlineto{\pgfqpoint{1.465818in}{3.177049in}}%
\pgfpathlineto{\pgfqpoint{1.470328in}{3.291245in}}%
\pgfpathlineto{\pgfqpoint{1.478528in}{3.573460in}}%
\pgfpathlineto{\pgfqpoint{1.489188in}{3.921316in}}%
\pgfpathlineto{\pgfqpoint{1.494928in}{4.043718in}}%
\pgfpathlineto{\pgfqpoint{1.499028in}{4.090554in}}%
\pgfpathlineto{\pgfqpoint{1.501488in}{4.099971in}}%
\pgfpathlineto{\pgfqpoint{1.502308in}{4.099765in}}%
\pgfpathlineto{\pgfqpoint{1.503538in}{4.096195in}}%
\pgfpathlineto{\pgfqpoint{1.505588in}{4.081274in}}%
\pgfpathlineto{\pgfqpoint{1.508458in}{4.040898in}}%
\pgfpathlineto{\pgfqpoint{1.512558in}{3.943915in}}%
\pgfpathlineto{\pgfqpoint{1.524448in}{3.618210in}}%
\pgfpathlineto{\pgfqpoint{1.525268in}{3.621427in}}%
\pgfpathlineto{\pgfqpoint{1.526908in}{3.648821in}}%
\pgfpathlineto{\pgfqpoint{1.529778in}{3.756891in}}%
\pgfpathlineto{\pgfqpoint{1.535518in}{3.984325in}}%
\pgfpathlineto{\pgfqpoint{1.536338in}{3.991335in}}%
\pgfpathlineto{\pgfqpoint{1.536748in}{3.991270in}}%
\pgfpathlineto{\pgfqpoint{1.537568in}{3.983972in}}%
\pgfpathlineto{\pgfqpoint{1.539208in}{3.942631in}}%
\pgfpathlineto{\pgfqpoint{1.543308in}{3.753194in}}%
\pgfpathlineto{\pgfqpoint{1.546588in}{3.644276in}}%
\pgfpathlineto{\pgfqpoint{1.547818in}{3.633422in}}%
\pgfpathlineto{\pgfqpoint{1.548228in}{3.633735in}}%
\pgfpathlineto{\pgfqpoint{1.549458in}{3.645744in}}%
\pgfpathlineto{\pgfqpoint{1.551508in}{3.697015in}}%
\pgfpathlineto{\pgfqpoint{1.556018in}{3.884463in}}%
\pgfpathlineto{\pgfqpoint{1.563398in}{4.174197in}}%
\pgfpathlineto{\pgfqpoint{1.565795in}{4.234000in}}%
\pgfpathmoveto{\pgfqpoint{1.577815in}{4.234000in}}%
\pgfpathlineto{\pgfqpoint{1.581438in}{4.140654in}}%
\pgfpathlineto{\pgfqpoint{1.586358in}{3.952443in}}%
\pgfpathlineto{\pgfqpoint{1.594558in}{3.533211in}}%
\pgfpathlineto{\pgfqpoint{1.601528in}{3.222371in}}%
\pgfpathlineto{\pgfqpoint{1.604808in}{3.156943in}}%
\pgfpathlineto{\pgfqpoint{1.606038in}{3.151450in}}%
\pgfpathlineto{\pgfqpoint{1.606448in}{3.152050in}}%
\pgfpathlineto{\pgfqpoint{1.607678in}{3.161134in}}%
\pgfpathlineto{\pgfqpoint{1.609728in}{3.199765in}}%
\pgfpathlineto{\pgfqpoint{1.613008in}{3.314730in}}%
\pgfpathlineto{\pgfqpoint{1.619158in}{3.635408in}}%
\pgfpathlineto{\pgfqpoint{1.628998in}{4.131990in}}%
\pgfpathlineto{\pgfqpoint{1.631999in}{4.234000in}}%
\pgfpathmoveto{\pgfqpoint{1.645476in}{4.234000in}}%
\pgfpathlineto{\pgfqpoint{1.649498in}{4.060071in}}%
\pgfpathlineto{\pgfqpoint{1.656058in}{3.767730in}}%
\pgfpathlineto{\pgfqpoint{1.656878in}{3.764319in}}%
\pgfpathlineto{\pgfqpoint{1.657698in}{3.775196in}}%
\pgfpathlineto{\pgfqpoint{1.659338in}{3.841163in}}%
\pgfpathlineto{\pgfqpoint{1.665898in}{4.228701in}}%
\pgfpathlineto{\pgfqpoint{1.666718in}{4.223951in}}%
\pgfpathlineto{\pgfqpoint{1.668358in}{4.163128in}}%
\pgfpathlineto{\pgfqpoint{1.672458in}{3.855760in}}%
\pgfpathlineto{\pgfqpoint{1.675328in}{3.713702in}}%
\pgfpathlineto{\pgfqpoint{1.676558in}{3.700861in}}%
\pgfpathlineto{\pgfqpoint{1.676968in}{3.702847in}}%
\pgfpathlineto{\pgfqpoint{1.678198in}{3.725064in}}%
\pgfpathlineto{\pgfqpoint{1.681068in}{3.841634in}}%
\pgfpathlineto{\pgfqpoint{1.689268in}{4.207468in}}%
\pgfpathlineto{\pgfqpoint{1.690309in}{4.234000in}}%
\pgfpathmoveto{\pgfqpoint{1.699154in}{4.234000in}}%
\pgfpathlineto{\pgfqpoint{1.702388in}{4.142435in}}%
\pgfpathlineto{\pgfqpoint{1.707308in}{3.932180in}}%
\pgfpathlineto{\pgfqpoint{1.722068in}{3.245179in}}%
\pgfpathlineto{\pgfqpoint{1.724528in}{3.212148in}}%
\pgfpathlineto{\pgfqpoint{1.725348in}{3.210935in}}%
\pgfpathlineto{\pgfqpoint{1.726168in}{3.214619in}}%
\pgfpathlineto{\pgfqpoint{1.727808in}{3.235801in}}%
\pgfpathlineto{\pgfqpoint{1.730678in}{3.308444in}}%
\pgfpathlineto{\pgfqpoint{1.741748in}{3.624406in}}%
\pgfpathlineto{\pgfqpoint{1.744618in}{3.650276in}}%
\pgfpathlineto{\pgfqpoint{1.746668in}{3.655047in}}%
\pgfpathlineto{\pgfqpoint{1.748308in}{3.654421in}}%
\pgfpathlineto{\pgfqpoint{1.750358in}{3.653785in}}%
\pgfpathlineto{\pgfqpoint{1.751588in}{3.656285in}}%
\pgfpathlineto{\pgfqpoint{1.753228in}{3.665753in}}%
\pgfpathlineto{\pgfqpoint{1.755688in}{3.696068in}}%
\pgfpathlineto{\pgfqpoint{1.760608in}{3.795971in}}%
\pgfpathlineto{\pgfqpoint{1.765118in}{3.869034in}}%
\pgfpathlineto{\pgfqpoint{1.767988in}{3.885791in}}%
\pgfpathlineto{\pgfqpoint{1.770038in}{3.894228in}}%
\pgfpathlineto{\pgfqpoint{1.771268in}{3.910938in}}%
\pgfpathlineto{\pgfqpoint{1.772908in}{3.976579in}}%
\pgfpathlineto{\pgfqpoint{1.774958in}{4.204370in}}%
\pgfpathlineto{\pgfqpoint{1.775129in}{4.234000in}}%
\pgfpathmoveto{\pgfqpoint{1.790010in}{4.234000in}}%
\pgfpathlineto{\pgfqpoint{1.792588in}{4.100604in}}%
\pgfpathlineto{\pgfqpoint{1.795048in}{4.069595in}}%
\pgfpathlineto{\pgfqpoint{1.796688in}{4.066604in}}%
\pgfpathlineto{\pgfqpoint{1.800788in}{4.066445in}}%
\pgfpathlineto{\pgfqpoint{1.802428in}{4.057358in}}%
\pgfpathlineto{\pgfqpoint{1.804478in}{4.027883in}}%
\pgfpathlineto{\pgfqpoint{1.806938in}{3.954896in}}%
\pgfpathlineto{\pgfqpoint{1.811038in}{3.744568in}}%
\pgfpathlineto{\pgfqpoint{1.815958in}{3.514551in}}%
\pgfpathlineto{\pgfqpoint{1.818418in}{3.474566in}}%
\pgfpathlineto{\pgfqpoint{1.819238in}{3.472636in}}%
\pgfpathlineto{\pgfqpoint{1.819648in}{3.473045in}}%
\pgfpathlineto{\pgfqpoint{1.823338in}{3.481799in}}%
\pgfpathlineto{\pgfqpoint{1.823748in}{3.481423in}}%
\pgfpathlineto{\pgfqpoint{1.826618in}{3.476764in}}%
\pgfpathlineto{\pgfqpoint{1.827028in}{3.477328in}}%
\pgfpathlineto{\pgfqpoint{1.828258in}{3.484001in}}%
\pgfpathlineto{\pgfqpoint{1.829898in}{3.508488in}}%
\pgfpathlineto{\pgfqpoint{1.832768in}{3.596011in}}%
\pgfpathlineto{\pgfqpoint{1.841378in}{3.887451in}}%
\pgfpathlineto{\pgfqpoint{1.847118in}{4.071994in}}%
\pgfpathlineto{\pgfqpoint{1.852391in}{4.234000in}}%
\pgfpathmoveto{\pgfqpoint{1.856657in}{4.234000in}}%
\pgfpathlineto{\pgfqpoint{1.859008in}{4.207483in}}%
\pgfpathlineto{\pgfqpoint{1.859828in}{4.220658in}}%
\pgfpathlineto{\pgfqpoint{1.860175in}{4.234000in}}%
\pgfpathmoveto{\pgfqpoint{1.918883in}{4.234000in}}%
\pgfpathlineto{\pgfqpoint{1.923378in}{3.954504in}}%
\pgfpathlineto{\pgfqpoint{1.923788in}{3.952674in}}%
\pgfpathlineto{\pgfqpoint{1.924608in}{3.970279in}}%
\pgfpathlineto{\pgfqpoint{1.930348in}{4.232916in}}%
\pgfpathlineto{\pgfqpoint{1.930758in}{4.227522in}}%
\pgfpathlineto{\pgfqpoint{1.932808in}{4.161890in}}%
\pgfpathlineto{\pgfqpoint{1.933218in}{4.165577in}}%
\pgfpathlineto{\pgfqpoint{1.934409in}{4.234000in}}%
\pgfpathlineto{\pgfqpoint{1.934409in}{4.234000in}}%
\pgfusepath{stroke}%
\end{pgfscope}%
\begin{pgfscope}%
\pgfpathrectangle{\pgfqpoint{0.800000in}{2.544000in}}{\pgfqpoint{2.254545in}{1.680000in}}%
\pgfusepath{clip}%
\pgfsetrectcap%
\pgfsetroundjoin%
\pgfsetlinewidth{1.505625pt}%
\definecolor{currentstroke}{rgb}{0.890196,0.466667,0.760784}%
\pgfsetstrokecolor{currentstroke}%
\pgfsetdash{}{0pt}%
\pgfpathmoveto{\pgfqpoint{0.902479in}{3.626600in}}%
\pgfpathlineto{\pgfqpoint{0.928719in}{3.625518in}}%
\pgfpathlineto{\pgfqpoint{0.941019in}{3.622867in}}%
\pgfpathlineto{\pgfqpoint{0.950449in}{3.618630in}}%
\pgfpathlineto{\pgfqpoint{0.959059in}{3.612240in}}%
\pgfpathlineto{\pgfqpoint{0.967259in}{3.603236in}}%
\pgfpathlineto{\pgfqpoint{0.975459in}{3.590834in}}%
\pgfpathlineto{\pgfqpoint{0.984479in}{3.572900in}}%
\pgfpathlineto{\pgfqpoint{0.994729in}{3.547284in}}%
\pgfpathlineto{\pgfqpoint{1.011949in}{3.496881in}}%
\pgfpathlineto{\pgfqpoint{1.023839in}{3.465295in}}%
\pgfpathlineto{\pgfqpoint{1.030809in}{3.452232in}}%
\pgfpathlineto{\pgfqpoint{1.036139in}{3.446487in}}%
\pgfpathlineto{\pgfqpoint{1.040239in}{3.445114in}}%
\pgfpathlineto{\pgfqpoint{1.043929in}{3.446403in}}%
\pgfpathlineto{\pgfqpoint{1.047619in}{3.450242in}}%
\pgfpathlineto{\pgfqpoint{1.052129in}{3.458530in}}%
\pgfpathlineto{\pgfqpoint{1.057459in}{3.473463in}}%
\pgfpathlineto{\pgfqpoint{1.064019in}{3.499178in}}%
\pgfpathlineto{\pgfqpoint{1.072219in}{3.541317in}}%
\pgfpathlineto{\pgfqpoint{1.083699in}{3.613578in}}%
\pgfpathlineto{\pgfqpoint{1.111169in}{3.791315in}}%
\pgfpathlineto{\pgfqpoint{1.120189in}{3.833800in}}%
\pgfpathlineto{\pgfqpoint{1.127159in}{3.856936in}}%
\pgfpathlineto{\pgfqpoint{1.132489in}{3.867984in}}%
\pgfpathlineto{\pgfqpoint{1.136589in}{3.872201in}}%
\pgfpathlineto{\pgfqpoint{1.139869in}{3.872740in}}%
\pgfpathlineto{\pgfqpoint{1.142739in}{3.871066in}}%
\pgfpathlineto{\pgfqpoint{1.146429in}{3.865875in}}%
\pgfpathlineto{\pgfqpoint{1.150939in}{3.854749in}}%
\pgfpathlineto{\pgfqpoint{1.156269in}{3.834625in}}%
\pgfpathlineto{\pgfqpoint{1.162419in}{3.801805in}}%
\pgfpathlineto{\pgfqpoint{1.169389in}{3.752088in}}%
\pgfpathlineto{\pgfqpoint{1.177589in}{3.676993in}}%
\pgfpathlineto{\pgfqpoint{1.187839in}{3.560446in}}%
\pgfpathlineto{\pgfqpoint{1.203419in}{3.352783in}}%
\pgfpathlineto{\pgfqpoint{1.217359in}{3.177767in}}%
\pgfpathlineto{\pgfqpoint{1.224329in}{3.114434in}}%
\pgfpathlineto{\pgfqpoint{1.229249in}{3.085895in}}%
\pgfpathlineto{\pgfqpoint{1.232939in}{3.075066in}}%
\pgfpathlineto{\pgfqpoint{1.235399in}{3.073330in}}%
\pgfpathlineto{\pgfqpoint{1.237039in}{3.074700in}}%
\pgfpathlineto{\pgfqpoint{1.239499in}{3.080606in}}%
\pgfpathlineto{\pgfqpoint{1.242779in}{3.095672in}}%
\pgfpathlineto{\pgfqpoint{1.246879in}{3.125751in}}%
\pgfpathlineto{\pgfqpoint{1.252619in}{3.187072in}}%
\pgfpathlineto{\pgfqpoint{1.260409in}{3.298794in}}%
\pgfpathlineto{\pgfqpoint{1.277629in}{3.591946in}}%
\pgfpathlineto{\pgfqpoint{1.287879in}{3.743325in}}%
\pgfpathlineto{\pgfqpoint{1.294849in}{3.817598in}}%
\pgfpathlineto{\pgfqpoint{1.300179in}{3.853284in}}%
\pgfpathlineto{\pgfqpoint{1.303869in}{3.865538in}}%
\pgfpathlineto{\pgfqpoint{1.306329in}{3.867536in}}%
\pgfpathlineto{\pgfqpoint{1.307969in}{3.865986in}}%
\pgfpathlineto{\pgfqpoint{1.310429in}{3.859188in}}%
\pgfpathlineto{\pgfqpoint{1.313709in}{3.841532in}}%
\pgfpathlineto{\pgfqpoint{1.317809in}{3.805435in}}%
\pgfpathlineto{\pgfqpoint{1.323139in}{3.736172in}}%
\pgfpathlineto{\pgfqpoint{1.331749in}{3.587706in}}%
\pgfpathlineto{\pgfqpoint{1.338719in}{3.480937in}}%
\pgfpathlineto{\pgfqpoint{1.341999in}{3.456298in}}%
\pgfpathlineto{\pgfqpoint{1.343639in}{3.453388in}}%
\pgfpathlineto{\pgfqpoint{1.344869in}{3.455708in}}%
\pgfpathlineto{\pgfqpoint{1.346919in}{3.468071in}}%
\pgfpathlineto{\pgfqpoint{1.350199in}{3.506683in}}%
\pgfpathlineto{\pgfqpoint{1.358809in}{3.619900in}}%
\pgfpathlineto{\pgfqpoint{1.360859in}{3.625060in}}%
\pgfpathlineto{\pgfqpoint{1.362089in}{3.621768in}}%
\pgfpathlineto{\pgfqpoint{1.364139in}{3.605982in}}%
\pgfpathlineto{\pgfqpoint{1.367829in}{3.552797in}}%
\pgfpathlineto{\pgfqpoint{1.374799in}{3.449336in}}%
\pgfpathlineto{\pgfqpoint{1.377259in}{3.437325in}}%
\pgfpathlineto{\pgfqpoint{1.378079in}{3.437377in}}%
\pgfpathlineto{\pgfqpoint{1.379309in}{3.441260in}}%
\pgfpathlineto{\pgfqpoint{1.381359in}{3.457293in}}%
\pgfpathlineto{\pgfqpoint{1.384639in}{3.503509in}}%
\pgfpathlineto{\pgfqpoint{1.391609in}{3.643297in}}%
\pgfpathlineto{\pgfqpoint{1.399399in}{3.783153in}}%
\pgfpathlineto{\pgfqpoint{1.403909in}{3.831852in}}%
\pgfpathlineto{\pgfqpoint{1.407189in}{3.848754in}}%
\pgfpathlineto{\pgfqpoint{1.409239in}{3.851080in}}%
\pgfpathlineto{\pgfqpoint{1.410469in}{3.849416in}}%
\pgfpathlineto{\pgfqpoint{1.412519in}{3.841573in}}%
\pgfpathlineto{\pgfqpoint{1.415389in}{3.820098in}}%
\pgfpathlineto{\pgfqpoint{1.419489in}{3.768983in}}%
\pgfpathlineto{\pgfqpoint{1.424819in}{3.669775in}}%
\pgfpathlineto{\pgfqpoint{1.432199in}{3.485586in}}%
\pgfpathlineto{\pgfqpoint{1.445728in}{3.142178in}}%
\pgfpathlineto{\pgfqpoint{1.450238in}{3.081615in}}%
\pgfpathlineto{\pgfqpoint{1.452698in}{3.069048in}}%
\pgfpathlineto{\pgfqpoint{1.453518in}{3.068363in}}%
\pgfpathlineto{\pgfqpoint{1.453928in}{3.068690in}}%
\pgfpathlineto{\pgfqpoint{1.455158in}{3.072347in}}%
\pgfpathlineto{\pgfqpoint{1.457208in}{3.087266in}}%
\pgfpathlineto{\pgfqpoint{1.460488in}{3.132772in}}%
\pgfpathlineto{\pgfqpoint{1.464998in}{3.231477in}}%
\pgfpathlineto{\pgfqpoint{1.474018in}{3.491834in}}%
\pgfpathlineto{\pgfqpoint{1.482628in}{3.716253in}}%
\pgfpathlineto{\pgfqpoint{1.487958in}{3.803888in}}%
\pgfpathlineto{\pgfqpoint{1.491238in}{3.830594in}}%
\pgfpathlineto{\pgfqpoint{1.493288in}{3.835368in}}%
\pgfpathlineto{\pgfqpoint{1.494518in}{3.833584in}}%
\pgfpathlineto{\pgfqpoint{1.496158in}{3.825620in}}%
\pgfpathlineto{\pgfqpoint{1.498618in}{3.801518in}}%
\pgfpathlineto{\pgfqpoint{1.502308in}{3.738588in}}%
\pgfpathlineto{\pgfqpoint{1.508048in}{3.591597in}}%
\pgfpathlineto{\pgfqpoint{1.514198in}{3.448705in}}%
\pgfpathlineto{\pgfqpoint{1.516658in}{3.431951in}}%
\pgfpathlineto{\pgfqpoint{1.517478in}{3.434001in}}%
\pgfpathlineto{\pgfqpoint{1.519118in}{3.449315in}}%
\pgfpathlineto{\pgfqpoint{1.522398in}{3.511206in}}%
\pgfpathlineto{\pgfqpoint{1.526498in}{3.576494in}}%
\pgfpathlineto{\pgfqpoint{1.527318in}{3.579384in}}%
\pgfpathlineto{\pgfqpoint{1.527728in}{3.579159in}}%
\pgfpathlineto{\pgfqpoint{1.528958in}{3.571789in}}%
\pgfpathlineto{\pgfqpoint{1.531008in}{3.539778in}}%
\pgfpathlineto{\pgfqpoint{1.537978in}{3.409331in}}%
\pgfpathlineto{\pgfqpoint{1.538798in}{3.407486in}}%
\pgfpathlineto{\pgfqpoint{1.539208in}{3.408140in}}%
\pgfpathlineto{\pgfqpoint{1.540438in}{3.416233in}}%
\pgfpathlineto{\pgfqpoint{1.542898in}{3.456531in}}%
\pgfpathlineto{\pgfqpoint{1.547818in}{3.589625in}}%
\pgfpathlineto{\pgfqpoint{1.554378in}{3.746960in}}%
\pgfpathlineto{\pgfqpoint{1.557658in}{3.785498in}}%
\pgfpathlineto{\pgfqpoint{1.559708in}{3.793057in}}%
\pgfpathlineto{\pgfqpoint{1.560528in}{3.792462in}}%
\pgfpathlineto{\pgfqpoint{1.562168in}{3.785116in}}%
\pgfpathlineto{\pgfqpoint{1.564628in}{3.759060in}}%
\pgfpathlineto{\pgfqpoint{1.568318in}{3.688282in}}%
\pgfpathlineto{\pgfqpoint{1.573238in}{3.544302in}}%
\pgfpathlineto{\pgfqpoint{1.587998in}{3.078766in}}%
\pgfpathlineto{\pgfqpoint{1.590458in}{3.057698in}}%
\pgfpathlineto{\pgfqpoint{1.591278in}{3.056957in}}%
\pgfpathlineto{\pgfqpoint{1.592508in}{3.061845in}}%
\pgfpathlineto{\pgfqpoint{1.594558in}{3.085618in}}%
\pgfpathlineto{\pgfqpoint{1.597838in}{3.159888in}}%
\pgfpathlineto{\pgfqpoint{1.603168in}{3.343404in}}%
\pgfpathlineto{\pgfqpoint{1.613008in}{3.678861in}}%
\pgfpathlineto{\pgfqpoint{1.617108in}{3.752765in}}%
\pgfpathlineto{\pgfqpoint{1.619568in}{3.768488in}}%
\pgfpathlineto{\pgfqpoint{1.620388in}{3.768484in}}%
\pgfpathlineto{\pgfqpoint{1.621618in}{3.763378in}}%
\pgfpathlineto{\pgfqpoint{1.623668in}{3.741082in}}%
\pgfpathlineto{\pgfqpoint{1.626538in}{3.681702in}}%
\pgfpathlineto{\pgfqpoint{1.631868in}{3.510711in}}%
\pgfpathlineto{\pgfqpoint{1.635968in}{3.403044in}}%
\pgfpathlineto{\pgfqpoint{1.637608in}{3.390921in}}%
\pgfpathlineto{\pgfqpoint{1.638428in}{3.393492in}}%
\pgfpathlineto{\pgfqpoint{1.640068in}{3.414194in}}%
\pgfpathlineto{\pgfqpoint{1.645808in}{3.503856in}}%
\pgfpathlineto{\pgfqpoint{1.646628in}{3.499229in}}%
\pgfpathlineto{\pgfqpoint{1.648268in}{3.473127in}}%
\pgfpathlineto{\pgfqpoint{1.654008in}{3.361175in}}%
\pgfpathlineto{\pgfqpoint{1.654418in}{3.360818in}}%
\pgfpathlineto{\pgfqpoint{1.655238in}{3.364606in}}%
\pgfpathlineto{\pgfqpoint{1.656878in}{3.388881in}}%
\pgfpathlineto{\pgfqpoint{1.660158in}{3.482679in}}%
\pgfpathlineto{\pgfqpoint{1.666718in}{3.670492in}}%
\pgfpathlineto{\pgfqpoint{1.669588in}{3.703682in}}%
\pgfpathlineto{\pgfqpoint{1.670818in}{3.706057in}}%
\pgfpathlineto{\pgfqpoint{1.672048in}{3.701257in}}%
\pgfpathlineto{\pgfqpoint{1.674098in}{3.677684in}}%
\pgfpathlineto{\pgfqpoint{1.676968in}{3.614174in}}%
\pgfpathlineto{\pgfqpoint{1.681478in}{3.456417in}}%
\pgfpathlineto{\pgfqpoint{1.691728in}{3.072666in}}%
\pgfpathlineto{\pgfqpoint{1.694188in}{3.041278in}}%
\pgfpathlineto{\pgfqpoint{1.695008in}{3.039693in}}%
\pgfpathlineto{\pgfqpoint{1.695418in}{3.040615in}}%
\pgfpathlineto{\pgfqpoint{1.696648in}{3.050179in}}%
\pgfpathlineto{\pgfqpoint{1.698698in}{3.087617in}}%
\pgfpathlineto{\pgfqpoint{1.702388in}{3.208905in}}%
\pgfpathlineto{\pgfqpoint{1.714278in}{3.642701in}}%
\pgfpathlineto{\pgfqpoint{1.717148in}{3.673735in}}%
\pgfpathlineto{\pgfqpoint{1.717968in}{3.674247in}}%
\pgfpathlineto{\pgfqpoint{1.719198in}{3.667716in}}%
\pgfpathlineto{\pgfqpoint{1.721248in}{3.637319in}}%
\pgfpathlineto{\pgfqpoint{1.724528in}{3.543151in}}%
\pgfpathlineto{\pgfqpoint{1.731088in}{3.338998in}}%
\pgfpathlineto{\pgfqpoint{1.732318in}{3.333436in}}%
\pgfpathlineto{\pgfqpoint{1.733138in}{3.338160in}}%
\pgfpathlineto{\pgfqpoint{1.735188in}{3.370280in}}%
\pgfpathlineto{\pgfqpoint{1.738058in}{3.408060in}}%
\pgfpathlineto{\pgfqpoint{1.738468in}{3.408136in}}%
\pgfpathlineto{\pgfqpoint{1.739288in}{3.403147in}}%
\pgfpathlineto{\pgfqpoint{1.741338in}{3.366113in}}%
\pgfpathlineto{\pgfqpoint{1.745028in}{3.300794in}}%
\pgfpathlineto{\pgfqpoint{1.745438in}{3.300116in}}%
\pgfpathlineto{\pgfqpoint{1.745848in}{3.301220in}}%
\pgfpathlineto{\pgfqpoint{1.747078in}{3.314949in}}%
\pgfpathlineto{\pgfqpoint{1.749538in}{3.379436in}}%
\pgfpathlineto{\pgfqpoint{1.756508in}{3.581011in}}%
\pgfpathlineto{\pgfqpoint{1.758558in}{3.596938in}}%
\pgfpathlineto{\pgfqpoint{1.759378in}{3.595864in}}%
\pgfpathlineto{\pgfqpoint{1.760608in}{3.586307in}}%
\pgfpathlineto{\pgfqpoint{1.762658in}{3.550053in}}%
\pgfpathlineto{\pgfqpoint{1.765938in}{3.446185in}}%
\pgfpathlineto{\pgfqpoint{1.777828in}{3.020166in}}%
\pgfpathlineto{\pgfqpoint{1.778648in}{3.017530in}}%
\pgfpathlineto{\pgfqpoint{1.779058in}{3.018347in}}%
\pgfpathlineto{\pgfqpoint{1.780288in}{3.029244in}}%
\pgfpathlineto{\pgfqpoint{1.782338in}{3.073608in}}%
\pgfpathlineto{\pgfqpoint{1.786028in}{3.213284in}}%
\pgfpathlineto{\pgfqpoint{1.793818in}{3.516496in}}%
\pgfpathlineto{\pgfqpoint{1.796688in}{3.558066in}}%
\pgfpathlineto{\pgfqpoint{1.797508in}{3.559422in}}%
\pgfpathlineto{\pgfqpoint{1.798328in}{3.555844in}}%
\pgfpathlineto{\pgfqpoint{1.799968in}{3.533906in}}%
\pgfpathlineto{\pgfqpoint{1.802838in}{3.453166in}}%
\pgfpathlineto{\pgfqpoint{1.808988in}{3.265729in}}%
\pgfpathlineto{\pgfqpoint{1.809808in}{3.262423in}}%
\pgfpathlineto{\pgfqpoint{1.810218in}{3.263316in}}%
\pgfpathlineto{\pgfqpoint{1.811858in}{3.278923in}}%
\pgfpathlineto{\pgfqpoint{1.814318in}{3.302215in}}%
\pgfpathlineto{\pgfqpoint{1.814728in}{3.302138in}}%
\pgfpathlineto{\pgfqpoint{1.815958in}{3.292771in}}%
\pgfpathlineto{\pgfqpoint{1.820468in}{3.229359in}}%
\pgfpathlineto{\pgfqpoint{1.821288in}{3.232714in}}%
\pgfpathlineto{\pgfqpoint{1.822928in}{3.260219in}}%
\pgfpathlineto{\pgfqpoint{1.831948in}{3.474089in}}%
\pgfpathlineto{\pgfqpoint{1.832768in}{3.471386in}}%
\pgfpathlineto{\pgfqpoint{1.834408in}{3.450881in}}%
\pgfpathlineto{\pgfqpoint{1.836868in}{3.385822in}}%
\pgfpathlineto{\pgfqpoint{1.841788in}{3.179346in}}%
\pgfpathlineto{\pgfqpoint{1.846708in}{3.009626in}}%
\pgfpathlineto{\pgfqpoint{1.848758in}{2.991637in}}%
\pgfpathlineto{\pgfqpoint{1.849578in}{2.995558in}}%
\pgfpathlineto{\pgfqpoint{1.851218in}{3.021783in}}%
\pgfpathlineto{\pgfqpoint{1.854088in}{3.115127in}}%
\pgfpathlineto{\pgfqpoint{1.862698in}{3.420483in}}%
\pgfpathlineto{\pgfqpoint{1.864338in}{3.432847in}}%
\pgfpathlineto{\pgfqpoint{1.864748in}{3.432437in}}%
\pgfpathlineto{\pgfqpoint{1.865978in}{3.422684in}}%
\pgfpathlineto{\pgfqpoint{1.868028in}{3.379694in}}%
\pgfpathlineto{\pgfqpoint{1.875818in}{3.183007in}}%
\pgfpathlineto{\pgfqpoint{1.877048in}{3.188194in}}%
\pgfpathlineto{\pgfqpoint{1.879098in}{3.196590in}}%
\pgfpathlineto{\pgfqpoint{1.879918in}{3.193764in}}%
\pgfpathlineto{\pgfqpoint{1.881968in}{3.171310in}}%
\pgfpathlineto{\pgfqpoint{1.884018in}{3.153542in}}%
\pgfpathlineto{\pgfqpoint{1.884428in}{3.153711in}}%
\pgfpathlineto{\pgfqpoint{1.885658in}{3.164078in}}%
\pgfpathlineto{\pgfqpoint{1.888118in}{3.221842in}}%
\pgfpathlineto{\pgfqpoint{1.893038in}{3.340513in}}%
\pgfpathlineto{\pgfqpoint{1.894268in}{3.346893in}}%
\pgfpathlineto{\pgfqpoint{1.894678in}{3.346242in}}%
\pgfpathlineto{\pgfqpoint{1.895908in}{3.335938in}}%
\pgfpathlineto{\pgfqpoint{1.897958in}{3.292693in}}%
\pgfpathlineto{\pgfqpoint{1.902058in}{3.139813in}}%
\pgfpathlineto{\pgfqpoint{1.906978in}{2.977984in}}%
\pgfpathlineto{\pgfqpoint{1.908618in}{2.963428in}}%
\pgfpathlineto{\pgfqpoint{1.909028in}{2.963886in}}%
\pgfpathlineto{\pgfqpoint{1.910258in}{2.975008in}}%
\pgfpathlineto{\pgfqpoint{1.912308in}{3.022481in}}%
\pgfpathlineto{\pgfqpoint{1.922148in}{3.304751in}}%
\pgfpathlineto{\pgfqpoint{1.922968in}{3.302457in}}%
\pgfpathlineto{\pgfqpoint{1.924608in}{3.280519in}}%
\pgfpathlineto{\pgfqpoint{1.927888in}{3.187917in}}%
\pgfpathlineto{\pgfqpoint{1.931578in}{3.105953in}}%
\pgfpathlineto{\pgfqpoint{1.933218in}{3.100125in}}%
\pgfpathlineto{\pgfqpoint{1.934858in}{3.100199in}}%
\pgfpathlineto{\pgfqpoint{1.936088in}{3.095891in}}%
\pgfpathlineto{\pgfqpoint{1.939368in}{3.077989in}}%
\pgfpathlineto{\pgfqpoint{1.939778in}{3.079016in}}%
\pgfpathlineto{\pgfqpoint{1.941008in}{3.090154in}}%
\pgfpathlineto{\pgfqpoint{1.943468in}{3.142045in}}%
\pgfpathlineto{\pgfqpoint{1.947568in}{3.221844in}}%
\pgfpathlineto{\pgfqpoint{1.948388in}{3.225262in}}%
\pgfpathlineto{\pgfqpoint{1.948798in}{3.224902in}}%
\pgfpathlineto{\pgfqpoint{1.950028in}{3.215485in}}%
\pgfpathlineto{\pgfqpoint{1.952078in}{3.174110in}}%
\pgfpathlineto{\pgfqpoint{1.961098in}{2.934807in}}%
\pgfpathlineto{\pgfqpoint{1.961508in}{2.935376in}}%
\pgfpathlineto{\pgfqpoint{1.962738in}{2.946292in}}%
\pgfpathlineto{\pgfqpoint{1.965198in}{3.002753in}}%
\pgfpathlineto{\pgfqpoint{1.972168in}{3.182561in}}%
\pgfpathlineto{\pgfqpoint{1.972988in}{3.185287in}}%
\pgfpathlineto{\pgfqpoint{1.973398in}{3.184498in}}%
\pgfpathlineto{\pgfqpoint{1.974628in}{3.173765in}}%
\pgfpathlineto{\pgfqpoint{1.977088in}{3.122337in}}%
\pgfpathlineto{\pgfqpoint{1.981598in}{3.030324in}}%
\pgfpathlineto{\pgfqpoint{1.984058in}{3.019024in}}%
\pgfpathlineto{\pgfqpoint{1.987748in}{3.008492in}}%
\pgfpathlineto{\pgfqpoint{1.988158in}{3.009265in}}%
\pgfpathlineto{\pgfqpoint{1.989388in}{3.017374in}}%
\pgfpathlineto{\pgfqpoint{1.991848in}{3.057377in}}%
\pgfpathlineto{\pgfqpoint{1.995538in}{3.114864in}}%
\pgfpathlineto{\pgfqpoint{1.996358in}{3.117486in}}%
\pgfpathlineto{\pgfqpoint{1.996768in}{3.116915in}}%
\pgfpathlineto{\pgfqpoint{1.997998in}{3.107601in}}%
\pgfpathlineto{\pgfqpoint{2.000048in}{3.069343in}}%
\pgfpathlineto{\pgfqpoint{2.007428in}{2.908063in}}%
\pgfpathlineto{\pgfqpoint{2.007838in}{2.907871in}}%
\pgfpathlineto{\pgfqpoint{2.008658in}{2.911614in}}%
\pgfpathlineto{\pgfqpoint{2.010298in}{2.934105in}}%
\pgfpathlineto{\pgfqpoint{2.018088in}{3.082087in}}%
\pgfpathlineto{\pgfqpoint{2.018498in}{3.081660in}}%
\pgfpathlineto{\pgfqpoint{2.019728in}{3.073077in}}%
\pgfpathlineto{\pgfqpoint{2.022188in}{3.031262in}}%
\pgfpathlineto{\pgfqpoint{2.026288in}{2.965387in}}%
\pgfpathlineto{\pgfqpoint{2.029158in}{2.951739in}}%
\pgfpathlineto{\pgfqpoint{2.030798in}{2.949574in}}%
\pgfpathlineto{\pgfqpoint{2.031618in}{2.951058in}}%
\pgfpathlineto{\pgfqpoint{2.033258in}{2.962114in}}%
\pgfpathlineto{\pgfqpoint{2.039408in}{3.028444in}}%
\pgfpathlineto{\pgfqpoint{2.039818in}{3.027613in}}%
\pgfpathlineto{\pgfqpoint{2.041048in}{3.018754in}}%
\pgfpathlineto{\pgfqpoint{2.043508in}{2.977301in}}%
\pgfpathlineto{\pgfqpoint{2.048838in}{2.885728in}}%
\pgfpathlineto{\pgfqpoint{2.049658in}{2.884043in}}%
\pgfpathlineto{\pgfqpoint{2.050068in}{2.884912in}}%
\pgfpathlineto{\pgfqpoint{2.051298in}{2.893978in}}%
\pgfpathlineto{\pgfqpoint{2.054168in}{2.941208in}}%
\pgfpathlineto{\pgfqpoint{2.057858in}{2.995694in}}%
\pgfpathlineto{\pgfqpoint{2.059088in}{2.999254in}}%
\pgfpathlineto{\pgfqpoint{2.059908in}{2.996545in}}%
\pgfpathlineto{\pgfqpoint{2.061958in}{2.975056in}}%
\pgfpathlineto{\pgfqpoint{2.067288in}{2.911111in}}%
\pgfpathlineto{\pgfqpoint{2.069748in}{2.904281in}}%
\pgfpathlineto{\pgfqpoint{2.070568in}{2.904934in}}%
\pgfpathlineto{\pgfqpoint{2.071797in}{2.909410in}}%
\pgfpathlineto{\pgfqpoint{2.074257in}{2.930747in}}%
\pgfpathlineto{\pgfqpoint{2.077947in}{2.959548in}}%
\pgfpathlineto{\pgfqpoint{2.078767in}{2.959023in}}%
\pgfpathlineto{\pgfqpoint{2.079997in}{2.952016in}}%
\pgfpathlineto{\pgfqpoint{2.082457in}{2.920305in}}%
\pgfpathlineto{\pgfqpoint{2.086967in}{2.865713in}}%
\pgfpathlineto{\pgfqpoint{2.087787in}{2.864529in}}%
\pgfpathlineto{\pgfqpoint{2.088197in}{2.865251in}}%
\pgfpathlineto{\pgfqpoint{2.089427in}{2.872293in}}%
\pgfpathlineto{\pgfqpoint{2.092707in}{2.911914in}}%
\pgfpathlineto{\pgfqpoint{2.095577in}{2.936361in}}%
\pgfpathlineto{\pgfqpoint{2.096397in}{2.937149in}}%
\pgfpathlineto{\pgfqpoint{2.096807in}{2.936386in}}%
\pgfpathlineto{\pgfqpoint{2.098447in}{2.926548in}}%
\pgfpathlineto{\pgfqpoint{2.105007in}{2.873556in}}%
\pgfpathlineto{\pgfqpoint{2.106237in}{2.872522in}}%
\pgfpathlineto{\pgfqpoint{2.106647in}{2.872922in}}%
\pgfpathlineto{\pgfqpoint{2.108287in}{2.878336in}}%
\pgfpathlineto{\pgfqpoint{2.114027in}{2.909765in}}%
\pgfpathlineto{\pgfqpoint{2.114847in}{2.908243in}}%
\pgfpathlineto{\pgfqpoint{2.116487in}{2.898239in}}%
\pgfpathlineto{\pgfqpoint{2.122637in}{2.849497in}}%
\pgfpathlineto{\pgfqpoint{2.123457in}{2.850597in}}%
\pgfpathlineto{\pgfqpoint{2.125097in}{2.859357in}}%
\pgfpathlineto{\pgfqpoint{2.130427in}{2.893454in}}%
\pgfpathlineto{\pgfqpoint{2.131247in}{2.892638in}}%
\pgfpathlineto{\pgfqpoint{2.132887in}{2.885272in}}%
\pgfpathlineto{\pgfqpoint{2.139037in}{2.851715in}}%
\pgfpathlineto{\pgfqpoint{2.140267in}{2.851965in}}%
\pgfpathlineto{\pgfqpoint{2.141907in}{2.856454in}}%
\pgfpathlineto{\pgfqpoint{2.147237in}{2.875313in}}%
\pgfpathlineto{\pgfqpoint{2.148467in}{2.872522in}}%
\pgfpathlineto{\pgfqpoint{2.150927in}{2.858070in}}%
\pgfpathlineto{\pgfqpoint{2.154617in}{2.839080in}}%
\pgfpathlineto{\pgfqpoint{2.155437in}{2.838787in}}%
\pgfpathlineto{\pgfqpoint{2.155847in}{2.839285in}}%
\pgfpathlineto{\pgfqpoint{2.157487in}{2.844883in}}%
\pgfpathlineto{\pgfqpoint{2.162407in}{2.864247in}}%
\pgfpathlineto{\pgfqpoint{2.163637in}{2.862670in}}%
\pgfpathlineto{\pgfqpoint{2.166097in}{2.852551in}}%
\pgfpathlineto{\pgfqpoint{2.169787in}{2.839011in}}%
\pgfpathlineto{\pgfqpoint{2.171427in}{2.838555in}}%
\pgfpathlineto{\pgfqpoint{2.173067in}{2.841652in}}%
\pgfpathlineto{\pgfqpoint{2.177987in}{2.852863in}}%
\pgfpathlineto{\pgfqpoint{2.179217in}{2.851053in}}%
\pgfpathlineto{\pgfqpoint{2.182087in}{2.839989in}}%
\pgfpathlineto{\pgfqpoint{2.184957in}{2.831575in}}%
\pgfpathlineto{\pgfqpoint{2.186187in}{2.831777in}}%
\pgfpathlineto{\pgfqpoint{2.188237in}{2.836691in}}%
\pgfpathlineto{\pgfqpoint{2.191927in}{2.845604in}}%
\pgfpathlineto{\pgfqpoint{2.193157in}{2.845070in}}%
\pgfpathlineto{\pgfqpoint{2.195207in}{2.840335in}}%
\pgfpathlineto{\pgfqpoint{2.199307in}{2.830591in}}%
\pgfpathlineto{\pgfqpoint{2.200947in}{2.830681in}}%
\pgfpathlineto{\pgfqpoint{2.203407in}{2.834711in}}%
\pgfpathlineto{\pgfqpoint{2.206277in}{2.838820in}}%
\pgfpathlineto{\pgfqpoint{2.207917in}{2.837870in}}%
\pgfpathlineto{\pgfqpoint{2.210787in}{2.831252in}}%
\pgfpathlineto{\pgfqpoint{2.213657in}{2.826663in}}%
\pgfpathlineto{\pgfqpoint{2.215297in}{2.827593in}}%
\pgfpathlineto{\pgfqpoint{2.221037in}{2.834093in}}%
\pgfpathlineto{\pgfqpoint{2.223497in}{2.830535in}}%
\pgfpathlineto{\pgfqpoint{2.227187in}{2.825711in}}%
\pgfpathlineto{\pgfqpoint{2.229237in}{2.826476in}}%
\pgfpathlineto{\pgfqpoint{2.234157in}{2.830374in}}%
\pgfpathlineto{\pgfqpoint{2.236207in}{2.828503in}}%
\pgfpathlineto{\pgfqpoint{2.240717in}{2.823752in}}%
\pgfpathlineto{\pgfqpoint{2.242767in}{2.824933in}}%
\pgfpathlineto{\pgfqpoint{2.246867in}{2.827789in}}%
\pgfpathlineto{\pgfqpoint{2.249327in}{2.826128in}}%
\pgfpathlineto{\pgfqpoint{2.253427in}{2.823052in}}%
\pgfpathlineto{\pgfqpoint{2.255887in}{2.823961in}}%
\pgfpathlineto{\pgfqpoint{2.259987in}{2.825524in}}%
\pgfpathlineto{\pgfqpoint{2.263267in}{2.823293in}}%
\pgfpathlineto{\pgfqpoint{2.266547in}{2.822135in}}%
\pgfpathlineto{\pgfqpoint{2.274337in}{2.823133in}}%
\pgfpathlineto{\pgfqpoint{2.278847in}{2.821687in}}%
\pgfpathlineto{\pgfqpoint{2.286227in}{2.822260in}}%
\pgfpathlineto{\pgfqpoint{2.290737in}{2.821207in}}%
\pgfpathlineto{\pgfqpoint{2.298117in}{2.821561in}}%
\pgfpathlineto{\pgfqpoint{2.303037in}{2.821032in}}%
\pgfpathlineto{\pgfqpoint{2.309597in}{2.821089in}}%
\pgfpathlineto{\pgfqpoint{2.314927in}{2.820858in}}%
\pgfpathlineto{\pgfqpoint{2.321077in}{2.820737in}}%
\pgfpathlineto{\pgfqpoint{2.328047in}{2.820812in}}%
\pgfpathlineto{\pgfqpoint{2.364947in}{2.820242in}}%
\pgfpathlineto{\pgfqpoint{2.576507in}{2.820175in}}%
\pgfpathlineto{\pgfqpoint{2.952066in}{2.820175in}}%
\pgfpathlineto{\pgfqpoint{2.952066in}{2.820175in}}%
\pgfusepath{stroke}%
\end{pgfscope}%
\begin{pgfscope}%
\pgfpathrectangle{\pgfqpoint{0.800000in}{2.544000in}}{\pgfqpoint{2.254545in}{1.680000in}}%
\pgfusepath{clip}%
\pgfsetrectcap%
\pgfsetroundjoin%
\pgfsetlinewidth{1.505625pt}%
\definecolor{currentstroke}{rgb}{0.498039,0.498039,0.498039}%
\pgfsetstrokecolor{currentstroke}%
\pgfsetdash{}{0pt}%
\pgfpathmoveto{\pgfqpoint{0.902479in}{3.626600in}}%
\pgfpathlineto{\pgfqpoint{0.910269in}{3.625595in}}%
\pgfpathlineto{\pgfqpoint{0.917239in}{3.622544in}}%
\pgfpathlineto{\pgfqpoint{0.923799in}{3.617315in}}%
\pgfpathlineto{\pgfqpoint{0.930359in}{3.609142in}}%
\pgfpathlineto{\pgfqpoint{0.936919in}{3.597322in}}%
\pgfpathlineto{\pgfqpoint{0.944299in}{3.578974in}}%
\pgfpathlineto{\pgfqpoint{0.952499in}{3.552157in}}%
\pgfpathlineto{\pgfqpoint{0.964799in}{3.502985in}}%
\pgfpathlineto{\pgfqpoint{0.977099in}{3.456574in}}%
\pgfpathlineto{\pgfqpoint{0.983249in}{3.440945in}}%
\pgfpathlineto{\pgfqpoint{0.987759in}{3.434550in}}%
\pgfpathlineto{\pgfqpoint{0.991039in}{3.432928in}}%
\pgfpathlineto{\pgfqpoint{0.993909in}{3.433636in}}%
\pgfpathlineto{\pgfqpoint{0.997189in}{3.436775in}}%
\pgfpathlineto{\pgfqpoint{1.001699in}{3.444660in}}%
\pgfpathlineto{\pgfqpoint{1.008669in}{3.462325in}}%
\pgfpathlineto{\pgfqpoint{1.018919in}{3.487860in}}%
\pgfpathlineto{\pgfqpoint{1.023019in}{3.493179in}}%
\pgfpathlineto{\pgfqpoint{1.025889in}{3.493987in}}%
\pgfpathlineto{\pgfqpoint{1.028349in}{3.492403in}}%
\pgfpathlineto{\pgfqpoint{1.031219in}{3.487612in}}%
\pgfpathlineto{\pgfqpoint{1.034909in}{3.476484in}}%
\pgfpathlineto{\pgfqpoint{1.039419in}{3.455166in}}%
\pgfpathlineto{\pgfqpoint{1.045569in}{3.413629in}}%
\pgfpathlineto{\pgfqpoint{1.055819in}{3.325100in}}%
\pgfpathlineto{\pgfqpoint{1.065249in}{3.250184in}}%
\pgfpathlineto{\pgfqpoint{1.070579in}{3.223460in}}%
\pgfpathlineto{\pgfqpoint{1.074269in}{3.214734in}}%
\pgfpathlineto{\pgfqpoint{1.076319in}{3.213686in}}%
\pgfpathlineto{\pgfqpoint{1.078369in}{3.215370in}}%
\pgfpathlineto{\pgfqpoint{1.080829in}{3.220860in}}%
\pgfpathlineto{\pgfqpoint{1.084519in}{3.235372in}}%
\pgfpathlineto{\pgfqpoint{1.091079in}{3.272739in}}%
\pgfpathlineto{\pgfqpoint{1.097639in}{3.306248in}}%
\pgfpathlineto{\pgfqpoint{1.100919in}{3.313319in}}%
\pgfpathlineto{\pgfqpoint{1.102559in}{3.313197in}}%
\pgfpathlineto{\pgfqpoint{1.104199in}{3.310416in}}%
\pgfpathlineto{\pgfqpoint{1.106659in}{3.301423in}}%
\pgfpathlineto{\pgfqpoint{1.111579in}{3.271926in}}%
\pgfpathlineto{\pgfqpoint{1.116089in}{3.250381in}}%
\pgfpathlineto{\pgfqpoint{1.117729in}{3.249300in}}%
\pgfpathlineto{\pgfqpoint{1.118959in}{3.251958in}}%
\pgfpathlineto{\pgfqpoint{1.121009in}{3.263682in}}%
\pgfpathlineto{\pgfqpoint{1.124289in}{3.300383in}}%
\pgfpathlineto{\pgfqpoint{1.133719in}{3.422179in}}%
\pgfpathlineto{\pgfqpoint{1.135359in}{3.426318in}}%
\pgfpathlineto{\pgfqpoint{1.135769in}{3.426144in}}%
\pgfpathlineto{\pgfqpoint{1.136999in}{3.422804in}}%
\pgfpathlineto{\pgfqpoint{1.139049in}{3.408790in}}%
\pgfpathlineto{\pgfqpoint{1.143559in}{3.355991in}}%
\pgfpathlineto{\pgfqpoint{1.148889in}{3.302476in}}%
\pgfpathlineto{\pgfqpoint{1.151759in}{3.292557in}}%
\pgfpathlineto{\pgfqpoint{1.152989in}{3.292670in}}%
\pgfpathlineto{\pgfqpoint{1.154629in}{3.296218in}}%
\pgfpathlineto{\pgfqpoint{1.157909in}{3.310925in}}%
\pgfpathlineto{\pgfqpoint{1.163649in}{3.335746in}}%
\pgfpathlineto{\pgfqpoint{1.165699in}{3.337374in}}%
\pgfpathlineto{\pgfqpoint{1.167339in}{3.334541in}}%
\pgfpathlineto{\pgfqpoint{1.169799in}{3.322996in}}%
\pgfpathlineto{\pgfqpoint{1.173489in}{3.290707in}}%
\pgfpathlineto{\pgfqpoint{1.187839in}{3.143968in}}%
\pgfpathlineto{\pgfqpoint{1.189889in}{3.140514in}}%
\pgfpathlineto{\pgfqpoint{1.191119in}{3.141547in}}%
\pgfpathlineto{\pgfqpoint{1.193169in}{3.147981in}}%
\pgfpathlineto{\pgfqpoint{1.196859in}{3.170348in}}%
\pgfpathlineto{\pgfqpoint{1.203009in}{3.206573in}}%
\pgfpathlineto{\pgfqpoint{1.204649in}{3.208699in}}%
\pgfpathlineto{\pgfqpoint{1.205879in}{3.207111in}}%
\pgfpathlineto{\pgfqpoint{1.207929in}{3.198585in}}%
\pgfpathlineto{\pgfqpoint{1.214489in}{3.161719in}}%
\pgfpathlineto{\pgfqpoint{1.214899in}{3.161869in}}%
\pgfpathlineto{\pgfqpoint{1.216129in}{3.165781in}}%
\pgfpathlineto{\pgfqpoint{1.218179in}{3.184239in}}%
\pgfpathlineto{\pgfqpoint{1.226789in}{3.292124in}}%
\pgfpathlineto{\pgfqpoint{1.227199in}{3.291958in}}%
\pgfpathlineto{\pgfqpoint{1.228429in}{3.287289in}}%
\pgfpathlineto{\pgfqpoint{1.230889in}{3.263425in}}%
\pgfpathlineto{\pgfqpoint{1.236629in}{3.203557in}}%
\pgfpathlineto{\pgfqpoint{1.238679in}{3.199286in}}%
\pgfpathlineto{\pgfqpoint{1.239909in}{3.201495in}}%
\pgfpathlineto{\pgfqpoint{1.242369in}{3.212803in}}%
\pgfpathlineto{\pgfqpoint{1.246879in}{3.233038in}}%
\pgfpathlineto{\pgfqpoint{1.248109in}{3.233772in}}%
\pgfpathlineto{\pgfqpoint{1.249339in}{3.231447in}}%
\pgfpathlineto{\pgfqpoint{1.251389in}{3.220529in}}%
\pgfpathlineto{\pgfqpoint{1.254669in}{3.187501in}}%
\pgfpathlineto{\pgfqpoint{1.263279in}{3.094121in}}%
\pgfpathlineto{\pgfqpoint{1.265329in}{3.088384in}}%
\pgfpathlineto{\pgfqpoint{1.266559in}{3.089211in}}%
\pgfpathlineto{\pgfqpoint{1.268199in}{3.094747in}}%
\pgfpathlineto{\pgfqpoint{1.271889in}{3.118392in}}%
\pgfpathlineto{\pgfqpoint{1.275579in}{3.137128in}}%
\pgfpathlineto{\pgfqpoint{1.276809in}{3.137965in}}%
\pgfpathlineto{\pgfqpoint{1.278039in}{3.135316in}}%
\pgfpathlineto{\pgfqpoint{1.280499in}{3.121389in}}%
\pgfpathlineto{\pgfqpoint{1.284189in}{3.102238in}}%
\pgfpathlineto{\pgfqpoint{1.285009in}{3.103059in}}%
\pgfpathlineto{\pgfqpoint{1.286239in}{3.109667in}}%
\pgfpathlineto{\pgfqpoint{1.288699in}{3.140465in}}%
\pgfpathlineto{\pgfqpoint{1.293209in}{3.199935in}}%
\pgfpathlineto{\pgfqpoint{1.294439in}{3.202540in}}%
\pgfpathlineto{\pgfqpoint{1.295669in}{3.197702in}}%
\pgfpathlineto{\pgfqpoint{1.298129in}{3.173045in}}%
\pgfpathlineto{\pgfqpoint{1.302229in}{3.136501in}}%
\pgfpathlineto{\pgfqpoint{1.303459in}{3.134444in}}%
\pgfpathlineto{\pgfqpoint{1.303869in}{3.134738in}}%
\pgfpathlineto{\pgfqpoint{1.305509in}{3.139784in}}%
\pgfpathlineto{\pgfqpoint{1.310839in}{3.161944in}}%
\pgfpathlineto{\pgfqpoint{1.311659in}{3.161416in}}%
\pgfpathlineto{\pgfqpoint{1.313299in}{3.155282in}}%
\pgfpathlineto{\pgfqpoint{1.315759in}{3.134402in}}%
\pgfpathlineto{\pgfqpoint{1.324369in}{3.050778in}}%
\pgfpathlineto{\pgfqpoint{1.325599in}{3.049777in}}%
\pgfpathlineto{\pgfqpoint{1.326829in}{3.052358in}}%
\pgfpathlineto{\pgfqpoint{1.329289in}{3.065358in}}%
\pgfpathlineto{\pgfqpoint{1.333389in}{3.087000in}}%
\pgfpathlineto{\pgfqpoint{1.334619in}{3.087538in}}%
\pgfpathlineto{\pgfqpoint{1.335849in}{3.084178in}}%
\pgfpathlineto{\pgfqpoint{1.340769in}{3.059339in}}%
\pgfpathlineto{\pgfqpoint{1.341589in}{3.060718in}}%
\pgfpathlineto{\pgfqpoint{1.343229in}{3.072938in}}%
\pgfpathlineto{\pgfqpoint{1.349379in}{3.139427in}}%
\pgfpathlineto{\pgfqpoint{1.349789in}{3.138714in}}%
\pgfpathlineto{\pgfqpoint{1.351429in}{3.128254in}}%
\pgfpathlineto{\pgfqpoint{1.356759in}{3.087276in}}%
\pgfpathlineto{\pgfqpoint{1.357579in}{3.087709in}}%
\pgfpathlineto{\pgfqpoint{1.359219in}{3.093485in}}%
\pgfpathlineto{\pgfqpoint{1.363319in}{3.109684in}}%
\pgfpathlineto{\pgfqpoint{1.364139in}{3.109168in}}%
\pgfpathlineto{\pgfqpoint{1.365369in}{3.104855in}}%
\pgfpathlineto{\pgfqpoint{1.367829in}{3.084711in}}%
\pgfpathlineto{\pgfqpoint{1.374389in}{3.022304in}}%
\pgfpathlineto{\pgfqpoint{1.375619in}{3.020139in}}%
\pgfpathlineto{\pgfqpoint{1.376029in}{3.020341in}}%
\pgfpathlineto{\pgfqpoint{1.377259in}{3.023425in}}%
\pgfpathlineto{\pgfqpoint{1.380539in}{3.041747in}}%
\pgfpathlineto{\pgfqpoint{1.382999in}{3.050324in}}%
\pgfpathlineto{\pgfqpoint{1.383819in}{3.049868in}}%
\pgfpathlineto{\pgfqpoint{1.385459in}{3.043691in}}%
\pgfpathlineto{\pgfqpoint{1.389149in}{3.027187in}}%
\pgfpathlineto{\pgfqpoint{1.389969in}{3.029297in}}%
\pgfpathlineto{\pgfqpoint{1.391609in}{3.043275in}}%
\pgfpathlineto{\pgfqpoint{1.396529in}{3.092840in}}%
\pgfpathlineto{\pgfqpoint{1.397349in}{3.090916in}}%
\pgfpathlineto{\pgfqpoint{1.399399in}{3.074755in}}%
\pgfpathlineto{\pgfqpoint{1.402679in}{3.051988in}}%
\pgfpathlineto{\pgfqpoint{1.403499in}{3.051519in}}%
\pgfpathlineto{\pgfqpoint{1.403909in}{3.052091in}}%
\pgfpathlineto{\pgfqpoint{1.405959in}{3.060230in}}%
\pgfpathlineto{\pgfqpoint{1.408829in}{3.070256in}}%
\pgfpathlineto{\pgfqpoint{1.409649in}{3.069727in}}%
\pgfpathlineto{\pgfqpoint{1.410879in}{3.065142in}}%
\pgfpathlineto{\pgfqpoint{1.413339in}{3.044316in}}%
\pgfpathlineto{\pgfqpoint{1.418669in}{2.998305in}}%
\pgfpathlineto{\pgfqpoint{1.419899in}{2.997010in}}%
\pgfpathlineto{\pgfqpoint{1.421129in}{2.999853in}}%
\pgfpathlineto{\pgfqpoint{1.426459in}{3.021543in}}%
\pgfpathlineto{\pgfqpoint{1.426869in}{3.021099in}}%
\pgfpathlineto{\pgfqpoint{1.428509in}{3.014930in}}%
\pgfpathlineto{\pgfqpoint{1.431379in}{3.002203in}}%
\pgfpathlineto{\pgfqpoint{1.431789in}{3.002349in}}%
\pgfpathlineto{\pgfqpoint{1.433019in}{3.007870in}}%
\pgfpathlineto{\pgfqpoint{1.435889in}{3.041415in}}%
\pgfpathlineto{\pgfqpoint{1.438348in}{3.057192in}}%
\pgfpathlineto{\pgfqpoint{1.439168in}{3.055042in}}%
\pgfpathlineto{\pgfqpoint{1.441628in}{3.035821in}}%
\pgfpathlineto{\pgfqpoint{1.444088in}{3.023597in}}%
\pgfpathlineto{\pgfqpoint{1.444908in}{3.024112in}}%
\pgfpathlineto{\pgfqpoint{1.446958in}{3.032028in}}%
\pgfpathlineto{\pgfqpoint{1.449418in}{3.039633in}}%
\pgfpathlineto{\pgfqpoint{1.450238in}{3.038830in}}%
\pgfpathlineto{\pgfqpoint{1.451878in}{3.030937in}}%
\pgfpathlineto{\pgfqpoint{1.458848in}{2.978261in}}%
\pgfpathlineto{\pgfqpoint{1.459668in}{2.978873in}}%
\pgfpathlineto{\pgfqpoint{1.461308in}{2.984902in}}%
\pgfpathlineto{\pgfqpoint{1.464998in}{2.998781in}}%
\pgfpathlineto{\pgfqpoint{1.465818in}{2.997691in}}%
\pgfpathlineto{\pgfqpoint{1.467868in}{2.988384in}}%
\pgfpathlineto{\pgfqpoint{1.469918in}{2.982341in}}%
\pgfpathlineto{\pgfqpoint{1.470738in}{2.984866in}}%
\pgfpathlineto{\pgfqpoint{1.472788in}{3.004435in}}%
\pgfpathlineto{\pgfqpoint{1.475658in}{3.028977in}}%
\pgfpathlineto{\pgfqpoint{1.476478in}{3.028351in}}%
\pgfpathlineto{\pgfqpoint{1.478118in}{3.018295in}}%
\pgfpathlineto{\pgfqpoint{1.480988in}{3.001501in}}%
\pgfpathlineto{\pgfqpoint{1.481808in}{3.001495in}}%
\pgfpathlineto{\pgfqpoint{1.483448in}{3.006863in}}%
\pgfpathlineto{\pgfqpoint{1.486318in}{3.015163in}}%
\pgfpathlineto{\pgfqpoint{1.487548in}{3.012112in}}%
\pgfpathlineto{\pgfqpoint{1.489598in}{2.997317in}}%
\pgfpathlineto{\pgfqpoint{1.494108in}{2.963601in}}%
\pgfpathlineto{\pgfqpoint{1.494928in}{2.962929in}}%
\pgfpathlineto{\pgfqpoint{1.495338in}{2.963368in}}%
\pgfpathlineto{\pgfqpoint{1.496978in}{2.969068in}}%
\pgfpathlineto{\pgfqpoint{1.500258in}{2.980323in}}%
\pgfpathlineto{\pgfqpoint{1.501078in}{2.979146in}}%
\pgfpathlineto{\pgfqpoint{1.504768in}{2.966110in}}%
\pgfpathlineto{\pgfqpoint{1.505178in}{2.966913in}}%
\pgfpathlineto{\pgfqpoint{1.506408in}{2.974529in}}%
\pgfpathlineto{\pgfqpoint{1.510508in}{3.006585in}}%
\pgfpathlineto{\pgfqpoint{1.511738in}{3.001734in}}%
\pgfpathlineto{\pgfqpoint{1.515428in}{2.983013in}}%
\pgfpathlineto{\pgfqpoint{1.516658in}{2.985593in}}%
\pgfpathlineto{\pgfqpoint{1.519938in}{2.995257in}}%
\pgfpathlineto{\pgfqpoint{1.521168in}{2.992080in}}%
\pgfpathlineto{\pgfqpoint{1.523628in}{2.973494in}}%
\pgfpathlineto{\pgfqpoint{1.526908in}{2.951066in}}%
\pgfpathlineto{\pgfqpoint{1.527728in}{2.950113in}}%
\pgfpathlineto{\pgfqpoint{1.528138in}{2.950441in}}%
\pgfpathlineto{\pgfqpoint{1.529778in}{2.955798in}}%
\pgfpathlineto{\pgfqpoint{1.532648in}{2.965113in}}%
\pgfpathlineto{\pgfqpoint{1.533468in}{2.964091in}}%
\pgfpathlineto{\pgfqpoint{1.536748in}{2.952585in}}%
\pgfpathlineto{\pgfqpoint{1.537568in}{2.954348in}}%
\pgfpathlineto{\pgfqpoint{1.539208in}{2.967461in}}%
\pgfpathlineto{\pgfqpoint{1.542078in}{2.988191in}}%
\pgfpathlineto{\pgfqpoint{1.542898in}{2.986550in}}%
\pgfpathlineto{\pgfqpoint{1.546998in}{2.967999in}}%
\pgfpathlineto{\pgfqpoint{1.547408in}{2.968606in}}%
\pgfpathlineto{\pgfqpoint{1.551098in}{2.978752in}}%
\pgfpathlineto{\pgfqpoint{1.551508in}{2.978181in}}%
\pgfpathlineto{\pgfqpoint{1.553148in}{2.970357in}}%
\pgfpathlineto{\pgfqpoint{1.558068in}{2.939327in}}%
\pgfpathlineto{\pgfqpoint{1.558888in}{2.940030in}}%
\pgfpathlineto{\pgfqpoint{1.561348in}{2.949336in}}%
\pgfpathlineto{\pgfqpoint{1.562988in}{2.952324in}}%
\pgfpathlineto{\pgfqpoint{1.564218in}{2.949430in}}%
\pgfpathlineto{\pgfqpoint{1.566678in}{2.941277in}}%
\pgfpathlineto{\pgfqpoint{1.567088in}{2.941777in}}%
\pgfpathlineto{\pgfqpoint{1.568318in}{2.948505in}}%
\pgfpathlineto{\pgfqpoint{1.571598in}{2.972781in}}%
\pgfpathlineto{\pgfqpoint{1.572008in}{2.972546in}}%
\pgfpathlineto{\pgfqpoint{1.573238in}{2.967226in}}%
\pgfpathlineto{\pgfqpoint{1.576108in}{2.955196in}}%
\pgfpathlineto{\pgfqpoint{1.577338in}{2.957794in}}%
\pgfpathlineto{\pgfqpoint{1.579798in}{2.964959in}}%
\pgfpathlineto{\pgfqpoint{1.580208in}{2.964831in}}%
\pgfpathlineto{\pgfqpoint{1.581438in}{2.960952in}}%
\pgfpathlineto{\pgfqpoint{1.586768in}{2.930054in}}%
\pgfpathlineto{\pgfqpoint{1.587588in}{2.931193in}}%
\pgfpathlineto{\pgfqpoint{1.591278in}{2.941459in}}%
\pgfpathlineto{\pgfqpoint{1.591688in}{2.940868in}}%
\pgfpathlineto{\pgfqpoint{1.594558in}{2.931650in}}%
\pgfpathlineto{\pgfqpoint{1.595378in}{2.933062in}}%
\pgfpathlineto{\pgfqpoint{1.597018in}{2.945229in}}%
\pgfpathlineto{\pgfqpoint{1.599478in}{2.959811in}}%
\pgfpathlineto{\pgfqpoint{1.600298in}{2.957794in}}%
\pgfpathlineto{\pgfqpoint{1.603578in}{2.944368in}}%
\pgfpathlineto{\pgfqpoint{1.603988in}{2.944806in}}%
\pgfpathlineto{\pgfqpoint{1.607268in}{2.953179in}}%
\pgfpathlineto{\pgfqpoint{1.608088in}{2.951636in}}%
\pgfpathlineto{\pgfqpoint{1.610138in}{2.939092in}}%
\pgfpathlineto{\pgfqpoint{1.613418in}{2.922076in}}%
\pgfpathlineto{\pgfqpoint{1.614238in}{2.922839in}}%
\pgfpathlineto{\pgfqpoint{1.617518in}{2.932275in}}%
\pgfpathlineto{\pgfqpoint{1.618338in}{2.931340in}}%
\pgfpathlineto{\pgfqpoint{1.621208in}{2.923458in}}%
\pgfpathlineto{\pgfqpoint{1.621618in}{2.924370in}}%
\pgfpathlineto{\pgfqpoint{1.623258in}{2.935702in}}%
\pgfpathlineto{\pgfqpoint{1.625718in}{2.948617in}}%
\pgfpathlineto{\pgfqpoint{1.626948in}{2.944033in}}%
\pgfpathlineto{\pgfqpoint{1.629408in}{2.935023in}}%
\pgfpathlineto{\pgfqpoint{1.630638in}{2.937539in}}%
\pgfpathlineto{\pgfqpoint{1.633098in}{2.942952in}}%
\pgfpathlineto{\pgfqpoint{1.634328in}{2.939098in}}%
\pgfpathlineto{\pgfqpoint{1.638838in}{2.915080in}}%
\pgfpathlineto{\pgfqpoint{1.639248in}{2.915346in}}%
\pgfpathlineto{\pgfqpoint{1.640888in}{2.920192in}}%
\pgfpathlineto{\pgfqpoint{1.642938in}{2.924101in}}%
\pgfpathlineto{\pgfqpoint{1.644168in}{2.920895in}}%
\pgfpathlineto{\pgfqpoint{1.645808in}{2.916131in}}%
\pgfpathlineto{\pgfqpoint{1.646218in}{2.916373in}}%
\pgfpathlineto{\pgfqpoint{1.647448in}{2.922286in}}%
\pgfpathlineto{\pgfqpoint{1.650318in}{2.939118in}}%
\pgfpathlineto{\pgfqpoint{1.651548in}{2.935069in}}%
\pgfpathlineto{\pgfqpoint{1.654008in}{2.926955in}}%
\pgfpathlineto{\pgfqpoint{1.655238in}{2.929770in}}%
\pgfpathlineto{\pgfqpoint{1.657288in}{2.934258in}}%
\pgfpathlineto{\pgfqpoint{1.658108in}{2.932644in}}%
\pgfpathlineto{\pgfqpoint{1.660158in}{2.920401in}}%
\pgfpathlineto{\pgfqpoint{1.662618in}{2.909036in}}%
\pgfpathlineto{\pgfqpoint{1.663438in}{2.909510in}}%
\pgfpathlineto{\pgfqpoint{1.666718in}{2.917081in}}%
\pgfpathlineto{\pgfqpoint{1.667128in}{2.916455in}}%
\pgfpathlineto{\pgfqpoint{1.669588in}{2.909784in}}%
\pgfpathlineto{\pgfqpoint{1.669998in}{2.910312in}}%
\pgfpathlineto{\pgfqpoint{1.671228in}{2.916944in}}%
\pgfpathlineto{\pgfqpoint{1.673688in}{2.930802in}}%
\pgfpathlineto{\pgfqpoint{1.674508in}{2.929034in}}%
\pgfpathlineto{\pgfqpoint{1.676968in}{2.919823in}}%
\pgfpathlineto{\pgfqpoint{1.677378in}{2.919923in}}%
\pgfpathlineto{\pgfqpoint{1.679018in}{2.924300in}}%
\pgfpathlineto{\pgfqpoint{1.680248in}{2.926594in}}%
\pgfpathlineto{\pgfqpoint{1.680658in}{2.926331in}}%
\pgfpathlineto{\pgfqpoint{1.681888in}{2.921951in}}%
\pgfpathlineto{\pgfqpoint{1.685578in}{2.903533in}}%
\pgfpathlineto{\pgfqpoint{1.685988in}{2.903675in}}%
\pgfpathlineto{\pgfqpoint{1.687628in}{2.908085in}}%
\pgfpathlineto{\pgfqpoint{1.689268in}{2.910958in}}%
\pgfpathlineto{\pgfqpoint{1.690498in}{2.908140in}}%
\pgfpathlineto{\pgfqpoint{1.692138in}{2.904200in}}%
\pgfpathlineto{\pgfqpoint{1.692548in}{2.904802in}}%
\pgfpathlineto{\pgfqpoint{1.694188in}{2.914767in}}%
\pgfpathlineto{\pgfqpoint{1.696238in}{2.923343in}}%
\pgfpathlineto{\pgfqpoint{1.697878in}{2.917036in}}%
\pgfpathlineto{\pgfqpoint{1.699518in}{2.913609in}}%
\pgfpathlineto{\pgfqpoint{1.701158in}{2.917809in}}%
\pgfpathlineto{\pgfqpoint{1.702388in}{2.919803in}}%
\pgfpathlineto{\pgfqpoint{1.702798in}{2.919392in}}%
\pgfpathlineto{\pgfqpoint{1.704028in}{2.914615in}}%
\pgfpathlineto{\pgfqpoint{1.707718in}{2.898783in}}%
\pgfpathlineto{\pgfqpoint{1.709358in}{2.902960in}}%
\pgfpathlineto{\pgfqpoint{1.710998in}{2.905383in}}%
\pgfpathlineto{\pgfqpoint{1.712638in}{2.900916in}}%
\pgfpathlineto{\pgfqpoint{1.713458in}{2.899236in}}%
\pgfpathlineto{\pgfqpoint{1.713868in}{2.899450in}}%
\pgfpathlineto{\pgfqpoint{1.715098in}{2.905196in}}%
\pgfpathlineto{\pgfqpoint{1.717558in}{2.916897in}}%
\pgfpathlineto{\pgfqpoint{1.718788in}{2.912716in}}%
\pgfpathlineto{\pgfqpoint{1.720428in}{2.907932in}}%
\pgfpathlineto{\pgfqpoint{1.720838in}{2.908160in}}%
\pgfpathlineto{\pgfqpoint{1.723708in}{2.913608in}}%
\pgfpathlineto{\pgfqpoint{1.724118in}{2.912815in}}%
\pgfpathlineto{\pgfqpoint{1.726168in}{2.901997in}}%
\pgfpathlineto{\pgfqpoint{1.728218in}{2.894351in}}%
\pgfpathlineto{\pgfqpoint{1.729038in}{2.895179in}}%
\pgfpathlineto{\pgfqpoint{1.731498in}{2.900594in}}%
\pgfpathlineto{\pgfqpoint{1.731908in}{2.900207in}}%
\pgfpathlineto{\pgfqpoint{1.734368in}{2.894885in}}%
\pgfpathlineto{\pgfqpoint{1.734778in}{2.895810in}}%
\pgfpathlineto{\pgfqpoint{1.737648in}{2.911180in}}%
\pgfpathlineto{\pgfqpoint{1.738468in}{2.910107in}}%
\pgfpathlineto{\pgfqpoint{1.740928in}{2.902977in}}%
\pgfpathlineto{\pgfqpoint{1.741338in}{2.903475in}}%
\pgfpathlineto{\pgfqpoint{1.743798in}{2.908331in}}%
\pgfpathlineto{\pgfqpoint{1.744208in}{2.907693in}}%
\pgfpathlineto{\pgfqpoint{1.745848in}{2.899928in}}%
\pgfpathlineto{\pgfqpoint{1.748308in}{2.890429in}}%
\pgfpathlineto{\pgfqpoint{1.749538in}{2.892611in}}%
\pgfpathlineto{\pgfqpoint{1.751178in}{2.896190in}}%
\pgfpathlineto{\pgfqpoint{1.751588in}{2.896031in}}%
\pgfpathlineto{\pgfqpoint{1.754048in}{2.890817in}}%
\pgfpathlineto{\pgfqpoint{1.754458in}{2.891592in}}%
\pgfpathlineto{\pgfqpoint{1.757328in}{2.906072in}}%
\pgfpathlineto{\pgfqpoint{1.758148in}{2.904758in}}%
\pgfpathlineto{\pgfqpoint{1.760198in}{2.898467in}}%
\pgfpathlineto{\pgfqpoint{1.760608in}{2.898655in}}%
\pgfpathlineto{\pgfqpoint{1.763068in}{2.903570in}}%
\pgfpathlineto{\pgfqpoint{1.763478in}{2.903119in}}%
\pgfpathlineto{\pgfqpoint{1.765118in}{2.895888in}}%
\pgfpathlineto{\pgfqpoint{1.767578in}{2.886905in}}%
\pgfpathlineto{\pgfqpoint{1.768808in}{2.889257in}}%
\pgfpathlineto{\pgfqpoint{1.770448in}{2.892216in}}%
\pgfpathlineto{\pgfqpoint{1.770858in}{2.891797in}}%
\pgfpathlineto{\pgfqpoint{1.772908in}{2.887144in}}%
\pgfpathlineto{\pgfqpoint{1.773318in}{2.887646in}}%
\pgfpathlineto{\pgfqpoint{1.774958in}{2.896681in}}%
\pgfpathlineto{\pgfqpoint{1.776188in}{2.901438in}}%
\pgfpathlineto{\pgfqpoint{1.776598in}{2.901186in}}%
\pgfpathlineto{\pgfqpoint{1.779058in}{2.894410in}}%
\pgfpathlineto{\pgfqpoint{1.779878in}{2.895595in}}%
\pgfpathlineto{\pgfqpoint{1.781928in}{2.899071in}}%
\pgfpathlineto{\pgfqpoint{1.783158in}{2.894997in}}%
\pgfpathlineto{\pgfqpoint{1.786028in}{2.883683in}}%
\pgfpathlineto{\pgfqpoint{1.787258in}{2.885934in}}%
\pgfpathlineto{\pgfqpoint{1.788898in}{2.888580in}}%
\pgfpathlineto{\pgfqpoint{1.790538in}{2.884700in}}%
\pgfpathlineto{\pgfqpoint{1.791358in}{2.883960in}}%
\pgfpathlineto{\pgfqpoint{1.792588in}{2.889005in}}%
\pgfpathlineto{\pgfqpoint{1.794638in}{2.897176in}}%
\pgfpathlineto{\pgfqpoint{1.797098in}{2.890738in}}%
\pgfpathlineto{\pgfqpoint{1.797918in}{2.891864in}}%
\pgfpathlineto{\pgfqpoint{1.799558in}{2.895278in}}%
\pgfpathlineto{\pgfqpoint{1.799968in}{2.895048in}}%
\pgfpathlineto{\pgfqpoint{1.801198in}{2.890724in}}%
\pgfpathlineto{\pgfqpoint{1.803658in}{2.880777in}}%
\pgfpathlineto{\pgfqpoint{1.804068in}{2.880878in}}%
\pgfpathlineto{\pgfqpoint{1.806528in}{2.885350in}}%
\pgfpathlineto{\pgfqpoint{1.807348in}{2.884026in}}%
\pgfpathlineto{\pgfqpoint{1.808988in}{2.880968in}}%
\pgfpathlineto{\pgfqpoint{1.810218in}{2.885977in}}%
\pgfpathlineto{\pgfqpoint{1.811858in}{2.893440in}}%
\pgfpathlineto{\pgfqpoint{1.812268in}{2.893246in}}%
\pgfpathlineto{\pgfqpoint{1.814728in}{2.887475in}}%
\pgfpathlineto{\pgfqpoint{1.815138in}{2.888055in}}%
\pgfpathlineto{\pgfqpoint{1.817188in}{2.891603in}}%
\pgfpathlineto{\pgfqpoint{1.818418in}{2.887789in}}%
\pgfpathlineto{\pgfqpoint{1.820878in}{2.878084in}}%
\pgfpathlineto{\pgfqpoint{1.821288in}{2.878212in}}%
\pgfpathlineto{\pgfqpoint{1.823748in}{2.882302in}}%
\pgfpathlineto{\pgfqpoint{1.824158in}{2.881731in}}%
\pgfpathlineto{\pgfqpoint{1.825798in}{2.878140in}}%
\pgfpathlineto{\pgfqpoint{1.826208in}{2.878526in}}%
\pgfpathlineto{\pgfqpoint{1.827848in}{2.886852in}}%
\pgfpathlineto{\pgfqpoint{1.829078in}{2.889940in}}%
\pgfpathlineto{\pgfqpoint{1.831538in}{2.884399in}}%
\pgfpathlineto{\pgfqpoint{1.832358in}{2.885883in}}%
\pgfpathlineto{\pgfqpoint{1.833998in}{2.888287in}}%
\pgfpathlineto{\pgfqpoint{1.835228in}{2.884298in}}%
\pgfpathlineto{\pgfqpoint{1.837688in}{2.875601in}}%
\pgfpathlineto{\pgfqpoint{1.838918in}{2.877815in}}%
\pgfpathlineto{\pgfqpoint{1.840148in}{2.879637in}}%
\pgfpathlineto{\pgfqpoint{1.840558in}{2.879248in}}%
\pgfpathlineto{\pgfqpoint{1.842198in}{2.875655in}}%
\pgfpathlineto{\pgfqpoint{1.842608in}{2.875886in}}%
\pgfpathlineto{\pgfqpoint{1.843838in}{2.881409in}}%
\pgfpathlineto{\pgfqpoint{1.845478in}{2.886759in}}%
\pgfpathlineto{\pgfqpoint{1.847938in}{2.881686in}}%
\pgfpathlineto{\pgfqpoint{1.848348in}{2.882352in}}%
\pgfpathlineto{\pgfqpoint{1.849988in}{2.885417in}}%
\pgfpathlineto{\pgfqpoint{1.850398in}{2.885043in}}%
\pgfpathlineto{\pgfqpoint{1.852038in}{2.878304in}}%
\pgfpathlineto{\pgfqpoint{1.853678in}{2.873324in}}%
\pgfpathlineto{\pgfqpoint{1.854088in}{2.873556in}}%
\pgfpathlineto{\pgfqpoint{1.856138in}{2.877128in}}%
\pgfpathlineto{\pgfqpoint{1.856548in}{2.876771in}}%
\pgfpathlineto{\pgfqpoint{1.858188in}{2.873332in}}%
\pgfpathlineto{\pgfqpoint{1.858598in}{2.873654in}}%
\pgfpathlineto{\pgfqpoint{1.861058in}{2.883937in}}%
\pgfpathlineto{\pgfqpoint{1.862288in}{2.881323in}}%
\pgfpathlineto{\pgfqpoint{1.863518in}{2.878979in}}%
\pgfpathlineto{\pgfqpoint{1.863928in}{2.879390in}}%
\pgfpathlineto{\pgfqpoint{1.865978in}{2.882486in}}%
\pgfpathlineto{\pgfqpoint{1.867208in}{2.878354in}}%
\pgfpathlineto{\pgfqpoint{1.869258in}{2.871228in}}%
\pgfpathlineto{\pgfqpoint{1.869668in}{2.871436in}}%
\pgfpathlineto{\pgfqpoint{1.871718in}{2.874795in}}%
\pgfpathlineto{\pgfqpoint{1.872128in}{2.874356in}}%
\pgfpathlineto{\pgfqpoint{1.873768in}{2.871203in}}%
\pgfpathlineto{\pgfqpoint{1.874178in}{2.871798in}}%
\pgfpathlineto{\pgfqpoint{1.876638in}{2.881262in}}%
\pgfpathlineto{\pgfqpoint{1.877048in}{2.880602in}}%
\pgfpathlineto{\pgfqpoint{1.878688in}{2.876603in}}%
\pgfpathlineto{\pgfqpoint{1.879098in}{2.876820in}}%
\pgfpathlineto{\pgfqpoint{1.881148in}{2.880021in}}%
\pgfpathlineto{\pgfqpoint{1.881558in}{2.879310in}}%
\pgfpathlineto{\pgfqpoint{1.884428in}{2.869278in}}%
\pgfpathlineto{\pgfqpoint{1.885248in}{2.870273in}}%
\pgfpathlineto{\pgfqpoint{1.886888in}{2.872590in}}%
\pgfpathlineto{\pgfqpoint{1.888938in}{2.869365in}}%
\pgfpathlineto{\pgfqpoint{1.889348in}{2.870351in}}%
\pgfpathlineto{\pgfqpoint{1.891398in}{2.878828in}}%
\pgfpathlineto{\pgfqpoint{1.891808in}{2.878497in}}%
\pgfpathlineto{\pgfqpoint{1.893858in}{2.874521in}}%
\pgfpathlineto{\pgfqpoint{1.894268in}{2.875161in}}%
\pgfpathlineto{\pgfqpoint{1.895908in}{2.877703in}}%
\pgfpathlineto{\pgfqpoint{1.897138in}{2.873985in}}%
\pgfpathlineto{\pgfqpoint{1.899188in}{2.867476in}}%
\pgfpathlineto{\pgfqpoint{1.901648in}{2.870468in}}%
\pgfpathlineto{\pgfqpoint{1.902058in}{2.869722in}}%
\pgfpathlineto{\pgfqpoint{1.903288in}{2.867395in}}%
\pgfpathlineto{\pgfqpoint{1.903698in}{2.867883in}}%
\pgfpathlineto{\pgfqpoint{1.906158in}{2.876442in}}%
\pgfpathlineto{\pgfqpoint{1.906568in}{2.875615in}}%
\pgfpathlineto{\pgfqpoint{1.908208in}{2.872422in}}%
\pgfpathlineto{\pgfqpoint{1.910258in}{2.875546in}}%
\pgfpathlineto{\pgfqpoint{1.910668in}{2.874931in}}%
\pgfpathlineto{\pgfqpoint{1.913538in}{2.865819in}}%
\pgfpathlineto{\pgfqpoint{1.914358in}{2.867097in}}%
\pgfpathlineto{\pgfqpoint{1.915588in}{2.868836in}}%
\pgfpathlineto{\pgfqpoint{1.915998in}{2.868426in}}%
\pgfpathlineto{\pgfqpoint{1.917638in}{2.865803in}}%
\pgfpathlineto{\pgfqpoint{1.918868in}{2.870690in}}%
\pgfpathlineto{\pgfqpoint{1.920098in}{2.874447in}}%
\pgfpathlineto{\pgfqpoint{1.920508in}{2.873890in}}%
\pgfpathlineto{\pgfqpoint{1.922148in}{2.870488in}}%
\pgfpathlineto{\pgfqpoint{1.922558in}{2.870913in}}%
\pgfpathlineto{\pgfqpoint{1.924198in}{2.873549in}}%
\pgfpathlineto{\pgfqpoint{1.924608in}{2.872997in}}%
\pgfpathlineto{\pgfqpoint{1.927478in}{2.864282in}}%
\pgfpathlineto{\pgfqpoint{1.927888in}{2.864798in}}%
\pgfpathlineto{\pgfqpoint{1.929528in}{2.867065in}}%
\pgfpathlineto{\pgfqpoint{1.929938in}{2.866502in}}%
\pgfpathlineto{\pgfqpoint{1.931168in}{2.864120in}}%
\pgfpathlineto{\pgfqpoint{1.931578in}{2.864478in}}%
\pgfpathlineto{\pgfqpoint{1.933628in}{2.872452in}}%
\pgfpathlineto{\pgfqpoint{1.934448in}{2.871346in}}%
\pgfpathlineto{\pgfqpoint{1.935678in}{2.868722in}}%
\pgfpathlineto{\pgfqpoint{1.936088in}{2.868968in}}%
\pgfpathlineto{\pgfqpoint{1.937728in}{2.871695in}}%
\pgfpathlineto{\pgfqpoint{1.938138in}{2.871268in}}%
\pgfpathlineto{\pgfqpoint{1.941008in}{2.862828in}}%
\pgfpathlineto{\pgfqpoint{1.941828in}{2.864207in}}%
\pgfpathlineto{\pgfqpoint{1.943058in}{2.865407in}}%
\pgfpathlineto{\pgfqpoint{1.944698in}{2.862649in}}%
\pgfpathlineto{\pgfqpoint{1.945108in}{2.863285in}}%
\pgfpathlineto{\pgfqpoint{1.947158in}{2.870668in}}%
\pgfpathlineto{\pgfqpoint{1.947568in}{2.870115in}}%
\pgfpathlineto{\pgfqpoint{1.949208in}{2.867145in}}%
\pgfpathlineto{\pgfqpoint{1.949618in}{2.867718in}}%
\pgfpathlineto{\pgfqpoint{1.950848in}{2.869931in}}%
\pgfpathlineto{\pgfqpoint{1.951258in}{2.869726in}}%
\pgfpathlineto{\pgfqpoint{1.952898in}{2.863738in}}%
\pgfpathlineto{\pgfqpoint{1.954128in}{2.861432in}}%
\pgfpathlineto{\pgfqpoint{1.954538in}{2.861914in}}%
\pgfpathlineto{\pgfqpoint{1.956178in}{2.863905in}}%
\pgfpathlineto{\pgfqpoint{1.957818in}{2.861295in}}%
\pgfpathlineto{\pgfqpoint{1.958228in}{2.862062in}}%
\pgfpathlineto{\pgfqpoint{1.960278in}{2.868899in}}%
\pgfpathlineto{\pgfqpoint{1.960688in}{2.868147in}}%
\pgfpathlineto{\pgfqpoint{1.961918in}{2.865551in}}%
\pgfpathlineto{\pgfqpoint{1.962328in}{2.865780in}}%
\pgfpathlineto{\pgfqpoint{1.963968in}{2.868281in}}%
\pgfpathlineto{\pgfqpoint{1.964378in}{2.867704in}}%
\pgfpathlineto{\pgfqpoint{1.966838in}{2.860111in}}%
\pgfpathlineto{\pgfqpoint{1.967658in}{2.861191in}}%
\pgfpathlineto{\pgfqpoint{1.968888in}{2.862568in}}%
\pgfpathlineto{\pgfqpoint{1.970528in}{2.859985in}}%
\pgfpathlineto{\pgfqpoint{1.970938in}{2.860689in}}%
\pgfpathlineto{\pgfqpoint{1.972988in}{2.867283in}}%
\pgfpathlineto{\pgfqpoint{1.973398in}{2.866488in}}%
\pgfpathlineto{\pgfqpoint{1.974628in}{2.864108in}}%
\pgfpathlineto{\pgfqpoint{1.975038in}{2.864457in}}%
\pgfpathlineto{\pgfqpoint{1.976678in}{2.866657in}}%
\pgfpathlineto{\pgfqpoint{1.977908in}{2.862709in}}%
\pgfpathlineto{\pgfqpoint{1.979548in}{2.858992in}}%
\pgfpathlineto{\pgfqpoint{1.981188in}{2.861331in}}%
\pgfpathlineto{\pgfqpoint{1.981598in}{2.860971in}}%
\pgfpathlineto{\pgfqpoint{1.982828in}{2.858751in}}%
\pgfpathlineto{\pgfqpoint{1.983238in}{2.859156in}}%
\pgfpathlineto{\pgfqpoint{1.985288in}{2.865861in}}%
\pgfpathlineto{\pgfqpoint{1.985698in}{2.865182in}}%
\pgfpathlineto{\pgfqpoint{1.986928in}{2.862763in}}%
\pgfpathlineto{\pgfqpoint{1.987338in}{2.863074in}}%
\pgfpathlineto{\pgfqpoint{1.988978in}{2.865206in}}%
\pgfpathlineto{\pgfqpoint{1.990618in}{2.859556in}}%
\pgfpathlineto{\pgfqpoint{1.991438in}{2.857797in}}%
\pgfpathlineto{\pgfqpoint{1.991848in}{2.857942in}}%
\pgfpathlineto{\pgfqpoint{1.993488in}{2.860050in}}%
\pgfpathlineto{\pgfqpoint{1.993898in}{2.859504in}}%
\pgfpathlineto{\pgfqpoint{1.995128in}{2.857660in}}%
\pgfpathlineto{\pgfqpoint{1.995538in}{2.858487in}}%
\pgfpathlineto{\pgfqpoint{1.997178in}{2.864497in}}%
\pgfpathlineto{\pgfqpoint{1.997588in}{2.864184in}}%
\pgfpathlineto{\pgfqpoint{1.999228in}{2.861633in}}%
\pgfpathlineto{\pgfqpoint{2.000868in}{2.863953in}}%
\pgfpathlineto{\pgfqpoint{2.001278in}{2.863301in}}%
\pgfpathlineto{\pgfqpoint{2.003738in}{2.856841in}}%
\pgfpathlineto{\pgfqpoint{2.004148in}{2.857463in}}%
\pgfpathlineto{\pgfqpoint{2.005378in}{2.858887in}}%
\pgfpathlineto{\pgfqpoint{2.005788in}{2.858327in}}%
\pgfpathlineto{\pgfqpoint{2.007018in}{2.856631in}}%
\pgfpathlineto{\pgfqpoint{2.009068in}{2.863201in}}%
\pgfpathlineto{\pgfqpoint{2.009888in}{2.861648in}}%
\pgfpathlineto{\pgfqpoint{2.010708in}{2.860312in}}%
\pgfpathlineto{\pgfqpoint{2.011118in}{2.860621in}}%
\pgfpathlineto{\pgfqpoint{2.012348in}{2.862723in}}%
\pgfpathlineto{\pgfqpoint{2.012758in}{2.862499in}}%
\pgfpathlineto{\pgfqpoint{2.015218in}{2.855718in}}%
\pgfpathlineto{\pgfqpoint{2.016038in}{2.856930in}}%
\pgfpathlineto{\pgfqpoint{2.016858in}{2.857853in}}%
\pgfpathlineto{\pgfqpoint{2.017268in}{2.857465in}}%
\pgfpathlineto{\pgfqpoint{2.018498in}{2.855547in}}%
\pgfpathlineto{\pgfqpoint{2.018908in}{2.856274in}}%
\pgfpathlineto{\pgfqpoint{2.020548in}{2.861973in}}%
\pgfpathlineto{\pgfqpoint{2.020958in}{2.861549in}}%
\pgfpathlineto{\pgfqpoint{2.022188in}{2.859194in}}%
\pgfpathlineto{\pgfqpoint{2.022598in}{2.859498in}}%
\pgfpathlineto{\pgfqpoint{2.023828in}{2.861542in}}%
\pgfpathlineto{\pgfqpoint{2.024238in}{2.861257in}}%
\pgfpathlineto{\pgfqpoint{2.026698in}{2.854822in}}%
\pgfpathlineto{\pgfqpoint{2.027108in}{2.855372in}}%
\pgfpathlineto{\pgfqpoint{2.028338in}{2.856768in}}%
\pgfpathlineto{\pgfqpoint{2.028748in}{2.856193in}}%
\pgfpathlineto{\pgfqpoint{2.029978in}{2.854768in}}%
\pgfpathlineto{\pgfqpoint{2.032028in}{2.860703in}}%
\pgfpathlineto{\pgfqpoint{2.032438in}{2.859903in}}%
\pgfpathlineto{\pgfqpoint{2.033668in}{2.858251in}}%
\pgfpathlineto{\pgfqpoint{2.035308in}{2.860303in}}%
\pgfpathlineto{\pgfqpoint{2.035718in}{2.859479in}}%
\pgfpathlineto{\pgfqpoint{2.037768in}{2.853886in}}%
\pgfpathlineto{\pgfqpoint{2.038178in}{2.854366in}}%
\pgfpathlineto{\pgfqpoint{2.039408in}{2.855816in}}%
\pgfpathlineto{\pgfqpoint{2.039818in}{2.855271in}}%
\pgfpathlineto{\pgfqpoint{2.041048in}{2.853880in}}%
\pgfpathlineto{\pgfqpoint{2.043098in}{2.859549in}}%
\pgfpathlineto{\pgfqpoint{2.043508in}{2.858684in}}%
\pgfpathlineto{\pgfqpoint{2.044328in}{2.857164in}}%
\pgfpathlineto{\pgfqpoint{2.044738in}{2.857355in}}%
\pgfpathlineto{\pgfqpoint{2.045968in}{2.859363in}}%
\pgfpathlineto{\pgfqpoint{2.046378in}{2.859078in}}%
\pgfpathlineto{\pgfqpoint{2.048838in}{2.853185in}}%
\pgfpathlineto{\pgfqpoint{2.049248in}{2.853842in}}%
\pgfpathlineto{\pgfqpoint{2.050068in}{2.854944in}}%
\pgfpathlineto{\pgfqpoint{2.050478in}{2.854683in}}%
\pgfpathlineto{\pgfqpoint{2.051708in}{2.852848in}}%
\pgfpathlineto{\pgfqpoint{2.052118in}{2.853621in}}%
\pgfpathlineto{\pgfqpoint{2.053758in}{2.858652in}}%
\pgfpathlineto{\pgfqpoint{2.054168in}{2.857958in}}%
\pgfpathlineto{\pgfqpoint{2.055398in}{2.856319in}}%
\pgfpathlineto{\pgfqpoint{2.057038in}{2.858162in}}%
\pgfpathlineto{\pgfqpoint{2.059498in}{2.852383in}}%
\pgfpathlineto{\pgfqpoint{2.059908in}{2.853040in}}%
\pgfpathlineto{\pgfqpoint{2.060728in}{2.854082in}}%
\pgfpathlineto{\pgfqpoint{2.061138in}{2.853765in}}%
\pgfpathlineto{\pgfqpoint{2.062368in}{2.852095in}}%
\pgfpathlineto{\pgfqpoint{2.064418in}{2.857571in}}%
\pgfpathlineto{\pgfqpoint{2.064828in}{2.856727in}}%
\pgfpathlineto{\pgfqpoint{2.065648in}{2.855320in}}%
\pgfpathlineto{\pgfqpoint{2.066058in}{2.855603in}}%
\pgfpathlineto{\pgfqpoint{2.067288in}{2.857421in}}%
\pgfpathlineto{\pgfqpoint{2.067698in}{2.856904in}}%
\pgfpathlineto{\pgfqpoint{2.069748in}{2.851468in}}%
\pgfpathlineto{\pgfqpoint{2.070158in}{2.851944in}}%
\pgfpathlineto{\pgfqpoint{2.071388in}{2.853139in}}%
\pgfpathlineto{\pgfqpoint{2.072617in}{2.851257in}}%
\pgfpathlineto{\pgfqpoint{2.073027in}{2.851888in}}%
\pgfpathlineto{\pgfqpoint{2.074667in}{2.856747in}}%
\pgfpathlineto{\pgfqpoint{2.075077in}{2.856018in}}%
\pgfpathlineto{\pgfqpoint{2.075897in}{2.854497in}}%
\pgfpathlineto{\pgfqpoint{2.076307in}{2.854691in}}%
\pgfpathlineto{\pgfqpoint{2.077537in}{2.856533in}}%
\pgfpathlineto{\pgfqpoint{2.077947in}{2.856057in}}%
\pgfpathlineto{\pgfqpoint{2.079997in}{2.850761in}}%
\pgfpathlineto{\pgfqpoint{2.080407in}{2.851270in}}%
\pgfpathlineto{\pgfqpoint{2.081637in}{2.852297in}}%
\pgfpathlineto{\pgfqpoint{2.082867in}{2.850577in}}%
\pgfpathlineto{\pgfqpoint{2.083277in}{2.851444in}}%
\pgfpathlineto{\pgfqpoint{2.084507in}{2.855871in}}%
\pgfpathlineto{\pgfqpoint{2.084917in}{2.855727in}}%
\pgfpathlineto{\pgfqpoint{2.086147in}{2.853687in}}%
\pgfpathlineto{\pgfqpoint{2.086557in}{2.854140in}}%
\pgfpathlineto{\pgfqpoint{2.087787in}{2.855575in}}%
\pgfpathlineto{\pgfqpoint{2.089837in}{2.850038in}}%
\pgfpathlineto{\pgfqpoint{2.090657in}{2.851011in}}%
\pgfpathlineto{\pgfqpoint{2.091477in}{2.851699in}}%
\pgfpathlineto{\pgfqpoint{2.091887in}{2.851125in}}%
\pgfpathlineto{\pgfqpoint{2.092707in}{2.849845in}}%
\pgfpathlineto{\pgfqpoint{2.093117in}{2.850404in}}%
\pgfpathlineto{\pgfqpoint{2.094757in}{2.855013in}}%
\pgfpathlineto{\pgfqpoint{2.095167in}{2.854221in}}%
\pgfpathlineto{\pgfqpoint{2.095987in}{2.852924in}}%
\pgfpathlineto{\pgfqpoint{2.096397in}{2.853299in}}%
\pgfpathlineto{\pgfqpoint{2.097627in}{2.854798in}}%
\pgfpathlineto{\pgfqpoint{2.099677in}{2.849384in}}%
\pgfpathlineto{\pgfqpoint{2.100497in}{2.850400in}}%
\pgfpathlineto{\pgfqpoint{2.101317in}{2.850960in}}%
\pgfpathlineto{\pgfqpoint{2.101727in}{2.850317in}}%
\pgfpathlineto{\pgfqpoint{2.102547in}{2.849214in}}%
\pgfpathlineto{\pgfqpoint{2.102957in}{2.849991in}}%
\pgfpathlineto{\pgfqpoint{2.104187in}{2.854294in}}%
\pgfpathlineto{\pgfqpoint{2.104597in}{2.854074in}}%
\pgfpathlineto{\pgfqpoint{2.105827in}{2.852259in}}%
\pgfpathlineto{\pgfqpoint{2.106237in}{2.852846in}}%
\pgfpathlineto{\pgfqpoint{2.107057in}{2.854123in}}%
\pgfpathlineto{\pgfqpoint{2.107467in}{2.853800in}}%
\pgfpathlineto{\pgfqpoint{2.109517in}{2.848848in}}%
\pgfpathlineto{\pgfqpoint{2.109927in}{2.849428in}}%
\pgfpathlineto{\pgfqpoint{2.110747in}{2.850386in}}%
\pgfpathlineto{\pgfqpoint{2.111157in}{2.849972in}}%
\pgfpathlineto{\pgfqpoint{2.111977in}{2.848588in}}%
\pgfpathlineto{\pgfqpoint{2.112387in}{2.848982in}}%
\pgfpathlineto{\pgfqpoint{2.114027in}{2.853489in}}%
\pgfpathlineto{\pgfqpoint{2.114437in}{2.852688in}}%
\pgfpathlineto{\pgfqpoint{2.115257in}{2.851538in}}%
\pgfpathlineto{\pgfqpoint{2.115667in}{2.852017in}}%
\pgfpathlineto{\pgfqpoint{2.116487in}{2.853381in}}%
\pgfpathlineto{\pgfqpoint{2.116897in}{2.853169in}}%
\pgfpathlineto{\pgfqpoint{2.118947in}{2.848251in}}%
\pgfpathlineto{\pgfqpoint{2.119357in}{2.848822in}}%
\pgfpathlineto{\pgfqpoint{2.120177in}{2.849753in}}%
\pgfpathlineto{\pgfqpoint{2.120587in}{2.849311in}}%
\pgfpathlineto{\pgfqpoint{2.121407in}{2.847989in}}%
\pgfpathlineto{\pgfqpoint{2.121817in}{2.848500in}}%
\pgfpathlineto{\pgfqpoint{2.123047in}{2.852819in}}%
\pgfpathlineto{\pgfqpoint{2.123457in}{2.852688in}}%
\pgfpathlineto{\pgfqpoint{2.124687in}{2.850948in}}%
\pgfpathlineto{\pgfqpoint{2.125097in}{2.851569in}}%
\pgfpathlineto{\pgfqpoint{2.125917in}{2.852724in}}%
\pgfpathlineto{\pgfqpoint{2.126327in}{2.852242in}}%
\pgfpathlineto{\pgfqpoint{2.127967in}{2.847625in}}%
\pgfpathlineto{\pgfqpoint{2.128377in}{2.847874in}}%
\pgfpathlineto{\pgfqpoint{2.129607in}{2.849025in}}%
\pgfpathlineto{\pgfqpoint{2.130837in}{2.847557in}}%
\pgfpathlineto{\pgfqpoint{2.132477in}{2.852202in}}%
\pgfpathlineto{\pgfqpoint{2.132887in}{2.851522in}}%
\pgfpathlineto{\pgfqpoint{2.133707in}{2.850269in}}%
\pgfpathlineto{\pgfqpoint{2.134117in}{2.850703in}}%
\pgfpathlineto{\pgfqpoint{2.134937in}{2.852052in}}%
\pgfpathlineto{\pgfqpoint{2.135347in}{2.851805in}}%
\pgfpathlineto{\pgfqpoint{2.137397in}{2.847254in}}%
\pgfpathlineto{\pgfqpoint{2.137807in}{2.847908in}}%
\pgfpathlineto{\pgfqpoint{2.138627in}{2.848499in}}%
\pgfpathlineto{\pgfqpoint{2.139037in}{2.847843in}}%
\pgfpathlineto{\pgfqpoint{2.139857in}{2.846998in}}%
\pgfpathlineto{\pgfqpoint{2.141497in}{2.851564in}}%
\pgfpathlineto{\pgfqpoint{2.141907in}{2.850865in}}%
\pgfpathlineto{\pgfqpoint{2.142727in}{2.849694in}}%
\pgfpathlineto{\pgfqpoint{2.143137in}{2.850186in}}%
\pgfpathlineto{\pgfqpoint{2.143957in}{2.851453in}}%
\pgfpathlineto{\pgfqpoint{2.144367in}{2.851076in}}%
\pgfpathlineto{\pgfqpoint{2.146007in}{2.846577in}}%
\pgfpathlineto{\pgfqpoint{2.146417in}{2.846874in}}%
\pgfpathlineto{\pgfqpoint{2.147647in}{2.847816in}}%
\pgfpathlineto{\pgfqpoint{2.148467in}{2.846420in}}%
\pgfpathlineto{\pgfqpoint{2.148877in}{2.846743in}}%
\pgfpathlineto{\pgfqpoint{2.150107in}{2.850938in}}%
\pgfpathlineto{\pgfqpoint{2.150927in}{2.849857in}}%
\pgfpathlineto{\pgfqpoint{2.151747in}{2.849296in}}%
\pgfpathlineto{\pgfqpoint{2.152977in}{2.850776in}}%
\pgfpathlineto{\pgfqpoint{2.155027in}{2.846204in}}%
\pgfpathlineto{\pgfqpoint{2.155437in}{2.846823in}}%
\pgfpathlineto{\pgfqpoint{2.156257in}{2.847445in}}%
\pgfpathlineto{\pgfqpoint{2.156667in}{2.846789in}}%
\pgfpathlineto{\pgfqpoint{2.157487in}{2.846061in}}%
\pgfpathlineto{\pgfqpoint{2.159127in}{2.850294in}}%
\pgfpathlineto{\pgfqpoint{2.159537in}{2.849462in}}%
\pgfpathlineto{\pgfqpoint{2.160357in}{2.848708in}}%
\pgfpathlineto{\pgfqpoint{2.160767in}{2.849398in}}%
\pgfpathlineto{\pgfqpoint{2.161587in}{2.850247in}}%
\pgfpathlineto{\pgfqpoint{2.161997in}{2.849444in}}%
\pgfpathlineto{\pgfqpoint{2.163637in}{2.845747in}}%
\pgfpathlineto{\pgfqpoint{2.164867in}{2.846927in}}%
\pgfpathlineto{\pgfqpoint{2.165277in}{2.846235in}}%
\pgfpathlineto{\pgfqpoint{2.166097in}{2.845686in}}%
\pgfpathlineto{\pgfqpoint{2.167327in}{2.849819in}}%
\pgfpathlineto{\pgfqpoint{2.168147in}{2.848742in}}%
\pgfpathlineto{\pgfqpoint{2.168967in}{2.848328in}}%
\pgfpathlineto{\pgfqpoint{2.170197in}{2.849569in}}%
\pgfpathlineto{\pgfqpoint{2.171837in}{2.845190in}}%
\pgfpathlineto{\pgfqpoint{2.172657in}{2.846198in}}%
\pgfpathlineto{\pgfqpoint{2.173477in}{2.846212in}}%
\pgfpathlineto{\pgfqpoint{2.174297in}{2.845016in}}%
\pgfpathlineto{\pgfqpoint{2.174707in}{2.845774in}}%
\pgfpathlineto{\pgfqpoint{2.175937in}{2.849342in}}%
\pgfpathlineto{\pgfqpoint{2.176347in}{2.848646in}}%
\pgfpathlineto{\pgfqpoint{2.177167in}{2.847670in}}%
\pgfpathlineto{\pgfqpoint{2.177577in}{2.848298in}}%
\pgfpathlineto{\pgfqpoint{2.178397in}{2.849224in}}%
\pgfpathlineto{\pgfqpoint{2.178807in}{2.848447in}}%
\pgfpathlineto{\pgfqpoint{2.180037in}{2.844819in}}%
\pgfpathlineto{\pgfqpoint{2.180447in}{2.844942in}}%
\pgfpathlineto{\pgfqpoint{2.181677in}{2.845897in}}%
\pgfpathlineto{\pgfqpoint{2.182497in}{2.844594in}}%
\pgfpathlineto{\pgfqpoint{2.182907in}{2.845159in}}%
\pgfpathlineto{\pgfqpoint{2.184137in}{2.848871in}}%
\pgfpathlineto{\pgfqpoint{2.184547in}{2.848241in}}%
\pgfpathlineto{\pgfqpoint{2.185367in}{2.847201in}}%
\pgfpathlineto{\pgfqpoint{2.185777in}{2.847810in}}%
\pgfpathlineto{\pgfqpoint{2.186597in}{2.848741in}}%
\pgfpathlineto{\pgfqpoint{2.187007in}{2.847948in}}%
\pgfpathlineto{\pgfqpoint{2.188237in}{2.844396in}}%
\pgfpathlineto{\pgfqpoint{2.188647in}{2.844588in}}%
\pgfpathlineto{\pgfqpoint{2.189467in}{2.845694in}}%
\pgfpathlineto{\pgfqpoint{2.189877in}{2.845384in}}%
\pgfpathlineto{\pgfqpoint{2.190697in}{2.844203in}}%
\pgfpathlineto{\pgfqpoint{2.191107in}{2.845009in}}%
\pgfpathlineto{\pgfqpoint{2.192337in}{2.848329in}}%
\pgfpathlineto{\pgfqpoint{2.192747in}{2.847536in}}%
\pgfpathlineto{\pgfqpoint{2.193567in}{2.846884in}}%
\pgfpathlineto{\pgfqpoint{2.194797in}{2.848112in}}%
\pgfpathlineto{\pgfqpoint{2.196437in}{2.843997in}}%
\pgfpathlineto{\pgfqpoint{2.196847in}{2.844476in}}%
\pgfpathlineto{\pgfqpoint{2.197667in}{2.845239in}}%
\pgfpathlineto{\pgfqpoint{2.198077in}{2.844609in}}%
\pgfpathlineto{\pgfqpoint{2.198897in}{2.844083in}}%
\pgfpathlineto{\pgfqpoint{2.200127in}{2.847967in}}%
\pgfpathlineto{\pgfqpoint{2.200537in}{2.847480in}}%
\pgfpathlineto{\pgfqpoint{2.201357in}{2.846334in}}%
\pgfpathlineto{\pgfqpoint{2.201767in}{2.846910in}}%
\pgfpathlineto{\pgfqpoint{2.202587in}{2.847838in}}%
\pgfpathlineto{\pgfqpoint{2.202997in}{2.847007in}}%
\pgfpathlineto{\pgfqpoint{2.204227in}{2.843622in}}%
\pgfpathlineto{\pgfqpoint{2.204637in}{2.843947in}}%
\pgfpathlineto{\pgfqpoint{2.205457in}{2.844896in}}%
\pgfpathlineto{\pgfqpoint{2.205867in}{2.844384in}}%
\pgfpathlineto{\pgfqpoint{2.206687in}{2.843605in}}%
\pgfpathlineto{\pgfqpoint{2.207917in}{2.847529in}}%
\pgfpathlineto{\pgfqpoint{2.208327in}{2.847138in}}%
\pgfpathlineto{\pgfqpoint{2.209147in}{2.845930in}}%
\pgfpathlineto{\pgfqpoint{2.209557in}{2.846479in}}%
\pgfpathlineto{\pgfqpoint{2.210377in}{2.847424in}}%
\pgfpathlineto{\pgfqpoint{2.210787in}{2.846587in}}%
\pgfpathlineto{\pgfqpoint{2.212017in}{2.843271in}}%
\pgfpathlineto{\pgfqpoint{2.212427in}{2.843646in}}%
\pgfpathlineto{\pgfqpoint{2.213247in}{2.844506in}}%
\pgfpathlineto{\pgfqpoint{2.213657in}{2.843916in}}%
\pgfpathlineto{\pgfqpoint{2.214477in}{2.843382in}}%
\pgfpathlineto{\pgfqpoint{2.215707in}{2.847141in}}%
\pgfpathlineto{\pgfqpoint{2.216117in}{2.846538in}}%
\pgfpathlineto{\pgfqpoint{2.216937in}{2.845628in}}%
\pgfpathlineto{\pgfqpoint{2.217347in}{2.846330in}}%
\pgfpathlineto{\pgfqpoint{2.218167in}{2.846872in}}%
\pgfpathlineto{\pgfqpoint{2.219807in}{2.842995in}}%
\pgfpathlineto{\pgfqpoint{2.220217in}{2.843582in}}%
\pgfpathlineto{\pgfqpoint{2.221037in}{2.843982in}}%
\pgfpathlineto{\pgfqpoint{2.221857in}{2.842783in}}%
\pgfpathlineto{\pgfqpoint{2.222267in}{2.843609in}}%
\pgfpathlineto{\pgfqpoint{2.223497in}{2.846549in}}%
\pgfpathlineto{\pgfqpoint{2.224317in}{2.845182in}}%
\pgfpathlineto{\pgfqpoint{2.224727in}{2.845617in}}%
\pgfpathlineto{\pgfqpoint{2.225547in}{2.846675in}}%
\pgfpathlineto{\pgfqpoint{2.225957in}{2.845899in}}%
\pgfpathlineto{\pgfqpoint{2.227187in}{2.842621in}}%
\pgfpathlineto{\pgfqpoint{2.227597in}{2.843040in}}%
\pgfpathlineto{\pgfqpoint{2.228417in}{2.843776in}}%
\pgfpathlineto{\pgfqpoint{2.228827in}{2.843097in}}%
\pgfpathlineto{\pgfqpoint{2.229237in}{2.842477in}}%
\pgfpathlineto{\pgfqpoint{2.229647in}{2.842985in}}%
\pgfpathlineto{\pgfqpoint{2.230877in}{2.846292in}}%
\pgfpathlineto{\pgfqpoint{2.231287in}{2.845460in}}%
\pgfpathlineto{\pgfqpoint{2.231697in}{2.844854in}}%
\pgfpathlineto{\pgfqpoint{2.232107in}{2.845176in}}%
\pgfpathlineto{\pgfqpoint{2.232927in}{2.846336in}}%
\pgfpathlineto{\pgfqpoint{2.233337in}{2.845648in}}%
\pgfpathlineto{\pgfqpoint{2.234567in}{2.842317in}}%
\pgfpathlineto{\pgfqpoint{2.234977in}{2.842720in}}%
\pgfpathlineto{\pgfqpoint{2.235797in}{2.843452in}}%
\pgfpathlineto{\pgfqpoint{2.236207in}{2.842761in}}%
\pgfpathlineto{\pgfqpoint{2.236617in}{2.842169in}}%
\pgfpathlineto{\pgfqpoint{2.237027in}{2.842759in}}%
\pgfpathlineto{\pgfqpoint{2.238257in}{2.845885in}}%
\pgfpathlineto{\pgfqpoint{2.238667in}{2.845012in}}%
\pgfpathlineto{\pgfqpoint{2.239077in}{2.844516in}}%
\pgfpathlineto{\pgfqpoint{2.239487in}{2.844968in}}%
\pgfpathlineto{\pgfqpoint{2.240307in}{2.845946in}}%
\pgfpathlineto{\pgfqpoint{2.240717in}{2.845056in}}%
\pgfpathlineto{\pgfqpoint{2.241947in}{2.842059in}}%
\pgfpathlineto{\pgfqpoint{2.242357in}{2.842616in}}%
\pgfpathlineto{\pgfqpoint{2.243177in}{2.842999in}}%
\pgfpathlineto{\pgfqpoint{2.243997in}{2.841919in}}%
\pgfpathlineto{\pgfqpoint{2.244407in}{2.842984in}}%
\pgfpathlineto{\pgfqpoint{2.245227in}{2.845709in}}%
\pgfpathlineto{\pgfqpoint{2.245637in}{2.845250in}}%
\pgfpathlineto{\pgfqpoint{2.246457in}{2.844298in}}%
\pgfpathlineto{\pgfqpoint{2.246867in}{2.845028in}}%
\pgfpathlineto{\pgfqpoint{2.247277in}{2.845653in}}%
\pgfpathlineto{\pgfqpoint{2.247687in}{2.845325in}}%
\pgfpathlineto{\pgfqpoint{2.248917in}{2.841778in}}%
\pgfpathlineto{\pgfqpoint{2.249327in}{2.842021in}}%
\pgfpathlineto{\pgfqpoint{2.250147in}{2.842899in}}%
\pgfpathlineto{\pgfqpoint{2.250557in}{2.842268in}}%
\pgfpathlineto{\pgfqpoint{2.250967in}{2.841622in}}%
\pgfpathlineto{\pgfqpoint{2.251377in}{2.842146in}}%
\pgfpathlineto{\pgfqpoint{2.252607in}{2.845209in}}%
\pgfpathlineto{\pgfqpoint{2.253017in}{2.844318in}}%
\pgfpathlineto{\pgfqpoint{2.253427in}{2.843920in}}%
\pgfpathlineto{\pgfqpoint{2.253837in}{2.844481in}}%
\pgfpathlineto{\pgfqpoint{2.254657in}{2.845209in}}%
\pgfpathlineto{\pgfqpoint{2.256297in}{2.841647in}}%
\pgfpathlineto{\pgfqpoint{2.256707in}{2.842323in}}%
\pgfpathlineto{\pgfqpoint{2.257117in}{2.842652in}}%
\pgfpathlineto{\pgfqpoint{2.257527in}{2.842121in}}%
\pgfpathlineto{\pgfqpoint{2.257937in}{2.841399in}}%
\pgfpathlineto{\pgfqpoint{2.258347in}{2.841748in}}%
\pgfpathlineto{\pgfqpoint{2.259577in}{2.844957in}}%
\pgfpathlineto{\pgfqpoint{2.259987in}{2.844077in}}%
\pgfpathlineto{\pgfqpoint{2.260397in}{2.843637in}}%
\pgfpathlineto{\pgfqpoint{2.260807in}{2.844179in}}%
\pgfpathlineto{\pgfqpoint{2.261627in}{2.844911in}}%
\pgfpathlineto{\pgfqpoint{2.263267in}{2.841433in}}%
\pgfpathlineto{\pgfqpoint{2.263677in}{2.842119in}}%
\pgfpathlineto{\pgfqpoint{2.264497in}{2.841766in}}%
\pgfpathlineto{\pgfqpoint{2.264907in}{2.841114in}}%
\pgfpathlineto{\pgfqpoint{2.265317in}{2.841677in}}%
\pgfpathlineto{\pgfqpoint{2.266137in}{2.844725in}}%
\pgfpathlineto{\pgfqpoint{2.266547in}{2.844540in}}%
\pgfpathlineto{\pgfqpoint{2.267367in}{2.843418in}}%
\pgfpathlineto{\pgfqpoint{2.267777in}{2.844116in}}%
\pgfpathlineto{\pgfqpoint{2.268187in}{2.844769in}}%
\pgfpathlineto{\pgfqpoint{2.268597in}{2.844423in}}%
\pgfpathlineto{\pgfqpoint{2.269827in}{2.841000in}}%
\pgfpathlineto{\pgfqpoint{2.270237in}{2.841403in}}%
\pgfpathlineto{\pgfqpoint{2.271057in}{2.841987in}}%
\pgfpathlineto{\pgfqpoint{2.271877in}{2.840914in}}%
\pgfpathlineto{\pgfqpoint{2.273107in}{2.844554in}}%
\pgfpathlineto{\pgfqpoint{2.273517in}{2.843863in}}%
\pgfpathlineto{\pgfqpoint{2.273927in}{2.843145in}}%
\pgfpathlineto{\pgfqpoint{2.274337in}{2.843430in}}%
\pgfpathlineto{\pgfqpoint{2.275157in}{2.844479in}}%
\pgfpathlineto{\pgfqpoint{2.275567in}{2.843535in}}%
\pgfpathlineto{\pgfqpoint{2.276797in}{2.840912in}}%
\pgfpathlineto{\pgfqpoint{2.277207in}{2.841592in}}%
\pgfpathlineto{\pgfqpoint{2.277617in}{2.841891in}}%
\pgfpathlineto{\pgfqpoint{2.278027in}{2.841297in}}%
\pgfpathlineto{\pgfqpoint{2.278437in}{2.840651in}}%
\pgfpathlineto{\pgfqpoint{2.278847in}{2.841284in}}%
\pgfpathlineto{\pgfqpoint{2.279667in}{2.844262in}}%
\pgfpathlineto{\pgfqpoint{2.280077in}{2.843902in}}%
\pgfpathlineto{\pgfqpoint{2.280897in}{2.843014in}}%
\pgfpathlineto{\pgfqpoint{2.281307in}{2.843814in}}%
\pgfpathlineto{\pgfqpoint{2.281717in}{2.844280in}}%
\pgfpathlineto{\pgfqpoint{2.282127in}{2.843594in}}%
\pgfpathlineto{\pgfqpoint{2.283357in}{2.840617in}}%
\pgfpathlineto{\pgfqpoint{2.283767in}{2.841252in}}%
\pgfpathlineto{\pgfqpoint{2.284177in}{2.841679in}}%
\pgfpathlineto{\pgfqpoint{2.284587in}{2.841194in}}%
\pgfpathlineto{\pgfqpoint{2.284997in}{2.840469in}}%
\pgfpathlineto{\pgfqpoint{2.285407in}{2.840923in}}%
\pgfpathlineto{\pgfqpoint{2.286227in}{2.843986in}}%
\pgfpathlineto{\pgfqpoint{2.286637in}{2.843722in}}%
\pgfpathlineto{\pgfqpoint{2.287457in}{2.842768in}}%
\pgfpathlineto{\pgfqpoint{2.287867in}{2.843564in}}%
\pgfpathlineto{\pgfqpoint{2.288277in}{2.844042in}}%
\pgfpathlineto{\pgfqpoint{2.288687in}{2.843349in}}%
\pgfpathlineto{\pgfqpoint{2.289917in}{2.840431in}}%
\pgfpathlineto{\pgfqpoint{2.290327in}{2.841090in}}%
\pgfpathlineto{\pgfqpoint{2.290737in}{2.841459in}}%
\pgfpathlineto{\pgfqpoint{2.291147in}{2.840900in}}%
\pgfpathlineto{\pgfqpoint{2.291557in}{2.840238in}}%
\pgfpathlineto{\pgfqpoint{2.291967in}{2.840894in}}%
\pgfpathlineto{\pgfqpoint{2.292787in}{2.843823in}}%
\pgfpathlineto{\pgfqpoint{2.293197in}{2.843340in}}%
\pgfpathlineto{\pgfqpoint{2.294017in}{2.842674in}}%
\pgfpathlineto{\pgfqpoint{2.294837in}{2.843769in}}%
\pgfpathlineto{\pgfqpoint{2.295247in}{2.842794in}}%
\pgfpathlineto{\pgfqpoint{2.296067in}{2.840221in}}%
\pgfpathlineto{\pgfqpoint{2.296477in}{2.840384in}}%
\pgfpathlineto{\pgfqpoint{2.297297in}{2.841170in}}%
\pgfpathlineto{\pgfqpoint{2.297707in}{2.840415in}}%
\pgfpathlineto{\pgfqpoint{2.298117in}{2.840084in}}%
\pgfpathlineto{\pgfqpoint{2.299347in}{2.843576in}}%
\pgfpathlineto{\pgfqpoint{2.299757in}{2.842731in}}%
\pgfpathlineto{\pgfqpoint{2.300167in}{2.842233in}}%
\pgfpathlineto{\pgfqpoint{2.300577in}{2.842825in}}%
\pgfpathlineto{\pgfqpoint{2.300987in}{2.843559in}}%
\pgfpathlineto{\pgfqpoint{2.301397in}{2.843244in}}%
\pgfpathlineto{\pgfqpoint{2.302627in}{2.839987in}}%
\pgfpathlineto{\pgfqpoint{2.303037in}{2.840572in}}%
\pgfpathlineto{\pgfqpoint{2.303447in}{2.841065in}}%
\pgfpathlineto{\pgfqpoint{2.303857in}{2.840611in}}%
\pgfpathlineto{\pgfqpoint{2.304267in}{2.839885in}}%
\pgfpathlineto{\pgfqpoint{2.304677in}{2.840434in}}%
\pgfpathlineto{\pgfqpoint{2.305497in}{2.843409in}}%
\pgfpathlineto{\pgfqpoint{2.305907in}{2.842902in}}%
\pgfpathlineto{\pgfqpoint{2.306317in}{2.842089in}}%
\pgfpathlineto{\pgfqpoint{2.306727in}{2.842340in}}%
\pgfpathlineto{\pgfqpoint{2.307547in}{2.843288in}}%
\pgfpathlineto{\pgfqpoint{2.308777in}{2.839798in}}%
\pgfpathlineto{\pgfqpoint{2.309187in}{2.840198in}}%
\pgfpathlineto{\pgfqpoint{2.310007in}{2.840611in}}%
\pgfpathlineto{\pgfqpoint{2.310827in}{2.840013in}}%
\pgfpathlineto{\pgfqpoint{2.311647in}{2.843148in}}%
\pgfpathlineto{\pgfqpoint{2.312057in}{2.842864in}}%
\pgfpathlineto{\pgfqpoint{2.312877in}{2.842066in}}%
\pgfpathlineto{\pgfqpoint{2.313697in}{2.843149in}}%
\pgfpathlineto{\pgfqpoint{2.314107in}{2.842079in}}%
\pgfpathlineto{\pgfqpoint{2.314927in}{2.839639in}}%
\pgfpathlineto{\pgfqpoint{2.315337in}{2.839999in}}%
\pgfpathlineto{\pgfqpoint{2.316157in}{2.840451in}}%
\pgfpathlineto{\pgfqpoint{2.316567in}{2.839643in}}%
\pgfpathlineto{\pgfqpoint{2.316977in}{2.839867in}}%
\pgfpathlineto{\pgfqpoint{2.317797in}{2.842995in}}%
\pgfpathlineto{\pgfqpoint{2.318207in}{2.842633in}}%
\pgfpathlineto{\pgfqpoint{2.319027in}{2.841957in}}%
\pgfpathlineto{\pgfqpoint{2.319847in}{2.842912in}}%
\pgfpathlineto{\pgfqpoint{2.321077in}{2.839468in}}%
\pgfpathlineto{\pgfqpoint{2.321487in}{2.839969in}}%
\pgfpathlineto{\pgfqpoint{2.321897in}{2.840539in}}%
\pgfpathlineto{\pgfqpoint{2.322307in}{2.840135in}}%
\pgfpathlineto{\pgfqpoint{2.322717in}{2.839396in}}%
\pgfpathlineto{\pgfqpoint{2.323127in}{2.840017in}}%
\pgfpathlineto{\pgfqpoint{2.323947in}{2.842894in}}%
\pgfpathlineto{\pgfqpoint{2.324357in}{2.842195in}}%
\pgfpathlineto{\pgfqpoint{2.324767in}{2.841527in}}%
\pgfpathlineto{\pgfqpoint{2.325177in}{2.842049in}}%
\pgfpathlineto{\pgfqpoint{2.325587in}{2.842824in}}%
\pgfpathlineto{\pgfqpoint{2.325997in}{2.842473in}}%
\pgfpathlineto{\pgfqpoint{2.327227in}{2.839406in}}%
\pgfpathlineto{\pgfqpoint{2.327637in}{2.840107in}}%
\pgfpathlineto{\pgfqpoint{2.328047in}{2.840353in}}%
\pgfpathlineto{\pgfqpoint{2.328457in}{2.839623in}}%
\pgfpathlineto{\pgfqpoint{2.328867in}{2.839287in}}%
\pgfpathlineto{\pgfqpoint{2.330097in}{2.842548in}}%
\pgfpathlineto{\pgfqpoint{2.330507in}{2.841596in}}%
\pgfpathlineto{\pgfqpoint{2.331327in}{2.842364in}}%
\pgfpathlineto{\pgfqpoint{2.331737in}{2.842667in}}%
\pgfpathlineto{\pgfqpoint{2.332147in}{2.841593in}}%
\pgfpathlineto{\pgfqpoint{2.332967in}{2.839181in}}%
\pgfpathlineto{\pgfqpoint{2.333377in}{2.839643in}}%
\pgfpathlineto{\pgfqpoint{2.333787in}{2.840240in}}%
\pgfpathlineto{\pgfqpoint{2.334197in}{2.839833in}}%
\pgfpathlineto{\pgfqpoint{2.334607in}{2.839107in}}%
\pgfpathlineto{\pgfqpoint{2.335017in}{2.839847in}}%
\pgfpathlineto{\pgfqpoint{2.335837in}{2.842579in}}%
\pgfpathlineto{\pgfqpoint{2.336247in}{2.841742in}}%
\pgfpathlineto{\pgfqpoint{2.336657in}{2.841247in}}%
\pgfpathlineto{\pgfqpoint{2.337067in}{2.841952in}}%
\pgfpathlineto{\pgfqpoint{2.337477in}{2.842572in}}%
\pgfpathlineto{\pgfqpoint{2.337887in}{2.841851in}}%
\pgfpathlineto{\pgfqpoint{2.338707in}{2.839109in}}%
\pgfpathlineto{\pgfqpoint{2.339117in}{2.839339in}}%
\pgfpathlineto{\pgfqpoint{2.339937in}{2.839860in}}%
\pgfpathlineto{\pgfqpoint{2.340347in}{2.839047in}}%
\pgfpathlineto{\pgfqpoint{2.340757in}{2.839459in}}%
\pgfpathlineto{\pgfqpoint{2.341577in}{2.842461in}}%
\pgfpathlineto{\pgfqpoint{2.341987in}{2.841744in}}%
\pgfpathlineto{\pgfqpoint{2.342397in}{2.841104in}}%
\pgfpathlineto{\pgfqpoint{2.342807in}{2.841723in}}%
\pgfpathlineto{\pgfqpoint{2.343217in}{2.842428in}}%
\pgfpathlineto{\pgfqpoint{2.343627in}{2.841818in}}%
\pgfpathlineto{\pgfqpoint{2.344447in}{2.839007in}}%
\pgfpathlineto{\pgfqpoint{2.344857in}{2.839191in}}%
\pgfpathlineto{\pgfqpoint{2.345677in}{2.839741in}}%
\pgfpathlineto{\pgfqpoint{2.346087in}{2.838924in}}%
\pgfpathlineto{\pgfqpoint{2.346497in}{2.839373in}}%
\pgfpathlineto{\pgfqpoint{2.347317in}{2.842338in}}%
\pgfpathlineto{\pgfqpoint{2.347727in}{2.841555in}}%
\pgfpathlineto{\pgfqpoint{2.348137in}{2.840989in}}%
\pgfpathlineto{\pgfqpoint{2.348547in}{2.841688in}}%
\pgfpathlineto{\pgfqpoint{2.348957in}{2.842313in}}%
\pgfpathlineto{\pgfqpoint{2.349367in}{2.841530in}}%
\pgfpathlineto{\pgfqpoint{2.350187in}{2.838835in}}%
\pgfpathlineto{\pgfqpoint{2.350597in}{2.839194in}}%
\pgfpathlineto{\pgfqpoint{2.351007in}{2.839850in}}%
\pgfpathlineto{\pgfqpoint{2.351417in}{2.839476in}}%
\pgfpathlineto{\pgfqpoint{2.351827in}{2.838747in}}%
\pgfpathlineto{\pgfqpoint{2.352237in}{2.839613in}}%
\pgfpathlineto{\pgfqpoint{2.353057in}{2.842151in}}%
\pgfpathlineto{\pgfqpoint{2.353467in}{2.841190in}}%
\pgfpathlineto{\pgfqpoint{2.353877in}{2.840965in}}%
\pgfpathlineto{\pgfqpoint{2.354287in}{2.841840in}}%
\pgfpathlineto{\pgfqpoint{2.354697in}{2.842129in}}%
\pgfpathlineto{\pgfqpoint{2.355927in}{2.838716in}}%
\pgfpathlineto{\pgfqpoint{2.356337in}{2.839368in}}%
\pgfpathlineto{\pgfqpoint{2.356747in}{2.839747in}}%
\pgfpathlineto{\pgfqpoint{2.357157in}{2.839028in}}%
\pgfpathlineto{\pgfqpoint{2.357567in}{2.838732in}}%
\pgfpathlineto{\pgfqpoint{2.358387in}{2.842008in}}%
\pgfpathlineto{\pgfqpoint{2.359207in}{2.840803in}}%
\pgfpathlineto{\pgfqpoint{2.360027in}{2.842046in}}%
\pgfpathlineto{\pgfqpoint{2.360437in}{2.841603in}}%
\pgfpathlineto{\pgfqpoint{2.361257in}{2.838693in}}%
\pgfpathlineto{\pgfqpoint{2.361667in}{2.838882in}}%
\pgfpathlineto{\pgfqpoint{2.362077in}{2.839606in}}%
\pgfpathlineto{\pgfqpoint{2.362487in}{2.839342in}}%
\pgfpathlineto{\pgfqpoint{2.362897in}{2.838563in}}%
\pgfpathlineto{\pgfqpoint{2.363307in}{2.839370in}}%
\pgfpathlineto{\pgfqpoint{2.364127in}{2.841940in}}%
\pgfpathlineto{\pgfqpoint{2.364537in}{2.840945in}}%
\pgfpathlineto{\pgfqpoint{2.365357in}{2.841733in}}%
\pgfpathlineto{\pgfqpoint{2.365767in}{2.841848in}}%
\pgfpathlineto{\pgfqpoint{2.366997in}{2.838599in}}%
\pgfpathlineto{\pgfqpoint{2.367407in}{2.839355in}}%
\pgfpathlineto{\pgfqpoint{2.368637in}{2.838880in}}%
\pgfpathlineto{\pgfqpoint{2.369457in}{2.841943in}}%
\pgfpathlineto{\pgfqpoint{2.369867in}{2.841067in}}%
\pgfpathlineto{\pgfqpoint{2.370277in}{2.840631in}}%
\pgfpathlineto{\pgfqpoint{2.370687in}{2.841484in}}%
\pgfpathlineto{\pgfqpoint{2.371097in}{2.841860in}}%
\pgfpathlineto{\pgfqpoint{2.372327in}{2.838460in}}%
\pgfpathlineto{\pgfqpoint{2.372737in}{2.839184in}}%
\pgfpathlineto{\pgfqpoint{2.374377in}{2.840684in}}%
\pgfpathlineto{\pgfqpoint{2.374787in}{2.841874in}}%
\pgfpathlineto{\pgfqpoint{2.375197in}{2.841034in}}%
\pgfpathlineto{\pgfqpoint{2.375607in}{2.840546in}}%
\pgfpathlineto{\pgfqpoint{2.376017in}{2.841396in}}%
\pgfpathlineto{\pgfqpoint{2.376427in}{2.841782in}}%
\pgfpathlineto{\pgfqpoint{2.377657in}{2.838397in}}%
\pgfpathlineto{\pgfqpoint{2.378067in}{2.839148in}}%
\pgfpathlineto{\pgfqpoint{2.379707in}{2.840840in}}%
\pgfpathlineto{\pgfqpoint{2.380117in}{2.841790in}}%
\pgfpathlineto{\pgfqpoint{2.380527in}{2.840825in}}%
\pgfpathlineto{\pgfqpoint{2.380937in}{2.840543in}}%
\pgfpathlineto{\pgfqpoint{2.381347in}{2.841476in}}%
\pgfpathlineto{\pgfqpoint{2.381757in}{2.841611in}}%
\pgfpathlineto{\pgfqpoint{2.382987in}{2.838435in}}%
\pgfpathlineto{\pgfqpoint{2.383397in}{2.839226in}}%
\pgfpathlineto{\pgfqpoint{2.384627in}{2.839151in}}%
\pgfpathlineto{\pgfqpoint{2.385447in}{2.841557in}}%
\pgfpathlineto{\pgfqpoint{2.385857in}{2.840507in}}%
\pgfpathlineto{\pgfqpoint{2.387087in}{2.841188in}}%
\pgfpathlineto{\pgfqpoint{2.387907in}{2.838239in}}%
\pgfpathlineto{\pgfqpoint{2.388317in}{2.838671in}}%
\pgfpathlineto{\pgfqpoint{2.388727in}{2.839285in}}%
\pgfpathlineto{\pgfqpoint{2.389137in}{2.838649in}}%
\pgfpathlineto{\pgfqpoint{2.389547in}{2.838283in}}%
\pgfpathlineto{\pgfqpoint{2.390367in}{2.841683in}}%
\pgfpathlineto{\pgfqpoint{2.390777in}{2.841002in}}%
\pgfpathlineto{\pgfqpoint{2.391187in}{2.840346in}}%
\pgfpathlineto{\pgfqpoint{2.391597in}{2.841178in}}%
\pgfpathlineto{\pgfqpoint{2.392007in}{2.841606in}}%
\pgfpathlineto{\pgfqpoint{2.393237in}{2.838259in}}%
\pgfpathlineto{\pgfqpoint{2.393647in}{2.839069in}}%
\pgfpathlineto{\pgfqpoint{2.394877in}{2.839073in}}%
\pgfpathlineto{\pgfqpoint{2.395697in}{2.841416in}}%
\pgfpathlineto{\pgfqpoint{2.396107in}{2.840363in}}%
\pgfpathlineto{\pgfqpoint{2.396927in}{2.841603in}}%
\pgfpathlineto{\pgfqpoint{2.397337in}{2.840847in}}%
\pgfpathlineto{\pgfqpoint{2.398157in}{2.838108in}}%
\pgfpathlineto{\pgfqpoint{2.398567in}{2.838784in}}%
\pgfpathlineto{\pgfqpoint{2.402257in}{2.841094in}}%
\pgfpathlineto{\pgfqpoint{2.403077in}{2.838077in}}%
\pgfpathlineto{\pgfqpoint{2.403487in}{2.838606in}}%
\pgfpathlineto{\pgfqpoint{2.403897in}{2.839140in}}%
\pgfpathlineto{\pgfqpoint{2.404307in}{2.838360in}}%
\pgfpathlineto{\pgfqpoint{2.404717in}{2.838363in}}%
\pgfpathlineto{\pgfqpoint{2.405537in}{2.841604in}}%
\pgfpathlineto{\pgfqpoint{2.405947in}{2.840537in}}%
\pgfpathlineto{\pgfqpoint{2.407587in}{2.839246in}}%
\pgfpathlineto{\pgfqpoint{2.407997in}{2.838043in}}%
\pgfpathlineto{\pgfqpoint{2.408407in}{2.838575in}}%
\pgfpathlineto{\pgfqpoint{2.411687in}{2.841482in}}%
\pgfpathlineto{\pgfqpoint{2.414147in}{2.838136in}}%
\pgfpathlineto{\pgfqpoint{2.414557in}{2.838703in}}%
\pgfpathlineto{\pgfqpoint{2.415377in}{2.841386in}}%
\pgfpathlineto{\pgfqpoint{2.415787in}{2.840264in}}%
\pgfpathlineto{\pgfqpoint{2.416607in}{2.841540in}}%
\pgfpathlineto{\pgfqpoint{2.417017in}{2.840444in}}%
\pgfpathlineto{\pgfqpoint{2.417837in}{2.838062in}}%
\pgfpathlineto{\pgfqpoint{2.418247in}{2.838927in}}%
\pgfpathlineto{\pgfqpoint{2.419067in}{2.838002in}}%
\pgfpathlineto{\pgfqpoint{2.419477in}{2.839482in}}%
\pgfpathlineto{\pgfqpoint{2.419887in}{2.841544in}}%
\pgfpathlineto{\pgfqpoint{2.420707in}{2.840223in}}%
\pgfpathlineto{\pgfqpoint{2.421527in}{2.841347in}}%
\pgfpathlineto{\pgfqpoint{2.422347in}{2.838037in}}%
\pgfpathlineto{\pgfqpoint{2.423167in}{2.839070in}}%
\pgfpathlineto{\pgfqpoint{2.423987in}{2.838361in}}%
\pgfpathlineto{\pgfqpoint{2.424807in}{2.841539in}}%
\pgfpathlineto{\pgfqpoint{2.425217in}{2.840343in}}%
\pgfpathlineto{\pgfqpoint{2.426037in}{2.841549in}}%
\pgfpathlineto{\pgfqpoint{2.426447in}{2.840521in}}%
\pgfpathlineto{\pgfqpoint{2.427267in}{2.838041in}}%
\pgfpathlineto{\pgfqpoint{2.427677in}{2.838935in}}%
\pgfpathlineto{\pgfqpoint{2.428497in}{2.838009in}}%
\pgfpathlineto{\pgfqpoint{2.429317in}{2.841664in}}%
\pgfpathlineto{\pgfqpoint{2.430137in}{2.840335in}}%
\pgfpathlineto{\pgfqpoint{2.430547in}{2.841411in}}%
\pgfpathlineto{\pgfqpoint{2.430957in}{2.841087in}}%
\pgfpathlineto{\pgfqpoint{2.431777in}{2.837945in}}%
\pgfpathlineto{\pgfqpoint{2.432187in}{2.838703in}}%
\pgfpathlineto{\pgfqpoint{2.433417in}{2.839161in}}%
\pgfpathlineto{\pgfqpoint{2.433827in}{2.841526in}}%
\pgfpathlineto{\pgfqpoint{2.434647in}{2.840255in}}%
\pgfpathlineto{\pgfqpoint{2.435467in}{2.841349in}}%
\pgfpathlineto{\pgfqpoint{2.436287in}{2.837972in}}%
\pgfpathlineto{\pgfqpoint{2.436697in}{2.838562in}}%
\pgfpathlineto{\pgfqpoint{2.439977in}{2.841434in}}%
\pgfpathlineto{\pgfqpoint{2.440797in}{2.837990in}}%
\pgfpathlineto{\pgfqpoint{2.441617in}{2.839063in}}%
\pgfpathlineto{\pgfqpoint{2.442027in}{2.838129in}}%
\pgfpathlineto{\pgfqpoint{2.442437in}{2.839006in}}%
\pgfpathlineto{\pgfqpoint{2.442847in}{2.841545in}}%
\pgfpathlineto{\pgfqpoint{2.443667in}{2.840317in}}%
\pgfpathlineto{\pgfqpoint{2.444487in}{2.841377in}}%
\pgfpathlineto{\pgfqpoint{2.445307in}{2.837975in}}%
\pgfpathlineto{\pgfqpoint{2.445717in}{2.838713in}}%
\pgfpathlineto{\pgfqpoint{2.446947in}{2.839463in}}%
\pgfpathlineto{\pgfqpoint{2.447357in}{2.841801in}}%
\pgfpathlineto{\pgfqpoint{2.448177in}{2.840457in}}%
\pgfpathlineto{\pgfqpoint{2.448587in}{2.841596in}}%
\pgfpathlineto{\pgfqpoint{2.448997in}{2.841092in}}%
\pgfpathlineto{\pgfqpoint{2.449817in}{2.838033in}}%
\pgfpathlineto{\pgfqpoint{2.450227in}{2.838984in}}%
\pgfpathlineto{\pgfqpoint{2.451047in}{2.838145in}}%
\pgfpathlineto{\pgfqpoint{2.451867in}{2.841932in}}%
\pgfpathlineto{\pgfqpoint{2.452277in}{2.840631in}}%
\pgfpathlineto{\pgfqpoint{2.453097in}{2.841791in}}%
\pgfpathlineto{\pgfqpoint{2.453507in}{2.840410in}}%
\pgfpathlineto{\pgfqpoint{2.454327in}{2.838347in}}%
\pgfpathlineto{\pgfqpoint{2.454737in}{2.839207in}}%
\pgfpathlineto{\pgfqpoint{2.455147in}{2.838402in}}%
\pgfpathlineto{\pgfqpoint{2.455557in}{2.838761in}}%
\pgfpathlineto{\pgfqpoint{2.456377in}{2.841550in}}%
\pgfpathlineto{\pgfqpoint{2.456787in}{2.840464in}}%
\pgfpathlineto{\pgfqpoint{2.457607in}{2.841520in}}%
\pgfpathlineto{\pgfqpoint{2.458427in}{2.838055in}}%
\pgfpathlineto{\pgfqpoint{2.458837in}{2.838949in}}%
\pgfpathlineto{\pgfqpoint{2.459657in}{2.838198in}}%
\pgfpathlineto{\pgfqpoint{2.460477in}{2.842114in}}%
\pgfpathlineto{\pgfqpoint{2.461297in}{2.841003in}}%
\pgfpathlineto{\pgfqpoint{2.461707in}{2.841932in}}%
\pgfpathlineto{\pgfqpoint{2.462117in}{2.840363in}}%
\pgfpathlineto{\pgfqpoint{2.462527in}{2.838277in}}%
\pgfpathlineto{\pgfqpoint{2.463347in}{2.839298in}}%
\pgfpathlineto{\pgfqpoint{2.463757in}{2.838349in}}%
\pgfpathlineto{\pgfqpoint{2.464167in}{2.839348in}}%
\pgfpathlineto{\pgfqpoint{2.464577in}{2.842068in}}%
\pgfpathlineto{\pgfqpoint{2.465397in}{2.840729in}}%
\pgfpathlineto{\pgfqpoint{2.465807in}{2.841938in}}%
\pgfpathlineto{\pgfqpoint{2.466217in}{2.841119in}}%
\pgfpathlineto{\pgfqpoint{2.467037in}{2.838320in}}%
\pgfpathlineto{\pgfqpoint{2.467447in}{2.839338in}}%
\pgfpathlineto{\pgfqpoint{2.468267in}{2.838837in}}%
\pgfpathlineto{\pgfqpoint{2.469087in}{2.841791in}}%
\pgfpathlineto{\pgfqpoint{2.469497in}{2.840700in}}%
\pgfpathlineto{\pgfqpoint{2.469907in}{2.841864in}}%
\pgfpathlineto{\pgfqpoint{2.470317in}{2.841552in}}%
\pgfpathlineto{\pgfqpoint{2.471137in}{2.838260in}}%
\pgfpathlineto{\pgfqpoint{2.471547in}{2.839325in}}%
\pgfpathlineto{\pgfqpoint{2.472367in}{2.838688in}}%
\pgfpathlineto{\pgfqpoint{2.473187in}{2.842052in}}%
\pgfpathlineto{\pgfqpoint{2.473597in}{2.840777in}}%
\pgfpathlineto{\pgfqpoint{2.474417in}{2.841746in}}%
\pgfpathlineto{\pgfqpoint{2.475237in}{2.838298in}}%
\pgfpathlineto{\pgfqpoint{2.475647in}{2.839379in}}%
\pgfpathlineto{\pgfqpoint{2.476467in}{2.838769in}}%
\pgfpathlineto{\pgfqpoint{2.477287in}{2.842131in}}%
\pgfpathlineto{\pgfqpoint{2.477697in}{2.840878in}}%
\pgfpathlineto{\pgfqpoint{2.478107in}{2.842050in}}%
\pgfpathlineto{\pgfqpoint{2.478517in}{2.841739in}}%
\pgfpathlineto{\pgfqpoint{2.479337in}{2.838414in}}%
\pgfpathlineto{\pgfqpoint{2.479747in}{2.839522in}}%
\pgfpathlineto{\pgfqpoint{2.480567in}{2.839094in}}%
\pgfpathlineto{\pgfqpoint{2.480977in}{2.842246in}}%
\pgfpathlineto{\pgfqpoint{2.481797in}{2.841045in}}%
\pgfpathlineto{\pgfqpoint{2.482207in}{2.842332in}}%
\pgfpathlineto{\pgfqpoint{2.482617in}{2.841463in}}%
\pgfpathlineto{\pgfqpoint{2.483437in}{2.838672in}}%
\pgfpathlineto{\pgfqpoint{2.483847in}{2.839678in}}%
\pgfpathlineto{\pgfqpoint{2.484257in}{2.838748in}}%
\pgfpathlineto{\pgfqpoint{2.484667in}{2.839848in}}%
\pgfpathlineto{\pgfqpoint{2.485077in}{2.842781in}}%
\pgfpathlineto{\pgfqpoint{2.485897in}{2.841435in}}%
\pgfpathlineto{\pgfqpoint{2.486307in}{2.842566in}}%
\pgfpathlineto{\pgfqpoint{2.486717in}{2.840796in}}%
\pgfpathlineto{\pgfqpoint{2.487127in}{2.838554in}}%
\pgfpathlineto{\pgfqpoint{2.487947in}{2.839645in}}%
\pgfpathlineto{\pgfqpoint{2.488357in}{2.838697in}}%
\pgfpathlineto{\pgfqpoint{2.489177in}{2.842917in}}%
\pgfpathlineto{\pgfqpoint{2.489997in}{2.842147in}}%
\pgfpathlineto{\pgfqpoint{2.490407in}{2.842370in}}%
\pgfpathlineto{\pgfqpoint{2.491227in}{2.838577in}}%
\pgfpathlineto{\pgfqpoint{2.491637in}{2.839752in}}%
\pgfpathlineto{\pgfqpoint{2.492457in}{2.839429in}}%
\pgfpathlineto{\pgfqpoint{2.492867in}{2.842794in}}%
\pgfpathlineto{\pgfqpoint{2.493687in}{2.841504in}}%
\pgfpathlineto{\pgfqpoint{2.494097in}{2.842820in}}%
\pgfpathlineto{\pgfqpoint{2.494507in}{2.841316in}}%
\pgfpathlineto{\pgfqpoint{2.494917in}{2.838802in}}%
\pgfpathlineto{\pgfqpoint{2.495737in}{2.839880in}}%
\pgfpathlineto{\pgfqpoint{2.496147in}{2.838896in}}%
\pgfpathlineto{\pgfqpoint{2.496967in}{2.843209in}}%
\pgfpathlineto{\pgfqpoint{2.497787in}{2.842549in}}%
\pgfpathlineto{\pgfqpoint{2.498197in}{2.842459in}}%
\pgfpathlineto{\pgfqpoint{2.499017in}{2.838864in}}%
\pgfpathlineto{\pgfqpoint{2.499427in}{2.840066in}}%
\pgfpathlineto{\pgfqpoint{2.499837in}{2.839196in}}%
\pgfpathlineto{\pgfqpoint{2.500247in}{2.840344in}}%
\pgfpathlineto{\pgfqpoint{2.500657in}{2.843488in}}%
\pgfpathlineto{\pgfqpoint{2.501477in}{2.842182in}}%
\pgfpathlineto{\pgfqpoint{2.501887in}{2.843046in}}%
\pgfpathlineto{\pgfqpoint{2.502707in}{2.838798in}}%
\pgfpathlineto{\pgfqpoint{2.503527in}{2.839649in}}%
\pgfpathlineto{\pgfqpoint{2.503937in}{2.839708in}}%
\pgfpathlineto{\pgfqpoint{2.504347in}{2.843323in}}%
\pgfpathlineto{\pgfqpoint{2.505167in}{2.842011in}}%
\pgfpathlineto{\pgfqpoint{2.505577in}{2.843298in}}%
\pgfpathlineto{\pgfqpoint{2.505987in}{2.841273in}}%
\pgfpathlineto{\pgfqpoint{2.506397in}{2.838934in}}%
\pgfpathlineto{\pgfqpoint{2.507217in}{2.839985in}}%
\pgfpathlineto{\pgfqpoint{2.507627in}{2.839519in}}%
\pgfpathlineto{\pgfqpoint{2.508037in}{2.843154in}}%
\pgfpathlineto{\pgfqpoint{2.508857in}{2.842036in}}%
\pgfpathlineto{\pgfqpoint{2.509267in}{2.843450in}}%
\pgfpathlineto{\pgfqpoint{2.509677in}{2.841688in}}%
\pgfpathlineto{\pgfqpoint{2.510087in}{2.839093in}}%
\pgfpathlineto{\pgfqpoint{2.510907in}{2.840181in}}%
\pgfpathlineto{\pgfqpoint{2.511317in}{2.839578in}}%
\pgfpathlineto{\pgfqpoint{2.512137in}{2.843336in}}%
\pgfpathlineto{\pgfqpoint{2.512547in}{2.842186in}}%
\pgfpathlineto{\pgfqpoint{2.512957in}{2.843624in}}%
\pgfpathlineto{\pgfqpoint{2.513367in}{2.841784in}}%
\pgfpathlineto{\pgfqpoint{2.513777in}{2.839179in}}%
\pgfpathlineto{\pgfqpoint{2.514597in}{2.840253in}}%
\pgfpathlineto{\pgfqpoint{2.515007in}{2.839824in}}%
\pgfpathlineto{\pgfqpoint{2.515417in}{2.843673in}}%
\pgfpathlineto{\pgfqpoint{2.516237in}{2.842459in}}%
\pgfpathlineto{\pgfqpoint{2.516647in}{2.843813in}}%
\pgfpathlineto{\pgfqpoint{2.517057in}{2.841539in}}%
\pgfpathlineto{\pgfqpoint{2.517467in}{2.839220in}}%
\pgfpathlineto{\pgfqpoint{2.518287in}{2.840178in}}%
\pgfpathlineto{\pgfqpoint{2.518697in}{2.840385in}}%
\pgfpathlineto{\pgfqpoint{2.519107in}{2.844315in}}%
\pgfpathlineto{\pgfqpoint{2.519927in}{2.842941in}}%
\pgfpathlineto{\pgfqpoint{2.520337in}{2.843865in}}%
\pgfpathlineto{\pgfqpoint{2.521157in}{2.839380in}}%
\pgfpathlineto{\pgfqpoint{2.521977in}{2.839964in}}%
\pgfpathlineto{\pgfqpoint{2.522797in}{2.844767in}}%
\pgfpathlineto{\pgfqpoint{2.524027in}{2.843470in}}%
\pgfpathlineto{\pgfqpoint{2.524847in}{2.839889in}}%
\pgfpathlineto{\pgfqpoint{2.525257in}{2.840971in}}%
\pgfpathlineto{\pgfqpoint{2.525667in}{2.839926in}}%
\pgfpathlineto{\pgfqpoint{2.526487in}{2.844440in}}%
\pgfpathlineto{\pgfqpoint{2.526897in}{2.842804in}}%
\pgfpathlineto{\pgfqpoint{2.527307in}{2.844353in}}%
\pgfpathlineto{\pgfqpoint{2.527717in}{2.842332in}}%
\pgfpathlineto{\pgfqpoint{2.528127in}{2.839594in}}%
\pgfpathlineto{\pgfqpoint{2.528947in}{2.840606in}}%
\pgfpathlineto{\pgfqpoint{2.529357in}{2.840992in}}%
\pgfpathlineto{\pgfqpoint{2.529767in}{2.845122in}}%
\pgfpathlineto{\pgfqpoint{2.530587in}{2.843755in}}%
\pgfpathlineto{\pgfqpoint{2.530997in}{2.844139in}}%
\pgfpathlineto{\pgfqpoint{2.531817in}{2.840000in}}%
\pgfpathlineto{\pgfqpoint{2.532227in}{2.841295in}}%
\pgfpathlineto{\pgfqpoint{2.532637in}{2.840211in}}%
\pgfpathlineto{\pgfqpoint{2.533457in}{2.844884in}}%
\pgfpathlineto{\pgfqpoint{2.534277in}{2.844760in}}%
\pgfpathlineto{\pgfqpoint{2.535097in}{2.839799in}}%
\pgfpathlineto{\pgfqpoint{2.535917in}{2.840699in}}%
\pgfpathlineto{\pgfqpoint{2.536737in}{2.845723in}}%
\pgfpathlineto{\pgfqpoint{2.537967in}{2.843833in}}%
\pgfpathlineto{\pgfqpoint{2.538377in}{2.840278in}}%
\pgfpathlineto{\pgfqpoint{2.539197in}{2.841339in}}%
\pgfpathlineto{\pgfqpoint{2.539607in}{2.841006in}}%
\pgfpathlineto{\pgfqpoint{2.540017in}{2.845609in}}%
\pgfpathlineto{\pgfqpoint{2.540837in}{2.844221in}}%
\pgfpathlineto{\pgfqpoint{2.541247in}{2.844738in}}%
\pgfpathlineto{\pgfqpoint{2.542067in}{2.840454in}}%
\pgfpathlineto{\pgfqpoint{2.542477in}{2.841721in}}%
\pgfpathlineto{\pgfqpoint{2.542887in}{2.840746in}}%
\pgfpathlineto{\pgfqpoint{2.543297in}{2.845101in}}%
\pgfpathlineto{\pgfqpoint{2.544117in}{2.844011in}}%
\pgfpathlineto{\pgfqpoint{2.544527in}{2.845232in}}%
\pgfpathlineto{\pgfqpoint{2.545347in}{2.840306in}}%
\pgfpathlineto{\pgfqpoint{2.546167in}{2.840788in}}%
\pgfpathlineto{\pgfqpoint{2.546987in}{2.845727in}}%
\pgfpathlineto{\pgfqpoint{2.547807in}{2.845502in}}%
\pgfpathlineto{\pgfqpoint{2.548627in}{2.840279in}}%
\pgfpathlineto{\pgfqpoint{2.549447in}{2.841033in}}%
\pgfpathlineto{\pgfqpoint{2.550267in}{2.846328in}}%
\pgfpathlineto{\pgfqpoint{2.551087in}{2.845590in}}%
\pgfpathlineto{\pgfqpoint{2.551907in}{2.840412in}}%
\pgfpathlineto{\pgfqpoint{2.552727in}{2.841411in}}%
\pgfpathlineto{\pgfqpoint{2.553547in}{2.846710in}}%
\pgfpathlineto{\pgfqpoint{2.554777in}{2.844382in}}%
\pgfpathlineto{\pgfqpoint{2.555187in}{2.840728in}}%
\pgfpathlineto{\pgfqpoint{2.556007in}{2.841836in}}%
\pgfpathlineto{\pgfqpoint{2.556417in}{2.842067in}}%
\pgfpathlineto{\pgfqpoint{2.556827in}{2.846787in}}%
\pgfpathlineto{\pgfqpoint{2.557647in}{2.845326in}}%
\pgfpathlineto{\pgfqpoint{2.558057in}{2.845149in}}%
\pgfpathlineto{\pgfqpoint{2.558467in}{2.841232in}}%
\pgfpathlineto{\pgfqpoint{2.559287in}{2.842225in}}%
\pgfpathlineto{\pgfqpoint{2.559697in}{2.841618in}}%
\pgfpathlineto{\pgfqpoint{2.560107in}{2.846527in}}%
\pgfpathlineto{\pgfqpoint{2.560927in}{2.845084in}}%
\pgfpathlineto{\pgfqpoint{2.561337in}{2.845780in}}%
\pgfpathlineto{\pgfqpoint{2.562157in}{2.841117in}}%
\pgfpathlineto{\pgfqpoint{2.562567in}{2.842512in}}%
\pgfpathlineto{\pgfqpoint{2.562977in}{2.841440in}}%
\pgfpathlineto{\pgfqpoint{2.563797in}{2.846065in}}%
\pgfpathlineto{\pgfqpoint{2.564207in}{2.844870in}}%
\pgfpathlineto{\pgfqpoint{2.564617in}{2.846237in}}%
\pgfpathlineto{\pgfqpoint{2.565437in}{2.840978in}}%
\pgfpathlineto{\pgfqpoint{2.566257in}{2.841523in}}%
\pgfpathlineto{\pgfqpoint{2.567077in}{2.846819in}}%
\pgfpathlineto{\pgfqpoint{2.567897in}{2.846496in}}%
\pgfpathlineto{\pgfqpoint{2.568717in}{2.840977in}}%
\pgfpathlineto{\pgfqpoint{2.569537in}{2.841817in}}%
\pgfpathlineto{\pgfqpoint{2.570357in}{2.847436in}}%
\pgfpathlineto{\pgfqpoint{2.571177in}{2.846561in}}%
\pgfpathlineto{\pgfqpoint{2.571997in}{2.841153in}}%
\pgfpathlineto{\pgfqpoint{2.572817in}{2.842237in}}%
\pgfpathlineto{\pgfqpoint{2.573637in}{2.847795in}}%
\pgfpathlineto{\pgfqpoint{2.574867in}{2.845452in}}%
\pgfpathlineto{\pgfqpoint{2.575277in}{2.841529in}}%
\pgfpathlineto{\pgfqpoint{2.576097in}{2.842691in}}%
\pgfpathlineto{\pgfqpoint{2.576507in}{2.842727in}}%
\pgfpathlineto{\pgfqpoint{2.576917in}{2.847812in}}%
\pgfpathlineto{\pgfqpoint{2.577737in}{2.846238in}}%
\pgfpathlineto{\pgfqpoint{2.578147in}{2.846247in}}%
\pgfpathlineto{\pgfqpoint{2.578967in}{2.842024in}}%
\pgfpathlineto{\pgfqpoint{2.579377in}{2.843089in}}%
\pgfpathlineto{\pgfqpoint{2.579787in}{2.842305in}}%
\pgfpathlineto{\pgfqpoint{2.580197in}{2.847460in}}%
\pgfpathlineto{\pgfqpoint{2.581017in}{2.845979in}}%
\pgfpathlineto{\pgfqpoint{2.581427in}{2.846887in}}%
\pgfpathlineto{\pgfqpoint{2.582247in}{2.841802in}}%
\pgfpathlineto{\pgfqpoint{2.582657in}{2.843365in}}%
\pgfpathlineto{\pgfqpoint{2.583067in}{2.842180in}}%
\pgfpathlineto{\pgfqpoint{2.583887in}{2.847251in}}%
\pgfpathlineto{\pgfqpoint{2.584297in}{2.845767in}}%
\pgfpathlineto{\pgfqpoint{2.584707in}{2.847330in}}%
\pgfpathlineto{\pgfqpoint{2.585527in}{2.841685in}}%
\pgfpathlineto{\pgfqpoint{2.586347in}{2.842331in}}%
\pgfpathlineto{\pgfqpoint{2.587167in}{2.848040in}}%
\pgfpathlineto{\pgfqpoint{2.587987in}{2.847560in}}%
\pgfpathlineto{\pgfqpoint{2.588807in}{2.841728in}}%
\pgfpathlineto{\pgfqpoint{2.589627in}{2.842689in}}%
\pgfpathlineto{\pgfqpoint{2.590447in}{2.848649in}}%
\pgfpathlineto{\pgfqpoint{2.591267in}{2.847582in}}%
\pgfpathlineto{\pgfqpoint{2.592087in}{2.841972in}}%
\pgfpathlineto{\pgfqpoint{2.592907in}{2.843160in}}%
\pgfpathlineto{\pgfqpoint{2.593317in}{2.844045in}}%
\pgfpathlineto{\pgfqpoint{2.593727in}{2.848948in}}%
\pgfpathlineto{\pgfqpoint{2.594547in}{2.847434in}}%
\pgfpathlineto{\pgfqpoint{2.594957in}{2.846667in}}%
\pgfpathlineto{\pgfqpoint{2.595367in}{2.842435in}}%
\pgfpathlineto{\pgfqpoint{2.596187in}{2.843637in}}%
\pgfpathlineto{\pgfqpoint{2.596597in}{2.843384in}}%
\pgfpathlineto{\pgfqpoint{2.597007in}{2.848858in}}%
\pgfpathlineto{\pgfqpoint{2.597827in}{2.847179in}}%
\pgfpathlineto{\pgfqpoint{2.598237in}{2.847477in}}%
\pgfpathlineto{\pgfqpoint{2.599057in}{2.842722in}}%
\pgfpathlineto{\pgfqpoint{2.599467in}{2.844030in}}%
\pgfpathlineto{\pgfqpoint{2.599877in}{2.843025in}}%
\pgfpathlineto{\pgfqpoint{2.600287in}{2.848373in}}%
\pgfpathlineto{\pgfqpoint{2.601107in}{2.846905in}}%
\pgfpathlineto{\pgfqpoint{2.601517in}{2.848105in}}%
\pgfpathlineto{\pgfqpoint{2.602337in}{2.842513in}}%
\pgfpathlineto{\pgfqpoint{2.602747in}{2.844274in}}%
\pgfpathlineto{\pgfqpoint{2.603157in}{2.842988in}}%
\pgfpathlineto{\pgfqpoint{2.603977in}{2.848594in}}%
\pgfpathlineto{\pgfqpoint{2.604387in}{2.846712in}}%
\pgfpathlineto{\pgfqpoint{2.604797in}{2.848512in}}%
\pgfpathlineto{\pgfqpoint{2.605617in}{2.842438in}}%
\pgfpathlineto{\pgfqpoint{2.606437in}{2.843233in}}%
\pgfpathlineto{\pgfqpoint{2.607257in}{2.849400in}}%
\pgfpathlineto{\pgfqpoint{2.608077in}{2.848686in}}%
\pgfpathlineto{\pgfqpoint{2.608897in}{2.842555in}}%
\pgfpathlineto{\pgfqpoint{2.609717in}{2.843671in}}%
\pgfpathlineto{\pgfqpoint{2.610537in}{2.849960in}}%
\pgfpathlineto{\pgfqpoint{2.611357in}{2.848645in}}%
\pgfpathlineto{\pgfqpoint{2.612177in}{2.842898in}}%
\pgfpathlineto{\pgfqpoint{2.613407in}{2.844653in}}%
\pgfpathlineto{\pgfqpoint{2.613817in}{2.850146in}}%
\pgfpathlineto{\pgfqpoint{2.614637in}{2.848438in}}%
\pgfpathlineto{\pgfqpoint{2.615047in}{2.848042in}}%
\pgfpathlineto{\pgfqpoint{2.615457in}{2.843480in}}%
\pgfpathlineto{\pgfqpoint{2.616277in}{2.844678in}}%
\pgfpathlineto{\pgfqpoint{2.616687in}{2.844056in}}%
\pgfpathlineto{\pgfqpoint{2.617097in}{2.849893in}}%
\pgfpathlineto{\pgfqpoint{2.617917in}{2.848145in}}%
\pgfpathlineto{\pgfqpoint{2.618327in}{2.848845in}}%
\pgfpathlineto{\pgfqpoint{2.619147in}{2.843438in}}%
\pgfpathlineto{\pgfqpoint{2.619557in}{2.845042in}}%
\pgfpathlineto{\pgfqpoint{2.619967in}{2.843805in}}%
\pgfpathlineto{\pgfqpoint{2.620377in}{2.849225in}}%
\pgfpathlineto{\pgfqpoint{2.621197in}{2.847870in}}%
\pgfpathlineto{\pgfqpoint{2.621607in}{2.849433in}}%
\pgfpathlineto{\pgfqpoint{2.622427in}{2.843265in}}%
\pgfpathlineto{\pgfqpoint{2.623247in}{2.843895in}}%
\pgfpathlineto{\pgfqpoint{2.624067in}{2.850112in}}%
\pgfpathlineto{\pgfqpoint{2.624887in}{2.849772in}}%
\pgfpathlineto{\pgfqpoint{2.625707in}{2.843264in}}%
\pgfpathlineto{\pgfqpoint{2.626527in}{2.844259in}}%
\pgfpathlineto{\pgfqpoint{2.627347in}{2.850895in}}%
\pgfpathlineto{\pgfqpoint{2.628167in}{2.849861in}}%
\pgfpathlineto{\pgfqpoint{2.628987in}{2.843491in}}%
\pgfpathlineto{\pgfqpoint{2.629807in}{2.844782in}}%
\pgfpathlineto{\pgfqpoint{2.630217in}{2.846078in}}%
\pgfpathlineto{\pgfqpoint{2.630627in}{2.851345in}}%
\pgfpathlineto{\pgfqpoint{2.631447in}{2.849734in}}%
\pgfpathlineto{\pgfqpoint{2.631857in}{2.848617in}}%
\pgfpathlineto{\pgfqpoint{2.632267in}{2.843974in}}%
\pgfpathlineto{\pgfqpoint{2.633087in}{2.845337in}}%
\pgfpathlineto{\pgfqpoint{2.633497in}{2.845254in}}%
\pgfpathlineto{\pgfqpoint{2.633907in}{2.851343in}}%
\pgfpathlineto{\pgfqpoint{2.634727in}{2.849459in}}%
\pgfpathlineto{\pgfqpoint{2.635137in}{2.849583in}}%
\pgfpathlineto{\pgfqpoint{2.635957in}{2.844438in}}%
\pgfpathlineto{\pgfqpoint{2.636367in}{2.845808in}}%
\pgfpathlineto{\pgfqpoint{2.636777in}{2.844778in}}%
\pgfpathlineto{\pgfqpoint{2.637187in}{2.850859in}}%
\pgfpathlineto{\pgfqpoint{2.638007in}{2.849137in}}%
\pgfpathlineto{\pgfqpoint{2.638417in}{2.850346in}}%
\pgfpathlineto{\pgfqpoint{2.639237in}{2.844186in}}%
\pgfpathlineto{\pgfqpoint{2.639647in}{2.846110in}}%
\pgfpathlineto{\pgfqpoint{2.640057in}{2.844689in}}%
\pgfpathlineto{\pgfqpoint{2.640877in}{2.850844in}}%
\pgfpathlineto{\pgfqpoint{2.641287in}{2.848889in}}%
\pgfpathlineto{\pgfqpoint{2.641697in}{2.850855in}}%
\pgfpathlineto{\pgfqpoint{2.642517in}{2.844084in}}%
\pgfpathlineto{\pgfqpoint{2.643337in}{2.844945in}}%
\pgfpathlineto{\pgfqpoint{2.644157in}{2.851808in}}%
\pgfpathlineto{\pgfqpoint{2.644977in}{2.851087in}}%
\pgfpathlineto{\pgfqpoint{2.645797in}{2.844204in}}%
\pgfpathlineto{\pgfqpoint{2.646617in}{2.845440in}}%
\pgfpathlineto{\pgfqpoint{2.647437in}{2.852497in}}%
\pgfpathlineto{\pgfqpoint{2.648257in}{2.851060in}}%
\pgfpathlineto{\pgfqpoint{2.649077in}{2.844591in}}%
\pgfpathlineto{\pgfqpoint{2.650307in}{2.846587in}}%
\pgfpathlineto{\pgfqpoint{2.650717in}{2.852744in}}%
\pgfpathlineto{\pgfqpoint{2.651537in}{2.850830in}}%
\pgfpathlineto{\pgfqpoint{2.651947in}{2.850355in}}%
\pgfpathlineto{\pgfqpoint{2.652357in}{2.845260in}}%
\pgfpathlineto{\pgfqpoint{2.653177in}{2.846598in}}%
\pgfpathlineto{\pgfqpoint{2.653587in}{2.845891in}}%
\pgfpathlineto{\pgfqpoint{2.653997in}{2.852464in}}%
\pgfpathlineto{\pgfqpoint{2.654817in}{2.850491in}}%
\pgfpathlineto{\pgfqpoint{2.655227in}{2.851285in}}%
\pgfpathlineto{\pgfqpoint{2.656047in}{2.845184in}}%
\pgfpathlineto{\pgfqpoint{2.656457in}{2.847009in}}%
\pgfpathlineto{\pgfqpoint{2.656867in}{2.845612in}}%
\pgfpathlineto{\pgfqpoint{2.657277in}{2.851680in}}%
\pgfpathlineto{\pgfqpoint{2.658097in}{2.850169in}}%
\pgfpathlineto{\pgfqpoint{2.658507in}{2.851960in}}%
\pgfpathlineto{\pgfqpoint{2.659327in}{2.844997in}}%
\pgfpathlineto{\pgfqpoint{2.660147in}{2.845741in}}%
\pgfpathlineto{\pgfqpoint{2.660967in}{2.852755in}}%
\pgfpathlineto{\pgfqpoint{2.661787in}{2.852340in}}%
\pgfpathlineto{\pgfqpoint{2.662607in}{2.845022in}}%
\pgfpathlineto{\pgfqpoint{2.663427in}{2.846186in}}%
\pgfpathlineto{\pgfqpoint{2.664247in}{2.853654in}}%
\pgfpathlineto{\pgfqpoint{2.665067in}{2.852423in}}%
\pgfpathlineto{\pgfqpoint{2.665887in}{2.845323in}}%
\pgfpathlineto{\pgfqpoint{2.666707in}{2.846804in}}%
\pgfpathlineto{\pgfqpoint{2.667117in}{2.848019in}}%
\pgfpathlineto{\pgfqpoint{2.667527in}{2.854133in}}%
\pgfpathlineto{\pgfqpoint{2.668347in}{2.852252in}}%
\pgfpathlineto{\pgfqpoint{2.668757in}{2.851200in}}%
\pgfpathlineto{\pgfqpoint{2.669167in}{2.845930in}}%
\pgfpathlineto{\pgfqpoint{2.669987in}{2.847437in}}%
\pgfpathlineto{\pgfqpoint{2.670397in}{2.847115in}}%
\pgfpathlineto{\pgfqpoint{2.670807in}{2.854054in}}%
\pgfpathlineto{\pgfqpoint{2.671627in}{2.851916in}}%
\pgfpathlineto{\pgfqpoint{2.672037in}{2.852287in}}%
\pgfpathlineto{\pgfqpoint{2.672857in}{2.846242in}}%
\pgfpathlineto{\pgfqpoint{2.673267in}{2.847946in}}%
\pgfpathlineto{\pgfqpoint{2.673677in}{2.846648in}}%
\pgfpathlineto{\pgfqpoint{2.674087in}{2.853395in}}%
\pgfpathlineto{\pgfqpoint{2.674907in}{2.851543in}}%
\pgfpathlineto{\pgfqpoint{2.675317in}{2.853121in}}%
\pgfpathlineto{\pgfqpoint{2.676137in}{2.845986in}}%
\pgfpathlineto{\pgfqpoint{2.676547in}{2.848237in}}%
\pgfpathlineto{\pgfqpoint{2.676957in}{2.846644in}}%
\pgfpathlineto{\pgfqpoint{2.677777in}{2.853787in}}%
\pgfpathlineto{\pgfqpoint{2.678187in}{2.851286in}}%
\pgfpathlineto{\pgfqpoint{2.678597in}{2.853643in}}%
\pgfpathlineto{\pgfqpoint{2.679417in}{2.845933in}}%
\pgfpathlineto{\pgfqpoint{2.680237in}{2.847029in}}%
\pgfpathlineto{\pgfqpoint{2.681057in}{2.854867in}}%
\pgfpathlineto{\pgfqpoint{2.681877in}{2.853836in}}%
\pgfpathlineto{\pgfqpoint{2.682697in}{2.846162in}}%
\pgfpathlineto{\pgfqpoint{2.683517in}{2.847657in}}%
\pgfpathlineto{\pgfqpoint{2.684337in}{2.855555in}}%
\pgfpathlineto{\pgfqpoint{2.685567in}{2.852155in}}%
\pgfpathlineto{\pgfqpoint{2.685977in}{2.846721in}}%
\pgfpathlineto{\pgfqpoint{2.686797in}{2.848351in}}%
\pgfpathlineto{\pgfqpoint{2.687207in}{2.848417in}}%
\pgfpathlineto{\pgfqpoint{2.687617in}{2.855661in}}%
\pgfpathlineto{\pgfqpoint{2.688437in}{2.853406in}}%
\pgfpathlineto{\pgfqpoint{2.688847in}{2.853390in}}%
\pgfpathlineto{\pgfqpoint{2.689667in}{2.847350in}}%
\pgfpathlineto{\pgfqpoint{2.690077in}{2.848946in}}%
\pgfpathlineto{\pgfqpoint{2.690487in}{2.847775in}}%
\pgfpathlineto{\pgfqpoint{2.690897in}{2.855117in}}%
\pgfpathlineto{\pgfqpoint{2.691717in}{2.852996in}}%
\pgfpathlineto{\pgfqpoint{2.692127in}{2.854372in}}%
\pgfpathlineto{\pgfqpoint{2.692947in}{2.847037in}}%
\pgfpathlineto{\pgfqpoint{2.693357in}{2.849323in}}%
\pgfpathlineto{\pgfqpoint{2.693767in}{2.847648in}}%
\pgfpathlineto{\pgfqpoint{2.694587in}{2.854950in}}%
\pgfpathlineto{\pgfqpoint{2.694997in}{2.852665in}}%
\pgfpathlineto{\pgfqpoint{2.695407in}{2.855027in}}%
\pgfpathlineto{\pgfqpoint{2.696227in}{2.846922in}}%
\pgfpathlineto{\pgfqpoint{2.697047in}{2.847978in}}%
\pgfpathlineto{\pgfqpoint{2.697867in}{2.856191in}}%
\pgfpathlineto{\pgfqpoint{2.698687in}{2.855320in}}%
\pgfpathlineto{\pgfqpoint{2.699507in}{2.847100in}}%
\pgfpathlineto{\pgfqpoint{2.700327in}{2.848615in}}%
\pgfpathlineto{\pgfqpoint{2.701147in}{2.857060in}}%
\pgfpathlineto{\pgfqpoint{2.701967in}{2.855272in}}%
\pgfpathlineto{\pgfqpoint{2.702787in}{2.847631in}}%
\pgfpathlineto{\pgfqpoint{2.704017in}{2.849759in}}%
\pgfpathlineto{\pgfqpoint{2.704427in}{2.857320in}}%
\pgfpathlineto{\pgfqpoint{2.705247in}{2.854960in}}%
\pgfpathlineto{\pgfqpoint{2.705657in}{2.854637in}}%
\pgfpathlineto{\pgfqpoint{2.706477in}{2.848497in}}%
\pgfpathlineto{\pgfqpoint{2.706887in}{2.850036in}}%
\pgfpathlineto{\pgfqpoint{2.707297in}{2.848965in}}%
\pgfpathlineto{\pgfqpoint{2.707706in}{2.856862in}}%
\pgfpathlineto{\pgfqpoint{2.708526in}{2.854519in}}%
\pgfpathlineto{\pgfqpoint{2.708936in}{2.855755in}}%
\pgfpathlineto{\pgfqpoint{2.709756in}{2.848139in}}%
\pgfpathlineto{\pgfqpoint{2.710166in}{2.850484in}}%
\pgfpathlineto{\pgfqpoint{2.710576in}{2.848738in}}%
\pgfpathlineto{\pgfqpoint{2.711396in}{2.856287in}}%
\pgfpathlineto{\pgfqpoint{2.711806in}{2.854129in}}%
\pgfpathlineto{\pgfqpoint{2.712216in}{2.856525in}}%
\pgfpathlineto{\pgfqpoint{2.713036in}{2.847980in}}%
\pgfpathlineto{\pgfqpoint{2.713856in}{2.849033in}}%
\pgfpathlineto{\pgfqpoint{2.714676in}{2.857675in}}%
\pgfpathlineto{\pgfqpoint{2.715496in}{2.856901in}}%
\pgfpathlineto{\pgfqpoint{2.716316in}{2.848130in}}%
\pgfpathlineto{\pgfqpoint{2.717136in}{2.849694in}}%
\pgfpathlineto{\pgfqpoint{2.717956in}{2.858696in}}%
\pgfpathlineto{\pgfqpoint{2.718776in}{2.856893in}}%
\pgfpathlineto{\pgfqpoint{2.719596in}{2.848665in}}%
\pgfpathlineto{\pgfqpoint{2.720826in}{2.851106in}}%
\pgfpathlineto{\pgfqpoint{2.721236in}{2.859068in}}%
\pgfpathlineto{\pgfqpoint{2.722056in}{2.856580in}}%
\pgfpathlineto{\pgfqpoint{2.722466in}{2.856069in}}%
\pgfpathlineto{\pgfqpoint{2.722876in}{2.849606in}}%
\pgfpathlineto{\pgfqpoint{2.723696in}{2.851240in}}%
\pgfpathlineto{\pgfqpoint{2.724106in}{2.850192in}}%
\pgfpathlineto{\pgfqpoint{2.724516in}{2.858643in}}%
\pgfpathlineto{\pgfqpoint{2.725336in}{2.856105in}}%
\pgfpathlineto{\pgfqpoint{2.725746in}{2.857308in}}%
\pgfpathlineto{\pgfqpoint{2.726566in}{2.849283in}}%
\pgfpathlineto{\pgfqpoint{2.726976in}{2.851740in}}%
\pgfpathlineto{\pgfqpoint{2.727386in}{2.849904in}}%
\pgfpathlineto{\pgfqpoint{2.728206in}{2.857841in}}%
\pgfpathlineto{\pgfqpoint{2.728616in}{2.855666in}}%
\pgfpathlineto{\pgfqpoint{2.729026in}{2.858171in}}%
\pgfpathlineto{\pgfqpoint{2.729846in}{2.849100in}}%
\pgfpathlineto{\pgfqpoint{2.730666in}{2.850202in}}%
\pgfpathlineto{\pgfqpoint{2.731486in}{2.859370in}}%
\pgfpathlineto{\pgfqpoint{2.732306in}{2.858600in}}%
\pgfpathlineto{\pgfqpoint{2.733126in}{2.849256in}}%
\pgfpathlineto{\pgfqpoint{2.733946in}{2.850911in}}%
\pgfpathlineto{\pgfqpoint{2.734766in}{2.860509in}}%
\pgfpathlineto{\pgfqpoint{2.735586in}{2.858604in}}%
\pgfpathlineto{\pgfqpoint{2.736406in}{2.849833in}}%
\pgfpathlineto{\pgfqpoint{2.737636in}{2.852423in}}%
\pgfpathlineto{\pgfqpoint{2.738046in}{2.860931in}}%
\pgfpathlineto{\pgfqpoint{2.738866in}{2.858267in}}%
\pgfpathlineto{\pgfqpoint{2.739276in}{2.857733in}}%
\pgfpathlineto{\pgfqpoint{2.739686in}{2.850858in}}%
\pgfpathlineto{\pgfqpoint{2.740506in}{2.852581in}}%
\pgfpathlineto{\pgfqpoint{2.740916in}{2.851434in}}%
\pgfpathlineto{\pgfqpoint{2.741326in}{2.860464in}}%
\pgfpathlineto{\pgfqpoint{2.742146in}{2.857748in}}%
\pgfpathlineto{\pgfqpoint{2.742556in}{2.859072in}}%
\pgfpathlineto{\pgfqpoint{2.743376in}{2.850464in}}%
\pgfpathlineto{\pgfqpoint{2.743786in}{2.853108in}}%
\pgfpathlineto{\pgfqpoint{2.744196in}{2.851144in}}%
\pgfpathlineto{\pgfqpoint{2.745016in}{2.859660in}}%
\pgfpathlineto{\pgfqpoint{2.745426in}{2.857271in}}%
\pgfpathlineto{\pgfqpoint{2.745836in}{2.859994in}}%
\pgfpathlineto{\pgfqpoint{2.746656in}{2.850285in}}%
\pgfpathlineto{\pgfqpoint{2.747476in}{2.851498in}}%
\pgfpathlineto{\pgfqpoint{2.748296in}{2.861326in}}%
\pgfpathlineto{\pgfqpoint{2.749116in}{2.860437in}}%
\pgfpathlineto{\pgfqpoint{2.749936in}{2.850488in}}%
\pgfpathlineto{\pgfqpoint{2.750756in}{2.852292in}}%
\pgfpathlineto{\pgfqpoint{2.751576in}{2.862535in}}%
\pgfpathlineto{\pgfqpoint{2.752396in}{2.860412in}}%
\pgfpathlineto{\pgfqpoint{2.753216in}{2.851161in}}%
\pgfpathlineto{\pgfqpoint{2.754446in}{2.853682in}}%
\pgfpathlineto{\pgfqpoint{2.754856in}{2.862920in}}%
\pgfpathlineto{\pgfqpoint{2.755676in}{2.860017in}}%
\pgfpathlineto{\pgfqpoint{2.756086in}{2.859674in}}%
\pgfpathlineto{\pgfqpoint{2.756906in}{2.852109in}}%
\pgfpathlineto{\pgfqpoint{2.757316in}{2.854079in}}%
\pgfpathlineto{\pgfqpoint{2.757726in}{2.852686in}}%
\pgfpathlineto{\pgfqpoint{2.758136in}{2.862304in}}%
\pgfpathlineto{\pgfqpoint{2.758956in}{2.859441in}}%
\pgfpathlineto{\pgfqpoint{2.759366in}{2.861082in}}%
\pgfpathlineto{\pgfqpoint{2.760186in}{2.851682in}}%
\pgfpathlineto{\pgfqpoint{2.760596in}{2.854597in}}%
\pgfpathlineto{\pgfqpoint{2.761006in}{2.852473in}}%
\pgfpathlineto{\pgfqpoint{2.761826in}{2.861806in}}%
\pgfpathlineto{\pgfqpoint{2.762236in}{2.858941in}}%
\pgfpathlineto{\pgfqpoint{2.762646in}{2.862017in}}%
\pgfpathlineto{\pgfqpoint{2.763466in}{2.851547in}}%
\pgfpathlineto{\pgfqpoint{2.764286in}{2.852954in}}%
\pgfpathlineto{\pgfqpoint{2.765106in}{2.863589in}}%
\pgfpathlineto{\pgfqpoint{2.765926in}{2.862416in}}%
\pgfpathlineto{\pgfqpoint{2.766746in}{2.851856in}}%
\pgfpathlineto{\pgfqpoint{2.767566in}{2.853873in}}%
\pgfpathlineto{\pgfqpoint{2.768386in}{2.864790in}}%
\pgfpathlineto{\pgfqpoint{2.769616in}{2.860088in}}%
\pgfpathlineto{\pgfqpoint{2.770026in}{2.852697in}}%
\pgfpathlineto{\pgfqpoint{2.770846in}{2.854904in}}%
\pgfpathlineto{\pgfqpoint{2.771256in}{2.854869in}}%
\pgfpathlineto{\pgfqpoint{2.771666in}{2.865011in}}%
\pgfpathlineto{\pgfqpoint{2.772486in}{2.861818in}}%
\pgfpathlineto{\pgfqpoint{2.772896in}{2.861934in}}%
\pgfpathlineto{\pgfqpoint{2.773716in}{2.853349in}}%
\pgfpathlineto{\pgfqpoint{2.774126in}{2.855746in}}%
\pgfpathlineto{\pgfqpoint{2.774536in}{2.853967in}}%
\pgfpathlineto{\pgfqpoint{2.774946in}{2.864101in}}%
\pgfpathlineto{\pgfqpoint{2.775766in}{2.861177in}}%
\pgfpathlineto{\pgfqpoint{2.776176in}{2.863362in}}%
\pgfpathlineto{\pgfqpoint{2.776996in}{2.852947in}}%
\pgfpathlineto{\pgfqpoint{2.777406in}{2.856198in}}%
\pgfpathlineto{\pgfqpoint{2.777816in}{2.853938in}}%
\pgfpathlineto{\pgfqpoint{2.778636in}{2.864344in}}%
\pgfpathlineto{\pgfqpoint{2.779046in}{2.860690in}}%
\pgfpathlineto{\pgfqpoint{2.779456in}{2.864242in}}%
\pgfpathlineto{\pgfqpoint{2.779866in}{2.858126in}}%
\pgfpathlineto{\pgfqpoint{2.780276in}{2.852919in}}%
\pgfpathlineto{\pgfqpoint{2.781096in}{2.854623in}}%
\pgfpathlineto{\pgfqpoint{2.781916in}{2.866194in}}%
\pgfpathlineto{\pgfqpoint{2.782736in}{2.864525in}}%
\pgfpathlineto{\pgfqpoint{2.783556in}{2.853419in}}%
\pgfpathlineto{\pgfqpoint{2.784376in}{2.855697in}}%
\pgfpathlineto{\pgfqpoint{2.784786in}{2.857619in}}%
\pgfpathlineto{\pgfqpoint{2.785196in}{2.867251in}}%
\pgfpathlineto{\pgfqpoint{2.786016in}{2.864268in}}%
\pgfpathlineto{\pgfqpoint{2.786426in}{2.862658in}}%
\pgfpathlineto{\pgfqpoint{2.786836in}{2.854523in}}%
\pgfpathlineto{\pgfqpoint{2.787656in}{2.856785in}}%
\pgfpathlineto{\pgfqpoint{2.788066in}{2.856006in}}%
\pgfpathlineto{\pgfqpoint{2.788476in}{2.867120in}}%
\pgfpathlineto{\pgfqpoint{2.789296in}{2.863647in}}%
\pgfpathlineto{\pgfqpoint{2.789706in}{2.864547in}}%
\pgfpathlineto{\pgfqpoint{2.790526in}{2.854608in}}%
\pgfpathlineto{\pgfqpoint{2.790936in}{2.857572in}}%
\pgfpathlineto{\pgfqpoint{2.791346in}{2.855340in}}%
\pgfpathlineto{\pgfqpoint{2.791756in}{2.865736in}}%
\pgfpathlineto{\pgfqpoint{2.792576in}{2.862955in}}%
\pgfpathlineto{\pgfqpoint{2.792986in}{2.865916in}}%
\pgfpathlineto{\pgfqpoint{2.793806in}{2.854295in}}%
\pgfpathlineto{\pgfqpoint{2.794626in}{2.855621in}}%
\pgfpathlineto{\pgfqpoint{2.795446in}{2.867340in}}%
\pgfpathlineto{\pgfqpoint{2.796266in}{2.866645in}}%
\pgfpathlineto{\pgfqpoint{2.797086in}{2.854471in}}%
\pgfpathlineto{\pgfqpoint{2.797906in}{2.856580in}}%
\pgfpathlineto{\pgfqpoint{2.798726in}{2.869124in}}%
\pgfpathlineto{\pgfqpoint{2.799546in}{2.866717in}}%
\pgfpathlineto{\pgfqpoint{2.800366in}{2.855279in}}%
\pgfpathlineto{\pgfqpoint{2.801596in}{2.858529in}}%
\pgfpathlineto{\pgfqpoint{2.802006in}{2.869810in}}%
\pgfpathlineto{\pgfqpoint{2.802826in}{2.866244in}}%
\pgfpathlineto{\pgfqpoint{2.803236in}{2.865671in}}%
\pgfpathlineto{\pgfqpoint{2.804056in}{2.856451in}}%
\pgfpathlineto{\pgfqpoint{2.804466in}{2.858881in}}%
\pgfpathlineto{\pgfqpoint{2.804876in}{2.857178in}}%
\pgfpathlineto{\pgfqpoint{2.805286in}{2.869071in}}%
\pgfpathlineto{\pgfqpoint{2.806106in}{2.865480in}}%
\pgfpathlineto{\pgfqpoint{2.806516in}{2.867513in}}%
\pgfpathlineto{\pgfqpoint{2.807336in}{2.855917in}}%
\pgfpathlineto{\pgfqpoint{2.807746in}{2.859510in}}%
\pgfpathlineto{\pgfqpoint{2.808156in}{2.856936in}}%
\pgfpathlineto{\pgfqpoint{2.808976in}{2.868451in}}%
\pgfpathlineto{\pgfqpoint{2.809386in}{2.864803in}}%
\pgfpathlineto{\pgfqpoint{2.809796in}{2.868705in}}%
\pgfpathlineto{\pgfqpoint{2.810616in}{2.855807in}}%
\pgfpathlineto{\pgfqpoint{2.811436in}{2.857648in}}%
\pgfpathlineto{\pgfqpoint{2.812256in}{2.870807in}}%
\pgfpathlineto{\pgfqpoint{2.813076in}{2.869156in}}%
\pgfpathlineto{\pgfqpoint{2.813896in}{2.856334in}}%
\pgfpathlineto{\pgfqpoint{2.814716in}{2.858902in}}%
\pgfpathlineto{\pgfqpoint{2.815126in}{2.861414in}}%
\pgfpathlineto{\pgfqpoint{2.815536in}{2.872257in}}%
\pgfpathlineto{\pgfqpoint{2.816356in}{2.868912in}}%
\pgfpathlineto{\pgfqpoint{2.816766in}{2.866848in}}%
\pgfpathlineto{\pgfqpoint{2.817176in}{2.857608in}}%
\pgfpathlineto{\pgfqpoint{2.817996in}{2.860203in}}%
\pgfpathlineto{\pgfqpoint{2.818406in}{2.859395in}}%
\pgfpathlineto{\pgfqpoint{2.818816in}{2.872223in}}%
\pgfpathlineto{\pgfqpoint{2.819636in}{2.868179in}}%
\pgfpathlineto{\pgfqpoint{2.820046in}{2.869131in}}%
\pgfpathlineto{\pgfqpoint{2.820866in}{2.857719in}}%
\pgfpathlineto{\pgfqpoint{2.821276in}{2.861132in}}%
\pgfpathlineto{\pgfqpoint{2.821686in}{2.858576in}}%
\pgfpathlineto{\pgfqpoint{2.822096in}{2.870572in}}%
\pgfpathlineto{\pgfqpoint{2.822916in}{2.867325in}}%
\pgfpathlineto{\pgfqpoint{2.823326in}{2.870776in}}%
\pgfpathlineto{\pgfqpoint{2.824146in}{2.857371in}}%
\pgfpathlineto{\pgfqpoint{2.824966in}{2.858953in}}%
\pgfpathlineto{\pgfqpoint{2.825786in}{2.872493in}}%
\pgfpathlineto{\pgfqpoint{2.826606in}{2.871621in}}%
\pgfpathlineto{\pgfqpoint{2.827426in}{2.857650in}}%
\pgfpathlineto{\pgfqpoint{2.828246in}{2.860154in}}%
\pgfpathlineto{\pgfqpoint{2.829066in}{2.874611in}}%
\pgfpathlineto{\pgfqpoint{2.829886in}{2.871644in}}%
\pgfpathlineto{\pgfqpoint{2.830706in}{2.858731in}}%
\pgfpathlineto{\pgfqpoint{2.831936in}{2.861886in}}%
\pgfpathlineto{\pgfqpoint{2.832346in}{2.875260in}}%
\pgfpathlineto{\pgfqpoint{2.833166in}{2.871000in}}%
\pgfpathlineto{\pgfqpoint{2.833576in}{2.870884in}}%
\pgfpathlineto{\pgfqpoint{2.834396in}{2.859648in}}%
\pgfpathlineto{\pgfqpoint{2.834806in}{2.862813in}}%
\pgfpathlineto{\pgfqpoint{2.835216in}{2.860490in}}%
\pgfpathlineto{\pgfqpoint{2.835626in}{2.874052in}}%
\pgfpathlineto{\pgfqpoint{2.836446in}{2.870043in}}%
\pgfpathlineto{\pgfqpoint{2.836856in}{2.872958in}}%
\pgfpathlineto{\pgfqpoint{2.837676in}{2.859108in}}%
\pgfpathlineto{\pgfqpoint{2.838496in}{2.860504in}}%
\pgfpathlineto{\pgfqpoint{2.839316in}{2.874360in}}%
\pgfpathlineto{\pgfqpoint{2.839726in}{2.869292in}}%
\pgfpathlineto{\pgfqpoint{2.840136in}{2.874182in}}%
\pgfpathlineto{\pgfqpoint{2.840546in}{2.866488in}}%
\pgfpathlineto{\pgfqpoint{2.840956in}{2.859185in}}%
\pgfpathlineto{\pgfqpoint{2.841776in}{2.861613in}}%
\pgfpathlineto{\pgfqpoint{2.842596in}{2.877047in}}%
\pgfpathlineto{\pgfqpoint{2.843416in}{2.874469in}}%
\pgfpathlineto{\pgfqpoint{2.844236in}{2.860120in}}%
\pgfpathlineto{\pgfqpoint{2.845056in}{2.863203in}}%
\pgfpathlineto{\pgfqpoint{2.845466in}{2.864521in}}%
\pgfpathlineto{\pgfqpoint{2.845876in}{2.878302in}}%
\pgfpathlineto{\pgfqpoint{2.846696in}{2.873930in}}%
\pgfpathlineto{\pgfqpoint{2.847106in}{2.872889in}}%
\pgfpathlineto{\pgfqpoint{2.847926in}{2.861659in}}%
\pgfpathlineto{\pgfqpoint{2.848336in}{2.864624in}}%
\pgfpathlineto{\pgfqpoint{2.848746in}{2.862594in}}%
\pgfpathlineto{\pgfqpoint{2.849156in}{2.877492in}}%
\pgfpathlineto{\pgfqpoint{2.849976in}{2.872905in}}%
\pgfpathlineto{\pgfqpoint{2.850386in}{2.875361in}}%
\pgfpathlineto{\pgfqpoint{2.851206in}{2.860969in}}%
\pgfpathlineto{\pgfqpoint{2.851616in}{2.865410in}}%
\pgfpathlineto{\pgfqpoint{2.852026in}{2.862278in}}%
\pgfpathlineto{\pgfqpoint{2.852846in}{2.876560in}}%
\pgfpathlineto{\pgfqpoint{2.853256in}{2.871953in}}%
\pgfpathlineto{\pgfqpoint{2.853666in}{2.876924in}}%
\pgfpathlineto{\pgfqpoint{2.854486in}{2.860901in}}%
\pgfpathlineto{\pgfqpoint{2.855306in}{2.863311in}}%
\pgfpathlineto{\pgfqpoint{2.856126in}{2.879739in}}%
\pgfpathlineto{\pgfqpoint{2.856946in}{2.877438in}}%
\pgfpathlineto{\pgfqpoint{2.857766in}{2.861761in}}%
\pgfpathlineto{\pgfqpoint{2.858586in}{2.865022in}}%
\pgfpathlineto{\pgfqpoint{2.858996in}{2.867162in}}%
\pgfpathlineto{\pgfqpoint{2.859406in}{2.881491in}}%
\pgfpathlineto{\pgfqpoint{2.860226in}{2.876974in}}%
\pgfpathlineto{\pgfqpoint{2.860636in}{2.875269in}}%
\pgfpathlineto{\pgfqpoint{2.861046in}{2.863689in}}%
\pgfpathlineto{\pgfqpoint{2.861866in}{2.866638in}}%
\pgfpathlineto{\pgfqpoint{2.862276in}{2.864787in}}%
\pgfpathlineto{\pgfqpoint{2.862686in}{2.880946in}}%
\pgfpathlineto{\pgfqpoint{2.863506in}{2.875878in}}%
\pgfpathlineto{\pgfqpoint{2.863916in}{2.878099in}}%
\pgfpathlineto{\pgfqpoint{2.864736in}{2.862915in}}%
\pgfpathlineto{\pgfqpoint{2.865146in}{2.867582in}}%
\pgfpathlineto{\pgfqpoint{2.865556in}{2.864236in}}%
\pgfpathlineto{\pgfqpoint{2.866376in}{2.879230in}}%
\pgfpathlineto{\pgfqpoint{2.866786in}{2.874760in}}%
\pgfpathlineto{\pgfqpoint{2.867196in}{2.879941in}}%
\pgfpathlineto{\pgfqpoint{2.868016in}{2.862771in}}%
\pgfpathlineto{\pgfqpoint{2.868836in}{2.865275in}}%
\pgfpathlineto{\pgfqpoint{2.869656in}{2.882854in}}%
\pgfpathlineto{\pgfqpoint{2.870476in}{2.880603in}}%
\pgfpathlineto{\pgfqpoint{2.871296in}{2.863655in}}%
\pgfpathlineto{\pgfqpoint{2.872116in}{2.867138in}}%
\pgfpathlineto{\pgfqpoint{2.872526in}{2.869671in}}%
\pgfpathlineto{\pgfqpoint{2.872936in}{2.884953in}}%
\pgfpathlineto{\pgfqpoint{2.873756in}{2.880142in}}%
\pgfpathlineto{\pgfqpoint{2.874166in}{2.878148in}}%
\pgfpathlineto{\pgfqpoint{2.874576in}{2.865743in}}%
\pgfpathlineto{\pgfqpoint{2.875396in}{2.868921in}}%
\pgfpathlineto{\pgfqpoint{2.875806in}{2.866974in}}%
\pgfpathlineto{\pgfqpoint{2.876216in}{2.884448in}}%
\pgfpathlineto{\pgfqpoint{2.877036in}{2.878931in}}%
\pgfpathlineto{\pgfqpoint{2.877446in}{2.881285in}}%
\pgfpathlineto{\pgfqpoint{2.878266in}{2.864919in}}%
\pgfpathlineto{\pgfqpoint{2.878676in}{2.869944in}}%
\pgfpathlineto{\pgfqpoint{2.879086in}{2.866359in}}%
\pgfpathlineto{\pgfqpoint{2.879906in}{2.882514in}}%
\pgfpathlineto{\pgfqpoint{2.880316in}{2.877671in}}%
\pgfpathlineto{\pgfqpoint{2.880726in}{2.883312in}}%
\pgfpathlineto{\pgfqpoint{2.881546in}{2.864792in}}%
\pgfpathlineto{\pgfqpoint{2.882366in}{2.867546in}}%
\pgfpathlineto{\pgfqpoint{2.883186in}{2.886544in}}%
\pgfpathlineto{\pgfqpoint{2.884006in}{2.884001in}}%
\pgfpathlineto{\pgfqpoint{2.884826in}{2.865843in}}%
\pgfpathlineto{\pgfqpoint{2.885646in}{2.869619in}}%
\pgfpathlineto{\pgfqpoint{2.886056in}{2.871921in}}%
\pgfpathlineto{\pgfqpoint{2.886466in}{2.888774in}}%
\pgfpathlineto{\pgfqpoint{2.887286in}{2.883424in}}%
\pgfpathlineto{\pgfqpoint{2.887696in}{2.881656in}}%
\pgfpathlineto{\pgfqpoint{2.888516in}{2.867880in}}%
\pgfpathlineto{\pgfqpoint{2.888926in}{2.871524in}}%
\pgfpathlineto{\pgfqpoint{2.889336in}{2.869100in}}%
\pgfpathlineto{\pgfqpoint{2.889746in}{2.887961in}}%
\pgfpathlineto{\pgfqpoint{2.890566in}{2.882030in}}%
\pgfpathlineto{\pgfqpoint{2.890976in}{2.885021in}}%
\pgfpathlineto{\pgfqpoint{2.891796in}{2.866975in}}%
\pgfpathlineto{\pgfqpoint{2.892206in}{2.872506in}}%
\pgfpathlineto{\pgfqpoint{2.892616in}{2.868679in}}%
\pgfpathlineto{\pgfqpoint{2.893436in}{2.886584in}}%
\pgfpathlineto{\pgfqpoint{2.893846in}{2.880660in}}%
\pgfpathlineto{\pgfqpoint{2.894256in}{2.887089in}}%
\pgfpathlineto{\pgfqpoint{2.894666in}{2.876524in}}%
\pgfpathlineto{\pgfqpoint{2.895076in}{2.867005in}}%
\pgfpathlineto{\pgfqpoint{2.895896in}{2.870207in}}%
\pgfpathlineto{\pgfqpoint{2.896716in}{2.890933in}}%
\pgfpathlineto{\pgfqpoint{2.897536in}{2.887628in}}%
\pgfpathlineto{\pgfqpoint{2.898356in}{2.868423in}}%
\pgfpathlineto{\pgfqpoint{2.899176in}{2.872538in}}%
\pgfpathlineto{\pgfqpoint{2.899586in}{2.873819in}}%
\pgfpathlineto{\pgfqpoint{2.899996in}{2.892932in}}%
\pgfpathlineto{\pgfqpoint{2.900816in}{2.886773in}}%
\pgfpathlineto{\pgfqpoint{2.901226in}{2.885920in}}%
\pgfpathlineto{\pgfqpoint{2.902046in}{2.869928in}}%
\pgfpathlineto{\pgfqpoint{2.902456in}{2.874467in}}%
\pgfpathlineto{\pgfqpoint{2.902866in}{2.871201in}}%
\pgfpathlineto{\pgfqpoint{2.903276in}{2.891296in}}%
\pgfpathlineto{\pgfqpoint{2.904096in}{2.885122in}}%
\pgfpathlineto{\pgfqpoint{2.904506in}{2.889371in}}%
\pgfpathlineto{\pgfqpoint{2.905326in}{2.869117in}}%
\pgfpathlineto{\pgfqpoint{2.906146in}{2.871324in}}%
\pgfpathlineto{\pgfqpoint{2.906966in}{2.891641in}}%
\pgfpathlineto{\pgfqpoint{2.907786in}{2.891261in}}%
\pgfpathlineto{\pgfqpoint{2.908606in}{2.869525in}}%
\pgfpathlineto{\pgfqpoint{2.909426in}{2.873390in}}%
\pgfpathlineto{\pgfqpoint{2.910246in}{2.896050in}}%
\pgfpathlineto{\pgfqpoint{2.911066in}{2.891404in}}%
\pgfpathlineto{\pgfqpoint{2.911886in}{2.871600in}}%
\pgfpathlineto{\pgfqpoint{2.913116in}{2.875380in}}%
\pgfpathlineto{\pgfqpoint{2.913526in}{2.897194in}}%
\pgfpathlineto{\pgfqpoint{2.914346in}{2.890084in}}%
\pgfpathlineto{\pgfqpoint{2.914756in}{2.891030in}}%
\pgfpathlineto{\pgfqpoint{2.915576in}{2.871955in}}%
\pgfpathlineto{\pgfqpoint{2.915986in}{2.877694in}}%
\pgfpathlineto{\pgfqpoint{2.916396in}{2.873474in}}%
\pgfpathlineto{\pgfqpoint{2.916806in}{2.894032in}}%
\pgfpathlineto{\pgfqpoint{2.917626in}{2.888165in}}%
\pgfpathlineto{\pgfqpoint{2.918036in}{2.894317in}}%
\pgfpathlineto{\pgfqpoint{2.918856in}{2.871469in}}%
\pgfpathlineto{\pgfqpoint{2.919676in}{2.874543in}}%
\pgfpathlineto{\pgfqpoint{2.920496in}{2.897843in}}%
\pgfpathlineto{\pgfqpoint{2.921316in}{2.895703in}}%
\pgfpathlineto{\pgfqpoint{2.922136in}{2.872607in}}%
\pgfpathlineto{\pgfqpoint{2.922956in}{2.877242in}}%
\pgfpathlineto{\pgfqpoint{2.923366in}{2.881047in}}%
\pgfpathlineto{\pgfqpoint{2.923776in}{2.901663in}}%
\pgfpathlineto{\pgfqpoint{2.924596in}{2.895142in}}%
\pgfpathlineto{\pgfqpoint{2.925006in}{2.892353in}}%
\pgfpathlineto{\pgfqpoint{2.925826in}{2.875267in}}%
\pgfpathlineto{\pgfqpoint{2.926236in}{2.879818in}}%
\pgfpathlineto{\pgfqpoint{2.926646in}{2.876866in}}%
\pgfpathlineto{\pgfqpoint{2.927056in}{2.900946in}}%
\pgfpathlineto{\pgfqpoint{2.927876in}{2.893211in}}%
\pgfpathlineto{\pgfqpoint{2.928286in}{2.896971in}}%
\pgfpathlineto{\pgfqpoint{2.929106in}{2.874076in}}%
\pgfpathlineto{\pgfqpoint{2.929516in}{2.881012in}}%
\pgfpathlineto{\pgfqpoint{2.929926in}{2.876351in}}%
\pgfpathlineto{\pgfqpoint{2.930746in}{2.899094in}}%
\pgfpathlineto{\pgfqpoint{2.931156in}{2.891214in}}%
\pgfpathlineto{\pgfqpoint{2.931566in}{2.899683in}}%
\pgfpathlineto{\pgfqpoint{2.931976in}{2.887035in}}%
\pgfpathlineto{\pgfqpoint{2.932386in}{2.874339in}}%
\pgfpathlineto{\pgfqpoint{2.933206in}{2.878662in}}%
\pgfpathlineto{\pgfqpoint{2.934026in}{2.905057in}}%
\pgfpathlineto{\pgfqpoint{2.934846in}{2.900118in}}%
\pgfpathlineto{\pgfqpoint{2.935666in}{2.876729in}}%
\pgfpathlineto{\pgfqpoint{2.936896in}{2.881390in}}%
\pgfpathlineto{\pgfqpoint{2.937306in}{2.906962in}}%
\pgfpathlineto{\pgfqpoint{2.938126in}{2.898548in}}%
\pgfpathlineto{\pgfqpoint{2.938536in}{2.899490in}}%
\pgfpathlineto{\pgfqpoint{2.939356in}{2.877145in}}%
\pgfpathlineto{\pgfqpoint{2.939766in}{2.883880in}}%
\pgfpathlineto{\pgfqpoint{2.940176in}{2.878960in}}%
\pgfpathlineto{\pgfqpoint{2.940586in}{2.903049in}}%
\pgfpathlineto{\pgfqpoint{2.941406in}{2.896086in}}%
\pgfpathlineto{\pgfqpoint{2.941816in}{2.903505in}}%
\pgfpathlineto{\pgfqpoint{2.942636in}{2.876657in}}%
\pgfpathlineto{\pgfqpoint{2.943456in}{2.880456in}}%
\pgfpathlineto{\pgfqpoint{2.944276in}{2.907973in}}%
\pgfpathlineto{\pgfqpoint{2.945096in}{2.905023in}}%
\pgfpathlineto{\pgfqpoint{2.945916in}{2.878358in}}%
\pgfpathlineto{\pgfqpoint{2.946736in}{2.883859in}}%
\pgfpathlineto{\pgfqpoint{2.947146in}{2.886705in}}%
\pgfpathlineto{\pgfqpoint{2.947556in}{2.912302in}}%
\pgfpathlineto{\pgfqpoint{2.948376in}{2.904009in}}%
\pgfpathlineto{\pgfqpoint{2.948786in}{2.902174in}}%
\pgfpathlineto{\pgfqpoint{2.949606in}{2.880493in}}%
\pgfpathlineto{\pgfqpoint{2.950016in}{2.886722in}}%
\pgfpathlineto{\pgfqpoint{2.950426in}{2.882287in}}%
\pgfpathlineto{\pgfqpoint{2.950836in}{2.910207in}}%
\pgfpathlineto{\pgfqpoint{2.951656in}{2.901364in}}%
\pgfpathlineto{\pgfqpoint{2.952066in}{2.907418in}}%
\pgfpathlineto{\pgfqpoint{2.952066in}{2.907418in}}%
\pgfusepath{stroke}%
\end{pgfscope}%
\begin{pgfscope}%
\pgfpathrectangle{\pgfqpoint{0.800000in}{2.544000in}}{\pgfqpoint{2.254545in}{1.680000in}}%
\pgfusepath{clip}%
\pgfsetrectcap%
\pgfsetroundjoin%
\pgfsetlinewidth{1.505625pt}%
\definecolor{currentstroke}{rgb}{0.737255,0.741176,0.133333}%
\pgfsetstrokecolor{currentstroke}%
\pgfsetdash{}{0pt}%
\pgfpathmoveto{\pgfqpoint{0.902479in}{3.626600in}}%
\pgfpathlineto{\pgfqpoint{0.914369in}{3.625512in}}%
\pgfpathlineto{\pgfqpoint{0.922979in}{3.622536in}}%
\pgfpathlineto{\pgfqpoint{0.929949in}{3.617639in}}%
\pgfpathlineto{\pgfqpoint{0.936099in}{3.610519in}}%
\pgfpathlineto{\pgfqpoint{0.942249in}{3.599924in}}%
\pgfpathlineto{\pgfqpoint{0.948809in}{3.584012in}}%
\pgfpathlineto{\pgfqpoint{0.956189in}{3.559956in}}%
\pgfpathlineto{\pgfqpoint{0.965619in}{3.520952in}}%
\pgfpathlineto{\pgfqpoint{0.984069in}{3.442890in}}%
\pgfpathlineto{\pgfqpoint{0.989809in}{3.428005in}}%
\pgfpathlineto{\pgfqpoint{0.993909in}{3.422272in}}%
\pgfpathlineto{\pgfqpoint{0.997189in}{3.420853in}}%
\pgfpathlineto{\pgfqpoint{1.000059in}{3.421874in}}%
\pgfpathlineto{\pgfqpoint{1.003749in}{3.425994in}}%
\pgfpathlineto{\pgfqpoint{1.008669in}{3.435309in}}%
\pgfpathlineto{\pgfqpoint{1.023019in}{3.464690in}}%
\pgfpathlineto{\pgfqpoint{1.026299in}{3.466461in}}%
\pgfpathlineto{\pgfqpoint{1.028759in}{3.465513in}}%
\pgfpathlineto{\pgfqpoint{1.031629in}{3.461606in}}%
\pgfpathlineto{\pgfqpoint{1.035319in}{3.451815in}}%
\pgfpathlineto{\pgfqpoint{1.039829in}{3.432461in}}%
\pgfpathlineto{\pgfqpoint{1.045979in}{3.394417in}}%
\pgfpathlineto{\pgfqpoint{1.068119in}{3.243207in}}%
\pgfpathlineto{\pgfqpoint{1.071809in}{3.234634in}}%
\pgfpathlineto{\pgfqpoint{1.073859in}{3.233586in}}%
\pgfpathlineto{\pgfqpoint{1.075909in}{3.235297in}}%
\pgfpathlineto{\pgfqpoint{1.078369in}{3.240967in}}%
\pgfpathlineto{\pgfqpoint{1.082059in}{3.256383in}}%
\pgfpathlineto{\pgfqpoint{1.087389in}{3.290070in}}%
\pgfpathlineto{\pgfqpoint{1.098869in}{3.364848in}}%
\pgfpathlineto{\pgfqpoint{1.102149in}{3.373258in}}%
\pgfpathlineto{\pgfqpoint{1.103789in}{3.373793in}}%
\pgfpathlineto{\pgfqpoint{1.105429in}{3.371783in}}%
\pgfpathlineto{\pgfqpoint{1.107889in}{3.364304in}}%
\pgfpathlineto{\pgfqpoint{1.112399in}{3.341197in}}%
\pgfpathlineto{\pgfqpoint{1.117319in}{3.319363in}}%
\pgfpathlineto{\pgfqpoint{1.118959in}{3.317584in}}%
\pgfpathlineto{\pgfqpoint{1.120189in}{3.319044in}}%
\pgfpathlineto{\pgfqpoint{1.122239in}{3.327485in}}%
\pgfpathlineto{\pgfqpoint{1.125109in}{3.351937in}}%
\pgfpathlineto{\pgfqpoint{1.136589in}{3.469679in}}%
\pgfpathlineto{\pgfqpoint{1.137409in}{3.469368in}}%
\pgfpathlineto{\pgfqpoint{1.138639in}{3.465431in}}%
\pgfpathlineto{\pgfqpoint{1.141099in}{3.446091in}}%
\pgfpathlineto{\pgfqpoint{1.145609in}{3.385915in}}%
\pgfpathlineto{\pgfqpoint{1.150939in}{3.322402in}}%
\pgfpathlineto{\pgfqpoint{1.153809in}{3.307578in}}%
\pgfpathlineto{\pgfqpoint{1.155449in}{3.305770in}}%
\pgfpathlineto{\pgfqpoint{1.156679in}{3.307042in}}%
\pgfpathlineto{\pgfqpoint{1.159139in}{3.314207in}}%
\pgfpathlineto{\pgfqpoint{1.164879in}{3.332499in}}%
\pgfpathlineto{\pgfqpoint{1.166519in}{3.332917in}}%
\pgfpathlineto{\pgfqpoint{1.168159in}{3.329847in}}%
\pgfpathlineto{\pgfqpoint{1.170619in}{3.318282in}}%
\pgfpathlineto{\pgfqpoint{1.174309in}{3.286849in}}%
\pgfpathlineto{\pgfqpoint{1.187019in}{3.165037in}}%
\pgfpathlineto{\pgfqpoint{1.189069in}{3.161048in}}%
\pgfpathlineto{\pgfqpoint{1.190299in}{3.161847in}}%
\pgfpathlineto{\pgfqpoint{1.191939in}{3.166481in}}%
\pgfpathlineto{\pgfqpoint{1.194809in}{3.182986in}}%
\pgfpathlineto{\pgfqpoint{1.205059in}{3.251812in}}%
\pgfpathlineto{\pgfqpoint{1.206289in}{3.252205in}}%
\pgfpathlineto{\pgfqpoint{1.207519in}{3.249971in}}%
\pgfpathlineto{\pgfqpoint{1.209979in}{3.238781in}}%
\pgfpathlineto{\pgfqpoint{1.215719in}{3.209357in}}%
\pgfpathlineto{\pgfqpoint{1.216539in}{3.209308in}}%
\pgfpathlineto{\pgfqpoint{1.217769in}{3.212649in}}%
\pgfpathlineto{\pgfqpoint{1.219819in}{3.227452in}}%
\pgfpathlineto{\pgfqpoint{1.228019in}{3.305075in}}%
\pgfpathlineto{\pgfqpoint{1.228429in}{3.304658in}}%
\pgfpathlineto{\pgfqpoint{1.229659in}{3.299592in}}%
\pgfpathlineto{\pgfqpoint{1.232119in}{3.275029in}}%
\pgfpathlineto{\pgfqpoint{1.239089in}{3.195277in}}%
\pgfpathlineto{\pgfqpoint{1.241139in}{3.190135in}}%
\pgfpathlineto{\pgfqpoint{1.242369in}{3.191115in}}%
\pgfpathlineto{\pgfqpoint{1.245239in}{3.199228in}}%
\pgfpathlineto{\pgfqpoint{1.248109in}{3.204970in}}%
\pgfpathlineto{\pgfqpoint{1.249339in}{3.204148in}}%
\pgfpathlineto{\pgfqpoint{1.250979in}{3.198980in}}%
\pgfpathlineto{\pgfqpoint{1.253439in}{3.182571in}}%
\pgfpathlineto{\pgfqpoint{1.259589in}{3.118676in}}%
\pgfpathlineto{\pgfqpoint{1.263279in}{3.091838in}}%
\pgfpathlineto{\pgfqpoint{1.265329in}{3.087314in}}%
\pgfpathlineto{\pgfqpoint{1.266559in}{3.088498in}}%
\pgfpathlineto{\pgfqpoint{1.268609in}{3.096223in}}%
\pgfpathlineto{\pgfqpoint{1.277629in}{3.141788in}}%
\pgfpathlineto{\pgfqpoint{1.278859in}{3.139687in}}%
\pgfpathlineto{\pgfqpoint{1.281319in}{3.128528in}}%
\pgfpathlineto{\pgfqpoint{1.285419in}{3.110785in}}%
\pgfpathlineto{\pgfqpoint{1.286239in}{3.110988in}}%
\pgfpathlineto{\pgfqpoint{1.287469in}{3.114800in}}%
\pgfpathlineto{\pgfqpoint{1.289929in}{3.133335in}}%
\pgfpathlineto{\pgfqpoint{1.294439in}{3.166206in}}%
\pgfpathlineto{\pgfqpoint{1.295259in}{3.166308in}}%
\pgfpathlineto{\pgfqpoint{1.296489in}{3.161886in}}%
\pgfpathlineto{\pgfqpoint{1.298949in}{3.140100in}}%
\pgfpathlineto{\pgfqpoint{1.303869in}{3.094864in}}%
\pgfpathlineto{\pgfqpoint{1.305919in}{3.090244in}}%
\pgfpathlineto{\pgfqpoint{1.307149in}{3.091132in}}%
\pgfpathlineto{\pgfqpoint{1.311659in}{3.097072in}}%
\pgfpathlineto{\pgfqpoint{1.312889in}{3.094911in}}%
\pgfpathlineto{\pgfqpoint{1.314939in}{3.085617in}}%
\pgfpathlineto{\pgfqpoint{1.319039in}{3.052441in}}%
\pgfpathlineto{\pgfqpoint{1.323139in}{3.024570in}}%
\pgfpathlineto{\pgfqpoint{1.325189in}{3.020626in}}%
\pgfpathlineto{\pgfqpoint{1.326419in}{3.021941in}}%
\pgfpathlineto{\pgfqpoint{1.328469in}{3.029009in}}%
\pgfpathlineto{\pgfqpoint{1.334209in}{3.050492in}}%
\pgfpathlineto{\pgfqpoint{1.335439in}{3.049695in}}%
\pgfpathlineto{\pgfqpoint{1.337489in}{3.043241in}}%
\pgfpathlineto{\pgfqpoint{1.341589in}{3.028916in}}%
\pgfpathlineto{\pgfqpoint{1.342409in}{3.029140in}}%
\pgfpathlineto{\pgfqpoint{1.343639in}{3.032345in}}%
\pgfpathlineto{\pgfqpoint{1.346919in}{3.050999in}}%
\pgfpathlineto{\pgfqpoint{1.349379in}{3.058798in}}%
\pgfpathlineto{\pgfqpoint{1.350199in}{3.057825in}}%
\pgfpathlineto{\pgfqpoint{1.351839in}{3.050393in}}%
\pgfpathlineto{\pgfqpoint{1.358809in}{3.010187in}}%
\pgfpathlineto{\pgfqpoint{1.360449in}{3.010490in}}%
\pgfpathlineto{\pgfqpoint{1.363729in}{3.012404in}}%
\pgfpathlineto{\pgfqpoint{1.365369in}{3.009487in}}%
\pgfpathlineto{\pgfqpoint{1.367829in}{2.998432in}}%
\pgfpathlineto{\pgfqpoint{1.374389in}{2.965277in}}%
\pgfpathlineto{\pgfqpoint{1.376029in}{2.964779in}}%
\pgfpathlineto{\pgfqpoint{1.377669in}{2.967763in}}%
\pgfpathlineto{\pgfqpoint{1.382999in}{2.979333in}}%
\pgfpathlineto{\pgfqpoint{1.384229in}{2.978002in}}%
\pgfpathlineto{\pgfqpoint{1.387099in}{2.969584in}}%
\pgfpathlineto{\pgfqpoint{1.389559in}{2.964628in}}%
\pgfpathlineto{\pgfqpoint{1.390789in}{2.965484in}}%
\pgfpathlineto{\pgfqpoint{1.392839in}{2.971421in}}%
\pgfpathlineto{\pgfqpoint{1.395709in}{2.979036in}}%
\pgfpathlineto{\pgfqpoint{1.396939in}{2.978287in}}%
\pgfpathlineto{\pgfqpoint{1.398579in}{2.972793in}}%
\pgfpathlineto{\pgfqpoint{1.404319in}{2.949341in}}%
\pgfpathlineto{\pgfqpoint{1.406369in}{2.948900in}}%
\pgfpathlineto{\pgfqpoint{1.408829in}{2.949030in}}%
\pgfpathlineto{\pgfqpoint{1.410469in}{2.946629in}}%
\pgfpathlineto{\pgfqpoint{1.413339in}{2.936585in}}%
\pgfpathlineto{\pgfqpoint{1.417849in}{2.920980in}}%
\pgfpathlineto{\pgfqpoint{1.419489in}{2.919963in}}%
\pgfpathlineto{\pgfqpoint{1.421129in}{2.921483in}}%
\pgfpathlineto{\pgfqpoint{1.425229in}{2.926395in}}%
\pgfpathlineto{\pgfqpoint{1.426869in}{2.925041in}}%
\pgfpathlineto{\pgfqpoint{1.432609in}{2.916759in}}%
\pgfpathlineto{\pgfqpoint{1.434659in}{2.919729in}}%
\pgfpathlineto{\pgfqpoint{1.437118in}{2.922685in}}%
\pgfpathlineto{\pgfqpoint{1.438348in}{2.921753in}}%
\pgfpathlineto{\pgfqpoint{1.440398in}{2.916395in}}%
\pgfpathlineto{\pgfqpoint{1.444908in}{2.904762in}}%
\pgfpathlineto{\pgfqpoint{1.447368in}{2.904033in}}%
\pgfpathlineto{\pgfqpoint{1.449828in}{2.902788in}}%
\pgfpathlineto{\pgfqpoint{1.452288in}{2.897944in}}%
\pgfpathlineto{\pgfqpoint{1.457618in}{2.886607in}}%
\pgfpathlineto{\pgfqpoint{1.459668in}{2.886737in}}%
\pgfpathlineto{\pgfqpoint{1.463768in}{2.888500in}}%
\pgfpathlineto{\pgfqpoint{1.465818in}{2.886495in}}%
\pgfpathlineto{\pgfqpoint{1.469918in}{2.882092in}}%
\pgfpathlineto{\pgfqpoint{1.472378in}{2.883388in}}%
\pgfpathlineto{\pgfqpoint{1.474838in}{2.884003in}}%
\pgfpathlineto{\pgfqpoint{1.476888in}{2.881385in}}%
\pgfpathlineto{\pgfqpoint{1.482218in}{2.873256in}}%
\pgfpathlineto{\pgfqpoint{1.487548in}{2.870069in}}%
\pgfpathlineto{\pgfqpoint{1.494108in}{2.862308in}}%
\pgfpathlineto{\pgfqpoint{1.500258in}{2.861525in}}%
\pgfpathlineto{\pgfqpoint{1.505178in}{2.858484in}}%
\pgfpathlineto{\pgfqpoint{1.509688in}{2.858184in}}%
\pgfpathlineto{\pgfqpoint{1.525268in}{2.846357in}}%
\pgfpathlineto{\pgfqpoint{1.529368in}{2.845847in}}%
\pgfpathlineto{\pgfqpoint{1.533058in}{2.844447in}}%
\pgfpathlineto{\pgfqpoint{1.537568in}{2.842929in}}%
\pgfpathlineto{\pgfqpoint{1.541668in}{2.842029in}}%
\pgfpathlineto{\pgfqpoint{1.549458in}{2.838503in}}%
\pgfpathlineto{\pgfqpoint{1.577338in}{2.830654in}}%
\pgfpathlineto{\pgfqpoint{1.623258in}{2.823818in}}%
\pgfpathlineto{\pgfqpoint{1.660568in}{2.821422in}}%
\pgfpathlineto{\pgfqpoint{1.726988in}{2.820300in}}%
\pgfpathlineto{\pgfqpoint{1.988978in}{2.820175in}}%
\pgfpathlineto{\pgfqpoint{2.952066in}{2.820175in}}%
\pgfpathlineto{\pgfqpoint{2.952066in}{2.820175in}}%
\pgfusepath{stroke}%
\end{pgfscope}%
\begin{pgfscope}%
\pgfpathrectangle{\pgfqpoint{0.800000in}{2.544000in}}{\pgfqpoint{2.254545in}{1.680000in}}%
\pgfusepath{clip}%
\pgfsetrectcap%
\pgfsetroundjoin%
\pgfsetlinewidth{1.505625pt}%
\definecolor{currentstroke}{rgb}{0.090196,0.745098,0.811765}%
\pgfsetstrokecolor{currentstroke}%
\pgfsetdash{}{0pt}%
\pgfpathmoveto{\pgfqpoint{0.902479in}{3.626600in}}%
\pgfpathlineto{\pgfqpoint{0.914369in}{3.625522in}}%
\pgfpathlineto{\pgfqpoint{0.922979in}{3.622550in}}%
\pgfpathlineto{\pgfqpoint{0.929539in}{3.618022in}}%
\pgfpathlineto{\pgfqpoint{0.935689in}{3.611100in}}%
\pgfpathlineto{\pgfqpoint{0.941839in}{3.600749in}}%
\pgfpathlineto{\pgfqpoint{0.948399in}{3.585133in}}%
\pgfpathlineto{\pgfqpoint{0.955779in}{3.561407in}}%
\pgfpathlineto{\pgfqpoint{0.965209in}{3.522691in}}%
\pgfpathlineto{\pgfqpoint{0.984479in}{3.441401in}}%
\pgfpathlineto{\pgfqpoint{0.990219in}{3.427016in}}%
\pgfpathlineto{\pgfqpoint{0.994319in}{3.421689in}}%
\pgfpathlineto{\pgfqpoint{0.997599in}{3.420594in}}%
\pgfpathlineto{\pgfqpoint{1.000469in}{3.421874in}}%
\pgfpathlineto{\pgfqpoint{1.004159in}{3.426269in}}%
\pgfpathlineto{\pgfqpoint{1.009489in}{3.436716in}}%
\pgfpathlineto{\pgfqpoint{1.022199in}{3.463102in}}%
\pgfpathlineto{\pgfqpoint{1.025889in}{3.465679in}}%
\pgfpathlineto{\pgfqpoint{1.028349in}{3.465042in}}%
\pgfpathlineto{\pgfqpoint{1.031219in}{3.461545in}}%
\pgfpathlineto{\pgfqpoint{1.034499in}{3.453620in}}%
\pgfpathlineto{\pgfqpoint{1.038599in}{3.437659in}}%
\pgfpathlineto{\pgfqpoint{1.044339in}{3.404826in}}%
\pgfpathlineto{\pgfqpoint{1.053359in}{3.336901in}}%
\pgfpathlineto{\pgfqpoint{1.063609in}{3.263253in}}%
\pgfpathlineto{\pgfqpoint{1.068529in}{3.241071in}}%
\pgfpathlineto{\pgfqpoint{1.071809in}{3.233814in}}%
\pgfpathlineto{\pgfqpoint{1.073859in}{3.232728in}}%
\pgfpathlineto{\pgfqpoint{1.075909in}{3.234392in}}%
\pgfpathlineto{\pgfqpoint{1.078369in}{3.239990in}}%
\pgfpathlineto{\pgfqpoint{1.082059in}{3.255271in}}%
\pgfpathlineto{\pgfqpoint{1.087389in}{3.288707in}}%
\pgfpathlineto{\pgfqpoint{1.098869in}{3.362848in}}%
\pgfpathlineto{\pgfqpoint{1.102149in}{3.371115in}}%
\pgfpathlineto{\pgfqpoint{1.103789in}{3.371595in}}%
\pgfpathlineto{\pgfqpoint{1.105429in}{3.369545in}}%
\pgfpathlineto{\pgfqpoint{1.107889in}{3.362029in}}%
\pgfpathlineto{\pgfqpoint{1.112399in}{3.338903in}}%
\pgfpathlineto{\pgfqpoint{1.117319in}{3.316970in}}%
\pgfpathlineto{\pgfqpoint{1.119369in}{3.315258in}}%
\pgfpathlineto{\pgfqpoint{1.120599in}{3.317466in}}%
\pgfpathlineto{\pgfqpoint{1.122649in}{3.327166in}}%
\pgfpathlineto{\pgfqpoint{1.125929in}{3.357654in}}%
\pgfpathlineto{\pgfqpoint{1.135769in}{3.462141in}}%
\pgfpathlineto{\pgfqpoint{1.136999in}{3.463616in}}%
\pgfpathlineto{\pgfqpoint{1.137409in}{3.463192in}}%
\pgfpathlineto{\pgfqpoint{1.138639in}{3.459192in}}%
\pgfpathlineto{\pgfqpoint{1.141099in}{3.439904in}}%
\pgfpathlineto{\pgfqpoint{1.146019in}{3.374368in}}%
\pgfpathlineto{\pgfqpoint{1.151349in}{3.314123in}}%
\pgfpathlineto{\pgfqpoint{1.154219in}{3.301420in}}%
\pgfpathlineto{\pgfqpoint{1.155859in}{3.300535in}}%
\pgfpathlineto{\pgfqpoint{1.157499in}{3.303240in}}%
\pgfpathlineto{\pgfqpoint{1.161189in}{3.315977in}}%
\pgfpathlineto{\pgfqpoint{1.164879in}{3.325846in}}%
\pgfpathlineto{\pgfqpoint{1.166519in}{3.326145in}}%
\pgfpathlineto{\pgfqpoint{1.168159in}{3.323026in}}%
\pgfpathlineto{\pgfqpoint{1.170619in}{3.311525in}}%
\pgfpathlineto{\pgfqpoint{1.174309in}{3.280473in}}%
\pgfpathlineto{\pgfqpoint{1.186609in}{3.162175in}}%
\pgfpathlineto{\pgfqpoint{1.189069in}{3.156700in}}%
\pgfpathlineto{\pgfqpoint{1.190299in}{3.157479in}}%
\pgfpathlineto{\pgfqpoint{1.191939in}{3.162008in}}%
\pgfpathlineto{\pgfqpoint{1.194809in}{3.178113in}}%
\pgfpathlineto{\pgfqpoint{1.205059in}{3.244300in}}%
\pgfpathlineto{\pgfqpoint{1.206289in}{3.244467in}}%
\pgfpathlineto{\pgfqpoint{1.207929in}{3.240744in}}%
\pgfpathlineto{\pgfqpoint{1.210799in}{3.225770in}}%
\pgfpathlineto{\pgfqpoint{1.215309in}{3.202521in}}%
\pgfpathlineto{\pgfqpoint{1.216539in}{3.201838in}}%
\pgfpathlineto{\pgfqpoint{1.217769in}{3.205061in}}%
\pgfpathlineto{\pgfqpoint{1.219819in}{3.219288in}}%
\pgfpathlineto{\pgfqpoint{1.228019in}{3.292103in}}%
\pgfpathlineto{\pgfqpoint{1.228839in}{3.290407in}}%
\pgfpathlineto{\pgfqpoint{1.230479in}{3.279982in}}%
\pgfpathlineto{\pgfqpoint{1.234169in}{3.234520in}}%
\pgfpathlineto{\pgfqpoint{1.238679in}{3.186980in}}%
\pgfpathlineto{\pgfqpoint{1.241139in}{3.179722in}}%
\pgfpathlineto{\pgfqpoint{1.242369in}{3.180545in}}%
\pgfpathlineto{\pgfqpoint{1.245239in}{3.187956in}}%
\pgfpathlineto{\pgfqpoint{1.248109in}{3.192993in}}%
\pgfpathlineto{\pgfqpoint{1.249339in}{3.191992in}}%
\pgfpathlineto{\pgfqpoint{1.250979in}{3.186762in}}%
\pgfpathlineto{\pgfqpoint{1.253849in}{3.167163in}}%
\pgfpathlineto{\pgfqpoint{1.264509in}{3.080191in}}%
\pgfpathlineto{\pgfqpoint{1.265739in}{3.079582in}}%
\pgfpathlineto{\pgfqpoint{1.266969in}{3.081687in}}%
\pgfpathlineto{\pgfqpoint{1.269429in}{3.092630in}}%
\pgfpathlineto{\pgfqpoint{1.276399in}{3.129402in}}%
\pgfpathlineto{\pgfqpoint{1.277629in}{3.129755in}}%
\pgfpathlineto{\pgfqpoint{1.279269in}{3.125993in}}%
\pgfpathlineto{\pgfqpoint{1.282959in}{3.107239in}}%
\pgfpathlineto{\pgfqpoint{1.285419in}{3.099448in}}%
\pgfpathlineto{\pgfqpoint{1.286239in}{3.099737in}}%
\pgfpathlineto{\pgfqpoint{1.287469in}{3.103439in}}%
\pgfpathlineto{\pgfqpoint{1.289929in}{3.120701in}}%
\pgfpathlineto{\pgfqpoint{1.294029in}{3.148829in}}%
\pgfpathlineto{\pgfqpoint{1.294849in}{3.149603in}}%
\pgfpathlineto{\pgfqpoint{1.295259in}{3.149135in}}%
\pgfpathlineto{\pgfqpoint{1.296489in}{3.144384in}}%
\pgfpathlineto{\pgfqpoint{1.298949in}{3.123249in}}%
\pgfpathlineto{\pgfqpoint{1.303869in}{3.080973in}}%
\pgfpathlineto{\pgfqpoint{1.305919in}{3.076668in}}%
\pgfpathlineto{\pgfqpoint{1.307149in}{3.077397in}}%
\pgfpathlineto{\pgfqpoint{1.311659in}{3.081988in}}%
\pgfpathlineto{\pgfqpoint{1.313299in}{3.078341in}}%
\pgfpathlineto{\pgfqpoint{1.315759in}{3.065216in}}%
\pgfpathlineto{\pgfqpoint{1.324369in}{3.010836in}}%
\pgfpathlineto{\pgfqpoint{1.325599in}{3.010563in}}%
\pgfpathlineto{\pgfqpoint{1.327239in}{3.013837in}}%
\pgfpathlineto{\pgfqpoint{1.334209in}{3.036512in}}%
\pgfpathlineto{\pgfqpoint{1.335029in}{3.035939in}}%
\pgfpathlineto{\pgfqpoint{1.336669in}{3.031834in}}%
\pgfpathlineto{\pgfqpoint{1.341999in}{3.015644in}}%
\pgfpathlineto{\pgfqpoint{1.343229in}{3.017687in}}%
\pgfpathlineto{\pgfqpoint{1.345689in}{3.028866in}}%
\pgfpathlineto{\pgfqpoint{1.348969in}{3.040819in}}%
\pgfpathlineto{\pgfqpoint{1.349789in}{3.040325in}}%
\pgfpathlineto{\pgfqpoint{1.351429in}{3.034338in}}%
\pgfpathlineto{\pgfqpoint{1.359219in}{2.995402in}}%
\pgfpathlineto{\pgfqpoint{1.361269in}{2.996357in}}%
\pgfpathlineto{\pgfqpoint{1.363319in}{2.996984in}}%
\pgfpathlineto{\pgfqpoint{1.364959in}{2.994695in}}%
\pgfpathlineto{\pgfqpoint{1.367419in}{2.985282in}}%
\pgfpathlineto{\pgfqpoint{1.374389in}{2.953805in}}%
\pgfpathlineto{\pgfqpoint{1.376029in}{2.953573in}}%
\pgfpathlineto{\pgfqpoint{1.378079in}{2.957246in}}%
\pgfpathlineto{\pgfqpoint{1.382179in}{2.965172in}}%
\pgfpathlineto{\pgfqpoint{1.383409in}{2.964684in}}%
\pgfpathlineto{\pgfqpoint{1.385459in}{2.960326in}}%
\pgfpathlineto{\pgfqpoint{1.389559in}{2.951295in}}%
\pgfpathlineto{\pgfqpoint{1.390789in}{2.952154in}}%
\pgfpathlineto{\pgfqpoint{1.393659in}{2.959401in}}%
\pgfpathlineto{\pgfqpoint{1.395709in}{2.962462in}}%
\pgfpathlineto{\pgfqpoint{1.396939in}{2.961184in}}%
\pgfpathlineto{\pgfqpoint{1.398989in}{2.954008in}}%
\pgfpathlineto{\pgfqpoint{1.403909in}{2.935948in}}%
\pgfpathlineto{\pgfqpoint{1.405959in}{2.934896in}}%
\pgfpathlineto{\pgfqpoint{1.408829in}{2.934515in}}%
\pgfpathlineto{\pgfqpoint{1.410879in}{2.931054in}}%
\pgfpathlineto{\pgfqpoint{1.414979in}{2.916866in}}%
\pgfpathlineto{\pgfqpoint{1.418259in}{2.909396in}}%
\pgfpathlineto{\pgfqpoint{1.419899in}{2.909253in}}%
\pgfpathlineto{\pgfqpoint{1.423179in}{2.912798in}}%
\pgfpathlineto{\pgfqpoint{1.425229in}{2.913536in}}%
\pgfpathlineto{\pgfqpoint{1.427279in}{2.911205in}}%
\pgfpathlineto{\pgfqpoint{1.431789in}{2.904491in}}%
\pgfpathlineto{\pgfqpoint{1.433429in}{2.905388in}}%
\pgfpathlineto{\pgfqpoint{1.437118in}{2.908529in}}%
\pgfpathlineto{\pgfqpoint{1.438758in}{2.906573in}}%
\pgfpathlineto{\pgfqpoint{1.446548in}{2.892140in}}%
\pgfpathlineto{\pgfqpoint{1.449828in}{2.890509in}}%
\pgfpathlineto{\pgfqpoint{1.452698in}{2.885242in}}%
\pgfpathlineto{\pgfqpoint{1.457208in}{2.877219in}}%
\pgfpathlineto{\pgfqpoint{1.459668in}{2.877061in}}%
\pgfpathlineto{\pgfqpoint{1.463768in}{2.877736in}}%
\pgfpathlineto{\pgfqpoint{1.466638in}{2.874687in}}%
\pgfpathlineto{\pgfqpoint{1.469918in}{2.872111in}}%
\pgfpathlineto{\pgfqpoint{1.476068in}{2.871544in}}%
\pgfpathlineto{\pgfqpoint{1.483038in}{2.863535in}}%
\pgfpathlineto{\pgfqpoint{1.486728in}{2.861618in}}%
\pgfpathlineto{\pgfqpoint{1.494928in}{2.854588in}}%
\pgfpathlineto{\pgfqpoint{1.499438in}{2.853867in}}%
\pgfpathlineto{\pgfqpoint{1.505998in}{2.850805in}}%
\pgfpathlineto{\pgfqpoint{1.509688in}{2.849975in}}%
\pgfpathlineto{\pgfqpoint{1.518708in}{2.844597in}}%
\pgfpathlineto{\pgfqpoint{1.537568in}{2.837387in}}%
\pgfpathlineto{\pgfqpoint{1.542488in}{2.836036in}}%
\pgfpathlineto{\pgfqpoint{1.549048in}{2.833800in}}%
\pgfpathlineto{\pgfqpoint{1.584308in}{2.826207in}}%
\pgfpathlineto{\pgfqpoint{1.625718in}{2.822282in}}%
\pgfpathlineto{\pgfqpoint{1.660158in}{2.820892in}}%
\pgfpathlineto{\pgfqpoint{1.739288in}{2.820207in}}%
\pgfpathlineto{\pgfqpoint{2.512547in}{2.820175in}}%
\pgfpathlineto{\pgfqpoint{2.952066in}{2.820175in}}%
\pgfpathlineto{\pgfqpoint{2.952066in}{2.820175in}}%
\pgfusepath{stroke}%
\end{pgfscope}%
\begin{pgfscope}%
\pgfsetrectcap%
\pgfsetmiterjoin%
\pgfsetlinewidth{0.803000pt}%
\definecolor{currentstroke}{rgb}{0.000000,0.000000,0.000000}%
\pgfsetstrokecolor{currentstroke}%
\pgfsetdash{}{0pt}%
\pgfpathmoveto{\pgfqpoint{0.800000in}{2.544000in}}%
\pgfpathlineto{\pgfqpoint{0.800000in}{4.224000in}}%
\pgfusepath{stroke}%
\end{pgfscope}%
\begin{pgfscope}%
\pgfsetrectcap%
\pgfsetmiterjoin%
\pgfsetlinewidth{0.803000pt}%
\definecolor{currentstroke}{rgb}{0.000000,0.000000,0.000000}%
\pgfsetstrokecolor{currentstroke}%
\pgfsetdash{}{0pt}%
\pgfpathmoveto{\pgfqpoint{3.054545in}{2.544000in}}%
\pgfpathlineto{\pgfqpoint{3.054545in}{4.224000in}}%
\pgfusepath{stroke}%
\end{pgfscope}%
\begin{pgfscope}%
\pgfsetrectcap%
\pgfsetmiterjoin%
\pgfsetlinewidth{0.803000pt}%
\definecolor{currentstroke}{rgb}{0.000000,0.000000,0.000000}%
\pgfsetstrokecolor{currentstroke}%
\pgfsetdash{}{0pt}%
\pgfpathmoveto{\pgfqpoint{0.800000in}{2.544000in}}%
\pgfpathlineto{\pgfqpoint{3.054545in}{2.544000in}}%
\pgfusepath{stroke}%
\end{pgfscope}%
\begin{pgfscope}%
\pgfsetrectcap%
\pgfsetmiterjoin%
\pgfsetlinewidth{0.803000pt}%
\definecolor{currentstroke}{rgb}{0.000000,0.000000,0.000000}%
\pgfsetstrokecolor{currentstroke}%
\pgfsetdash{}{0pt}%
\pgfpathmoveto{\pgfqpoint{0.800000in}{4.224000in}}%
\pgfpathlineto{\pgfqpoint{3.054545in}{4.224000in}}%
\pgfusepath{stroke}%
\end{pgfscope}%
\begin{pgfscope}%
\pgfsetbuttcap%
\pgfsetmiterjoin%
\definecolor{currentfill}{rgb}{1.000000,1.000000,1.000000}%
\pgfsetfillcolor{currentfill}%
\pgfsetlinewidth{0.000000pt}%
\definecolor{currentstroke}{rgb}{0.000000,0.000000,0.000000}%
\pgfsetstrokecolor{currentstroke}%
\pgfsetstrokeopacity{0.000000}%
\pgfsetdash{}{0pt}%
\pgfpathmoveto{\pgfqpoint{3.505455in}{2.544000in}}%
\pgfpathlineto{\pgfqpoint{5.760000in}{2.544000in}}%
\pgfpathlineto{\pgfqpoint{5.760000in}{4.224000in}}%
\pgfpathlineto{\pgfqpoint{3.505455in}{4.224000in}}%
\pgfpathlineto{\pgfqpoint{3.505455in}{2.544000in}}%
\pgfpathclose%
\pgfusepath{fill}%
\end{pgfscope}%
\begin{pgfscope}%
\pgfsetbuttcap%
\pgfsetroundjoin%
\definecolor{currentfill}{rgb}{0.000000,0.000000,0.000000}%
\pgfsetfillcolor{currentfill}%
\pgfsetlinewidth{0.803000pt}%
\definecolor{currentstroke}{rgb}{0.000000,0.000000,0.000000}%
\pgfsetstrokecolor{currentstroke}%
\pgfsetdash{}{0pt}%
\pgfsys@defobject{currentmarker}{\pgfqpoint{0.000000in}{-0.048611in}}{\pgfqpoint{0.000000in}{0.000000in}}{%
\pgfpathmoveto{\pgfqpoint{0.000000in}{0.000000in}}%
\pgfpathlineto{\pgfqpoint{0.000000in}{-0.048611in}}%
\pgfusepath{stroke,fill}%
}%
\begin{pgfscope}%
\pgfsys@transformshift{3.607934in}{2.544000in}%
\pgfsys@useobject{currentmarker}{}%
\end{pgfscope}%
\end{pgfscope}%
\begin{pgfscope}%
\definecolor{textcolor}{rgb}{0.000000,0.000000,0.000000}%
\pgfsetstrokecolor{textcolor}%
\pgfsetfillcolor{textcolor}%
\pgftext[x=3.607934in,y=2.446778in,,top]{\color{textcolor}\rmfamily\fontsize{10.000000}{12.000000}\selectfont \(\displaystyle {0}\)}%
\end{pgfscope}%
\begin{pgfscope}%
\pgfsetbuttcap%
\pgfsetroundjoin%
\definecolor{currentfill}{rgb}{0.000000,0.000000,0.000000}%
\pgfsetfillcolor{currentfill}%
\pgfsetlinewidth{0.803000pt}%
\definecolor{currentstroke}{rgb}{0.000000,0.000000,0.000000}%
\pgfsetstrokecolor{currentstroke}%
\pgfsetdash{}{0pt}%
\pgfsys@defobject{currentmarker}{\pgfqpoint{0.000000in}{-0.048611in}}{\pgfqpoint{0.000000in}{0.000000in}}{%
\pgfpathmoveto{\pgfqpoint{0.000000in}{0.000000in}}%
\pgfpathlineto{\pgfqpoint{0.000000in}{-0.048611in}}%
\pgfusepath{stroke,fill}%
}%
\begin{pgfscope}%
\pgfsys@transformshift{4.427933in}{2.544000in}%
\pgfsys@useobject{currentmarker}{}%
\end{pgfscope}%
\end{pgfscope}%
\begin{pgfscope}%
\definecolor{textcolor}{rgb}{0.000000,0.000000,0.000000}%
\pgfsetstrokecolor{textcolor}%
\pgfsetfillcolor{textcolor}%
\pgftext[x=4.427933in,y=2.446778in,,top]{\color{textcolor}\rmfamily\fontsize{10.000000}{12.000000}\selectfont \(\displaystyle {2000}\)}%
\end{pgfscope}%
\begin{pgfscope}%
\pgfsetbuttcap%
\pgfsetroundjoin%
\definecolor{currentfill}{rgb}{0.000000,0.000000,0.000000}%
\pgfsetfillcolor{currentfill}%
\pgfsetlinewidth{0.803000pt}%
\definecolor{currentstroke}{rgb}{0.000000,0.000000,0.000000}%
\pgfsetstrokecolor{currentstroke}%
\pgfsetdash{}{0pt}%
\pgfsys@defobject{currentmarker}{\pgfqpoint{0.000000in}{-0.048611in}}{\pgfqpoint{0.000000in}{0.000000in}}{%
\pgfpathmoveto{\pgfqpoint{0.000000in}{0.000000in}}%
\pgfpathlineto{\pgfqpoint{0.000000in}{-0.048611in}}%
\pgfusepath{stroke,fill}%
}%
\begin{pgfscope}%
\pgfsys@transformshift{5.247931in}{2.544000in}%
\pgfsys@useobject{currentmarker}{}%
\end{pgfscope}%
\end{pgfscope}%
\begin{pgfscope}%
\definecolor{textcolor}{rgb}{0.000000,0.000000,0.000000}%
\pgfsetstrokecolor{textcolor}%
\pgfsetfillcolor{textcolor}%
\pgftext[x=5.247931in,y=2.446778in,,top]{\color{textcolor}\rmfamily\fontsize{10.000000}{12.000000}\selectfont \(\displaystyle {4000}\)}%
\end{pgfscope}%
\begin{pgfscope}%
\pgfsetbuttcap%
\pgfsetroundjoin%
\definecolor{currentfill}{rgb}{0.000000,0.000000,0.000000}%
\pgfsetfillcolor{currentfill}%
\pgfsetlinewidth{0.803000pt}%
\definecolor{currentstroke}{rgb}{0.000000,0.000000,0.000000}%
\pgfsetstrokecolor{currentstroke}%
\pgfsetdash{}{0pt}%
\pgfsys@defobject{currentmarker}{\pgfqpoint{-0.048611in}{0.000000in}}{\pgfqpoint{-0.000000in}{0.000000in}}{%
\pgfpathmoveto{\pgfqpoint{-0.000000in}{0.000000in}}%
\pgfpathlineto{\pgfqpoint{-0.048611in}{0.000000in}}%
\pgfusepath{stroke,fill}%
}%
\begin{pgfscope}%
\pgfsys@transformshift{3.505455in}{2.590774in}%
\pgfsys@useobject{currentmarker}{}%
\end{pgfscope}%
\end{pgfscope}%
\begin{pgfscope}%
\definecolor{textcolor}{rgb}{0.000000,0.000000,0.000000}%
\pgfsetstrokecolor{textcolor}%
\pgfsetfillcolor{textcolor}%
\pgftext[x=3.161318in, y=2.542549in, left, base]{\color{textcolor}\rmfamily\fontsize{10.000000}{12.000000}\selectfont \(\displaystyle {0.00}\)}%
\end{pgfscope}%
\begin{pgfscope}%
\pgfsetbuttcap%
\pgfsetroundjoin%
\definecolor{currentfill}{rgb}{0.000000,0.000000,0.000000}%
\pgfsetfillcolor{currentfill}%
\pgfsetlinewidth{0.803000pt}%
\definecolor{currentstroke}{rgb}{0.000000,0.000000,0.000000}%
\pgfsetstrokecolor{currentstroke}%
\pgfsetdash{}{0pt}%
\pgfsys@defobject{currentmarker}{\pgfqpoint{-0.048611in}{0.000000in}}{\pgfqpoint{-0.000000in}{0.000000in}}{%
\pgfpathmoveto{\pgfqpoint{-0.000000in}{0.000000in}}%
\pgfpathlineto{\pgfqpoint{-0.048611in}{0.000000in}}%
\pgfusepath{stroke,fill}%
}%
\begin{pgfscope}%
\pgfsys@transformshift{3.505455in}{2.999081in}%
\pgfsys@useobject{currentmarker}{}%
\end{pgfscope}%
\end{pgfscope}%
\begin{pgfscope}%
\definecolor{textcolor}{rgb}{0.000000,0.000000,0.000000}%
\pgfsetstrokecolor{textcolor}%
\pgfsetfillcolor{textcolor}%
\pgftext[x=3.161318in, y=2.950855in, left, base]{\color{textcolor}\rmfamily\fontsize{10.000000}{12.000000}\selectfont \(\displaystyle {0.05}\)}%
\end{pgfscope}%
\begin{pgfscope}%
\pgfsetbuttcap%
\pgfsetroundjoin%
\definecolor{currentfill}{rgb}{0.000000,0.000000,0.000000}%
\pgfsetfillcolor{currentfill}%
\pgfsetlinewidth{0.803000pt}%
\definecolor{currentstroke}{rgb}{0.000000,0.000000,0.000000}%
\pgfsetstrokecolor{currentstroke}%
\pgfsetdash{}{0pt}%
\pgfsys@defobject{currentmarker}{\pgfqpoint{-0.048611in}{0.000000in}}{\pgfqpoint{-0.000000in}{0.000000in}}{%
\pgfpathmoveto{\pgfqpoint{-0.000000in}{0.000000in}}%
\pgfpathlineto{\pgfqpoint{-0.048611in}{0.000000in}}%
\pgfusepath{stroke,fill}%
}%
\begin{pgfscope}%
\pgfsys@transformshift{3.505455in}{3.407387in}%
\pgfsys@useobject{currentmarker}{}%
\end{pgfscope}%
\end{pgfscope}%
\begin{pgfscope}%
\definecolor{textcolor}{rgb}{0.000000,0.000000,0.000000}%
\pgfsetstrokecolor{textcolor}%
\pgfsetfillcolor{textcolor}%
\pgftext[x=3.161318in, y=3.359162in, left, base]{\color{textcolor}\rmfamily\fontsize{10.000000}{12.000000}\selectfont \(\displaystyle {0.10}\)}%
\end{pgfscope}%
\begin{pgfscope}%
\pgfsetbuttcap%
\pgfsetroundjoin%
\definecolor{currentfill}{rgb}{0.000000,0.000000,0.000000}%
\pgfsetfillcolor{currentfill}%
\pgfsetlinewidth{0.803000pt}%
\definecolor{currentstroke}{rgb}{0.000000,0.000000,0.000000}%
\pgfsetstrokecolor{currentstroke}%
\pgfsetdash{}{0pt}%
\pgfsys@defobject{currentmarker}{\pgfqpoint{-0.048611in}{0.000000in}}{\pgfqpoint{-0.000000in}{0.000000in}}{%
\pgfpathmoveto{\pgfqpoint{-0.000000in}{0.000000in}}%
\pgfpathlineto{\pgfqpoint{-0.048611in}{0.000000in}}%
\pgfusepath{stroke,fill}%
}%
\begin{pgfscope}%
\pgfsys@transformshift{3.505455in}{3.815694in}%
\pgfsys@useobject{currentmarker}{}%
\end{pgfscope}%
\end{pgfscope}%
\begin{pgfscope}%
\definecolor{textcolor}{rgb}{0.000000,0.000000,0.000000}%
\pgfsetstrokecolor{textcolor}%
\pgfsetfillcolor{textcolor}%
\pgftext[x=3.161318in, y=3.767468in, left, base]{\color{textcolor}\rmfamily\fontsize{10.000000}{12.000000}\selectfont \(\displaystyle {0.15}\)}%
\end{pgfscope}%
\begin{pgfscope}%
\pgfsetbuttcap%
\pgfsetroundjoin%
\definecolor{currentfill}{rgb}{0.000000,0.000000,0.000000}%
\pgfsetfillcolor{currentfill}%
\pgfsetlinewidth{0.803000pt}%
\definecolor{currentstroke}{rgb}{0.000000,0.000000,0.000000}%
\pgfsetstrokecolor{currentstroke}%
\pgfsetdash{}{0pt}%
\pgfsys@defobject{currentmarker}{\pgfqpoint{-0.048611in}{0.000000in}}{\pgfqpoint{-0.000000in}{0.000000in}}{%
\pgfpathmoveto{\pgfqpoint{-0.000000in}{0.000000in}}%
\pgfpathlineto{\pgfqpoint{-0.048611in}{0.000000in}}%
\pgfusepath{stroke,fill}%
}%
\begin{pgfscope}%
\pgfsys@transformshift{3.505455in}{4.224000in}%
\pgfsys@useobject{currentmarker}{}%
\end{pgfscope}%
\end{pgfscope}%
\begin{pgfscope}%
\definecolor{textcolor}{rgb}{0.000000,0.000000,0.000000}%
\pgfsetstrokecolor{textcolor}%
\pgfsetfillcolor{textcolor}%
\pgftext[x=3.161318in, y=4.175775in, left, base]{\color{textcolor}\rmfamily\fontsize{10.000000}{12.000000}\selectfont \(\displaystyle {0.20}\)}%
\end{pgfscope}%
\begin{pgfscope}%
\pgfpathrectangle{\pgfqpoint{3.505455in}{2.544000in}}{\pgfqpoint{2.254545in}{1.680000in}}%
\pgfusepath{clip}%
\pgfsetrectcap%
\pgfsetroundjoin%
\pgfsetlinewidth{1.505625pt}%
\definecolor{currentstroke}{rgb}{0.121569,0.466667,0.705882}%
\pgfsetstrokecolor{currentstroke}%
\pgfsetdash{}{0pt}%
\pgfpathmoveto{\pgfqpoint{3.607934in}{3.528978in}}%
\pgfpathlineto{\pgfqpoint{3.608754in}{3.514618in}}%
\pgfpathlineto{\pgfqpoint{3.611214in}{3.431665in}}%
\pgfpathlineto{\pgfqpoint{3.621874in}{3.014583in}}%
\pgfpathlineto{\pgfqpoint{3.627614in}{2.879514in}}%
\pgfpathlineto{\pgfqpoint{3.632534in}{2.810839in}}%
\pgfpathlineto{\pgfqpoint{3.637454in}{2.772145in}}%
\pgfpathlineto{\pgfqpoint{3.642784in}{2.747351in}}%
\pgfpathlineto{\pgfqpoint{3.650984in}{2.720604in}}%
\pgfpathlineto{\pgfqpoint{3.659594in}{2.698950in}}%
\pgfpathlineto{\pgfqpoint{3.667384in}{2.685156in}}%
\pgfpathlineto{\pgfqpoint{3.677224in}{2.672654in}}%
\pgfpathlineto{\pgfqpoint{3.689114in}{2.661059in}}%
\pgfpathlineto{\pgfqpoint{3.701414in}{2.652264in}}%
\pgfpathlineto{\pgfqpoint{3.717404in}{2.643669in}}%
\pgfpathlineto{\pgfqpoint{3.735854in}{2.636396in}}%
\pgfpathlineto{\pgfqpoint{3.758814in}{2.629760in}}%
\pgfpathlineto{\pgfqpoint{3.788334in}{2.623659in}}%
\pgfpathlineto{\pgfqpoint{3.826464in}{2.618165in}}%
\pgfpathlineto{\pgfqpoint{3.877303in}{2.613208in}}%
\pgfpathlineto{\pgfqpoint{3.947003in}{2.608778in}}%
\pgfpathlineto{\pgfqpoint{4.046223in}{2.604853in}}%
\pgfpathlineto{\pgfqpoint{4.193003in}{2.601442in}}%
\pgfpathlineto{\pgfqpoint{4.422603in}{2.598529in}}%
\pgfpathlineto{\pgfqpoint{4.806772in}{2.596111in}}%
\pgfpathlineto{\pgfqpoint{5.512381in}{2.594179in}}%
\pgfpathlineto{\pgfqpoint{5.657521in}{2.593944in}}%
\pgfpathlineto{\pgfqpoint{5.657521in}{2.593944in}}%
\pgfusepath{stroke}%
\end{pgfscope}%
\begin{pgfscope}%
\pgfpathrectangle{\pgfqpoint{3.505455in}{2.544000in}}{\pgfqpoint{2.254545in}{1.680000in}}%
\pgfusepath{clip}%
\pgfsetrectcap%
\pgfsetroundjoin%
\pgfsetlinewidth{1.505625pt}%
\definecolor{currentstroke}{rgb}{1.000000,0.498039,0.054902}%
\pgfsetstrokecolor{currentstroke}%
\pgfsetdash{}{0pt}%
\pgfpathmoveto{\pgfqpoint{3.607934in}{3.528978in}}%
\pgfpathlineto{\pgfqpoint{3.609574in}{3.518903in}}%
\pgfpathlineto{\pgfqpoint{3.612444in}{3.488557in}}%
\pgfpathlineto{\pgfqpoint{3.618184in}{3.395065in}}%
\pgfpathlineto{\pgfqpoint{3.628844in}{3.225446in}}%
\pgfpathlineto{\pgfqpoint{3.638274in}{3.114324in}}%
\pgfpathlineto{\pgfqpoint{3.659594in}{2.897755in}}%
\pgfpathlineto{\pgfqpoint{3.670254in}{2.807824in}}%
\pgfpathlineto{\pgfqpoint{3.678864in}{2.753118in}}%
\pgfpathlineto{\pgfqpoint{3.686244in}{2.719779in}}%
\pgfpathlineto{\pgfqpoint{3.692804in}{2.699520in}}%
\pgfpathlineto{\pgfqpoint{3.698954in}{2.687066in}}%
\pgfpathlineto{\pgfqpoint{3.705104in}{2.679376in}}%
\pgfpathlineto{\pgfqpoint{3.711664in}{2.674715in}}%
\pgfpathlineto{\pgfqpoint{3.720274in}{2.671539in}}%
\pgfpathlineto{\pgfqpoint{3.742004in}{2.664539in}}%
\pgfpathlineto{\pgfqpoint{3.755124in}{2.657158in}}%
\pgfpathlineto{\pgfqpoint{3.783824in}{2.640180in}}%
\pgfpathlineto{\pgfqpoint{3.796944in}{2.635639in}}%
\pgfpathlineto{\pgfqpoint{3.813344in}{2.632557in}}%
\pgfpathlineto{\pgfqpoint{3.858443in}{2.625092in}}%
\pgfpathlineto{\pgfqpoint{3.889193in}{2.620133in}}%
\pgfpathlineto{\pgfqpoint{3.928963in}{2.616770in}}%
\pgfpathlineto{\pgfqpoint{4.015473in}{2.610657in}}%
\pgfpathlineto{\pgfqpoint{4.146263in}{2.605411in}}%
\pgfpathlineto{\pgfqpoint{4.281563in}{2.602289in}}%
\pgfpathlineto{\pgfqpoint{4.532482in}{2.598997in}}%
\pgfpathlineto{\pgfqpoint{4.921572in}{2.596460in}}%
\pgfpathlineto{\pgfqpoint{5.619391in}{2.594424in}}%
\pgfpathlineto{\pgfqpoint{5.657521in}{2.594354in}}%
\pgfpathlineto{\pgfqpoint{5.657521in}{2.594354in}}%
\pgfusepath{stroke}%
\end{pgfscope}%
\begin{pgfscope}%
\pgfpathrectangle{\pgfqpoint{3.505455in}{2.544000in}}{\pgfqpoint{2.254545in}{1.680000in}}%
\pgfusepath{clip}%
\pgfsetrectcap%
\pgfsetroundjoin%
\pgfsetlinewidth{1.505625pt}%
\definecolor{currentstroke}{rgb}{0.172549,0.627451,0.172549}%
\pgfsetstrokecolor{currentstroke}%
\pgfsetdash{}{0pt}%
\pgfpathmoveto{\pgfqpoint{3.607934in}{3.528978in}}%
\pgfpathlineto{\pgfqpoint{3.610394in}{3.496928in}}%
\pgfpathlineto{\pgfqpoint{3.614904in}{3.393245in}}%
\pgfpathlineto{\pgfqpoint{3.622694in}{3.222544in}}%
\pgfpathlineto{\pgfqpoint{3.630894in}{3.096263in}}%
\pgfpathlineto{\pgfqpoint{3.647704in}{2.876199in}}%
\pgfpathlineto{\pgfqpoint{3.655904in}{2.796305in}}%
\pgfpathlineto{\pgfqpoint{3.662464in}{2.752271in}}%
\pgfpathlineto{\pgfqpoint{3.668614in}{2.725299in}}%
\pgfpathlineto{\pgfqpoint{3.674354in}{2.709590in}}%
\pgfpathlineto{\pgfqpoint{3.680094in}{2.699924in}}%
\pgfpathlineto{\pgfqpoint{3.687064in}{2.692564in}}%
\pgfpathlineto{\pgfqpoint{3.736264in}{2.651428in}}%
\pgfpathlineto{\pgfqpoint{3.748154in}{2.646017in}}%
\pgfpathlineto{\pgfqpoint{3.772344in}{2.638352in}}%
\pgfpathlineto{\pgfqpoint{3.805144in}{2.629253in}}%
\pgfpathlineto{\pgfqpoint{3.834664in}{2.624235in}}%
\pgfpathlineto{\pgfqpoint{3.883043in}{2.617899in}}%
\pgfpathlineto{\pgfqpoint{3.950693in}{2.612300in}}%
\pgfpathlineto{\pgfqpoint{4.040483in}{2.607652in}}%
\pgfpathlineto{\pgfqpoint{4.158973in}{2.603895in}}%
\pgfpathlineto{\pgfqpoint{4.338553in}{2.600578in}}%
\pgfpathlineto{\pgfqpoint{4.622272in}{2.597776in}}%
\pgfpathlineto{\pgfqpoint{5.104022in}{2.595486in}}%
\pgfpathlineto{\pgfqpoint{5.657521in}{2.594197in}}%
\pgfpathlineto{\pgfqpoint{5.657521in}{2.594197in}}%
\pgfusepath{stroke}%
\end{pgfscope}%
\begin{pgfscope}%
\pgfpathrectangle{\pgfqpoint{3.505455in}{2.544000in}}{\pgfqpoint{2.254545in}{1.680000in}}%
\pgfusepath{clip}%
\pgfsetrectcap%
\pgfsetroundjoin%
\pgfsetlinewidth{1.505625pt}%
\definecolor{currentstroke}{rgb}{0.839216,0.152941,0.156863}%
\pgfsetstrokecolor{currentstroke}%
\pgfsetdash{}{0pt}%
\pgfpathmoveto{\pgfqpoint{3.607934in}{3.528978in}}%
\pgfpathlineto{\pgfqpoint{3.609164in}{3.490242in}}%
\pgfpathlineto{\pgfqpoint{3.612034in}{3.348239in}}%
\pgfpathlineto{\pgfqpoint{3.621874in}{2.798022in}}%
\pgfpathlineto{\pgfqpoint{3.626384in}{2.674477in}}%
\pgfpathlineto{\pgfqpoint{3.630074in}{2.625458in}}%
\pgfpathlineto{\pgfqpoint{3.632944in}{2.609059in}}%
\pgfpathlineto{\pgfqpoint{3.634994in}{2.606395in}}%
\pgfpathlineto{\pgfqpoint{3.636634in}{2.607920in}}%
\pgfpathlineto{\pgfqpoint{3.644014in}{2.618141in}}%
\pgfpathlineto{\pgfqpoint{3.646474in}{2.616934in}}%
\pgfpathlineto{\pgfqpoint{3.650164in}{2.611688in}}%
\pgfpathlineto{\pgfqpoint{3.657954in}{2.600369in}}%
\pgfpathlineto{\pgfqpoint{3.661234in}{2.599162in}}%
\pgfpathlineto{\pgfqpoint{3.665744in}{2.600308in}}%
\pgfpathlineto{\pgfqpoint{3.671484in}{2.601321in}}%
\pgfpathlineto{\pgfqpoint{3.676814in}{2.599684in}}%
\pgfpathlineto{\pgfqpoint{3.686244in}{2.596382in}}%
\pgfpathlineto{\pgfqpoint{3.694854in}{2.596854in}}%
\pgfpathlineto{\pgfqpoint{3.703054in}{2.596181in}}%
\pgfpathlineto{\pgfqpoint{3.716174in}{2.594852in}}%
\pgfpathlineto{\pgfqpoint{3.733394in}{2.594217in}}%
\pgfpathlineto{\pgfqpoint{3.752664in}{2.593821in}}%
\pgfpathlineto{\pgfqpoint{3.803914in}{2.592814in}}%
\pgfpathlineto{\pgfqpoint{3.928553in}{2.591880in}}%
\pgfpathlineto{\pgfqpoint{4.427523in}{2.591146in}}%
\pgfpathlineto{\pgfqpoint{5.657521in}{2.590909in}}%
\pgfpathlineto{\pgfqpoint{5.657521in}{2.590909in}}%
\pgfusepath{stroke}%
\end{pgfscope}%
\begin{pgfscope}%
\pgfpathrectangle{\pgfqpoint{3.505455in}{2.544000in}}{\pgfqpoint{2.254545in}{1.680000in}}%
\pgfusepath{clip}%
\pgfsetrectcap%
\pgfsetroundjoin%
\pgfsetlinewidth{1.505625pt}%
\definecolor{currentstroke}{rgb}{0.580392,0.403922,0.741176}%
\pgfsetstrokecolor{currentstroke}%
\pgfsetdash{}{0pt}%
\pgfpathmoveto{\pgfqpoint{3.607934in}{3.528978in}}%
\pgfpathlineto{\pgfqpoint{3.614494in}{3.016579in}}%
\pgfpathlineto{\pgfqpoint{3.619414in}{2.734086in}}%
\pgfpathlineto{\pgfqpoint{3.622694in}{2.647114in}}%
\pgfpathlineto{\pgfqpoint{3.625564in}{2.619182in}}%
\pgfpathlineto{\pgfqpoint{3.627614in}{2.615841in}}%
\pgfpathlineto{\pgfqpoint{3.629664in}{2.618447in}}%
\pgfpathlineto{\pgfqpoint{3.632944in}{2.621745in}}%
\pgfpathlineto{\pgfqpoint{3.634584in}{2.620799in}}%
\pgfpathlineto{\pgfqpoint{3.637454in}{2.615653in}}%
\pgfpathlineto{\pgfqpoint{3.643604in}{2.604339in}}%
\pgfpathlineto{\pgfqpoint{3.646474in}{2.603205in}}%
\pgfpathlineto{\pgfqpoint{3.655494in}{2.602407in}}%
\pgfpathlineto{\pgfqpoint{3.664514in}{2.598694in}}%
\pgfpathlineto{\pgfqpoint{3.756764in}{2.593253in}}%
\pgfpathlineto{\pgfqpoint{3.896573in}{2.591907in}}%
\pgfpathlineto{\pgfqpoint{4.290993in}{2.591201in}}%
\pgfpathlineto{\pgfqpoint{5.657521in}{2.590904in}}%
\pgfpathlineto{\pgfqpoint{5.657521in}{2.590904in}}%
\pgfusepath{stroke}%
\end{pgfscope}%
\begin{pgfscope}%
\pgfpathrectangle{\pgfqpoint{3.505455in}{2.544000in}}{\pgfqpoint{2.254545in}{1.680000in}}%
\pgfusepath{clip}%
\pgfsetrectcap%
\pgfsetroundjoin%
\pgfsetlinewidth{1.505625pt}%
\definecolor{currentstroke}{rgb}{0.549020,0.337255,0.294118}%
\pgfsetstrokecolor{currentstroke}%
\pgfsetdash{}{0pt}%
\pgfpathmoveto{\pgfqpoint{3.607934in}{3.528978in}}%
\pgfpathlineto{\pgfqpoint{3.638684in}{3.527881in}}%
\pgfpathlineto{\pgfqpoint{3.655494in}{3.525157in}}%
\pgfpathlineto{\pgfqpoint{3.668204in}{3.520875in}}%
\pgfpathlineto{\pgfqpoint{3.679274in}{3.514764in}}%
\pgfpathlineto{\pgfqpoint{3.689524in}{3.506501in}}%
\pgfpathlineto{\pgfqpoint{3.699774in}{3.495206in}}%
\pgfpathlineto{\pgfqpoint{3.710024in}{3.480482in}}%
\pgfpathlineto{\pgfqpoint{3.721094in}{3.460520in}}%
\pgfpathlineto{\pgfqpoint{3.733804in}{3.432639in}}%
\pgfpathlineto{\pgfqpoint{3.749794in}{3.391671in}}%
\pgfpathlineto{\pgfqpoint{3.793254in}{3.277220in}}%
\pgfpathlineto{\pgfqpoint{3.807194in}{3.248241in}}%
\pgfpathlineto{\pgfqpoint{3.819904in}{3.226641in}}%
\pgfpathlineto{\pgfqpoint{3.832614in}{3.209284in}}%
\pgfpathlineto{\pgfqpoint{3.846554in}{3.194058in}}%
\pgfpathlineto{\pgfqpoint{3.868693in}{3.173838in}}%
\pgfpathlineto{\pgfqpoint{3.885093in}{3.157504in}}%
\pgfpathlineto{\pgfqpoint{3.896573in}{3.142895in}}%
\pgfpathlineto{\pgfqpoint{3.906823in}{3.126224in}}%
\pgfpathlineto{\pgfqpoint{3.917073in}{3.105059in}}%
\pgfpathlineto{\pgfqpoint{3.927323in}{3.078584in}}%
\pgfpathlineto{\pgfqpoint{3.938393in}{3.043621in}}%
\pgfpathlineto{\pgfqpoint{3.951513in}{2.994341in}}%
\pgfpathlineto{\pgfqpoint{3.972013in}{2.907119in}}%
\pgfpathlineto{\pgfqpoint{3.996203in}{2.806820in}}%
\pgfpathlineto{\pgfqpoint{4.010553in}{2.756451in}}%
\pgfpathlineto{\pgfqpoint{4.022853in}{2.720962in}}%
\pgfpathlineto{\pgfqpoint{4.033103in}{2.697566in}}%
\pgfpathlineto{\pgfqpoint{4.041713in}{2.682822in}}%
\pgfpathlineto{\pgfqpoint{4.049093in}{2.674294in}}%
\pgfpathlineto{\pgfqpoint{4.055243in}{2.670447in}}%
\pgfpathlineto{\pgfqpoint{4.060573in}{2.669581in}}%
\pgfpathlineto{\pgfqpoint{4.066313in}{2.670952in}}%
\pgfpathlineto{\pgfqpoint{4.073693in}{2.675226in}}%
\pgfpathlineto{\pgfqpoint{4.121253in}{2.707493in}}%
\pgfpathlineto{\pgfqpoint{4.136833in}{2.715240in}}%
\pgfpathlineto{\pgfqpoint{4.146263in}{2.717678in}}%
\pgfpathlineto{\pgfqpoint{4.154053in}{2.717447in}}%
\pgfpathlineto{\pgfqpoint{4.161023in}{2.714964in}}%
\pgfpathlineto{\pgfqpoint{4.167993in}{2.710100in}}%
\pgfpathlineto{\pgfqpoint{4.176193in}{2.701582in}}%
\pgfpathlineto{\pgfqpoint{4.188493in}{2.685105in}}%
\pgfpathlineto{\pgfqpoint{4.213913in}{2.650513in}}%
\pgfpathlineto{\pgfqpoint{4.224163in}{2.640486in}}%
\pgfpathlineto{\pgfqpoint{4.231133in}{2.636417in}}%
\pgfpathlineto{\pgfqpoint{4.236873in}{2.635414in}}%
\pgfpathlineto{\pgfqpoint{4.243023in}{2.636614in}}%
\pgfpathlineto{\pgfqpoint{4.253683in}{2.641496in}}%
\pgfpathlineto{\pgfqpoint{4.294683in}{2.661062in}}%
\pgfpathlineto{\pgfqpoint{4.303293in}{2.662534in}}%
\pgfpathlineto{\pgfqpoint{4.310263in}{2.661602in}}%
\pgfpathlineto{\pgfqpoint{4.317233in}{2.658431in}}%
\pgfpathlineto{\pgfqpoint{4.326253in}{2.651650in}}%
\pgfpathlineto{\pgfqpoint{4.359873in}{2.623801in}}%
\pgfpathlineto{\pgfqpoint{4.366433in}{2.621995in}}%
\pgfpathlineto{\pgfqpoint{4.372993in}{2.622671in}}%
\pgfpathlineto{\pgfqpoint{4.386113in}{2.626999in}}%
\pgfpathlineto{\pgfqpoint{4.418093in}{2.637834in}}%
\pgfpathlineto{\pgfqpoint{4.426703in}{2.637963in}}%
\pgfpathlineto{\pgfqpoint{4.434903in}{2.635827in}}%
\pgfpathlineto{\pgfqpoint{4.448433in}{2.629420in}}%
\pgfpathlineto{\pgfqpoint{4.463193in}{2.623380in}}%
\pgfpathlineto{\pgfqpoint{4.484513in}{2.617261in}}%
\pgfpathlineto{\pgfqpoint{4.490252in}{2.618286in}}%
\pgfpathlineto{\pgfqpoint{4.506242in}{2.622285in}}%
\pgfpathlineto{\pgfqpoint{4.521412in}{2.623127in}}%
\pgfpathlineto{\pgfqpoint{4.543142in}{2.623410in}}%
\pgfpathlineto{\pgfqpoint{4.553392in}{2.619785in}}%
\pgfpathlineto{\pgfqpoint{4.570202in}{2.613459in}}%
\pgfpathlineto{\pgfqpoint{4.593572in}{2.614193in}}%
\pgfpathlineto{\pgfqpoint{4.613252in}{2.618903in}}%
\pgfpathlineto{\pgfqpoint{4.623092in}{2.618454in}}%
\pgfpathlineto{\pgfqpoint{4.660402in}{2.611925in}}%
\pgfpathlineto{\pgfqpoint{4.672292in}{2.608440in}}%
\pgfpathlineto{\pgfqpoint{4.700172in}{2.609331in}}%
\pgfpathlineto{\pgfqpoint{4.731332in}{2.620005in}}%
\pgfpathlineto{\pgfqpoint{4.740352in}{2.621268in}}%
\pgfpathlineto{\pgfqpoint{4.752242in}{2.620415in}}%
\pgfpathlineto{\pgfqpoint{4.769872in}{2.617904in}}%
\pgfpathlineto{\pgfqpoint{4.784632in}{2.614081in}}%
\pgfpathlineto{\pgfqpoint{4.816612in}{2.602638in}}%
\pgfpathlineto{\pgfqpoint{4.830962in}{2.601535in}}%
\pgfpathlineto{\pgfqpoint{4.845312in}{2.598090in}}%
\pgfpathlineto{\pgfqpoint{4.855152in}{2.596664in}}%
\pgfpathlineto{\pgfqpoint{4.870732in}{2.596939in}}%
\pgfpathlineto{\pgfqpoint{4.893692in}{2.596109in}}%
\pgfpathlineto{\pgfqpoint{4.908042in}{2.596637in}}%
\pgfpathlineto{\pgfqpoint{4.921162in}{2.597415in}}%
\pgfpathlineto{\pgfqpoint{4.937152in}{2.599037in}}%
\pgfpathlineto{\pgfqpoint{4.949042in}{2.600745in}}%
\pgfpathlineto{\pgfqpoint{4.965442in}{2.603009in}}%
\pgfpathlineto{\pgfqpoint{4.978562in}{2.605907in}}%
\pgfpathlineto{\pgfqpoint{5.011362in}{2.612922in}}%
\pgfpathlineto{\pgfqpoint{5.100742in}{2.640608in}}%
\pgfpathlineto{\pgfqpoint{5.218821in}{2.688832in}}%
\pgfpathlineto{\pgfqpoint{5.375441in}{2.764234in}}%
\pgfpathlineto{\pgfqpoint{5.436121in}{2.795345in}}%
\pgfpathlineto{\pgfqpoint{5.461951in}{2.808766in}}%
\pgfpathlineto{\pgfqpoint{5.522631in}{2.840507in}}%
\pgfpathlineto{\pgfqpoint{5.555841in}{2.858041in}}%
\pgfpathlineto{\pgfqpoint{5.598891in}{2.880728in}}%
\pgfpathlineto{\pgfqpoint{5.657521in}{2.911686in}}%
\pgfpathlineto{\pgfqpoint{5.657521in}{2.911686in}}%
\pgfusepath{stroke}%
\end{pgfscope}%
\begin{pgfscope}%
\pgfpathrectangle{\pgfqpoint{3.505455in}{2.544000in}}{\pgfqpoint{2.254545in}{1.680000in}}%
\pgfusepath{clip}%
\pgfsetrectcap%
\pgfsetroundjoin%
\pgfsetlinewidth{1.505625pt}%
\definecolor{currentstroke}{rgb}{0.890196,0.466667,0.760784}%
\pgfsetstrokecolor{currentstroke}%
\pgfsetdash{}{0pt}%
\pgfpathmoveto{\pgfqpoint{3.607934in}{3.528978in}}%
\pgfpathlineto{\pgfqpoint{3.644014in}{3.527897in}}%
\pgfpathlineto{\pgfqpoint{3.659594in}{3.525303in}}%
\pgfpathlineto{\pgfqpoint{3.671894in}{3.521039in}}%
\pgfpathlineto{\pgfqpoint{3.682554in}{3.514980in}}%
\pgfpathlineto{\pgfqpoint{3.692394in}{3.506853in}}%
\pgfpathlineto{\pgfqpoint{3.702234in}{3.495810in}}%
\pgfpathlineto{\pgfqpoint{3.712484in}{3.480822in}}%
\pgfpathlineto{\pgfqpoint{3.723554in}{3.460442in}}%
\pgfpathlineto{\pgfqpoint{3.735854in}{3.432940in}}%
\pgfpathlineto{\pgfqpoint{3.751844in}{3.391228in}}%
\pgfpathlineto{\pgfqpoint{3.795304in}{3.274552in}}%
\pgfpathlineto{\pgfqpoint{3.809654in}{3.244041in}}%
\pgfpathlineto{\pgfqpoint{3.822774in}{3.221159in}}%
\pgfpathlineto{\pgfqpoint{3.836304in}{3.202045in}}%
\pgfpathlineto{\pgfqpoint{3.851474in}{3.184543in}}%
\pgfpathlineto{\pgfqpoint{3.889193in}{3.142922in}}%
\pgfpathlineto{\pgfqpoint{3.900673in}{3.125451in}}%
\pgfpathlineto{\pgfqpoint{3.911333in}{3.104994in}}%
\pgfpathlineto{\pgfqpoint{3.921993in}{3.079441in}}%
\pgfpathlineto{\pgfqpoint{3.933063in}{3.046873in}}%
\pgfpathlineto{\pgfqpoint{3.945773in}{3.002088in}}%
\pgfpathlineto{\pgfqpoint{3.962993in}{2.932227in}}%
\pgfpathlineto{\pgfqpoint{3.995383in}{2.799957in}}%
\pgfpathlineto{\pgfqpoint{4.009733in}{2.751455in}}%
\pgfpathlineto{\pgfqpoint{4.021623in}{2.718607in}}%
\pgfpathlineto{\pgfqpoint{4.031873in}{2.696381in}}%
\pgfpathlineto{\pgfqpoint{4.040483in}{2.682631in}}%
\pgfpathlineto{\pgfqpoint{4.047453in}{2.675239in}}%
\pgfpathlineto{\pgfqpoint{4.053603in}{2.671767in}}%
\pgfpathlineto{\pgfqpoint{4.059343in}{2.671109in}}%
\pgfpathlineto{\pgfqpoint{4.065493in}{2.672797in}}%
\pgfpathlineto{\pgfqpoint{4.074103in}{2.677832in}}%
\pgfpathlineto{\pgfqpoint{4.102803in}{2.696158in}}%
\pgfpathlineto{\pgfqpoint{4.123303in}{2.706219in}}%
\pgfpathlineto{\pgfqpoint{4.135193in}{2.709855in}}%
\pgfpathlineto{\pgfqpoint{4.144213in}{2.710362in}}%
\pgfpathlineto{\pgfqpoint{4.152003in}{2.708515in}}%
\pgfpathlineto{\pgfqpoint{4.159383in}{2.704440in}}%
\pgfpathlineto{\pgfqpoint{4.167583in}{2.697296in}}%
\pgfpathlineto{\pgfqpoint{4.178653in}{2.684405in}}%
\pgfpathlineto{\pgfqpoint{4.208993in}{2.647549in}}%
\pgfpathlineto{\pgfqpoint{4.218013in}{2.640630in}}%
\pgfpathlineto{\pgfqpoint{4.224983in}{2.638031in}}%
\pgfpathlineto{\pgfqpoint{4.231543in}{2.638111in}}%
\pgfpathlineto{\pgfqpoint{4.240153in}{2.640804in}}%
\pgfpathlineto{\pgfqpoint{4.278283in}{2.654951in}}%
\pgfpathlineto{\pgfqpoint{4.288123in}{2.655883in}}%
\pgfpathlineto{\pgfqpoint{4.296323in}{2.654367in}}%
\pgfpathlineto{\pgfqpoint{4.304523in}{2.650514in}}%
\pgfpathlineto{\pgfqpoint{4.316413in}{2.642092in}}%
\pgfpathlineto{\pgfqpoint{4.336093in}{2.628188in}}%
\pgfpathlineto{\pgfqpoint{4.343883in}{2.625339in}}%
\pgfpathlineto{\pgfqpoint{4.350853in}{2.625207in}}%
\pgfpathlineto{\pgfqpoint{4.361103in}{2.627718in}}%
\pgfpathlineto{\pgfqpoint{4.388983in}{2.635077in}}%
\pgfpathlineto{\pgfqpoint{4.398003in}{2.635023in}}%
\pgfpathlineto{\pgfqpoint{4.406613in}{2.632670in}}%
\pgfpathlineto{\pgfqpoint{4.418913in}{2.626547in}}%
\pgfpathlineto{\pgfqpoint{4.434083in}{2.619391in}}%
\pgfpathlineto{\pgfqpoint{4.441463in}{2.618262in}}%
\pgfpathlineto{\pgfqpoint{4.450073in}{2.619401in}}%
\pgfpathlineto{\pgfqpoint{4.479593in}{2.624743in}}%
\pgfpathlineto{\pgfqpoint{4.488612in}{2.623364in}}%
\pgfpathlineto{\pgfqpoint{4.500912in}{2.618805in}}%
\pgfpathlineto{\pgfqpoint{4.514032in}{2.614496in}}%
\pgfpathlineto{\pgfqpoint{4.521822in}{2.614336in}}%
\pgfpathlineto{\pgfqpoint{4.537812in}{2.617157in}}%
\pgfpathlineto{\pgfqpoint{4.550932in}{2.618386in}}%
\pgfpathlineto{\pgfqpoint{4.560362in}{2.616947in}}%
\pgfpathlineto{\pgfqpoint{4.585782in}{2.611452in}}%
\pgfpathlineto{\pgfqpoint{4.601362in}{2.613570in}}%
\pgfpathlineto{\pgfqpoint{4.613662in}{2.614094in}}%
\pgfpathlineto{\pgfqpoint{4.624732in}{2.612029in}}%
\pgfpathlineto{\pgfqpoint{4.639492in}{2.609400in}}%
\pgfpathlineto{\pgfqpoint{4.651382in}{2.610412in}}%
\pgfpathlineto{\pgfqpoint{4.666552in}{2.611181in}}%
\pgfpathlineto{\pgfqpoint{4.679262in}{2.609082in}}%
\pgfpathlineto{\pgfqpoint{4.690742in}{2.607913in}}%
\pgfpathlineto{\pgfqpoint{4.723542in}{2.607714in}}%
\pgfpathlineto{\pgfqpoint{4.736252in}{2.606761in}}%
\pgfpathlineto{\pgfqpoint{4.762902in}{2.606674in}}%
\pgfpathlineto{\pgfqpoint{4.778072in}{2.605845in}}%
\pgfpathlineto{\pgfqpoint{4.799802in}{2.605677in}}%
\pgfpathlineto{\pgfqpoint{4.816612in}{2.605071in}}%
\pgfpathlineto{\pgfqpoint{4.835882in}{2.604674in}}%
\pgfpathlineto{\pgfqpoint{4.853512in}{2.604396in}}%
\pgfpathlineto{\pgfqpoint{4.875652in}{2.603652in}}%
\pgfpathlineto{\pgfqpoint{5.060152in}{2.600708in}}%
\pgfpathlineto{\pgfqpoint{5.343461in}{2.597717in}}%
\pgfpathlineto{\pgfqpoint{5.657521in}{2.595749in}}%
\pgfpathlineto{\pgfqpoint{5.657521in}{2.595749in}}%
\pgfusepath{stroke}%
\end{pgfscope}%
\begin{pgfscope}%
\pgfpathrectangle{\pgfqpoint{3.505455in}{2.544000in}}{\pgfqpoint{2.254545in}{1.680000in}}%
\pgfusepath{clip}%
\pgfsetrectcap%
\pgfsetroundjoin%
\pgfsetlinewidth{1.505625pt}%
\definecolor{currentstroke}{rgb}{0.498039,0.498039,0.498039}%
\pgfsetstrokecolor{currentstroke}%
\pgfsetdash{}{0pt}%
\pgfpathmoveto{\pgfqpoint{3.607934in}{3.528978in}}%
\pgfpathlineto{\pgfqpoint{3.620234in}{3.527873in}}%
\pgfpathlineto{\pgfqpoint{3.630484in}{3.524728in}}%
\pgfpathlineto{\pgfqpoint{3.639094in}{3.519695in}}%
\pgfpathlineto{\pgfqpoint{3.646884in}{3.512453in}}%
\pgfpathlineto{\pgfqpoint{3.654264in}{3.502521in}}%
\pgfpathlineto{\pgfqpoint{3.662054in}{3.488101in}}%
\pgfpathlineto{\pgfqpoint{3.670254in}{3.467965in}}%
\pgfpathlineto{\pgfqpoint{3.679274in}{3.439678in}}%
\pgfpathlineto{\pgfqpoint{3.689934in}{3.398599in}}%
\pgfpathlineto{\pgfqpoint{3.705104in}{3.330096in}}%
\pgfpathlineto{\pgfqpoint{3.758814in}{3.069830in}}%
\pgfpathlineto{\pgfqpoint{3.774394in}{2.980958in}}%
\pgfpathlineto{\pgfqpoint{3.813754in}{2.748302in}}%
\pgfpathlineto{\pgfqpoint{3.821134in}{2.720754in}}%
\pgfpathlineto{\pgfqpoint{3.826874in}{2.706803in}}%
\pgfpathlineto{\pgfqpoint{3.831384in}{2.700783in}}%
\pgfpathlineto{\pgfqpoint{3.835074in}{2.698951in}}%
\pgfpathlineto{\pgfqpoint{3.838764in}{2.699548in}}%
\pgfpathlineto{\pgfqpoint{3.843274in}{2.702807in}}%
\pgfpathlineto{\pgfqpoint{3.851884in}{2.712675in}}%
\pgfpathlineto{\pgfqpoint{3.861313in}{2.722255in}}%
\pgfpathlineto{\pgfqpoint{3.868283in}{2.726199in}}%
\pgfpathlineto{\pgfqpoint{3.874433in}{2.727140in}}%
\pgfpathlineto{\pgfqpoint{3.880173in}{2.725720in}}%
\pgfpathlineto{\pgfqpoint{3.885913in}{2.721881in}}%
\pgfpathlineto{\pgfqpoint{3.892063in}{2.714959in}}%
\pgfpathlineto{\pgfqpoint{3.899853in}{2.702392in}}%
\pgfpathlineto{\pgfqpoint{3.921993in}{2.663736in}}%
\pgfpathlineto{\pgfqpoint{3.926503in}{2.660882in}}%
\pgfpathlineto{\pgfqpoint{3.930603in}{2.660634in}}%
\pgfpathlineto{\pgfqpoint{3.935933in}{2.662811in}}%
\pgfpathlineto{\pgfqpoint{3.953563in}{2.671997in}}%
\pgfpathlineto{\pgfqpoint{3.960123in}{2.672160in}}%
\pgfpathlineto{\pgfqpoint{3.966273in}{2.669926in}}%
\pgfpathlineto{\pgfqpoint{3.972833in}{2.664903in}}%
\pgfpathlineto{\pgfqpoint{3.983083in}{2.653496in}}%
\pgfpathlineto{\pgfqpoint{3.990873in}{2.646090in}}%
\pgfpathlineto{\pgfqpoint{3.995793in}{2.644156in}}%
\pgfpathlineto{\pgfqpoint{4.000713in}{2.644614in}}%
\pgfpathlineto{\pgfqpoint{4.011373in}{2.648900in}}%
\pgfpathlineto{\pgfqpoint{4.019163in}{2.650635in}}%
\pgfpathlineto{\pgfqpoint{4.025723in}{2.649866in}}%
\pgfpathlineto{\pgfqpoint{4.032283in}{2.646723in}}%
\pgfpathlineto{\pgfqpoint{4.052373in}{2.634634in}}%
\pgfpathlineto{\pgfqpoint{4.058523in}{2.635701in}}%
\pgfpathlineto{\pgfqpoint{4.072053in}{2.638811in}}%
\pgfpathlineto{\pgfqpoint{4.078613in}{2.637878in}}%
\pgfpathlineto{\pgfqpoint{4.085993in}{2.634424in}}%
\pgfpathlineto{\pgfqpoint{4.097883in}{2.628588in}}%
\pgfpathlineto{\pgfqpoint{4.103623in}{2.628742in}}%
\pgfpathlineto{\pgfqpoint{4.120843in}{2.631043in}}%
\pgfpathlineto{\pgfqpoint{4.128223in}{2.628791in}}%
\pgfpathlineto{\pgfqpoint{4.141753in}{2.623925in}}%
\pgfpathlineto{\pgfqpoint{4.149133in}{2.624891in}}%
\pgfpathlineto{\pgfqpoint{4.159383in}{2.625964in}}%
\pgfpathlineto{\pgfqpoint{4.166763in}{2.624379in}}%
\pgfpathlineto{\pgfqpoint{4.180703in}{2.620542in}}%
\pgfpathlineto{\pgfqpoint{4.203663in}{2.620454in}}%
\pgfpathlineto{\pgfqpoint{4.214733in}{2.617864in}}%
\pgfpathlineto{\pgfqpoint{4.238103in}{2.617261in}}%
\pgfpathlineto{\pgfqpoint{4.247123in}{2.615712in}}%
\pgfpathlineto{\pgfqpoint{4.267623in}{2.615368in}}%
\pgfpathlineto{\pgfqpoint{4.277463in}{2.613946in}}%
\pgfpathlineto{\pgfqpoint{4.294683in}{2.613891in}}%
\pgfpathlineto{\pgfqpoint{4.306163in}{2.612485in}}%
\pgfpathlineto{\pgfqpoint{4.320513in}{2.612527in}}%
\pgfpathlineto{\pgfqpoint{4.333633in}{2.611284in}}%
\pgfpathlineto{\pgfqpoint{4.345523in}{2.611242in}}%
\pgfpathlineto{\pgfqpoint{4.359053in}{2.610202in}}%
\pgfpathlineto{\pgfqpoint{4.370123in}{2.609998in}}%
\pgfpathlineto{\pgfqpoint{4.382013in}{2.609166in}}%
\pgfpathlineto{\pgfqpoint{4.393493in}{2.608918in}}%
\pgfpathlineto{\pgfqpoint{4.404563in}{2.608298in}}%
\pgfpathlineto{\pgfqpoint{4.415633in}{2.607990in}}%
\pgfpathlineto{\pgfqpoint{4.426293in}{2.607536in}}%
\pgfpathlineto{\pgfqpoint{4.437363in}{2.607095in}}%
\pgfpathlineto{\pgfqpoint{4.447203in}{2.606854in}}%
\pgfpathlineto{\pgfqpoint{4.457863in}{2.606344in}}%
\pgfpathlineto{\pgfqpoint{4.467703in}{2.606252in}}%
\pgfpathlineto{\pgfqpoint{4.479183in}{2.605539in}}%
\pgfpathlineto{\pgfqpoint{4.505422in}{2.605150in}}%
\pgfpathlineto{\pgfqpoint{4.528792in}{2.604368in}}%
\pgfpathlineto{\pgfqpoint{4.542322in}{2.604162in}}%
\pgfpathlineto{\pgfqpoint{4.586192in}{2.603036in}}%
\pgfpathlineto{\pgfqpoint{4.609972in}{2.602313in}}%
\pgfpathlineto{\pgfqpoint{4.802262in}{2.599262in}}%
\pgfpathlineto{\pgfqpoint{4.849412in}{2.598690in}}%
\pgfpathlineto{\pgfqpoint{4.926082in}{2.597846in}}%
\pgfpathlineto{\pgfqpoint{4.974462in}{2.597412in}}%
\pgfpathlineto{\pgfqpoint{5.252851in}{2.595515in}}%
\pgfpathlineto{\pgfqpoint{5.368471in}{2.595004in}}%
\pgfpathlineto{\pgfqpoint{5.454161in}{2.594702in}}%
\pgfpathlineto{\pgfqpoint{5.560761in}{2.594394in}}%
\pgfpathlineto{\pgfqpoint{5.657521in}{2.594217in}}%
\pgfpathlineto{\pgfqpoint{5.657521in}{2.594217in}}%
\pgfusepath{stroke}%
\end{pgfscope}%
\begin{pgfscope}%
\pgfpathrectangle{\pgfqpoint{3.505455in}{2.544000in}}{\pgfqpoint{2.254545in}{1.680000in}}%
\pgfusepath{clip}%
\pgfsetrectcap%
\pgfsetroundjoin%
\pgfsetlinewidth{1.505625pt}%
\definecolor{currentstroke}{rgb}{0.737255,0.741176,0.133333}%
\pgfsetstrokecolor{currentstroke}%
\pgfsetdash{}{0pt}%
\pgfpathmoveto{\pgfqpoint{3.607934in}{3.528978in}}%
\pgfpathlineto{\pgfqpoint{3.625154in}{3.527912in}}%
\pgfpathlineto{\pgfqpoint{3.635814in}{3.525201in}}%
\pgfpathlineto{\pgfqpoint{3.644014in}{3.520878in}}%
\pgfpathlineto{\pgfqpoint{3.651394in}{3.514420in}}%
\pgfpathlineto{\pgfqpoint{3.658364in}{3.505329in}}%
\pgfpathlineto{\pgfqpoint{3.665744in}{3.491784in}}%
\pgfpathlineto{\pgfqpoint{3.673534in}{3.472423in}}%
\pgfpathlineto{\pgfqpoint{3.682144in}{3.444512in}}%
\pgfpathlineto{\pgfqpoint{3.691984in}{3.404526in}}%
\pgfpathlineto{\pgfqpoint{3.704694in}{3.342789in}}%
\pgfpathlineto{\pgfqpoint{3.734214in}{3.183904in}}%
\pgfpathlineto{\pgfqpoint{3.764554in}{3.014555in}}%
\pgfpathlineto{\pgfqpoint{3.787514in}{2.870035in}}%
\pgfpathlineto{\pgfqpoint{3.804324in}{2.769714in}}%
\pgfpathlineto{\pgfqpoint{3.814164in}{2.723688in}}%
\pgfpathlineto{\pgfqpoint{3.821544in}{2.699214in}}%
\pgfpathlineto{\pgfqpoint{3.827284in}{2.687354in}}%
\pgfpathlineto{\pgfqpoint{3.831794in}{2.682681in}}%
\pgfpathlineto{\pgfqpoint{3.835484in}{2.681775in}}%
\pgfpathlineto{\pgfqpoint{3.839174in}{2.683178in}}%
\pgfpathlineto{\pgfqpoint{3.844094in}{2.687771in}}%
\pgfpathlineto{\pgfqpoint{3.863773in}{2.709306in}}%
\pgfpathlineto{\pgfqpoint{3.869513in}{2.711280in}}%
\pgfpathlineto{\pgfqpoint{3.875253in}{2.710971in}}%
\pgfpathlineto{\pgfqpoint{3.880993in}{2.708458in}}%
\pgfpathlineto{\pgfqpoint{3.887143in}{2.703286in}}%
\pgfpathlineto{\pgfqpoint{3.894113in}{2.694380in}}%
\pgfpathlineto{\pgfqpoint{3.903543in}{2.678337in}}%
\pgfpathlineto{\pgfqpoint{3.917893in}{2.654093in}}%
\pgfpathlineto{\pgfqpoint{3.923633in}{2.648636in}}%
\pgfpathlineto{\pgfqpoint{3.928143in}{2.647184in}}%
\pgfpathlineto{\pgfqpoint{3.932653in}{2.648096in}}%
\pgfpathlineto{\pgfqpoint{3.939623in}{2.652252in}}%
\pgfpathlineto{\pgfqpoint{3.949463in}{2.657578in}}%
\pgfpathlineto{\pgfqpoint{3.956023in}{2.658620in}}%
\pgfpathlineto{\pgfqpoint{3.962583in}{2.657351in}}%
\pgfpathlineto{\pgfqpoint{3.969143in}{2.653734in}}%
\pgfpathlineto{\pgfqpoint{3.977343in}{2.646461in}}%
\pgfpathlineto{\pgfqpoint{3.990463in}{2.634716in}}%
\pgfpathlineto{\pgfqpoint{3.995793in}{2.632989in}}%
\pgfpathlineto{\pgfqpoint{4.001123in}{2.633662in}}%
\pgfpathlineto{\pgfqpoint{4.020393in}{2.638428in}}%
\pgfpathlineto{\pgfqpoint{4.027363in}{2.636782in}}%
\pgfpathlineto{\pgfqpoint{4.035563in}{2.632381in}}%
\pgfpathlineto{\pgfqpoint{4.048273in}{2.625433in}}%
\pgfpathlineto{\pgfqpoint{4.054013in}{2.625016in}}%
\pgfpathlineto{\pgfqpoint{4.065083in}{2.627360in}}%
\pgfpathlineto{\pgfqpoint{4.073283in}{2.627681in}}%
\pgfpathlineto{\pgfqpoint{4.081073in}{2.625763in}}%
\pgfpathlineto{\pgfqpoint{4.100343in}{2.619669in}}%
\pgfpathlineto{\pgfqpoint{4.125763in}{2.619536in}}%
\pgfpathlineto{\pgfqpoint{4.141753in}{2.615900in}}%
\pgfpathlineto{\pgfqpoint{4.166763in}{2.615034in}}%
\pgfpathlineto{\pgfqpoint{4.179473in}{2.613056in}}%
\pgfpathlineto{\pgfqpoint{4.203253in}{2.611984in}}%
\pgfpathlineto{\pgfqpoint{4.215963in}{2.610829in}}%
\pgfpathlineto{\pgfqpoint{4.234003in}{2.610085in}}%
\pgfpathlineto{\pgfqpoint{4.249993in}{2.609015in}}%
\pgfpathlineto{\pgfqpoint{4.266803in}{2.608054in}}%
\pgfpathlineto{\pgfqpoint{4.284433in}{2.607397in}}%
\pgfpathlineto{\pgfqpoint{4.548882in}{2.599295in}}%
\pgfpathlineto{\pgfqpoint{4.777662in}{2.596291in}}%
\pgfpathlineto{\pgfqpoint{5.139281in}{2.593995in}}%
\pgfpathlineto{\pgfqpoint{5.657521in}{2.592573in}}%
\pgfpathlineto{\pgfqpoint{5.657521in}{2.592573in}}%
\pgfusepath{stroke}%
\end{pgfscope}%
\begin{pgfscope}%
\pgfpathrectangle{\pgfqpoint{3.505455in}{2.544000in}}{\pgfqpoint{2.254545in}{1.680000in}}%
\pgfusepath{clip}%
\pgfsetrectcap%
\pgfsetroundjoin%
\pgfsetlinewidth{1.505625pt}%
\definecolor{currentstroke}{rgb}{0.090196,0.745098,0.811765}%
\pgfsetstrokecolor{currentstroke}%
\pgfsetdash{}{0pt}%
\pgfpathmoveto{\pgfqpoint{3.607934in}{3.528978in}}%
\pgfpathlineto{\pgfqpoint{3.625154in}{3.527921in}}%
\pgfpathlineto{\pgfqpoint{3.635814in}{3.525213in}}%
\pgfpathlineto{\pgfqpoint{3.644014in}{3.520889in}}%
\pgfpathlineto{\pgfqpoint{3.651394in}{3.514428in}}%
\pgfpathlineto{\pgfqpoint{3.658364in}{3.505332in}}%
\pgfpathlineto{\pgfqpoint{3.665744in}{3.491780in}}%
\pgfpathlineto{\pgfqpoint{3.673534in}{3.472408in}}%
\pgfpathlineto{\pgfqpoint{3.682144in}{3.444485in}}%
\pgfpathlineto{\pgfqpoint{3.691984in}{3.404487in}}%
\pgfpathlineto{\pgfqpoint{3.704694in}{3.342741in}}%
\pgfpathlineto{\pgfqpoint{3.734214in}{3.183861in}}%
\pgfpathlineto{\pgfqpoint{3.764554in}{3.014531in}}%
\pgfpathlineto{\pgfqpoint{3.787514in}{2.870072in}}%
\pgfpathlineto{\pgfqpoint{3.804324in}{2.769820in}}%
\pgfpathlineto{\pgfqpoint{3.813754in}{2.725469in}}%
\pgfpathlineto{\pgfqpoint{3.821134in}{2.700474in}}%
\pgfpathlineto{\pgfqpoint{3.826874in}{2.688148in}}%
\pgfpathlineto{\pgfqpoint{3.831384in}{2.683096in}}%
\pgfpathlineto{\pgfqpoint{3.835074in}{2.681892in}}%
\pgfpathlineto{\pgfqpoint{3.838764in}{2.683032in}}%
\pgfpathlineto{\pgfqpoint{3.843684in}{2.687359in}}%
\pgfpathlineto{\pgfqpoint{3.864593in}{2.709587in}}%
\pgfpathlineto{\pgfqpoint{3.870333in}{2.711197in}}%
\pgfpathlineto{\pgfqpoint{3.876073in}{2.710550in}}%
\pgfpathlineto{\pgfqpoint{3.881813in}{2.707705in}}%
\pgfpathlineto{\pgfqpoint{3.887963in}{2.702184in}}%
\pgfpathlineto{\pgfqpoint{3.895343in}{2.692302in}}%
\pgfpathlineto{\pgfqpoint{3.906003in}{2.673658in}}%
\pgfpathlineto{\pgfqpoint{3.917483in}{2.654645in}}%
\pgfpathlineto{\pgfqpoint{3.923223in}{2.648985in}}%
\pgfpathlineto{\pgfqpoint{3.927733in}{2.647321in}}%
\pgfpathlineto{\pgfqpoint{3.932243in}{2.648025in}}%
\pgfpathlineto{\pgfqpoint{3.938803in}{2.651710in}}%
\pgfpathlineto{\pgfqpoint{3.949463in}{2.657447in}}%
\pgfpathlineto{\pgfqpoint{3.956023in}{2.658428in}}%
\pgfpathlineto{\pgfqpoint{3.962583in}{2.657120in}}%
\pgfpathlineto{\pgfqpoint{3.969143in}{2.653491in}}%
\pgfpathlineto{\pgfqpoint{3.977753in}{2.645857in}}%
\pgfpathlineto{\pgfqpoint{3.990053in}{2.634960in}}%
\pgfpathlineto{\pgfqpoint{3.995383in}{2.633100in}}%
\pgfpathlineto{\pgfqpoint{4.000713in}{2.633608in}}%
\pgfpathlineto{\pgfqpoint{4.020803in}{2.638185in}}%
\pgfpathlineto{\pgfqpoint{4.027773in}{2.636376in}}%
\pgfpathlineto{\pgfqpoint{4.036383in}{2.631654in}}%
\pgfpathlineto{\pgfqpoint{4.047863in}{2.625552in}}%
\pgfpathlineto{\pgfqpoint{4.054013in}{2.625061in}}%
\pgfpathlineto{\pgfqpoint{4.077793in}{2.626596in}}%
\pgfpathlineto{\pgfqpoint{4.088043in}{2.622607in}}%
\pgfpathlineto{\pgfqpoint{4.097063in}{2.619832in}}%
\pgfpathlineto{\pgfqpoint{4.104443in}{2.620051in}}%
\pgfpathlineto{\pgfqpoint{4.118383in}{2.620851in}}%
\pgfpathlineto{\pgfqpoint{4.127813in}{2.618728in}}%
\pgfpathlineto{\pgfqpoint{4.140523in}{2.615918in}}%
\pgfpathlineto{\pgfqpoint{4.171273in}{2.613894in}}%
\pgfpathlineto{\pgfqpoint{4.181113in}{2.613047in}}%
\pgfpathlineto{\pgfqpoint{4.199973in}{2.612411in}}%
\pgfpathlineto{\pgfqpoint{4.216783in}{2.610789in}}%
\pgfpathlineto{\pgfqpoint{4.234003in}{2.609968in}}%
\pgfpathlineto{\pgfqpoint{4.250813in}{2.608936in}}%
\pgfpathlineto{\pgfqpoint{4.269263in}{2.607763in}}%
\pgfpathlineto{\pgfqpoint{4.305343in}{2.606159in}}%
\pgfpathlineto{\pgfqpoint{4.365613in}{2.603865in}}%
\pgfpathlineto{\pgfqpoint{4.459913in}{2.601109in}}%
\pgfpathlineto{\pgfqpoint{4.652202in}{2.597659in}}%
\pgfpathlineto{\pgfqpoint{4.935922in}{2.595034in}}%
\pgfpathlineto{\pgfqpoint{5.414801in}{2.593076in}}%
\pgfpathlineto{\pgfqpoint{5.657521in}{2.592564in}}%
\pgfpathlineto{\pgfqpoint{5.657521in}{2.592564in}}%
\pgfusepath{stroke}%
\end{pgfscope}%
\begin{pgfscope}%
\pgfsetrectcap%
\pgfsetmiterjoin%
\pgfsetlinewidth{0.803000pt}%
\definecolor{currentstroke}{rgb}{0.000000,0.000000,0.000000}%
\pgfsetstrokecolor{currentstroke}%
\pgfsetdash{}{0pt}%
\pgfpathmoveto{\pgfqpoint{3.505455in}{2.544000in}}%
\pgfpathlineto{\pgfqpoint{3.505455in}{4.224000in}}%
\pgfusepath{stroke}%
\end{pgfscope}%
\begin{pgfscope}%
\pgfsetrectcap%
\pgfsetmiterjoin%
\pgfsetlinewidth{0.803000pt}%
\definecolor{currentstroke}{rgb}{0.000000,0.000000,0.000000}%
\pgfsetstrokecolor{currentstroke}%
\pgfsetdash{}{0pt}%
\pgfpathmoveto{\pgfqpoint{5.760000in}{2.544000in}}%
\pgfpathlineto{\pgfqpoint{5.760000in}{4.224000in}}%
\pgfusepath{stroke}%
\end{pgfscope}%
\begin{pgfscope}%
\pgfsetrectcap%
\pgfsetmiterjoin%
\pgfsetlinewidth{0.803000pt}%
\definecolor{currentstroke}{rgb}{0.000000,0.000000,0.000000}%
\pgfsetstrokecolor{currentstroke}%
\pgfsetdash{}{0pt}%
\pgfpathmoveto{\pgfqpoint{3.505455in}{2.544000in}}%
\pgfpathlineto{\pgfqpoint{5.760000in}{2.544000in}}%
\pgfusepath{stroke}%
\end{pgfscope}%
\begin{pgfscope}%
\pgfsetrectcap%
\pgfsetmiterjoin%
\pgfsetlinewidth{0.803000pt}%
\definecolor{currentstroke}{rgb}{0.000000,0.000000,0.000000}%
\pgfsetstrokecolor{currentstroke}%
\pgfsetdash{}{0pt}%
\pgfpathmoveto{\pgfqpoint{3.505455in}{4.224000in}}%
\pgfpathlineto{\pgfqpoint{5.760000in}{4.224000in}}%
\pgfusepath{stroke}%
\end{pgfscope}%
\begin{pgfscope}%
\pgfsetbuttcap%
\pgfsetmiterjoin%
\definecolor{currentfill}{rgb}{1.000000,1.000000,1.000000}%
\pgfsetfillcolor{currentfill}%
\pgfsetlinewidth{0.000000pt}%
\definecolor{currentstroke}{rgb}{0.000000,0.000000,0.000000}%
\pgfsetstrokecolor{currentstroke}%
\pgfsetstrokeopacity{0.000000}%
\pgfsetdash{}{0pt}%
\pgfpathmoveto{\pgfqpoint{0.800000in}{0.528000in}}%
\pgfpathlineto{\pgfqpoint{3.054545in}{0.528000in}}%
\pgfpathlineto{\pgfqpoint{3.054545in}{2.208000in}}%
\pgfpathlineto{\pgfqpoint{0.800000in}{2.208000in}}%
\pgfpathlineto{\pgfqpoint{0.800000in}{0.528000in}}%
\pgfpathclose%
\pgfusepath{fill}%
\end{pgfscope}%
\begin{pgfscope}%
\pgfsetbuttcap%
\pgfsetroundjoin%
\definecolor{currentfill}{rgb}{0.000000,0.000000,0.000000}%
\pgfsetfillcolor{currentfill}%
\pgfsetlinewidth{0.803000pt}%
\definecolor{currentstroke}{rgb}{0.000000,0.000000,0.000000}%
\pgfsetstrokecolor{currentstroke}%
\pgfsetdash{}{0pt}%
\pgfsys@defobject{currentmarker}{\pgfqpoint{0.000000in}{-0.048611in}}{\pgfqpoint{0.000000in}{0.000000in}}{%
\pgfpathmoveto{\pgfqpoint{0.000000in}{0.000000in}}%
\pgfpathlineto{\pgfqpoint{0.000000in}{-0.048611in}}%
\pgfusepath{stroke,fill}%
}%
\begin{pgfscope}%
\pgfsys@transformshift{0.902479in}{0.528000in}%
\pgfsys@useobject{currentmarker}{}%
\end{pgfscope}%
\end{pgfscope}%
\begin{pgfscope}%
\definecolor{textcolor}{rgb}{0.000000,0.000000,0.000000}%
\pgfsetstrokecolor{textcolor}%
\pgfsetfillcolor{textcolor}%
\pgftext[x=0.902479in,y=0.430778in,,top]{\color{textcolor}\rmfamily\fontsize{10.000000}{12.000000}\selectfont \(\displaystyle {0}\)}%
\end{pgfscope}%
\begin{pgfscope}%
\pgfsetbuttcap%
\pgfsetroundjoin%
\definecolor{currentfill}{rgb}{0.000000,0.000000,0.000000}%
\pgfsetfillcolor{currentfill}%
\pgfsetlinewidth{0.803000pt}%
\definecolor{currentstroke}{rgb}{0.000000,0.000000,0.000000}%
\pgfsetstrokecolor{currentstroke}%
\pgfsetdash{}{0pt}%
\pgfsys@defobject{currentmarker}{\pgfqpoint{0.000000in}{-0.048611in}}{\pgfqpoint{0.000000in}{0.000000in}}{%
\pgfpathmoveto{\pgfqpoint{0.000000in}{0.000000in}}%
\pgfpathlineto{\pgfqpoint{0.000000in}{-0.048611in}}%
\pgfusepath{stroke,fill}%
}%
\begin{pgfscope}%
\pgfsys@transformshift{1.722478in}{0.528000in}%
\pgfsys@useobject{currentmarker}{}%
\end{pgfscope}%
\end{pgfscope}%
\begin{pgfscope}%
\definecolor{textcolor}{rgb}{0.000000,0.000000,0.000000}%
\pgfsetstrokecolor{textcolor}%
\pgfsetfillcolor{textcolor}%
\pgftext[x=1.722478in,y=0.430778in,,top]{\color{textcolor}\rmfamily\fontsize{10.000000}{12.000000}\selectfont \(\displaystyle {2000}\)}%
\end{pgfscope}%
\begin{pgfscope}%
\pgfsetbuttcap%
\pgfsetroundjoin%
\definecolor{currentfill}{rgb}{0.000000,0.000000,0.000000}%
\pgfsetfillcolor{currentfill}%
\pgfsetlinewidth{0.803000pt}%
\definecolor{currentstroke}{rgb}{0.000000,0.000000,0.000000}%
\pgfsetstrokecolor{currentstroke}%
\pgfsetdash{}{0pt}%
\pgfsys@defobject{currentmarker}{\pgfqpoint{0.000000in}{-0.048611in}}{\pgfqpoint{0.000000in}{0.000000in}}{%
\pgfpathmoveto{\pgfqpoint{0.000000in}{0.000000in}}%
\pgfpathlineto{\pgfqpoint{0.000000in}{-0.048611in}}%
\pgfusepath{stroke,fill}%
}%
\begin{pgfscope}%
\pgfsys@transformshift{2.542477in}{0.528000in}%
\pgfsys@useobject{currentmarker}{}%
\end{pgfscope}%
\end{pgfscope}%
\begin{pgfscope}%
\definecolor{textcolor}{rgb}{0.000000,0.000000,0.000000}%
\pgfsetstrokecolor{textcolor}%
\pgfsetfillcolor{textcolor}%
\pgftext[x=2.542477in,y=0.430778in,,top]{\color{textcolor}\rmfamily\fontsize{10.000000}{12.000000}\selectfont \(\displaystyle {4000}\)}%
\end{pgfscope}%
\begin{pgfscope}%
\pgfsetbuttcap%
\pgfsetroundjoin%
\definecolor{currentfill}{rgb}{0.000000,0.000000,0.000000}%
\pgfsetfillcolor{currentfill}%
\pgfsetlinewidth{0.803000pt}%
\definecolor{currentstroke}{rgb}{0.000000,0.000000,0.000000}%
\pgfsetstrokecolor{currentstroke}%
\pgfsetdash{}{0pt}%
\pgfsys@defobject{currentmarker}{\pgfqpoint{-0.048611in}{0.000000in}}{\pgfqpoint{-0.000000in}{0.000000in}}{%
\pgfpathmoveto{\pgfqpoint{-0.000000in}{0.000000in}}%
\pgfpathlineto{\pgfqpoint{-0.048611in}{0.000000in}}%
\pgfusepath{stroke,fill}%
}%
\begin{pgfscope}%
\pgfsys@transformshift{0.800000in}{0.685365in}%
\pgfsys@useobject{currentmarker}{}%
\end{pgfscope}%
\end{pgfscope}%
\begin{pgfscope}%
\definecolor{textcolor}{rgb}{0.000000,0.000000,0.000000}%
\pgfsetstrokecolor{textcolor}%
\pgfsetfillcolor{textcolor}%
\pgftext[x=0.455863in, y=0.637139in, left, base]{\color{textcolor}\rmfamily\fontsize{10.000000}{12.000000}\selectfont \(\displaystyle {0.00}\)}%
\end{pgfscope}%
\begin{pgfscope}%
\pgfsetbuttcap%
\pgfsetroundjoin%
\definecolor{currentfill}{rgb}{0.000000,0.000000,0.000000}%
\pgfsetfillcolor{currentfill}%
\pgfsetlinewidth{0.803000pt}%
\definecolor{currentstroke}{rgb}{0.000000,0.000000,0.000000}%
\pgfsetstrokecolor{currentstroke}%
\pgfsetdash{}{0pt}%
\pgfsys@defobject{currentmarker}{\pgfqpoint{-0.048611in}{0.000000in}}{\pgfqpoint{-0.000000in}{0.000000in}}{%
\pgfpathmoveto{\pgfqpoint{-0.000000in}{0.000000in}}%
\pgfpathlineto{\pgfqpoint{-0.048611in}{0.000000in}}%
\pgfusepath{stroke,fill}%
}%
\begin{pgfscope}%
\pgfsys@transformshift{0.800000in}{1.066023in}%
\pgfsys@useobject{currentmarker}{}%
\end{pgfscope}%
\end{pgfscope}%
\begin{pgfscope}%
\definecolor{textcolor}{rgb}{0.000000,0.000000,0.000000}%
\pgfsetstrokecolor{textcolor}%
\pgfsetfillcolor{textcolor}%
\pgftext[x=0.455863in, y=1.017798in, left, base]{\color{textcolor}\rmfamily\fontsize{10.000000}{12.000000}\selectfont \(\displaystyle {0.05}\)}%
\end{pgfscope}%
\begin{pgfscope}%
\pgfsetbuttcap%
\pgfsetroundjoin%
\definecolor{currentfill}{rgb}{0.000000,0.000000,0.000000}%
\pgfsetfillcolor{currentfill}%
\pgfsetlinewidth{0.803000pt}%
\definecolor{currentstroke}{rgb}{0.000000,0.000000,0.000000}%
\pgfsetstrokecolor{currentstroke}%
\pgfsetdash{}{0pt}%
\pgfsys@defobject{currentmarker}{\pgfqpoint{-0.048611in}{0.000000in}}{\pgfqpoint{-0.000000in}{0.000000in}}{%
\pgfpathmoveto{\pgfqpoint{-0.000000in}{0.000000in}}%
\pgfpathlineto{\pgfqpoint{-0.048611in}{0.000000in}}%
\pgfusepath{stroke,fill}%
}%
\begin{pgfscope}%
\pgfsys@transformshift{0.800000in}{1.446682in}%
\pgfsys@useobject{currentmarker}{}%
\end{pgfscope}%
\end{pgfscope}%
\begin{pgfscope}%
\definecolor{textcolor}{rgb}{0.000000,0.000000,0.000000}%
\pgfsetstrokecolor{textcolor}%
\pgfsetfillcolor{textcolor}%
\pgftext[x=0.455863in, y=1.398457in, left, base]{\color{textcolor}\rmfamily\fontsize{10.000000}{12.000000}\selectfont \(\displaystyle {0.10}\)}%
\end{pgfscope}%
\begin{pgfscope}%
\pgfsetbuttcap%
\pgfsetroundjoin%
\definecolor{currentfill}{rgb}{0.000000,0.000000,0.000000}%
\pgfsetfillcolor{currentfill}%
\pgfsetlinewidth{0.803000pt}%
\definecolor{currentstroke}{rgb}{0.000000,0.000000,0.000000}%
\pgfsetstrokecolor{currentstroke}%
\pgfsetdash{}{0pt}%
\pgfsys@defobject{currentmarker}{\pgfqpoint{-0.048611in}{0.000000in}}{\pgfqpoint{-0.000000in}{0.000000in}}{%
\pgfpathmoveto{\pgfqpoint{-0.000000in}{0.000000in}}%
\pgfpathlineto{\pgfqpoint{-0.048611in}{0.000000in}}%
\pgfusepath{stroke,fill}%
}%
\begin{pgfscope}%
\pgfsys@transformshift{0.800000in}{1.827341in}%
\pgfsys@useobject{currentmarker}{}%
\end{pgfscope}%
\end{pgfscope}%
\begin{pgfscope}%
\definecolor{textcolor}{rgb}{0.000000,0.000000,0.000000}%
\pgfsetstrokecolor{textcolor}%
\pgfsetfillcolor{textcolor}%
\pgftext[x=0.455863in, y=1.779116in, left, base]{\color{textcolor}\rmfamily\fontsize{10.000000}{12.000000}\selectfont \(\displaystyle {0.15}\)}%
\end{pgfscope}%
\begin{pgfscope}%
\pgfsetbuttcap%
\pgfsetroundjoin%
\definecolor{currentfill}{rgb}{0.000000,0.000000,0.000000}%
\pgfsetfillcolor{currentfill}%
\pgfsetlinewidth{0.803000pt}%
\definecolor{currentstroke}{rgb}{0.000000,0.000000,0.000000}%
\pgfsetstrokecolor{currentstroke}%
\pgfsetdash{}{0pt}%
\pgfsys@defobject{currentmarker}{\pgfqpoint{-0.048611in}{0.000000in}}{\pgfqpoint{-0.000000in}{0.000000in}}{%
\pgfpathmoveto{\pgfqpoint{-0.000000in}{0.000000in}}%
\pgfpathlineto{\pgfqpoint{-0.048611in}{0.000000in}}%
\pgfusepath{stroke,fill}%
}%
\begin{pgfscope}%
\pgfsys@transformshift{0.800000in}{2.208000in}%
\pgfsys@useobject{currentmarker}{}%
\end{pgfscope}%
\end{pgfscope}%
\begin{pgfscope}%
\definecolor{textcolor}{rgb}{0.000000,0.000000,0.000000}%
\pgfsetstrokecolor{textcolor}%
\pgfsetfillcolor{textcolor}%
\pgftext[x=0.455863in, y=2.159775in, left, base]{\color{textcolor}\rmfamily\fontsize{10.000000}{12.000000}\selectfont \(\displaystyle {0.20}\)}%
\end{pgfscope}%
\begin{pgfscope}%
\pgfpathrectangle{\pgfqpoint{0.800000in}{0.528000in}}{\pgfqpoint{2.254545in}{1.680000in}}%
\pgfusepath{clip}%
\pgfsetrectcap%
\pgfsetroundjoin%
\pgfsetlinewidth{1.505625pt}%
\definecolor{currentstroke}{rgb}{0.121569,0.466667,0.705882}%
\pgfsetstrokecolor{currentstroke}%
\pgfsetdash{}{0pt}%
\pgfpathmoveto{\pgfqpoint{0.902479in}{1.531273in}}%
\pgfpathlineto{\pgfqpoint{0.903299in}{1.550220in}}%
\pgfpathlineto{\pgfqpoint{0.903709in}{1.546774in}}%
\pgfpathlineto{\pgfqpoint{0.904939in}{1.501320in}}%
\pgfpathlineto{\pgfqpoint{0.909039in}{1.203328in}}%
\pgfpathlineto{\pgfqpoint{0.913959in}{0.913005in}}%
\pgfpathlineto{\pgfqpoint{0.917649in}{0.802696in}}%
\pgfpathlineto{\pgfqpoint{0.920109in}{0.779654in}}%
\pgfpathlineto{\pgfqpoint{0.922159in}{0.764873in}}%
\pgfpathlineto{\pgfqpoint{0.922569in}{0.765226in}}%
\pgfpathlineto{\pgfqpoint{0.924209in}{0.770688in}}%
\pgfpathlineto{\pgfqpoint{0.933229in}{0.814026in}}%
\pgfpathlineto{\pgfqpoint{0.934459in}{0.812927in}}%
\pgfpathlineto{\pgfqpoint{0.936509in}{0.807039in}}%
\pgfpathlineto{\pgfqpoint{0.941429in}{0.782724in}}%
\pgfpathlineto{\pgfqpoint{0.946349in}{0.763280in}}%
\pgfpathlineto{\pgfqpoint{0.949219in}{0.758696in}}%
\pgfpathlineto{\pgfqpoint{0.951679in}{0.758230in}}%
\pgfpathlineto{\pgfqpoint{0.955369in}{0.760543in}}%
\pgfpathlineto{\pgfqpoint{0.959469in}{0.762408in}}%
\pgfpathlineto{\pgfqpoint{0.962339in}{0.761386in}}%
\pgfpathlineto{\pgfqpoint{0.966029in}{0.757355in}}%
\pgfpathlineto{\pgfqpoint{0.975459in}{0.746100in}}%
\pgfpathlineto{\pgfqpoint{0.979969in}{0.744637in}}%
\pgfpathlineto{\pgfqpoint{0.991449in}{0.742639in}}%
\pgfpathlineto{\pgfqpoint{1.008259in}{0.735377in}}%
\pgfpathlineto{\pgfqpoint{1.023019in}{0.732125in}}%
\pgfpathlineto{\pgfqpoint{1.036139in}{0.729036in}}%
\pgfpathlineto{\pgfqpoint{1.121829in}{0.717993in}}%
\pgfpathlineto{\pgfqpoint{1.197679in}{0.712775in}}%
\pgfpathlineto{\pgfqpoint{1.298129in}{0.708502in}}%
\pgfpathlineto{\pgfqpoint{1.443268in}{0.704716in}}%
\pgfpathlineto{\pgfqpoint{1.666718in}{0.701285in}}%
\pgfpathlineto{\pgfqpoint{2.023828in}{0.698227in}}%
\pgfpathlineto{\pgfqpoint{2.624887in}{0.695530in}}%
\pgfpathlineto{\pgfqpoint{2.952066in}{0.694614in}}%
\pgfpathlineto{\pgfqpoint{2.952066in}{0.694614in}}%
\pgfusepath{stroke}%
\end{pgfscope}%
\begin{pgfscope}%
\pgfpathrectangle{\pgfqpoint{0.800000in}{0.528000in}}{\pgfqpoint{2.254545in}{1.680000in}}%
\pgfusepath{clip}%
\pgfsetrectcap%
\pgfsetroundjoin%
\pgfsetlinewidth{1.505625pt}%
\definecolor{currentstroke}{rgb}{1.000000,0.498039,0.054902}%
\pgfsetstrokecolor{currentstroke}%
\pgfsetdash{}{0pt}%
\pgfpathmoveto{\pgfqpoint{0.902479in}{1.531273in}}%
\pgfpathlineto{\pgfqpoint{0.904119in}{1.547105in}}%
\pgfpathlineto{\pgfqpoint{0.904529in}{1.547290in}}%
\pgfpathlineto{\pgfqpoint{0.905759in}{1.540586in}}%
\pgfpathlineto{\pgfqpoint{0.907809in}{1.510744in}}%
\pgfpathlineto{\pgfqpoint{0.911499in}{1.417637in}}%
\pgfpathlineto{\pgfqpoint{0.929129in}{0.921581in}}%
\pgfpathlineto{\pgfqpoint{0.931589in}{0.880748in}}%
\pgfpathlineto{\pgfqpoint{0.935689in}{0.919906in}}%
\pgfpathlineto{\pgfqpoint{0.939379in}{0.936369in}}%
\pgfpathlineto{\pgfqpoint{0.941839in}{0.939436in}}%
\pgfpathlineto{\pgfqpoint{0.943479in}{0.938538in}}%
\pgfpathlineto{\pgfqpoint{0.945939in}{0.933406in}}%
\pgfpathlineto{\pgfqpoint{0.949629in}{0.918640in}}%
\pgfpathlineto{\pgfqpoint{0.954959in}{0.886067in}}%
\pgfpathlineto{\pgfqpoint{0.963159in}{0.819659in}}%
\pgfpathlineto{\pgfqpoint{0.972179in}{0.737611in}}%
\pgfpathlineto{\pgfqpoint{0.972589in}{0.741665in}}%
\pgfpathlineto{\pgfqpoint{0.976279in}{0.766220in}}%
\pgfpathlineto{\pgfqpoint{0.978329in}{0.769909in}}%
\pgfpathlineto{\pgfqpoint{0.979559in}{0.768850in}}%
\pgfpathlineto{\pgfqpoint{0.981609in}{0.761895in}}%
\pgfpathlineto{\pgfqpoint{0.984479in}{0.742072in}}%
\pgfpathlineto{\pgfqpoint{0.986119in}{0.729972in}}%
\pgfpathlineto{\pgfqpoint{0.994729in}{0.797734in}}%
\pgfpathlineto{\pgfqpoint{0.999649in}{0.821041in}}%
\pgfpathlineto{\pgfqpoint{1.003339in}{0.829143in}}%
\pgfpathlineto{\pgfqpoint{1.005799in}{0.830156in}}%
\pgfpathlineto{\pgfqpoint{1.007849in}{0.828532in}}%
\pgfpathlineto{\pgfqpoint{1.010719in}{0.822928in}}%
\pgfpathlineto{\pgfqpoint{1.014819in}{0.809476in}}%
\pgfpathlineto{\pgfqpoint{1.021789in}{0.777508in}}%
\pgfpathlineto{\pgfqpoint{1.027119in}{0.751147in}}%
\pgfpathlineto{\pgfqpoint{1.027529in}{0.751936in}}%
\pgfpathlineto{\pgfqpoint{1.032449in}{0.765416in}}%
\pgfpathlineto{\pgfqpoint{1.036549in}{0.770776in}}%
\pgfpathlineto{\pgfqpoint{1.039419in}{0.771525in}}%
\pgfpathlineto{\pgfqpoint{1.042289in}{0.770015in}}%
\pgfpathlineto{\pgfqpoint{1.045979in}{0.765132in}}%
\pgfpathlineto{\pgfqpoint{1.050899in}{0.754384in}}%
\pgfpathlineto{\pgfqpoint{1.058279in}{0.732169in}}%
\pgfpathlineto{\pgfqpoint{1.069759in}{0.695435in}}%
\pgfpathlineto{\pgfqpoint{1.089439in}{0.748185in}}%
\pgfpathlineto{\pgfqpoint{1.094359in}{0.753945in}}%
\pgfpathlineto{\pgfqpoint{1.098049in}{0.755258in}}%
\pgfpathlineto{\pgfqpoint{1.101739in}{0.754088in}}%
\pgfpathlineto{\pgfqpoint{1.105839in}{0.750217in}}%
\pgfpathlineto{\pgfqpoint{1.111579in}{0.741301in}}%
\pgfpathlineto{\pgfqpoint{1.120599in}{0.723703in}}%
\pgfpathlineto{\pgfqpoint{1.121419in}{0.724778in}}%
\pgfpathlineto{\pgfqpoint{1.126749in}{0.729808in}}%
\pgfpathlineto{\pgfqpoint{1.131259in}{0.731230in}}%
\pgfpathlineto{\pgfqpoint{1.135769in}{0.730164in}}%
\pgfpathlineto{\pgfqpoint{1.140689in}{0.726519in}}%
\pgfpathlineto{\pgfqpoint{1.147249in}{0.718629in}}%
\pgfpathlineto{\pgfqpoint{1.159549in}{0.700504in}}%
\pgfpathlineto{\pgfqpoint{1.160369in}{0.701348in}}%
\pgfpathlineto{\pgfqpoint{1.183329in}{0.727405in}}%
\pgfpathlineto{\pgfqpoint{1.188659in}{0.729240in}}%
\pgfpathlineto{\pgfqpoint{1.193989in}{0.728728in}}%
\pgfpathlineto{\pgfqpoint{1.199729in}{0.725841in}}%
\pgfpathlineto{\pgfqpoint{1.207519in}{0.719206in}}%
\pgfpathlineto{\pgfqpoint{1.214899in}{0.713044in}}%
\pgfpathlineto{\pgfqpoint{1.221459in}{0.715209in}}%
\pgfpathlineto{\pgfqpoint{1.227609in}{0.714836in}}%
\pgfpathlineto{\pgfqpoint{1.234579in}{0.711948in}}%
\pgfpathlineto{\pgfqpoint{1.244419in}{0.705041in}}%
\pgfpathlineto{\pgfqpoint{1.249749in}{0.701992in}}%
\pgfpathlineto{\pgfqpoint{1.262459in}{0.710101in}}%
\pgfpathlineto{\pgfqpoint{1.273119in}{0.715734in}}%
\pgfpathlineto{\pgfqpoint{1.280909in}{0.717324in}}%
\pgfpathlineto{\pgfqpoint{1.288289in}{0.716506in}}%
\pgfpathlineto{\pgfqpoint{1.297309in}{0.713018in}}%
\pgfpathlineto{\pgfqpoint{1.310839in}{0.707338in}}%
\pgfpathlineto{\pgfqpoint{1.319039in}{0.707433in}}%
\pgfpathlineto{\pgfqpoint{1.328059in}{0.705253in}}%
\pgfpathlineto{\pgfqpoint{1.339949in}{0.702163in}}%
\pgfpathlineto{\pgfqpoint{1.355529in}{0.707402in}}%
\pgfpathlineto{\pgfqpoint{1.367829in}{0.710464in}}%
\pgfpathlineto{\pgfqpoint{1.377669in}{0.710628in}}%
\pgfpathlineto{\pgfqpoint{1.388739in}{0.708444in}}%
\pgfpathlineto{\pgfqpoint{1.411289in}{0.703268in}}%
\pgfpathlineto{\pgfqpoint{1.434249in}{0.702409in}}%
\pgfpathlineto{\pgfqpoint{1.466638in}{0.706860in}}%
\pgfpathlineto{\pgfqpoint{1.480578in}{0.705403in}}%
\pgfpathlineto{\pgfqpoint{1.512968in}{0.701013in}}%
\pgfpathlineto{\pgfqpoint{1.528958in}{0.701991in}}%
\pgfpathlineto{\pgfqpoint{1.558888in}{0.704241in}}%
\pgfpathlineto{\pgfqpoint{1.577748in}{0.702701in}}%
\pgfpathlineto{\pgfqpoint{1.605218in}{0.700518in}}%
\pgfpathlineto{\pgfqpoint{1.627358in}{0.701550in}}%
\pgfpathlineto{\pgfqpoint{1.653598in}{0.702330in}}%
\pgfpathlineto{\pgfqpoint{1.731088in}{0.700964in}}%
\pgfpathlineto{\pgfqpoint{1.758148in}{0.700487in}}%
\pgfpathlineto{\pgfqpoint{1.800788in}{0.699438in}}%
\pgfpathlineto{\pgfqpoint{1.854088in}{0.699423in}}%
\pgfpathlineto{\pgfqpoint{1.898778in}{0.698894in}}%
\pgfpathlineto{\pgfqpoint{1.950028in}{0.698598in}}%
\pgfpathlineto{\pgfqpoint{2.001688in}{0.698381in}}%
\pgfpathlineto{\pgfqpoint{2.069338in}{0.697709in}}%
\pgfpathlineto{\pgfqpoint{2.341167in}{0.696460in}}%
\pgfpathlineto{\pgfqpoint{2.952066in}{0.694580in}}%
\pgfpathlineto{\pgfqpoint{2.952066in}{0.694580in}}%
\pgfusepath{stroke}%
\end{pgfscope}%
\begin{pgfscope}%
\pgfpathrectangle{\pgfqpoint{0.800000in}{0.528000in}}{\pgfqpoint{2.254545in}{1.680000in}}%
\pgfusepath{clip}%
\pgfsetrectcap%
\pgfsetroundjoin%
\pgfsetlinewidth{1.505625pt}%
\definecolor{currentstroke}{rgb}{0.172549,0.627451,0.172549}%
\pgfsetstrokecolor{currentstroke}%
\pgfsetdash{}{0pt}%
\pgfpathmoveto{\pgfqpoint{0.902479in}{1.531273in}}%
\pgfpathlineto{\pgfqpoint{0.903709in}{1.545881in}}%
\pgfpathlineto{\pgfqpoint{0.904119in}{1.545091in}}%
\pgfpathlineto{\pgfqpoint{0.905349in}{1.529247in}}%
\pgfpathlineto{\pgfqpoint{0.907809in}{1.455627in}}%
\pgfpathlineto{\pgfqpoint{0.922979in}{0.914377in}}%
\pgfpathlineto{\pgfqpoint{0.925439in}{0.863551in}}%
\pgfpathlineto{\pgfqpoint{0.925849in}{0.867380in}}%
\pgfpathlineto{\pgfqpoint{0.928719in}{0.884091in}}%
\pgfpathlineto{\pgfqpoint{0.930769in}{0.886895in}}%
\pgfpathlineto{\pgfqpoint{0.931999in}{0.885611in}}%
\pgfpathlineto{\pgfqpoint{0.934049in}{0.879379in}}%
\pgfpathlineto{\pgfqpoint{0.937329in}{0.861080in}}%
\pgfpathlineto{\pgfqpoint{0.943069in}{0.813550in}}%
\pgfpathlineto{\pgfqpoint{0.955369in}{0.697628in}}%
\pgfpathlineto{\pgfqpoint{0.955779in}{0.698558in}}%
\pgfpathlineto{\pgfqpoint{0.969309in}{0.792718in}}%
\pgfpathlineto{\pgfqpoint{0.972999in}{0.802613in}}%
\pgfpathlineto{\pgfqpoint{0.975459in}{0.803990in}}%
\pgfpathlineto{\pgfqpoint{0.977509in}{0.802247in}}%
\pgfpathlineto{\pgfqpoint{0.980379in}{0.796096in}}%
\pgfpathlineto{\pgfqpoint{0.984889in}{0.780221in}}%
\pgfpathlineto{\pgfqpoint{0.993909in}{0.742234in}}%
\pgfpathlineto{\pgfqpoint{0.994729in}{0.743340in}}%
\pgfpathlineto{\pgfqpoint{0.998009in}{0.745632in}}%
\pgfpathlineto{\pgfqpoint{1.000469in}{0.744928in}}%
\pgfpathlineto{\pgfqpoint{1.003749in}{0.741232in}}%
\pgfpathlineto{\pgfqpoint{1.008259in}{0.732265in}}%
\pgfpathlineto{\pgfqpoint{1.016869in}{0.710893in}}%
\pgfpathlineto{\pgfqpoint{1.017689in}{0.712187in}}%
\pgfpathlineto{\pgfqpoint{1.024659in}{0.725837in}}%
\pgfpathlineto{\pgfqpoint{1.032859in}{0.740423in}}%
\pgfpathlineto{\pgfqpoint{1.037369in}{0.744424in}}%
\pgfpathlineto{\pgfqpoint{1.041059in}{0.744967in}}%
\pgfpathlineto{\pgfqpoint{1.044749in}{0.743203in}}%
\pgfpathlineto{\pgfqpoint{1.049669in}{0.738050in}}%
\pgfpathlineto{\pgfqpoint{1.066069in}{0.718292in}}%
\pgfpathlineto{\pgfqpoint{1.070989in}{0.715606in}}%
\pgfpathlineto{\pgfqpoint{1.077139in}{0.712317in}}%
\pgfpathlineto{\pgfqpoint{1.082469in}{0.714509in}}%
\pgfpathlineto{\pgfqpoint{1.103789in}{0.725989in}}%
\pgfpathlineto{\pgfqpoint{1.109529in}{0.725132in}}%
\pgfpathlineto{\pgfqpoint{1.116909in}{0.721550in}}%
\pgfpathlineto{\pgfqpoint{1.133719in}{0.712651in}}%
\pgfpathlineto{\pgfqpoint{1.141099in}{0.711821in}}%
\pgfpathlineto{\pgfqpoint{1.149709in}{0.713365in}}%
\pgfpathlineto{\pgfqpoint{1.167339in}{0.717225in}}%
\pgfpathlineto{\pgfqpoint{1.176359in}{0.716331in}}%
\pgfpathlineto{\pgfqpoint{1.209159in}{0.710248in}}%
\pgfpathlineto{\pgfqpoint{1.243599in}{0.711411in}}%
\pgfpathlineto{\pgfqpoint{1.272709in}{0.708323in}}%
\pgfpathlineto{\pgfqpoint{1.315759in}{0.707855in}}%
\pgfpathlineto{\pgfqpoint{1.339539in}{0.706670in}}%
\pgfpathlineto{\pgfqpoint{1.382999in}{0.705890in}}%
\pgfpathlineto{\pgfqpoint{1.413339in}{0.705248in}}%
\pgfpathlineto{\pgfqpoint{1.452698in}{0.704313in}}%
\pgfpathlineto{\pgfqpoint{1.524858in}{0.703051in}}%
\pgfpathlineto{\pgfqpoint{1.765938in}{0.700189in}}%
\pgfpathlineto{\pgfqpoint{2.064008in}{0.697948in}}%
\pgfpathlineto{\pgfqpoint{2.745016in}{0.695147in}}%
\pgfpathlineto{\pgfqpoint{2.952066in}{0.694599in}}%
\pgfpathlineto{\pgfqpoint{2.952066in}{0.694599in}}%
\pgfusepath{stroke}%
\end{pgfscope}%
\begin{pgfscope}%
\pgfpathrectangle{\pgfqpoint{0.800000in}{0.528000in}}{\pgfqpoint{2.254545in}{1.680000in}}%
\pgfusepath{clip}%
\pgfsetrectcap%
\pgfsetroundjoin%
\pgfsetlinewidth{1.505625pt}%
\definecolor{currentstroke}{rgb}{0.839216,0.152941,0.156863}%
\pgfsetstrokecolor{currentstroke}%
\pgfsetdash{}{0pt}%
\pgfpathmoveto{\pgfqpoint{0.902479in}{1.531273in}}%
\pgfpathlineto{\pgfqpoint{0.902889in}{1.552242in}}%
\pgfpathlineto{\pgfqpoint{0.903709in}{1.540263in}}%
\pgfpathlineto{\pgfqpoint{0.905349in}{1.425829in}}%
\pgfpathlineto{\pgfqpoint{0.909859in}{0.890052in}}%
\pgfpathlineto{\pgfqpoint{0.910679in}{0.965426in}}%
\pgfpathlineto{\pgfqpoint{0.913139in}{1.088756in}}%
\pgfpathlineto{\pgfqpoint{0.914369in}{1.099939in}}%
\pgfpathlineto{\pgfqpoint{0.914779in}{1.097942in}}%
\pgfpathlineto{\pgfqpoint{0.916009in}{1.077837in}}%
\pgfpathlineto{\pgfqpoint{0.918469in}{0.985940in}}%
\pgfpathlineto{\pgfqpoint{0.918879in}{0.965754in}}%
\pgfpathlineto{\pgfqpoint{0.922159in}{1.260430in}}%
\pgfpathlineto{\pgfqpoint{0.922979in}{1.278268in}}%
\pgfpathlineto{\pgfqpoint{0.923389in}{1.277526in}}%
\pgfpathlineto{\pgfqpoint{0.924619in}{1.236953in}}%
\pgfpathlineto{\pgfqpoint{0.926669in}{1.055846in}}%
\pgfpathlineto{\pgfqpoint{0.927899in}{0.939555in}}%
\pgfpathlineto{\pgfqpoint{0.928309in}{0.964596in}}%
\pgfpathlineto{\pgfqpoint{0.930359in}{1.030979in}}%
\pgfpathlineto{\pgfqpoint{0.930769in}{1.032070in}}%
\pgfpathlineto{\pgfqpoint{0.931589in}{1.022805in}}%
\pgfpathlineto{\pgfqpoint{0.933229in}{0.965445in}}%
\pgfpathlineto{\pgfqpoint{0.936509in}{0.787725in}}%
\pgfpathlineto{\pgfqpoint{0.936919in}{0.806135in}}%
\pgfpathlineto{\pgfqpoint{0.939789in}{0.906913in}}%
\pgfpathlineto{\pgfqpoint{0.941019in}{0.918321in}}%
\pgfpathlineto{\pgfqpoint{0.941429in}{0.918231in}}%
\pgfpathlineto{\pgfqpoint{0.942659in}{0.907329in}}%
\pgfpathlineto{\pgfqpoint{0.944709in}{0.858300in}}%
\pgfpathlineto{\pgfqpoint{0.945119in}{0.844555in}}%
\pgfpathlineto{\pgfqpoint{0.945529in}{0.849544in}}%
\pgfpathlineto{\pgfqpoint{0.948399in}{0.985491in}}%
\pgfpathlineto{\pgfqpoint{0.949219in}{0.993745in}}%
\pgfpathlineto{\pgfqpoint{0.949629in}{0.992420in}}%
\pgfpathlineto{\pgfqpoint{0.950859in}{0.967205in}}%
\pgfpathlineto{\pgfqpoint{0.952909in}{0.863412in}}%
\pgfpathlineto{\pgfqpoint{0.953729in}{0.820907in}}%
\pgfpathlineto{\pgfqpoint{0.954139in}{0.835236in}}%
\pgfpathlineto{\pgfqpoint{0.956599in}{0.880398in}}%
\pgfpathlineto{\pgfqpoint{0.957009in}{0.880660in}}%
\pgfpathlineto{\pgfqpoint{0.957829in}{0.875190in}}%
\pgfpathlineto{\pgfqpoint{0.959469in}{0.843297in}}%
\pgfpathlineto{\pgfqpoint{0.962339in}{0.747855in}}%
\pgfpathlineto{\pgfqpoint{0.963159in}{0.763164in}}%
\pgfpathlineto{\pgfqpoint{0.966029in}{0.822999in}}%
\pgfpathlineto{\pgfqpoint{0.967259in}{0.829428in}}%
\pgfpathlineto{\pgfqpoint{0.967669in}{0.829054in}}%
\pgfpathlineto{\pgfqpoint{0.968899in}{0.820860in}}%
\pgfpathlineto{\pgfqpoint{0.970949in}{0.786634in}}%
\pgfpathlineto{\pgfqpoint{0.971359in}{0.777253in}}%
\pgfpathlineto{\pgfqpoint{0.971769in}{0.784390in}}%
\pgfpathlineto{\pgfqpoint{0.974639in}{0.856582in}}%
\pgfpathlineto{\pgfqpoint{0.975049in}{0.858412in}}%
\pgfpathlineto{\pgfqpoint{0.975459in}{0.858029in}}%
\pgfpathlineto{\pgfqpoint{0.976689in}{0.843951in}}%
\pgfpathlineto{\pgfqpoint{0.978739in}{0.782730in}}%
\pgfpathlineto{\pgfqpoint{0.979149in}{0.766055in}}%
\pgfpathlineto{\pgfqpoint{0.979969in}{0.777442in}}%
\pgfpathlineto{\pgfqpoint{0.982429in}{0.805840in}}%
\pgfpathlineto{\pgfqpoint{0.982839in}{0.806370in}}%
\pgfpathlineto{\pgfqpoint{0.983249in}{0.805702in}}%
\pgfpathlineto{\pgfqpoint{0.984479in}{0.796930in}}%
\pgfpathlineto{\pgfqpoint{0.986939in}{0.755329in}}%
\pgfpathlineto{\pgfqpoint{0.988579in}{0.725984in}}%
\pgfpathlineto{\pgfqpoint{0.988989in}{0.734079in}}%
\pgfpathlineto{\pgfqpoint{0.992269in}{0.776390in}}%
\pgfpathlineto{\pgfqpoint{0.993499in}{0.778783in}}%
\pgfpathlineto{\pgfqpoint{0.994729in}{0.773912in}}%
\pgfpathlineto{\pgfqpoint{0.996779in}{0.751831in}}%
\pgfpathlineto{\pgfqpoint{0.997599in}{0.739144in}}%
\pgfpathlineto{\pgfqpoint{0.998009in}{0.746784in}}%
\pgfpathlineto{\pgfqpoint{1.000469in}{0.782761in}}%
\pgfpathlineto{\pgfqpoint{1.000879in}{0.783960in}}%
\pgfpathlineto{\pgfqpoint{1.001289in}{0.783749in}}%
\pgfpathlineto{\pgfqpoint{1.002519in}{0.774892in}}%
\pgfpathlineto{\pgfqpoint{1.004979in}{0.734713in}}%
\pgfpathlineto{\pgfqpoint{1.005389in}{0.740680in}}%
\pgfpathlineto{\pgfqpoint{1.008259in}{0.764113in}}%
\pgfpathlineto{\pgfqpoint{1.009079in}{0.764180in}}%
\pgfpathlineto{\pgfqpoint{1.010309in}{0.758892in}}%
\pgfpathlineto{\pgfqpoint{1.012769in}{0.732746in}}%
\pgfpathlineto{\pgfqpoint{1.014409in}{0.713451in}}%
\pgfpathlineto{\pgfqpoint{1.014819in}{0.718299in}}%
\pgfpathlineto{\pgfqpoint{1.018099in}{0.746487in}}%
\pgfpathlineto{\pgfqpoint{1.019329in}{0.748214in}}%
\pgfpathlineto{\pgfqpoint{1.020559in}{0.745045in}}%
\pgfpathlineto{\pgfqpoint{1.022609in}{0.730306in}}%
\pgfpathlineto{\pgfqpoint{1.023839in}{0.717197in}}%
\pgfpathlineto{\pgfqpoint{1.024249in}{0.721873in}}%
\pgfpathlineto{\pgfqpoint{1.026709in}{0.740009in}}%
\pgfpathlineto{\pgfqpoint{1.027529in}{0.738736in}}%
\pgfpathlineto{\pgfqpoint{1.029169in}{0.725773in}}%
\pgfpathlineto{\pgfqpoint{1.029989in}{0.714757in}}%
\pgfpathlineto{\pgfqpoint{1.030399in}{0.715800in}}%
\pgfpathlineto{\pgfqpoint{1.033679in}{0.738219in}}%
\pgfpathlineto{\pgfqpoint{1.034909in}{0.739016in}}%
\pgfpathlineto{\pgfqpoint{1.036139in}{0.735576in}}%
\pgfpathlineto{\pgfqpoint{1.038599in}{0.718410in}}%
\pgfpathlineto{\pgfqpoint{1.040239in}{0.705610in}}%
\pgfpathlineto{\pgfqpoint{1.040649in}{0.708827in}}%
\pgfpathlineto{\pgfqpoint{1.043929in}{0.727725in}}%
\pgfpathlineto{\pgfqpoint{1.045159in}{0.728876in}}%
\pgfpathlineto{\pgfqpoint{1.045569in}{0.728509in}}%
\pgfpathlineto{\pgfqpoint{1.047209in}{0.723540in}}%
\pgfpathlineto{\pgfqpoint{1.050079in}{0.704388in}}%
\pgfpathlineto{\pgfqpoint{1.050899in}{0.707437in}}%
\pgfpathlineto{\pgfqpoint{1.052539in}{0.712966in}}%
\pgfpathlineto{\pgfqpoint{1.052949in}{0.712797in}}%
\pgfpathlineto{\pgfqpoint{1.054179in}{0.708680in}}%
\pgfpathlineto{\pgfqpoint{1.055409in}{0.701729in}}%
\pgfpathlineto{\pgfqpoint{1.055819in}{0.704864in}}%
\pgfpathlineto{\pgfqpoint{1.059099in}{0.721790in}}%
\pgfpathlineto{\pgfqpoint{1.060739in}{0.723072in}}%
\pgfpathlineto{\pgfqpoint{1.061969in}{0.720723in}}%
\pgfpathlineto{\pgfqpoint{1.064429in}{0.709113in}}%
\pgfpathlineto{\pgfqpoint{1.066069in}{0.700514in}}%
\pgfpathlineto{\pgfqpoint{1.066479in}{0.702766in}}%
\pgfpathlineto{\pgfqpoint{1.069759in}{0.715540in}}%
\pgfpathlineto{\pgfqpoint{1.071399in}{0.716006in}}%
\pgfpathlineto{\pgfqpoint{1.073039in}{0.712517in}}%
\pgfpathlineto{\pgfqpoint{1.075909in}{0.699263in}}%
\pgfpathlineto{\pgfqpoint{1.077139in}{0.693815in}}%
\pgfpathlineto{\pgfqpoint{1.077549in}{0.694933in}}%
\pgfpathlineto{\pgfqpoint{1.078779in}{0.695737in}}%
\pgfpathlineto{\pgfqpoint{1.080009in}{0.692836in}}%
\pgfpathlineto{\pgfqpoint{1.080419in}{0.693884in}}%
\pgfpathlineto{\pgfqpoint{1.084519in}{0.711107in}}%
\pgfpathlineto{\pgfqpoint{1.086569in}{0.712640in}}%
\pgfpathlineto{\pgfqpoint{1.088209in}{0.710038in}}%
\pgfpathlineto{\pgfqpoint{1.091079in}{0.699336in}}%
\pgfpathlineto{\pgfqpoint{1.091899in}{0.697103in}}%
\pgfpathlineto{\pgfqpoint{1.092309in}{0.698716in}}%
\pgfpathlineto{\pgfqpoint{1.095589in}{0.707419in}}%
\pgfpathlineto{\pgfqpoint{1.097229in}{0.707670in}}%
\pgfpathlineto{\pgfqpoint{1.098869in}{0.705176in}}%
\pgfpathlineto{\pgfqpoint{1.102149in}{0.694262in}}%
\pgfpathlineto{\pgfqpoint{1.104199in}{0.686148in}}%
\pgfpathlineto{\pgfqpoint{1.104609in}{0.686424in}}%
\pgfpathlineto{\pgfqpoint{1.110349in}{0.704674in}}%
\pgfpathlineto{\pgfqpoint{1.112399in}{0.705642in}}%
\pgfpathlineto{\pgfqpoint{1.114039in}{0.703777in}}%
\pgfpathlineto{\pgfqpoint{1.116909in}{0.696279in}}%
\pgfpathlineto{\pgfqpoint{1.117729in}{0.694762in}}%
\pgfpathlineto{\pgfqpoint{1.118139in}{0.695923in}}%
\pgfpathlineto{\pgfqpoint{1.121419in}{0.701894in}}%
\pgfpathlineto{\pgfqpoint{1.123469in}{0.701726in}}%
\pgfpathlineto{\pgfqpoint{1.125929in}{0.697793in}}%
\pgfpathlineto{\pgfqpoint{1.130029in}{0.687113in}}%
\pgfpathlineto{\pgfqpoint{1.130849in}{0.688751in}}%
\pgfpathlineto{\pgfqpoint{1.135769in}{0.699814in}}%
\pgfpathlineto{\pgfqpoint{1.138229in}{0.700852in}}%
\pgfpathlineto{\pgfqpoint{1.140279in}{0.698930in}}%
\pgfpathlineto{\pgfqpoint{1.143559in}{0.693122in}}%
\pgfpathlineto{\pgfqpoint{1.143969in}{0.693948in}}%
\pgfpathlineto{\pgfqpoint{1.147249in}{0.698071in}}%
\pgfpathlineto{\pgfqpoint{1.149299in}{0.697890in}}%
\pgfpathlineto{\pgfqpoint{1.151759in}{0.695049in}}%
\pgfpathlineto{\pgfqpoint{1.155859in}{0.687727in}}%
\pgfpathlineto{\pgfqpoint{1.156269in}{0.688327in}}%
\pgfpathlineto{\pgfqpoint{1.162009in}{0.697148in}}%
\pgfpathlineto{\pgfqpoint{1.164469in}{0.697370in}}%
\pgfpathlineto{\pgfqpoint{1.166929in}{0.695135in}}%
\pgfpathlineto{\pgfqpoint{1.169389in}{0.691951in}}%
\pgfpathlineto{\pgfqpoint{1.169799in}{0.692525in}}%
\pgfpathlineto{\pgfqpoint{1.173079in}{0.695385in}}%
\pgfpathlineto{\pgfqpoint{1.175539in}{0.694990in}}%
\pgfpathlineto{\pgfqpoint{1.178819in}{0.691575in}}%
\pgfpathlineto{\pgfqpoint{1.181689in}{0.688174in}}%
\pgfpathlineto{\pgfqpoint{1.182099in}{0.688738in}}%
\pgfpathlineto{\pgfqpoint{1.187839in}{0.694941in}}%
\pgfpathlineto{\pgfqpoint{1.190709in}{0.694884in}}%
\pgfpathlineto{\pgfqpoint{1.193989in}{0.692134in}}%
\pgfpathlineto{\pgfqpoint{1.195219in}{0.691098in}}%
\pgfpathlineto{\pgfqpoint{1.195629in}{0.691480in}}%
\pgfpathlineto{\pgfqpoint{1.199319in}{0.693530in}}%
\pgfpathlineto{\pgfqpoint{1.202189in}{0.692705in}}%
\pgfpathlineto{\pgfqpoint{1.208339in}{0.689367in}}%
\pgfpathlineto{\pgfqpoint{1.213669in}{0.693346in}}%
\pgfpathlineto{\pgfqpoint{1.216949in}{0.693102in}}%
\pgfpathlineto{\pgfqpoint{1.222689in}{0.691437in}}%
\pgfpathlineto{\pgfqpoint{1.226379in}{0.692041in}}%
\pgfpathlineto{\pgfqpoint{1.230479in}{0.690008in}}%
\pgfpathlineto{\pgfqpoint{1.233349in}{0.688768in}}%
\pgfpathlineto{\pgfqpoint{1.239909in}{0.692245in}}%
\pgfpathlineto{\pgfqpoint{1.243599in}{0.691661in}}%
\pgfpathlineto{\pgfqpoint{1.248519in}{0.690634in}}%
\pgfpathlineto{\pgfqpoint{1.252619in}{0.690962in}}%
\pgfpathlineto{\pgfqpoint{1.261229in}{0.689922in}}%
\pgfpathlineto{\pgfqpoint{1.266559in}{0.691385in}}%
\pgfpathlineto{\pgfqpoint{1.271479in}{0.690121in}}%
\pgfpathlineto{\pgfqpoint{1.274349in}{0.690031in}}%
\pgfpathlineto{\pgfqpoint{1.279269in}{0.690083in}}%
\pgfpathlineto{\pgfqpoint{1.286239in}{0.689370in}}%
\pgfpathlineto{\pgfqpoint{1.292799in}{0.690688in}}%
\pgfpathlineto{\pgfqpoint{1.316989in}{0.690155in}}%
\pgfpathlineto{\pgfqpoint{1.323549in}{0.689371in}}%
\pgfpathlineto{\pgfqpoint{1.328059in}{0.689370in}}%
\pgfpathlineto{\pgfqpoint{1.382999in}{0.688658in}}%
\pgfpathlineto{\pgfqpoint{1.391199in}{0.689040in}}%
\pgfpathlineto{\pgfqpoint{1.401449in}{0.688826in}}%
\pgfpathlineto{\pgfqpoint{1.414159in}{0.688683in}}%
\pgfpathlineto{\pgfqpoint{1.427689in}{0.688688in}}%
\pgfpathlineto{\pgfqpoint{1.442038in}{0.688728in}}%
\pgfpathlineto{\pgfqpoint{1.457618in}{0.688417in}}%
\pgfpathlineto{\pgfqpoint{1.495748in}{0.688584in}}%
\pgfpathlineto{\pgfqpoint{1.555198in}{0.688346in}}%
\pgfpathlineto{\pgfqpoint{1.622848in}{0.688190in}}%
\pgfpathlineto{\pgfqpoint{1.802018in}{0.687865in}}%
\pgfpathlineto{\pgfqpoint{2.952066in}{0.686988in}}%
\pgfpathlineto{\pgfqpoint{2.952066in}{0.686988in}}%
\pgfusepath{stroke}%
\end{pgfscope}%
\begin{pgfscope}%
\pgfpathrectangle{\pgfqpoint{0.800000in}{0.528000in}}{\pgfqpoint{2.254545in}{1.680000in}}%
\pgfusepath{clip}%
\pgfsetrectcap%
\pgfsetroundjoin%
\pgfsetlinewidth{1.505625pt}%
\definecolor{currentstroke}{rgb}{0.580392,0.403922,0.741176}%
\pgfsetstrokecolor{currentstroke}%
\pgfsetdash{}{0pt}%
\pgfpathmoveto{\pgfqpoint{0.902479in}{1.531273in}}%
\pgfpathlineto{\pgfqpoint{0.902889in}{1.541146in}}%
\pgfpathlineto{\pgfqpoint{0.903299in}{1.534147in}}%
\pgfpathlineto{\pgfqpoint{0.904529in}{1.414374in}}%
\pgfpathlineto{\pgfqpoint{0.907809in}{0.873690in}}%
\pgfpathlineto{\pgfqpoint{0.908629in}{0.951203in}}%
\pgfpathlineto{\pgfqpoint{0.910679in}{1.030672in}}%
\pgfpathlineto{\pgfqpoint{0.911499in}{1.022639in}}%
\pgfpathlineto{\pgfqpoint{0.913139in}{0.962332in}}%
\pgfpathlineto{\pgfqpoint{0.914369in}{0.908298in}}%
\pgfpathlineto{\pgfqpoint{0.916829in}{1.124177in}}%
\pgfpathlineto{\pgfqpoint{0.917239in}{1.124721in}}%
\pgfpathlineto{\pgfqpoint{0.918059in}{1.095284in}}%
\pgfpathlineto{\pgfqpoint{0.920519in}{0.869058in}}%
\pgfpathlineto{\pgfqpoint{0.921339in}{0.912423in}}%
\pgfpathlineto{\pgfqpoint{0.922569in}{0.934526in}}%
\pgfpathlineto{\pgfqpoint{0.922979in}{0.930751in}}%
\pgfpathlineto{\pgfqpoint{0.924619in}{0.872457in}}%
\pgfpathlineto{\pgfqpoint{0.926669in}{0.762978in}}%
\pgfpathlineto{\pgfqpoint{0.927079in}{0.783228in}}%
\pgfpathlineto{\pgfqpoint{0.929539in}{0.842776in}}%
\pgfpathlineto{\pgfqpoint{0.929949in}{0.842696in}}%
\pgfpathlineto{\pgfqpoint{0.931179in}{0.827934in}}%
\pgfpathlineto{\pgfqpoint{0.932819in}{0.781187in}}%
\pgfpathlineto{\pgfqpoint{0.933229in}{0.788327in}}%
\pgfpathlineto{\pgfqpoint{0.935689in}{0.857253in}}%
\pgfpathlineto{\pgfqpoint{0.936509in}{0.843263in}}%
\pgfpathlineto{\pgfqpoint{0.938559in}{0.772296in}}%
\pgfpathlineto{\pgfqpoint{0.938969in}{0.778765in}}%
\pgfpathlineto{\pgfqpoint{0.941019in}{0.805961in}}%
\pgfpathlineto{\pgfqpoint{0.941839in}{0.800466in}}%
\pgfpathlineto{\pgfqpoint{0.943889in}{0.756471in}}%
\pgfpathlineto{\pgfqpoint{0.945119in}{0.729387in}}%
\pgfpathlineto{\pgfqpoint{0.945529in}{0.739497in}}%
\pgfpathlineto{\pgfqpoint{0.947989in}{0.770128in}}%
\pgfpathlineto{\pgfqpoint{0.948399in}{0.769993in}}%
\pgfpathlineto{\pgfqpoint{0.949629in}{0.761448in}}%
\pgfpathlineto{\pgfqpoint{0.951269in}{0.734860in}}%
\pgfpathlineto{\pgfqpoint{0.952089in}{0.738087in}}%
\pgfpathlineto{\pgfqpoint{0.953729in}{0.754334in}}%
\pgfpathlineto{\pgfqpoint{0.954549in}{0.750069in}}%
\pgfpathlineto{\pgfqpoint{0.956599in}{0.729949in}}%
\pgfpathlineto{\pgfqpoint{0.959059in}{0.752924in}}%
\pgfpathlineto{\pgfqpoint{0.959469in}{0.752849in}}%
\pgfpathlineto{\pgfqpoint{0.960699in}{0.745943in}}%
\pgfpathlineto{\pgfqpoint{0.963159in}{0.711629in}}%
\pgfpathlineto{\pgfqpoint{0.963979in}{0.719854in}}%
\pgfpathlineto{\pgfqpoint{0.966439in}{0.735283in}}%
\pgfpathlineto{\pgfqpoint{0.967259in}{0.733742in}}%
\pgfpathlineto{\pgfqpoint{0.968899in}{0.722090in}}%
\pgfpathlineto{\pgfqpoint{0.970129in}{0.709383in}}%
\pgfpathlineto{\pgfqpoint{0.970949in}{0.710455in}}%
\pgfpathlineto{\pgfqpoint{0.972589in}{0.706539in}}%
\pgfpathlineto{\pgfqpoint{0.972999in}{0.708455in}}%
\pgfpathlineto{\pgfqpoint{0.975049in}{0.715276in}}%
\pgfpathlineto{\pgfqpoint{0.977509in}{0.726969in}}%
\pgfpathlineto{\pgfqpoint{0.978329in}{0.725657in}}%
\pgfpathlineto{\pgfqpoint{0.979969in}{0.716365in}}%
\pgfpathlineto{\pgfqpoint{0.981609in}{0.702818in}}%
\pgfpathlineto{\pgfqpoint{0.982019in}{0.706107in}}%
\pgfpathlineto{\pgfqpoint{0.984479in}{0.716697in}}%
\pgfpathlineto{\pgfqpoint{0.985299in}{0.716303in}}%
\pgfpathlineto{\pgfqpoint{0.986939in}{0.710187in}}%
\pgfpathlineto{\pgfqpoint{0.990629in}{0.692586in}}%
\pgfpathlineto{\pgfqpoint{0.993089in}{0.704932in}}%
\pgfpathlineto{\pgfqpoint{0.995549in}{0.712813in}}%
\pgfpathlineto{\pgfqpoint{0.996369in}{0.712414in}}%
\pgfpathlineto{\pgfqpoint{0.998009in}{0.707503in}}%
\pgfpathlineto{\pgfqpoint{1.000059in}{0.698947in}}%
\pgfpathlineto{\pgfqpoint{1.000469in}{0.700817in}}%
\pgfpathlineto{\pgfqpoint{1.002929in}{0.706363in}}%
\pgfpathlineto{\pgfqpoint{1.004159in}{0.705130in}}%
\pgfpathlineto{\pgfqpoint{1.006209in}{0.698306in}}%
\pgfpathlineto{\pgfqpoint{1.008669in}{0.688790in}}%
\pgfpathlineto{\pgfqpoint{1.009079in}{0.690607in}}%
\pgfpathlineto{\pgfqpoint{1.013179in}{0.704268in}}%
\pgfpathlineto{\pgfqpoint{1.014819in}{0.704226in}}%
\pgfpathlineto{\pgfqpoint{1.016869in}{0.699603in}}%
\pgfpathlineto{\pgfqpoint{1.018099in}{0.695410in}}%
\pgfpathlineto{\pgfqpoint{1.018509in}{0.696238in}}%
\pgfpathlineto{\pgfqpoint{1.020969in}{0.700184in}}%
\pgfpathlineto{\pgfqpoint{1.022609in}{0.699145in}}%
\pgfpathlineto{\pgfqpoint{1.025069in}{0.693395in}}%
\pgfpathlineto{\pgfqpoint{1.026709in}{0.689651in}}%
\pgfpathlineto{\pgfqpoint{1.027119in}{0.690495in}}%
\pgfpathlineto{\pgfqpoint{1.031219in}{0.699362in}}%
\pgfpathlineto{\pgfqpoint{1.032859in}{0.699595in}}%
\pgfpathlineto{\pgfqpoint{1.034909in}{0.696929in}}%
\pgfpathlineto{\pgfqpoint{1.036549in}{0.693716in}}%
\pgfpathlineto{\pgfqpoint{1.036959in}{0.694305in}}%
\pgfpathlineto{\pgfqpoint{1.039419in}{0.696444in}}%
\pgfpathlineto{\pgfqpoint{1.041469in}{0.695055in}}%
\pgfpathlineto{\pgfqpoint{1.045159in}{0.690455in}}%
\pgfpathlineto{\pgfqpoint{1.045569in}{0.691174in}}%
\pgfpathlineto{\pgfqpoint{1.049669in}{0.696518in}}%
\pgfpathlineto{\pgfqpoint{1.051719in}{0.696218in}}%
\pgfpathlineto{\pgfqpoint{1.056229in}{0.693587in}}%
\pgfpathlineto{\pgfqpoint{1.058689in}{0.693811in}}%
\pgfpathlineto{\pgfqpoint{1.064429in}{0.691818in}}%
\pgfpathlineto{\pgfqpoint{1.068119in}{0.694595in}}%
\pgfpathlineto{\pgfqpoint{1.070579in}{0.693996in}}%
\pgfpathlineto{\pgfqpoint{1.074679in}{0.692279in}}%
\pgfpathlineto{\pgfqpoint{1.077549in}{0.692086in}}%
\pgfpathlineto{\pgfqpoint{1.082059in}{0.690979in}}%
\pgfpathlineto{\pgfqpoint{1.086569in}{0.693260in}}%
\pgfpathlineto{\pgfqpoint{1.089849in}{0.692312in}}%
\pgfpathlineto{\pgfqpoint{1.093129in}{0.691534in}}%
\pgfpathlineto{\pgfqpoint{1.097639in}{0.690741in}}%
\pgfpathlineto{\pgfqpoint{1.100099in}{0.690705in}}%
\pgfpathlineto{\pgfqpoint{1.105429in}{0.692298in}}%
\pgfpathlineto{\pgfqpoint{1.136589in}{0.690396in}}%
\pgfpathlineto{\pgfqpoint{1.143559in}{0.690946in}}%
\pgfpathlineto{\pgfqpoint{1.155039in}{0.690268in}}%
\pgfpathlineto{\pgfqpoint{1.163239in}{0.690419in}}%
\pgfpathlineto{\pgfqpoint{1.173079in}{0.690083in}}%
\pgfpathlineto{\pgfqpoint{1.182919in}{0.690040in}}%
\pgfpathlineto{\pgfqpoint{1.193169in}{0.690060in}}%
\pgfpathlineto{\pgfqpoint{1.206289in}{0.689688in}}%
\pgfpathlineto{\pgfqpoint{1.268609in}{0.689474in}}%
\pgfpathlineto{\pgfqpoint{2.046378in}{0.687571in}}%
\pgfpathlineto{\pgfqpoint{2.952066in}{0.686990in}}%
\pgfpathlineto{\pgfqpoint{2.952066in}{0.686990in}}%
\pgfusepath{stroke}%
\end{pgfscope}%
\begin{pgfscope}%
\pgfpathrectangle{\pgfqpoint{0.800000in}{0.528000in}}{\pgfqpoint{2.254545in}{1.680000in}}%
\pgfusepath{clip}%
\pgfsetrectcap%
\pgfsetroundjoin%
\pgfsetlinewidth{1.505625pt}%
\definecolor{currentstroke}{rgb}{0.549020,0.337255,0.294118}%
\pgfsetstrokecolor{currentstroke}%
\pgfsetdash{}{0pt}%
\pgfpathmoveto{\pgfqpoint{0.902479in}{1.531273in}}%
\pgfpathlineto{\pgfqpoint{0.912729in}{1.530204in}}%
\pgfpathlineto{\pgfqpoint{0.922569in}{1.526896in}}%
\pgfpathlineto{\pgfqpoint{0.931589in}{1.521487in}}%
\pgfpathlineto{\pgfqpoint{0.940199in}{1.513635in}}%
\pgfpathlineto{\pgfqpoint{0.948399in}{1.503057in}}%
\pgfpathlineto{\pgfqpoint{0.956599in}{1.488684in}}%
\pgfpathlineto{\pgfqpoint{0.964799in}{1.469637in}}%
\pgfpathlineto{\pgfqpoint{0.973409in}{1.443550in}}%
\pgfpathlineto{\pgfqpoint{0.982019in}{1.410092in}}%
\pgfpathlineto{\pgfqpoint{0.991039in}{1.365952in}}%
\pgfpathlineto{\pgfqpoint{1.000879in}{1.306041in}}%
\pgfpathlineto{\pgfqpoint{1.004159in}{1.285518in}}%
\pgfpathlineto{\pgfqpoint{1.020559in}{1.494318in}}%
\pgfpathlineto{\pgfqpoint{1.048849in}{1.889905in}}%
\pgfpathlineto{\pgfqpoint{1.069349in}{2.160587in}}%
\pgfpathlineto{\pgfqpoint{1.074213in}{2.218000in}}%
\pgfpathmoveto{\pgfqpoint{1.183798in}{2.218000in}}%
\pgfpathlineto{\pgfqpoint{1.192349in}{2.066009in}}%
\pgfpathlineto{\pgfqpoint{1.201779in}{1.856263in}}%
\pgfpathlineto{\pgfqpoint{1.211619in}{1.586109in}}%
\pgfpathlineto{\pgfqpoint{1.223099in}{1.205103in}}%
\pgfpathlineto{\pgfqpoint{1.228429in}{1.012562in}}%
\pgfpathlineto{\pgfqpoint{1.229249in}{1.015428in}}%
\pgfpathlineto{\pgfqpoint{1.259589in}{1.210155in}}%
\pgfpathlineto{\pgfqpoint{1.269019in}{1.254862in}}%
\pgfpathlineto{\pgfqpoint{1.276399in}{1.280308in}}%
\pgfpathlineto{\pgfqpoint{1.282139in}{1.293358in}}%
\pgfpathlineto{\pgfqpoint{1.286649in}{1.299003in}}%
\pgfpathlineto{\pgfqpoint{1.290339in}{1.300312in}}%
\pgfpathlineto{\pgfqpoint{1.293209in}{1.299093in}}%
\pgfpathlineto{\pgfqpoint{1.296489in}{1.295091in}}%
\pgfpathlineto{\pgfqpoint{1.300179in}{1.286927in}}%
\pgfpathlineto{\pgfqpoint{1.304689in}{1.270963in}}%
\pgfpathlineto{\pgfqpoint{1.309609in}{1.244658in}}%
\pgfpathlineto{\pgfqpoint{1.314939in}{1.203162in}}%
\pgfpathlineto{\pgfqpoint{1.320679in}{1.138790in}}%
\pgfpathlineto{\pgfqpoint{1.326419in}{1.046882in}}%
\pgfpathlineto{\pgfqpoint{1.329699in}{0.978285in}}%
\pgfpathlineto{\pgfqpoint{1.330519in}{0.986665in}}%
\pgfpathlineto{\pgfqpoint{1.337079in}{1.113640in}}%
\pgfpathlineto{\pgfqpoint{1.344869in}{1.314594in}}%
\pgfpathlineto{\pgfqpoint{1.355529in}{1.587034in}}%
\pgfpathlineto{\pgfqpoint{1.358809in}{1.625719in}}%
\pgfpathlineto{\pgfqpoint{1.360449in}{1.630767in}}%
\pgfpathlineto{\pgfqpoint{1.361269in}{1.629299in}}%
\pgfpathlineto{\pgfqpoint{1.362909in}{1.617986in}}%
\pgfpathlineto{\pgfqpoint{1.365369in}{1.579513in}}%
\pgfpathlineto{\pgfqpoint{1.368649in}{1.489021in}}%
\pgfpathlineto{\pgfqpoint{1.372749in}{1.339674in}}%
\pgfpathlineto{\pgfqpoint{1.373159in}{1.361385in}}%
\pgfpathlineto{\pgfqpoint{1.384229in}{2.080289in}}%
\pgfpathlineto{\pgfqpoint{1.386978in}{2.218000in}}%
\pgfpathmoveto{\pgfqpoint{1.435603in}{2.218000in}}%
\pgfpathlineto{\pgfqpoint{1.442038in}{1.914259in}}%
\pgfpathlineto{\pgfqpoint{1.449418in}{1.455832in}}%
\pgfpathlineto{\pgfqpoint{1.455158in}{1.028617in}}%
\pgfpathlineto{\pgfqpoint{1.456388in}{1.040203in}}%
\pgfpathlineto{\pgfqpoint{1.471968in}{1.238999in}}%
\pgfpathlineto{\pgfqpoint{1.478938in}{1.300281in}}%
\pgfpathlineto{\pgfqpoint{1.484268in}{1.329915in}}%
\pgfpathlineto{\pgfqpoint{1.488368in}{1.341627in}}%
\pgfpathlineto{\pgfqpoint{1.490828in}{1.343696in}}%
\pgfpathlineto{\pgfqpoint{1.492468in}{1.342865in}}%
\pgfpathlineto{\pgfqpoint{1.494518in}{1.339170in}}%
\pgfpathlineto{\pgfqpoint{1.497388in}{1.328579in}}%
\pgfpathlineto{\pgfqpoint{1.501078in}{1.304207in}}%
\pgfpathlineto{\pgfqpoint{1.505178in}{1.259334in}}%
\pgfpathlineto{\pgfqpoint{1.509278in}{1.189031in}}%
\pgfpathlineto{\pgfqpoint{1.513788in}{1.068929in}}%
\pgfpathlineto{\pgfqpoint{1.516248in}{1.004130in}}%
\pgfpathlineto{\pgfqpoint{1.519938in}{1.126623in}}%
\pgfpathlineto{\pgfqpoint{1.525678in}{1.395548in}}%
\pgfpathlineto{\pgfqpoint{1.532648in}{1.710694in}}%
\pgfpathlineto{\pgfqpoint{1.535108in}{1.744887in}}%
\pgfpathlineto{\pgfqpoint{1.535518in}{1.744339in}}%
\pgfpathlineto{\pgfqpoint{1.536748in}{1.731084in}}%
\pgfpathlineto{\pgfqpoint{1.538798in}{1.669472in}}%
\pgfpathlineto{\pgfqpoint{1.542078in}{1.475974in}}%
\pgfpathlineto{\pgfqpoint{1.543308in}{1.412699in}}%
\pgfpathlineto{\pgfqpoint{1.545358in}{1.632692in}}%
\pgfpathlineto{\pgfqpoint{1.551147in}{2.218000in}}%
\pgfpathmoveto{\pgfqpoint{1.590702in}{2.218000in}}%
\pgfpathlineto{\pgfqpoint{1.596198in}{1.779767in}}%
\pgfpathlineto{\pgfqpoint{1.603988in}{1.062067in}}%
\pgfpathlineto{\pgfqpoint{1.604808in}{1.076992in}}%
\pgfpathlineto{\pgfqpoint{1.616288in}{1.290099in}}%
\pgfpathlineto{\pgfqpoint{1.622028in}{1.358633in}}%
\pgfpathlineto{\pgfqpoint{1.626538in}{1.389568in}}%
\pgfpathlineto{\pgfqpoint{1.629818in}{1.399082in}}%
\pgfpathlineto{\pgfqpoint{1.631458in}{1.399546in}}%
\pgfpathlineto{\pgfqpoint{1.633098in}{1.396957in}}%
\pgfpathlineto{\pgfqpoint{1.635558in}{1.386821in}}%
\pgfpathlineto{\pgfqpoint{1.638428in}{1.363997in}}%
\pgfpathlineto{\pgfqpoint{1.641708in}{1.319414in}}%
\pgfpathlineto{\pgfqpoint{1.645398in}{1.235570in}}%
\pgfpathlineto{\pgfqpoint{1.649088in}{1.105264in}}%
\pgfpathlineto{\pgfqpoint{1.650728in}{1.106394in}}%
\pgfpathlineto{\pgfqpoint{1.652368in}{1.106327in}}%
\pgfpathlineto{\pgfqpoint{1.653188in}{1.134831in}}%
\pgfpathlineto{\pgfqpoint{1.658108in}{1.474593in}}%
\pgfpathlineto{\pgfqpoint{1.663028in}{1.781303in}}%
\pgfpathlineto{\pgfqpoint{1.664668in}{1.810950in}}%
\pgfpathlineto{\pgfqpoint{1.665078in}{1.810516in}}%
\pgfpathlineto{\pgfqpoint{1.666308in}{1.789835in}}%
\pgfpathlineto{\pgfqpoint{1.668358in}{1.708059in}}%
\pgfpathlineto{\pgfqpoint{1.668768in}{1.726143in}}%
\pgfpathlineto{\pgfqpoint{1.671228in}{1.789057in}}%
\pgfpathlineto{\pgfqpoint{1.672458in}{1.795167in}}%
\pgfpathlineto{\pgfqpoint{1.672868in}{1.794304in}}%
\pgfpathlineto{\pgfqpoint{1.674508in}{1.780607in}}%
\pgfpathlineto{\pgfqpoint{1.674918in}{1.786564in}}%
\pgfpathlineto{\pgfqpoint{1.678792in}{2.218000in}}%
\pgfpathmoveto{\pgfqpoint{1.708389in}{2.218000in}}%
\pgfpathlineto{\pgfqpoint{1.716328in}{1.714168in}}%
\pgfpathlineto{\pgfqpoint{1.720838in}{1.465976in}}%
\pgfpathlineto{\pgfqpoint{1.726168in}{1.099400in}}%
\pgfpathlineto{\pgfqpoint{1.726988in}{1.110391in}}%
\pgfpathlineto{\pgfqpoint{1.731498in}{1.155232in}}%
\pgfpathlineto{\pgfqpoint{1.734368in}{1.166149in}}%
\pgfpathlineto{\pgfqpoint{1.735598in}{1.165913in}}%
\pgfpathlineto{\pgfqpoint{1.737238in}{1.160737in}}%
\pgfpathlineto{\pgfqpoint{1.738058in}{1.156024in}}%
\pgfpathlineto{\pgfqpoint{1.738468in}{1.156317in}}%
\pgfpathlineto{\pgfqpoint{1.744618in}{1.234225in}}%
\pgfpathlineto{\pgfqpoint{1.747488in}{1.251642in}}%
\pgfpathlineto{\pgfqpoint{1.748718in}{1.253036in}}%
\pgfpathlineto{\pgfqpoint{1.749948in}{1.250059in}}%
\pgfpathlineto{\pgfqpoint{1.751998in}{1.234820in}}%
\pgfpathlineto{\pgfqpoint{1.755278in}{1.186447in}}%
\pgfpathlineto{\pgfqpoint{1.760198in}{1.084779in}}%
\pgfpathlineto{\pgfqpoint{1.761428in}{1.092351in}}%
\pgfpathlineto{\pgfqpoint{1.762248in}{1.094707in}}%
\pgfpathlineto{\pgfqpoint{1.762658in}{1.094595in}}%
\pgfpathlineto{\pgfqpoint{1.763888in}{1.089610in}}%
\pgfpathlineto{\pgfqpoint{1.765938in}{1.072467in}}%
\pgfpathlineto{\pgfqpoint{1.766758in}{1.074073in}}%
\pgfpathlineto{\pgfqpoint{1.767578in}{1.072347in}}%
\pgfpathlineto{\pgfqpoint{1.769218in}{1.060008in}}%
\pgfpathlineto{\pgfqpoint{1.771268in}{1.030446in}}%
\pgfpathlineto{\pgfqpoint{1.771678in}{1.043648in}}%
\pgfpathlineto{\pgfqpoint{1.774958in}{1.345239in}}%
\pgfpathlineto{\pgfqpoint{1.780288in}{1.957028in}}%
\pgfpathlineto{\pgfqpoint{1.780698in}{1.958755in}}%
\pgfpathlineto{\pgfqpoint{1.781518in}{1.928439in}}%
\pgfpathlineto{\pgfqpoint{1.781928in}{1.896945in}}%
\pgfpathlineto{\pgfqpoint{1.783317in}{2.218000in}}%
\pgfpathmoveto{\pgfqpoint{1.792881in}{2.218000in}}%
\pgfpathlineto{\pgfqpoint{1.794228in}{2.184205in}}%
\pgfpathlineto{\pgfqpoint{1.796278in}{2.100952in}}%
\pgfpathlineto{\pgfqpoint{1.796688in}{2.116559in}}%
\pgfpathlineto{\pgfqpoint{1.799968in}{2.198606in}}%
\pgfpathlineto{\pgfqpoint{1.802018in}{2.215058in}}%
\pgfpathlineto{\pgfqpoint{1.802838in}{2.214229in}}%
\pgfpathlineto{\pgfqpoint{1.804068in}{2.204680in}}%
\pgfpathlineto{\pgfqpoint{1.806118in}{2.165083in}}%
\pgfpathlineto{\pgfqpoint{1.808988in}{2.054794in}}%
\pgfpathlineto{\pgfqpoint{1.813088in}{1.789157in}}%
\pgfpathlineto{\pgfqpoint{1.818828in}{1.405772in}}%
\pgfpathlineto{\pgfqpoint{1.822518in}{1.701530in}}%
\pgfpathlineto{\pgfqpoint{1.824158in}{1.731087in}}%
\pgfpathlineto{\pgfqpoint{1.824978in}{1.720098in}}%
\pgfpathlineto{\pgfqpoint{1.826618in}{1.650108in}}%
\pgfpathlineto{\pgfqpoint{1.828258in}{1.598015in}}%
\pgfpathlineto{\pgfqpoint{1.828668in}{1.594897in}}%
\pgfpathlineto{\pgfqpoint{1.829078in}{1.599539in}}%
\pgfpathlineto{\pgfqpoint{1.831948in}{1.684783in}}%
\pgfpathlineto{\pgfqpoint{1.833178in}{1.694579in}}%
\pgfpathlineto{\pgfqpoint{1.833588in}{1.694212in}}%
\pgfpathlineto{\pgfqpoint{1.834818in}{1.682166in}}%
\pgfpathlineto{\pgfqpoint{1.836868in}{1.626059in}}%
\pgfpathlineto{\pgfqpoint{1.840148in}{1.450725in}}%
\pgfpathlineto{\pgfqpoint{1.843018in}{1.261713in}}%
\pgfpathlineto{\pgfqpoint{1.843428in}{1.269192in}}%
\pgfpathlineto{\pgfqpoint{1.847528in}{1.322679in}}%
\pgfpathlineto{\pgfqpoint{1.849578in}{1.330939in}}%
\pgfpathlineto{\pgfqpoint{1.850398in}{1.329731in}}%
\pgfpathlineto{\pgfqpoint{1.851628in}{1.322073in}}%
\pgfpathlineto{\pgfqpoint{1.853678in}{1.289555in}}%
\pgfpathlineto{\pgfqpoint{1.856138in}{1.213905in}}%
\pgfpathlineto{\pgfqpoint{1.856548in}{1.230694in}}%
\pgfpathlineto{\pgfqpoint{1.859418in}{1.401281in}}%
\pgfpathlineto{\pgfqpoint{1.864338in}{1.756902in}}%
\pgfpathlineto{\pgfqpoint{1.865158in}{1.741580in}}%
\pgfpathlineto{\pgfqpoint{1.865568in}{1.724336in}}%
\pgfpathlineto{\pgfqpoint{1.867527in}{2.218000in}}%
\pgfpathmoveto{\pgfqpoint{1.872046in}{2.218000in}}%
\pgfpathlineto{\pgfqpoint{1.881148in}{1.520544in}}%
\pgfpathlineto{\pgfqpoint{1.881558in}{1.567718in}}%
\pgfpathlineto{\pgfqpoint{1.884018in}{1.905623in}}%
\pgfpathlineto{\pgfqpoint{1.884838in}{1.918438in}}%
\pgfpathlineto{\pgfqpoint{1.886068in}{1.877072in}}%
\pgfpathlineto{\pgfqpoint{1.886799in}{2.218000in}}%
\pgfpathmoveto{\pgfqpoint{1.915597in}{2.218000in}}%
\pgfpathlineto{\pgfqpoint{1.918868in}{1.937921in}}%
\pgfpathlineto{\pgfqpoint{1.923788in}{1.334381in}}%
\pgfpathlineto{\pgfqpoint{1.924608in}{1.393224in}}%
\pgfpathlineto{\pgfqpoint{1.927888in}{1.714851in}}%
\pgfpathlineto{\pgfqpoint{1.928298in}{1.716464in}}%
\pgfpathlineto{\pgfqpoint{1.929528in}{1.678724in}}%
\pgfpathlineto{\pgfqpoint{1.931988in}{2.136546in}}%
\pgfpathlineto{\pgfqpoint{1.932808in}{2.148561in}}%
\pgfpathlineto{\pgfqpoint{1.933218in}{2.142768in}}%
\pgfpathlineto{\pgfqpoint{1.935268in}{2.066176in}}%
\pgfpathlineto{\pgfqpoint{1.936908in}{2.006959in}}%
\pgfpathlineto{\pgfqpoint{1.938958in}{2.078394in}}%
\pgfpathlineto{\pgfqpoint{1.939368in}{2.080307in}}%
\pgfpathlineto{\pgfqpoint{1.939778in}{2.078702in}}%
\pgfpathlineto{\pgfqpoint{1.941008in}{2.054459in}}%
\pgfpathlineto{\pgfqpoint{1.943058in}{1.955744in}}%
\pgfpathlineto{\pgfqpoint{1.946338in}{1.665471in}}%
\pgfpathlineto{\pgfqpoint{1.948798in}{1.371432in}}%
\pgfpathlineto{\pgfqpoint{1.949618in}{1.382326in}}%
\pgfpathlineto{\pgfqpoint{1.952898in}{1.424731in}}%
\pgfpathlineto{\pgfqpoint{1.954128in}{1.429047in}}%
\pgfpathlineto{\pgfqpoint{1.954538in}{1.428602in}}%
\pgfpathlineto{\pgfqpoint{1.955768in}{1.420412in}}%
\pgfpathlineto{\pgfqpoint{1.957408in}{1.388391in}}%
\pgfpathlineto{\pgfqpoint{1.959458in}{1.293911in}}%
\pgfpathlineto{\pgfqpoint{1.959868in}{1.321666in}}%
\pgfpathlineto{\pgfqpoint{1.961918in}{1.554711in}}%
\pgfpathlineto{\pgfqpoint{1.965086in}{2.218000in}}%
\pgfpathmoveto{\pgfqpoint{1.968477in}{2.218000in}}%
\pgfpathlineto{\pgfqpoint{1.968478in}{2.217982in}}%
\pgfpathlineto{\pgfqpoint{1.968478in}{2.218000in}}%
\pgfpathmoveto{\pgfqpoint{1.982640in}{2.218000in}}%
\pgfpathlineto{\pgfqpoint{1.984878in}{1.742019in}}%
\pgfpathlineto{\pgfqpoint{1.985288in}{1.890933in}}%
\pgfpathlineto{\pgfqpoint{1.986298in}{2.218000in}}%
\pgfpathmoveto{\pgfqpoint{2.044480in}{2.218000in}}%
\pgfpathlineto{\pgfqpoint{2.048018in}{1.613634in}}%
\pgfpathlineto{\pgfqpoint{2.050068in}{1.273655in}}%
\pgfpathlineto{\pgfqpoint{2.050478in}{1.326394in}}%
\pgfpathlineto{\pgfqpoint{2.052528in}{1.750738in}}%
\pgfpathlineto{\pgfqpoint{2.054153in}{2.218000in}}%
\pgfpathmoveto{\pgfqpoint{2.072647in}{2.218000in}}%
\pgfpathlineto{\pgfqpoint{2.075897in}{1.734628in}}%
\pgfpathlineto{\pgfqpoint{2.077127in}{1.507672in}}%
\pgfpathlineto{\pgfqpoint{2.077947in}{1.538288in}}%
\pgfpathlineto{\pgfqpoint{2.083277in}{1.699346in}}%
\pgfpathlineto{\pgfqpoint{2.086967in}{1.751305in}}%
\pgfpathlineto{\pgfqpoint{2.090247in}{1.769931in}}%
\pgfpathlineto{\pgfqpoint{2.091887in}{1.772048in}}%
\pgfpathlineto{\pgfqpoint{2.093117in}{1.770606in}}%
\pgfpathlineto{\pgfqpoint{2.094757in}{1.764148in}}%
\pgfpathlineto{\pgfqpoint{2.096807in}{1.746447in}}%
\pgfpathlineto{\pgfqpoint{2.099267in}{1.700209in}}%
\pgfpathlineto{\pgfqpoint{2.101317in}{1.611660in}}%
\pgfpathlineto{\pgfqpoint{2.102547in}{1.522061in}}%
\pgfpathlineto{\pgfqpoint{2.102957in}{1.563561in}}%
\pgfpathlineto{\pgfqpoint{2.104597in}{1.877833in}}%
\pgfpathlineto{\pgfqpoint{2.105698in}{2.218000in}}%
\pgfpathmoveto{\pgfqpoint{2.121134in}{2.218000in}}%
\pgfpathlineto{\pgfqpoint{2.123867in}{1.811281in}}%
\pgfpathlineto{\pgfqpoint{2.125507in}{1.515763in}}%
\pgfpathlineto{\pgfqpoint{2.126327in}{1.548167in}}%
\pgfpathlineto{\pgfqpoint{2.131247in}{1.699516in}}%
\pgfpathlineto{\pgfqpoint{2.134527in}{1.739678in}}%
\pgfpathlineto{\pgfqpoint{2.136167in}{1.744750in}}%
\pgfpathlineto{\pgfqpoint{2.136577in}{1.744528in}}%
\pgfpathlineto{\pgfqpoint{2.137807in}{1.739953in}}%
\pgfpathlineto{\pgfqpoint{2.139447in}{1.722657in}}%
\pgfpathlineto{\pgfqpoint{2.141497in}{1.670922in}}%
\pgfpathlineto{\pgfqpoint{2.143137in}{1.574974in}}%
\pgfpathlineto{\pgfqpoint{2.143547in}{1.535097in}}%
\pgfpathlineto{\pgfqpoint{2.143957in}{1.537448in}}%
\pgfpathlineto{\pgfqpoint{2.145597in}{1.820796in}}%
\pgfpathlineto{\pgfqpoint{2.147001in}{2.218000in}}%
\pgfpathmoveto{\pgfqpoint{2.148833in}{2.218000in}}%
\pgfpathlineto{\pgfqpoint{2.148877in}{2.206930in}}%
\pgfpathlineto{\pgfqpoint{2.148987in}{2.218000in}}%
\pgfpathmoveto{\pgfqpoint{2.157969in}{2.218000in}}%
\pgfpathlineto{\pgfqpoint{2.160357in}{1.923116in}}%
\pgfpathlineto{\pgfqpoint{2.162817in}{1.508647in}}%
\pgfpathlineto{\pgfqpoint{2.163637in}{1.542380in}}%
\pgfpathlineto{\pgfqpoint{2.168147in}{1.682095in}}%
\pgfpathlineto{\pgfqpoint{2.170607in}{1.710109in}}%
\pgfpathlineto{\pgfqpoint{2.171427in}{1.712077in}}%
\pgfpathlineto{\pgfqpoint{2.171837in}{1.711623in}}%
\pgfpathlineto{\pgfqpoint{2.173067in}{1.703811in}}%
\pgfpathlineto{\pgfqpoint{2.174707in}{1.673437in}}%
\pgfpathlineto{\pgfqpoint{2.176347in}{1.601377in}}%
\pgfpathlineto{\pgfqpoint{2.177577in}{1.520905in}}%
\pgfpathlineto{\pgfqpoint{2.179217in}{1.787287in}}%
\pgfpathlineto{\pgfqpoint{2.181267in}{2.208374in}}%
\pgfpathlineto{\pgfqpoint{2.181677in}{2.176273in}}%
\pgfpathlineto{\pgfqpoint{2.182497in}{2.027753in}}%
\pgfpathlineto{\pgfqpoint{2.182960in}{2.218000in}}%
\pgfpathmoveto{\pgfqpoint{2.184754in}{2.218000in}}%
\pgfpathlineto{\pgfqpoint{2.185777in}{2.133166in}}%
\pgfpathlineto{\pgfqpoint{2.186187in}{2.154853in}}%
\pgfpathlineto{\pgfqpoint{2.187397in}{2.218000in}}%
\pgfpathmoveto{\pgfqpoint{2.187922in}{2.218000in}}%
\pgfpathlineto{\pgfqpoint{2.188647in}{2.198731in}}%
\pgfpathlineto{\pgfqpoint{2.190287in}{2.077277in}}%
\pgfpathlineto{\pgfqpoint{2.192747in}{1.704850in}}%
\pgfpathlineto{\pgfqpoint{2.193977in}{1.492940in}}%
\pgfpathlineto{\pgfqpoint{2.194387in}{1.510528in}}%
\pgfpathlineto{\pgfqpoint{2.198897in}{1.660509in}}%
\pgfpathlineto{\pgfqpoint{2.201357in}{1.685946in}}%
\pgfpathlineto{\pgfqpoint{2.201767in}{1.685885in}}%
\pgfpathlineto{\pgfqpoint{2.202587in}{1.681586in}}%
\pgfpathlineto{\pgfqpoint{2.204227in}{1.651620in}}%
\pgfpathlineto{\pgfqpoint{2.205867in}{1.571641in}}%
\pgfpathlineto{\pgfqpoint{2.206687in}{1.498975in}}%
\pgfpathlineto{\pgfqpoint{2.207097in}{1.541125in}}%
\pgfpathlineto{\pgfqpoint{2.209147in}{1.937185in}}%
\pgfpathlineto{\pgfqpoint{2.210377in}{2.097140in}}%
\pgfpathlineto{\pgfqpoint{2.211197in}{1.932448in}}%
\pgfpathlineto{\pgfqpoint{2.211607in}{1.961162in}}%
\pgfpathlineto{\pgfqpoint{2.212837in}{2.142924in}}%
\pgfpathlineto{\pgfqpoint{2.213247in}{2.127933in}}%
\pgfpathlineto{\pgfqpoint{2.214477in}{2.043879in}}%
\pgfpathlineto{\pgfqpoint{2.214887in}{2.068405in}}%
\pgfpathlineto{\pgfqpoint{2.216117in}{2.123179in}}%
\pgfpathlineto{\pgfqpoint{2.216527in}{2.118714in}}%
\pgfpathlineto{\pgfqpoint{2.217757in}{2.053273in}}%
\pgfpathlineto{\pgfqpoint{2.219807in}{1.791620in}}%
\pgfpathlineto{\pgfqpoint{2.221447in}{1.476679in}}%
\pgfpathlineto{\pgfqpoint{2.222267in}{1.513076in}}%
\pgfpathlineto{\pgfqpoint{2.226367in}{1.647471in}}%
\pgfpathlineto{\pgfqpoint{2.228417in}{1.665520in}}%
\pgfpathlineto{\pgfqpoint{2.229237in}{1.661685in}}%
\pgfpathlineto{\pgfqpoint{2.230467in}{1.640411in}}%
\pgfpathlineto{\pgfqpoint{2.232107in}{1.564496in}}%
\pgfpathlineto{\pgfqpoint{2.232927in}{1.488804in}}%
\pgfpathlineto{\pgfqpoint{2.233337in}{1.530475in}}%
\pgfpathlineto{\pgfqpoint{2.236207in}{2.018897in}}%
\pgfpathlineto{\pgfqpoint{2.237027in}{1.921173in}}%
\pgfpathlineto{\pgfqpoint{2.237437in}{1.796178in}}%
\pgfpathlineto{\pgfqpoint{2.237847in}{1.922694in}}%
\pgfpathlineto{\pgfqpoint{2.239077in}{2.039044in}}%
\pgfpathlineto{\pgfqpoint{2.240307in}{1.971714in}}%
\pgfpathlineto{\pgfqpoint{2.240717in}{1.996292in}}%
\pgfpathlineto{\pgfqpoint{2.241947in}{2.044128in}}%
\pgfpathlineto{\pgfqpoint{2.242767in}{2.015116in}}%
\pgfpathlineto{\pgfqpoint{2.244407in}{1.845901in}}%
\pgfpathlineto{\pgfqpoint{2.246457in}{1.466008in}}%
\pgfpathlineto{\pgfqpoint{2.247277in}{1.503752in}}%
\pgfpathlineto{\pgfqpoint{2.251377in}{1.637421in}}%
\pgfpathlineto{\pgfqpoint{2.253017in}{1.648603in}}%
\pgfpathlineto{\pgfqpoint{2.253837in}{1.642592in}}%
\pgfpathlineto{\pgfqpoint{2.255067in}{1.613812in}}%
\pgfpathlineto{\pgfqpoint{2.257117in}{1.494734in}}%
\pgfpathlineto{\pgfqpoint{2.257527in}{1.539660in}}%
\pgfpathlineto{\pgfqpoint{2.259987in}{1.952985in}}%
\pgfpathlineto{\pgfqpoint{2.260807in}{1.885617in}}%
\pgfpathlineto{\pgfqpoint{2.261217in}{1.772364in}}%
\pgfpathlineto{\pgfqpoint{2.261627in}{1.807466in}}%
\pgfpathlineto{\pgfqpoint{2.262857in}{1.962695in}}%
\pgfpathlineto{\pgfqpoint{2.263267in}{1.952974in}}%
\pgfpathlineto{\pgfqpoint{2.264087in}{1.904711in}}%
\pgfpathlineto{\pgfqpoint{2.264497in}{1.943969in}}%
\pgfpathlineto{\pgfqpoint{2.265317in}{1.979487in}}%
\pgfpathlineto{\pgfqpoint{2.265727in}{1.975154in}}%
\pgfpathlineto{\pgfqpoint{2.266957in}{1.892711in}}%
\pgfpathlineto{\pgfqpoint{2.269827in}{1.471732in}}%
\pgfpathlineto{\pgfqpoint{2.271057in}{1.528455in}}%
\pgfpathlineto{\pgfqpoint{2.274337in}{1.626949in}}%
\pgfpathlineto{\pgfqpoint{2.275567in}{1.634683in}}%
\pgfpathlineto{\pgfqpoint{2.275977in}{1.632914in}}%
\pgfpathlineto{\pgfqpoint{2.277207in}{1.611102in}}%
\pgfpathlineto{\pgfqpoint{2.278847in}{1.519869in}}%
\pgfpathlineto{\pgfqpoint{2.279257in}{1.477311in}}%
\pgfpathlineto{\pgfqpoint{2.279667in}{1.514706in}}%
\pgfpathlineto{\pgfqpoint{2.282127in}{1.903849in}}%
\pgfpathlineto{\pgfqpoint{2.282947in}{1.840423in}}%
\pgfpathlineto{\pgfqpoint{2.283357in}{1.730894in}}%
\pgfpathlineto{\pgfqpoint{2.283767in}{1.739664in}}%
\pgfpathlineto{\pgfqpoint{2.284997in}{1.899092in}}%
\pgfpathlineto{\pgfqpoint{2.285407in}{1.890675in}}%
\pgfpathlineto{\pgfqpoint{2.286227in}{1.866056in}}%
\pgfpathlineto{\pgfqpoint{2.287457in}{1.922935in}}%
\pgfpathlineto{\pgfqpoint{2.288277in}{1.883749in}}%
\pgfpathlineto{\pgfqpoint{2.289917in}{1.669196in}}%
\pgfpathlineto{\pgfqpoint{2.291147in}{1.456191in}}%
\pgfpathlineto{\pgfqpoint{2.291557in}{1.476520in}}%
\pgfpathlineto{\pgfqpoint{2.295247in}{1.610477in}}%
\pgfpathlineto{\pgfqpoint{2.296477in}{1.622620in}}%
\pgfpathlineto{\pgfqpoint{2.296887in}{1.621950in}}%
\pgfpathlineto{\pgfqpoint{2.297707in}{1.612138in}}%
\pgfpathlineto{\pgfqpoint{2.298937in}{1.568788in}}%
\pgfpathlineto{\pgfqpoint{2.300167in}{1.472817in}}%
\pgfpathlineto{\pgfqpoint{2.300577in}{1.515963in}}%
\pgfpathlineto{\pgfqpoint{2.303037in}{1.870393in}}%
\pgfpathlineto{\pgfqpoint{2.303447in}{1.840923in}}%
\pgfpathlineto{\pgfqpoint{2.304267in}{1.635698in}}%
\pgfpathlineto{\pgfqpoint{2.304677in}{1.744312in}}%
\pgfpathlineto{\pgfqpoint{2.305907in}{1.843420in}}%
\pgfpathlineto{\pgfqpoint{2.306727in}{1.811785in}}%
\pgfpathlineto{\pgfqpoint{2.307957in}{1.874660in}}%
\pgfpathlineto{\pgfqpoint{2.308777in}{1.831285in}}%
\pgfpathlineto{\pgfqpoint{2.310827in}{1.513109in}}%
\pgfpathlineto{\pgfqpoint{2.311237in}{1.446019in}}%
\pgfpathlineto{\pgfqpoint{2.312057in}{1.487658in}}%
\pgfpathlineto{\pgfqpoint{2.315337in}{1.603041in}}%
\pgfpathlineto{\pgfqpoint{2.316567in}{1.611917in}}%
\pgfpathlineto{\pgfqpoint{2.317387in}{1.603530in}}%
\pgfpathlineto{\pgfqpoint{2.318617in}{1.560278in}}%
\pgfpathlineto{\pgfqpoint{2.319847in}{1.473942in}}%
\pgfpathlineto{\pgfqpoint{2.320257in}{1.518893in}}%
\pgfpathlineto{\pgfqpoint{2.322717in}{1.831282in}}%
\pgfpathlineto{\pgfqpoint{2.323127in}{1.775376in}}%
\pgfpathlineto{\pgfqpoint{2.323947in}{1.634987in}}%
\pgfpathlineto{\pgfqpoint{2.324357in}{1.738938in}}%
\pgfpathlineto{\pgfqpoint{2.325177in}{1.799221in}}%
\pgfpathlineto{\pgfqpoint{2.325587in}{1.790824in}}%
\pgfpathlineto{\pgfqpoint{2.325997in}{1.768324in}}%
\pgfpathlineto{\pgfqpoint{2.326407in}{1.806639in}}%
\pgfpathlineto{\pgfqpoint{2.327227in}{1.832583in}}%
\pgfpathlineto{\pgfqpoint{2.327637in}{1.817726in}}%
\pgfpathlineto{\pgfqpoint{2.329277in}{1.616923in}}%
\pgfpathlineto{\pgfqpoint{2.330097in}{1.448624in}}%
\pgfpathlineto{\pgfqpoint{2.330917in}{1.475231in}}%
\pgfpathlineto{\pgfqpoint{2.334197in}{1.594848in}}%
\pgfpathlineto{\pgfqpoint{2.335017in}{1.602520in}}%
\pgfpathlineto{\pgfqpoint{2.335427in}{1.601958in}}%
\pgfpathlineto{\pgfqpoint{2.336247in}{1.590312in}}%
\pgfpathlineto{\pgfqpoint{2.337477in}{1.536124in}}%
\pgfpathlineto{\pgfqpoint{2.338297in}{1.460720in}}%
\pgfpathlineto{\pgfqpoint{2.338707in}{1.504203in}}%
\pgfpathlineto{\pgfqpoint{2.340757in}{1.802663in}}%
\pgfpathlineto{\pgfqpoint{2.341577in}{1.742479in}}%
\pgfpathlineto{\pgfqpoint{2.342397in}{1.607362in}}%
\pgfpathlineto{\pgfqpoint{2.342807in}{1.704735in}}%
\pgfpathlineto{\pgfqpoint{2.343627in}{1.758375in}}%
\pgfpathlineto{\pgfqpoint{2.344037in}{1.747336in}}%
\pgfpathlineto{\pgfqpoint{2.344447in}{1.743049in}}%
\pgfpathlineto{\pgfqpoint{2.345267in}{1.795615in}}%
\pgfpathlineto{\pgfqpoint{2.345677in}{1.788929in}}%
\pgfpathlineto{\pgfqpoint{2.346907in}{1.669271in}}%
\pgfpathlineto{\pgfqpoint{2.348137in}{1.432685in}}%
\pgfpathlineto{\pgfqpoint{2.348957in}{1.472810in}}%
\pgfpathlineto{\pgfqpoint{2.352237in}{1.589881in}}%
\pgfpathlineto{\pgfqpoint{2.353057in}{1.593949in}}%
\pgfpathlineto{\pgfqpoint{2.353877in}{1.583637in}}%
\pgfpathlineto{\pgfqpoint{2.355107in}{1.529234in}}%
\pgfpathlineto{\pgfqpoint{2.355927in}{1.460960in}}%
\pgfpathlineto{\pgfqpoint{2.356337in}{1.505936in}}%
\pgfpathlineto{\pgfqpoint{2.358387in}{1.784136in}}%
\pgfpathlineto{\pgfqpoint{2.358797in}{1.761994in}}%
\pgfpathlineto{\pgfqpoint{2.359617in}{1.572372in}}%
\pgfpathlineto{\pgfqpoint{2.360027in}{1.619203in}}%
\pgfpathlineto{\pgfqpoint{2.360847in}{1.719460in}}%
\pgfpathlineto{\pgfqpoint{2.361257in}{1.718932in}}%
\pgfpathlineto{\pgfqpoint{2.361667in}{1.699604in}}%
\pgfpathlineto{\pgfqpoint{2.362487in}{1.759872in}}%
\pgfpathlineto{\pgfqpoint{2.362897in}{1.757763in}}%
\pgfpathlineto{\pgfqpoint{2.364127in}{1.641135in}}%
\pgfpathlineto{\pgfqpoint{2.365357in}{1.429569in}}%
\pgfpathlineto{\pgfqpoint{2.366177in}{1.475173in}}%
\pgfpathlineto{\pgfqpoint{2.369047in}{1.580736in}}%
\pgfpathlineto{\pgfqpoint{2.369867in}{1.586373in}}%
\pgfpathlineto{\pgfqpoint{2.370277in}{1.583665in}}%
\pgfpathlineto{\pgfqpoint{2.371507in}{1.546083in}}%
\pgfpathlineto{\pgfqpoint{2.372737in}{1.465452in}}%
\pgfpathlineto{\pgfqpoint{2.375197in}{1.760319in}}%
\pgfpathlineto{\pgfqpoint{2.375607in}{1.712706in}}%
\pgfpathlineto{\pgfqpoint{2.376427in}{1.523700in}}%
\pgfpathlineto{\pgfqpoint{2.376837in}{1.628995in}}%
\pgfpathlineto{\pgfqpoint{2.378477in}{1.697438in}}%
\pgfpathlineto{\pgfqpoint{2.379297in}{1.728596in}}%
\pgfpathlineto{\pgfqpoint{2.379707in}{1.709100in}}%
\pgfpathlineto{\pgfqpoint{2.381757in}{1.431275in}}%
\pgfpathlineto{\pgfqpoint{2.382987in}{1.498775in}}%
\pgfpathlineto{\pgfqpoint{2.385447in}{1.577172in}}%
\pgfpathlineto{\pgfqpoint{2.385857in}{1.579329in}}%
\pgfpathlineto{\pgfqpoint{2.386267in}{1.577463in}}%
\pgfpathlineto{\pgfqpoint{2.387497in}{1.540803in}}%
\pgfpathlineto{\pgfqpoint{2.388727in}{1.465858in}}%
\pgfpathlineto{\pgfqpoint{2.390777in}{1.740479in}}%
\pgfpathlineto{\pgfqpoint{2.391187in}{1.732875in}}%
\pgfpathlineto{\pgfqpoint{2.392417in}{1.540377in}}%
\pgfpathlineto{\pgfqpoint{2.392827in}{1.622617in}}%
\pgfpathlineto{\pgfqpoint{2.394877in}{1.701789in}}%
\pgfpathlineto{\pgfqpoint{2.395697in}{1.654656in}}%
\pgfpathlineto{\pgfqpoint{2.397337in}{1.428604in}}%
\pgfpathlineto{\pgfqpoint{2.398157in}{1.476024in}}%
\pgfpathlineto{\pgfqpoint{2.401027in}{1.572117in}}%
\pgfpathlineto{\pgfqpoint{2.401437in}{1.572314in}}%
\pgfpathlineto{\pgfqpoint{2.402257in}{1.558002in}}%
\pgfpathlineto{\pgfqpoint{2.403897in}{1.454450in}}%
\pgfpathlineto{\pgfqpoint{2.404307in}{1.502061in}}%
\pgfpathlineto{\pgfqpoint{2.405947in}{1.723942in}}%
\pgfpathlineto{\pgfqpoint{2.406357in}{1.715181in}}%
\pgfpathlineto{\pgfqpoint{2.407587in}{1.521713in}}%
\pgfpathlineto{\pgfqpoint{2.407997in}{1.599887in}}%
\pgfpathlineto{\pgfqpoint{2.410047in}{1.673604in}}%
\pgfpathlineto{\pgfqpoint{2.410867in}{1.608630in}}%
\pgfpathlineto{\pgfqpoint{2.412097in}{1.417144in}}%
\pgfpathlineto{\pgfqpoint{2.412507in}{1.442555in}}%
\pgfpathlineto{\pgfqpoint{2.415377in}{1.562072in}}%
\pgfpathlineto{\pgfqpoint{2.415787in}{1.566214in}}%
\pgfpathlineto{\pgfqpoint{2.416197in}{1.565886in}}%
\pgfpathlineto{\pgfqpoint{2.417017in}{1.549011in}}%
\pgfpathlineto{\pgfqpoint{2.418247in}{1.462796in}}%
\pgfpathlineto{\pgfqpoint{2.418657in}{1.469368in}}%
\pgfpathlineto{\pgfqpoint{2.420707in}{1.710574in}}%
\pgfpathlineto{\pgfqpoint{2.421117in}{1.666860in}}%
\pgfpathlineto{\pgfqpoint{2.421937in}{1.461822in}}%
\pgfpathlineto{\pgfqpoint{2.422347in}{1.554509in}}%
\pgfpathlineto{\pgfqpoint{2.423987in}{1.642581in}}%
\pgfpathlineto{\pgfqpoint{2.424397in}{1.650041in}}%
\pgfpathlineto{\pgfqpoint{2.425217in}{1.586593in}}%
\pgfpathlineto{\pgfqpoint{2.426447in}{1.419197in}}%
\pgfpathlineto{\pgfqpoint{2.426857in}{1.445017in}}%
\pgfpathlineto{\pgfqpoint{2.429727in}{1.559228in}}%
\pgfpathlineto{\pgfqpoint{2.430137in}{1.560876in}}%
\pgfpathlineto{\pgfqpoint{2.430957in}{1.547724in}}%
\pgfpathlineto{\pgfqpoint{2.432597in}{1.459985in}}%
\pgfpathlineto{\pgfqpoint{2.434647in}{1.696209in}}%
\pgfpathlineto{\pgfqpoint{2.435057in}{1.649691in}}%
\pgfpathlineto{\pgfqpoint{2.435877in}{1.443551in}}%
\pgfpathlineto{\pgfqpoint{2.436287in}{1.538419in}}%
\pgfpathlineto{\pgfqpoint{2.437927in}{1.625205in}}%
\pgfpathlineto{\pgfqpoint{2.438337in}{1.623434in}}%
\pgfpathlineto{\pgfqpoint{2.439567in}{1.467697in}}%
\pgfpathlineto{\pgfqpoint{2.439977in}{1.405361in}}%
\pgfpathlineto{\pgfqpoint{2.440387in}{1.432449in}}%
\pgfpathlineto{\pgfqpoint{2.443257in}{1.553626in}}%
\pgfpathlineto{\pgfqpoint{2.443667in}{1.555535in}}%
\pgfpathlineto{\pgfqpoint{2.444487in}{1.541740in}}%
\pgfpathlineto{\pgfqpoint{2.445717in}{1.454933in}}%
\pgfpathlineto{\pgfqpoint{2.446127in}{1.467678in}}%
\pgfpathlineto{\pgfqpoint{2.447767in}{1.682624in}}%
\pgfpathlineto{\pgfqpoint{2.448177in}{1.672746in}}%
\pgfpathlineto{\pgfqpoint{2.449407in}{1.472491in}}%
\pgfpathlineto{\pgfqpoint{2.449817in}{1.540392in}}%
\pgfpathlineto{\pgfqpoint{2.451457in}{1.606132in}}%
\pgfpathlineto{\pgfqpoint{2.452277in}{1.537297in}}%
\pgfpathlineto{\pgfqpoint{2.453097in}{1.399692in}}%
\pgfpathlineto{\pgfqpoint{2.453507in}{1.427645in}}%
\pgfpathlineto{\pgfqpoint{2.456377in}{1.549469in}}%
\pgfpathlineto{\pgfqpoint{2.456787in}{1.550098in}}%
\pgfpathlineto{\pgfqpoint{2.457607in}{1.531654in}}%
\pgfpathlineto{\pgfqpoint{2.458837in}{1.440471in}}%
\pgfpathlineto{\pgfqpoint{2.459247in}{1.490162in}}%
\pgfpathlineto{\pgfqpoint{2.460887in}{1.671783in}}%
\pgfpathlineto{\pgfqpoint{2.462117in}{1.418808in}}%
\pgfpathlineto{\pgfqpoint{2.462937in}{1.537701in}}%
\pgfpathlineto{\pgfqpoint{2.463347in}{1.531351in}}%
\pgfpathlineto{\pgfqpoint{2.464167in}{1.586702in}}%
\pgfpathlineto{\pgfqpoint{2.464577in}{1.567784in}}%
\pgfpathlineto{\pgfqpoint{2.465807in}{1.400439in}}%
\pgfpathlineto{\pgfqpoint{2.466627in}{1.455402in}}%
\pgfpathlineto{\pgfqpoint{2.469087in}{1.545675in}}%
\pgfpathlineto{\pgfqpoint{2.469497in}{1.543539in}}%
\pgfpathlineto{\pgfqpoint{2.470317in}{1.516330in}}%
\pgfpathlineto{\pgfqpoint{2.471137in}{1.444506in}}%
\pgfpathlineto{\pgfqpoint{2.471547in}{1.469893in}}%
\pgfpathlineto{\pgfqpoint{2.473187in}{1.663548in}}%
\pgfpathlineto{\pgfqpoint{2.473597in}{1.625633in}}%
\pgfpathlineto{\pgfqpoint{2.474417in}{1.424925in}}%
\pgfpathlineto{\pgfqpoint{2.474827in}{1.480242in}}%
\pgfpathlineto{\pgfqpoint{2.476467in}{1.567734in}}%
\pgfpathlineto{\pgfqpoint{2.477287in}{1.495688in}}%
\pgfpathlineto{\pgfqpoint{2.478107in}{1.405642in}}%
\pgfpathlineto{\pgfqpoint{2.478517in}{1.434266in}}%
\pgfpathlineto{\pgfqpoint{2.480977in}{1.540067in}}%
\pgfpathlineto{\pgfqpoint{2.481387in}{1.540721in}}%
\pgfpathlineto{\pgfqpoint{2.482207in}{1.519387in}}%
\pgfpathlineto{\pgfqpoint{2.483027in}{1.454569in}}%
\pgfpathlineto{\pgfqpoint{2.483437in}{1.455631in}}%
\pgfpathlineto{\pgfqpoint{2.485077in}{1.654139in}}%
\pgfpathlineto{\pgfqpoint{2.485487in}{1.621615in}}%
\pgfpathlineto{\pgfqpoint{2.486307in}{1.423955in}}%
\pgfpathlineto{\pgfqpoint{2.487127in}{1.498797in}}%
\pgfpathlineto{\pgfqpoint{2.487537in}{1.494015in}}%
\pgfpathlineto{\pgfqpoint{2.488357in}{1.549057in}}%
\pgfpathlineto{\pgfqpoint{2.488767in}{1.522366in}}%
\pgfpathlineto{\pgfqpoint{2.489587in}{1.383445in}}%
\pgfpathlineto{\pgfqpoint{2.490407in}{1.441825in}}%
\pgfpathlineto{\pgfqpoint{2.492867in}{1.536944in}}%
\pgfpathlineto{\pgfqpoint{2.493687in}{1.520954in}}%
\pgfpathlineto{\pgfqpoint{2.494917in}{1.443584in}}%
\pgfpathlineto{\pgfqpoint{2.496557in}{1.645042in}}%
\pgfpathlineto{\pgfqpoint{2.496967in}{1.616019in}}%
\pgfpathlineto{\pgfqpoint{2.497787in}{1.420061in}}%
\pgfpathlineto{\pgfqpoint{2.498607in}{1.481893in}}%
\pgfpathlineto{\pgfqpoint{2.499017in}{1.476628in}}%
\pgfpathlineto{\pgfqpoint{2.499837in}{1.530305in}}%
\pgfpathlineto{\pgfqpoint{2.500247in}{1.497134in}}%
\pgfpathlineto{\pgfqpoint{2.501067in}{1.390774in}}%
\pgfpathlineto{\pgfqpoint{2.501477in}{1.421470in}}%
\pgfpathlineto{\pgfqpoint{2.503937in}{1.532304in}}%
\pgfpathlineto{\pgfqpoint{2.504347in}{1.531406in}}%
\pgfpathlineto{\pgfqpoint{2.505167in}{1.502649in}}%
\pgfpathlineto{\pgfqpoint{2.505987in}{1.430714in}}%
\pgfpathlineto{\pgfqpoint{2.506397in}{1.483212in}}%
\pgfpathlineto{\pgfqpoint{2.507627in}{1.636111in}}%
\pgfpathlineto{\pgfqpoint{2.508037in}{1.612530in}}%
\pgfpathlineto{\pgfqpoint{2.508857in}{1.419620in}}%
\pgfpathlineto{\pgfqpoint{2.509677in}{1.465517in}}%
\pgfpathlineto{\pgfqpoint{2.510087in}{1.466241in}}%
\pgfpathlineto{\pgfqpoint{2.510497in}{1.512721in}}%
\pgfpathlineto{\pgfqpoint{2.510907in}{1.512179in}}%
\pgfpathlineto{\pgfqpoint{2.512137in}{1.397734in}}%
\pgfpathlineto{\pgfqpoint{2.512547in}{1.428414in}}%
\pgfpathlineto{\pgfqpoint{2.515007in}{1.528855in}}%
\pgfpathlineto{\pgfqpoint{2.515827in}{1.507896in}}%
\pgfpathlineto{\pgfqpoint{2.516647in}{1.435758in}}%
\pgfpathlineto{\pgfqpoint{2.517057in}{1.465045in}}%
\pgfpathlineto{\pgfqpoint{2.518287in}{1.625855in}}%
\pgfpathlineto{\pgfqpoint{2.518697in}{1.613291in}}%
\pgfpathlineto{\pgfqpoint{2.519927in}{1.404576in}}%
\pgfpathlineto{\pgfqpoint{2.520337in}{1.448779in}}%
\pgfpathlineto{\pgfqpoint{2.520747in}{1.450070in}}%
\pgfpathlineto{\pgfqpoint{2.521157in}{1.497877in}}%
\pgfpathlineto{\pgfqpoint{2.521567in}{1.496375in}}%
\pgfpathlineto{\pgfqpoint{2.522387in}{1.380951in}}%
\pgfpathlineto{\pgfqpoint{2.523207in}{1.432555in}}%
\pgfpathlineto{\pgfqpoint{2.525257in}{1.524465in}}%
\pgfpathlineto{\pgfqpoint{2.525667in}{1.523790in}}%
\pgfpathlineto{\pgfqpoint{2.526487in}{1.492057in}}%
\pgfpathlineto{\pgfqpoint{2.527307in}{1.440755in}}%
\pgfpathlineto{\pgfqpoint{2.528947in}{1.616867in}}%
\pgfpathlineto{\pgfqpoint{2.530177in}{1.369847in}}%
\pgfpathlineto{\pgfqpoint{2.530997in}{1.432632in}}%
\pgfpathlineto{\pgfqpoint{2.531817in}{1.484288in}}%
\pgfpathlineto{\pgfqpoint{2.532637in}{1.371703in}}%
\pgfpathlineto{\pgfqpoint{2.533457in}{1.432444in}}%
\pgfpathlineto{\pgfqpoint{2.535507in}{1.521506in}}%
\pgfpathlineto{\pgfqpoint{2.535917in}{1.518402in}}%
\pgfpathlineto{\pgfqpoint{2.536737in}{1.477931in}}%
\pgfpathlineto{\pgfqpoint{2.537147in}{1.433836in}}%
\pgfpathlineto{\pgfqpoint{2.537557in}{1.461271in}}%
\pgfpathlineto{\pgfqpoint{2.538787in}{1.614540in}}%
\pgfpathlineto{\pgfqpoint{2.539197in}{1.586323in}}%
\pgfpathlineto{\pgfqpoint{2.540017in}{1.386570in}}%
\pgfpathlineto{\pgfqpoint{2.540837in}{1.422483in}}%
\pgfpathlineto{\pgfqpoint{2.541657in}{1.474391in}}%
\pgfpathlineto{\pgfqpoint{2.542067in}{1.446209in}}%
\pgfpathlineto{\pgfqpoint{2.542477in}{1.379358in}}%
\pgfpathlineto{\pgfqpoint{2.543297in}{1.426513in}}%
\pgfpathlineto{\pgfqpoint{2.545347in}{1.518171in}}%
\pgfpathlineto{\pgfqpoint{2.545757in}{1.514712in}}%
\pgfpathlineto{\pgfqpoint{2.546577in}{1.472025in}}%
\pgfpathlineto{\pgfqpoint{2.546987in}{1.425830in}}%
\pgfpathlineto{\pgfqpoint{2.547397in}{1.466037in}}%
\pgfpathlineto{\pgfqpoint{2.548627in}{1.607800in}}%
\pgfpathlineto{\pgfqpoint{2.549037in}{1.568707in}}%
\pgfpathlineto{\pgfqpoint{2.549857in}{1.367268in}}%
\pgfpathlineto{\pgfqpoint{2.550677in}{1.407556in}}%
\pgfpathlineto{\pgfqpoint{2.551497in}{1.458122in}}%
\pgfpathlineto{\pgfqpoint{2.551907in}{1.421924in}}%
\pgfpathlineto{\pgfqpoint{2.552317in}{1.366216in}}%
\pgfpathlineto{\pgfqpoint{2.552727in}{1.400721in}}%
\pgfpathlineto{\pgfqpoint{2.555187in}{1.515189in}}%
\pgfpathlineto{\pgfqpoint{2.556007in}{1.491413in}}%
\pgfpathlineto{\pgfqpoint{2.556827in}{1.425121in}}%
\pgfpathlineto{\pgfqpoint{2.557237in}{1.479361in}}%
\pgfpathlineto{\pgfqpoint{2.558467in}{1.597475in}}%
\pgfpathlineto{\pgfqpoint{2.559697in}{1.342221in}}%
\pgfpathlineto{\pgfqpoint{2.560517in}{1.391759in}}%
\pgfpathlineto{\pgfqpoint{2.560927in}{1.442419in}}%
\pgfpathlineto{\pgfqpoint{2.561337in}{1.439530in}}%
\pgfpathlineto{\pgfqpoint{2.562157in}{1.375313in}}%
\pgfpathlineto{\pgfqpoint{2.562567in}{1.408707in}}%
\pgfpathlineto{\pgfqpoint{2.565027in}{1.511505in}}%
\pgfpathlineto{\pgfqpoint{2.565847in}{1.480368in}}%
\pgfpathlineto{\pgfqpoint{2.566667in}{1.439847in}}%
\pgfpathlineto{\pgfqpoint{2.567897in}{1.591385in}}%
\pgfpathlineto{\pgfqpoint{2.568307in}{1.581098in}}%
\pgfpathlineto{\pgfqpoint{2.569537in}{1.342985in}}%
\pgfpathlineto{\pgfqpoint{2.570357in}{1.400873in}}%
\pgfpathlineto{\pgfqpoint{2.570767in}{1.433833in}}%
\pgfpathlineto{\pgfqpoint{2.571177in}{1.417245in}}%
\pgfpathlineto{\pgfqpoint{2.571587in}{1.360858in}}%
\pgfpathlineto{\pgfqpoint{2.571997in}{1.385665in}}%
\pgfpathlineto{\pgfqpoint{2.574457in}{1.509496in}}%
\pgfpathlineto{\pgfqpoint{2.574867in}{1.506651in}}%
\pgfpathlineto{\pgfqpoint{2.575687in}{1.466534in}}%
\pgfpathlineto{\pgfqpoint{2.576097in}{1.423019in}}%
\pgfpathlineto{\pgfqpoint{2.576507in}{1.457289in}}%
\pgfpathlineto{\pgfqpoint{2.577737in}{1.588801in}}%
\pgfpathlineto{\pgfqpoint{2.578147in}{1.556845in}}%
\pgfpathlineto{\pgfqpoint{2.579377in}{1.351742in}}%
\pgfpathlineto{\pgfqpoint{2.579787in}{1.371897in}}%
\pgfpathlineto{\pgfqpoint{2.580607in}{1.421002in}}%
\pgfpathlineto{\pgfqpoint{2.581017in}{1.389971in}}%
\pgfpathlineto{\pgfqpoint{2.581427in}{1.363130in}}%
\pgfpathlineto{\pgfqpoint{2.581837in}{1.396984in}}%
\pgfpathlineto{\pgfqpoint{2.584297in}{1.506896in}}%
\pgfpathlineto{\pgfqpoint{2.585117in}{1.481829in}}%
\pgfpathlineto{\pgfqpoint{2.585937in}{1.425396in}}%
\pgfpathlineto{\pgfqpoint{2.587577in}{1.577875in}}%
\pgfpathlineto{\pgfqpoint{2.588807in}{1.338289in}}%
\pgfpathlineto{\pgfqpoint{2.589627in}{1.359398in}}%
\pgfpathlineto{\pgfqpoint{2.590037in}{1.406635in}}%
\pgfpathlineto{\pgfqpoint{2.590447in}{1.402332in}}%
\pgfpathlineto{\pgfqpoint{2.590857in}{1.356596in}}%
\pgfpathlineto{\pgfqpoint{2.591267in}{1.376662in}}%
\pgfpathlineto{\pgfqpoint{2.593727in}{1.504041in}}%
\pgfpathlineto{\pgfqpoint{2.594137in}{1.502681in}}%
\pgfpathlineto{\pgfqpoint{2.594957in}{1.467191in}}%
\pgfpathlineto{\pgfqpoint{2.595367in}{1.427419in}}%
\pgfpathlineto{\pgfqpoint{2.595777in}{1.445964in}}%
\pgfpathlineto{\pgfqpoint{2.597007in}{1.577485in}}%
\pgfpathlineto{\pgfqpoint{2.597417in}{1.556173in}}%
\pgfpathlineto{\pgfqpoint{2.598647in}{1.320057in}}%
\pgfpathlineto{\pgfqpoint{2.599467in}{1.375785in}}%
\pgfpathlineto{\pgfqpoint{2.599877in}{1.399400in}}%
\pgfpathlineto{\pgfqpoint{2.600287in}{1.376517in}}%
\pgfpathlineto{\pgfqpoint{2.600697in}{1.356903in}}%
\pgfpathlineto{\pgfqpoint{2.603567in}{1.502157in}}%
\pgfpathlineto{\pgfqpoint{2.604387in}{1.479240in}}%
\pgfpathlineto{\pgfqpoint{2.605207in}{1.419396in}}%
\pgfpathlineto{\pgfqpoint{2.605617in}{1.468973in}}%
\pgfpathlineto{\pgfqpoint{2.606847in}{1.569488in}}%
\pgfpathlineto{\pgfqpoint{2.608487in}{1.330774in}}%
\pgfpathlineto{\pgfqpoint{2.609307in}{1.382666in}}%
\pgfpathlineto{\pgfqpoint{2.609717in}{1.383460in}}%
\pgfpathlineto{\pgfqpoint{2.610127in}{1.342576in}}%
\pgfpathlineto{\pgfqpoint{2.610537in}{1.373027in}}%
\pgfpathlineto{\pgfqpoint{2.612997in}{1.499370in}}%
\pgfpathlineto{\pgfqpoint{2.613407in}{1.498105in}}%
\pgfpathlineto{\pgfqpoint{2.614227in}{1.463365in}}%
\pgfpathlineto{\pgfqpoint{2.614637in}{1.424546in}}%
\pgfpathlineto{\pgfqpoint{2.615047in}{1.442670in}}%
\pgfpathlineto{\pgfqpoint{2.616277in}{1.567764in}}%
\pgfpathlineto{\pgfqpoint{2.616687in}{1.547949in}}%
\pgfpathlineto{\pgfqpoint{2.617917in}{1.306829in}}%
\pgfpathlineto{\pgfqpoint{2.618737in}{1.353636in}}%
\pgfpathlineto{\pgfqpoint{2.619147in}{1.378740in}}%
\pgfpathlineto{\pgfqpoint{2.619557in}{1.357824in}}%
\pgfpathlineto{\pgfqpoint{2.619967in}{1.355900in}}%
\pgfpathlineto{\pgfqpoint{2.622427in}{1.494953in}}%
\pgfpathlineto{\pgfqpoint{2.622837in}{1.497621in}}%
\pgfpathlineto{\pgfqpoint{2.623657in}{1.473735in}}%
\pgfpathlineto{\pgfqpoint{2.624477in}{1.419985in}}%
\pgfpathlineto{\pgfqpoint{2.626117in}{1.559184in}}%
\pgfpathlineto{\pgfqpoint{2.627757in}{1.312725in}}%
\pgfpathlineto{\pgfqpoint{2.628577in}{1.363675in}}%
\pgfpathlineto{\pgfqpoint{2.628987in}{1.363128in}}%
\pgfpathlineto{\pgfqpoint{2.629397in}{1.339393in}}%
\pgfpathlineto{\pgfqpoint{2.629807in}{1.374025in}}%
\pgfpathlineto{\pgfqpoint{2.632267in}{1.495409in}}%
\pgfpathlineto{\pgfqpoint{2.632677in}{1.493064in}}%
\pgfpathlineto{\pgfqpoint{2.633497in}{1.455869in}}%
\pgfpathlineto{\pgfqpoint{2.633907in}{1.415787in}}%
\pgfpathlineto{\pgfqpoint{2.634317in}{1.445766in}}%
\pgfpathlineto{\pgfqpoint{2.635547in}{1.559398in}}%
\pgfpathlineto{\pgfqpoint{2.635957in}{1.534489in}}%
\pgfpathlineto{\pgfqpoint{2.637187in}{1.300628in}}%
\pgfpathlineto{\pgfqpoint{2.638007in}{1.339580in}}%
\pgfpathlineto{\pgfqpoint{2.638827in}{1.335348in}}%
\pgfpathlineto{\pgfqpoint{2.642107in}{1.493075in}}%
\pgfpathlineto{\pgfqpoint{2.642927in}{1.465500in}}%
\pgfpathlineto{\pgfqpoint{2.643747in}{1.426067in}}%
\pgfpathlineto{\pgfqpoint{2.644977in}{1.552016in}}%
\pgfpathlineto{\pgfqpoint{2.645387in}{1.546482in}}%
\pgfpathlineto{\pgfqpoint{2.647027in}{1.299293in}}%
\pgfpathlineto{\pgfqpoint{2.647847in}{1.348423in}}%
\pgfpathlineto{\pgfqpoint{2.648257in}{1.341222in}}%
\pgfpathlineto{\pgfqpoint{2.648667in}{1.345290in}}%
\pgfpathlineto{\pgfqpoint{2.651537in}{1.491705in}}%
\pgfpathlineto{\pgfqpoint{2.652357in}{1.472307in}}%
\pgfpathlineto{\pgfqpoint{2.653177in}{1.408883in}}%
\pgfpathlineto{\pgfqpoint{2.653587in}{1.454167in}}%
\pgfpathlineto{\pgfqpoint{2.654817in}{1.550123in}}%
\pgfpathlineto{\pgfqpoint{2.655227in}{1.515405in}}%
\pgfpathlineto{\pgfqpoint{2.656457in}{1.287238in}}%
\pgfpathlineto{\pgfqpoint{2.657277in}{1.330630in}}%
\pgfpathlineto{\pgfqpoint{2.658507in}{1.366464in}}%
\pgfpathlineto{\pgfqpoint{2.660967in}{1.489316in}}%
\pgfpathlineto{\pgfqpoint{2.661377in}{1.487958in}}%
\pgfpathlineto{\pgfqpoint{2.662197in}{1.454116in}}%
\pgfpathlineto{\pgfqpoint{2.662607in}{1.416891in}}%
\pgfpathlineto{\pgfqpoint{2.663017in}{1.436929in}}%
\pgfpathlineto{\pgfqpoint{2.664247in}{1.547567in}}%
\pgfpathlineto{\pgfqpoint{2.664657in}{1.529205in}}%
\pgfpathlineto{\pgfqpoint{2.666297in}{1.287420in}}%
\pgfpathlineto{\pgfqpoint{2.666707in}{1.307308in}}%
\pgfpathlineto{\pgfqpoint{2.670397in}{1.486192in}}%
\pgfpathlineto{\pgfqpoint{2.670807in}{1.487551in}}%
\pgfpathlineto{\pgfqpoint{2.671627in}{1.461174in}}%
\pgfpathlineto{\pgfqpoint{2.672447in}{1.421701in}}%
\pgfpathlineto{\pgfqpoint{2.673677in}{1.540864in}}%
\pgfpathlineto{\pgfqpoint{2.674087in}{1.536660in}}%
\pgfpathlineto{\pgfqpoint{2.675727in}{1.274098in}}%
\pgfpathlineto{\pgfqpoint{2.676957in}{1.318495in}}%
\pgfpathlineto{\pgfqpoint{2.680237in}{1.486276in}}%
\pgfpathlineto{\pgfqpoint{2.680647in}{1.481444in}}%
\pgfpathlineto{\pgfqpoint{2.681877in}{1.408275in}}%
\pgfpathlineto{\pgfqpoint{2.682287in}{1.451928in}}%
\pgfpathlineto{\pgfqpoint{2.683517in}{1.539025in}}%
\pgfpathlineto{\pgfqpoint{2.684747in}{1.358073in}}%
\pgfpathlineto{\pgfqpoint{2.685567in}{1.273401in}}%
\pgfpathlineto{\pgfqpoint{2.685977in}{1.307253in}}%
\pgfpathlineto{\pgfqpoint{2.686797in}{1.332788in}}%
\pgfpathlineto{\pgfqpoint{2.689667in}{1.484362in}}%
\pgfpathlineto{\pgfqpoint{2.690077in}{1.482029in}}%
\pgfpathlineto{\pgfqpoint{2.690897in}{1.446231in}}%
\pgfpathlineto{\pgfqpoint{2.691307in}{1.408253in}}%
\pgfpathlineto{\pgfqpoint{2.691717in}{1.438482in}}%
\pgfpathlineto{\pgfqpoint{2.692947in}{1.537602in}}%
\pgfpathlineto{\pgfqpoint{2.693357in}{1.515549in}}%
\pgfpathlineto{\pgfqpoint{2.694997in}{1.266217in}}%
\pgfpathlineto{\pgfqpoint{2.695817in}{1.311054in}}%
\pgfpathlineto{\pgfqpoint{2.696637in}{1.357346in}}%
\pgfpathlineto{\pgfqpoint{2.699097in}{1.481991in}}%
\pgfpathlineto{\pgfqpoint{2.699507in}{1.481836in}}%
\pgfpathlineto{\pgfqpoint{2.700327in}{1.452030in}}%
\pgfpathlineto{\pgfqpoint{2.700737in}{1.418163in}}%
\pgfpathlineto{\pgfqpoint{2.701147in}{1.426502in}}%
\pgfpathlineto{\pgfqpoint{2.702377in}{1.533559in}}%
\pgfpathlineto{\pgfqpoint{2.702787in}{1.522240in}}%
\pgfpathlineto{\pgfqpoint{2.704427in}{1.263335in}}%
\pgfpathlineto{\pgfqpoint{2.705247in}{1.302538in}}%
\pgfpathlineto{\pgfqpoint{2.706067in}{1.348636in}}%
\pgfpathlineto{\pgfqpoint{2.708526in}{1.479308in}}%
\pgfpathlineto{\pgfqpoint{2.708936in}{1.481055in}}%
\pgfpathlineto{\pgfqpoint{2.709756in}{1.456456in}}%
\pgfpathlineto{\pgfqpoint{2.710576in}{1.415862in}}%
\pgfpathlineto{\pgfqpoint{2.711806in}{1.527846in}}%
\pgfpathlineto{\pgfqpoint{2.712216in}{1.525652in}}%
\pgfpathlineto{\pgfqpoint{2.714266in}{1.251393in}}%
\pgfpathlineto{\pgfqpoint{2.715086in}{1.304730in}}%
\pgfpathlineto{\pgfqpoint{2.717956in}{1.476429in}}%
\pgfpathlineto{\pgfqpoint{2.718366in}{1.479837in}}%
\pgfpathlineto{\pgfqpoint{2.718776in}{1.474834in}}%
\pgfpathlineto{\pgfqpoint{2.720006in}{1.406430in}}%
\pgfpathlineto{\pgfqpoint{2.720416in}{1.448015in}}%
\pgfpathlineto{\pgfqpoint{2.721646in}{1.526526in}}%
\pgfpathlineto{\pgfqpoint{2.722876in}{1.357054in}}%
\pgfpathlineto{\pgfqpoint{2.723696in}{1.246981in}}%
\pgfpathlineto{\pgfqpoint{2.724106in}{1.278943in}}%
\pgfpathlineto{\pgfqpoint{2.725746in}{1.395699in}}%
\pgfpathlineto{\pgfqpoint{2.727796in}{1.478305in}}%
\pgfpathlineto{\pgfqpoint{2.728206in}{1.475073in}}%
\pgfpathlineto{\pgfqpoint{2.729026in}{1.437731in}}%
\pgfpathlineto{\pgfqpoint{2.729436in}{1.399453in}}%
\pgfpathlineto{\pgfqpoint{2.729846in}{1.438615in}}%
\pgfpathlineto{\pgfqpoint{2.731076in}{1.525531in}}%
\pgfpathlineto{\pgfqpoint{2.731486in}{1.501014in}}%
\pgfpathlineto{\pgfqpoint{2.733126in}{1.249342in}}%
\pgfpathlineto{\pgfqpoint{2.733946in}{1.289194in}}%
\pgfpathlineto{\pgfqpoint{2.736816in}{1.470433in}}%
\pgfpathlineto{\pgfqpoint{2.737226in}{1.476558in}}%
\pgfpathlineto{\pgfqpoint{2.737636in}{1.474871in}}%
\pgfpathlineto{\pgfqpoint{2.738456in}{1.441795in}}%
\pgfpathlineto{\pgfqpoint{2.738866in}{1.406417in}}%
\pgfpathlineto{\pgfqpoint{2.739276in}{1.430229in}}%
\pgfpathlineto{\pgfqpoint{2.740506in}{1.523229in}}%
\pgfpathlineto{\pgfqpoint{2.740916in}{1.505803in}}%
\pgfpathlineto{\pgfqpoint{2.742966in}{1.250074in}}%
\pgfpathlineto{\pgfqpoint{2.743376in}{1.282377in}}%
\pgfpathlineto{\pgfqpoint{2.746246in}{1.467450in}}%
\pgfpathlineto{\pgfqpoint{2.746656in}{1.474675in}}%
\pgfpathlineto{\pgfqpoint{2.747066in}{1.474333in}}%
\pgfpathlineto{\pgfqpoint{2.747886in}{1.444971in}}%
\pgfpathlineto{\pgfqpoint{2.748296in}{1.412129in}}%
\pgfpathlineto{\pgfqpoint{2.748706in}{1.422781in}}%
\pgfpathlineto{\pgfqpoint{2.749936in}{1.520073in}}%
\pgfpathlineto{\pgfqpoint{2.750346in}{1.508791in}}%
\pgfpathlineto{\pgfqpoint{2.752396in}{1.235231in}}%
\pgfpathlineto{\pgfqpoint{2.752806in}{1.276188in}}%
\pgfpathlineto{\pgfqpoint{2.755676in}{1.464542in}}%
\pgfpathlineto{\pgfqpoint{2.756496in}{1.473547in}}%
\pgfpathlineto{\pgfqpoint{2.757316in}{1.447411in}}%
\pgfpathlineto{\pgfqpoint{2.758136in}{1.416194in}}%
\pgfpathlineto{\pgfqpoint{2.759366in}{1.516412in}}%
\pgfpathlineto{\pgfqpoint{2.759776in}{1.510355in}}%
\pgfpathlineto{\pgfqpoint{2.761826in}{1.231436in}}%
\pgfpathlineto{\pgfqpoint{2.762646in}{1.308395in}}%
\pgfpathlineto{\pgfqpoint{2.765516in}{1.470743in}}%
\pgfpathlineto{\pgfqpoint{2.765926in}{1.472580in}}%
\pgfpathlineto{\pgfqpoint{2.766746in}{1.449239in}}%
\pgfpathlineto{\pgfqpoint{2.767566in}{1.410398in}}%
\pgfpathlineto{\pgfqpoint{2.768796in}{1.512511in}}%
\pgfpathlineto{\pgfqpoint{2.769206in}{1.510825in}}%
\pgfpathlineto{\pgfqpoint{2.771256in}{1.237742in}}%
\pgfpathlineto{\pgfqpoint{2.772076in}{1.303645in}}%
\pgfpathlineto{\pgfqpoint{2.774946in}{1.468785in}}%
\pgfpathlineto{\pgfqpoint{2.775356in}{1.471489in}}%
\pgfpathlineto{\pgfqpoint{2.776176in}{1.450557in}}%
\pgfpathlineto{\pgfqpoint{2.776996in}{1.405325in}}%
\pgfpathlineto{\pgfqpoint{2.778636in}{1.510480in}}%
\pgfpathlineto{\pgfqpoint{2.779866in}{1.353081in}}%
\pgfpathlineto{\pgfqpoint{2.780686in}{1.243586in}}%
\pgfpathlineto{\pgfqpoint{2.781096in}{1.261295in}}%
\pgfpathlineto{\pgfqpoint{2.784376in}{1.466877in}}%
\pgfpathlineto{\pgfqpoint{2.784786in}{1.470319in}}%
\pgfpathlineto{\pgfqpoint{2.785196in}{1.465676in}}%
\pgfpathlineto{\pgfqpoint{2.786426in}{1.400916in}}%
\pgfpathlineto{\pgfqpoint{2.786836in}{1.439291in}}%
\pgfpathlineto{\pgfqpoint{2.788066in}{1.509548in}}%
\pgfpathlineto{\pgfqpoint{2.789296in}{1.359881in}}%
\pgfpathlineto{\pgfqpoint{2.790116in}{1.248927in}}%
\pgfpathlineto{\pgfqpoint{2.790526in}{1.257512in}}%
\pgfpathlineto{\pgfqpoint{2.793806in}{1.465045in}}%
\pgfpathlineto{\pgfqpoint{2.794216in}{1.469106in}}%
\pgfpathlineto{\pgfqpoint{2.794626in}{1.465237in}}%
\pgfpathlineto{\pgfqpoint{2.795856in}{1.397119in}}%
\pgfpathlineto{\pgfqpoint{2.796266in}{1.434950in}}%
\pgfpathlineto{\pgfqpoint{2.797496in}{1.508212in}}%
\pgfpathlineto{\pgfqpoint{2.797906in}{1.483696in}}%
\pgfpathlineto{\pgfqpoint{2.799546in}{1.253735in}}%
\pgfpathlineto{\pgfqpoint{2.800366in}{1.292553in}}%
\pgfpathlineto{\pgfqpoint{2.803236in}{1.463306in}}%
\pgfpathlineto{\pgfqpoint{2.803646in}{1.467878in}}%
\pgfpathlineto{\pgfqpoint{2.804056in}{1.464653in}}%
\pgfpathlineto{\pgfqpoint{2.805286in}{1.393885in}}%
\pgfpathlineto{\pgfqpoint{2.805696in}{1.431208in}}%
\pgfpathlineto{\pgfqpoint{2.806926in}{1.506612in}}%
\pgfpathlineto{\pgfqpoint{2.807336in}{1.484874in}}%
\pgfpathlineto{\pgfqpoint{2.809386in}{1.251614in}}%
\pgfpathlineto{\pgfqpoint{2.809796in}{1.289861in}}%
\pgfpathlineto{\pgfqpoint{2.812666in}{1.461672in}}%
\pgfpathlineto{\pgfqpoint{2.813076in}{1.466656in}}%
\pgfpathlineto{\pgfqpoint{2.813486in}{1.463957in}}%
\pgfpathlineto{\pgfqpoint{2.814306in}{1.429690in}}%
\pgfpathlineto{\pgfqpoint{2.814716in}{1.394713in}}%
\pgfpathlineto{\pgfqpoint{2.815126in}{1.428026in}}%
\pgfpathlineto{\pgfqpoint{2.816356in}{1.504857in}}%
\pgfpathlineto{\pgfqpoint{2.816766in}{1.485404in}}%
\pgfpathlineto{\pgfqpoint{2.818816in}{1.249459in}}%
\pgfpathlineto{\pgfqpoint{2.819226in}{1.287646in}}%
\pgfpathlineto{\pgfqpoint{2.822096in}{1.460154in}}%
\pgfpathlineto{\pgfqpoint{2.822506in}{1.465457in}}%
\pgfpathlineto{\pgfqpoint{2.822916in}{1.463172in}}%
\pgfpathlineto{\pgfqpoint{2.823736in}{1.430141in}}%
\pgfpathlineto{\pgfqpoint{2.824146in}{1.396059in}}%
\pgfpathlineto{\pgfqpoint{2.824556in}{1.425367in}}%
\pgfpathlineto{\pgfqpoint{2.825786in}{1.503026in}}%
\pgfpathlineto{\pgfqpoint{2.826196in}{1.485399in}}%
\pgfpathlineto{\pgfqpoint{2.828246in}{1.247803in}}%
\pgfpathlineto{\pgfqpoint{2.828656in}{1.285890in}}%
\pgfpathlineto{\pgfqpoint{2.831526in}{1.458757in}}%
\pgfpathlineto{\pgfqpoint{2.831936in}{1.464292in}}%
\pgfpathlineto{\pgfqpoint{2.832346in}{1.462320in}}%
\pgfpathlineto{\pgfqpoint{2.833166in}{1.430253in}}%
\pgfpathlineto{\pgfqpoint{2.833576in}{1.396893in}}%
\pgfpathlineto{\pgfqpoint{2.833986in}{1.423198in}}%
\pgfpathlineto{\pgfqpoint{2.835216in}{1.501176in}}%
\pgfpathlineto{\pgfqpoint{2.835626in}{1.484951in}}%
\pgfpathlineto{\pgfqpoint{2.837676in}{1.246625in}}%
\pgfpathlineto{\pgfqpoint{2.838086in}{1.284576in}}%
\pgfpathlineto{\pgfqpoint{2.840956in}{1.457482in}}%
\pgfpathlineto{\pgfqpoint{2.841366in}{1.463169in}}%
\pgfpathlineto{\pgfqpoint{2.841776in}{1.461413in}}%
\pgfpathlineto{\pgfqpoint{2.842596in}{1.430059in}}%
\pgfpathlineto{\pgfqpoint{2.843006in}{1.397260in}}%
\pgfpathlineto{\pgfqpoint{2.843416in}{1.421487in}}%
\pgfpathlineto{\pgfqpoint{2.844646in}{1.499344in}}%
\pgfpathlineto{\pgfqpoint{2.845056in}{1.484128in}}%
\pgfpathlineto{\pgfqpoint{2.847106in}{1.245901in}}%
\pgfpathlineto{\pgfqpoint{2.847516in}{1.283684in}}%
\pgfpathlineto{\pgfqpoint{2.850386in}{1.456330in}}%
\pgfpathlineto{\pgfqpoint{2.850796in}{1.462092in}}%
\pgfpathlineto{\pgfqpoint{2.851206in}{1.460461in}}%
\pgfpathlineto{\pgfqpoint{2.852026in}{1.429584in}}%
\pgfpathlineto{\pgfqpoint{2.852436in}{1.397197in}}%
\pgfpathlineto{\pgfqpoint{2.852846in}{1.420206in}}%
\pgfpathlineto{\pgfqpoint{2.854076in}{1.497553in}}%
\pgfpathlineto{\pgfqpoint{2.854486in}{1.482987in}}%
\pgfpathlineto{\pgfqpoint{2.856536in}{1.245610in}}%
\pgfpathlineto{\pgfqpoint{2.856946in}{1.283197in}}%
\pgfpathlineto{\pgfqpoint{2.859816in}{1.455297in}}%
\pgfpathlineto{\pgfqpoint{2.860226in}{1.461062in}}%
\pgfpathlineto{\pgfqpoint{2.860636in}{1.459472in}}%
\pgfpathlineto{\pgfqpoint{2.861456in}{1.428848in}}%
\pgfpathlineto{\pgfqpoint{2.861866in}{1.396733in}}%
\pgfpathlineto{\pgfqpoint{2.862276in}{1.419329in}}%
\pgfpathlineto{\pgfqpoint{2.863506in}{1.495812in}}%
\pgfpathlineto{\pgfqpoint{2.863916in}{1.481565in}}%
\pgfpathlineto{\pgfqpoint{2.865966in}{1.245730in}}%
\pgfpathlineto{\pgfqpoint{2.866376in}{1.283097in}}%
\pgfpathlineto{\pgfqpoint{2.869246in}{1.454380in}}%
\pgfpathlineto{\pgfqpoint{2.869656in}{1.460080in}}%
\pgfpathlineto{\pgfqpoint{2.870066in}{1.458448in}}%
\pgfpathlineto{\pgfqpoint{2.870886in}{1.427868in}}%
\pgfpathlineto{\pgfqpoint{2.871296in}{1.395894in}}%
\pgfpathlineto{\pgfqpoint{2.871706in}{1.418832in}}%
\pgfpathlineto{\pgfqpoint{2.872936in}{1.494121in}}%
\pgfpathlineto{\pgfqpoint{2.873346in}{1.479889in}}%
\pgfpathlineto{\pgfqpoint{2.875396in}{1.246238in}}%
\pgfpathlineto{\pgfqpoint{2.875806in}{1.283364in}}%
\pgfpathlineto{\pgfqpoint{2.878676in}{1.453572in}}%
\pgfpathlineto{\pgfqpoint{2.879086in}{1.459140in}}%
\pgfpathlineto{\pgfqpoint{2.879496in}{1.457389in}}%
\pgfpathlineto{\pgfqpoint{2.880316in}{1.426652in}}%
\pgfpathlineto{\pgfqpoint{2.880726in}{1.394696in}}%
\pgfpathlineto{\pgfqpoint{2.881136in}{1.418693in}}%
\pgfpathlineto{\pgfqpoint{2.882366in}{1.492473in}}%
\pgfpathlineto{\pgfqpoint{2.882776in}{1.477974in}}%
\pgfpathlineto{\pgfqpoint{2.884826in}{1.247114in}}%
\pgfpathlineto{\pgfqpoint{2.885236in}{1.283981in}}%
\pgfpathlineto{\pgfqpoint{2.888106in}{1.452865in}}%
\pgfpathlineto{\pgfqpoint{2.888516in}{1.458238in}}%
\pgfpathlineto{\pgfqpoint{2.888926in}{1.456293in}}%
\pgfpathlineto{\pgfqpoint{2.889746in}{1.425209in}}%
\pgfpathlineto{\pgfqpoint{2.890156in}{1.393152in}}%
\pgfpathlineto{\pgfqpoint{2.890566in}{1.418891in}}%
\pgfpathlineto{\pgfqpoint{2.891796in}{1.490852in}}%
\pgfpathlineto{\pgfqpoint{2.892206in}{1.475827in}}%
\pgfpathlineto{\pgfqpoint{2.894256in}{1.248338in}}%
\pgfpathlineto{\pgfqpoint{2.894666in}{1.284929in}}%
\pgfpathlineto{\pgfqpoint{2.897536in}{1.452252in}}%
\pgfpathlineto{\pgfqpoint{2.897946in}{1.457367in}}%
\pgfpathlineto{\pgfqpoint{2.898356in}{1.455155in}}%
\pgfpathlineto{\pgfqpoint{2.899176in}{1.423540in}}%
\pgfpathlineto{\pgfqpoint{2.899586in}{1.391274in}}%
\pgfpathlineto{\pgfqpoint{2.899996in}{1.419406in}}%
\pgfpathlineto{\pgfqpoint{2.901226in}{1.489238in}}%
\pgfpathlineto{\pgfqpoint{2.901636in}{1.473445in}}%
\pgfpathlineto{\pgfqpoint{2.903686in}{1.249891in}}%
\pgfpathlineto{\pgfqpoint{2.904096in}{1.286192in}}%
\pgfpathlineto{\pgfqpoint{2.906966in}{1.451721in}}%
\pgfpathlineto{\pgfqpoint{2.907376in}{1.456520in}}%
\pgfpathlineto{\pgfqpoint{2.907786in}{1.453970in}}%
\pgfpathlineto{\pgfqpoint{2.908606in}{1.421646in}}%
\pgfpathlineto{\pgfqpoint{2.909016in}{1.389066in}}%
\pgfpathlineto{\pgfqpoint{2.909426in}{1.420219in}}%
\pgfpathlineto{\pgfqpoint{2.910656in}{1.487605in}}%
\pgfpathlineto{\pgfqpoint{2.911066in}{1.470823in}}%
\pgfpathlineto{\pgfqpoint{2.913116in}{1.251755in}}%
\pgfpathlineto{\pgfqpoint{2.913526in}{1.287751in}}%
\pgfpathlineto{\pgfqpoint{2.916396in}{1.451261in}}%
\pgfpathlineto{\pgfqpoint{2.916806in}{1.455685in}}%
\pgfpathlineto{\pgfqpoint{2.917216in}{1.452729in}}%
\pgfpathlineto{\pgfqpoint{2.918446in}{1.387722in}}%
\pgfpathlineto{\pgfqpoint{2.918856in}{1.421312in}}%
\pgfpathlineto{\pgfqpoint{2.920086in}{1.485923in}}%
\pgfpathlineto{\pgfqpoint{2.920496in}{1.467948in}}%
\pgfpathlineto{\pgfqpoint{2.922546in}{1.253911in}}%
\pgfpathlineto{\pgfqpoint{2.922956in}{1.289590in}}%
\pgfpathlineto{\pgfqpoint{2.925826in}{1.450862in}}%
\pgfpathlineto{\pgfqpoint{2.926236in}{1.454852in}}%
\pgfpathlineto{\pgfqpoint{2.926646in}{1.451423in}}%
\pgfpathlineto{\pgfqpoint{2.927876in}{1.389302in}}%
\pgfpathlineto{\pgfqpoint{2.928286in}{1.422667in}}%
\pgfpathlineto{\pgfqpoint{2.929516in}{1.484160in}}%
\pgfpathlineto{\pgfqpoint{2.929926in}{1.464805in}}%
\pgfpathlineto{\pgfqpoint{2.931976in}{1.256344in}}%
\pgfpathlineto{\pgfqpoint{2.932386in}{1.291694in}}%
\pgfpathlineto{\pgfqpoint{2.935256in}{1.450510in}}%
\pgfpathlineto{\pgfqpoint{2.935666in}{1.454010in}}%
\pgfpathlineto{\pgfqpoint{2.936076in}{1.450040in}}%
\pgfpathlineto{\pgfqpoint{2.937306in}{1.391136in}}%
\pgfpathlineto{\pgfqpoint{2.937716in}{1.424266in}}%
\pgfpathlineto{\pgfqpoint{2.938946in}{1.482278in}}%
\pgfpathlineto{\pgfqpoint{2.940176in}{1.363317in}}%
\pgfpathlineto{\pgfqpoint{2.941406in}{1.259036in}}%
\pgfpathlineto{\pgfqpoint{2.941816in}{1.294045in}}%
\pgfpathlineto{\pgfqpoint{2.944686in}{1.450191in}}%
\pgfpathlineto{\pgfqpoint{2.945096in}{1.453144in}}%
\pgfpathlineto{\pgfqpoint{2.945506in}{1.448569in}}%
\pgfpathlineto{\pgfqpoint{2.946736in}{1.393213in}}%
\pgfpathlineto{\pgfqpoint{2.947146in}{1.426089in}}%
\pgfpathlineto{\pgfqpoint{2.948376in}{1.480242in}}%
\pgfpathlineto{\pgfqpoint{2.949606in}{1.359056in}}%
\pgfpathlineto{\pgfqpoint{2.950836in}{1.261973in}}%
\pgfpathlineto{\pgfqpoint{2.952066in}{1.358199in}}%
\pgfpathlineto{\pgfqpoint{2.952066in}{1.358199in}}%
\pgfusepath{stroke}%
\end{pgfscope}%
\begin{pgfscope}%
\pgfpathrectangle{\pgfqpoint{0.800000in}{0.528000in}}{\pgfqpoint{2.254545in}{1.680000in}}%
\pgfusepath{clip}%
\pgfsetrectcap%
\pgfsetroundjoin%
\pgfsetlinewidth{1.505625pt}%
\definecolor{currentstroke}{rgb}{0.890196,0.466667,0.760784}%
\pgfsetstrokecolor{currentstroke}%
\pgfsetdash{}{0pt}%
\pgfpathmoveto{\pgfqpoint{0.902479in}{1.531273in}}%
\pgfpathlineto{\pgfqpoint{0.913139in}{1.530186in}}%
\pgfpathlineto{\pgfqpoint{0.922979in}{1.526993in}}%
\pgfpathlineto{\pgfqpoint{0.931999in}{1.521792in}}%
\pgfpathlineto{\pgfqpoint{0.940609in}{1.514224in}}%
\pgfpathlineto{\pgfqpoint{0.948809in}{1.503985in}}%
\pgfpathlineto{\pgfqpoint{0.957009in}{1.490009in}}%
\pgfpathlineto{\pgfqpoint{0.965209in}{1.471404in}}%
\pgfpathlineto{\pgfqpoint{0.973409in}{1.447188in}}%
\pgfpathlineto{\pgfqpoint{0.982019in}{1.414606in}}%
\pgfpathlineto{\pgfqpoint{0.991039in}{1.371403in}}%
\pgfpathlineto{\pgfqpoint{1.000469in}{1.315170in}}%
\pgfpathlineto{\pgfqpoint{1.004569in}{1.286991in}}%
\pgfpathlineto{\pgfqpoint{1.004979in}{1.290306in}}%
\pgfpathlineto{\pgfqpoint{1.021789in}{1.502939in}}%
\pgfpathlineto{\pgfqpoint{1.075732in}{2.218000in}}%
\pgfpathmoveto{\pgfqpoint{1.180331in}{2.218000in}}%
\pgfpathlineto{\pgfqpoint{1.188659in}{2.079510in}}%
\pgfpathlineto{\pgfqpoint{1.198089in}{1.883406in}}%
\pgfpathlineto{\pgfqpoint{1.207929in}{1.630015in}}%
\pgfpathlineto{\pgfqpoint{1.218999in}{1.284181in}}%
\pgfpathlineto{\pgfqpoint{1.226789in}{1.006484in}}%
\pgfpathlineto{\pgfqpoint{1.228019in}{1.012961in}}%
\pgfpathlineto{\pgfqpoint{1.257129in}{1.194020in}}%
\pgfpathlineto{\pgfqpoint{1.266559in}{1.237907in}}%
\pgfpathlineto{\pgfqpoint{1.273939in}{1.262861in}}%
\pgfpathlineto{\pgfqpoint{1.279679in}{1.275539in}}%
\pgfpathlineto{\pgfqpoint{1.284189in}{1.280865in}}%
\pgfpathlineto{\pgfqpoint{1.287469in}{1.281931in}}%
\pgfpathlineto{\pgfqpoint{1.290339in}{1.280762in}}%
\pgfpathlineto{\pgfqpoint{1.293619in}{1.276828in}}%
\pgfpathlineto{\pgfqpoint{1.297309in}{1.268767in}}%
\pgfpathlineto{\pgfqpoint{1.301819in}{1.253011in}}%
\pgfpathlineto{\pgfqpoint{1.306739in}{1.227142in}}%
\pgfpathlineto{\pgfqpoint{1.312069in}{1.186600in}}%
\pgfpathlineto{\pgfqpoint{1.317809in}{1.124349in}}%
\pgfpathlineto{\pgfqpoint{1.323549in}{1.036731in}}%
\pgfpathlineto{\pgfqpoint{1.327239in}{0.970761in}}%
\pgfpathlineto{\pgfqpoint{1.327649in}{0.976957in}}%
\pgfpathlineto{\pgfqpoint{1.334619in}{1.100459in}}%
\pgfpathlineto{\pgfqpoint{1.343229in}{1.299483in}}%
\pgfpathlineto{\pgfqpoint{1.352659in}{1.507446in}}%
\pgfpathlineto{\pgfqpoint{1.356349in}{1.546254in}}%
\pgfpathlineto{\pgfqpoint{1.357989in}{1.550292in}}%
\pgfpathlineto{\pgfqpoint{1.358809in}{1.548869in}}%
\pgfpathlineto{\pgfqpoint{1.360449in}{1.538791in}}%
\pgfpathlineto{\pgfqpoint{1.362909in}{1.505091in}}%
\pgfpathlineto{\pgfqpoint{1.366189in}{1.426144in}}%
\pgfpathlineto{\pgfqpoint{1.370289in}{1.286932in}}%
\pgfpathlineto{\pgfqpoint{1.370699in}{1.314098in}}%
\pgfpathlineto{\pgfqpoint{1.381769in}{1.973445in}}%
\pgfpathlineto{\pgfqpoint{1.387490in}{2.218000in}}%
\pgfpathmoveto{\pgfqpoint{1.426062in}{2.218000in}}%
\pgfpathlineto{\pgfqpoint{1.432199in}{1.989835in}}%
\pgfpathlineto{\pgfqpoint{1.439168in}{1.643382in}}%
\pgfpathlineto{\pgfqpoint{1.447778in}{1.094758in}}%
\pgfpathlineto{\pgfqpoint{1.449418in}{1.006635in}}%
\pgfpathlineto{\pgfqpoint{1.449828in}{1.011687in}}%
\pgfpathlineto{\pgfqpoint{1.464998in}{1.186089in}}%
\pgfpathlineto{\pgfqpoint{1.471968in}{1.241092in}}%
\pgfpathlineto{\pgfqpoint{1.476888in}{1.264630in}}%
\pgfpathlineto{\pgfqpoint{1.480168in}{1.272289in}}%
\pgfpathlineto{\pgfqpoint{1.482218in}{1.273517in}}%
\pgfpathlineto{\pgfqpoint{1.483858in}{1.272388in}}%
\pgfpathlineto{\pgfqpoint{1.486318in}{1.266918in}}%
\pgfpathlineto{\pgfqpoint{1.489188in}{1.254241in}}%
\pgfpathlineto{\pgfqpoint{1.492878in}{1.226408in}}%
\pgfpathlineto{\pgfqpoint{1.496978in}{1.176749in}}%
\pgfpathlineto{\pgfqpoint{1.501488in}{1.092051in}}%
\pgfpathlineto{\pgfqpoint{1.505998in}{0.964307in}}%
\pgfpathlineto{\pgfqpoint{1.506818in}{0.983538in}}%
\pgfpathlineto{\pgfqpoint{1.512558in}{1.152875in}}%
\pgfpathlineto{\pgfqpoint{1.524038in}{1.514886in}}%
\pgfpathlineto{\pgfqpoint{1.525678in}{1.524726in}}%
\pgfpathlineto{\pgfqpoint{1.526498in}{1.522019in}}%
\pgfpathlineto{\pgfqpoint{1.528138in}{1.500364in}}%
\pgfpathlineto{\pgfqpoint{1.530598in}{1.427211in}}%
\pgfpathlineto{\pgfqpoint{1.533878in}{1.269326in}}%
\pgfpathlineto{\pgfqpoint{1.534288in}{1.309307in}}%
\pgfpathlineto{\pgfqpoint{1.542488in}{2.004279in}}%
\pgfpathlineto{\pgfqpoint{1.546215in}{2.218000in}}%
\pgfpathmoveto{\pgfqpoint{1.570645in}{2.218000in}}%
\pgfpathlineto{\pgfqpoint{1.575288in}{1.990392in}}%
\pgfpathlineto{\pgfqpoint{1.581438in}{1.569144in}}%
\pgfpathlineto{\pgfqpoint{1.587998in}{0.997653in}}%
\pgfpathlineto{\pgfqpoint{1.589228in}{1.018150in}}%
\pgfpathlineto{\pgfqpoint{1.600298in}{1.184105in}}%
\pgfpathlineto{\pgfqpoint{1.605628in}{1.234191in}}%
\pgfpathlineto{\pgfqpoint{1.609318in}{1.251957in}}%
\pgfpathlineto{\pgfqpoint{1.611368in}{1.255051in}}%
\pgfpathlineto{\pgfqpoint{1.612598in}{1.254382in}}%
\pgfpathlineto{\pgfqpoint{1.614238in}{1.250333in}}%
\pgfpathlineto{\pgfqpoint{1.616698in}{1.236857in}}%
\pgfpathlineto{\pgfqpoint{1.619978in}{1.202824in}}%
\pgfpathlineto{\pgfqpoint{1.623668in}{1.136774in}}%
\pgfpathlineto{\pgfqpoint{1.627768in}{1.015943in}}%
\pgfpathlineto{\pgfqpoint{1.629408in}{0.960823in}}%
\pgfpathlineto{\pgfqpoint{1.629818in}{0.973448in}}%
\pgfpathlineto{\pgfqpoint{1.635148in}{1.176420in}}%
\pgfpathlineto{\pgfqpoint{1.642528in}{1.463733in}}%
\pgfpathlineto{\pgfqpoint{1.644168in}{1.480181in}}%
\pgfpathlineto{\pgfqpoint{1.644578in}{1.479475in}}%
\pgfpathlineto{\pgfqpoint{1.645808in}{1.464875in}}%
\pgfpathlineto{\pgfqpoint{1.647858in}{1.398431in}}%
\pgfpathlineto{\pgfqpoint{1.650728in}{1.254379in}}%
\pgfpathlineto{\pgfqpoint{1.657698in}{1.979916in}}%
\pgfpathlineto{\pgfqpoint{1.661381in}{2.218000in}}%
\pgfpathmoveto{\pgfqpoint{1.677429in}{2.218000in}}%
\pgfpathlineto{\pgfqpoint{1.681478in}{1.998483in}}%
\pgfpathlineto{\pgfqpoint{1.686808in}{1.572247in}}%
\pgfpathlineto{\pgfqpoint{1.692548in}{0.990143in}}%
\pgfpathlineto{\pgfqpoint{1.693368in}{1.006609in}}%
\pgfpathlineto{\pgfqpoint{1.702388in}{1.166683in}}%
\pgfpathlineto{\pgfqpoint{1.706898in}{1.213638in}}%
\pgfpathlineto{\pgfqpoint{1.709768in}{1.226533in}}%
\pgfpathlineto{\pgfqpoint{1.710998in}{1.227457in}}%
\pgfpathlineto{\pgfqpoint{1.711408in}{1.227106in}}%
\pgfpathlineto{\pgfqpoint{1.713048in}{1.222209in}}%
\pgfpathlineto{\pgfqpoint{1.715098in}{1.207530in}}%
\pgfpathlineto{\pgfqpoint{1.717968in}{1.168226in}}%
\pgfpathlineto{\pgfqpoint{1.721248in}{1.089157in}}%
\pgfpathlineto{\pgfqpoint{1.724938in}{0.947480in}}%
\pgfpathlineto{\pgfqpoint{1.725758in}{0.977338in}}%
\pgfpathlineto{\pgfqpoint{1.731498in}{1.240214in}}%
\pgfpathlineto{\pgfqpoint{1.736008in}{1.410705in}}%
\pgfpathlineto{\pgfqpoint{1.737238in}{1.419970in}}%
\pgfpathlineto{\pgfqpoint{1.737648in}{1.417920in}}%
\pgfpathlineto{\pgfqpoint{1.738878in}{1.395360in}}%
\pgfpathlineto{\pgfqpoint{1.740928in}{1.304033in}}%
\pgfpathlineto{\pgfqpoint{1.742568in}{1.217851in}}%
\pgfpathlineto{\pgfqpoint{1.748718in}{1.936765in}}%
\pgfpathlineto{\pgfqpoint{1.752818in}{2.199552in}}%
\pgfpathlineto{\pgfqpoint{1.753281in}{2.218000in}}%
\pgfpathmoveto{\pgfqpoint{1.762048in}{2.218000in}}%
\pgfpathlineto{\pgfqpoint{1.765118in}{2.077854in}}%
\pgfpathlineto{\pgfqpoint{1.769218in}{1.779703in}}%
\pgfpathlineto{\pgfqpoint{1.774548in}{1.211596in}}%
\pgfpathlineto{\pgfqpoint{1.776598in}{0.975343in}}%
\pgfpathlineto{\pgfqpoint{1.777418in}{0.993923in}}%
\pgfpathlineto{\pgfqpoint{1.785208in}{1.145177in}}%
\pgfpathlineto{\pgfqpoint{1.788898in}{1.183501in}}%
\pgfpathlineto{\pgfqpoint{1.791358in}{1.191534in}}%
\pgfpathlineto{\pgfqpoint{1.792178in}{1.190644in}}%
\pgfpathlineto{\pgfqpoint{1.793818in}{1.182924in}}%
\pgfpathlineto{\pgfqpoint{1.796278in}{1.154537in}}%
\pgfpathlineto{\pgfqpoint{1.799148in}{1.090046in}}%
\pgfpathlineto{\pgfqpoint{1.802428in}{0.961008in}}%
\pgfpathlineto{\pgfqpoint{1.803248in}{0.937179in}}%
\pgfpathlineto{\pgfqpoint{1.803658in}{0.953459in}}%
\pgfpathlineto{\pgfqpoint{1.813498in}{1.347945in}}%
\pgfpathlineto{\pgfqpoint{1.813908in}{1.346541in}}%
\pgfpathlineto{\pgfqpoint{1.815138in}{1.323429in}}%
\pgfpathlineto{\pgfqpoint{1.817188in}{1.223416in}}%
\pgfpathlineto{\pgfqpoint{1.818008in}{1.165191in}}%
\pgfpathlineto{\pgfqpoint{1.818418in}{1.193674in}}%
\pgfpathlineto{\pgfqpoint{1.824158in}{1.892662in}}%
\pgfpathlineto{\pgfqpoint{1.827848in}{2.114891in}}%
\pgfpathlineto{\pgfqpoint{1.830308in}{2.166893in}}%
\pgfpathlineto{\pgfqpoint{1.831128in}{2.168888in}}%
\pgfpathlineto{\pgfqpoint{1.831948in}{2.163636in}}%
\pgfpathlineto{\pgfqpoint{1.833588in}{2.132114in}}%
\pgfpathlineto{\pgfqpoint{1.836048in}{2.033971in}}%
\pgfpathlineto{\pgfqpoint{1.839738in}{1.774984in}}%
\pgfpathlineto{\pgfqpoint{1.844658in}{1.234968in}}%
\pgfpathlineto{\pgfqpoint{1.846708in}{0.959926in}}%
\pgfpathlineto{\pgfqpoint{1.847528in}{0.973202in}}%
\pgfpathlineto{\pgfqpoint{1.854088in}{1.108560in}}%
\pgfpathlineto{\pgfqpoint{1.857368in}{1.143118in}}%
\pgfpathlineto{\pgfqpoint{1.859008in}{1.148245in}}%
\pgfpathlineto{\pgfqpoint{1.859418in}{1.148085in}}%
\pgfpathlineto{\pgfqpoint{1.860648in}{1.143852in}}%
\pgfpathlineto{\pgfqpoint{1.862698in}{1.122972in}}%
\pgfpathlineto{\pgfqpoint{1.865158in}{1.070778in}}%
\pgfpathlineto{\pgfqpoint{1.868438in}{0.941500in}}%
\pgfpathlineto{\pgfqpoint{1.868848in}{0.919541in}}%
\pgfpathlineto{\pgfqpoint{1.869258in}{0.919680in}}%
\pgfpathlineto{\pgfqpoint{1.878278in}{1.268140in}}%
\pgfpathlineto{\pgfqpoint{1.878688in}{1.265930in}}%
\pgfpathlineto{\pgfqpoint{1.879918in}{1.239304in}}%
\pgfpathlineto{\pgfqpoint{1.881968in}{1.131485in}}%
\pgfpathlineto{\pgfqpoint{1.882378in}{1.102161in}}%
\pgfpathlineto{\pgfqpoint{1.888118in}{1.811546in}}%
\pgfpathlineto{\pgfqpoint{1.891398in}{1.992628in}}%
\pgfpathlineto{\pgfqpoint{1.893448in}{2.022866in}}%
\pgfpathlineto{\pgfqpoint{1.894268in}{2.018548in}}%
\pgfpathlineto{\pgfqpoint{1.895908in}{1.983267in}}%
\pgfpathlineto{\pgfqpoint{1.898368in}{1.866531in}}%
\pgfpathlineto{\pgfqpoint{1.902058in}{1.555734in}}%
\pgfpathlineto{\pgfqpoint{1.906978in}{0.933428in}}%
\pgfpathlineto{\pgfqpoint{1.908208in}{0.963789in}}%
\pgfpathlineto{\pgfqpoint{1.913948in}{1.076808in}}%
\pgfpathlineto{\pgfqpoint{1.916818in}{1.098828in}}%
\pgfpathlineto{\pgfqpoint{1.917638in}{1.099214in}}%
\pgfpathlineto{\pgfqpoint{1.918868in}{1.094126in}}%
\pgfpathlineto{\pgfqpoint{1.920918in}{1.068577in}}%
\pgfpathlineto{\pgfqpoint{1.923378in}{1.004226in}}%
\pgfpathlineto{\pgfqpoint{1.925838in}{0.894437in}}%
\pgfpathlineto{\pgfqpoint{1.926658in}{0.916993in}}%
\pgfpathlineto{\pgfqpoint{1.933628in}{1.183141in}}%
\pgfpathlineto{\pgfqpoint{1.934038in}{1.185483in}}%
\pgfpathlineto{\pgfqpoint{1.934448in}{1.184786in}}%
\pgfpathlineto{\pgfqpoint{1.935678in}{1.163342in}}%
\pgfpathlineto{\pgfqpoint{1.938138in}{1.062074in}}%
\pgfpathlineto{\pgfqpoint{1.943058in}{1.668677in}}%
\pgfpathlineto{\pgfqpoint{1.946338in}{1.845430in}}%
\pgfpathlineto{\pgfqpoint{1.947978in}{1.863444in}}%
\pgfpathlineto{\pgfqpoint{1.948798in}{1.855980in}}%
\pgfpathlineto{\pgfqpoint{1.950438in}{1.809693in}}%
\pgfpathlineto{\pgfqpoint{1.952898in}{1.666023in}}%
\pgfpathlineto{\pgfqpoint{1.956588in}{1.300953in}}%
\pgfpathlineto{\pgfqpoint{1.959868in}{0.916445in}}%
\pgfpathlineto{\pgfqpoint{1.960278in}{0.926732in}}%
\pgfpathlineto{\pgfqpoint{1.965198in}{1.025245in}}%
\pgfpathlineto{\pgfqpoint{1.967658in}{1.046163in}}%
\pgfpathlineto{\pgfqpoint{1.968478in}{1.047231in}}%
\pgfpathlineto{\pgfqpoint{1.968888in}{1.046514in}}%
\pgfpathlineto{\pgfqpoint{1.970118in}{1.038959in}}%
\pgfpathlineto{\pgfqpoint{1.972168in}{1.006173in}}%
\pgfpathlineto{\pgfqpoint{1.974628in}{0.927415in}}%
\pgfpathlineto{\pgfqpoint{1.975858in}{0.869582in}}%
\pgfpathlineto{\pgfqpoint{1.976268in}{0.875384in}}%
\pgfpathlineto{\pgfqpoint{1.982828in}{1.102272in}}%
\pgfpathlineto{\pgfqpoint{1.983238in}{1.104601in}}%
\pgfpathlineto{\pgfqpoint{1.983648in}{1.104161in}}%
\pgfpathlineto{\pgfqpoint{1.984878in}{1.085130in}}%
\pgfpathlineto{\pgfqpoint{1.986928in}{0.996290in}}%
\pgfpathlineto{\pgfqpoint{1.987338in}{1.030437in}}%
\pgfpathlineto{\pgfqpoint{1.991848in}{1.548532in}}%
\pgfpathlineto{\pgfqpoint{1.994718in}{1.687387in}}%
\pgfpathlineto{\pgfqpoint{1.995948in}{1.698941in}}%
\pgfpathlineto{\pgfqpoint{1.996358in}{1.696651in}}%
\pgfpathlineto{\pgfqpoint{1.997588in}{1.671990in}}%
\pgfpathlineto{\pgfqpoint{1.999638in}{1.574466in}}%
\pgfpathlineto{\pgfqpoint{2.002918in}{1.288012in}}%
\pgfpathlineto{\pgfqpoint{2.006198in}{0.883528in}}%
\pgfpathlineto{\pgfqpoint{2.007428in}{0.911530in}}%
\pgfpathlineto{\pgfqpoint{2.011528in}{0.982778in}}%
\pgfpathlineto{\pgfqpoint{2.013578in}{0.994435in}}%
\pgfpathlineto{\pgfqpoint{2.014398in}{0.993121in}}%
\pgfpathlineto{\pgfqpoint{2.015628in}{0.983799in}}%
\pgfpathlineto{\pgfqpoint{2.017678in}{0.946129in}}%
\pgfpathlineto{\pgfqpoint{2.020548in}{0.839988in}}%
\pgfpathlineto{\pgfqpoint{2.021368in}{0.867690in}}%
\pgfpathlineto{\pgfqpoint{2.026288in}{1.021719in}}%
\pgfpathlineto{\pgfqpoint{2.027518in}{1.029216in}}%
\pgfpathlineto{\pgfqpoint{2.028338in}{1.021917in}}%
\pgfpathlineto{\pgfqpoint{2.029978in}{0.976232in}}%
\pgfpathlineto{\pgfqpoint{2.030798in}{0.939348in}}%
\pgfpathlineto{\pgfqpoint{2.037768in}{1.525022in}}%
\pgfpathlineto{\pgfqpoint{2.038998in}{1.536870in}}%
\pgfpathlineto{\pgfqpoint{2.039408in}{1.534345in}}%
\pgfpathlineto{\pgfqpoint{2.040638in}{1.508004in}}%
\pgfpathlineto{\pgfqpoint{2.042688in}{1.405210in}}%
\pgfpathlineto{\pgfqpoint{2.045968in}{1.111198in}}%
\pgfpathlineto{\pgfqpoint{2.048428in}{0.861266in}}%
\pgfpathlineto{\pgfqpoint{2.048838in}{0.870495in}}%
\pgfpathlineto{\pgfqpoint{2.052938in}{0.936273in}}%
\pgfpathlineto{\pgfqpoint{2.054578in}{0.943454in}}%
\pgfpathlineto{\pgfqpoint{2.055398in}{0.941640in}}%
\pgfpathlineto{\pgfqpoint{2.057038in}{0.925658in}}%
\pgfpathlineto{\pgfqpoint{2.059498in}{0.866933in}}%
\pgfpathlineto{\pgfqpoint{2.060728in}{0.820908in}}%
\pgfpathlineto{\pgfqpoint{2.061138in}{0.821123in}}%
\pgfpathlineto{\pgfqpoint{2.066058in}{0.955523in}}%
\pgfpathlineto{\pgfqpoint{2.067288in}{0.961917in}}%
\pgfpathlineto{\pgfqpoint{2.068108in}{0.955778in}}%
\pgfpathlineto{\pgfqpoint{2.069748in}{0.917313in}}%
\pgfpathlineto{\pgfqpoint{2.070568in}{0.886574in}}%
\pgfpathlineto{\pgfqpoint{2.075077in}{1.305375in}}%
\pgfpathlineto{\pgfqpoint{2.077537in}{1.383085in}}%
\pgfpathlineto{\pgfqpoint{2.077947in}{1.384536in}}%
\pgfpathlineto{\pgfqpoint{2.078357in}{1.382758in}}%
\pgfpathlineto{\pgfqpoint{2.079587in}{1.358552in}}%
\pgfpathlineto{\pgfqpoint{2.081637in}{1.259521in}}%
\pgfpathlineto{\pgfqpoint{2.085327in}{0.935429in}}%
\pgfpathlineto{\pgfqpoint{2.086557in}{0.832619in}}%
\pgfpathlineto{\pgfqpoint{2.086967in}{0.840981in}}%
\pgfpathlineto{\pgfqpoint{2.090657in}{0.892033in}}%
\pgfpathlineto{\pgfqpoint{2.091887in}{0.896204in}}%
\pgfpathlineto{\pgfqpoint{2.092297in}{0.895817in}}%
\pgfpathlineto{\pgfqpoint{2.093527in}{0.888812in}}%
\pgfpathlineto{\pgfqpoint{2.095577in}{0.855915in}}%
\pgfpathlineto{\pgfqpoint{2.097627in}{0.795170in}}%
\pgfpathlineto{\pgfqpoint{2.098447in}{0.809683in}}%
\pgfpathlineto{\pgfqpoint{2.102547in}{0.899412in}}%
\pgfpathlineto{\pgfqpoint{2.103367in}{0.903841in}}%
\pgfpathlineto{\pgfqpoint{2.103777in}{0.903548in}}%
\pgfpathlineto{\pgfqpoint{2.105007in}{0.891896in}}%
\pgfpathlineto{\pgfqpoint{2.106647in}{0.851858in}}%
\pgfpathlineto{\pgfqpoint{2.107057in}{0.858775in}}%
\pgfpathlineto{\pgfqpoint{2.111157in}{1.185490in}}%
\pgfpathlineto{\pgfqpoint{2.113617in}{1.246699in}}%
\pgfpathlineto{\pgfqpoint{2.114027in}{1.245985in}}%
\pgfpathlineto{\pgfqpoint{2.115257in}{1.225689in}}%
\pgfpathlineto{\pgfqpoint{2.117307in}{1.135807in}}%
\pgfpathlineto{\pgfqpoint{2.120997in}{0.841550in}}%
\pgfpathlineto{\pgfqpoint{2.121407in}{0.802587in}}%
\pgfpathlineto{\pgfqpoint{2.122227in}{0.816491in}}%
\pgfpathlineto{\pgfqpoint{2.125507in}{0.852353in}}%
\pgfpathlineto{\pgfqpoint{2.126327in}{0.854036in}}%
\pgfpathlineto{\pgfqpoint{2.126737in}{0.853585in}}%
\pgfpathlineto{\pgfqpoint{2.127967in}{0.846697in}}%
\pgfpathlineto{\pgfqpoint{2.130017in}{0.815606in}}%
\pgfpathlineto{\pgfqpoint{2.131657in}{0.773143in}}%
\pgfpathlineto{\pgfqpoint{2.132067in}{0.775258in}}%
\pgfpathlineto{\pgfqpoint{2.136167in}{0.851409in}}%
\pgfpathlineto{\pgfqpoint{2.136987in}{0.855010in}}%
\pgfpathlineto{\pgfqpoint{2.137397in}{0.854746in}}%
\pgfpathlineto{\pgfqpoint{2.138627in}{0.845196in}}%
\pgfpathlineto{\pgfqpoint{2.140267in}{0.812723in}}%
\pgfpathlineto{\pgfqpoint{2.140677in}{0.825360in}}%
\pgfpathlineto{\pgfqpoint{2.144777in}{1.090572in}}%
\pgfpathlineto{\pgfqpoint{2.146827in}{1.126804in}}%
\pgfpathlineto{\pgfqpoint{2.147647in}{1.121354in}}%
\pgfpathlineto{\pgfqpoint{2.149287in}{1.078079in}}%
\pgfpathlineto{\pgfqpoint{2.152157in}{0.915673in}}%
\pgfpathlineto{\pgfqpoint{2.154207in}{0.782808in}}%
\pgfpathlineto{\pgfqpoint{2.154617in}{0.788991in}}%
\pgfpathlineto{\pgfqpoint{2.157487in}{0.816088in}}%
\pgfpathlineto{\pgfqpoint{2.158307in}{0.817593in}}%
\pgfpathlineto{\pgfqpoint{2.158717in}{0.817152in}}%
\pgfpathlineto{\pgfqpoint{2.159947in}{0.810781in}}%
\pgfpathlineto{\pgfqpoint{2.161997in}{0.782779in}}%
\pgfpathlineto{\pgfqpoint{2.163637in}{0.754023in}}%
\pgfpathlineto{\pgfqpoint{2.164047in}{0.762602in}}%
\pgfpathlineto{\pgfqpoint{2.167327in}{0.811175in}}%
\pgfpathlineto{\pgfqpoint{2.168557in}{0.814923in}}%
\pgfpathlineto{\pgfqpoint{2.169377in}{0.811701in}}%
\pgfpathlineto{\pgfqpoint{2.171017in}{0.791373in}}%
\pgfpathlineto{\pgfqpoint{2.171427in}{0.783648in}}%
\pgfpathlineto{\pgfqpoint{2.171837in}{0.787470in}}%
\pgfpathlineto{\pgfqpoint{2.175937in}{1.002533in}}%
\pgfpathlineto{\pgfqpoint{2.177577in}{1.025437in}}%
\pgfpathlineto{\pgfqpoint{2.177987in}{1.024831in}}%
\pgfpathlineto{\pgfqpoint{2.179217in}{1.008239in}}%
\pgfpathlineto{\pgfqpoint{2.181267in}{0.935963in}}%
\pgfpathlineto{\pgfqpoint{2.184547in}{0.761564in}}%
\pgfpathlineto{\pgfqpoint{2.185367in}{0.771292in}}%
\pgfpathlineto{\pgfqpoint{2.187827in}{0.786725in}}%
\pgfpathlineto{\pgfqpoint{2.188647in}{0.786533in}}%
\pgfpathlineto{\pgfqpoint{2.189877in}{0.780702in}}%
\pgfpathlineto{\pgfqpoint{2.191927in}{0.756114in}}%
\pgfpathlineto{\pgfqpoint{2.193157in}{0.736201in}}%
\pgfpathlineto{\pgfqpoint{2.193567in}{0.743237in}}%
\pgfpathlineto{\pgfqpoint{2.196847in}{0.780841in}}%
\pgfpathlineto{\pgfqpoint{2.197667in}{0.782800in}}%
\pgfpathlineto{\pgfqpoint{2.198077in}{0.782406in}}%
\pgfpathlineto{\pgfqpoint{2.199307in}{0.775598in}}%
\pgfpathlineto{\pgfqpoint{2.200947in}{0.754978in}}%
\pgfpathlineto{\pgfqpoint{2.204637in}{0.918243in}}%
\pgfpathlineto{\pgfqpoint{2.206687in}{0.942334in}}%
\pgfpathlineto{\pgfqpoint{2.207507in}{0.936830in}}%
\pgfpathlineto{\pgfqpoint{2.209147in}{0.901590in}}%
\pgfpathlineto{\pgfqpoint{2.212017in}{0.779740in}}%
\pgfpathlineto{\pgfqpoint{2.212837in}{0.741806in}}%
\pgfpathlineto{\pgfqpoint{2.213657in}{0.749949in}}%
\pgfpathlineto{\pgfqpoint{2.216117in}{0.761920in}}%
\pgfpathlineto{\pgfqpoint{2.216937in}{0.761158in}}%
\pgfpathlineto{\pgfqpoint{2.218167in}{0.755236in}}%
\pgfpathlineto{\pgfqpoint{2.221037in}{0.723892in}}%
\pgfpathlineto{\pgfqpoint{2.221857in}{0.734655in}}%
\pgfpathlineto{\pgfqpoint{2.224727in}{0.756796in}}%
\pgfpathlineto{\pgfqpoint{2.225547in}{0.757233in}}%
\pgfpathlineto{\pgfqpoint{2.226777in}{0.752336in}}%
\pgfpathlineto{\pgfqpoint{2.228417in}{0.736253in}}%
\pgfpathlineto{\pgfqpoint{2.232107in}{0.861428in}}%
\pgfpathlineto{\pgfqpoint{2.233747in}{0.875958in}}%
\pgfpathlineto{\pgfqpoint{2.234567in}{0.872355in}}%
\pgfpathlineto{\pgfqpoint{2.236207in}{0.844803in}}%
\pgfpathlineto{\pgfqpoint{2.239077in}{0.747110in}}%
\pgfpathlineto{\pgfqpoint{2.239897in}{0.729680in}}%
\pgfpathlineto{\pgfqpoint{2.240307in}{0.732820in}}%
\pgfpathlineto{\pgfqpoint{2.242767in}{0.741997in}}%
\pgfpathlineto{\pgfqpoint{2.243587in}{0.740977in}}%
\pgfpathlineto{\pgfqpoint{2.245227in}{0.732620in}}%
\pgfpathlineto{\pgfqpoint{2.247277in}{0.713278in}}%
\pgfpathlineto{\pgfqpoint{2.247687in}{0.717614in}}%
\pgfpathlineto{\pgfqpoint{2.250557in}{0.736861in}}%
\pgfpathlineto{\pgfqpoint{2.251377in}{0.737747in}}%
\pgfpathlineto{\pgfqpoint{2.251787in}{0.737318in}}%
\pgfpathlineto{\pgfqpoint{2.253017in}{0.732575in}}%
\pgfpathlineto{\pgfqpoint{2.254247in}{0.723106in}}%
\pgfpathlineto{\pgfqpoint{2.259577in}{0.823900in}}%
\pgfpathlineto{\pgfqpoint{2.260397in}{0.819712in}}%
\pgfpathlineto{\pgfqpoint{2.262037in}{0.795153in}}%
\pgfpathlineto{\pgfqpoint{2.264907in}{0.715771in}}%
\pgfpathlineto{\pgfqpoint{2.266137in}{0.722681in}}%
\pgfpathlineto{\pgfqpoint{2.267777in}{0.726421in}}%
\pgfpathlineto{\pgfqpoint{2.268187in}{0.726291in}}%
\pgfpathlineto{\pgfqpoint{2.269417in}{0.723306in}}%
\pgfpathlineto{\pgfqpoint{2.271467in}{0.710394in}}%
\pgfpathlineto{\pgfqpoint{2.272287in}{0.706652in}}%
\pgfpathlineto{\pgfqpoint{2.275157in}{0.722047in}}%
\pgfpathlineto{\pgfqpoint{2.275977in}{0.722821in}}%
\pgfpathlineto{\pgfqpoint{2.276387in}{0.722529in}}%
\pgfpathlineto{\pgfqpoint{2.277617in}{0.718986in}}%
\pgfpathlineto{\pgfqpoint{2.278847in}{0.711835in}}%
\pgfpathlineto{\pgfqpoint{2.282947in}{0.782485in}}%
\pgfpathlineto{\pgfqpoint{2.283767in}{0.784355in}}%
\pgfpathlineto{\pgfqpoint{2.284177in}{0.783526in}}%
\pgfpathlineto{\pgfqpoint{2.285407in}{0.774292in}}%
\pgfpathlineto{\pgfqpoint{2.287867in}{0.731061in}}%
\pgfpathlineto{\pgfqpoint{2.289097in}{0.708115in}}%
\pgfpathlineto{\pgfqpoint{2.289507in}{0.709978in}}%
\pgfpathlineto{\pgfqpoint{2.291557in}{0.714560in}}%
\pgfpathlineto{\pgfqpoint{2.292787in}{0.713220in}}%
\pgfpathlineto{\pgfqpoint{2.294837in}{0.704569in}}%
\pgfpathlineto{\pgfqpoint{2.295657in}{0.699326in}}%
\pgfpathlineto{\pgfqpoint{2.296067in}{0.701599in}}%
\pgfpathlineto{\pgfqpoint{2.298937in}{0.711544in}}%
\pgfpathlineto{\pgfqpoint{2.300167in}{0.710902in}}%
\pgfpathlineto{\pgfqpoint{2.302217in}{0.704269in}}%
\pgfpathlineto{\pgfqpoint{2.305497in}{0.750645in}}%
\pgfpathlineto{\pgfqpoint{2.306727in}{0.754732in}}%
\pgfpathlineto{\pgfqpoint{2.307137in}{0.754264in}}%
\pgfpathlineto{\pgfqpoint{2.308367in}{0.747602in}}%
\pgfpathlineto{\pgfqpoint{2.310827in}{0.715227in}}%
\pgfpathlineto{\pgfqpoint{2.311647in}{0.701012in}}%
\pgfpathlineto{\pgfqpoint{2.312467in}{0.703228in}}%
\pgfpathlineto{\pgfqpoint{2.314517in}{0.705667in}}%
\pgfpathlineto{\pgfqpoint{2.315747in}{0.703931in}}%
\pgfpathlineto{\pgfqpoint{2.318207in}{0.695489in}}%
\pgfpathlineto{\pgfqpoint{2.318617in}{0.697242in}}%
\pgfpathlineto{\pgfqpoint{2.321077in}{0.703360in}}%
\pgfpathlineto{\pgfqpoint{2.322307in}{0.703044in}}%
\pgfpathlineto{\pgfqpoint{2.324357in}{0.698573in}}%
\pgfpathlineto{\pgfqpoint{2.327637in}{0.731015in}}%
\pgfpathlineto{\pgfqpoint{2.328867in}{0.732912in}}%
\pgfpathlineto{\pgfqpoint{2.330097in}{0.728683in}}%
\pgfpathlineto{\pgfqpoint{2.332557in}{0.705592in}}%
\pgfpathlineto{\pgfqpoint{2.333377in}{0.695921in}}%
\pgfpathlineto{\pgfqpoint{2.334197in}{0.697818in}}%
\pgfpathlineto{\pgfqpoint{2.336247in}{0.699136in}}%
\pgfpathlineto{\pgfqpoint{2.337887in}{0.696608in}}%
\pgfpathlineto{\pgfqpoint{2.339527in}{0.692195in}}%
\pgfpathlineto{\pgfqpoint{2.339937in}{0.693448in}}%
\pgfpathlineto{\pgfqpoint{2.342397in}{0.697603in}}%
\pgfpathlineto{\pgfqpoint{2.343627in}{0.697156in}}%
\pgfpathlineto{\pgfqpoint{2.345267in}{0.694088in}}%
\pgfpathlineto{\pgfqpoint{2.349367in}{0.717467in}}%
\pgfpathlineto{\pgfqpoint{2.349777in}{0.717352in}}%
\pgfpathlineto{\pgfqpoint{2.351007in}{0.714036in}}%
\pgfpathlineto{\pgfqpoint{2.353467in}{0.696965in}}%
\pgfpathlineto{\pgfqpoint{2.354287in}{0.692958in}}%
\pgfpathlineto{\pgfqpoint{2.354697in}{0.693597in}}%
\pgfpathlineto{\pgfqpoint{2.356747in}{0.694583in}}%
\pgfpathlineto{\pgfqpoint{2.358387in}{0.692732in}}%
\pgfpathlineto{\pgfqpoint{2.360027in}{0.690265in}}%
\pgfpathlineto{\pgfqpoint{2.360437in}{0.691099in}}%
\pgfpathlineto{\pgfqpoint{2.362897in}{0.693525in}}%
\pgfpathlineto{\pgfqpoint{2.364537in}{0.692414in}}%
\pgfpathlineto{\pgfqpoint{2.365357in}{0.691137in}}%
\pgfpathlineto{\pgfqpoint{2.369457in}{0.706572in}}%
\pgfpathlineto{\pgfqpoint{2.370687in}{0.704713in}}%
\pgfpathlineto{\pgfqpoint{2.372737in}{0.695633in}}%
\pgfpathlineto{\pgfqpoint{2.373967in}{0.690366in}}%
\pgfpathlineto{\pgfqpoint{2.374377in}{0.690801in}}%
\pgfpathlineto{\pgfqpoint{2.376427in}{0.691370in}}%
\pgfpathlineto{\pgfqpoint{2.378477in}{0.689373in}}%
\pgfpathlineto{\pgfqpoint{2.379297in}{0.688353in}}%
\pgfpathlineto{\pgfqpoint{2.379707in}{0.688945in}}%
\pgfpathlineto{\pgfqpoint{2.382167in}{0.690677in}}%
\pgfpathlineto{\pgfqpoint{2.384217in}{0.689483in}}%
\pgfpathlineto{\pgfqpoint{2.384627in}{0.689000in}}%
\pgfpathlineto{\pgfqpoint{2.387907in}{0.698926in}}%
\pgfpathlineto{\pgfqpoint{2.389137in}{0.698632in}}%
\pgfpathlineto{\pgfqpoint{2.390777in}{0.694897in}}%
\pgfpathlineto{\pgfqpoint{2.392827in}{0.688636in}}%
\pgfpathlineto{\pgfqpoint{2.393237in}{0.688916in}}%
\pgfpathlineto{\pgfqpoint{2.395697in}{0.688992in}}%
\pgfpathlineto{\pgfqpoint{2.398977in}{0.688224in}}%
\pgfpathlineto{\pgfqpoint{2.401437in}{0.688602in}}%
\pgfpathlineto{\pgfqpoint{2.403077in}{0.688013in}}%
\pgfpathlineto{\pgfqpoint{2.405947in}{0.693878in}}%
\pgfpathlineto{\pgfqpoint{2.407587in}{0.693468in}}%
\pgfpathlineto{\pgfqpoint{2.409637in}{0.689511in}}%
\pgfpathlineto{\pgfqpoint{2.410867in}{0.687459in}}%
\pgfpathlineto{\pgfqpoint{2.411277in}{0.687634in}}%
\pgfpathlineto{\pgfqpoint{2.414147in}{0.687390in}}%
\pgfpathlineto{\pgfqpoint{2.416607in}{0.687119in}}%
\pgfpathlineto{\pgfqpoint{2.419477in}{0.687261in}}%
\pgfpathlineto{\pgfqpoint{2.420707in}{0.687196in}}%
\pgfpathlineto{\pgfqpoint{2.423577in}{0.690718in}}%
\pgfpathlineto{\pgfqpoint{2.425217in}{0.690157in}}%
\pgfpathlineto{\pgfqpoint{2.429727in}{0.686887in}}%
\pgfpathlineto{\pgfqpoint{2.437927in}{0.686993in}}%
\pgfpathlineto{\pgfqpoint{2.440797in}{0.688625in}}%
\pgfpathlineto{\pgfqpoint{2.443257in}{0.687105in}}%
\pgfpathlineto{\pgfqpoint{2.445307in}{0.686252in}}%
\pgfpathlineto{\pgfqpoint{2.459247in}{0.686323in}}%
\pgfpathlineto{\pgfqpoint{2.461707in}{0.685905in}}%
\pgfpathlineto{\pgfqpoint{2.564617in}{0.685377in}}%
\pgfpathlineto{\pgfqpoint{2.952066in}{0.685365in}}%
\pgfpathlineto{\pgfqpoint{2.952066in}{0.685365in}}%
\pgfusepath{stroke}%
\end{pgfscope}%
\begin{pgfscope}%
\pgfpathrectangle{\pgfqpoint{0.800000in}{0.528000in}}{\pgfqpoint{2.254545in}{1.680000in}}%
\pgfusepath{clip}%
\pgfsetrectcap%
\pgfsetroundjoin%
\pgfsetlinewidth{1.505625pt}%
\definecolor{currentstroke}{rgb}{0.498039,0.498039,0.498039}%
\pgfsetstrokecolor{currentstroke}%
\pgfsetdash{}{0pt}%
\pgfpathmoveto{\pgfqpoint{0.902479in}{1.531273in}}%
\pgfpathlineto{\pgfqpoint{0.907809in}{1.530279in}}%
\pgfpathlineto{\pgfqpoint{0.913139in}{1.526966in}}%
\pgfpathlineto{\pgfqpoint{0.918469in}{1.521045in}}%
\pgfpathlineto{\pgfqpoint{0.924209in}{1.511206in}}%
\pgfpathlineto{\pgfqpoint{0.930359in}{1.495775in}}%
\pgfpathlineto{\pgfqpoint{0.936509in}{1.474076in}}%
\pgfpathlineto{\pgfqpoint{0.943069in}{1.442436in}}%
\pgfpathlineto{\pgfqpoint{0.950039in}{1.397283in}}%
\pgfpathlineto{\pgfqpoint{0.957419in}{1.334472in}}%
\pgfpathlineto{\pgfqpoint{0.962339in}{1.285536in}}%
\pgfpathlineto{\pgfqpoint{0.962749in}{1.293418in}}%
\pgfpathlineto{\pgfqpoint{0.982019in}{1.686808in}}%
\pgfpathlineto{\pgfqpoint{0.996779in}{1.968831in}}%
\pgfpathlineto{\pgfqpoint{1.005389in}{2.095845in}}%
\pgfpathlineto{\pgfqpoint{1.011949in}{2.163156in}}%
\pgfpathlineto{\pgfqpoint{1.016869in}{2.193035in}}%
\pgfpathlineto{\pgfqpoint{1.020149in}{2.201903in}}%
\pgfpathlineto{\pgfqpoint{1.022199in}{2.202636in}}%
\pgfpathlineto{\pgfqpoint{1.023839in}{2.200447in}}%
\pgfpathlineto{\pgfqpoint{1.026299in}{2.192386in}}%
\pgfpathlineto{\pgfqpoint{1.029579in}{2.172405in}}%
\pgfpathlineto{\pgfqpoint{1.033679in}{2.132006in}}%
\pgfpathlineto{\pgfqpoint{1.038599in}{2.060054in}}%
\pgfpathlineto{\pgfqpoint{1.044749in}{1.933137in}}%
\pgfpathlineto{\pgfqpoint{1.051719in}{1.739640in}}%
\pgfpathlineto{\pgfqpoint{1.060329in}{1.431992in}}%
\pgfpathlineto{\pgfqpoint{1.068939in}{1.066570in}}%
\pgfpathlineto{\pgfqpoint{1.070169in}{1.076333in}}%
\pgfpathlineto{\pgfqpoint{1.080829in}{1.167981in}}%
\pgfpathlineto{\pgfqpoint{1.087389in}{1.207406in}}%
\pgfpathlineto{\pgfqpoint{1.091489in}{1.220845in}}%
\pgfpathlineto{\pgfqpoint{1.093949in}{1.223118in}}%
\pgfpathlineto{\pgfqpoint{1.095589in}{1.221647in}}%
\pgfpathlineto{\pgfqpoint{1.097639in}{1.215863in}}%
\pgfpathlineto{\pgfqpoint{1.100509in}{1.199034in}}%
\pgfpathlineto{\pgfqpoint{1.103789in}{1.164523in}}%
\pgfpathlineto{\pgfqpoint{1.107479in}{1.101469in}}%
\pgfpathlineto{\pgfqpoint{1.111989in}{0.982135in}}%
\pgfpathlineto{\pgfqpoint{1.114449in}{0.895627in}}%
\pgfpathlineto{\pgfqpoint{1.114859in}{0.898428in}}%
\pgfpathlineto{\pgfqpoint{1.130029in}{1.398596in}}%
\pgfpathlineto{\pgfqpoint{1.132899in}{1.432934in}}%
\pgfpathlineto{\pgfqpoint{1.134129in}{1.436562in}}%
\pgfpathlineto{\pgfqpoint{1.134539in}{1.436213in}}%
\pgfpathlineto{\pgfqpoint{1.135769in}{1.430415in}}%
\pgfpathlineto{\pgfqpoint{1.137819in}{1.404956in}}%
\pgfpathlineto{\pgfqpoint{1.140689in}{1.337808in}}%
\pgfpathlineto{\pgfqpoint{1.144379in}{1.207057in}}%
\pgfpathlineto{\pgfqpoint{1.144789in}{1.226470in}}%
\pgfpathlineto{\pgfqpoint{1.152579in}{1.729228in}}%
\pgfpathlineto{\pgfqpoint{1.157499in}{1.928298in}}%
\pgfpathlineto{\pgfqpoint{1.161189in}{2.004788in}}%
\pgfpathlineto{\pgfqpoint{1.163649in}{2.019969in}}%
\pgfpathlineto{\pgfqpoint{1.164469in}{2.018687in}}%
\pgfpathlineto{\pgfqpoint{1.166109in}{2.006724in}}%
\pgfpathlineto{\pgfqpoint{1.168569in}{1.965756in}}%
\pgfpathlineto{\pgfqpoint{1.171849in}{1.869983in}}%
\pgfpathlineto{\pgfqpoint{1.176359in}{1.668185in}}%
\pgfpathlineto{\pgfqpoint{1.182919in}{1.257273in}}%
\pgfpathlineto{\pgfqpoint{1.186199in}{1.034844in}}%
\pgfpathlineto{\pgfqpoint{1.186609in}{1.041637in}}%
\pgfpathlineto{\pgfqpoint{1.193169in}{1.132484in}}%
\pgfpathlineto{\pgfqpoint{1.197269in}{1.165082in}}%
\pgfpathlineto{\pgfqpoint{1.199729in}{1.171035in}}%
\pgfpathlineto{\pgfqpoint{1.200959in}{1.169039in}}%
\pgfpathlineto{\pgfqpoint{1.202599in}{1.160200in}}%
\pgfpathlineto{\pgfqpoint{1.205059in}{1.131148in}}%
\pgfpathlineto{\pgfqpoint{1.207929in}{1.067829in}}%
\pgfpathlineto{\pgfqpoint{1.211619in}{0.930325in}}%
\pgfpathlineto{\pgfqpoint{1.212849in}{0.872152in}}%
\pgfpathlineto{\pgfqpoint{1.213259in}{0.885847in}}%
\pgfpathlineto{\pgfqpoint{1.223099in}{1.309332in}}%
\pgfpathlineto{\pgfqpoint{1.225559in}{1.341704in}}%
\pgfpathlineto{\pgfqpoint{1.225969in}{1.342289in}}%
\pgfpathlineto{\pgfqpoint{1.226379in}{1.341447in}}%
\pgfpathlineto{\pgfqpoint{1.227609in}{1.330333in}}%
\pgfpathlineto{\pgfqpoint{1.229659in}{1.284068in}}%
\pgfpathlineto{\pgfqpoint{1.232939in}{1.149190in}}%
\pgfpathlineto{\pgfqpoint{1.233349in}{1.172592in}}%
\pgfpathlineto{\pgfqpoint{1.239499in}{1.665820in}}%
\pgfpathlineto{\pgfqpoint{1.243599in}{1.842085in}}%
\pgfpathlineto{\pgfqpoint{1.246059in}{1.880212in}}%
\pgfpathlineto{\pgfqpoint{1.246879in}{1.881605in}}%
\pgfpathlineto{\pgfqpoint{1.247699in}{1.877429in}}%
\pgfpathlineto{\pgfqpoint{1.249339in}{1.852722in}}%
\pgfpathlineto{\pgfqpoint{1.251799in}{1.776549in}}%
\pgfpathlineto{\pgfqpoint{1.255489in}{1.582888in}}%
\pgfpathlineto{\pgfqpoint{1.260819in}{1.171817in}}%
\pgfpathlineto{\pgfqpoint{1.262869in}{1.009230in}}%
\pgfpathlineto{\pgfqpoint{1.263279in}{1.017725in}}%
\pgfpathlineto{\pgfqpoint{1.268609in}{1.105197in}}%
\pgfpathlineto{\pgfqpoint{1.271889in}{1.129386in}}%
\pgfpathlineto{\pgfqpoint{1.272709in}{1.130423in}}%
\pgfpathlineto{\pgfqpoint{1.273119in}{1.130048in}}%
\pgfpathlineto{\pgfqpoint{1.274349in}{1.124977in}}%
\pgfpathlineto{\pgfqpoint{1.276399in}{1.101383in}}%
\pgfpathlineto{\pgfqpoint{1.278859in}{1.042657in}}%
\pgfpathlineto{\pgfqpoint{1.282139in}{0.902581in}}%
\pgfpathlineto{\pgfqpoint{1.282959in}{0.856766in}}%
\pgfpathlineto{\pgfqpoint{1.283369in}{0.869607in}}%
\pgfpathlineto{\pgfqpoint{1.291159in}{1.244500in}}%
\pgfpathlineto{\pgfqpoint{1.293209in}{1.272420in}}%
\pgfpathlineto{\pgfqpoint{1.293619in}{1.272311in}}%
\pgfpathlineto{\pgfqpoint{1.294439in}{1.266226in}}%
\pgfpathlineto{\pgfqpoint{1.296079in}{1.231120in}}%
\pgfpathlineto{\pgfqpoint{1.298949in}{1.106417in}}%
\pgfpathlineto{\pgfqpoint{1.299359in}{1.128353in}}%
\pgfpathlineto{\pgfqpoint{1.304689in}{1.603456in}}%
\pgfpathlineto{\pgfqpoint{1.307969in}{1.748737in}}%
\pgfpathlineto{\pgfqpoint{1.310019in}{1.774692in}}%
\pgfpathlineto{\pgfqpoint{1.310839in}{1.771195in}}%
\pgfpathlineto{\pgfqpoint{1.312479in}{1.741177in}}%
\pgfpathlineto{\pgfqpoint{1.314939in}{1.641978in}}%
\pgfpathlineto{\pgfqpoint{1.318629in}{1.388268in}}%
\pgfpathlineto{\pgfqpoint{1.323139in}{0.989015in}}%
\pgfpathlineto{\pgfqpoint{1.323959in}{1.008062in}}%
\pgfpathlineto{\pgfqpoint{1.328469in}{1.084362in}}%
\pgfpathlineto{\pgfqpoint{1.330929in}{1.097839in}}%
\pgfpathlineto{\pgfqpoint{1.331749in}{1.096228in}}%
\pgfpathlineto{\pgfqpoint{1.332979in}{1.086890in}}%
\pgfpathlineto{\pgfqpoint{1.335029in}{1.049596in}}%
\pgfpathlineto{\pgfqpoint{1.337899in}{0.942221in}}%
\pgfpathlineto{\pgfqpoint{1.339539in}{0.849402in}}%
\pgfpathlineto{\pgfqpoint{1.339949in}{0.852583in}}%
\pgfpathlineto{\pgfqpoint{1.346509in}{1.191620in}}%
\pgfpathlineto{\pgfqpoint{1.348559in}{1.218902in}}%
\pgfpathlineto{\pgfqpoint{1.349379in}{1.213310in}}%
\pgfpathlineto{\pgfqpoint{1.351019in}{1.174045in}}%
\pgfpathlineto{\pgfqpoint{1.353479in}{1.079140in}}%
\pgfpathlineto{\pgfqpoint{1.358399in}{1.548919in}}%
\pgfpathlineto{\pgfqpoint{1.361269in}{1.673670in}}%
\pgfpathlineto{\pgfqpoint{1.362909in}{1.688927in}}%
\pgfpathlineto{\pgfqpoint{1.363729in}{1.681468in}}%
\pgfpathlineto{\pgfqpoint{1.365369in}{1.637600in}}%
\pgfpathlineto{\pgfqpoint{1.368239in}{1.476177in}}%
\pgfpathlineto{\pgfqpoint{1.372749in}{1.056405in}}%
\pgfpathlineto{\pgfqpoint{1.373569in}{0.972437in}}%
\pgfpathlineto{\pgfqpoint{1.374389in}{0.986107in}}%
\pgfpathlineto{\pgfqpoint{1.378489in}{1.060881in}}%
\pgfpathlineto{\pgfqpoint{1.380539in}{1.070954in}}%
\pgfpathlineto{\pgfqpoint{1.381359in}{1.068016in}}%
\pgfpathlineto{\pgfqpoint{1.382999in}{1.047555in}}%
\pgfpathlineto{\pgfqpoint{1.385049in}{0.989463in}}%
\pgfpathlineto{\pgfqpoint{1.387919in}{0.838336in}}%
\pgfpathlineto{\pgfqpoint{1.388739in}{0.869968in}}%
\pgfpathlineto{\pgfqpoint{1.394069in}{1.152035in}}%
\pgfpathlineto{\pgfqpoint{1.395709in}{1.176132in}}%
\pgfpathlineto{\pgfqpoint{1.396119in}{1.175311in}}%
\pgfpathlineto{\pgfqpoint{1.397349in}{1.156154in}}%
\pgfpathlineto{\pgfqpoint{1.399399in}{1.073585in}}%
\pgfpathlineto{\pgfqpoint{1.399809in}{1.050807in}}%
\pgfpathlineto{\pgfqpoint{1.400219in}{1.053195in}}%
\pgfpathlineto{\pgfqpoint{1.404729in}{1.501694in}}%
\pgfpathlineto{\pgfqpoint{1.407599in}{1.614584in}}%
\pgfpathlineto{\pgfqpoint{1.408419in}{1.619179in}}%
\pgfpathlineto{\pgfqpoint{1.408829in}{1.616920in}}%
\pgfpathlineto{\pgfqpoint{1.410059in}{1.592384in}}%
\pgfpathlineto{\pgfqpoint{1.412109in}{1.495851in}}%
\pgfpathlineto{\pgfqpoint{1.415389in}{1.220507in}}%
\pgfpathlineto{\pgfqpoint{1.417849in}{0.955754in}}%
\pgfpathlineto{\pgfqpoint{1.418669in}{0.972951in}}%
\pgfpathlineto{\pgfqpoint{1.422359in}{1.041642in}}%
\pgfpathlineto{\pgfqpoint{1.423589in}{1.048368in}}%
\pgfpathlineto{\pgfqpoint{1.423999in}{1.048296in}}%
\pgfpathlineto{\pgfqpoint{1.424819in}{1.044243in}}%
\pgfpathlineto{\pgfqpoint{1.426459in}{1.018339in}}%
\pgfpathlineto{\pgfqpoint{1.428919in}{0.926452in}}%
\pgfpathlineto{\pgfqpoint{1.430559in}{0.826834in}}%
\pgfpathlineto{\pgfqpoint{1.430969in}{0.838604in}}%
\pgfpathlineto{\pgfqpoint{1.436298in}{1.125397in}}%
\pgfpathlineto{\pgfqpoint{1.437528in}{1.141193in}}%
\pgfpathlineto{\pgfqpoint{1.437938in}{1.140224in}}%
\pgfpathlineto{\pgfqpoint{1.439168in}{1.118502in}}%
\pgfpathlineto{\pgfqpoint{1.441218in}{1.026500in}}%
\pgfpathlineto{\pgfqpoint{1.441628in}{1.032984in}}%
\pgfpathlineto{\pgfqpoint{1.446138in}{1.480826in}}%
\pgfpathlineto{\pgfqpoint{1.448598in}{1.559963in}}%
\pgfpathlineto{\pgfqpoint{1.449008in}{1.560618in}}%
\pgfpathlineto{\pgfqpoint{1.449828in}{1.551392in}}%
\pgfpathlineto{\pgfqpoint{1.451468in}{1.492559in}}%
\pgfpathlineto{\pgfqpoint{1.454338in}{1.276287in}}%
\pgfpathlineto{\pgfqpoint{1.457618in}{0.943617in}}%
\pgfpathlineto{\pgfqpoint{1.458438in}{0.966845in}}%
\pgfpathlineto{\pgfqpoint{1.461718in}{1.025517in}}%
\pgfpathlineto{\pgfqpoint{1.462538in}{1.029154in}}%
\pgfpathlineto{\pgfqpoint{1.462948in}{1.028898in}}%
\pgfpathlineto{\pgfqpoint{1.464178in}{1.018761in}}%
\pgfpathlineto{\pgfqpoint{1.465818in}{0.979963in}}%
\pgfpathlineto{\pgfqpoint{1.468278in}{0.858812in}}%
\pgfpathlineto{\pgfqpoint{1.469098in}{0.820510in}}%
\pgfpathlineto{\pgfqpoint{1.469508in}{0.848579in}}%
\pgfpathlineto{\pgfqpoint{1.474018in}{1.095371in}}%
\pgfpathlineto{\pgfqpoint{1.475248in}{1.112057in}}%
\pgfpathlineto{\pgfqpoint{1.475658in}{1.110714in}}%
\pgfpathlineto{\pgfqpoint{1.476888in}{1.085938in}}%
\pgfpathlineto{\pgfqpoint{1.478938in}{1.007171in}}%
\pgfpathlineto{\pgfqpoint{1.483038in}{1.432158in}}%
\pgfpathlineto{\pgfqpoint{1.485498in}{1.510966in}}%
\pgfpathlineto{\pgfqpoint{1.485908in}{1.509866in}}%
\pgfpathlineto{\pgfqpoint{1.487138in}{1.482907in}}%
\pgfpathlineto{\pgfqpoint{1.489188in}{1.364972in}}%
\pgfpathlineto{\pgfqpoint{1.492878in}{0.977302in}}%
\pgfpathlineto{\pgfqpoint{1.493288in}{0.932617in}}%
\pgfpathlineto{\pgfqpoint{1.494108in}{0.949333in}}%
\pgfpathlineto{\pgfqpoint{1.497388in}{1.010210in}}%
\pgfpathlineto{\pgfqpoint{1.498208in}{1.012468in}}%
\pgfpathlineto{\pgfqpoint{1.498618in}{1.011142in}}%
\pgfpathlineto{\pgfqpoint{1.499848in}{0.996058in}}%
\pgfpathlineto{\pgfqpoint{1.501898in}{0.927940in}}%
\pgfpathlineto{\pgfqpoint{1.503948in}{0.812574in}}%
\pgfpathlineto{\pgfqpoint{1.504358in}{0.837385in}}%
\pgfpathlineto{\pgfqpoint{1.508868in}{1.078754in}}%
\pgfpathlineto{\pgfqpoint{1.509688in}{1.087271in}}%
\pgfpathlineto{\pgfqpoint{1.510098in}{1.085831in}}%
\pgfpathlineto{\pgfqpoint{1.511328in}{1.058923in}}%
\pgfpathlineto{\pgfqpoint{1.512968in}{0.983898in}}%
\pgfpathlineto{\pgfqpoint{1.518298in}{1.456060in}}%
\pgfpathlineto{\pgfqpoint{1.519118in}{1.467964in}}%
\pgfpathlineto{\pgfqpoint{1.519528in}{1.467104in}}%
\pgfpathlineto{\pgfqpoint{1.520758in}{1.438055in}}%
\pgfpathlineto{\pgfqpoint{1.522808in}{1.308637in}}%
\pgfpathlineto{\pgfqpoint{1.526498in}{0.920269in}}%
\pgfpathlineto{\pgfqpoint{1.527728in}{0.956494in}}%
\pgfpathlineto{\pgfqpoint{1.530188in}{0.996474in}}%
\pgfpathlineto{\pgfqpoint{1.530598in}{0.997832in}}%
\pgfpathlineto{\pgfqpoint{1.531008in}{0.997378in}}%
\pgfpathlineto{\pgfqpoint{1.532238in}{0.983926in}}%
\pgfpathlineto{\pgfqpoint{1.533878in}{0.933471in}}%
\pgfpathlineto{\pgfqpoint{1.535928in}{0.812702in}}%
\pgfpathlineto{\pgfqpoint{1.536748in}{0.845426in}}%
\pgfpathlineto{\pgfqpoint{1.540848in}{1.060037in}}%
\pgfpathlineto{\pgfqpoint{1.541668in}{1.065757in}}%
\pgfpathlineto{\pgfqpoint{1.542078in}{1.062428in}}%
\pgfpathlineto{\pgfqpoint{1.543308in}{1.028411in}}%
\pgfpathlineto{\pgfqpoint{1.544538in}{0.968662in}}%
\pgfpathlineto{\pgfqpoint{1.549868in}{1.426603in}}%
\pgfpathlineto{\pgfqpoint{1.550278in}{1.430534in}}%
\pgfpathlineto{\pgfqpoint{1.550688in}{1.429461in}}%
\pgfpathlineto{\pgfqpoint{1.551918in}{1.397062in}}%
\pgfpathlineto{\pgfqpoint{1.553968in}{1.254678in}}%
\pgfpathlineto{\pgfqpoint{1.557248in}{0.915514in}}%
\pgfpathlineto{\pgfqpoint{1.558068in}{0.940664in}}%
\pgfpathlineto{\pgfqpoint{1.560528in}{0.983632in}}%
\pgfpathlineto{\pgfqpoint{1.560938in}{0.984900in}}%
\pgfpathlineto{\pgfqpoint{1.561348in}{0.984144in}}%
\pgfpathlineto{\pgfqpoint{1.562578in}{0.968352in}}%
\pgfpathlineto{\pgfqpoint{1.564628in}{0.889785in}}%
\pgfpathlineto{\pgfqpoint{1.565858in}{0.809380in}}%
\pgfpathlineto{\pgfqpoint{1.566268in}{0.809975in}}%
\pgfpathlineto{\pgfqpoint{1.570368in}{1.039335in}}%
\pgfpathlineto{\pgfqpoint{1.571188in}{1.047305in}}%
\pgfpathlineto{\pgfqpoint{1.571598in}{1.044668in}}%
\pgfpathlineto{\pgfqpoint{1.572828in}{1.010917in}}%
\pgfpathlineto{\pgfqpoint{1.574058in}{0.960053in}}%
\pgfpathlineto{\pgfqpoint{1.578158in}{1.370579in}}%
\pgfpathlineto{\pgfqpoint{1.579388in}{1.397655in}}%
\pgfpathlineto{\pgfqpoint{1.579798in}{1.395705in}}%
\pgfpathlineto{\pgfqpoint{1.581028in}{1.358099in}}%
\pgfpathlineto{\pgfqpoint{1.583078in}{1.200508in}}%
\pgfpathlineto{\pgfqpoint{1.585538in}{0.905810in}}%
\pgfpathlineto{\pgfqpoint{1.586358in}{0.923860in}}%
\pgfpathlineto{\pgfqpoint{1.588818in}{0.971684in}}%
\pgfpathlineto{\pgfqpoint{1.589228in}{0.973269in}}%
\pgfpathlineto{\pgfqpoint{1.589638in}{0.972646in}}%
\pgfpathlineto{\pgfqpoint{1.590868in}{0.955993in}}%
\pgfpathlineto{\pgfqpoint{1.592508in}{0.894316in}}%
\pgfpathlineto{\pgfqpoint{1.594148in}{0.797990in}}%
\pgfpathlineto{\pgfqpoint{1.594558in}{0.824895in}}%
\pgfpathlineto{\pgfqpoint{1.598248in}{1.025190in}}%
\pgfpathlineto{\pgfqpoint{1.599068in}{1.030458in}}%
\pgfpathlineto{\pgfqpoint{1.599888in}{1.016915in}}%
\pgfpathlineto{\pgfqpoint{1.601528in}{0.944174in}}%
\pgfpathlineto{\pgfqpoint{1.606448in}{1.367145in}}%
\pgfpathlineto{\pgfqpoint{1.606858in}{1.367916in}}%
\pgfpathlineto{\pgfqpoint{1.607678in}{1.351925in}}%
\pgfpathlineto{\pgfqpoint{1.609318in}{1.254559in}}%
\pgfpathlineto{\pgfqpoint{1.612598in}{0.895429in}}%
\pgfpathlineto{\pgfqpoint{1.613828in}{0.933726in}}%
\pgfpathlineto{\pgfqpoint{1.615878in}{0.962839in}}%
\pgfpathlineto{\pgfqpoint{1.616288in}{0.962099in}}%
\pgfpathlineto{\pgfqpoint{1.617518in}{0.943735in}}%
\pgfpathlineto{\pgfqpoint{1.619568in}{0.851477in}}%
\pgfpathlineto{\pgfqpoint{1.620388in}{0.794293in}}%
\pgfpathlineto{\pgfqpoint{1.620798in}{0.810632in}}%
\pgfpathlineto{\pgfqpoint{1.624488in}{1.012217in}}%
\pgfpathlineto{\pgfqpoint{1.624898in}{1.016126in}}%
\pgfpathlineto{\pgfqpoint{1.625308in}{1.015032in}}%
\pgfpathlineto{\pgfqpoint{1.626538in}{0.982376in}}%
\pgfpathlineto{\pgfqpoint{1.627358in}{0.938999in}}%
\pgfpathlineto{\pgfqpoint{1.627768in}{0.955803in}}%
\pgfpathlineto{\pgfqpoint{1.631048in}{1.306282in}}%
\pgfpathlineto{\pgfqpoint{1.632278in}{1.341617in}}%
\pgfpathlineto{\pgfqpoint{1.632688in}{1.340557in}}%
\pgfpathlineto{\pgfqpoint{1.633918in}{1.300196in}}%
\pgfpathlineto{\pgfqpoint{1.635968in}{1.123675in}}%
\pgfpathlineto{\pgfqpoint{1.638018in}{0.890320in}}%
\pgfpathlineto{\pgfqpoint{1.638428in}{0.904657in}}%
\pgfpathlineto{\pgfqpoint{1.640888in}{0.952974in}}%
\pgfpathlineto{\pgfqpoint{1.641298in}{0.953211in}}%
\pgfpathlineto{\pgfqpoint{1.642118in}{0.945340in}}%
\pgfpathlineto{\pgfqpoint{1.643758in}{0.891529in}}%
\pgfpathlineto{\pgfqpoint{1.645398in}{0.790729in}}%
\pgfpathlineto{\pgfqpoint{1.645808in}{0.812458in}}%
\pgfpathlineto{\pgfqpoint{1.649088in}{0.996807in}}%
\pgfpathlineto{\pgfqpoint{1.649908in}{1.002758in}}%
\pgfpathlineto{\pgfqpoint{1.650728in}{0.987656in}}%
\pgfpathlineto{\pgfqpoint{1.651958in}{0.929388in}}%
\pgfpathlineto{\pgfqpoint{1.652368in}{0.948345in}}%
\pgfpathlineto{\pgfqpoint{1.655648in}{1.292771in}}%
\pgfpathlineto{\pgfqpoint{1.656878in}{1.317912in}}%
\pgfpathlineto{\pgfqpoint{1.657288in}{1.312595in}}%
\pgfpathlineto{\pgfqpoint{1.658518in}{1.257594in}}%
\pgfpathlineto{\pgfqpoint{1.660978in}{1.001333in}}%
\pgfpathlineto{\pgfqpoint{1.662208in}{0.889179in}}%
\pgfpathlineto{\pgfqpoint{1.662618in}{0.903236in}}%
\pgfpathlineto{\pgfqpoint{1.665078in}{0.944769in}}%
\pgfpathlineto{\pgfqpoint{1.665898in}{0.937727in}}%
\pgfpathlineto{\pgfqpoint{1.667538in}{0.883167in}}%
\pgfpathlineto{\pgfqpoint{1.669178in}{0.787389in}}%
\pgfpathlineto{\pgfqpoint{1.669588in}{0.817839in}}%
\pgfpathlineto{\pgfqpoint{1.672868in}{0.989647in}}%
\pgfpathlineto{\pgfqpoint{1.673278in}{0.991373in}}%
\pgfpathlineto{\pgfqpoint{1.674098in}{0.978148in}}%
\pgfpathlineto{\pgfqpoint{1.675328in}{0.920466in}}%
\pgfpathlineto{\pgfqpoint{1.675738in}{0.942198in}}%
\pgfpathlineto{\pgfqpoint{1.679018in}{1.279475in}}%
\pgfpathlineto{\pgfqpoint{1.679838in}{1.296495in}}%
\pgfpathlineto{\pgfqpoint{1.680248in}{1.294016in}}%
\pgfpathlineto{\pgfqpoint{1.681478in}{1.244409in}}%
\pgfpathlineto{\pgfqpoint{1.683938in}{0.986998in}}%
\pgfpathlineto{\pgfqpoint{1.684758in}{0.879692in}}%
\pgfpathlineto{\pgfqpoint{1.685578in}{0.900408in}}%
\pgfpathlineto{\pgfqpoint{1.687628in}{0.936830in}}%
\pgfpathlineto{\pgfqpoint{1.688038in}{0.936032in}}%
\pgfpathlineto{\pgfqpoint{1.689268in}{0.913506in}}%
\pgfpathlineto{\pgfqpoint{1.691728in}{0.784468in}}%
\pgfpathlineto{\pgfqpoint{1.692138in}{0.816032in}}%
\pgfpathlineto{\pgfqpoint{1.695008in}{0.975535in}}%
\pgfpathlineto{\pgfqpoint{1.695418in}{0.980442in}}%
\pgfpathlineto{\pgfqpoint{1.695828in}{0.979510in}}%
\pgfpathlineto{\pgfqpoint{1.697058in}{0.942619in}}%
\pgfpathlineto{\pgfqpoint{1.697878in}{0.916556in}}%
\pgfpathlineto{\pgfqpoint{1.701158in}{1.261730in}}%
\pgfpathlineto{\pgfqpoint{1.701978in}{1.276869in}}%
\pgfpathlineto{\pgfqpoint{1.702388in}{1.272782in}}%
\pgfpathlineto{\pgfqpoint{1.703618in}{1.216078in}}%
\pgfpathlineto{\pgfqpoint{1.706078in}{0.939902in}}%
\pgfpathlineto{\pgfqpoint{1.706898in}{0.877715in}}%
\pgfpathlineto{\pgfqpoint{1.707308in}{0.892243in}}%
\pgfpathlineto{\pgfqpoint{1.709358in}{0.929625in}}%
\pgfpathlineto{\pgfqpoint{1.709768in}{0.928433in}}%
\pgfpathlineto{\pgfqpoint{1.710998in}{0.903234in}}%
\pgfpathlineto{\pgfqpoint{1.713048in}{0.782176in}}%
\pgfpathlineto{\pgfqpoint{1.713868in}{0.829745in}}%
\pgfpathlineto{\pgfqpoint{1.716738in}{0.970451in}}%
\pgfpathlineto{\pgfqpoint{1.717148in}{0.970055in}}%
\pgfpathlineto{\pgfqpoint{1.718378in}{0.933114in}}%
\pgfpathlineto{\pgfqpoint{1.718788in}{0.910250in}}%
\pgfpathlineto{\pgfqpoint{1.719198in}{0.918072in}}%
\pgfpathlineto{\pgfqpoint{1.722068in}{1.234763in}}%
\pgfpathlineto{\pgfqpoint{1.722888in}{1.258096in}}%
\pgfpathlineto{\pgfqpoint{1.723298in}{1.257375in}}%
\pgfpathlineto{\pgfqpoint{1.724528in}{1.207529in}}%
\pgfpathlineto{\pgfqpoint{1.726578in}{0.990351in}}%
\pgfpathlineto{\pgfqpoint{1.727808in}{0.874064in}}%
\pgfpathlineto{\pgfqpoint{1.728218in}{0.888609in}}%
\pgfpathlineto{\pgfqpoint{1.730268in}{0.922848in}}%
\pgfpathlineto{\pgfqpoint{1.731088in}{0.913795in}}%
\pgfpathlineto{\pgfqpoint{1.732728in}{0.846725in}}%
\pgfpathlineto{\pgfqpoint{1.733958in}{0.787070in}}%
\pgfpathlineto{\pgfqpoint{1.737238in}{0.961768in}}%
\pgfpathlineto{\pgfqpoint{1.737648in}{0.960625in}}%
\pgfpathlineto{\pgfqpoint{1.738878in}{0.920115in}}%
\pgfpathlineto{\pgfqpoint{1.739288in}{0.895865in}}%
\pgfpathlineto{\pgfqpoint{1.739698in}{0.932586in}}%
\pgfpathlineto{\pgfqpoint{1.742568in}{1.229479in}}%
\pgfpathlineto{\pgfqpoint{1.743388in}{1.241764in}}%
\pgfpathlineto{\pgfqpoint{1.743798in}{1.234913in}}%
\pgfpathlineto{\pgfqpoint{1.745028in}{1.165439in}}%
\pgfpathlineto{\pgfqpoint{1.747488in}{0.866158in}}%
\pgfpathlineto{\pgfqpoint{1.748718in}{0.898474in}}%
\pgfpathlineto{\pgfqpoint{1.749948in}{0.916558in}}%
\pgfpathlineto{\pgfqpoint{1.750358in}{0.915832in}}%
\pgfpathlineto{\pgfqpoint{1.751178in}{0.902555in}}%
\pgfpathlineto{\pgfqpoint{1.752818in}{0.823592in}}%
\pgfpathlineto{\pgfqpoint{1.753638in}{0.777827in}}%
\pgfpathlineto{\pgfqpoint{1.754048in}{0.810870in}}%
\pgfpathlineto{\pgfqpoint{1.756918in}{0.953752in}}%
\pgfpathlineto{\pgfqpoint{1.757328in}{0.951728in}}%
\pgfpathlineto{\pgfqpoint{1.758558in}{0.907345in}}%
\pgfpathlineto{\pgfqpoint{1.758968in}{0.893176in}}%
\pgfpathlineto{\pgfqpoint{1.762248in}{1.221833in}}%
\pgfpathlineto{\pgfqpoint{1.762658in}{1.226749in}}%
\pgfpathlineto{\pgfqpoint{1.763068in}{1.222527in}}%
\pgfpathlineto{\pgfqpoint{1.764298in}{1.157739in}}%
\pgfpathlineto{\pgfqpoint{1.766758in}{0.860120in}}%
\pgfpathlineto{\pgfqpoint{1.767988in}{0.895369in}}%
\pgfpathlineto{\pgfqpoint{1.769218in}{0.910939in}}%
\pgfpathlineto{\pgfqpoint{1.769628in}{0.908788in}}%
\pgfpathlineto{\pgfqpoint{1.770858in}{0.876302in}}%
\pgfpathlineto{\pgfqpoint{1.772498in}{0.775158in}}%
\pgfpathlineto{\pgfqpoint{1.772908in}{0.797703in}}%
\pgfpathlineto{\pgfqpoint{1.775778in}{0.946103in}}%
\pgfpathlineto{\pgfqpoint{1.776188in}{0.944240in}}%
\pgfpathlineto{\pgfqpoint{1.777828in}{0.890631in}}%
\pgfpathlineto{\pgfqpoint{1.781108in}{1.210918in}}%
\pgfpathlineto{\pgfqpoint{1.781518in}{1.211592in}}%
\pgfpathlineto{\pgfqpoint{1.782338in}{1.184675in}}%
\pgfpathlineto{\pgfqpoint{1.783978in}{1.030935in}}%
\pgfpathlineto{\pgfqpoint{1.785208in}{0.859914in}}%
\pgfpathlineto{\pgfqpoint{1.786028in}{0.878263in}}%
\pgfpathlineto{\pgfqpoint{1.787668in}{0.905510in}}%
\pgfpathlineto{\pgfqpoint{1.788488in}{0.895786in}}%
\pgfpathlineto{\pgfqpoint{1.790128in}{0.819323in}}%
\pgfpathlineto{\pgfqpoint{1.790948in}{0.774775in}}%
\pgfpathlineto{\pgfqpoint{1.791358in}{0.808379in}}%
\pgfpathlineto{\pgfqpoint{1.793818in}{0.938055in}}%
\pgfpathlineto{\pgfqpoint{1.794228in}{0.938603in}}%
\pgfpathlineto{\pgfqpoint{1.795048in}{0.918215in}}%
\pgfpathlineto{\pgfqpoint{1.795868in}{0.881997in}}%
\pgfpathlineto{\pgfqpoint{1.799148in}{1.198189in}}%
\pgfpathlineto{\pgfqpoint{1.799558in}{1.197665in}}%
\pgfpathlineto{\pgfqpoint{1.800788in}{1.137931in}}%
\pgfpathlineto{\pgfqpoint{1.803248in}{0.851367in}}%
\pgfpathlineto{\pgfqpoint{1.804478in}{0.890964in}}%
\pgfpathlineto{\pgfqpoint{1.805298in}{0.900429in}}%
\pgfpathlineto{\pgfqpoint{1.805708in}{0.899005in}}%
\pgfpathlineto{\pgfqpoint{1.806938in}{0.866222in}}%
\pgfpathlineto{\pgfqpoint{1.808578in}{0.771559in}}%
\pgfpathlineto{\pgfqpoint{1.808988in}{0.805485in}}%
\pgfpathlineto{\pgfqpoint{1.811448in}{0.932253in}}%
\pgfpathlineto{\pgfqpoint{1.811858in}{0.931128in}}%
\pgfpathlineto{\pgfqpoint{1.813088in}{0.884682in}}%
\pgfpathlineto{\pgfqpoint{1.813498in}{0.888302in}}%
\pgfpathlineto{\pgfqpoint{1.816368in}{1.183475in}}%
\pgfpathlineto{\pgfqpoint{1.816778in}{1.186584in}}%
\pgfpathlineto{\pgfqpoint{1.817598in}{1.161613in}}%
\pgfpathlineto{\pgfqpoint{1.819238in}{1.001901in}}%
\pgfpathlineto{\pgfqpoint{1.820468in}{0.847650in}}%
\pgfpathlineto{\pgfqpoint{1.820878in}{0.863977in}}%
\pgfpathlineto{\pgfqpoint{1.822518in}{0.895760in}}%
\pgfpathlineto{\pgfqpoint{1.823338in}{0.886206in}}%
\pgfpathlineto{\pgfqpoint{1.824978in}{0.804826in}}%
\pgfpathlineto{\pgfqpoint{1.825388in}{0.771875in}}%
\pgfpathlineto{\pgfqpoint{1.825798in}{0.784667in}}%
\pgfpathlineto{\pgfqpoint{1.828258in}{0.924831in}}%
\pgfpathlineto{\pgfqpoint{1.828668in}{0.926212in}}%
\pgfpathlineto{\pgfqpoint{1.829488in}{0.905783in}}%
\pgfpathlineto{\pgfqpoint{1.830308in}{0.877037in}}%
\pgfpathlineto{\pgfqpoint{1.833178in}{1.172370in}}%
\pgfpathlineto{\pgfqpoint{1.833588in}{1.174778in}}%
\pgfpathlineto{\pgfqpoint{1.834408in}{1.147085in}}%
\pgfpathlineto{\pgfqpoint{1.836048in}{0.978544in}}%
\pgfpathlineto{\pgfqpoint{1.837278in}{0.851066in}}%
\pgfpathlineto{\pgfqpoint{1.837688in}{0.866385in}}%
\pgfpathlineto{\pgfqpoint{1.838918in}{0.890885in}}%
\pgfpathlineto{\pgfqpoint{1.839328in}{0.890520in}}%
\pgfpathlineto{\pgfqpoint{1.840148in}{0.874630in}}%
\pgfpathlineto{\pgfqpoint{1.842198in}{0.776134in}}%
\pgfpathlineto{\pgfqpoint{1.842608in}{0.810288in}}%
\pgfpathlineto{\pgfqpoint{1.845068in}{0.920424in}}%
\pgfpathlineto{\pgfqpoint{1.845888in}{0.898854in}}%
\pgfpathlineto{\pgfqpoint{1.846298in}{0.877196in}}%
\pgfpathlineto{\pgfqpoint{1.846708in}{0.877931in}}%
\pgfpathlineto{\pgfqpoint{1.849168in}{1.154330in}}%
\pgfpathlineto{\pgfqpoint{1.849578in}{1.163772in}}%
\pgfpathlineto{\pgfqpoint{1.849988in}{1.161672in}}%
\pgfpathlineto{\pgfqpoint{1.851218in}{1.089694in}}%
\pgfpathlineto{\pgfqpoint{1.853268in}{0.842605in}}%
\pgfpathlineto{\pgfqpoint{1.854088in}{0.872063in}}%
\pgfpathlineto{\pgfqpoint{1.855318in}{0.886653in}}%
\pgfpathlineto{\pgfqpoint{1.856138in}{0.871590in}}%
\pgfpathlineto{\pgfqpoint{1.857778in}{0.773529in}}%
\pgfpathlineto{\pgfqpoint{1.858598in}{0.811520in}}%
\pgfpathlineto{\pgfqpoint{1.860648in}{0.915038in}}%
\pgfpathlineto{\pgfqpoint{1.861058in}{0.913861in}}%
\pgfpathlineto{\pgfqpoint{1.862288in}{0.862801in}}%
\pgfpathlineto{\pgfqpoint{1.865568in}{1.153514in}}%
\pgfpathlineto{\pgfqpoint{1.865978in}{1.143754in}}%
\pgfpathlineto{\pgfqpoint{1.867208in}{1.048636in}}%
\pgfpathlineto{\pgfqpoint{1.868848in}{0.838007in}}%
\pgfpathlineto{\pgfqpoint{1.869668in}{0.868204in}}%
\pgfpathlineto{\pgfqpoint{1.870898in}{0.882519in}}%
\pgfpathlineto{\pgfqpoint{1.871718in}{0.866034in}}%
\pgfpathlineto{\pgfqpoint{1.873358in}{0.765225in}}%
\pgfpathlineto{\pgfqpoint{1.873768in}{0.785332in}}%
\pgfpathlineto{\pgfqpoint{1.876228in}{0.910523in}}%
\pgfpathlineto{\pgfqpoint{1.877048in}{0.891475in}}%
\pgfpathlineto{\pgfqpoint{1.877458in}{0.870126in}}%
\pgfpathlineto{\pgfqpoint{1.877868in}{0.870207in}}%
\pgfpathlineto{\pgfqpoint{1.880328in}{1.139685in}}%
\pgfpathlineto{\pgfqpoint{1.880738in}{1.144429in}}%
\pgfpathlineto{\pgfqpoint{1.881148in}{1.136733in}}%
\pgfpathlineto{\pgfqpoint{1.882378in}{1.044469in}}%
\pgfpathlineto{\pgfqpoint{1.884018in}{0.836181in}}%
\pgfpathlineto{\pgfqpoint{1.884838in}{0.866109in}}%
\pgfpathlineto{\pgfqpoint{1.885658in}{0.879169in}}%
\pgfpathlineto{\pgfqpoint{1.886068in}{0.878103in}}%
\pgfpathlineto{\pgfqpoint{1.886888in}{0.858519in}}%
\pgfpathlineto{\pgfqpoint{1.888528in}{0.763801in}}%
\pgfpathlineto{\pgfqpoint{1.888938in}{0.797692in}}%
\pgfpathlineto{\pgfqpoint{1.890988in}{0.905418in}}%
\pgfpathlineto{\pgfqpoint{1.891398in}{0.903715in}}%
\pgfpathlineto{\pgfqpoint{1.892628in}{0.859004in}}%
\pgfpathlineto{\pgfqpoint{1.895498in}{1.135451in}}%
\pgfpathlineto{\pgfqpoint{1.896318in}{1.109725in}}%
\pgfpathlineto{\pgfqpoint{1.897958in}{0.926125in}}%
\pgfpathlineto{\pgfqpoint{1.898778in}{0.835888in}}%
\pgfpathlineto{\pgfqpoint{1.899188in}{0.852380in}}%
\pgfpathlineto{\pgfqpoint{1.900418in}{0.875874in}}%
\pgfpathlineto{\pgfqpoint{1.900828in}{0.873247in}}%
\pgfpathlineto{\pgfqpoint{1.902058in}{0.827910in}}%
\pgfpathlineto{\pgfqpoint{1.902878in}{0.766283in}}%
\pgfpathlineto{\pgfqpoint{1.903288in}{0.777058in}}%
\pgfpathlineto{\pgfqpoint{1.905748in}{0.900959in}}%
\pgfpathlineto{\pgfqpoint{1.906568in}{0.877074in}}%
\pgfpathlineto{\pgfqpoint{1.906978in}{0.852977in}}%
\pgfpathlineto{\pgfqpoint{1.907388in}{0.893229in}}%
\pgfpathlineto{\pgfqpoint{1.909848in}{1.126845in}}%
\pgfpathlineto{\pgfqpoint{1.910668in}{1.102340in}}%
\pgfpathlineto{\pgfqpoint{1.912308in}{0.916023in}}%
\pgfpathlineto{\pgfqpoint{1.913128in}{0.835949in}}%
\pgfpathlineto{\pgfqpoint{1.913538in}{0.852005in}}%
\pgfpathlineto{\pgfqpoint{1.914768in}{0.872376in}}%
\pgfpathlineto{\pgfqpoint{1.915178in}{0.868000in}}%
\pgfpathlineto{\pgfqpoint{1.916408in}{0.815503in}}%
\pgfpathlineto{\pgfqpoint{1.917228in}{0.761481in}}%
\pgfpathlineto{\pgfqpoint{1.917638in}{0.791563in}}%
\pgfpathlineto{\pgfqpoint{1.919688in}{0.897183in}}%
\pgfpathlineto{\pgfqpoint{1.920508in}{0.879745in}}%
\pgfpathlineto{\pgfqpoint{1.920918in}{0.857910in}}%
\pgfpathlineto{\pgfqpoint{1.921328in}{0.869259in}}%
\pgfpathlineto{\pgfqpoint{1.923788in}{1.118520in}}%
\pgfpathlineto{\pgfqpoint{1.924198in}{1.114197in}}%
\pgfpathlineto{\pgfqpoint{1.925428in}{1.022886in}}%
\pgfpathlineto{\pgfqpoint{1.927068in}{0.835334in}}%
\pgfpathlineto{\pgfqpoint{1.927478in}{0.851071in}}%
\pgfpathlineto{\pgfqpoint{1.928708in}{0.868741in}}%
\pgfpathlineto{\pgfqpoint{1.929528in}{0.850080in}}%
\pgfpathlineto{\pgfqpoint{1.931168in}{0.768223in}}%
\pgfpathlineto{\pgfqpoint{1.931578in}{0.803776in}}%
\pgfpathlineto{\pgfqpoint{1.933628in}{0.892200in}}%
\pgfpathlineto{\pgfqpoint{1.934858in}{0.851497in}}%
\pgfpathlineto{\pgfqpoint{1.937318in}{1.109987in}}%
\pgfpathlineto{\pgfqpoint{1.937728in}{1.108381in}}%
\pgfpathlineto{\pgfqpoint{1.938958in}{1.021550in}}%
\pgfpathlineto{\pgfqpoint{1.940598in}{0.833136in}}%
\pgfpathlineto{\pgfqpoint{1.941008in}{0.848929in}}%
\pgfpathlineto{\pgfqpoint{1.941828in}{0.865675in}}%
\pgfpathlineto{\pgfqpoint{1.942238in}{0.865322in}}%
\pgfpathlineto{\pgfqpoint{1.943058in}{0.844337in}}%
\pgfpathlineto{\pgfqpoint{1.944288in}{0.760093in}}%
\pgfpathlineto{\pgfqpoint{1.944698in}{0.777010in}}%
\pgfpathlineto{\pgfqpoint{1.946748in}{0.889407in}}%
\pgfpathlineto{\pgfqpoint{1.947568in}{0.871696in}}%
\pgfpathlineto{\pgfqpoint{1.947978in}{0.848952in}}%
\pgfpathlineto{\pgfqpoint{1.948388in}{0.874031in}}%
\pgfpathlineto{\pgfqpoint{1.950848in}{1.103199in}}%
\pgfpathlineto{\pgfqpoint{1.951668in}{1.065769in}}%
\pgfpathlineto{\pgfqpoint{1.953718in}{0.828428in}}%
\pgfpathlineto{\pgfqpoint{1.954538in}{0.856564in}}%
\pgfpathlineto{\pgfqpoint{1.954948in}{0.862660in}}%
\pgfpathlineto{\pgfqpoint{1.955358in}{0.862523in}}%
\pgfpathlineto{\pgfqpoint{1.956178in}{0.841246in}}%
\pgfpathlineto{\pgfqpoint{1.957408in}{0.758375in}}%
\pgfpathlineto{\pgfqpoint{1.957818in}{0.780189in}}%
\pgfpathlineto{\pgfqpoint{1.959868in}{0.885823in}}%
\pgfpathlineto{\pgfqpoint{1.960688in}{0.862412in}}%
\pgfpathlineto{\pgfqpoint{1.961098in}{0.842697in}}%
\pgfpathlineto{\pgfqpoint{1.963558in}{1.096460in}}%
\pgfpathlineto{\pgfqpoint{1.963968in}{1.092227in}}%
\pgfpathlineto{\pgfqpoint{1.965198in}{0.992949in}}%
\pgfpathlineto{\pgfqpoint{1.966428in}{0.820513in}}%
\pgfpathlineto{\pgfqpoint{1.967248in}{0.851313in}}%
\pgfpathlineto{\pgfqpoint{1.968068in}{0.860371in}}%
\pgfpathlineto{\pgfqpoint{1.968478in}{0.854781in}}%
\pgfpathlineto{\pgfqpoint{1.969708in}{0.792503in}}%
\pgfpathlineto{\pgfqpoint{1.970118in}{0.757394in}}%
\pgfpathlineto{\pgfqpoint{1.970528in}{0.776187in}}%
\pgfpathlineto{\pgfqpoint{1.972578in}{0.882345in}}%
\pgfpathlineto{\pgfqpoint{1.973808in}{0.843461in}}%
\pgfpathlineto{\pgfqpoint{1.976268in}{1.090245in}}%
\pgfpathlineto{\pgfqpoint{1.977088in}{1.057266in}}%
\pgfpathlineto{\pgfqpoint{1.979138in}{0.826722in}}%
\pgfpathlineto{\pgfqpoint{1.979958in}{0.853440in}}%
\pgfpathlineto{\pgfqpoint{1.980368in}{0.857939in}}%
\pgfpathlineto{\pgfqpoint{1.980778in}{0.855523in}}%
\pgfpathlineto{\pgfqpoint{1.982008in}{0.801824in}}%
\pgfpathlineto{\pgfqpoint{1.982828in}{0.763264in}}%
\pgfpathlineto{\pgfqpoint{1.984878in}{0.879313in}}%
\pgfpathlineto{\pgfqpoint{1.985698in}{0.857509in}}%
\pgfpathlineto{\pgfqpoint{1.986108in}{0.838353in}}%
\pgfpathlineto{\pgfqpoint{1.988568in}{1.083941in}}%
\pgfpathlineto{\pgfqpoint{1.988978in}{1.074491in}}%
\pgfpathlineto{\pgfqpoint{1.990208in}{0.956367in}}%
\pgfpathlineto{\pgfqpoint{1.991028in}{0.827055in}}%
\pgfpathlineto{\pgfqpoint{1.991848in}{0.843383in}}%
\pgfpathlineto{\pgfqpoint{1.992668in}{0.855484in}}%
\pgfpathlineto{\pgfqpoint{1.993078in}{0.850877in}}%
\pgfpathlineto{\pgfqpoint{1.994308in}{0.788628in}}%
\pgfpathlineto{\pgfqpoint{1.994718in}{0.755706in}}%
\pgfpathlineto{\pgfqpoint{1.995128in}{0.777633in}}%
\pgfpathlineto{\pgfqpoint{1.996768in}{0.874927in}}%
\pgfpathlineto{\pgfqpoint{1.997178in}{0.874665in}}%
\pgfpathlineto{\pgfqpoint{1.997998in}{0.840994in}}%
\pgfpathlineto{\pgfqpoint{1.998408in}{0.861903in}}%
\pgfpathlineto{\pgfqpoint{2.000458in}{1.077685in}}%
\pgfpathlineto{\pgfqpoint{2.000868in}{1.072423in}}%
\pgfpathlineto{\pgfqpoint{2.002098in}{0.962107in}}%
\pgfpathlineto{\pgfqpoint{2.003328in}{0.824540in}}%
\pgfpathlineto{\pgfqpoint{2.003738in}{0.840271in}}%
\pgfpathlineto{\pgfqpoint{2.004558in}{0.853104in}}%
\pgfpathlineto{\pgfqpoint{2.004968in}{0.848526in}}%
\pgfpathlineto{\pgfqpoint{2.006198in}{0.784808in}}%
\pgfpathlineto{\pgfqpoint{2.006608in}{0.754938in}}%
\pgfpathlineto{\pgfqpoint{2.007018in}{0.780587in}}%
\pgfpathlineto{\pgfqpoint{2.008658in}{0.872928in}}%
\pgfpathlineto{\pgfqpoint{2.009068in}{0.870064in}}%
\pgfpathlineto{\pgfqpoint{2.009888in}{0.831221in}}%
\pgfpathlineto{\pgfqpoint{2.012348in}{1.071851in}}%
\pgfpathlineto{\pgfqpoint{2.012758in}{1.059194in}}%
\pgfpathlineto{\pgfqpoint{2.013988in}{0.927674in}}%
\pgfpathlineto{\pgfqpoint{2.014808in}{0.814280in}}%
\pgfpathlineto{\pgfqpoint{2.015218in}{0.832437in}}%
\pgfpathlineto{\pgfqpoint{2.016038in}{0.850560in}}%
\pgfpathlineto{\pgfqpoint{2.016448in}{0.848706in}}%
\pgfpathlineto{\pgfqpoint{2.017678in}{0.792058in}}%
\pgfpathlineto{\pgfqpoint{2.018088in}{0.756620in}}%
\pgfpathlineto{\pgfqpoint{2.018498in}{0.770957in}}%
\pgfpathlineto{\pgfqpoint{2.020138in}{0.869643in}}%
\pgfpathlineto{\pgfqpoint{2.020548in}{0.868216in}}%
\pgfpathlineto{\pgfqpoint{2.021368in}{0.830718in}}%
\pgfpathlineto{\pgfqpoint{2.023828in}{1.066086in}}%
\pgfpathlineto{\pgfqpoint{2.024238in}{1.051890in}}%
\pgfpathlineto{\pgfqpoint{2.026288in}{0.816809in}}%
\pgfpathlineto{\pgfqpoint{2.027928in}{0.844356in}}%
\pgfpathlineto{\pgfqpoint{2.029158in}{0.778367in}}%
\pgfpathlineto{\pgfqpoint{2.029568in}{0.753444in}}%
\pgfpathlineto{\pgfqpoint{2.029978in}{0.785212in}}%
\pgfpathlineto{\pgfqpoint{2.031618in}{0.867799in}}%
\pgfpathlineto{\pgfqpoint{2.032438in}{0.840449in}}%
\pgfpathlineto{\pgfqpoint{2.032848in}{0.840694in}}%
\pgfpathlineto{\pgfqpoint{2.034898in}{1.061619in}}%
\pgfpathlineto{\pgfqpoint{2.035718in}{1.025874in}}%
\pgfpathlineto{\pgfqpoint{2.037358in}{0.811585in}}%
\pgfpathlineto{\pgfqpoint{2.038178in}{0.841801in}}%
\pgfpathlineto{\pgfqpoint{2.038588in}{0.846510in}}%
\pgfpathlineto{\pgfqpoint{2.038998in}{0.842981in}}%
\pgfpathlineto{\pgfqpoint{2.040228in}{0.777740in}}%
\pgfpathlineto{\pgfqpoint{2.040638in}{0.752760in}}%
\pgfpathlineto{\pgfqpoint{2.041048in}{0.784831in}}%
\pgfpathlineto{\pgfqpoint{2.042688in}{0.865008in}}%
\pgfpathlineto{\pgfqpoint{2.043508in}{0.834342in}}%
\pgfpathlineto{\pgfqpoint{2.043918in}{0.852381in}}%
\pgfpathlineto{\pgfqpoint{2.045968in}{1.056010in}}%
\pgfpathlineto{\pgfqpoint{2.046788in}{1.008844in}}%
\pgfpathlineto{\pgfqpoint{2.048428in}{0.818325in}}%
\pgfpathlineto{\pgfqpoint{2.049248in}{0.842986in}}%
\pgfpathlineto{\pgfqpoint{2.049658in}{0.843849in}}%
\pgfpathlineto{\pgfqpoint{2.050478in}{0.818214in}}%
\pgfpathlineto{\pgfqpoint{2.051298in}{0.755300in}}%
\pgfpathlineto{\pgfqpoint{2.051708in}{0.768588in}}%
\pgfpathlineto{\pgfqpoint{2.053348in}{0.862889in}}%
\pgfpathlineto{\pgfqpoint{2.053758in}{0.857311in}}%
\pgfpathlineto{\pgfqpoint{2.054578in}{0.834767in}}%
\pgfpathlineto{\pgfqpoint{2.056628in}{1.051753in}}%
\pgfpathlineto{\pgfqpoint{2.057448in}{1.008662in}}%
\pgfpathlineto{\pgfqpoint{2.059088in}{0.816522in}}%
\pgfpathlineto{\pgfqpoint{2.059908in}{0.841240in}}%
\pgfpathlineto{\pgfqpoint{2.060318in}{0.841714in}}%
\pgfpathlineto{\pgfqpoint{2.061138in}{0.814336in}}%
\pgfpathlineto{\pgfqpoint{2.061958in}{0.751446in}}%
\pgfpathlineto{\pgfqpoint{2.062368in}{0.773742in}}%
\pgfpathlineto{\pgfqpoint{2.064008in}{0.860431in}}%
\pgfpathlineto{\pgfqpoint{2.064828in}{0.829837in}}%
\pgfpathlineto{\pgfqpoint{2.065238in}{0.853069in}}%
\pgfpathlineto{\pgfqpoint{2.066878in}{1.044852in}}%
\pgfpathlineto{\pgfqpoint{2.067288in}{1.044594in}}%
\pgfpathlineto{\pgfqpoint{2.068518in}{0.929334in}}%
\pgfpathlineto{\pgfqpoint{2.069338in}{0.807009in}}%
\pgfpathlineto{\pgfqpoint{2.070158in}{0.836597in}}%
\pgfpathlineto{\pgfqpoint{2.070568in}{0.840735in}}%
\pgfpathlineto{\pgfqpoint{2.070978in}{0.835706in}}%
\pgfpathlineto{\pgfqpoint{2.072617in}{0.760801in}}%
\pgfpathlineto{\pgfqpoint{2.073027in}{0.798696in}}%
\pgfpathlineto{\pgfqpoint{2.074257in}{0.858371in}}%
\pgfpathlineto{\pgfqpoint{2.074667in}{0.852177in}}%
\pgfpathlineto{\pgfqpoint{2.075077in}{0.832733in}}%
\pgfpathlineto{\pgfqpoint{2.075487in}{0.838255in}}%
\pgfpathlineto{\pgfqpoint{2.077537in}{1.040828in}}%
\pgfpathlineto{\pgfqpoint{2.078357in}{0.982045in}}%
\pgfpathlineto{\pgfqpoint{2.079587in}{0.804971in}}%
\pgfpathlineto{\pgfqpoint{2.080407in}{0.835585in}}%
\pgfpathlineto{\pgfqpoint{2.080817in}{0.838864in}}%
\pgfpathlineto{\pgfqpoint{2.081637in}{0.815977in}}%
\pgfpathlineto{\pgfqpoint{2.082457in}{0.752402in}}%
\pgfpathlineto{\pgfqpoint{2.082867in}{0.768333in}}%
\pgfpathlineto{\pgfqpoint{2.084507in}{0.855875in}}%
\pgfpathlineto{\pgfqpoint{2.085327in}{0.821888in}}%
\pgfpathlineto{\pgfqpoint{2.087377in}{1.038595in}}%
\pgfpathlineto{\pgfqpoint{2.087787in}{1.030226in}}%
\pgfpathlineto{\pgfqpoint{2.089017in}{0.889004in}}%
\pgfpathlineto{\pgfqpoint{2.089837in}{0.814965in}}%
\pgfpathlineto{\pgfqpoint{2.090247in}{0.830043in}}%
\pgfpathlineto{\pgfqpoint{2.090657in}{0.836948in}}%
\pgfpathlineto{\pgfqpoint{2.091067in}{0.834426in}}%
\pgfpathlineto{\pgfqpoint{2.092707in}{0.755747in}}%
\pgfpathlineto{\pgfqpoint{2.093117in}{0.794552in}}%
\pgfpathlineto{\pgfqpoint{2.094347in}{0.854234in}}%
\pgfpathlineto{\pgfqpoint{2.094757in}{0.846546in}}%
\pgfpathlineto{\pgfqpoint{2.095167in}{0.825107in}}%
\pgfpathlineto{\pgfqpoint{2.095577in}{0.850326in}}%
\pgfpathlineto{\pgfqpoint{2.097217in}{1.034333in}}%
\pgfpathlineto{\pgfqpoint{2.097627in}{1.027283in}}%
\pgfpathlineto{\pgfqpoint{2.098857in}{0.886181in}}%
\pgfpathlineto{\pgfqpoint{2.099267in}{0.814461in}}%
\pgfpathlineto{\pgfqpoint{2.100087in}{0.829236in}}%
\pgfpathlineto{\pgfqpoint{2.100497in}{0.835482in}}%
\pgfpathlineto{\pgfqpoint{2.100907in}{0.831910in}}%
\pgfpathlineto{\pgfqpoint{2.102137in}{0.756371in}}%
\pgfpathlineto{\pgfqpoint{2.102547in}{0.762604in}}%
\pgfpathlineto{\pgfqpoint{2.104187in}{0.851667in}}%
\pgfpathlineto{\pgfqpoint{2.105007in}{0.817755in}}%
\pgfpathlineto{\pgfqpoint{2.107057in}{1.030311in}}%
\pgfpathlineto{\pgfqpoint{2.107467in}{1.013950in}}%
\pgfpathlineto{\pgfqpoint{2.109107in}{0.803231in}}%
\pgfpathlineto{\pgfqpoint{2.110337in}{0.833158in}}%
\pgfpathlineto{\pgfqpoint{2.111157in}{0.802149in}}%
\pgfpathlineto{\pgfqpoint{2.111977in}{0.748721in}}%
\pgfpathlineto{\pgfqpoint{2.112387in}{0.787473in}}%
\pgfpathlineto{\pgfqpoint{2.113617in}{0.850417in}}%
\pgfpathlineto{\pgfqpoint{2.114027in}{0.842508in}}%
\pgfpathlineto{\pgfqpoint{2.114437in}{0.820173in}}%
\pgfpathlineto{\pgfqpoint{2.114847in}{0.854853in}}%
\pgfpathlineto{\pgfqpoint{2.116487in}{1.027090in}}%
\pgfpathlineto{\pgfqpoint{2.117307in}{0.977775in}}%
\pgfpathlineto{\pgfqpoint{2.118537in}{0.801091in}}%
\pgfpathlineto{\pgfqpoint{2.119357in}{0.830725in}}%
\pgfpathlineto{\pgfqpoint{2.119767in}{0.831570in}}%
\pgfpathlineto{\pgfqpoint{2.120587in}{0.799429in}}%
\pgfpathlineto{\pgfqpoint{2.121407in}{0.750591in}}%
\pgfpathlineto{\pgfqpoint{2.123047in}{0.848374in}}%
\pgfpathlineto{\pgfqpoint{2.123867in}{0.814725in}}%
\pgfpathlineto{\pgfqpoint{2.125917in}{1.021920in}}%
\pgfpathlineto{\pgfqpoint{2.126327in}{1.001030in}}%
\pgfpathlineto{\pgfqpoint{2.127967in}{0.808367in}}%
\pgfpathlineto{\pgfqpoint{2.128787in}{0.830928in}}%
\pgfpathlineto{\pgfqpoint{2.129197in}{0.826878in}}%
\pgfpathlineto{\pgfqpoint{2.130427in}{0.747854in}}%
\pgfpathlineto{\pgfqpoint{2.130837in}{0.771757in}}%
\pgfpathlineto{\pgfqpoint{2.132067in}{0.846476in}}%
\pgfpathlineto{\pgfqpoint{2.132477in}{0.841987in}}%
\pgfpathlineto{\pgfqpoint{2.132887in}{0.821881in}}%
\pgfpathlineto{\pgfqpoint{2.133297in}{0.840909in}}%
\pgfpathlineto{\pgfqpoint{2.134937in}{1.020083in}}%
\pgfpathlineto{\pgfqpoint{2.135757in}{0.967273in}}%
\pgfpathlineto{\pgfqpoint{2.136987in}{0.803475in}}%
\pgfpathlineto{\pgfqpoint{2.137807in}{0.829199in}}%
\pgfpathlineto{\pgfqpoint{2.138217in}{0.826619in}}%
\pgfpathlineto{\pgfqpoint{2.139447in}{0.747357in}}%
\pgfpathlineto{\pgfqpoint{2.139857in}{0.768944in}}%
\pgfpathlineto{\pgfqpoint{2.141087in}{0.844857in}}%
\pgfpathlineto{\pgfqpoint{2.141497in}{0.840062in}}%
\pgfpathlineto{\pgfqpoint{2.141907in}{0.819303in}}%
\pgfpathlineto{\pgfqpoint{2.142317in}{0.844475in}}%
\pgfpathlineto{\pgfqpoint{2.143957in}{1.016264in}}%
\pgfpathlineto{\pgfqpoint{2.144777in}{0.955827in}}%
\pgfpathlineto{\pgfqpoint{2.146007in}{0.807077in}}%
\pgfpathlineto{\pgfqpoint{2.146417in}{0.822640in}}%
\pgfpathlineto{\pgfqpoint{2.146827in}{0.828275in}}%
\pgfpathlineto{\pgfqpoint{2.147237in}{0.822392in}}%
\pgfpathlineto{\pgfqpoint{2.148467in}{0.747114in}}%
\pgfpathlineto{\pgfqpoint{2.148877in}{0.782351in}}%
\pgfpathlineto{\pgfqpoint{2.150107in}{0.843334in}}%
\pgfpathlineto{\pgfqpoint{2.150517in}{0.831416in}}%
\pgfpathlineto{\pgfqpoint{2.150927in}{0.815637in}}%
\pgfpathlineto{\pgfqpoint{2.152567in}{1.012955in}}%
\pgfpathlineto{\pgfqpoint{2.152977in}{1.006296in}}%
\pgfpathlineto{\pgfqpoint{2.157077in}{0.746664in}}%
\pgfpathlineto{\pgfqpoint{2.158717in}{0.842150in}}%
\pgfpathlineto{\pgfqpoint{2.159127in}{0.833373in}}%
\pgfpathlineto{\pgfqpoint{2.159537in}{0.809854in}}%
\pgfpathlineto{\pgfqpoint{2.161177in}{1.009223in}}%
\pgfpathlineto{\pgfqpoint{2.161587in}{1.004725in}}%
\pgfpathlineto{\pgfqpoint{2.162817in}{0.847343in}}%
\pgfpathlineto{\pgfqpoint{2.163227in}{0.796806in}}%
\pgfpathlineto{\pgfqpoint{2.164047in}{0.825151in}}%
\pgfpathlineto{\pgfqpoint{2.164457in}{0.822697in}}%
\pgfpathlineto{\pgfqpoint{2.165687in}{0.746392in}}%
\pgfpathlineto{\pgfqpoint{2.166097in}{0.775687in}}%
\pgfpathlineto{\pgfqpoint{2.167327in}{0.840416in}}%
\pgfpathlineto{\pgfqpoint{2.167737in}{0.828300in}}%
\pgfpathlineto{\pgfqpoint{2.168147in}{0.815267in}}%
\pgfpathlineto{\pgfqpoint{2.169787in}{1.007807in}}%
\pgfpathlineto{\pgfqpoint{2.170197in}{0.995293in}}%
\pgfpathlineto{\pgfqpoint{2.171837in}{0.803974in}}%
\pgfpathlineto{\pgfqpoint{2.173067in}{0.816545in}}%
\pgfpathlineto{\pgfqpoint{2.174297in}{0.753284in}}%
\pgfpathlineto{\pgfqpoint{2.175527in}{0.838518in}}%
\pgfpathlineto{\pgfqpoint{2.175937in}{0.834523in}}%
\pgfpathlineto{\pgfqpoint{2.176347in}{0.812848in}}%
\pgfpathlineto{\pgfqpoint{2.176757in}{0.848788in}}%
\pgfpathlineto{\pgfqpoint{2.177987in}{1.002738in}}%
\pgfpathlineto{\pgfqpoint{2.178397in}{0.999914in}}%
\pgfpathlineto{\pgfqpoint{2.182497in}{0.745583in}}%
\pgfpathlineto{\pgfqpoint{2.183727in}{0.836460in}}%
\pgfpathlineto{\pgfqpoint{2.184137in}{0.834395in}}%
\pgfpathlineto{\pgfqpoint{2.184547in}{0.813985in}}%
\pgfpathlineto{\pgfqpoint{2.184957in}{0.842220in}}%
\pgfpathlineto{\pgfqpoint{2.186187in}{0.999963in}}%
\pgfpathlineto{\pgfqpoint{2.186597in}{0.997151in}}%
\pgfpathlineto{\pgfqpoint{2.190697in}{0.753001in}}%
\pgfpathlineto{\pgfqpoint{2.191927in}{0.836415in}}%
\pgfpathlineto{\pgfqpoint{2.192337in}{0.829670in}}%
\pgfpathlineto{\pgfqpoint{2.192747in}{0.805927in}}%
\pgfpathlineto{\pgfqpoint{2.194387in}{0.999728in}}%
\pgfpathlineto{\pgfqpoint{2.194797in}{0.986779in}}%
\pgfpathlineto{\pgfqpoint{2.196027in}{0.804624in}}%
\pgfpathlineto{\pgfqpoint{2.197667in}{0.806441in}}%
\pgfpathlineto{\pgfqpoint{2.198487in}{0.745143in}}%
\pgfpathlineto{\pgfqpoint{2.198897in}{0.774849in}}%
\pgfpathlineto{\pgfqpoint{2.200127in}{0.833682in}}%
\pgfpathlineto{\pgfqpoint{2.200537in}{0.815799in}}%
\pgfpathlineto{\pgfqpoint{2.200947in}{0.830070in}}%
\pgfpathlineto{\pgfqpoint{2.202177in}{0.994835in}}%
\pgfpathlineto{\pgfqpoint{2.202587in}{0.991767in}}%
\pgfpathlineto{\pgfqpoint{2.206277in}{0.744662in}}%
\pgfpathlineto{\pgfqpoint{2.209967in}{0.992230in}}%
\pgfpathlineto{\pgfqpoint{2.210787in}{0.956116in}}%
\pgfpathlineto{\pgfqpoint{2.212017in}{0.800251in}}%
\pgfpathlineto{\pgfqpoint{2.212427in}{0.815742in}}%
\pgfpathlineto{\pgfqpoint{2.212837in}{0.818454in}}%
\pgfpathlineto{\pgfqpoint{2.213657in}{0.778603in}}%
\pgfpathlineto{\pgfqpoint{2.214067in}{0.744609in}}%
\pgfpathlineto{\pgfqpoint{2.214477in}{0.774071in}}%
\pgfpathlineto{\pgfqpoint{2.215297in}{0.830554in}}%
\pgfpathlineto{\pgfqpoint{2.215707in}{0.830297in}}%
\pgfpathlineto{\pgfqpoint{2.216117in}{0.809673in}}%
\pgfpathlineto{\pgfqpoint{2.216527in}{0.843668in}}%
\pgfpathlineto{\pgfqpoint{2.217757in}{0.992458in}}%
\pgfpathlineto{\pgfqpoint{2.218167in}{0.980027in}}%
\pgfpathlineto{\pgfqpoint{2.221037in}{0.795524in}}%
\pgfpathlineto{\pgfqpoint{2.221857in}{0.749849in}}%
\pgfpathlineto{\pgfqpoint{2.223087in}{0.831990in}}%
\pgfpathlineto{\pgfqpoint{2.223497in}{0.820743in}}%
\pgfpathlineto{\pgfqpoint{2.223907in}{0.812563in}}%
\pgfpathlineto{\pgfqpoint{2.225547in}{0.986539in}}%
\pgfpathlineto{\pgfqpoint{2.229237in}{0.744165in}}%
\pgfpathlineto{\pgfqpoint{2.230467in}{0.830631in}}%
\pgfpathlineto{\pgfqpoint{2.230877in}{0.823278in}}%
\pgfpathlineto{\pgfqpoint{2.231287in}{0.807046in}}%
\pgfpathlineto{\pgfqpoint{2.232927in}{0.985889in}}%
\pgfpathlineto{\pgfqpoint{2.233747in}{0.897528in}}%
\pgfpathlineto{\pgfqpoint{2.234567in}{0.798156in}}%
\pgfpathlineto{\pgfqpoint{2.234977in}{0.813617in}}%
\pgfpathlineto{\pgfqpoint{2.235387in}{0.815102in}}%
\pgfpathlineto{\pgfqpoint{2.236617in}{0.743917in}}%
\pgfpathlineto{\pgfqpoint{2.237027in}{0.784656in}}%
\pgfpathlineto{\pgfqpoint{2.237847in}{0.829887in}}%
\pgfpathlineto{\pgfqpoint{2.238257in}{0.820179in}}%
\pgfpathlineto{\pgfqpoint{2.238667in}{0.810392in}}%
\pgfpathlineto{\pgfqpoint{2.239897in}{0.983029in}}%
\pgfpathlineto{\pgfqpoint{2.240307in}{0.980813in}}%
\pgfpathlineto{\pgfqpoint{2.243587in}{0.755121in}}%
\pgfpathlineto{\pgfqpoint{2.243997in}{0.755322in}}%
\pgfpathlineto{\pgfqpoint{2.245227in}{0.827870in}}%
\pgfpathlineto{\pgfqpoint{2.245637in}{0.809538in}}%
\pgfpathlineto{\pgfqpoint{2.246457in}{0.915526in}}%
\pgfpathlineto{\pgfqpoint{2.247277in}{0.984443in}}%
\pgfpathlineto{\pgfqpoint{2.247687in}{0.965829in}}%
\pgfpathlineto{\pgfqpoint{2.248917in}{0.791583in}}%
\pgfpathlineto{\pgfqpoint{2.249737in}{0.814254in}}%
\pgfpathlineto{\pgfqpoint{2.250967in}{0.743612in}}%
\pgfpathlineto{\pgfqpoint{2.251377in}{0.781975in}}%
\pgfpathlineto{\pgfqpoint{2.252197in}{0.828043in}}%
\pgfpathlineto{\pgfqpoint{2.252607in}{0.817153in}}%
\pgfpathlineto{\pgfqpoint{2.253017in}{0.811976in}}%
\pgfpathlineto{\pgfqpoint{2.254247in}{0.981524in}}%
\pgfpathlineto{\pgfqpoint{2.254657in}{0.972109in}}%
\pgfpathlineto{\pgfqpoint{2.257527in}{0.776519in}}%
\pgfpathlineto{\pgfqpoint{2.257937in}{0.743450in}}%
\pgfpathlineto{\pgfqpoint{2.258347in}{0.776618in}}%
\pgfpathlineto{\pgfqpoint{2.259167in}{0.826999in}}%
\pgfpathlineto{\pgfqpoint{2.259577in}{0.817790in}}%
\pgfpathlineto{\pgfqpoint{2.259987in}{0.809931in}}%
\pgfpathlineto{\pgfqpoint{2.261217in}{0.979708in}}%
\pgfpathlineto{\pgfqpoint{2.261627in}{0.970162in}}%
\pgfpathlineto{\pgfqpoint{2.264497in}{0.771586in}}%
\pgfpathlineto{\pgfqpoint{2.264907in}{0.743301in}}%
\pgfpathlineto{\pgfqpoint{2.265317in}{0.782275in}}%
\pgfpathlineto{\pgfqpoint{2.266137in}{0.826302in}}%
\pgfpathlineto{\pgfqpoint{2.266547in}{0.812523in}}%
\pgfpathlineto{\pgfqpoint{2.266957in}{0.817825in}}%
\pgfpathlineto{\pgfqpoint{2.268187in}{0.979156in}}%
\pgfpathlineto{\pgfqpoint{2.268597in}{0.959703in}}%
\pgfpathlineto{\pgfqpoint{2.269827in}{0.793723in}}%
\pgfpathlineto{\pgfqpoint{2.270647in}{0.810397in}}%
\pgfpathlineto{\pgfqpoint{2.271877in}{0.754594in}}%
\pgfpathlineto{\pgfqpoint{2.273107in}{0.822446in}}%
\pgfpathlineto{\pgfqpoint{2.273517in}{0.798191in}}%
\pgfpathlineto{\pgfqpoint{2.274747in}{0.972898in}}%
\pgfpathlineto{\pgfqpoint{2.275157in}{0.972257in}}%
\pgfpathlineto{\pgfqpoint{2.278437in}{0.743031in}}%
\pgfpathlineto{\pgfqpoint{2.279667in}{0.824399in}}%
\pgfpathlineto{\pgfqpoint{2.280077in}{0.807184in}}%
\pgfpathlineto{\pgfqpoint{2.280897in}{0.916004in}}%
\pgfpathlineto{\pgfqpoint{2.281717in}{0.974532in}}%
\pgfpathlineto{\pgfqpoint{2.282127in}{0.944091in}}%
\pgfpathlineto{\pgfqpoint{2.282947in}{0.798062in}}%
\pgfpathlineto{\pgfqpoint{2.283767in}{0.810957in}}%
\pgfpathlineto{\pgfqpoint{2.284587in}{0.775731in}}%
\pgfpathlineto{\pgfqpoint{2.284997in}{0.742947in}}%
\pgfpathlineto{\pgfqpoint{2.285407in}{0.778333in}}%
\pgfpathlineto{\pgfqpoint{2.288277in}{0.973123in}}%
\pgfpathlineto{\pgfqpoint{2.289097in}{0.875840in}}%
\pgfpathlineto{\pgfqpoint{2.289507in}{0.796380in}}%
\pgfpathlineto{\pgfqpoint{2.290327in}{0.810412in}}%
\pgfpathlineto{\pgfqpoint{2.291147in}{0.771475in}}%
\pgfpathlineto{\pgfqpoint{2.291557in}{0.742836in}}%
\pgfpathlineto{\pgfqpoint{2.291967in}{0.783332in}}%
\pgfpathlineto{\pgfqpoint{2.294837in}{0.968121in}}%
\pgfpathlineto{\pgfqpoint{2.298117in}{0.752905in}}%
\pgfpathlineto{\pgfqpoint{2.298937in}{0.820716in}}%
\pgfpathlineto{\pgfqpoint{2.299347in}{0.817066in}}%
\pgfpathlineto{\pgfqpoint{2.299757in}{0.804656in}}%
\pgfpathlineto{\pgfqpoint{2.300987in}{0.971742in}}%
\pgfpathlineto{\pgfqpoint{2.301397in}{0.952837in}}%
\pgfpathlineto{\pgfqpoint{2.302627in}{0.795841in}}%
\pgfpathlineto{\pgfqpoint{2.303447in}{0.802699in}}%
\pgfpathlineto{\pgfqpoint{2.304267in}{0.742707in}}%
\pgfpathlineto{\pgfqpoint{2.304677in}{0.779998in}}%
\pgfpathlineto{\pgfqpoint{2.307137in}{0.967241in}}%
\pgfpathlineto{\pgfqpoint{2.307547in}{0.962134in}}%
\pgfpathlineto{\pgfqpoint{2.310417in}{0.742348in}}%
\pgfpathlineto{\pgfqpoint{2.313287in}{0.963829in}}%
\pgfpathlineto{\pgfqpoint{2.313697in}{0.963443in}}%
\pgfpathlineto{\pgfqpoint{2.316567in}{0.742302in}}%
\pgfpathlineto{\pgfqpoint{2.319437in}{0.964768in}}%
\pgfpathlineto{\pgfqpoint{2.319847in}{0.959393in}}%
\pgfpathlineto{\pgfqpoint{2.322717in}{0.742573in}}%
\pgfpathlineto{\pgfqpoint{2.325587in}{0.967065in}}%
\pgfpathlineto{\pgfqpoint{2.325997in}{0.946713in}}%
\pgfpathlineto{\pgfqpoint{2.328047in}{0.794052in}}%
\pgfpathlineto{\pgfqpoint{2.328867in}{0.752882in}}%
\pgfpathlineto{\pgfqpoint{2.331737in}{0.961240in}}%
\pgfpathlineto{\pgfqpoint{2.334607in}{0.742545in}}%
\pgfpathlineto{\pgfqpoint{2.337477in}{0.964550in}}%
\pgfpathlineto{\pgfqpoint{2.340347in}{0.742380in}}%
\pgfpathlineto{\pgfqpoint{2.340757in}{0.775359in}}%
\pgfpathlineto{\pgfqpoint{2.343217in}{0.964213in}}%
\pgfpathlineto{\pgfqpoint{2.343627in}{0.935161in}}%
\pgfpathlineto{\pgfqpoint{2.344447in}{0.786557in}}%
\pgfpathlineto{\pgfqpoint{2.345267in}{0.803919in}}%
\pgfpathlineto{\pgfqpoint{2.346087in}{0.742409in}}%
\pgfpathlineto{\pgfqpoint{2.346497in}{0.776116in}}%
\pgfpathlineto{\pgfqpoint{2.348957in}{0.962704in}}%
\pgfpathlineto{\pgfqpoint{2.351827in}{0.742586in}}%
\pgfpathlineto{\pgfqpoint{2.354697in}{0.956920in}}%
\pgfpathlineto{\pgfqpoint{2.357157in}{0.756629in}}%
\pgfpathlineto{\pgfqpoint{2.357567in}{0.754970in}}%
\pgfpathlineto{\pgfqpoint{2.360027in}{0.961987in}}%
\pgfpathlineto{\pgfqpoint{2.360847in}{0.865766in}}%
\pgfpathlineto{\pgfqpoint{2.361257in}{0.787273in}}%
\pgfpathlineto{\pgfqpoint{2.362077in}{0.802244in}}%
\pgfpathlineto{\pgfqpoint{2.362897in}{0.742690in}}%
\pgfpathlineto{\pgfqpoint{2.363307in}{0.783838in}}%
\pgfpathlineto{\pgfqpoint{2.365357in}{0.956585in}}%
\pgfpathlineto{\pgfqpoint{2.365767in}{0.951763in}}%
\pgfpathlineto{\pgfqpoint{2.368227in}{0.742191in}}%
\pgfpathlineto{\pgfqpoint{2.371097in}{0.955975in}}%
\pgfpathlineto{\pgfqpoint{2.371507in}{0.905579in}}%
\pgfpathlineto{\pgfqpoint{2.372327in}{0.792783in}}%
\pgfpathlineto{\pgfqpoint{2.372737in}{0.804268in}}%
\pgfpathlineto{\pgfqpoint{2.373557in}{0.745595in}}%
\pgfpathlineto{\pgfqpoint{2.373967in}{0.767362in}}%
\pgfpathlineto{\pgfqpoint{2.376427in}{0.955432in}}%
\pgfpathlineto{\pgfqpoint{2.378887in}{0.742295in}}%
\pgfpathlineto{\pgfqpoint{2.381347in}{0.954250in}}%
\pgfpathlineto{\pgfqpoint{2.381757in}{0.950025in}}%
\pgfpathlineto{\pgfqpoint{2.384217in}{0.743016in}}%
\pgfpathlineto{\pgfqpoint{2.386677in}{0.959158in}}%
\pgfpathlineto{\pgfqpoint{2.387087in}{0.934742in}}%
\pgfpathlineto{\pgfqpoint{2.387907in}{0.781481in}}%
\pgfpathlineto{\pgfqpoint{2.388727in}{0.796516in}}%
\pgfpathlineto{\pgfqpoint{2.389547in}{0.753724in}}%
\pgfpathlineto{\pgfqpoint{2.392007in}{0.954265in}}%
\pgfpathlineto{\pgfqpoint{2.394467in}{0.743242in}}%
\pgfpathlineto{\pgfqpoint{2.396927in}{0.958765in}}%
\pgfpathlineto{\pgfqpoint{2.397337in}{0.924537in}}%
\pgfpathlineto{\pgfqpoint{2.398157in}{0.787469in}}%
\pgfpathlineto{\pgfqpoint{2.398977in}{0.789530in}}%
\pgfpathlineto{\pgfqpoint{2.399387in}{0.743708in}}%
\pgfpathlineto{\pgfqpoint{2.399797in}{0.771529in}}%
\pgfpathlineto{\pgfqpoint{2.401847in}{0.957740in}}%
\pgfpathlineto{\pgfqpoint{2.402257in}{0.935257in}}%
\pgfpathlineto{\pgfqpoint{2.403077in}{0.782569in}}%
\pgfpathlineto{\pgfqpoint{2.403897in}{0.793166in}}%
\pgfpathlineto{\pgfqpoint{2.404307in}{0.750854in}}%
\pgfpathlineto{\pgfqpoint{2.404717in}{0.765298in}}%
\pgfpathlineto{\pgfqpoint{2.406767in}{0.956990in}}%
\pgfpathlineto{\pgfqpoint{2.407177in}{0.936883in}}%
\pgfpathlineto{\pgfqpoint{2.409227in}{0.749142in}}%
\pgfpathlineto{\pgfqpoint{2.410867in}{0.813909in}}%
\pgfpathlineto{\pgfqpoint{2.411687in}{0.957925in}}%
\pgfpathlineto{\pgfqpoint{2.412097in}{0.930170in}}%
\pgfpathlineto{\pgfqpoint{2.412917in}{0.786367in}}%
\pgfpathlineto{\pgfqpoint{2.413737in}{0.787687in}}%
\pgfpathlineto{\pgfqpoint{2.414147in}{0.743371in}}%
\pgfpathlineto{\pgfqpoint{2.414557in}{0.778175in}}%
\pgfpathlineto{\pgfqpoint{2.416607in}{0.957262in}}%
\pgfpathlineto{\pgfqpoint{2.417017in}{0.911821in}}%
\pgfpathlineto{\pgfqpoint{2.418657in}{0.775549in}}%
\pgfpathlineto{\pgfqpoint{2.419067in}{0.744009in}}%
\pgfpathlineto{\pgfqpoint{2.419477in}{0.794675in}}%
\pgfpathlineto{\pgfqpoint{2.420707in}{0.882808in}}%
\pgfpathlineto{\pgfqpoint{2.421117in}{0.950773in}}%
\pgfpathlineto{\pgfqpoint{2.421527in}{0.946928in}}%
\pgfpathlineto{\pgfqpoint{2.423577in}{0.751297in}}%
\pgfpathlineto{\pgfqpoint{2.425217in}{0.827978in}}%
\pgfpathlineto{\pgfqpoint{2.426037in}{0.957832in}}%
\pgfpathlineto{\pgfqpoint{2.426447in}{0.914805in}}%
\pgfpathlineto{\pgfqpoint{2.428087in}{0.772861in}}%
\pgfpathlineto{\pgfqpoint{2.428497in}{0.744214in}}%
\pgfpathlineto{\pgfqpoint{2.430547in}{0.956160in}}%
\pgfpathlineto{\pgfqpoint{2.430957in}{0.935848in}}%
\pgfpathlineto{\pgfqpoint{2.432597in}{0.784198in}}%
\pgfpathlineto{\pgfqpoint{2.433007in}{0.744651in}}%
\pgfpathlineto{\pgfqpoint{2.433417in}{0.788387in}}%
\pgfpathlineto{\pgfqpoint{2.435057in}{0.951410in}}%
\pgfpathlineto{\pgfqpoint{2.435467in}{0.945125in}}%
\pgfpathlineto{\pgfqpoint{2.437517in}{0.744494in}}%
\pgfpathlineto{\pgfqpoint{2.439567in}{0.949501in}}%
\pgfpathlineto{\pgfqpoint{2.439977in}{0.947328in}}%
\pgfpathlineto{\pgfqpoint{2.442027in}{0.744794in}}%
\pgfpathlineto{\pgfqpoint{2.444077in}{0.952341in}}%
\pgfpathlineto{\pgfqpoint{2.444487in}{0.943689in}}%
\pgfpathlineto{\pgfqpoint{2.446537in}{0.745469in}}%
\pgfpathlineto{\pgfqpoint{2.448587in}{0.957602in}}%
\pgfpathlineto{\pgfqpoint{2.448997in}{0.931537in}}%
\pgfpathlineto{\pgfqpoint{2.450637in}{0.772404in}}%
\pgfpathlineto{\pgfqpoint{2.451047in}{0.748032in}}%
\pgfpathlineto{\pgfqpoint{2.453097in}{0.958430in}}%
\pgfpathlineto{\pgfqpoint{2.455147in}{0.749223in}}%
\pgfpathlineto{\pgfqpoint{2.455557in}{0.775342in}}%
\pgfpathlineto{\pgfqpoint{2.457197in}{0.952920in}}%
\pgfpathlineto{\pgfqpoint{2.457607in}{0.943521in}}%
\pgfpathlineto{\pgfqpoint{2.459247in}{0.777468in}}%
\pgfpathlineto{\pgfqpoint{2.459657in}{0.746144in}}%
\pgfpathlineto{\pgfqpoint{2.461707in}{0.958569in}}%
\pgfpathlineto{\pgfqpoint{2.463757in}{0.746056in}}%
\pgfpathlineto{\pgfqpoint{2.465807in}{0.960180in}}%
\pgfpathlineto{\pgfqpoint{2.466217in}{0.924581in}}%
\pgfpathlineto{\pgfqpoint{2.467857in}{0.755180in}}%
\pgfpathlineto{\pgfqpoint{2.468677in}{0.816144in}}%
\pgfpathlineto{\pgfqpoint{2.469087in}{0.797691in}}%
\pgfpathlineto{\pgfqpoint{2.469907in}{0.957311in}}%
\pgfpathlineto{\pgfqpoint{2.470317in}{0.937351in}}%
\pgfpathlineto{\pgfqpoint{2.471957in}{0.763366in}}%
\pgfpathlineto{\pgfqpoint{2.472367in}{0.764920in}}%
\pgfpathlineto{\pgfqpoint{2.474007in}{0.955840in}}%
\pgfpathlineto{\pgfqpoint{2.474417in}{0.941390in}}%
\pgfpathlineto{\pgfqpoint{2.476057in}{0.764149in}}%
\pgfpathlineto{\pgfqpoint{2.476467in}{0.765189in}}%
\pgfpathlineto{\pgfqpoint{2.478107in}{0.958028in}}%
\pgfpathlineto{\pgfqpoint{2.478517in}{0.938117in}}%
\pgfpathlineto{\pgfqpoint{2.480157in}{0.757504in}}%
\pgfpathlineto{\pgfqpoint{2.480977in}{0.817706in}}%
\pgfpathlineto{\pgfqpoint{2.481387in}{0.801592in}}%
\pgfpathlineto{\pgfqpoint{2.482207in}{0.962256in}}%
\pgfpathlineto{\pgfqpoint{2.482617in}{0.925551in}}%
\pgfpathlineto{\pgfqpoint{2.484257in}{0.747496in}}%
\pgfpathlineto{\pgfqpoint{2.486307in}{0.962770in}}%
\pgfpathlineto{\pgfqpoint{2.486717in}{0.898455in}}%
\pgfpathlineto{\pgfqpoint{2.488357in}{0.748850in}}%
\pgfpathlineto{\pgfqpoint{2.489997in}{0.952583in}}%
\pgfpathlineto{\pgfqpoint{2.490407in}{0.949704in}}%
\pgfpathlineto{\pgfqpoint{2.492047in}{0.759578in}}%
\pgfpathlineto{\pgfqpoint{2.492867in}{0.819428in}}%
\pgfpathlineto{\pgfqpoint{2.493277in}{0.808773in}}%
\pgfpathlineto{\pgfqpoint{2.494097in}{0.965423in}}%
\pgfpathlineto{\pgfqpoint{2.494507in}{0.910301in}}%
\pgfpathlineto{\pgfqpoint{2.496147in}{0.749732in}}%
\pgfpathlineto{\pgfqpoint{2.497787in}{0.958024in}}%
\pgfpathlineto{\pgfqpoint{2.498197in}{0.944951in}}%
\pgfpathlineto{\pgfqpoint{2.499837in}{0.748697in}}%
\pgfpathlineto{\pgfqpoint{2.501887in}{0.961213in}}%
\pgfpathlineto{\pgfqpoint{2.503527in}{0.764748in}}%
\pgfpathlineto{\pgfqpoint{2.503937in}{0.773670in}}%
\pgfpathlineto{\pgfqpoint{2.505577in}{0.967169in}}%
\pgfpathlineto{\pgfqpoint{2.505987in}{0.896997in}}%
\pgfpathlineto{\pgfqpoint{2.507217in}{0.774888in}}%
\pgfpathlineto{\pgfqpoint{2.507627in}{0.762707in}}%
\pgfpathlineto{\pgfqpoint{2.509267in}{0.969012in}}%
\pgfpathlineto{\pgfqpoint{2.509677in}{0.907545in}}%
\pgfpathlineto{\pgfqpoint{2.511317in}{0.759499in}}%
\pgfpathlineto{\pgfqpoint{2.512957in}{0.969984in}}%
\pgfpathlineto{\pgfqpoint{2.513367in}{0.906609in}}%
\pgfpathlineto{\pgfqpoint{2.514597in}{0.775752in}}%
\pgfpathlineto{\pgfqpoint{2.515007in}{0.764704in}}%
\pgfpathlineto{\pgfqpoint{2.516647in}{0.969879in}}%
\pgfpathlineto{\pgfqpoint{2.518287in}{0.766201in}}%
\pgfpathlineto{\pgfqpoint{2.518697in}{0.777977in}}%
\pgfpathlineto{\pgfqpoint{2.520337in}{0.964819in}}%
\pgfpathlineto{\pgfqpoint{2.521977in}{0.751376in}}%
\pgfpathlineto{\pgfqpoint{2.523617in}{0.963804in}}%
\pgfpathlineto{\pgfqpoint{2.524027in}{0.947347in}}%
\pgfpathlineto{\pgfqpoint{2.525667in}{0.753626in}}%
\pgfpathlineto{\pgfqpoint{2.527307in}{0.974254in}}%
\pgfpathlineto{\pgfqpoint{2.527717in}{0.907391in}}%
\pgfpathlineto{\pgfqpoint{2.528947in}{0.766761in}}%
\pgfpathlineto{\pgfqpoint{2.530177in}{0.832463in}}%
\pgfpathlineto{\pgfqpoint{2.530997in}{0.958580in}}%
\pgfpathlineto{\pgfqpoint{2.532637in}{0.754750in}}%
\pgfpathlineto{\pgfqpoint{2.534277in}{0.976194in}}%
\pgfpathlineto{\pgfqpoint{2.534687in}{0.900226in}}%
\pgfpathlineto{\pgfqpoint{2.535917in}{0.756683in}}%
\pgfpathlineto{\pgfqpoint{2.537557in}{0.971676in}}%
\pgfpathlineto{\pgfqpoint{2.537967in}{0.940642in}}%
\pgfpathlineto{\pgfqpoint{2.539607in}{0.770326in}}%
\pgfpathlineto{\pgfqpoint{2.541247in}{0.962582in}}%
\pgfpathlineto{\pgfqpoint{2.542887in}{0.756149in}}%
\pgfpathlineto{\pgfqpoint{2.544527in}{0.973482in}}%
\pgfpathlineto{\pgfqpoint{2.546167in}{0.756497in}}%
\pgfpathlineto{\pgfqpoint{2.547807in}{0.979601in}}%
\pgfpathlineto{\pgfqpoint{2.549447in}{0.755372in}}%
\pgfpathlineto{\pgfqpoint{2.551087in}{0.981357in}}%
\pgfpathlineto{\pgfqpoint{2.551497in}{0.916371in}}%
\pgfpathlineto{\pgfqpoint{2.552727in}{0.760149in}}%
\pgfpathlineto{\pgfqpoint{2.554367in}{0.978609in}}%
\pgfpathlineto{\pgfqpoint{2.554777in}{0.937896in}}%
\pgfpathlineto{\pgfqpoint{2.556007in}{0.774574in}}%
\pgfpathlineto{\pgfqpoint{2.556417in}{0.782193in}}%
\pgfpathlineto{\pgfqpoint{2.557647in}{0.971271in}}%
\pgfpathlineto{\pgfqpoint{2.558057in}{0.955912in}}%
\pgfpathlineto{\pgfqpoint{2.559697in}{0.766047in}}%
\pgfpathlineto{\pgfqpoint{2.561337in}{0.970085in}}%
\pgfpathlineto{\pgfqpoint{2.562977in}{0.757563in}}%
\pgfpathlineto{\pgfqpoint{2.564617in}{0.980123in}}%
\pgfpathlineto{\pgfqpoint{2.566257in}{0.757661in}}%
\pgfpathlineto{\pgfqpoint{2.567897in}{0.985779in}}%
\pgfpathlineto{\pgfqpoint{2.568307in}{0.900052in}}%
\pgfpathlineto{\pgfqpoint{2.569537in}{0.756234in}}%
\pgfpathlineto{\pgfqpoint{2.571177in}{0.986854in}}%
\pgfpathlineto{\pgfqpoint{2.571587in}{0.924971in}}%
\pgfpathlineto{\pgfqpoint{2.572817in}{0.764536in}}%
\pgfpathlineto{\pgfqpoint{2.574047in}{0.867463in}}%
\pgfpathlineto{\pgfqpoint{2.574457in}{0.983207in}}%
\pgfpathlineto{\pgfqpoint{2.574867in}{0.946546in}}%
\pgfpathlineto{\pgfqpoint{2.576097in}{0.778982in}}%
\pgfpathlineto{\pgfqpoint{2.576507in}{0.780739in}}%
\pgfpathlineto{\pgfqpoint{2.577737in}{0.974754in}}%
\pgfpathlineto{\pgfqpoint{2.578147in}{0.964392in}}%
\pgfpathlineto{\pgfqpoint{2.579787in}{0.763686in}}%
\pgfpathlineto{\pgfqpoint{2.581427in}{0.978163in}}%
\pgfpathlineto{\pgfqpoint{2.583067in}{0.759057in}}%
\pgfpathlineto{\pgfqpoint{2.584707in}{0.987557in}}%
\pgfpathlineto{\pgfqpoint{2.586347in}{0.758750in}}%
\pgfpathlineto{\pgfqpoint{2.587987in}{0.992321in}}%
\pgfpathlineto{\pgfqpoint{2.588397in}{0.910899in}}%
\pgfpathlineto{\pgfqpoint{2.589627in}{0.756879in}}%
\pgfpathlineto{\pgfqpoint{2.591267in}{0.992257in}}%
\pgfpathlineto{\pgfqpoint{2.591677in}{0.935792in}}%
\pgfpathlineto{\pgfqpoint{2.592907in}{0.770348in}}%
\pgfpathlineto{\pgfqpoint{2.594137in}{0.862646in}}%
\pgfpathlineto{\pgfqpoint{2.594547in}{0.987224in}}%
\pgfpathlineto{\pgfqpoint{2.594957in}{0.957086in}}%
\pgfpathlineto{\pgfqpoint{2.596597in}{0.777637in}}%
\pgfpathlineto{\pgfqpoint{2.597827in}{0.977148in}}%
\pgfpathlineto{\pgfqpoint{2.598237in}{0.974381in}}%
\pgfpathlineto{\pgfqpoint{2.599877in}{0.759670in}}%
\pgfpathlineto{\pgfqpoint{2.601517in}{0.987321in}}%
\pgfpathlineto{\pgfqpoint{2.603157in}{0.760528in}}%
\pgfpathlineto{\pgfqpoint{2.604797in}{0.995599in}}%
\pgfpathlineto{\pgfqpoint{2.606437in}{0.759644in}}%
\pgfpathlineto{\pgfqpoint{2.608077in}{0.998961in}}%
\pgfpathlineto{\pgfqpoint{2.608487in}{0.924344in}}%
\pgfpathlineto{\pgfqpoint{2.609717in}{0.761260in}}%
\pgfpathlineto{\pgfqpoint{2.611357in}{0.997212in}}%
\pgfpathlineto{\pgfqpoint{2.611767in}{0.948855in}}%
\pgfpathlineto{\pgfqpoint{2.612997in}{0.777584in}}%
\pgfpathlineto{\pgfqpoint{2.613407in}{0.790321in}}%
\pgfpathlineto{\pgfqpoint{2.614637in}{0.990222in}}%
\pgfpathlineto{\pgfqpoint{2.615047in}{0.969452in}}%
\pgfpathlineto{\pgfqpoint{2.616687in}{0.772539in}}%
\pgfpathlineto{\pgfqpoint{2.618327in}{0.985724in}}%
\pgfpathlineto{\pgfqpoint{2.619967in}{0.761686in}}%
\pgfpathlineto{\pgfqpoint{2.621607in}{0.997309in}}%
\pgfpathlineto{\pgfqpoint{2.623247in}{0.761810in}}%
\pgfpathlineto{\pgfqpoint{2.624887in}{1.003900in}}%
\pgfpathlineto{\pgfqpoint{2.625297in}{0.912880in}}%
\pgfpathlineto{\pgfqpoint{2.626527in}{0.760161in}}%
\pgfpathlineto{\pgfqpoint{2.628167in}{1.005247in}}%
\pgfpathlineto{\pgfqpoint{2.628577in}{0.940380in}}%
\pgfpathlineto{\pgfqpoint{2.629807in}{0.770424in}}%
\pgfpathlineto{\pgfqpoint{2.631037in}{0.877783in}}%
\pgfpathlineto{\pgfqpoint{2.631447in}{1.001168in}}%
\pgfpathlineto{\pgfqpoint{2.631857in}{0.964050in}}%
\pgfpathlineto{\pgfqpoint{2.633087in}{0.786152in}}%
\pgfpathlineto{\pgfqpoint{2.633497in}{0.784337in}}%
\pgfpathlineto{\pgfqpoint{2.634727in}{0.991552in}}%
\pgfpathlineto{\pgfqpoint{2.635137in}{0.983424in}}%
\pgfpathlineto{\pgfqpoint{2.636777in}{0.765037in}}%
\pgfpathlineto{\pgfqpoint{2.638417in}{0.998086in}}%
\pgfpathlineto{\pgfqpoint{2.640057in}{0.763499in}}%
\pgfpathlineto{\pgfqpoint{2.641697in}{1.007674in}}%
\pgfpathlineto{\pgfqpoint{2.643337in}{0.762666in}}%
\pgfpathlineto{\pgfqpoint{2.644977in}{1.011885in}}%
\pgfpathlineto{\pgfqpoint{2.645387in}{0.932332in}}%
\pgfpathlineto{\pgfqpoint{2.646617in}{0.763514in}}%
\pgfpathlineto{\pgfqpoint{2.648257in}{1.010486in}}%
\pgfpathlineto{\pgfqpoint{2.648667in}{0.958845in}}%
\pgfpathlineto{\pgfqpoint{2.649897in}{0.781107in}}%
\pgfpathlineto{\pgfqpoint{2.650717in}{0.844396in}}%
\pgfpathlineto{\pgfqpoint{2.651127in}{0.862957in}}%
\pgfpathlineto{\pgfqpoint{2.651537in}{1.003313in}}%
\pgfpathlineto{\pgfqpoint{2.651947in}{0.981081in}}%
\pgfpathlineto{\pgfqpoint{2.653587in}{0.775594in}}%
\pgfpathlineto{\pgfqpoint{2.655227in}{0.998566in}}%
\pgfpathlineto{\pgfqpoint{2.656867in}{0.764855in}}%
\pgfpathlineto{\pgfqpoint{2.658507in}{1.010887in}}%
\pgfpathlineto{\pgfqpoint{2.660147in}{0.764798in}}%
\pgfpathlineto{\pgfqpoint{2.661787in}{1.017687in}}%
\pgfpathlineto{\pgfqpoint{2.662197in}{0.925368in}}%
\pgfpathlineto{\pgfqpoint{2.663427in}{0.762780in}}%
\pgfpathlineto{\pgfqpoint{2.665067in}{1.018682in}}%
\pgfpathlineto{\pgfqpoint{2.665477in}{0.954495in}}%
\pgfpathlineto{\pgfqpoint{2.666707in}{0.776603in}}%
\pgfpathlineto{\pgfqpoint{2.667937in}{0.881676in}}%
\pgfpathlineto{\pgfqpoint{2.668347in}{1.013659in}}%
\pgfpathlineto{\pgfqpoint{2.668757in}{0.979341in}}%
\pgfpathlineto{\pgfqpoint{2.670397in}{0.784789in}}%
\pgfpathlineto{\pgfqpoint{2.671627in}{1.002492in}}%
\pgfpathlineto{\pgfqpoint{2.672037in}{0.999377in}}%
\pgfpathlineto{\pgfqpoint{2.673677in}{0.766041in}}%
\pgfpathlineto{\pgfqpoint{2.675317in}{1.014133in}}%
\pgfpathlineto{\pgfqpoint{2.676957in}{0.766680in}}%
\pgfpathlineto{\pgfqpoint{2.678597in}{1.023202in}}%
\pgfpathlineto{\pgfqpoint{2.680237in}{0.765176in}}%
\pgfpathlineto{\pgfqpoint{2.681877in}{1.026249in}}%
\pgfpathlineto{\pgfqpoint{2.682287in}{0.951642in}}%
\pgfpathlineto{\pgfqpoint{2.683517in}{0.773043in}}%
\pgfpathlineto{\pgfqpoint{2.685157in}{1.023016in}}%
\pgfpathlineto{\pgfqpoint{2.685567in}{0.978835in}}%
\pgfpathlineto{\pgfqpoint{2.686797in}{0.790966in}}%
\pgfpathlineto{\pgfqpoint{2.687207in}{0.792523in}}%
\pgfpathlineto{\pgfqpoint{2.688437in}{1.013332in}}%
\pgfpathlineto{\pgfqpoint{2.688847in}{1.001123in}}%
\pgfpathlineto{\pgfqpoint{2.690487in}{0.770838in}}%
\pgfpathlineto{\pgfqpoint{2.692127in}{1.017980in}}%
\pgfpathlineto{\pgfqpoint{2.693767in}{0.768441in}}%
\pgfpathlineto{\pgfqpoint{2.695407in}{1.028948in}}%
\pgfpathlineto{\pgfqpoint{2.697047in}{0.767333in}}%
\pgfpathlineto{\pgfqpoint{2.698687in}{1.033643in}}%
\pgfpathlineto{\pgfqpoint{2.699097in}{0.950910in}}%
\pgfpathlineto{\pgfqpoint{2.700327in}{0.770830in}}%
\pgfpathlineto{\pgfqpoint{2.701967in}{1.031764in}}%
\pgfpathlineto{\pgfqpoint{2.702377in}{0.980170in}}%
\pgfpathlineto{\pgfqpoint{2.703607in}{0.790101in}}%
\pgfpathlineto{\pgfqpoint{2.704017in}{0.798686in}}%
\pgfpathlineto{\pgfqpoint{2.705247in}{1.023099in}}%
\pgfpathlineto{\pgfqpoint{2.705657in}{1.004377in}}%
\pgfpathlineto{\pgfqpoint{2.707297in}{0.776407in}}%
\pgfpathlineto{\pgfqpoint{2.708936in}{1.022953in}}%
\pgfpathlineto{\pgfqpoint{2.710576in}{0.770196in}}%
\pgfpathlineto{\pgfqpoint{2.712216in}{1.035388in}}%
\pgfpathlineto{\pgfqpoint{2.713856in}{0.769322in}}%
\pgfpathlineto{\pgfqpoint{2.715496in}{1.041252in}}%
\pgfpathlineto{\pgfqpoint{2.715906in}{0.952904in}}%
\pgfpathlineto{\pgfqpoint{2.717136in}{0.770359in}}%
\pgfpathlineto{\pgfqpoint{2.718776in}{1.040203in}}%
\pgfpathlineto{\pgfqpoint{2.719186in}{0.983916in}}%
\pgfpathlineto{\pgfqpoint{2.720416in}{0.790765in}}%
\pgfpathlineto{\pgfqpoint{2.720826in}{0.803124in}}%
\pgfpathlineto{\pgfqpoint{2.722056in}{1.031992in}}%
\pgfpathlineto{\pgfqpoint{2.722466in}{1.009663in}}%
\pgfpathlineto{\pgfqpoint{2.724106in}{0.780009in}}%
\pgfpathlineto{\pgfqpoint{2.725746in}{1.029511in}}%
\pgfpathlineto{\pgfqpoint{2.727386in}{0.772033in}}%
\pgfpathlineto{\pgfqpoint{2.729026in}{1.042902in}}%
\pgfpathlineto{\pgfqpoint{2.730666in}{0.771180in}}%
\pgfpathlineto{\pgfqpoint{2.732306in}{1.049362in}}%
\pgfpathlineto{\pgfqpoint{2.732716in}{0.958193in}}%
\pgfpathlineto{\pgfqpoint{2.733946in}{0.772006in}}%
\pgfpathlineto{\pgfqpoint{2.735586in}{1.048510in}}%
\pgfpathlineto{\pgfqpoint{2.735996in}{0.990592in}}%
\pgfpathlineto{\pgfqpoint{2.737226in}{0.793278in}}%
\pgfpathlineto{\pgfqpoint{2.737636in}{0.805598in}}%
\pgfpathlineto{\pgfqpoint{2.738866in}{1.040069in}}%
\pgfpathlineto{\pgfqpoint{2.739276in}{1.017429in}}%
\pgfpathlineto{\pgfqpoint{2.740916in}{0.781338in}}%
\pgfpathlineto{\pgfqpoint{2.742556in}{1.038015in}}%
\pgfpathlineto{\pgfqpoint{2.744196in}{0.773984in}}%
\pgfpathlineto{\pgfqpoint{2.745836in}{1.051745in}}%
\pgfpathlineto{\pgfqpoint{2.747476in}{0.772888in}}%
\pgfpathlineto{\pgfqpoint{2.749116in}{1.058104in}}%
\pgfpathlineto{\pgfqpoint{2.749526in}{0.967299in}}%
\pgfpathlineto{\pgfqpoint{2.750756in}{0.776109in}}%
\pgfpathlineto{\pgfqpoint{2.752396in}{1.056682in}}%
\pgfpathlineto{\pgfqpoint{2.752806in}{1.000639in}}%
\pgfpathlineto{\pgfqpoint{2.754036in}{0.797900in}}%
\pgfpathlineto{\pgfqpoint{2.754446in}{0.805743in}}%
\pgfpathlineto{\pgfqpoint{2.755676in}{1.047178in}}%
\pgfpathlineto{\pgfqpoint{2.756086in}{1.028011in}}%
\pgfpathlineto{\pgfqpoint{2.757726in}{0.779976in}}%
\pgfpathlineto{\pgfqpoint{2.759366in}{1.048676in}}%
\pgfpathlineto{\pgfqpoint{2.761006in}{0.776000in}}%
\pgfpathlineto{\pgfqpoint{2.762646in}{1.061985in}}%
\pgfpathlineto{\pgfqpoint{2.764286in}{0.774343in}}%
\pgfpathlineto{\pgfqpoint{2.765926in}{1.067391in}}%
\pgfpathlineto{\pgfqpoint{2.766336in}{0.980659in}}%
\pgfpathlineto{\pgfqpoint{2.767566in}{0.782929in}}%
\pgfpathlineto{\pgfqpoint{2.769206in}{1.064457in}}%
\pgfpathlineto{\pgfqpoint{2.769616in}{1.014364in}}%
\pgfpathlineto{\pgfqpoint{2.770846in}{0.804783in}}%
\pgfpathlineto{\pgfqpoint{2.771256in}{0.803027in}}%
\pgfpathlineto{\pgfqpoint{2.772486in}{1.052876in}}%
\pgfpathlineto{\pgfqpoint{2.772896in}{1.041567in}}%
\pgfpathlineto{\pgfqpoint{2.774536in}{0.777377in}}%
\pgfpathlineto{\pgfqpoint{2.776176in}{1.061481in}}%
\pgfpathlineto{\pgfqpoint{2.777816in}{0.777915in}}%
\pgfpathlineto{\pgfqpoint{2.779456in}{1.073420in}}%
\pgfpathlineto{\pgfqpoint{2.781096in}{0.775329in}}%
\pgfpathlineto{\pgfqpoint{2.782736in}{1.076812in}}%
\pgfpathlineto{\pgfqpoint{2.783146in}{0.998558in}}%
\pgfpathlineto{\pgfqpoint{2.784376in}{0.792593in}}%
\pgfpathlineto{\pgfqpoint{2.785606in}{0.919800in}}%
\pgfpathlineto{\pgfqpoint{2.786016in}{1.071209in}}%
\pgfpathlineto{\pgfqpoint{2.786426in}{1.031868in}}%
\pgfpathlineto{\pgfqpoint{2.788066in}{0.796727in}}%
\pgfpathlineto{\pgfqpoint{2.789706in}{1.057987in}}%
\pgfpathlineto{\pgfqpoint{2.791346in}{0.780147in}}%
\pgfpathlineto{\pgfqpoint{2.792986in}{1.076086in}}%
\pgfpathlineto{\pgfqpoint{2.794626in}{0.779408in}}%
\pgfpathlineto{\pgfqpoint{2.796266in}{1.085453in}}%
\pgfpathlineto{\pgfqpoint{2.796676in}{0.982054in}}%
\pgfpathlineto{\pgfqpoint{2.797906in}{0.779673in}}%
\pgfpathlineto{\pgfqpoint{2.799546in}{1.085503in}}%
\pgfpathlineto{\pgfqpoint{2.799956in}{1.021029in}}%
\pgfpathlineto{\pgfqpoint{2.801186in}{0.805000in}}%
\pgfpathlineto{\pgfqpoint{2.801596in}{0.815894in}}%
\pgfpathlineto{\pgfqpoint{2.802826in}{1.075795in}}%
\pgfpathlineto{\pgfqpoint{2.803236in}{1.052918in}}%
\pgfpathlineto{\pgfqpoint{2.804876in}{0.785941in}}%
\pgfpathlineto{\pgfqpoint{2.806516in}{1.076744in}}%
\pgfpathlineto{\pgfqpoint{2.808156in}{0.782492in}}%
\pgfpathlineto{\pgfqpoint{2.809796in}{1.091650in}}%
\pgfpathlineto{\pgfqpoint{2.811436in}{0.779976in}}%
\pgfpathlineto{\pgfqpoint{2.813076in}{1.096912in}}%
\pgfpathlineto{\pgfqpoint{2.813486in}{1.010474in}}%
\pgfpathlineto{\pgfqpoint{2.814716in}{0.796085in}}%
\pgfpathlineto{\pgfqpoint{2.815946in}{0.937317in}}%
\pgfpathlineto{\pgfqpoint{2.816356in}{1.091955in}}%
\pgfpathlineto{\pgfqpoint{2.816766in}{1.047680in}}%
\pgfpathlineto{\pgfqpoint{2.818406in}{0.802807in}}%
\pgfpathlineto{\pgfqpoint{2.820046in}{1.076753in}}%
\pgfpathlineto{\pgfqpoint{2.821686in}{0.784857in}}%
\pgfpathlineto{\pgfqpoint{2.823326in}{1.096685in}}%
\pgfpathlineto{\pgfqpoint{2.824966in}{0.783702in}}%
\pgfpathlineto{\pgfqpoint{2.826606in}{1.106609in}}%
\pgfpathlineto{\pgfqpoint{2.827016in}{1.001586in}}%
\pgfpathlineto{\pgfqpoint{2.828246in}{0.788050in}}%
\pgfpathlineto{\pgfqpoint{2.829886in}{1.105814in}}%
\pgfpathlineto{\pgfqpoint{2.830296in}{1.043644in}}%
\pgfpathlineto{\pgfqpoint{2.831526in}{0.815056in}}%
\pgfpathlineto{\pgfqpoint{2.831936in}{0.817197in}}%
\pgfpathlineto{\pgfqpoint{2.833166in}{1.093762in}}%
\pgfpathlineto{\pgfqpoint{2.833576in}{1.077438in}}%
\pgfpathlineto{\pgfqpoint{2.835216in}{0.786839in}}%
\pgfpathlineto{\pgfqpoint{2.836856in}{1.101802in}}%
\pgfpathlineto{\pgfqpoint{2.838496in}{0.786897in}}%
\pgfpathlineto{\pgfqpoint{2.840136in}{1.115723in}}%
\pgfpathlineto{\pgfqpoint{2.841776in}{0.782829in}}%
\pgfpathlineto{\pgfqpoint{2.843416in}{1.118352in}}%
\pgfpathlineto{\pgfqpoint{2.843826in}{1.042134in}}%
\pgfpathlineto{\pgfqpoint{2.845056in}{0.811897in}}%
\pgfpathlineto{\pgfqpoint{2.845466in}{0.828882in}}%
\pgfpathlineto{\pgfqpoint{2.846696in}{1.109026in}}%
\pgfpathlineto{\pgfqpoint{2.847106in}{1.080065in}}%
\pgfpathlineto{\pgfqpoint{2.848746in}{0.793719in}}%
\pgfpathlineto{\pgfqpoint{2.850386in}{1.108171in}}%
\pgfpathlineto{\pgfqpoint{2.852026in}{0.789793in}}%
\pgfpathlineto{\pgfqpoint{2.853666in}{1.125285in}}%
\pgfpathlineto{\pgfqpoint{2.855306in}{0.786170in}}%
\pgfpathlineto{\pgfqpoint{2.856946in}{1.130424in}}%
\pgfpathlineto{\pgfqpoint{2.857356in}{1.044401in}}%
\pgfpathlineto{\pgfqpoint{2.858586in}{0.811038in}}%
\pgfpathlineto{\pgfqpoint{2.859816in}{0.947742in}}%
\pgfpathlineto{\pgfqpoint{2.860226in}{1.122801in}}%
\pgfpathlineto{\pgfqpoint{2.860636in}{1.085804in}}%
\pgfpathlineto{\pgfqpoint{2.862276in}{0.800922in}}%
\pgfpathlineto{\pgfqpoint{2.863916in}{1.116829in}}%
\pgfpathlineto{\pgfqpoint{2.865556in}{0.792576in}}%
\pgfpathlineto{\pgfqpoint{2.867196in}{1.136156in}}%
\pgfpathlineto{\pgfqpoint{2.868836in}{0.788999in}}%
\pgfpathlineto{\pgfqpoint{2.870476in}{1.142666in}}%
\pgfpathlineto{\pgfqpoint{2.870886in}{1.051605in}}%
\pgfpathlineto{\pgfqpoint{2.872116in}{0.813229in}}%
\pgfpathlineto{\pgfqpoint{2.873346in}{0.958449in}}%
\pgfpathlineto{\pgfqpoint{2.873756in}{1.135459in}}%
\pgfpathlineto{\pgfqpoint{2.874166in}{1.095677in}}%
\pgfpathlineto{\pgfqpoint{2.875806in}{0.804153in}}%
\pgfpathlineto{\pgfqpoint{2.877446in}{1.128608in}}%
\pgfpathlineto{\pgfqpoint{2.879086in}{0.795326in}}%
\pgfpathlineto{\pgfqpoint{2.880726in}{1.148922in}}%
\pgfpathlineto{\pgfqpoint{2.882366in}{0.791260in}}%
\pgfpathlineto{\pgfqpoint{2.884006in}{1.155373in}}%
\pgfpathlineto{\pgfqpoint{2.884416in}{1.064755in}}%
\pgfpathlineto{\pgfqpoint{2.885646in}{0.819085in}}%
\pgfpathlineto{\pgfqpoint{2.886876in}{0.961840in}}%
\pgfpathlineto{\pgfqpoint{2.887286in}{1.146956in}}%
\pgfpathlineto{\pgfqpoint{2.887696in}{1.110476in}}%
\pgfpathlineto{\pgfqpoint{2.889336in}{0.802597in}}%
\pgfpathlineto{\pgfqpoint{2.890976in}{1.144017in}}%
\pgfpathlineto{\pgfqpoint{2.892616in}{0.797933in}}%
\pgfpathlineto{\pgfqpoint{2.894256in}{1.163753in}}%
\pgfpathlineto{\pgfqpoint{2.895896in}{0.792714in}}%
\pgfpathlineto{\pgfqpoint{2.897536in}{1.168320in}}%
\pgfpathlineto{\pgfqpoint{2.897946in}{1.084610in}}%
\pgfpathlineto{\pgfqpoint{2.899176in}{0.828960in}}%
\pgfpathlineto{\pgfqpoint{2.899586in}{0.840782in}}%
\pgfpathlineto{\pgfqpoint{2.900816in}{1.156628in}}%
\pgfpathlineto{\pgfqpoint{2.901226in}{1.130621in}}%
\pgfpathlineto{\pgfqpoint{2.902866in}{0.801090in}}%
\pgfpathlineto{\pgfqpoint{2.904506in}{1.163063in}}%
\pgfpathlineto{\pgfqpoint{2.906146in}{0.800010in}}%
\pgfpathlineto{\pgfqpoint{2.907786in}{1.180179in}}%
\pgfpathlineto{\pgfqpoint{2.908196in}{1.053881in}}%
\pgfpathlineto{\pgfqpoint{2.909426in}{0.806992in}}%
\pgfpathlineto{\pgfqpoint{2.911066in}{1.180509in}}%
\pgfpathlineto{\pgfqpoint{2.911476in}{1.111501in}}%
\pgfpathlineto{\pgfqpoint{2.912706in}{0.842768in}}%
\pgfpathlineto{\pgfqpoint{2.913116in}{0.830597in}}%
\pgfpathlineto{\pgfqpoint{2.914346in}{1.162905in}}%
\pgfpathlineto{\pgfqpoint{2.914756in}{1.155926in}}%
\pgfpathlineto{\pgfqpoint{2.916396in}{0.804659in}}%
\pgfpathlineto{\pgfqpoint{2.918036in}{1.184971in}}%
\pgfpathlineto{\pgfqpoint{2.919676in}{0.800819in}}%
\pgfpathlineto{\pgfqpoint{2.921316in}{1.196771in}}%
\pgfpathlineto{\pgfqpoint{2.921726in}{1.089414in}}%
\pgfpathlineto{\pgfqpoint{2.922956in}{0.826884in}}%
\pgfpathlineto{\pgfqpoint{2.924186in}{0.997633in}}%
\pgfpathlineto{\pgfqpoint{2.924596in}{1.189806in}}%
\pgfpathlineto{\pgfqpoint{2.925006in}{1.145014in}}%
\pgfpathlineto{\pgfqpoint{2.926646in}{0.811960in}}%
\pgfpathlineto{\pgfqpoint{2.928286in}{1.185228in}}%
\pgfpathlineto{\pgfqpoint{2.929926in}{0.806850in}}%
\pgfpathlineto{\pgfqpoint{2.931566in}{1.207752in}}%
\pgfpathlineto{\pgfqpoint{2.933206in}{0.809869in}}%
\pgfpathlineto{\pgfqpoint{2.934846in}{1.210664in}}%
\pgfpathlineto{\pgfqpoint{2.935256in}{1.132968in}}%
\pgfpathlineto{\pgfqpoint{2.936486in}{0.850629in}}%
\pgfpathlineto{\pgfqpoint{2.936896in}{0.840024in}}%
\pgfpathlineto{\pgfqpoint{2.938126in}{1.192438in}}%
\pgfpathlineto{\pgfqpoint{2.938536in}{1.183490in}}%
\pgfpathlineto{\pgfqpoint{2.940176in}{0.811328in}}%
\pgfpathlineto{\pgfqpoint{2.941816in}{1.215816in}}%
\pgfpathlineto{\pgfqpoint{2.943456in}{0.806034in}}%
\pgfpathlineto{\pgfqpoint{2.945096in}{1.227592in}}%
\pgfpathlineto{\pgfqpoint{2.945506in}{1.122340in}}%
\pgfpathlineto{\pgfqpoint{2.946736in}{0.841830in}}%
\pgfpathlineto{\pgfqpoint{2.947146in}{0.864024in}}%
\pgfpathlineto{\pgfqpoint{2.948376in}{1.216906in}}%
\pgfpathlineto{\pgfqpoint{2.948786in}{1.182222in}}%
\pgfpathlineto{\pgfqpoint{2.950426in}{0.814778in}}%
\pgfpathlineto{\pgfqpoint{2.952066in}{1.223255in}}%
\pgfpathlineto{\pgfqpoint{2.952066in}{1.223255in}}%
\pgfusepath{stroke}%
\end{pgfscope}%
\begin{pgfscope}%
\pgfpathrectangle{\pgfqpoint{0.800000in}{0.528000in}}{\pgfqpoint{2.254545in}{1.680000in}}%
\pgfusepath{clip}%
\pgfsetrectcap%
\pgfsetroundjoin%
\pgfsetlinewidth{1.505625pt}%
\definecolor{currentstroke}{rgb}{0.737255,0.741176,0.133333}%
\pgfsetstrokecolor{currentstroke}%
\pgfsetdash{}{0pt}%
\pgfpathmoveto{\pgfqpoint{0.902479in}{1.531273in}}%
\pgfpathlineto{\pgfqpoint{0.908219in}{1.530283in}}%
\pgfpathlineto{\pgfqpoint{0.913959in}{1.527027in}}%
\pgfpathlineto{\pgfqpoint{0.919699in}{1.521114in}}%
\pgfpathlineto{\pgfqpoint{0.925439in}{1.511901in}}%
\pgfpathlineto{\pgfqpoint{0.931589in}{1.497358in}}%
\pgfpathlineto{\pgfqpoint{0.937739in}{1.476650in}}%
\pgfpathlineto{\pgfqpoint{0.944299in}{1.445987in}}%
\pgfpathlineto{\pgfqpoint{0.951269in}{1.401481in}}%
\pgfpathlineto{\pgfqpoint{0.958649in}{1.338449in}}%
\pgfpathlineto{\pgfqpoint{0.963159in}{1.295362in}}%
\pgfpathlineto{\pgfqpoint{0.982839in}{1.694579in}}%
\pgfpathlineto{\pgfqpoint{0.996779in}{1.956620in}}%
\pgfpathlineto{\pgfqpoint{1.004979in}{2.073920in}}%
\pgfpathlineto{\pgfqpoint{1.011129in}{2.133696in}}%
\pgfpathlineto{\pgfqpoint{1.015639in}{2.158735in}}%
\pgfpathlineto{\pgfqpoint{1.018509in}{2.165432in}}%
\pgfpathlineto{\pgfqpoint{1.020149in}{2.165837in}}%
\pgfpathlineto{\pgfqpoint{1.021789in}{2.163667in}}%
\pgfpathlineto{\pgfqpoint{1.024249in}{2.155445in}}%
\pgfpathlineto{\pgfqpoint{1.027529in}{2.134921in}}%
\pgfpathlineto{\pgfqpoint{1.031629in}{2.093393in}}%
\pgfpathlineto{\pgfqpoint{1.036549in}{2.019621in}}%
\pgfpathlineto{\pgfqpoint{1.042699in}{1.890139in}}%
\pgfpathlineto{\pgfqpoint{1.049669in}{1.693953in}}%
\pgfpathlineto{\pgfqpoint{1.058279in}{1.383856in}}%
\pgfpathlineto{\pgfqpoint{1.065659in}{1.073851in}}%
\pgfpathlineto{\pgfqpoint{1.066479in}{1.081586in}}%
\pgfpathlineto{\pgfqpoint{1.077549in}{1.175263in}}%
\pgfpathlineto{\pgfqpoint{1.084519in}{1.217856in}}%
\pgfpathlineto{\pgfqpoint{1.089029in}{1.233905in}}%
\pgfpathlineto{\pgfqpoint{1.091899in}{1.237486in}}%
\pgfpathlineto{\pgfqpoint{1.093539in}{1.236666in}}%
\pgfpathlineto{\pgfqpoint{1.095589in}{1.232218in}}%
\pgfpathlineto{\pgfqpoint{1.098459in}{1.218549in}}%
\pgfpathlineto{\pgfqpoint{1.101739in}{1.190159in}}%
\pgfpathlineto{\pgfqpoint{1.105429in}{1.138285in}}%
\pgfpathlineto{\pgfqpoint{1.109939in}{1.039914in}}%
\pgfpathlineto{\pgfqpoint{1.114039in}{0.912096in}}%
\pgfpathlineto{\pgfqpoint{1.114859in}{0.927608in}}%
\pgfpathlineto{\pgfqpoint{1.131259in}{1.432941in}}%
\pgfpathlineto{\pgfqpoint{1.134129in}{1.461052in}}%
\pgfpathlineto{\pgfqpoint{1.134949in}{1.462531in}}%
\pgfpathlineto{\pgfqpoint{1.135359in}{1.462109in}}%
\pgfpathlineto{\pgfqpoint{1.136589in}{1.456099in}}%
\pgfpathlineto{\pgfqpoint{1.138639in}{1.430037in}}%
\pgfpathlineto{\pgfqpoint{1.141509in}{1.360627in}}%
\pgfpathlineto{\pgfqpoint{1.145199in}{1.236389in}}%
\pgfpathlineto{\pgfqpoint{1.152989in}{1.744330in}}%
\pgfpathlineto{\pgfqpoint{1.157909in}{1.938054in}}%
\pgfpathlineto{\pgfqpoint{1.161189in}{2.000155in}}%
\pgfpathlineto{\pgfqpoint{1.163239in}{2.010885in}}%
\pgfpathlineto{\pgfqpoint{1.164059in}{2.009180in}}%
\pgfpathlineto{\pgfqpoint{1.165699in}{1.995670in}}%
\pgfpathlineto{\pgfqpoint{1.168159in}{1.950870in}}%
\pgfpathlineto{\pgfqpoint{1.171849in}{1.831822in}}%
\pgfpathlineto{\pgfqpoint{1.176769in}{1.587804in}}%
\pgfpathlineto{\pgfqpoint{1.183739in}{1.115477in}}%
\pgfpathlineto{\pgfqpoint{1.184969in}{1.046619in}}%
\pgfpathlineto{\pgfqpoint{1.185379in}{1.053319in}}%
\pgfpathlineto{\pgfqpoint{1.191939in}{1.143065in}}%
\pgfpathlineto{\pgfqpoint{1.196039in}{1.176487in}}%
\pgfpathlineto{\pgfqpoint{1.198499in}{1.184260in}}%
\pgfpathlineto{\pgfqpoint{1.199729in}{1.183769in}}%
\pgfpathlineto{\pgfqpoint{1.201369in}{1.177775in}}%
\pgfpathlineto{\pgfqpoint{1.203419in}{1.160357in}}%
\pgfpathlineto{\pgfqpoint{1.206289in}{1.113789in}}%
\pgfpathlineto{\pgfqpoint{1.209979in}{1.007851in}}%
\pgfpathlineto{\pgfqpoint{1.212849in}{0.885178in}}%
\pgfpathlineto{\pgfqpoint{1.213259in}{0.896146in}}%
\pgfpathlineto{\pgfqpoint{1.223509in}{1.313129in}}%
\pgfpathlineto{\pgfqpoint{1.225969in}{1.348941in}}%
\pgfpathlineto{\pgfqpoint{1.226789in}{1.350641in}}%
\pgfpathlineto{\pgfqpoint{1.227199in}{1.349442in}}%
\pgfpathlineto{\pgfqpoint{1.228429in}{1.337512in}}%
\pgfpathlineto{\pgfqpoint{1.230479in}{1.290179in}}%
\pgfpathlineto{\pgfqpoint{1.233759in}{1.155760in}}%
\pgfpathlineto{\pgfqpoint{1.234169in}{1.195037in}}%
\pgfpathlineto{\pgfqpoint{1.239909in}{1.644344in}}%
\pgfpathlineto{\pgfqpoint{1.243599in}{1.800324in}}%
\pgfpathlineto{\pgfqpoint{1.246059in}{1.837490in}}%
\pgfpathlineto{\pgfqpoint{1.246469in}{1.838462in}}%
\pgfpathlineto{\pgfqpoint{1.246879in}{1.837963in}}%
\pgfpathlineto{\pgfqpoint{1.248109in}{1.827764in}}%
\pgfpathlineto{\pgfqpoint{1.250159in}{1.782746in}}%
\pgfpathlineto{\pgfqpoint{1.253439in}{1.643842in}}%
\pgfpathlineto{\pgfqpoint{1.257949in}{1.341744in}}%
\pgfpathlineto{\pgfqpoint{1.262049in}{1.009629in}}%
\pgfpathlineto{\pgfqpoint{1.262869in}{1.025888in}}%
\pgfpathlineto{\pgfqpoint{1.268199in}{1.106067in}}%
\pgfpathlineto{\pgfqpoint{1.271069in}{1.125018in}}%
\pgfpathlineto{\pgfqpoint{1.272299in}{1.126097in}}%
\pgfpathlineto{\pgfqpoint{1.273529in}{1.122129in}}%
\pgfpathlineto{\pgfqpoint{1.275579in}{1.102543in}}%
\pgfpathlineto{\pgfqpoint{1.278039in}{1.053774in}}%
\pgfpathlineto{\pgfqpoint{1.281319in}{0.938340in}}%
\pgfpathlineto{\pgfqpoint{1.282959in}{0.857876in}}%
\pgfpathlineto{\pgfqpoint{1.283369in}{0.872937in}}%
\pgfpathlineto{\pgfqpoint{1.291159in}{1.208094in}}%
\pgfpathlineto{\pgfqpoint{1.293619in}{1.242173in}}%
\pgfpathlineto{\pgfqpoint{1.294029in}{1.242183in}}%
\pgfpathlineto{\pgfqpoint{1.294849in}{1.237040in}}%
\pgfpathlineto{\pgfqpoint{1.296489in}{1.206129in}}%
\pgfpathlineto{\pgfqpoint{1.299359in}{1.092306in}}%
\pgfpathlineto{\pgfqpoint{1.299769in}{1.097697in}}%
\pgfpathlineto{\pgfqpoint{1.305099in}{1.532760in}}%
\pgfpathlineto{\pgfqpoint{1.308379in}{1.654124in}}%
\pgfpathlineto{\pgfqpoint{1.310019in}{1.667428in}}%
\pgfpathlineto{\pgfqpoint{1.310839in}{1.662374in}}%
\pgfpathlineto{\pgfqpoint{1.312479in}{1.629738in}}%
\pgfpathlineto{\pgfqpoint{1.314939in}{1.528519in}}%
\pgfpathlineto{\pgfqpoint{1.319039in}{1.244806in}}%
\pgfpathlineto{\pgfqpoint{1.322319in}{0.971818in}}%
\pgfpathlineto{\pgfqpoint{1.323139in}{0.989377in}}%
\pgfpathlineto{\pgfqpoint{1.327239in}{1.053190in}}%
\pgfpathlineto{\pgfqpoint{1.329699in}{1.067140in}}%
\pgfpathlineto{\pgfqpoint{1.330519in}{1.066613in}}%
\pgfpathlineto{\pgfqpoint{1.331749in}{1.060188in}}%
\pgfpathlineto{\pgfqpoint{1.333799in}{1.032454in}}%
\pgfpathlineto{\pgfqpoint{1.336669in}{0.952217in}}%
\pgfpathlineto{\pgfqpoint{1.339539in}{0.838353in}}%
\pgfpathlineto{\pgfqpoint{1.339949in}{0.858405in}}%
\pgfpathlineto{\pgfqpoint{1.346099in}{1.113096in}}%
\pgfpathlineto{\pgfqpoint{1.348559in}{1.143422in}}%
\pgfpathlineto{\pgfqpoint{1.349379in}{1.139299in}}%
\pgfpathlineto{\pgfqpoint{1.351019in}{1.108959in}}%
\pgfpathlineto{\pgfqpoint{1.353479in}{1.014027in}}%
\pgfpathlineto{\pgfqpoint{1.353889in}{1.049406in}}%
\pgfpathlineto{\pgfqpoint{1.358399in}{1.403014in}}%
\pgfpathlineto{\pgfqpoint{1.361269in}{1.499179in}}%
\pgfpathlineto{\pgfqpoint{1.362499in}{1.506067in}}%
\pgfpathlineto{\pgfqpoint{1.363319in}{1.499367in}}%
\pgfpathlineto{\pgfqpoint{1.364959in}{1.460009in}}%
\pgfpathlineto{\pgfqpoint{1.367829in}{1.316219in}}%
\pgfpathlineto{\pgfqpoint{1.372749in}{0.931893in}}%
\pgfpathlineto{\pgfqpoint{1.373979in}{0.957812in}}%
\pgfpathlineto{\pgfqpoint{1.377669in}{1.005905in}}%
\pgfpathlineto{\pgfqpoint{1.378899in}{1.009889in}}%
\pgfpathlineto{\pgfqpoint{1.379309in}{1.009630in}}%
\pgfpathlineto{\pgfqpoint{1.380539in}{1.003657in}}%
\pgfpathlineto{\pgfqpoint{1.382589in}{0.974534in}}%
\pgfpathlineto{\pgfqpoint{1.385459in}{0.888730in}}%
\pgfpathlineto{\pgfqpoint{1.387509in}{0.815260in}}%
\pgfpathlineto{\pgfqpoint{1.393249in}{1.033484in}}%
\pgfpathlineto{\pgfqpoint{1.395299in}{1.056024in}}%
\pgfpathlineto{\pgfqpoint{1.396119in}{1.052665in}}%
\pgfpathlineto{\pgfqpoint{1.397759in}{1.024208in}}%
\pgfpathlineto{\pgfqpoint{1.399809in}{0.952233in}}%
\pgfpathlineto{\pgfqpoint{1.405139in}{1.316338in}}%
\pgfpathlineto{\pgfqpoint{1.407189in}{1.358328in}}%
\pgfpathlineto{\pgfqpoint{1.407599in}{1.359388in}}%
\pgfpathlineto{\pgfqpoint{1.408009in}{1.358028in}}%
\pgfpathlineto{\pgfqpoint{1.409239in}{1.339761in}}%
\pgfpathlineto{\pgfqpoint{1.411289in}{1.265043in}}%
\pgfpathlineto{\pgfqpoint{1.414569in}{1.051198in}}%
\pgfpathlineto{\pgfqpoint{1.416619in}{0.892071in}}%
\pgfpathlineto{\pgfqpoint{1.417439in}{0.909238in}}%
\pgfpathlineto{\pgfqpoint{1.420719in}{0.952235in}}%
\pgfpathlineto{\pgfqpoint{1.421949in}{0.955855in}}%
\pgfpathlineto{\pgfqpoint{1.422359in}{0.955310in}}%
\pgfpathlineto{\pgfqpoint{1.423589in}{0.947997in}}%
\pgfpathlineto{\pgfqpoint{1.425639in}{0.915344in}}%
\pgfpathlineto{\pgfqpoint{1.428509in}{0.824428in}}%
\pgfpathlineto{\pgfqpoint{1.429329in}{0.789489in}}%
\pgfpathlineto{\pgfqpoint{1.429739in}{0.795964in}}%
\pgfpathlineto{\pgfqpoint{1.434659in}{0.961208in}}%
\pgfpathlineto{\pgfqpoint{1.436708in}{0.980682in}}%
\pgfpathlineto{\pgfqpoint{1.437528in}{0.976854in}}%
\pgfpathlineto{\pgfqpoint{1.439168in}{0.949091in}}%
\pgfpathlineto{\pgfqpoint{1.440808in}{0.899475in}}%
\pgfpathlineto{\pgfqpoint{1.445318in}{1.185209in}}%
\pgfpathlineto{\pgfqpoint{1.447368in}{1.227837in}}%
\pgfpathlineto{\pgfqpoint{1.447778in}{1.229010in}}%
\pgfpathlineto{\pgfqpoint{1.448188in}{1.227751in}}%
\pgfpathlineto{\pgfqpoint{1.449418in}{1.209721in}}%
\pgfpathlineto{\pgfqpoint{1.451468in}{1.135592in}}%
\pgfpathlineto{\pgfqpoint{1.455158in}{0.895485in}}%
\pgfpathlineto{\pgfqpoint{1.455978in}{0.859022in}}%
\pgfpathlineto{\pgfqpoint{1.456388in}{0.866986in}}%
\pgfpathlineto{\pgfqpoint{1.459258in}{0.903028in}}%
\pgfpathlineto{\pgfqpoint{1.460488in}{0.906379in}}%
\pgfpathlineto{\pgfqpoint{1.460898in}{0.905677in}}%
\pgfpathlineto{\pgfqpoint{1.462128in}{0.897762in}}%
\pgfpathlineto{\pgfqpoint{1.464178in}{0.864183in}}%
\pgfpathlineto{\pgfqpoint{1.467458in}{0.769540in}}%
\pgfpathlineto{\pgfqpoint{1.467868in}{0.784486in}}%
\pgfpathlineto{\pgfqpoint{1.472378in}{0.906911in}}%
\pgfpathlineto{\pgfqpoint{1.474018in}{0.917017in}}%
\pgfpathlineto{\pgfqpoint{1.474838in}{0.913029in}}%
\pgfpathlineto{\pgfqpoint{1.476478in}{0.887233in}}%
\pgfpathlineto{\pgfqpoint{1.477708in}{0.854102in}}%
\pgfpathlineto{\pgfqpoint{1.483038in}{1.107795in}}%
\pgfpathlineto{\pgfqpoint{1.484268in}{1.116011in}}%
\pgfpathlineto{\pgfqpoint{1.485088in}{1.109873in}}%
\pgfpathlineto{\pgfqpoint{1.486728in}{1.071183in}}%
\pgfpathlineto{\pgfqpoint{1.489598in}{0.932191in}}%
\pgfpathlineto{\pgfqpoint{1.491238in}{0.824507in}}%
\pgfpathlineto{\pgfqpoint{1.492058in}{0.833985in}}%
\pgfpathlineto{\pgfqpoint{1.494928in}{0.861723in}}%
\pgfpathlineto{\pgfqpoint{1.495748in}{0.862146in}}%
\pgfpathlineto{\pgfqpoint{1.496978in}{0.855843in}}%
\pgfpathlineto{\pgfqpoint{1.499028in}{0.826146in}}%
\pgfpathlineto{\pgfqpoint{1.501898in}{0.748311in}}%
\pgfpathlineto{\pgfqpoint{1.502308in}{0.761065in}}%
\pgfpathlineto{\pgfqpoint{1.506408in}{0.855363in}}%
\pgfpathlineto{\pgfqpoint{1.508048in}{0.864354in}}%
\pgfpathlineto{\pgfqpoint{1.508868in}{0.860992in}}%
\pgfpathlineto{\pgfqpoint{1.510508in}{0.838913in}}%
\pgfpathlineto{\pgfqpoint{1.511328in}{0.821020in}}%
\pgfpathlineto{\pgfqpoint{1.511738in}{0.824166in}}%
\pgfpathlineto{\pgfqpoint{1.515428in}{0.993988in}}%
\pgfpathlineto{\pgfqpoint{1.517478in}{1.020686in}}%
\pgfpathlineto{\pgfqpoint{1.518298in}{1.016395in}}%
\pgfpathlineto{\pgfqpoint{1.519938in}{0.983455in}}%
\pgfpathlineto{\pgfqpoint{1.522808in}{0.861170in}}%
\pgfpathlineto{\pgfqpoint{1.524038in}{0.792066in}}%
\pgfpathlineto{\pgfqpoint{1.524858in}{0.804346in}}%
\pgfpathlineto{\pgfqpoint{1.527318in}{0.823869in}}%
\pgfpathlineto{\pgfqpoint{1.527728in}{0.824333in}}%
\pgfpathlineto{\pgfqpoint{1.528138in}{0.823945in}}%
\pgfpathlineto{\pgfqpoint{1.529368in}{0.817563in}}%
\pgfpathlineto{\pgfqpoint{1.531418in}{0.789556in}}%
\pgfpathlineto{\pgfqpoint{1.533878in}{0.736191in}}%
\pgfpathlineto{\pgfqpoint{1.534288in}{0.746675in}}%
\pgfpathlineto{\pgfqpoint{1.537978in}{0.814823in}}%
\pgfpathlineto{\pgfqpoint{1.539618in}{0.821416in}}%
\pgfpathlineto{\pgfqpoint{1.540438in}{0.818043in}}%
\pgfpathlineto{\pgfqpoint{1.542078in}{0.798515in}}%
\pgfpathlineto{\pgfqpoint{1.542898in}{0.785014in}}%
\pgfpathlineto{\pgfqpoint{1.546588in}{0.924076in}}%
\pgfpathlineto{\pgfqpoint{1.548228in}{0.941688in}}%
\pgfpathlineto{\pgfqpoint{1.548638in}{0.941340in}}%
\pgfpathlineto{\pgfqpoint{1.549868in}{0.929169in}}%
\pgfpathlineto{\pgfqpoint{1.551918in}{0.875285in}}%
\pgfpathlineto{\pgfqpoint{1.554378in}{0.770113in}}%
\pgfpathlineto{\pgfqpoint{1.555198in}{0.777362in}}%
\pgfpathlineto{\pgfqpoint{1.557658in}{0.792222in}}%
\pgfpathlineto{\pgfqpoint{1.558478in}{0.791223in}}%
\pgfpathlineto{\pgfqpoint{1.560118in}{0.779903in}}%
\pgfpathlineto{\pgfqpoint{1.562578in}{0.741445in}}%
\pgfpathlineto{\pgfqpoint{1.563398in}{0.724137in}}%
\pgfpathlineto{\pgfqpoint{1.563808in}{0.728797in}}%
\pgfpathlineto{\pgfqpoint{1.567498in}{0.783065in}}%
\pgfpathlineto{\pgfqpoint{1.568728in}{0.787351in}}%
\pgfpathlineto{\pgfqpoint{1.569138in}{0.786938in}}%
\pgfpathlineto{\pgfqpoint{1.570368in}{0.780214in}}%
\pgfpathlineto{\pgfqpoint{1.572008in}{0.759660in}}%
\pgfpathlineto{\pgfqpoint{1.576108in}{0.872269in}}%
\pgfpathlineto{\pgfqpoint{1.577338in}{0.877892in}}%
\pgfpathlineto{\pgfqpoint{1.578158in}{0.873540in}}%
\pgfpathlineto{\pgfqpoint{1.579798in}{0.846697in}}%
\pgfpathlineto{\pgfqpoint{1.583078in}{0.750085in}}%
\pgfpathlineto{\pgfqpoint{1.584308in}{0.760626in}}%
\pgfpathlineto{\pgfqpoint{1.585948in}{0.765981in}}%
\pgfpathlineto{\pgfqpoint{1.586768in}{0.764673in}}%
\pgfpathlineto{\pgfqpoint{1.588408in}{0.754077in}}%
\pgfpathlineto{\pgfqpoint{1.591278in}{0.713852in}}%
\pgfpathlineto{\pgfqpoint{1.592098in}{0.724997in}}%
\pgfpathlineto{\pgfqpoint{1.595378in}{0.758697in}}%
\pgfpathlineto{\pgfqpoint{1.596198in}{0.760514in}}%
\pgfpathlineto{\pgfqpoint{1.596608in}{0.760298in}}%
\pgfpathlineto{\pgfqpoint{1.597838in}{0.755209in}}%
\pgfpathlineto{\pgfqpoint{1.599478in}{0.740750in}}%
\pgfpathlineto{\pgfqpoint{1.602758in}{0.818102in}}%
\pgfpathlineto{\pgfqpoint{1.604398in}{0.827485in}}%
\pgfpathlineto{\pgfqpoint{1.605218in}{0.824069in}}%
\pgfpathlineto{\pgfqpoint{1.606858in}{0.802179in}}%
\pgfpathlineto{\pgfqpoint{1.609728in}{0.732621in}}%
\pgfpathlineto{\pgfqpoint{1.610958in}{0.741096in}}%
\pgfpathlineto{\pgfqpoint{1.612598in}{0.744955in}}%
\pgfpathlineto{\pgfqpoint{1.613418in}{0.743508in}}%
\pgfpathlineto{\pgfqpoint{1.615058in}{0.734043in}}%
\pgfpathlineto{\pgfqpoint{1.617518in}{0.706468in}}%
\pgfpathlineto{\pgfqpoint{1.618338in}{0.714680in}}%
\pgfpathlineto{\pgfqpoint{1.621208in}{0.737907in}}%
\pgfpathlineto{\pgfqpoint{1.622438in}{0.739824in}}%
\pgfpathlineto{\pgfqpoint{1.623668in}{0.736415in}}%
\pgfpathlineto{\pgfqpoint{1.625308in}{0.725124in}}%
\pgfpathlineto{\pgfqpoint{1.628588in}{0.783071in}}%
\pgfpathlineto{\pgfqpoint{1.629818in}{0.788455in}}%
\pgfpathlineto{\pgfqpoint{1.630228in}{0.788046in}}%
\pgfpathlineto{\pgfqpoint{1.631458in}{0.780425in}}%
\pgfpathlineto{\pgfqpoint{1.633918in}{0.740889in}}%
\pgfpathlineto{\pgfqpoint{1.635148in}{0.720615in}}%
\pgfpathlineto{\pgfqpoint{1.635558in}{0.723029in}}%
\pgfpathlineto{\pgfqpoint{1.637608in}{0.728604in}}%
\pgfpathlineto{\pgfqpoint{1.638428in}{0.727631in}}%
\pgfpathlineto{\pgfqpoint{1.640068in}{0.720372in}}%
\pgfpathlineto{\pgfqpoint{1.642528in}{0.700567in}}%
\pgfpathlineto{\pgfqpoint{1.642938in}{0.704431in}}%
\pgfpathlineto{\pgfqpoint{1.646218in}{0.723658in}}%
\pgfpathlineto{\pgfqpoint{1.647038in}{0.724228in}}%
\pgfpathlineto{\pgfqpoint{1.647448in}{0.723814in}}%
\pgfpathlineto{\pgfqpoint{1.649088in}{0.717760in}}%
\pgfpathlineto{\pgfqpoint{1.649498in}{0.715261in}}%
\pgfpathlineto{\pgfqpoint{1.649908in}{0.715683in}}%
\pgfpathlineto{\pgfqpoint{1.653188in}{0.756521in}}%
\pgfpathlineto{\pgfqpoint{1.654008in}{0.758794in}}%
\pgfpathlineto{\pgfqpoint{1.654418in}{0.758621in}}%
\pgfpathlineto{\pgfqpoint{1.655648in}{0.753038in}}%
\pgfpathlineto{\pgfqpoint{1.657698in}{0.729223in}}%
\pgfpathlineto{\pgfqpoint{1.658928in}{0.709444in}}%
\pgfpathlineto{\pgfqpoint{1.659748in}{0.713063in}}%
\pgfpathlineto{\pgfqpoint{1.661388in}{0.716141in}}%
\pgfpathlineto{\pgfqpoint{1.661798in}{0.715997in}}%
\pgfpathlineto{\pgfqpoint{1.663028in}{0.713405in}}%
\pgfpathlineto{\pgfqpoint{1.665488in}{0.699973in}}%
\pgfpathlineto{\pgfqpoint{1.665898in}{0.696978in}}%
\pgfpathlineto{\pgfqpoint{1.666308in}{0.697207in}}%
\pgfpathlineto{\pgfqpoint{1.669588in}{0.712148in}}%
\pgfpathlineto{\pgfqpoint{1.670818in}{0.712373in}}%
\pgfpathlineto{\pgfqpoint{1.672458in}{0.707894in}}%
\pgfpathlineto{\pgfqpoint{1.672868in}{0.706028in}}%
\pgfpathlineto{\pgfqpoint{1.673278in}{0.707672in}}%
\pgfpathlineto{\pgfqpoint{1.676148in}{0.734512in}}%
\pgfpathlineto{\pgfqpoint{1.677378in}{0.736766in}}%
\pgfpathlineto{\pgfqpoint{1.678608in}{0.733063in}}%
\pgfpathlineto{\pgfqpoint{1.680658in}{0.715643in}}%
\pgfpathlineto{\pgfqpoint{1.681888in}{0.702585in}}%
\pgfpathlineto{\pgfqpoint{1.682298in}{0.703970in}}%
\pgfpathlineto{\pgfqpoint{1.684348in}{0.706841in}}%
\pgfpathlineto{\pgfqpoint{1.685578in}{0.705197in}}%
\pgfpathlineto{\pgfqpoint{1.687628in}{0.697608in}}%
\pgfpathlineto{\pgfqpoint{1.688448in}{0.693365in}}%
\pgfpathlineto{\pgfqpoint{1.688858in}{0.693744in}}%
\pgfpathlineto{\pgfqpoint{1.691728in}{0.703581in}}%
\pgfpathlineto{\pgfqpoint{1.693368in}{0.703776in}}%
\pgfpathlineto{\pgfqpoint{1.695418in}{0.699957in}}%
\pgfpathlineto{\pgfqpoint{1.698288in}{0.719354in}}%
\pgfpathlineto{\pgfqpoint{1.699518in}{0.720677in}}%
\pgfpathlineto{\pgfqpoint{1.700748in}{0.717537in}}%
\pgfpathlineto{\pgfqpoint{1.703208in}{0.700582in}}%
\pgfpathlineto{\pgfqpoint{1.703618in}{0.696921in}}%
\pgfpathlineto{\pgfqpoint{1.704438in}{0.698804in}}%
\pgfpathlineto{\pgfqpoint{1.706078in}{0.700097in}}%
\pgfpathlineto{\pgfqpoint{1.707718in}{0.698036in}}%
\pgfpathlineto{\pgfqpoint{1.710178in}{0.690316in}}%
\pgfpathlineto{\pgfqpoint{1.710588in}{0.691793in}}%
\pgfpathlineto{\pgfqpoint{1.713458in}{0.698005in}}%
\pgfpathlineto{\pgfqpoint{1.715098in}{0.697449in}}%
\pgfpathlineto{\pgfqpoint{1.716328in}{0.695139in}}%
\pgfpathlineto{\pgfqpoint{1.716738in}{0.695676in}}%
\pgfpathlineto{\pgfqpoint{1.719608in}{0.708709in}}%
\pgfpathlineto{\pgfqpoint{1.720838in}{0.709061in}}%
\pgfpathlineto{\pgfqpoint{1.722478in}{0.704578in}}%
\pgfpathlineto{\pgfqpoint{1.724528in}{0.693162in}}%
\pgfpathlineto{\pgfqpoint{1.725348in}{0.694476in}}%
\pgfpathlineto{\pgfqpoint{1.727398in}{0.695074in}}%
\pgfpathlineto{\pgfqpoint{1.729448in}{0.692009in}}%
\pgfpathlineto{\pgfqpoint{1.730678in}{0.688935in}}%
\pgfpathlineto{\pgfqpoint{1.731088in}{0.689369in}}%
\pgfpathlineto{\pgfqpoint{1.733958in}{0.693806in}}%
\pgfpathlineto{\pgfqpoint{1.735598in}{0.693378in}}%
\pgfpathlineto{\pgfqpoint{1.736828in}{0.691705in}}%
\pgfpathlineto{\pgfqpoint{1.737238in}{0.692795in}}%
\pgfpathlineto{\pgfqpoint{1.740108in}{0.701118in}}%
\pgfpathlineto{\pgfqpoint{1.741338in}{0.700845in}}%
\pgfpathlineto{\pgfqpoint{1.742978in}{0.697033in}}%
\pgfpathlineto{\pgfqpoint{1.744618in}{0.690564in}}%
\pgfpathlineto{\pgfqpoint{1.745438in}{0.691450in}}%
\pgfpathlineto{\pgfqpoint{1.747488in}{0.691723in}}%
\pgfpathlineto{\pgfqpoint{1.749948in}{0.688729in}}%
\pgfpathlineto{\pgfqpoint{1.750768in}{0.687776in}}%
\pgfpathlineto{\pgfqpoint{1.751178in}{0.688462in}}%
\pgfpathlineto{\pgfqpoint{1.754048in}{0.690977in}}%
\pgfpathlineto{\pgfqpoint{1.756098in}{0.689888in}}%
\pgfpathlineto{\pgfqpoint{1.756508in}{0.689418in}}%
\pgfpathlineto{\pgfqpoint{1.756918in}{0.690550in}}%
\pgfpathlineto{\pgfqpoint{1.759788in}{0.695749in}}%
\pgfpathlineto{\pgfqpoint{1.761428in}{0.694708in}}%
\pgfpathlineto{\pgfqpoint{1.765528in}{0.689625in}}%
\pgfpathlineto{\pgfqpoint{1.767578in}{0.688989in}}%
\pgfpathlineto{\pgfqpoint{1.770448in}{0.687596in}}%
\pgfpathlineto{\pgfqpoint{1.773318in}{0.688978in}}%
\pgfpathlineto{\pgfqpoint{1.776188in}{0.689548in}}%
\pgfpathlineto{\pgfqpoint{1.778648in}{0.692055in}}%
\pgfpathlineto{\pgfqpoint{1.780288in}{0.691173in}}%
\pgfpathlineto{\pgfqpoint{1.783978in}{0.688081in}}%
\pgfpathlineto{\pgfqpoint{1.786848in}{0.687198in}}%
\pgfpathlineto{\pgfqpoint{1.788898in}{0.686847in}}%
\pgfpathlineto{\pgfqpoint{1.792178in}{0.687566in}}%
\pgfpathlineto{\pgfqpoint{1.794228in}{0.687869in}}%
\pgfpathlineto{\pgfqpoint{1.797098in}{0.689573in}}%
\pgfpathlineto{\pgfqpoint{1.799558in}{0.687815in}}%
\pgfpathlineto{\pgfqpoint{1.801608in}{0.687044in}}%
\pgfpathlineto{\pgfqpoint{1.804888in}{0.686398in}}%
\pgfpathlineto{\pgfqpoint{1.806938in}{0.686419in}}%
\pgfpathlineto{\pgfqpoint{1.810628in}{0.686550in}}%
\pgfpathlineto{\pgfqpoint{1.811858in}{0.686999in}}%
\pgfpathlineto{\pgfqpoint{1.814728in}{0.687953in}}%
\pgfpathlineto{\pgfqpoint{1.827028in}{0.686182in}}%
\pgfpathlineto{\pgfqpoint{1.829898in}{0.686794in}}%
\pgfpathlineto{\pgfqpoint{1.833178in}{0.686523in}}%
\pgfpathlineto{\pgfqpoint{1.837278in}{0.685959in}}%
\pgfpathlineto{\pgfqpoint{1.845478in}{0.686040in}}%
\pgfpathlineto{\pgfqpoint{1.849578in}{0.686028in}}%
\pgfpathlineto{\pgfqpoint{1.855318in}{0.685532in}}%
\pgfpathlineto{\pgfqpoint{1.863518in}{0.685931in}}%
\pgfpathlineto{\pgfqpoint{1.905748in}{0.685414in}}%
\pgfpathlineto{\pgfqpoint{1.931578in}{0.685382in}}%
\pgfpathlineto{\pgfqpoint{2.952066in}{0.685365in}}%
\pgfpathlineto{\pgfqpoint{2.952066in}{0.685365in}}%
\pgfusepath{stroke}%
\end{pgfscope}%
\begin{pgfscope}%
\pgfpathrectangle{\pgfqpoint{0.800000in}{0.528000in}}{\pgfqpoint{2.254545in}{1.680000in}}%
\pgfusepath{clip}%
\pgfsetrectcap%
\pgfsetroundjoin%
\pgfsetlinewidth{1.505625pt}%
\definecolor{currentstroke}{rgb}{0.090196,0.745098,0.811765}%
\pgfsetstrokecolor{currentstroke}%
\pgfsetdash{}{0pt}%
\pgfpathmoveto{\pgfqpoint{0.902479in}{1.531273in}}%
\pgfpathlineto{\pgfqpoint{0.908219in}{1.530286in}}%
\pgfpathlineto{\pgfqpoint{0.913959in}{1.527033in}}%
\pgfpathlineto{\pgfqpoint{0.919699in}{1.521123in}}%
\pgfpathlineto{\pgfqpoint{0.925439in}{1.511912in}}%
\pgfpathlineto{\pgfqpoint{0.931589in}{1.497369in}}%
\pgfpathlineto{\pgfqpoint{0.937739in}{1.476658in}}%
\pgfpathlineto{\pgfqpoint{0.944299in}{1.445991in}}%
\pgfpathlineto{\pgfqpoint{0.951269in}{1.401478in}}%
\pgfpathlineto{\pgfqpoint{0.958649in}{1.338438in}}%
\pgfpathlineto{\pgfqpoint{0.963159in}{1.295271in}}%
\pgfpathlineto{\pgfqpoint{0.982839in}{1.694357in}}%
\pgfpathlineto{\pgfqpoint{0.996779in}{1.956231in}}%
\pgfpathlineto{\pgfqpoint{1.004979in}{2.073414in}}%
\pgfpathlineto{\pgfqpoint{1.011129in}{2.133107in}}%
\pgfpathlineto{\pgfqpoint{1.015639in}{2.158097in}}%
\pgfpathlineto{\pgfqpoint{1.018509in}{2.164770in}}%
\pgfpathlineto{\pgfqpoint{1.020149in}{2.165164in}}%
\pgfpathlineto{\pgfqpoint{1.021789in}{2.162986in}}%
\pgfpathlineto{\pgfqpoint{1.024249in}{2.154757in}}%
\pgfpathlineto{\pgfqpoint{1.027529in}{2.134235in}}%
\pgfpathlineto{\pgfqpoint{1.031629in}{2.092727in}}%
\pgfpathlineto{\pgfqpoint{1.036549in}{2.019006in}}%
\pgfpathlineto{\pgfqpoint{1.042699in}{1.889629in}}%
\pgfpathlineto{\pgfqpoint{1.049669in}{1.693611in}}%
\pgfpathlineto{\pgfqpoint{1.058279in}{1.383785in}}%
\pgfpathlineto{\pgfqpoint{1.065659in}{1.073443in}}%
\pgfpathlineto{\pgfqpoint{1.066479in}{1.081171in}}%
\pgfpathlineto{\pgfqpoint{1.077549in}{1.174740in}}%
\pgfpathlineto{\pgfqpoint{1.084519in}{1.217223in}}%
\pgfpathlineto{\pgfqpoint{1.089029in}{1.233184in}}%
\pgfpathlineto{\pgfqpoint{1.091899in}{1.236704in}}%
\pgfpathlineto{\pgfqpoint{1.093539in}{1.235848in}}%
\pgfpathlineto{\pgfqpoint{1.095589in}{1.231355in}}%
\pgfpathlineto{\pgfqpoint{1.098459in}{1.217629in}}%
\pgfpathlineto{\pgfqpoint{1.101739in}{1.189190in}}%
\pgfpathlineto{\pgfqpoint{1.105839in}{1.130051in}}%
\pgfpathlineto{\pgfqpoint{1.110349in}{1.027972in}}%
\pgfpathlineto{\pgfqpoint{1.114039in}{0.911489in}}%
\pgfpathlineto{\pgfqpoint{1.114859in}{0.927156in}}%
\pgfpathlineto{\pgfqpoint{1.131259in}{1.429597in}}%
\pgfpathlineto{\pgfqpoint{1.134129in}{1.457332in}}%
\pgfpathlineto{\pgfqpoint{1.134949in}{1.458743in}}%
\pgfpathlineto{\pgfqpoint{1.135359in}{1.458295in}}%
\pgfpathlineto{\pgfqpoint{1.136589in}{1.452239in}}%
\pgfpathlineto{\pgfqpoint{1.138639in}{1.426217in}}%
\pgfpathlineto{\pgfqpoint{1.141509in}{1.357125in}}%
\pgfpathlineto{\pgfqpoint{1.145199in}{1.233128in}}%
\pgfpathlineto{\pgfqpoint{1.152989in}{1.738165in}}%
\pgfpathlineto{\pgfqpoint{1.157909in}{1.930563in}}%
\pgfpathlineto{\pgfqpoint{1.161189in}{1.992122in}}%
\pgfpathlineto{\pgfqpoint{1.163239in}{2.002664in}}%
\pgfpathlineto{\pgfqpoint{1.164059in}{2.000917in}}%
\pgfpathlineto{\pgfqpoint{1.165699in}{1.987378in}}%
\pgfpathlineto{\pgfqpoint{1.168159in}{1.942669in}}%
\pgfpathlineto{\pgfqpoint{1.171849in}{1.824045in}}%
\pgfpathlineto{\pgfqpoint{1.176769in}{1.581078in}}%
\pgfpathlineto{\pgfqpoint{1.183739in}{1.111071in}}%
\pgfpathlineto{\pgfqpoint{1.184969in}{1.045131in}}%
\pgfpathlineto{\pgfqpoint{1.185379in}{1.051792in}}%
\pgfpathlineto{\pgfqpoint{1.191939in}{1.140749in}}%
\pgfpathlineto{\pgfqpoint{1.196039in}{1.173497in}}%
\pgfpathlineto{\pgfqpoint{1.198499in}{1.180802in}}%
\pgfpathlineto{\pgfqpoint{1.199729in}{1.180062in}}%
\pgfpathlineto{\pgfqpoint{1.201369in}{1.173731in}}%
\pgfpathlineto{\pgfqpoint{1.203829in}{1.150847in}}%
\pgfpathlineto{\pgfqpoint{1.206699in}{1.099709in}}%
\pgfpathlineto{\pgfqpoint{1.210389in}{0.987310in}}%
\pgfpathlineto{\pgfqpoint{1.212849in}{0.880354in}}%
\pgfpathlineto{\pgfqpoint{1.213259in}{0.896942in}}%
\pgfpathlineto{\pgfqpoint{1.223509in}{1.306089in}}%
\pgfpathlineto{\pgfqpoint{1.225969in}{1.339997in}}%
\pgfpathlineto{\pgfqpoint{1.226789in}{1.341213in}}%
\pgfpathlineto{\pgfqpoint{1.227609in}{1.337046in}}%
\pgfpathlineto{\pgfqpoint{1.229249in}{1.312380in}}%
\pgfpathlineto{\pgfqpoint{1.231709in}{1.236420in}}%
\pgfpathlineto{\pgfqpoint{1.233759in}{1.153148in}}%
\pgfpathlineto{\pgfqpoint{1.239909in}{1.632879in}}%
\pgfpathlineto{\pgfqpoint{1.243599in}{1.784891in}}%
\pgfpathlineto{\pgfqpoint{1.246059in}{1.820287in}}%
\pgfpathlineto{\pgfqpoint{1.246469in}{1.821037in}}%
\pgfpathlineto{\pgfqpoint{1.246879in}{1.820336in}}%
\pgfpathlineto{\pgfqpoint{1.248109in}{1.809657in}}%
\pgfpathlineto{\pgfqpoint{1.250159in}{1.764248in}}%
\pgfpathlineto{\pgfqpoint{1.253439in}{1.625738in}}%
\pgfpathlineto{\pgfqpoint{1.257949in}{1.326057in}}%
\pgfpathlineto{\pgfqpoint{1.262049in}{1.007207in}}%
\pgfpathlineto{\pgfqpoint{1.262869in}{1.023184in}}%
\pgfpathlineto{\pgfqpoint{1.267789in}{1.097037in}}%
\pgfpathlineto{\pgfqpoint{1.270659in}{1.117480in}}%
\pgfpathlineto{\pgfqpoint{1.271889in}{1.119436in}}%
\pgfpathlineto{\pgfqpoint{1.272299in}{1.119036in}}%
\pgfpathlineto{\pgfqpoint{1.273529in}{1.114375in}}%
\pgfpathlineto{\pgfqpoint{1.275579in}{1.093650in}}%
\pgfpathlineto{\pgfqpoint{1.278449in}{1.032432in}}%
\pgfpathlineto{\pgfqpoint{1.281729in}{0.909082in}}%
\pgfpathlineto{\pgfqpoint{1.282959in}{0.856027in}}%
\pgfpathlineto{\pgfqpoint{1.283369in}{0.875594in}}%
\pgfpathlineto{\pgfqpoint{1.291159in}{1.198040in}}%
\pgfpathlineto{\pgfqpoint{1.293619in}{1.227993in}}%
\pgfpathlineto{\pgfqpoint{1.294439in}{1.225276in}}%
\pgfpathlineto{\pgfqpoint{1.296079in}{1.199907in}}%
\pgfpathlineto{\pgfqpoint{1.298539in}{1.115334in}}%
\pgfpathlineto{\pgfqpoint{1.299359in}{1.076685in}}%
\pgfpathlineto{\pgfqpoint{1.299769in}{1.097012in}}%
\pgfpathlineto{\pgfqpoint{1.305099in}{1.516058in}}%
\pgfpathlineto{\pgfqpoint{1.308379in}{1.630204in}}%
\pgfpathlineto{\pgfqpoint{1.309609in}{1.641216in}}%
\pgfpathlineto{\pgfqpoint{1.310019in}{1.641054in}}%
\pgfpathlineto{\pgfqpoint{1.311249in}{1.629314in}}%
\pgfpathlineto{\pgfqpoint{1.313299in}{1.573844in}}%
\pgfpathlineto{\pgfqpoint{1.316579in}{1.402164in}}%
\pgfpathlineto{\pgfqpoint{1.321499in}{1.003797in}}%
\pgfpathlineto{\pgfqpoint{1.321909in}{0.965678in}}%
\pgfpathlineto{\pgfqpoint{1.322729in}{0.976968in}}%
\pgfpathlineto{\pgfqpoint{1.326829in}{1.041641in}}%
\pgfpathlineto{\pgfqpoint{1.329289in}{1.056801in}}%
\pgfpathlineto{\pgfqpoint{1.330109in}{1.056863in}}%
\pgfpathlineto{\pgfqpoint{1.331339in}{1.051548in}}%
\pgfpathlineto{\pgfqpoint{1.333389in}{1.026398in}}%
\pgfpathlineto{\pgfqpoint{1.336259in}{0.951686in}}%
\pgfpathlineto{\pgfqpoint{1.339129in}{0.827575in}}%
\pgfpathlineto{\pgfqpoint{1.339539in}{0.842500in}}%
\pgfpathlineto{\pgfqpoint{1.346099in}{1.100692in}}%
\pgfpathlineto{\pgfqpoint{1.348149in}{1.125063in}}%
\pgfpathlineto{\pgfqpoint{1.348559in}{1.124925in}}%
\pgfpathlineto{\pgfqpoint{1.349379in}{1.119389in}}%
\pgfpathlineto{\pgfqpoint{1.351019in}{1.087513in}}%
\pgfpathlineto{\pgfqpoint{1.353479in}{1.009783in}}%
\pgfpathlineto{\pgfqpoint{1.358399in}{1.381786in}}%
\pgfpathlineto{\pgfqpoint{1.361269in}{1.468250in}}%
\pgfpathlineto{\pgfqpoint{1.362089in}{1.473129in}}%
\pgfpathlineto{\pgfqpoint{1.362499in}{1.472288in}}%
\pgfpathlineto{\pgfqpoint{1.363729in}{1.456935in}}%
\pgfpathlineto{\pgfqpoint{1.365779in}{1.390861in}}%
\pgfpathlineto{\pgfqpoint{1.369059in}{1.195202in}}%
\pgfpathlineto{\pgfqpoint{1.372339in}{0.921669in}}%
\pgfpathlineto{\pgfqpoint{1.373159in}{0.935683in}}%
\pgfpathlineto{\pgfqpoint{1.376849in}{0.989344in}}%
\pgfpathlineto{\pgfqpoint{1.378899in}{0.996828in}}%
\pgfpathlineto{\pgfqpoint{1.379719in}{0.994373in}}%
\pgfpathlineto{\pgfqpoint{1.381359in}{0.978874in}}%
\pgfpathlineto{\pgfqpoint{1.383819in}{0.926244in}}%
\pgfpathlineto{\pgfqpoint{1.387099in}{0.801436in}}%
\pgfpathlineto{\pgfqpoint{1.387509in}{0.819409in}}%
\pgfpathlineto{\pgfqpoint{1.392839in}{1.010825in}}%
\pgfpathlineto{\pgfqpoint{1.394889in}{1.035171in}}%
\pgfpathlineto{\pgfqpoint{1.395299in}{1.035198in}}%
\pgfpathlineto{\pgfqpoint{1.396119in}{1.030160in}}%
\pgfpathlineto{\pgfqpoint{1.397759in}{1.000085in}}%
\pgfpathlineto{\pgfqpoint{1.399399in}{0.946145in}}%
\pgfpathlineto{\pgfqpoint{1.399809in}{0.950470in}}%
\pgfpathlineto{\pgfqpoint{1.404319in}{1.257105in}}%
\pgfpathlineto{\pgfqpoint{1.406779in}{1.318893in}}%
\pgfpathlineto{\pgfqpoint{1.407189in}{1.321109in}}%
\pgfpathlineto{\pgfqpoint{1.407599in}{1.321025in}}%
\pgfpathlineto{\pgfqpoint{1.408419in}{1.314058in}}%
\pgfpathlineto{\pgfqpoint{1.410059in}{1.274011in}}%
\pgfpathlineto{\pgfqpoint{1.412929in}{1.130252in}}%
\pgfpathlineto{\pgfqpoint{1.416209in}{0.883588in}}%
\pgfpathlineto{\pgfqpoint{1.417439in}{0.901536in}}%
\pgfpathlineto{\pgfqpoint{1.420719in}{0.938787in}}%
\pgfpathlineto{\pgfqpoint{1.421539in}{0.940704in}}%
\pgfpathlineto{\pgfqpoint{1.421949in}{0.940405in}}%
\pgfpathlineto{\pgfqpoint{1.423179in}{0.934170in}}%
\pgfpathlineto{\pgfqpoint{1.425229in}{0.904650in}}%
\pgfpathlineto{\pgfqpoint{1.428099in}{0.821443in}}%
\pgfpathlineto{\pgfqpoint{1.429329in}{0.782838in}}%
\pgfpathlineto{\pgfqpoint{1.429739in}{0.798778in}}%
\pgfpathlineto{\pgfqpoint{1.434659in}{0.945427in}}%
\pgfpathlineto{\pgfqpoint{1.436298in}{0.958986in}}%
\pgfpathlineto{\pgfqpoint{1.436708in}{0.958614in}}%
\pgfpathlineto{\pgfqpoint{1.437938in}{0.948216in}}%
\pgfpathlineto{\pgfqpoint{1.439988in}{0.901837in}}%
\pgfpathlineto{\pgfqpoint{1.440398in}{0.888795in}}%
\pgfpathlineto{\pgfqpoint{1.440808in}{0.898724in}}%
\pgfpathlineto{\pgfqpoint{1.444908in}{1.141076in}}%
\pgfpathlineto{\pgfqpoint{1.447368in}{1.188295in}}%
\pgfpathlineto{\pgfqpoint{1.447778in}{1.188225in}}%
\pgfpathlineto{\pgfqpoint{1.448598in}{1.181386in}}%
\pgfpathlineto{\pgfqpoint{1.450238in}{1.142101in}}%
\pgfpathlineto{\pgfqpoint{1.453108in}{1.002802in}}%
\pgfpathlineto{\pgfqpoint{1.455568in}{0.843276in}}%
\pgfpathlineto{\pgfqpoint{1.456388in}{0.858034in}}%
\pgfpathlineto{\pgfqpoint{1.459258in}{0.888279in}}%
\pgfpathlineto{\pgfqpoint{1.460078in}{0.889971in}}%
\pgfpathlineto{\pgfqpoint{1.460488in}{0.889537in}}%
\pgfpathlineto{\pgfqpoint{1.461718in}{0.882865in}}%
\pgfpathlineto{\pgfqpoint{1.463768in}{0.853044in}}%
\pgfpathlineto{\pgfqpoint{1.467048in}{0.758805in}}%
\pgfpathlineto{\pgfqpoint{1.467868in}{0.784594in}}%
\pgfpathlineto{\pgfqpoint{1.471968in}{0.884695in}}%
\pgfpathlineto{\pgfqpoint{1.473608in}{0.895630in}}%
\pgfpathlineto{\pgfqpoint{1.474428in}{0.893100in}}%
\pgfpathlineto{\pgfqpoint{1.476068in}{0.872114in}}%
\pgfpathlineto{\pgfqpoint{1.477298in}{0.843857in}}%
\pgfpathlineto{\pgfqpoint{1.477708in}{0.846977in}}%
\pgfpathlineto{\pgfqpoint{1.481808in}{1.047307in}}%
\pgfpathlineto{\pgfqpoint{1.483858in}{1.075505in}}%
\pgfpathlineto{\pgfqpoint{1.484678in}{1.071971in}}%
\pgfpathlineto{\pgfqpoint{1.486318in}{1.040649in}}%
\pgfpathlineto{\pgfqpoint{1.489188in}{0.919566in}}%
\pgfpathlineto{\pgfqpoint{1.491238in}{0.811009in}}%
\pgfpathlineto{\pgfqpoint{1.491648in}{0.817605in}}%
\pgfpathlineto{\pgfqpoint{1.494518in}{0.844655in}}%
\pgfpathlineto{\pgfqpoint{1.495338in}{0.845524in}}%
\pgfpathlineto{\pgfqpoint{1.495748in}{0.844714in}}%
\pgfpathlineto{\pgfqpoint{1.496978in}{0.837153in}}%
\pgfpathlineto{\pgfqpoint{1.499028in}{0.807310in}}%
\pgfpathlineto{\pgfqpoint{1.501488in}{0.745988in}}%
\pgfpathlineto{\pgfqpoint{1.502308in}{0.759750in}}%
\pgfpathlineto{\pgfqpoint{1.505998in}{0.834494in}}%
\pgfpathlineto{\pgfqpoint{1.507638in}{0.844107in}}%
\pgfpathlineto{\pgfqpoint{1.508458in}{0.842118in}}%
\pgfpathlineto{\pgfqpoint{1.510098in}{0.824709in}}%
\pgfpathlineto{\pgfqpoint{1.511328in}{0.801220in}}%
\pgfpathlineto{\pgfqpoint{1.516248in}{0.975892in}}%
\pgfpathlineto{\pgfqpoint{1.517478in}{0.981657in}}%
\pgfpathlineto{\pgfqpoint{1.518298in}{0.976078in}}%
\pgfpathlineto{\pgfqpoint{1.519938in}{0.943649in}}%
\pgfpathlineto{\pgfqpoint{1.522808in}{0.831366in}}%
\pgfpathlineto{\pgfqpoint{1.524038in}{0.782806in}}%
\pgfpathlineto{\pgfqpoint{1.524448in}{0.788260in}}%
\pgfpathlineto{\pgfqpoint{1.526908in}{0.807293in}}%
\pgfpathlineto{\pgfqpoint{1.527728in}{0.807914in}}%
\pgfpathlineto{\pgfqpoint{1.528958in}{0.803100in}}%
\pgfpathlineto{\pgfqpoint{1.531008in}{0.779805in}}%
\pgfpathlineto{\pgfqpoint{1.533468in}{0.730115in}}%
\pgfpathlineto{\pgfqpoint{1.534288in}{0.743540in}}%
\pgfpathlineto{\pgfqpoint{1.537978in}{0.799010in}}%
\pgfpathlineto{\pgfqpoint{1.539208in}{0.803129in}}%
\pgfpathlineto{\pgfqpoint{1.539618in}{0.802615in}}%
\pgfpathlineto{\pgfqpoint{1.540848in}{0.795451in}}%
\pgfpathlineto{\pgfqpoint{1.542488in}{0.773919in}}%
\pgfpathlineto{\pgfqpoint{1.542898in}{0.778758in}}%
\pgfpathlineto{\pgfqpoint{1.546588in}{0.893809in}}%
\pgfpathlineto{\pgfqpoint{1.548228in}{0.906238in}}%
\pgfpathlineto{\pgfqpoint{1.549048in}{0.902663in}}%
\pgfpathlineto{\pgfqpoint{1.550688in}{0.877057in}}%
\pgfpathlineto{\pgfqpoint{1.553558in}{0.784834in}}%
\pgfpathlineto{\pgfqpoint{1.554378in}{0.757786in}}%
\pgfpathlineto{\pgfqpoint{1.554788in}{0.762296in}}%
\pgfpathlineto{\pgfqpoint{1.557248in}{0.776965in}}%
\pgfpathlineto{\pgfqpoint{1.558068in}{0.776717in}}%
\pgfpathlineto{\pgfqpoint{1.559298in}{0.771318in}}%
\pgfpathlineto{\pgfqpoint{1.561348in}{0.749543in}}%
\pgfpathlineto{\pgfqpoint{1.563398in}{0.718206in}}%
\pgfpathlineto{\pgfqpoint{1.563808in}{0.725345in}}%
\pgfpathlineto{\pgfqpoint{1.567498in}{0.768599in}}%
\pgfpathlineto{\pgfqpoint{1.568728in}{0.771126in}}%
\pgfpathlineto{\pgfqpoint{1.569548in}{0.768996in}}%
\pgfpathlineto{\pgfqpoint{1.571188in}{0.756084in}}%
\pgfpathlineto{\pgfqpoint{1.572008in}{0.750319in}}%
\pgfpathlineto{\pgfqpoint{1.575698in}{0.839201in}}%
\pgfpathlineto{\pgfqpoint{1.576928in}{0.846595in}}%
\pgfpathlineto{\pgfqpoint{1.577338in}{0.846328in}}%
\pgfpathlineto{\pgfqpoint{1.578568in}{0.837550in}}%
\pgfpathlineto{\pgfqpoint{1.580618in}{0.799142in}}%
\pgfpathlineto{\pgfqpoint{1.582668in}{0.739346in}}%
\pgfpathlineto{\pgfqpoint{1.583488in}{0.743952in}}%
\pgfpathlineto{\pgfqpoint{1.585538in}{0.752479in}}%
\pgfpathlineto{\pgfqpoint{1.586358in}{0.752024in}}%
\pgfpathlineto{\pgfqpoint{1.587588in}{0.747138in}}%
\pgfpathlineto{\pgfqpoint{1.590048in}{0.723727in}}%
\pgfpathlineto{\pgfqpoint{1.591278in}{0.709058in}}%
\pgfpathlineto{\pgfqpoint{1.591688in}{0.714519in}}%
\pgfpathlineto{\pgfqpoint{1.594968in}{0.744328in}}%
\pgfpathlineto{\pgfqpoint{1.596198in}{0.746787in}}%
\pgfpathlineto{\pgfqpoint{1.596608in}{0.746401in}}%
\pgfpathlineto{\pgfqpoint{1.597838in}{0.741712in}}%
\pgfpathlineto{\pgfqpoint{1.599068in}{0.732208in}}%
\pgfpathlineto{\pgfqpoint{1.599478in}{0.733051in}}%
\pgfpathlineto{\pgfqpoint{1.602758in}{0.794186in}}%
\pgfpathlineto{\pgfqpoint{1.604398in}{0.800684in}}%
\pgfpathlineto{\pgfqpoint{1.605628in}{0.794135in}}%
\pgfpathlineto{\pgfqpoint{1.607678in}{0.764121in}}%
\pgfpathlineto{\pgfqpoint{1.609728in}{0.724309in}}%
\pgfpathlineto{\pgfqpoint{1.610138in}{0.726902in}}%
\pgfpathlineto{\pgfqpoint{1.612188in}{0.733470in}}%
\pgfpathlineto{\pgfqpoint{1.613008in}{0.732926in}}%
\pgfpathlineto{\pgfqpoint{1.614648in}{0.726468in}}%
\pgfpathlineto{\pgfqpoint{1.617518in}{0.701612in}}%
\pgfpathlineto{\pgfqpoint{1.618338in}{0.709591in}}%
\pgfpathlineto{\pgfqpoint{1.621208in}{0.727275in}}%
\pgfpathlineto{\pgfqpoint{1.622438in}{0.728438in}}%
\pgfpathlineto{\pgfqpoint{1.623668in}{0.725457in}}%
\pgfpathlineto{\pgfqpoint{1.625308in}{0.717845in}}%
\pgfpathlineto{\pgfqpoint{1.628588in}{0.762602in}}%
\pgfpathlineto{\pgfqpoint{1.629818in}{0.766425in}}%
\pgfpathlineto{\pgfqpoint{1.630228in}{0.765969in}}%
\pgfpathlineto{\pgfqpoint{1.631458in}{0.759612in}}%
\pgfpathlineto{\pgfqpoint{1.633918in}{0.728095in}}%
\pgfpathlineto{\pgfqpoint{1.635148in}{0.713222in}}%
\pgfpathlineto{\pgfqpoint{1.635558in}{0.715065in}}%
\pgfpathlineto{\pgfqpoint{1.637608in}{0.719112in}}%
\pgfpathlineto{\pgfqpoint{1.638838in}{0.717231in}}%
\pgfpathlineto{\pgfqpoint{1.640888in}{0.707578in}}%
\pgfpathlineto{\pgfqpoint{1.642528in}{0.697432in}}%
\pgfpathlineto{\pgfqpoint{1.642938in}{0.700382in}}%
\pgfpathlineto{\pgfqpoint{1.645808in}{0.714076in}}%
\pgfpathlineto{\pgfqpoint{1.647038in}{0.715117in}}%
\pgfpathlineto{\pgfqpoint{1.647448in}{0.714760in}}%
\pgfpathlineto{\pgfqpoint{1.649088in}{0.710031in}}%
\pgfpathlineto{\pgfqpoint{1.649498in}{0.708113in}}%
\pgfpathlineto{\pgfqpoint{1.649908in}{0.708668in}}%
\pgfpathlineto{\pgfqpoint{1.653188in}{0.739495in}}%
\pgfpathlineto{\pgfqpoint{1.654008in}{0.741149in}}%
\pgfpathlineto{\pgfqpoint{1.654418in}{0.740983in}}%
\pgfpathlineto{\pgfqpoint{1.655648in}{0.736657in}}%
\pgfpathlineto{\pgfqpoint{1.658108in}{0.713677in}}%
\pgfpathlineto{\pgfqpoint{1.658928in}{0.703580in}}%
\pgfpathlineto{\pgfqpoint{1.659748in}{0.706294in}}%
\pgfpathlineto{\pgfqpoint{1.661798in}{0.708420in}}%
\pgfpathlineto{\pgfqpoint{1.663028in}{0.706414in}}%
\pgfpathlineto{\pgfqpoint{1.665488in}{0.696302in}}%
\pgfpathlineto{\pgfqpoint{1.665898in}{0.694068in}}%
\pgfpathlineto{\pgfqpoint{1.666308in}{0.694161in}}%
\pgfpathlineto{\pgfqpoint{1.669588in}{0.705177in}}%
\pgfpathlineto{\pgfqpoint{1.670818in}{0.705334in}}%
\pgfpathlineto{\pgfqpoint{1.672458in}{0.702057in}}%
\pgfpathlineto{\pgfqpoint{1.672868in}{0.700695in}}%
\pgfpathlineto{\pgfqpoint{1.673278in}{0.701454in}}%
\pgfpathlineto{\pgfqpoint{1.676558in}{0.722357in}}%
\pgfpathlineto{\pgfqpoint{1.677378in}{0.723018in}}%
\pgfpathlineto{\pgfqpoint{1.677788in}{0.722614in}}%
\pgfpathlineto{\pgfqpoint{1.679018in}{0.718619in}}%
\pgfpathlineto{\pgfqpoint{1.681888in}{0.697782in}}%
\pgfpathlineto{\pgfqpoint{1.683118in}{0.700271in}}%
\pgfpathlineto{\pgfqpoint{1.684758in}{0.700761in}}%
\pgfpathlineto{\pgfqpoint{1.686398in}{0.698078in}}%
\pgfpathlineto{\pgfqpoint{1.688858in}{0.691178in}}%
\pgfpathlineto{\pgfqpoint{1.689268in}{0.692622in}}%
\pgfpathlineto{\pgfqpoint{1.692138in}{0.698636in}}%
\pgfpathlineto{\pgfqpoint{1.693778in}{0.698109in}}%
\pgfpathlineto{\pgfqpoint{1.695418in}{0.695150in}}%
\pgfpathlineto{\pgfqpoint{1.698288in}{0.709173in}}%
\pgfpathlineto{\pgfqpoint{1.699518in}{0.710282in}}%
\pgfpathlineto{\pgfqpoint{1.699928in}{0.709942in}}%
\pgfpathlineto{\pgfqpoint{1.701568in}{0.705316in}}%
\pgfpathlineto{\pgfqpoint{1.704028in}{0.694024in}}%
\pgfpathlineto{\pgfqpoint{1.704848in}{0.695114in}}%
\pgfpathlineto{\pgfqpoint{1.706488in}{0.695526in}}%
\pgfpathlineto{\pgfqpoint{1.708538in}{0.692875in}}%
\pgfpathlineto{\pgfqpoint{1.710178in}{0.688928in}}%
\pgfpathlineto{\pgfqpoint{1.710588in}{0.689544in}}%
\pgfpathlineto{\pgfqpoint{1.713458in}{0.693977in}}%
\pgfpathlineto{\pgfqpoint{1.715098in}{0.693721in}}%
\pgfpathlineto{\pgfqpoint{1.716738in}{0.691785in}}%
\pgfpathlineto{\pgfqpoint{1.719608in}{0.701026in}}%
\pgfpathlineto{\pgfqpoint{1.721248in}{0.701126in}}%
\pgfpathlineto{\pgfqpoint{1.722888in}{0.697496in}}%
\pgfpathlineto{\pgfqpoint{1.724938in}{0.690930in}}%
\pgfpathlineto{\pgfqpoint{1.725758in}{0.691674in}}%
\pgfpathlineto{\pgfqpoint{1.727808in}{0.691708in}}%
\pgfpathlineto{\pgfqpoint{1.730268in}{0.688719in}}%
\pgfpathlineto{\pgfqpoint{1.731088in}{0.687757in}}%
\pgfpathlineto{\pgfqpoint{1.731498in}{0.688420in}}%
\pgfpathlineto{\pgfqpoint{1.734368in}{0.690965in}}%
\pgfpathlineto{\pgfqpoint{1.736418in}{0.690114in}}%
\pgfpathlineto{\pgfqpoint{1.737238in}{0.689672in}}%
\pgfpathlineto{\pgfqpoint{1.740108in}{0.695489in}}%
\pgfpathlineto{\pgfqpoint{1.741748in}{0.695189in}}%
\pgfpathlineto{\pgfqpoint{1.743798in}{0.691408in}}%
\pgfpathlineto{\pgfqpoint{1.745028in}{0.688886in}}%
\pgfpathlineto{\pgfqpoint{1.745438in}{0.689165in}}%
\pgfpathlineto{\pgfqpoint{1.747898in}{0.689343in}}%
\pgfpathlineto{\pgfqpoint{1.752408in}{0.688206in}}%
\pgfpathlineto{\pgfqpoint{1.755278in}{0.688730in}}%
\pgfpathlineto{\pgfqpoint{1.757328in}{0.688957in}}%
\pgfpathlineto{\pgfqpoint{1.760198in}{0.691827in}}%
\pgfpathlineto{\pgfqpoint{1.762248in}{0.690504in}}%
\pgfpathlineto{\pgfqpoint{1.765528in}{0.687943in}}%
\pgfpathlineto{\pgfqpoint{1.768398in}{0.687277in}}%
\pgfpathlineto{\pgfqpoint{1.770858in}{0.686804in}}%
\pgfpathlineto{\pgfqpoint{1.774138in}{0.687460in}}%
\pgfpathlineto{\pgfqpoint{1.776188in}{0.687541in}}%
\pgfpathlineto{\pgfqpoint{1.779058in}{0.689330in}}%
\pgfpathlineto{\pgfqpoint{1.781518in}{0.688017in}}%
\pgfpathlineto{\pgfqpoint{1.783978in}{0.686922in}}%
\pgfpathlineto{\pgfqpoint{1.787668in}{0.686211in}}%
\pgfpathlineto{\pgfqpoint{1.789718in}{0.686390in}}%
\pgfpathlineto{\pgfqpoint{1.793408in}{0.686441in}}%
\pgfpathlineto{\pgfqpoint{1.795048in}{0.687112in}}%
\pgfpathlineto{\pgfqpoint{1.797918in}{0.687659in}}%
\pgfpathlineto{\pgfqpoint{1.807348in}{0.685952in}}%
\pgfpathlineto{\pgfqpoint{1.816368in}{0.686504in}}%
\pgfpathlineto{\pgfqpoint{1.822518in}{0.685692in}}%
\pgfpathlineto{\pgfqpoint{1.828668in}{0.685716in}}%
\pgfpathlineto{\pgfqpoint{1.833178in}{0.686035in}}%
\pgfpathlineto{\pgfqpoint{1.841378in}{0.685577in}}%
\pgfpathlineto{\pgfqpoint{1.860238in}{0.685494in}}%
\pgfpathlineto{\pgfqpoint{1.900008in}{0.685390in}}%
\pgfpathlineto{\pgfqpoint{2.064008in}{0.685365in}}%
\pgfpathlineto{\pgfqpoint{2.952066in}{0.685365in}}%
\pgfpathlineto{\pgfqpoint{2.952066in}{0.685365in}}%
\pgfusepath{stroke}%
\end{pgfscope}%
\begin{pgfscope}%
\pgfsetrectcap%
\pgfsetmiterjoin%
\pgfsetlinewidth{0.803000pt}%
\definecolor{currentstroke}{rgb}{0.000000,0.000000,0.000000}%
\pgfsetstrokecolor{currentstroke}%
\pgfsetdash{}{0pt}%
\pgfpathmoveto{\pgfqpoint{0.800000in}{0.528000in}}%
\pgfpathlineto{\pgfqpoint{0.800000in}{2.208000in}}%
\pgfusepath{stroke}%
\end{pgfscope}%
\begin{pgfscope}%
\pgfsetrectcap%
\pgfsetmiterjoin%
\pgfsetlinewidth{0.803000pt}%
\definecolor{currentstroke}{rgb}{0.000000,0.000000,0.000000}%
\pgfsetstrokecolor{currentstroke}%
\pgfsetdash{}{0pt}%
\pgfpathmoveto{\pgfqpoint{3.054545in}{0.528000in}}%
\pgfpathlineto{\pgfqpoint{3.054545in}{2.208000in}}%
\pgfusepath{stroke}%
\end{pgfscope}%
\begin{pgfscope}%
\pgfsetrectcap%
\pgfsetmiterjoin%
\pgfsetlinewidth{0.803000pt}%
\definecolor{currentstroke}{rgb}{0.000000,0.000000,0.000000}%
\pgfsetstrokecolor{currentstroke}%
\pgfsetdash{}{0pt}%
\pgfpathmoveto{\pgfqpoint{0.800000in}{0.528000in}}%
\pgfpathlineto{\pgfqpoint{3.054545in}{0.528000in}}%
\pgfusepath{stroke}%
\end{pgfscope}%
\begin{pgfscope}%
\pgfsetrectcap%
\pgfsetmiterjoin%
\pgfsetlinewidth{0.803000pt}%
\definecolor{currentstroke}{rgb}{0.000000,0.000000,0.000000}%
\pgfsetstrokecolor{currentstroke}%
\pgfsetdash{}{0pt}%
\pgfpathmoveto{\pgfqpoint{0.800000in}{2.208000in}}%
\pgfpathlineto{\pgfqpoint{3.054545in}{2.208000in}}%
\pgfusepath{stroke}%
\end{pgfscope}%
\begin{pgfscope}%
\pgfsetbuttcap%
\pgfsetmiterjoin%
\definecolor{currentfill}{rgb}{1.000000,1.000000,1.000000}%
\pgfsetfillcolor{currentfill}%
\pgfsetlinewidth{0.000000pt}%
\definecolor{currentstroke}{rgb}{0.000000,0.000000,0.000000}%
\pgfsetstrokecolor{currentstroke}%
\pgfsetstrokeopacity{0.000000}%
\pgfsetdash{}{0pt}%
\pgfpathmoveto{\pgfqpoint{3.505455in}{0.528000in}}%
\pgfpathlineto{\pgfqpoint{5.760000in}{0.528000in}}%
\pgfpathlineto{\pgfqpoint{5.760000in}{2.208000in}}%
\pgfpathlineto{\pgfqpoint{3.505455in}{2.208000in}}%
\pgfpathlineto{\pgfqpoint{3.505455in}{0.528000in}}%
\pgfpathclose%
\pgfusepath{fill}%
\end{pgfscope}%
\begin{pgfscope}%
\pgfsetbuttcap%
\pgfsetroundjoin%
\definecolor{currentfill}{rgb}{0.000000,0.000000,0.000000}%
\pgfsetfillcolor{currentfill}%
\pgfsetlinewidth{0.803000pt}%
\definecolor{currentstroke}{rgb}{0.000000,0.000000,0.000000}%
\pgfsetstrokecolor{currentstroke}%
\pgfsetdash{}{0pt}%
\pgfsys@defobject{currentmarker}{\pgfqpoint{0.000000in}{-0.048611in}}{\pgfqpoint{0.000000in}{0.000000in}}{%
\pgfpathmoveto{\pgfqpoint{0.000000in}{0.000000in}}%
\pgfpathlineto{\pgfqpoint{0.000000in}{-0.048611in}}%
\pgfusepath{stroke,fill}%
}%
\begin{pgfscope}%
\pgfsys@transformshift{3.607934in}{0.528000in}%
\pgfsys@useobject{currentmarker}{}%
\end{pgfscope}%
\end{pgfscope}%
\begin{pgfscope}%
\definecolor{textcolor}{rgb}{0.000000,0.000000,0.000000}%
\pgfsetstrokecolor{textcolor}%
\pgfsetfillcolor{textcolor}%
\pgftext[x=3.607934in,y=0.430778in,,top]{\color{textcolor}\rmfamily\fontsize{10.000000}{12.000000}\selectfont \(\displaystyle {0}\)}%
\end{pgfscope}%
\begin{pgfscope}%
\pgfsetbuttcap%
\pgfsetroundjoin%
\definecolor{currentfill}{rgb}{0.000000,0.000000,0.000000}%
\pgfsetfillcolor{currentfill}%
\pgfsetlinewidth{0.803000pt}%
\definecolor{currentstroke}{rgb}{0.000000,0.000000,0.000000}%
\pgfsetstrokecolor{currentstroke}%
\pgfsetdash{}{0pt}%
\pgfsys@defobject{currentmarker}{\pgfqpoint{0.000000in}{-0.048611in}}{\pgfqpoint{0.000000in}{0.000000in}}{%
\pgfpathmoveto{\pgfqpoint{0.000000in}{0.000000in}}%
\pgfpathlineto{\pgfqpoint{0.000000in}{-0.048611in}}%
\pgfusepath{stroke,fill}%
}%
\begin{pgfscope}%
\pgfsys@transformshift{4.427933in}{0.528000in}%
\pgfsys@useobject{currentmarker}{}%
\end{pgfscope}%
\end{pgfscope}%
\begin{pgfscope}%
\definecolor{textcolor}{rgb}{0.000000,0.000000,0.000000}%
\pgfsetstrokecolor{textcolor}%
\pgfsetfillcolor{textcolor}%
\pgftext[x=4.427933in,y=0.430778in,,top]{\color{textcolor}\rmfamily\fontsize{10.000000}{12.000000}\selectfont \(\displaystyle {2000}\)}%
\end{pgfscope}%
\begin{pgfscope}%
\pgfsetbuttcap%
\pgfsetroundjoin%
\definecolor{currentfill}{rgb}{0.000000,0.000000,0.000000}%
\pgfsetfillcolor{currentfill}%
\pgfsetlinewidth{0.803000pt}%
\definecolor{currentstroke}{rgb}{0.000000,0.000000,0.000000}%
\pgfsetstrokecolor{currentstroke}%
\pgfsetdash{}{0pt}%
\pgfsys@defobject{currentmarker}{\pgfqpoint{0.000000in}{-0.048611in}}{\pgfqpoint{0.000000in}{0.000000in}}{%
\pgfpathmoveto{\pgfqpoint{0.000000in}{0.000000in}}%
\pgfpathlineto{\pgfqpoint{0.000000in}{-0.048611in}}%
\pgfusepath{stroke,fill}%
}%
\begin{pgfscope}%
\pgfsys@transformshift{5.247931in}{0.528000in}%
\pgfsys@useobject{currentmarker}{}%
\end{pgfscope}%
\end{pgfscope}%
\begin{pgfscope}%
\definecolor{textcolor}{rgb}{0.000000,0.000000,0.000000}%
\pgfsetstrokecolor{textcolor}%
\pgfsetfillcolor{textcolor}%
\pgftext[x=5.247931in,y=0.430778in,,top]{\color{textcolor}\rmfamily\fontsize{10.000000}{12.000000}\selectfont \(\displaystyle {4000}\)}%
\end{pgfscope}%
\begin{pgfscope}%
\pgfsetbuttcap%
\pgfsetroundjoin%
\definecolor{currentfill}{rgb}{0.000000,0.000000,0.000000}%
\pgfsetfillcolor{currentfill}%
\pgfsetlinewidth{0.803000pt}%
\definecolor{currentstroke}{rgb}{0.000000,0.000000,0.000000}%
\pgfsetstrokecolor{currentstroke}%
\pgfsetdash{}{0pt}%
\pgfsys@defobject{currentmarker}{\pgfqpoint{-0.048611in}{0.000000in}}{\pgfqpoint{-0.000000in}{0.000000in}}{%
\pgfpathmoveto{\pgfqpoint{-0.000000in}{0.000000in}}%
\pgfpathlineto{\pgfqpoint{-0.048611in}{0.000000in}}%
\pgfusepath{stroke,fill}%
}%
\begin{pgfscope}%
\pgfsys@transformshift{3.505455in}{0.528000in}%
\pgfsys@useobject{currentmarker}{}%
\end{pgfscope}%
\end{pgfscope}%
\begin{pgfscope}%
\definecolor{textcolor}{rgb}{0.000000,0.000000,0.000000}%
\pgfsetstrokecolor{textcolor}%
\pgfsetfillcolor{textcolor}%
\pgftext[x=3.161318in, y=0.479775in, left, base]{\color{textcolor}\rmfamily\fontsize{10.000000}{12.000000}\selectfont \(\displaystyle {0.00}\)}%
\end{pgfscope}%
\begin{pgfscope}%
\pgfsetbuttcap%
\pgfsetroundjoin%
\definecolor{currentfill}{rgb}{0.000000,0.000000,0.000000}%
\pgfsetfillcolor{currentfill}%
\pgfsetlinewidth{0.803000pt}%
\definecolor{currentstroke}{rgb}{0.000000,0.000000,0.000000}%
\pgfsetstrokecolor{currentstroke}%
\pgfsetdash{}{0pt}%
\pgfsys@defobject{currentmarker}{\pgfqpoint{-0.048611in}{0.000000in}}{\pgfqpoint{-0.000000in}{0.000000in}}{%
\pgfpathmoveto{\pgfqpoint{-0.000000in}{0.000000in}}%
\pgfpathlineto{\pgfqpoint{-0.048611in}{0.000000in}}%
\pgfusepath{stroke,fill}%
}%
\begin{pgfscope}%
\pgfsys@transformshift{3.505455in}{0.948000in}%
\pgfsys@useobject{currentmarker}{}%
\end{pgfscope}%
\end{pgfscope}%
\begin{pgfscope}%
\definecolor{textcolor}{rgb}{0.000000,0.000000,0.000000}%
\pgfsetstrokecolor{textcolor}%
\pgfsetfillcolor{textcolor}%
\pgftext[x=3.161318in, y=0.899775in, left, base]{\color{textcolor}\rmfamily\fontsize{10.000000}{12.000000}\selectfont \(\displaystyle {0.05}\)}%
\end{pgfscope}%
\begin{pgfscope}%
\pgfsetbuttcap%
\pgfsetroundjoin%
\definecolor{currentfill}{rgb}{0.000000,0.000000,0.000000}%
\pgfsetfillcolor{currentfill}%
\pgfsetlinewidth{0.803000pt}%
\definecolor{currentstroke}{rgb}{0.000000,0.000000,0.000000}%
\pgfsetstrokecolor{currentstroke}%
\pgfsetdash{}{0pt}%
\pgfsys@defobject{currentmarker}{\pgfqpoint{-0.048611in}{0.000000in}}{\pgfqpoint{-0.000000in}{0.000000in}}{%
\pgfpathmoveto{\pgfqpoint{-0.000000in}{0.000000in}}%
\pgfpathlineto{\pgfqpoint{-0.048611in}{0.000000in}}%
\pgfusepath{stroke,fill}%
}%
\begin{pgfscope}%
\pgfsys@transformshift{3.505455in}{1.368000in}%
\pgfsys@useobject{currentmarker}{}%
\end{pgfscope}%
\end{pgfscope}%
\begin{pgfscope}%
\definecolor{textcolor}{rgb}{0.000000,0.000000,0.000000}%
\pgfsetstrokecolor{textcolor}%
\pgfsetfillcolor{textcolor}%
\pgftext[x=3.161318in, y=1.319775in, left, base]{\color{textcolor}\rmfamily\fontsize{10.000000}{12.000000}\selectfont \(\displaystyle {0.10}\)}%
\end{pgfscope}%
\begin{pgfscope}%
\pgfsetbuttcap%
\pgfsetroundjoin%
\definecolor{currentfill}{rgb}{0.000000,0.000000,0.000000}%
\pgfsetfillcolor{currentfill}%
\pgfsetlinewidth{0.803000pt}%
\definecolor{currentstroke}{rgb}{0.000000,0.000000,0.000000}%
\pgfsetstrokecolor{currentstroke}%
\pgfsetdash{}{0pt}%
\pgfsys@defobject{currentmarker}{\pgfqpoint{-0.048611in}{0.000000in}}{\pgfqpoint{-0.000000in}{0.000000in}}{%
\pgfpathmoveto{\pgfqpoint{-0.000000in}{0.000000in}}%
\pgfpathlineto{\pgfqpoint{-0.048611in}{0.000000in}}%
\pgfusepath{stroke,fill}%
}%
\begin{pgfscope}%
\pgfsys@transformshift{3.505455in}{1.788000in}%
\pgfsys@useobject{currentmarker}{}%
\end{pgfscope}%
\end{pgfscope}%
\begin{pgfscope}%
\definecolor{textcolor}{rgb}{0.000000,0.000000,0.000000}%
\pgfsetstrokecolor{textcolor}%
\pgfsetfillcolor{textcolor}%
\pgftext[x=3.161318in, y=1.739775in, left, base]{\color{textcolor}\rmfamily\fontsize{10.000000}{12.000000}\selectfont \(\displaystyle {0.15}\)}%
\end{pgfscope}%
\begin{pgfscope}%
\pgfsetbuttcap%
\pgfsetroundjoin%
\definecolor{currentfill}{rgb}{0.000000,0.000000,0.000000}%
\pgfsetfillcolor{currentfill}%
\pgfsetlinewidth{0.803000pt}%
\definecolor{currentstroke}{rgb}{0.000000,0.000000,0.000000}%
\pgfsetstrokecolor{currentstroke}%
\pgfsetdash{}{0pt}%
\pgfsys@defobject{currentmarker}{\pgfqpoint{-0.048611in}{0.000000in}}{\pgfqpoint{-0.000000in}{0.000000in}}{%
\pgfpathmoveto{\pgfqpoint{-0.000000in}{0.000000in}}%
\pgfpathlineto{\pgfqpoint{-0.048611in}{0.000000in}}%
\pgfusepath{stroke,fill}%
}%
\begin{pgfscope}%
\pgfsys@transformshift{3.505455in}{2.208000in}%
\pgfsys@useobject{currentmarker}{}%
\end{pgfscope}%
\end{pgfscope}%
\begin{pgfscope}%
\definecolor{textcolor}{rgb}{0.000000,0.000000,0.000000}%
\pgfsetstrokecolor{textcolor}%
\pgfsetfillcolor{textcolor}%
\pgftext[x=3.161318in, y=2.159775in, left, base]{\color{textcolor}\rmfamily\fontsize{10.000000}{12.000000}\selectfont \(\displaystyle {0.20}\)}%
\end{pgfscope}%
\begin{pgfscope}%
\pgfpathrectangle{\pgfqpoint{3.505455in}{0.528000in}}{\pgfqpoint{2.254545in}{1.680000in}}%
\pgfusepath{clip}%
\pgfsetrectcap%
\pgfsetroundjoin%
\pgfsetlinewidth{1.505625pt}%
\definecolor{currentstroke}{rgb}{0.121569,0.466667,0.705882}%
\pgfsetstrokecolor{currentstroke}%
\pgfsetdash{}{0pt}%
\pgfpathmoveto{\pgfqpoint{3.607934in}{1.461333in}}%
\pgfpathlineto{\pgfqpoint{3.608754in}{1.482238in}}%
\pgfpathlineto{\pgfqpoint{3.609164in}{1.478436in}}%
\pgfpathlineto{\pgfqpoint{3.610394in}{1.428285in}}%
\pgfpathlineto{\pgfqpoint{3.614494in}{1.099495in}}%
\pgfpathlineto{\pgfqpoint{3.619414in}{0.779167in}}%
\pgfpathlineto{\pgfqpoint{3.623104in}{0.657458in}}%
\pgfpathlineto{\pgfqpoint{3.625564in}{0.632035in}}%
\pgfpathlineto{\pgfqpoint{3.627614in}{0.615726in}}%
\pgfpathlineto{\pgfqpoint{3.628024in}{0.616115in}}%
\pgfpathlineto{\pgfqpoint{3.629664in}{0.622142in}}%
\pgfpathlineto{\pgfqpoint{3.638684in}{0.669959in}}%
\pgfpathlineto{\pgfqpoint{3.639914in}{0.668746in}}%
\pgfpathlineto{\pgfqpoint{3.641964in}{0.662249in}}%
\pgfpathlineto{\pgfqpoint{3.646884in}{0.635421in}}%
\pgfpathlineto{\pgfqpoint{3.651804in}{0.613968in}}%
\pgfpathlineto{\pgfqpoint{3.654674in}{0.608910in}}%
\pgfpathlineto{\pgfqpoint{3.657134in}{0.608396in}}%
\pgfpathlineto{\pgfqpoint{3.660824in}{0.610948in}}%
\pgfpathlineto{\pgfqpoint{3.664924in}{0.613006in}}%
\pgfpathlineto{\pgfqpoint{3.667794in}{0.611878in}}%
\pgfpathlineto{\pgfqpoint{3.671484in}{0.607431in}}%
\pgfpathlineto{\pgfqpoint{3.680914in}{0.595013in}}%
\pgfpathlineto{\pgfqpoint{3.685424in}{0.593398in}}%
\pgfpathlineto{\pgfqpoint{3.696494in}{0.591411in}}%
\pgfpathlineto{\pgfqpoint{3.714534in}{0.583050in}}%
\pgfpathlineto{\pgfqpoint{3.727244in}{0.580028in}}%
\pgfpathlineto{\pgfqpoint{3.741184in}{0.576250in}}%
\pgfpathlineto{\pgfqpoint{3.821954in}{0.564500in}}%
\pgfpathlineto{\pgfqpoint{3.890013in}{0.559062in}}%
\pgfpathlineto{\pgfqpoint{3.974883in}{0.554662in}}%
\pgfpathlineto{\pgfqpoint{4.098293in}{0.550575in}}%
\pgfpathlineto{\pgfqpoint{4.283613in}{0.546824in}}%
\pgfpathlineto{\pgfqpoint{4.569382in}{0.543454in}}%
\pgfpathlineto{\pgfqpoint{5.029812in}{0.540454in}}%
\pgfpathlineto{\pgfqpoint{5.657521in}{0.538205in}}%
\pgfpathlineto{\pgfqpoint{5.657521in}{0.538205in}}%
\pgfusepath{stroke}%
\end{pgfscope}%
\begin{pgfscope}%
\pgfpathrectangle{\pgfqpoint{3.505455in}{0.528000in}}{\pgfqpoint{2.254545in}{1.680000in}}%
\pgfusepath{clip}%
\pgfsetrectcap%
\pgfsetroundjoin%
\pgfsetlinewidth{1.505625pt}%
\definecolor{currentstroke}{rgb}{1.000000,0.498039,0.054902}%
\pgfsetstrokecolor{currentstroke}%
\pgfsetdash{}{0pt}%
\pgfpathmoveto{\pgfqpoint{3.607934in}{1.461333in}}%
\pgfpathlineto{\pgfqpoint{3.609574in}{1.478801in}}%
\pgfpathlineto{\pgfqpoint{3.609984in}{1.479005in}}%
\pgfpathlineto{\pgfqpoint{3.611214in}{1.471609in}}%
\pgfpathlineto{\pgfqpoint{3.613264in}{1.438682in}}%
\pgfpathlineto{\pgfqpoint{3.616954in}{1.335953in}}%
\pgfpathlineto{\pgfqpoint{3.634584in}{0.788630in}}%
\pgfpathlineto{\pgfqpoint{3.637044in}{0.743576in}}%
\pgfpathlineto{\pgfqpoint{3.641144in}{0.786781in}}%
\pgfpathlineto{\pgfqpoint{3.644834in}{0.804946in}}%
\pgfpathlineto{\pgfqpoint{3.647294in}{0.808330in}}%
\pgfpathlineto{\pgfqpoint{3.648934in}{0.807339in}}%
\pgfpathlineto{\pgfqpoint{3.651394in}{0.801676in}}%
\pgfpathlineto{\pgfqpoint{3.655084in}{0.785384in}}%
\pgfpathlineto{\pgfqpoint{3.660414in}{0.749445in}}%
\pgfpathlineto{\pgfqpoint{3.668614in}{0.676174in}}%
\pgfpathlineto{\pgfqpoint{3.677634in}{0.585646in}}%
\pgfpathlineto{\pgfqpoint{3.678044in}{0.590119in}}%
\pgfpathlineto{\pgfqpoint{3.681734in}{0.617211in}}%
\pgfpathlineto{\pgfqpoint{3.683784in}{0.621282in}}%
\pgfpathlineto{\pgfqpoint{3.685014in}{0.620114in}}%
\pgfpathlineto{\pgfqpoint{3.687064in}{0.612440in}}%
\pgfpathlineto{\pgfqpoint{3.689934in}{0.590568in}}%
\pgfpathlineto{\pgfqpoint{3.691574in}{0.577217in}}%
\pgfpathlineto{\pgfqpoint{3.700184in}{0.651982in}}%
\pgfpathlineto{\pgfqpoint{3.705104in}{0.677698in}}%
\pgfpathlineto{\pgfqpoint{3.708794in}{0.686638in}}%
\pgfpathlineto{\pgfqpoint{3.711254in}{0.687755in}}%
\pgfpathlineto{\pgfqpoint{3.713304in}{0.685963in}}%
\pgfpathlineto{\pgfqpoint{3.716174in}{0.679781in}}%
\pgfpathlineto{\pgfqpoint{3.720274in}{0.664938in}}%
\pgfpathlineto{\pgfqpoint{3.727244in}{0.629666in}}%
\pgfpathlineto{\pgfqpoint{3.732574in}{0.600581in}}%
\pgfpathlineto{\pgfqpoint{3.732984in}{0.601451in}}%
\pgfpathlineto{\pgfqpoint{3.737904in}{0.616325in}}%
\pgfpathlineto{\pgfqpoint{3.741594in}{0.621904in}}%
\pgfpathlineto{\pgfqpoint{3.744464in}{0.623105in}}%
\pgfpathlineto{\pgfqpoint{3.747334in}{0.621781in}}%
\pgfpathlineto{\pgfqpoint{3.750614in}{0.617497in}}%
\pgfpathlineto{\pgfqpoint{3.755124in}{0.607544in}}%
\pgfpathlineto{\pgfqpoint{3.761684in}{0.586995in}}%
\pgfpathlineto{\pgfqpoint{3.775214in}{0.539111in}}%
\pgfpathlineto{\pgfqpoint{3.775624in}{0.540395in}}%
\pgfpathlineto{\pgfqpoint{3.794484in}{0.596561in}}%
\pgfpathlineto{\pgfqpoint{3.799404in}{0.603332in}}%
\pgfpathlineto{\pgfqpoint{3.803094in}{0.605095in}}%
\pgfpathlineto{\pgfqpoint{3.806374in}{0.604336in}}%
\pgfpathlineto{\pgfqpoint{3.810064in}{0.601118in}}%
\pgfpathlineto{\pgfqpoint{3.814984in}{0.593645in}}%
\pgfpathlineto{\pgfqpoint{3.823184in}{0.576269in}}%
\pgfpathlineto{\pgfqpoint{3.826054in}{0.570301in}}%
\pgfpathlineto{\pgfqpoint{3.826464in}{0.570898in}}%
\pgfpathlineto{\pgfqpoint{3.831794in}{0.576752in}}%
\pgfpathlineto{\pgfqpoint{3.836304in}{0.578580in}}%
\pgfpathlineto{\pgfqpoint{3.840404in}{0.577835in}}%
\pgfpathlineto{\pgfqpoint{3.844914in}{0.574651in}}%
\pgfpathlineto{\pgfqpoint{3.851064in}{0.567152in}}%
\pgfpathlineto{\pgfqpoint{3.862133in}{0.549080in}}%
\pgfpathlineto{\pgfqpoint{3.865003in}{0.544704in}}%
\pgfpathlineto{\pgfqpoint{3.865413in}{0.545097in}}%
\pgfpathlineto{\pgfqpoint{3.889193in}{0.574633in}}%
\pgfpathlineto{\pgfqpoint{3.894523in}{0.576457in}}%
\pgfpathlineto{\pgfqpoint{3.899443in}{0.575845in}}%
\pgfpathlineto{\pgfqpoint{3.905183in}{0.572660in}}%
\pgfpathlineto{\pgfqpoint{3.912973in}{0.565339in}}%
\pgfpathlineto{\pgfqpoint{3.920353in}{0.558541in}}%
\pgfpathlineto{\pgfqpoint{3.926503in}{0.560866in}}%
\pgfpathlineto{\pgfqpoint{3.932243in}{0.560714in}}%
\pgfpathlineto{\pgfqpoint{3.938393in}{0.558301in}}%
\pgfpathlineto{\pgfqpoint{3.946593in}{0.552485in}}%
\pgfpathlineto{\pgfqpoint{3.955203in}{0.546346in}}%
\pgfpathlineto{\pgfqpoint{3.967093in}{0.554701in}}%
\pgfpathlineto{\pgfqpoint{3.978163in}{0.561341in}}%
\pgfpathlineto{\pgfqpoint{3.985543in}{0.563210in}}%
\pgfpathlineto{\pgfqpoint{3.992513in}{0.562673in}}%
\pgfpathlineto{\pgfqpoint{4.000713in}{0.559582in}}%
\pgfpathlineto{\pgfqpoint{4.017113in}{0.552365in}}%
\pgfpathlineto{\pgfqpoint{4.024903in}{0.552292in}}%
\pgfpathlineto{\pgfqpoint{4.033513in}{0.549944in}}%
\pgfpathlineto{\pgfqpoint{4.045403in}{0.546535in}}%
\pgfpathlineto{\pgfqpoint{4.059753in}{0.551848in}}%
\pgfpathlineto{\pgfqpoint{4.072463in}{0.555567in}}%
\pgfpathlineto{\pgfqpoint{4.081893in}{0.555986in}}%
\pgfpathlineto{\pgfqpoint{4.092143in}{0.554067in}}%
\pgfpathlineto{\pgfqpoint{4.119613in}{0.547426in}}%
\pgfpathlineto{\pgfqpoint{4.136013in}{0.546186in}}%
\pgfpathlineto{\pgfqpoint{4.155693in}{0.550221in}}%
\pgfpathlineto{\pgfqpoint{4.169633in}{0.551735in}}%
\pgfpathlineto{\pgfqpoint{4.182343in}{0.550742in}}%
\pgfpathlineto{\pgfqpoint{4.223343in}{0.545354in}}%
\pgfpathlineto{\pgfqpoint{4.241793in}{0.547303in}}%
\pgfpathlineto{\pgfqpoint{4.262293in}{0.548857in}}%
\pgfpathlineto{\pgfqpoint{4.279103in}{0.547650in}}%
\pgfpathlineto{\pgfqpoint{4.311903in}{0.544718in}}%
\pgfpathlineto{\pgfqpoint{4.333633in}{0.545923in}}%
\pgfpathlineto{\pgfqpoint{4.358233in}{0.546745in}}%
\pgfpathlineto{\pgfqpoint{4.385293in}{0.544811in}}%
\pgfpathlineto{\pgfqpoint{4.409073in}{0.544127in}}%
\pgfpathlineto{\pgfqpoint{4.469343in}{0.544357in}}%
\pgfpathlineto{\pgfqpoint{4.503372in}{0.543470in}}%
\pgfpathlineto{\pgfqpoint{4.562002in}{0.543413in}}%
\pgfpathlineto{\pgfqpoint{4.602182in}{0.542894in}}%
\pgfpathlineto{\pgfqpoint{4.654252in}{0.542636in}}%
\pgfpathlineto{\pgfqpoint{4.703452in}{0.542331in}}%
\pgfpathlineto{\pgfqpoint{4.759212in}{0.541736in}}%
\pgfpathlineto{\pgfqpoint{4.880982in}{0.541131in}}%
\pgfpathlineto{\pgfqpoint{5.023662in}{0.540400in}}%
\pgfpathlineto{\pgfqpoint{5.240551in}{0.539471in}}%
\pgfpathlineto{\pgfqpoint{5.657521in}{0.538167in}}%
\pgfpathlineto{\pgfqpoint{5.657521in}{0.538167in}}%
\pgfusepath{stroke}%
\end{pgfscope}%
\begin{pgfscope}%
\pgfpathrectangle{\pgfqpoint{3.505455in}{0.528000in}}{\pgfqpoint{2.254545in}{1.680000in}}%
\pgfusepath{clip}%
\pgfsetrectcap%
\pgfsetroundjoin%
\pgfsetlinewidth{1.505625pt}%
\definecolor{currentstroke}{rgb}{0.172549,0.627451,0.172549}%
\pgfsetstrokecolor{currentstroke}%
\pgfsetdash{}{0pt}%
\pgfpathmoveto{\pgfqpoint{3.607934in}{1.461333in}}%
\pgfpathlineto{\pgfqpoint{3.609164in}{1.477450in}}%
\pgfpathlineto{\pgfqpoint{3.609574in}{1.476579in}}%
\pgfpathlineto{\pgfqpoint{3.610804in}{1.459098in}}%
\pgfpathlineto{\pgfqpoint{3.613264in}{1.377869in}}%
\pgfpathlineto{\pgfqpoint{3.628434in}{0.780681in}}%
\pgfpathlineto{\pgfqpoint{3.630894in}{0.724602in}}%
\pgfpathlineto{\pgfqpoint{3.631304in}{0.728826in}}%
\pgfpathlineto{\pgfqpoint{3.634174in}{0.747264in}}%
\pgfpathlineto{\pgfqpoint{3.636224in}{0.750358in}}%
\pgfpathlineto{\pgfqpoint{3.637454in}{0.748942in}}%
\pgfpathlineto{\pgfqpoint{3.639504in}{0.742066in}}%
\pgfpathlineto{\pgfqpoint{3.642784in}{0.721876in}}%
\pgfpathlineto{\pgfqpoint{3.648524in}{0.669433in}}%
\pgfpathlineto{\pgfqpoint{3.660824in}{0.541530in}}%
\pgfpathlineto{\pgfqpoint{3.661234in}{0.542557in}}%
\pgfpathlineto{\pgfqpoint{3.674764in}{0.646448in}}%
\pgfpathlineto{\pgfqpoint{3.678454in}{0.657366in}}%
\pgfpathlineto{\pgfqpoint{3.680914in}{0.658886in}}%
\pgfpathlineto{\pgfqpoint{3.682964in}{0.656962in}}%
\pgfpathlineto{\pgfqpoint{3.685834in}{0.650176in}}%
\pgfpathlineto{\pgfqpoint{3.690344in}{0.632660in}}%
\pgfpathlineto{\pgfqpoint{3.699364in}{0.590747in}}%
\pgfpathlineto{\pgfqpoint{3.700184in}{0.591968in}}%
\pgfpathlineto{\pgfqpoint{3.703054in}{0.594411in}}%
\pgfpathlineto{\pgfqpoint{3.705514in}{0.593995in}}%
\pgfpathlineto{\pgfqpoint{3.708384in}{0.590950in}}%
\pgfpathlineto{\pgfqpoint{3.712484in}{0.582828in}}%
\pgfpathlineto{\pgfqpoint{3.720274in}{0.560956in}}%
\pgfpathlineto{\pgfqpoint{3.722324in}{0.556167in}}%
\pgfpathlineto{\pgfqpoint{3.722734in}{0.556866in}}%
\pgfpathlineto{\pgfqpoint{3.729294in}{0.570763in}}%
\pgfpathlineto{\pgfqpoint{3.737904in}{0.588157in}}%
\pgfpathlineto{\pgfqpoint{3.742414in}{0.592928in}}%
\pgfpathlineto{\pgfqpoint{3.746104in}{0.593828in}}%
\pgfpathlineto{\pgfqpoint{3.749794in}{0.592145in}}%
\pgfpathlineto{\pgfqpoint{3.754304in}{0.587266in}}%
\pgfpathlineto{\pgfqpoint{3.762914in}{0.573701in}}%
\pgfpathlineto{\pgfqpoint{3.769474in}{0.565011in}}%
\pgfpathlineto{\pgfqpoint{3.782594in}{0.557738in}}%
\pgfpathlineto{\pgfqpoint{3.787924in}{0.560157in}}%
\pgfpathlineto{\pgfqpoint{3.809244in}{0.572823in}}%
\pgfpathlineto{\pgfqpoint{3.814984in}{0.571878in}}%
\pgfpathlineto{\pgfqpoint{3.821954in}{0.568196in}}%
\pgfpathlineto{\pgfqpoint{3.839174in}{0.558107in}}%
\pgfpathlineto{\pgfqpoint{3.846144in}{0.557175in}}%
\pgfpathlineto{\pgfqpoint{3.853933in}{0.558531in}}%
\pgfpathlineto{\pgfqpoint{3.873613in}{0.563169in}}%
\pgfpathlineto{\pgfqpoint{3.882223in}{0.562068in}}%
\pgfpathlineto{\pgfqpoint{3.913383in}{0.555382in}}%
\pgfpathlineto{\pgfqpoint{3.930603in}{0.557391in}}%
\pgfpathlineto{\pgfqpoint{3.942903in}{0.557500in}}%
\pgfpathlineto{\pgfqpoint{3.958893in}{0.555010in}}%
\pgfpathlineto{\pgfqpoint{3.974883in}{0.553336in}}%
\pgfpathlineto{\pgfqpoint{3.994563in}{0.554119in}}%
\pgfpathlineto{\pgfqpoint{4.012193in}{0.553718in}}%
\pgfpathlineto{\pgfqpoint{4.049503in}{0.551574in}}%
\pgfpathlineto{\pgfqpoint{4.081073in}{0.551132in}}%
\pgfpathlineto{\pgfqpoint{4.116743in}{0.549928in}}%
\pgfpathlineto{\pgfqpoint{4.155283in}{0.549018in}}%
\pgfpathlineto{\pgfqpoint{4.199153in}{0.548358in}}%
\pgfpathlineto{\pgfqpoint{4.432443in}{0.544757in}}%
\pgfpathlineto{\pgfqpoint{4.974872in}{0.540698in}}%
\pgfpathlineto{\pgfqpoint{5.657521in}{0.538189in}}%
\pgfpathlineto{\pgfqpoint{5.657521in}{0.538189in}}%
\pgfusepath{stroke}%
\end{pgfscope}%
\begin{pgfscope}%
\pgfpathrectangle{\pgfqpoint{3.505455in}{0.528000in}}{\pgfqpoint{2.254545in}{1.680000in}}%
\pgfusepath{clip}%
\pgfsetrectcap%
\pgfsetroundjoin%
\pgfsetlinewidth{1.505625pt}%
\definecolor{currentstroke}{rgb}{0.839216,0.152941,0.156863}%
\pgfsetstrokecolor{currentstroke}%
\pgfsetdash{}{0pt}%
\pgfpathmoveto{\pgfqpoint{3.607934in}{1.461333in}}%
\pgfpathlineto{\pgfqpoint{3.608344in}{1.484469in}}%
\pgfpathlineto{\pgfqpoint{3.609164in}{1.471252in}}%
\pgfpathlineto{\pgfqpoint{3.610804in}{1.344991in}}%
\pgfpathlineto{\pgfqpoint{3.615314in}{0.753842in}}%
\pgfpathlineto{\pgfqpoint{3.616134in}{0.837006in}}%
\pgfpathlineto{\pgfqpoint{3.618594in}{0.973082in}}%
\pgfpathlineto{\pgfqpoint{3.619824in}{0.985420in}}%
\pgfpathlineto{\pgfqpoint{3.620234in}{0.983217in}}%
\pgfpathlineto{\pgfqpoint{3.621464in}{0.961035in}}%
\pgfpathlineto{\pgfqpoint{3.623924in}{0.859640in}}%
\pgfpathlineto{\pgfqpoint{3.624334in}{0.837367in}}%
\pgfpathlineto{\pgfqpoint{3.627614in}{1.162498in}}%
\pgfpathlineto{\pgfqpoint{3.628434in}{1.182180in}}%
\pgfpathlineto{\pgfqpoint{3.628844in}{1.181362in}}%
\pgfpathlineto{\pgfqpoint{3.630074in}{1.136595in}}%
\pgfpathlineto{\pgfqpoint{3.632124in}{0.936770in}}%
\pgfpathlineto{\pgfqpoint{3.633354in}{0.808461in}}%
\pgfpathlineto{\pgfqpoint{3.633764in}{0.836090in}}%
\pgfpathlineto{\pgfqpoint{3.635814in}{0.909333in}}%
\pgfpathlineto{\pgfqpoint{3.636224in}{0.910537in}}%
\pgfpathlineto{\pgfqpoint{3.637044in}{0.900315in}}%
\pgfpathlineto{\pgfqpoint{3.638684in}{0.837027in}}%
\pgfpathlineto{\pgfqpoint{3.641964in}{0.640939in}}%
\pgfpathlineto{\pgfqpoint{3.642374in}{0.661252in}}%
\pgfpathlineto{\pgfqpoint{3.645244in}{0.772445in}}%
\pgfpathlineto{\pgfqpoint{3.646474in}{0.785033in}}%
\pgfpathlineto{\pgfqpoint{3.646884in}{0.784934in}}%
\pgfpathlineto{\pgfqpoint{3.648114in}{0.772905in}}%
\pgfpathlineto{\pgfqpoint{3.650164in}{0.718808in}}%
\pgfpathlineto{\pgfqpoint{3.650574in}{0.703643in}}%
\pgfpathlineto{\pgfqpoint{3.650984in}{0.709148in}}%
\pgfpathlineto{\pgfqpoint{3.653854in}{0.859144in}}%
\pgfpathlineto{\pgfqpoint{3.654674in}{0.868252in}}%
\pgfpathlineto{\pgfqpoint{3.655084in}{0.866790in}}%
\pgfpathlineto{\pgfqpoint{3.656314in}{0.838968in}}%
\pgfpathlineto{\pgfqpoint{3.658364in}{0.724448in}}%
\pgfpathlineto{\pgfqpoint{3.659184in}{0.677550in}}%
\pgfpathlineto{\pgfqpoint{3.659594in}{0.693360in}}%
\pgfpathlineto{\pgfqpoint{3.662054in}{0.743190in}}%
\pgfpathlineto{\pgfqpoint{3.662464in}{0.743479in}}%
\pgfpathlineto{\pgfqpoint{3.663284in}{0.737444in}}%
\pgfpathlineto{\pgfqpoint{3.664924in}{0.702255in}}%
\pgfpathlineto{\pgfqpoint{3.667794in}{0.596949in}}%
\pgfpathlineto{\pgfqpoint{3.668614in}{0.613840in}}%
\pgfpathlineto{\pgfqpoint{3.671484in}{0.679859in}}%
\pgfpathlineto{\pgfqpoint{3.672714in}{0.686953in}}%
\pgfpathlineto{\pgfqpoint{3.673124in}{0.686540in}}%
\pgfpathlineto{\pgfqpoint{3.674354in}{0.677499in}}%
\pgfpathlineto{\pgfqpoint{3.676404in}{0.639735in}}%
\pgfpathlineto{\pgfqpoint{3.676814in}{0.629386in}}%
\pgfpathlineto{\pgfqpoint{3.677224in}{0.637259in}}%
\pgfpathlineto{\pgfqpoint{3.680094in}{0.716913in}}%
\pgfpathlineto{\pgfqpoint{3.680504in}{0.718932in}}%
\pgfpathlineto{\pgfqpoint{3.680914in}{0.718509in}}%
\pgfpathlineto{\pgfqpoint{3.682144in}{0.702977in}}%
\pgfpathlineto{\pgfqpoint{3.684194in}{0.635428in}}%
\pgfpathlineto{\pgfqpoint{3.684604in}{0.617030in}}%
\pgfpathlineto{\pgfqpoint{3.685424in}{0.629594in}}%
\pgfpathlineto{\pgfqpoint{3.687884in}{0.660926in}}%
\pgfpathlineto{\pgfqpoint{3.688294in}{0.661511in}}%
\pgfpathlineto{\pgfqpoint{3.688704in}{0.660774in}}%
\pgfpathlineto{\pgfqpoint{3.689934in}{0.651096in}}%
\pgfpathlineto{\pgfqpoint{3.692394in}{0.605195in}}%
\pgfpathlineto{\pgfqpoint{3.694034in}{0.572817in}}%
\pgfpathlineto{\pgfqpoint{3.694444in}{0.581749in}}%
\pgfpathlineto{\pgfqpoint{3.697724in}{0.628433in}}%
\pgfpathlineto{\pgfqpoint{3.698954in}{0.631073in}}%
\pgfpathlineto{\pgfqpoint{3.700184in}{0.625699in}}%
\pgfpathlineto{\pgfqpoint{3.702234in}{0.601335in}}%
\pgfpathlineto{\pgfqpoint{3.703054in}{0.587338in}}%
\pgfpathlineto{\pgfqpoint{3.703464in}{0.595767in}}%
\pgfpathlineto{\pgfqpoint{3.705924in}{0.635462in}}%
\pgfpathlineto{\pgfqpoint{3.706334in}{0.636786in}}%
\pgfpathlineto{\pgfqpoint{3.706744in}{0.636553in}}%
\pgfpathlineto{\pgfqpoint{3.707974in}{0.626780in}}%
\pgfpathlineto{\pgfqpoint{3.710434in}{0.582448in}}%
\pgfpathlineto{\pgfqpoint{3.710844in}{0.589032in}}%
\pgfpathlineto{\pgfqpoint{3.713714in}{0.614887in}}%
\pgfpathlineto{\pgfqpoint{3.714124in}{0.615336in}}%
\pgfpathlineto{\pgfqpoint{3.714534in}{0.614961in}}%
\pgfpathlineto{\pgfqpoint{3.715764in}{0.609127in}}%
\pgfpathlineto{\pgfqpoint{3.718224in}{0.580278in}}%
\pgfpathlineto{\pgfqpoint{3.719864in}{0.558989in}}%
\pgfpathlineto{\pgfqpoint{3.720274in}{0.564339in}}%
\pgfpathlineto{\pgfqpoint{3.723554in}{0.595439in}}%
\pgfpathlineto{\pgfqpoint{3.724784in}{0.597344in}}%
\pgfpathlineto{\pgfqpoint{3.726014in}{0.593849in}}%
\pgfpathlineto{\pgfqpoint{3.728064in}{0.577587in}}%
\pgfpathlineto{\pgfqpoint{3.729294in}{0.563122in}}%
\pgfpathlineto{\pgfqpoint{3.729704in}{0.568281in}}%
\pgfpathlineto{\pgfqpoint{3.732164in}{0.588292in}}%
\pgfpathlineto{\pgfqpoint{3.732984in}{0.586887in}}%
\pgfpathlineto{\pgfqpoint{3.734624in}{0.572585in}}%
\pgfpathlineto{\pgfqpoint{3.735444in}{0.560431in}}%
\pgfpathlineto{\pgfqpoint{3.735854in}{0.561581in}}%
\pgfpathlineto{\pgfqpoint{3.739134in}{0.586317in}}%
\pgfpathlineto{\pgfqpoint{3.739954in}{0.587439in}}%
\pgfpathlineto{\pgfqpoint{3.740364in}{0.587196in}}%
\pgfpathlineto{\pgfqpoint{3.741594in}{0.583401in}}%
\pgfpathlineto{\pgfqpoint{3.744054in}{0.564461in}}%
\pgfpathlineto{\pgfqpoint{3.745694in}{0.550338in}}%
\pgfpathlineto{\pgfqpoint{3.746104in}{0.553887in}}%
\pgfpathlineto{\pgfqpoint{3.749384in}{0.574738in}}%
\pgfpathlineto{\pgfqpoint{3.750614in}{0.576009in}}%
\pgfpathlineto{\pgfqpoint{3.751024in}{0.575604in}}%
\pgfpathlineto{\pgfqpoint{3.752664in}{0.570121in}}%
\pgfpathlineto{\pgfqpoint{3.755534in}{0.548990in}}%
\pgfpathlineto{\pgfqpoint{3.756354in}{0.552353in}}%
\pgfpathlineto{\pgfqpoint{3.757994in}{0.558454in}}%
\pgfpathlineto{\pgfqpoint{3.758404in}{0.558268in}}%
\pgfpathlineto{\pgfqpoint{3.759634in}{0.553725in}}%
\pgfpathlineto{\pgfqpoint{3.760864in}{0.546055in}}%
\pgfpathlineto{\pgfqpoint{3.761274in}{0.549514in}}%
\pgfpathlineto{\pgfqpoint{3.764554in}{0.568190in}}%
\pgfpathlineto{\pgfqpoint{3.766194in}{0.569605in}}%
\pgfpathlineto{\pgfqpoint{3.767424in}{0.567012in}}%
\pgfpathlineto{\pgfqpoint{3.769884in}{0.554202in}}%
\pgfpathlineto{\pgfqpoint{3.771524in}{0.544715in}}%
\pgfpathlineto{\pgfqpoint{3.771934in}{0.547200in}}%
\pgfpathlineto{\pgfqpoint{3.775214in}{0.561294in}}%
\pgfpathlineto{\pgfqpoint{3.776854in}{0.561809in}}%
\pgfpathlineto{\pgfqpoint{3.778494in}{0.557959in}}%
\pgfpathlineto{\pgfqpoint{3.781364in}{0.543335in}}%
\pgfpathlineto{\pgfqpoint{3.782594in}{0.537324in}}%
\pgfpathlineto{\pgfqpoint{3.783004in}{0.538557in}}%
\pgfpathlineto{\pgfqpoint{3.784234in}{0.539445in}}%
\pgfpathlineto{\pgfqpoint{3.785464in}{0.536243in}}%
\pgfpathlineto{\pgfqpoint{3.785874in}{0.537400in}}%
\pgfpathlineto{\pgfqpoint{3.789974in}{0.556402in}}%
\pgfpathlineto{\pgfqpoint{3.792024in}{0.558094in}}%
\pgfpathlineto{\pgfqpoint{3.793664in}{0.555223in}}%
\pgfpathlineto{\pgfqpoint{3.796534in}{0.543415in}}%
\pgfpathlineto{\pgfqpoint{3.797354in}{0.540951in}}%
\pgfpathlineto{\pgfqpoint{3.797764in}{0.542731in}}%
\pgfpathlineto{\pgfqpoint{3.801044in}{0.552334in}}%
\pgfpathlineto{\pgfqpoint{3.802684in}{0.552610in}}%
\pgfpathlineto{\pgfqpoint{3.804324in}{0.549859in}}%
\pgfpathlineto{\pgfqpoint{3.807194in}{0.539611in}}%
\pgfpathlineto{\pgfqpoint{3.809654in}{0.528864in}}%
\pgfpathlineto{\pgfqpoint{3.810064in}{0.529169in}}%
\pgfpathlineto{\pgfqpoint{3.815804in}{0.549305in}}%
\pgfpathlineto{\pgfqpoint{3.817854in}{0.550373in}}%
\pgfpathlineto{\pgfqpoint{3.819494in}{0.548316in}}%
\pgfpathlineto{\pgfqpoint{3.822364in}{0.540042in}}%
\pgfpathlineto{\pgfqpoint{3.823184in}{0.538369in}}%
\pgfpathlineto{\pgfqpoint{3.823594in}{0.539649in}}%
\pgfpathlineto{\pgfqpoint{3.826874in}{0.546238in}}%
\pgfpathlineto{\pgfqpoint{3.828514in}{0.546363in}}%
\pgfpathlineto{\pgfqpoint{3.830564in}{0.543581in}}%
\pgfpathlineto{\pgfqpoint{3.834254in}{0.533283in}}%
\pgfpathlineto{\pgfqpoint{3.835484in}{0.529929in}}%
\pgfpathlineto{\pgfqpoint{3.835894in}{0.530482in}}%
\pgfpathlineto{\pgfqpoint{3.841224in}{0.543943in}}%
\pgfpathlineto{\pgfqpoint{3.843274in}{0.545188in}}%
\pgfpathlineto{\pgfqpoint{3.845324in}{0.543591in}}%
\pgfpathlineto{\pgfqpoint{3.849014in}{0.536560in}}%
\pgfpathlineto{\pgfqpoint{3.849834in}{0.538321in}}%
\pgfpathlineto{\pgfqpoint{3.852703in}{0.542019in}}%
\pgfpathlineto{\pgfqpoint{3.854753in}{0.541820in}}%
\pgfpathlineto{\pgfqpoint{3.857213in}{0.538686in}}%
\pgfpathlineto{\pgfqpoint{3.861313in}{0.530606in}}%
\pgfpathlineto{\pgfqpoint{3.861723in}{0.531268in}}%
\pgfpathlineto{\pgfqpoint{3.867053in}{0.540675in}}%
\pgfpathlineto{\pgfqpoint{3.869513in}{0.541409in}}%
\pgfpathlineto{\pgfqpoint{3.871973in}{0.539346in}}%
\pgfpathlineto{\pgfqpoint{3.874843in}{0.535267in}}%
\pgfpathlineto{\pgfqpoint{3.875253in}{0.535900in}}%
\pgfpathlineto{\pgfqpoint{3.878533in}{0.539056in}}%
\pgfpathlineto{\pgfqpoint{3.880993in}{0.538621in}}%
\pgfpathlineto{\pgfqpoint{3.884273in}{0.534853in}}%
\pgfpathlineto{\pgfqpoint{3.887143in}{0.531100in}}%
\pgfpathlineto{\pgfqpoint{3.887553in}{0.531722in}}%
\pgfpathlineto{\pgfqpoint{3.893293in}{0.538566in}}%
\pgfpathlineto{\pgfqpoint{3.895753in}{0.538683in}}%
\pgfpathlineto{\pgfqpoint{3.898623in}{0.536438in}}%
\pgfpathlineto{\pgfqpoint{3.900673in}{0.534326in}}%
\pgfpathlineto{\pgfqpoint{3.901083in}{0.534747in}}%
\pgfpathlineto{\pgfqpoint{3.904773in}{0.537010in}}%
\pgfpathlineto{\pgfqpoint{3.907643in}{0.536099in}}%
\pgfpathlineto{\pgfqpoint{3.913793in}{0.532416in}}%
\pgfpathlineto{\pgfqpoint{3.919123in}{0.536806in}}%
\pgfpathlineto{\pgfqpoint{3.921993in}{0.536711in}}%
\pgfpathlineto{\pgfqpoint{3.925683in}{0.534123in}}%
\pgfpathlineto{\pgfqpoint{3.926913in}{0.533888in}}%
\pgfpathlineto{\pgfqpoint{3.930603in}{0.535465in}}%
\pgfpathlineto{\pgfqpoint{3.933883in}{0.534546in}}%
\pgfpathlineto{\pgfqpoint{3.939213in}{0.532077in}}%
\pgfpathlineto{\pgfqpoint{3.945363in}{0.535591in}}%
\pgfpathlineto{\pgfqpoint{3.949053in}{0.534947in}}%
\pgfpathlineto{\pgfqpoint{3.953563in}{0.533642in}}%
\pgfpathlineto{\pgfqpoint{3.957663in}{0.534254in}}%
\pgfpathlineto{\pgfqpoint{3.962173in}{0.532360in}}%
\pgfpathlineto{\pgfqpoint{3.964633in}{0.531872in}}%
\pgfpathlineto{\pgfqpoint{3.971603in}{0.534636in}}%
\pgfpathlineto{\pgfqpoint{3.976113in}{0.533623in}}%
\pgfpathlineto{\pgfqpoint{3.979803in}{0.533149in}}%
\pgfpathlineto{\pgfqpoint{3.984313in}{0.533299in}}%
\pgfpathlineto{\pgfqpoint{3.992103in}{0.532586in}}%
\pgfpathlineto{\pgfqpoint{3.998253in}{0.533873in}}%
\pgfpathlineto{\pgfqpoint{4.014243in}{0.531655in}}%
\pgfpathlineto{\pgfqpoint{4.016703in}{0.532026in}}%
\pgfpathlineto{\pgfqpoint{4.024493in}{0.533277in}}%
\pgfpathlineto{\pgfqpoint{4.045403in}{0.532538in}}%
\pgfpathlineto{\pgfqpoint{4.052373in}{0.532629in}}%
\pgfpathlineto{\pgfqpoint{4.066313in}{0.531554in}}%
\pgfpathlineto{\pgfqpoint{4.078613in}{0.532292in}}%
\pgfpathlineto{\pgfqpoint{4.092963in}{0.531640in}}%
\pgfpathlineto{\pgfqpoint{4.105263in}{0.531993in}}%
\pgfpathlineto{\pgfqpoint{4.119203in}{0.531626in}}%
\pgfpathlineto{\pgfqpoint{4.132733in}{0.531700in}}%
\pgfpathlineto{\pgfqpoint{4.146673in}{0.531663in}}%
\pgfpathlineto{\pgfqpoint{4.161023in}{0.531453in}}%
\pgfpathlineto{\pgfqpoint{4.178653in}{0.531696in}}%
\pgfpathlineto{\pgfqpoint{4.303703in}{0.531191in}}%
\pgfpathlineto{\pgfqpoint{4.462373in}{0.530841in}}%
\pgfpathlineto{\pgfqpoint{5.269661in}{0.529998in}}%
\pgfpathlineto{\pgfqpoint{5.657521in}{0.529791in}}%
\pgfpathlineto{\pgfqpoint{5.657521in}{0.529791in}}%
\pgfusepath{stroke}%
\end{pgfscope}%
\begin{pgfscope}%
\pgfpathrectangle{\pgfqpoint{3.505455in}{0.528000in}}{\pgfqpoint{2.254545in}{1.680000in}}%
\pgfusepath{clip}%
\pgfsetrectcap%
\pgfsetroundjoin%
\pgfsetlinewidth{1.505625pt}%
\definecolor{currentstroke}{rgb}{0.580392,0.403922,0.741176}%
\pgfsetstrokecolor{currentstroke}%
\pgfsetdash{}{0pt}%
\pgfpathmoveto{\pgfqpoint{3.607934in}{1.461333in}}%
\pgfpathlineto{\pgfqpoint{3.608344in}{1.472227in}}%
\pgfpathlineto{\pgfqpoint{3.608754in}{1.464504in}}%
\pgfpathlineto{\pgfqpoint{3.609984in}{1.332353in}}%
\pgfpathlineto{\pgfqpoint{3.613264in}{0.735789in}}%
\pgfpathlineto{\pgfqpoint{3.614084in}{0.821313in}}%
\pgfpathlineto{\pgfqpoint{3.616134in}{0.908995in}}%
\pgfpathlineto{\pgfqpoint{3.616954in}{0.900131in}}%
\pgfpathlineto{\pgfqpoint{3.618594in}{0.833592in}}%
\pgfpathlineto{\pgfqpoint{3.619824in}{0.773973in}}%
\pgfpathlineto{\pgfqpoint{3.622284in}{1.012164in}}%
\pgfpathlineto{\pgfqpoint{3.622694in}{1.012764in}}%
\pgfpathlineto{\pgfqpoint{3.623514in}{0.980285in}}%
\pgfpathlineto{\pgfqpoint{3.625974in}{0.730678in}}%
\pgfpathlineto{\pgfqpoint{3.626794in}{0.778525in}}%
\pgfpathlineto{\pgfqpoint{3.628024in}{0.802912in}}%
\pgfpathlineto{\pgfqpoint{3.628434in}{0.798747in}}%
\pgfpathlineto{\pgfqpoint{3.630074in}{0.734429in}}%
\pgfpathlineto{\pgfqpoint{3.632124in}{0.613635in}}%
\pgfpathlineto{\pgfqpoint{3.632534in}{0.635977in}}%
\pgfpathlineto{\pgfqpoint{3.634994in}{0.701680in}}%
\pgfpathlineto{\pgfqpoint{3.635404in}{0.701592in}}%
\pgfpathlineto{\pgfqpoint{3.636634in}{0.685304in}}%
\pgfpathlineto{\pgfqpoint{3.638274in}{0.633726in}}%
\pgfpathlineto{\pgfqpoint{3.638684in}{0.641603in}}%
\pgfpathlineto{\pgfqpoint{3.641144in}{0.717653in}}%
\pgfpathlineto{\pgfqpoint{3.641964in}{0.702218in}}%
\pgfpathlineto{\pgfqpoint{3.644014in}{0.623916in}}%
\pgfpathlineto{\pgfqpoint{3.644424in}{0.631053in}}%
\pgfpathlineto{\pgfqpoint{3.646474in}{0.661060in}}%
\pgfpathlineto{\pgfqpoint{3.647294in}{0.654997in}}%
\pgfpathlineto{\pgfqpoint{3.649344in}{0.606455in}}%
\pgfpathlineto{\pgfqpoint{3.650574in}{0.576572in}}%
\pgfpathlineto{\pgfqpoint{3.650984in}{0.587727in}}%
\pgfpathlineto{\pgfqpoint{3.653444in}{0.621524in}}%
\pgfpathlineto{\pgfqpoint{3.653854in}{0.621374in}}%
\pgfpathlineto{\pgfqpoint{3.655084in}{0.611946in}}%
\pgfpathlineto{\pgfqpoint{3.656724in}{0.582611in}}%
\pgfpathlineto{\pgfqpoint{3.657544in}{0.586171in}}%
\pgfpathlineto{\pgfqpoint{3.659184in}{0.604097in}}%
\pgfpathlineto{\pgfqpoint{3.660004in}{0.599392in}}%
\pgfpathlineto{\pgfqpoint{3.662054in}{0.577193in}}%
\pgfpathlineto{\pgfqpoint{3.664514in}{0.602542in}}%
\pgfpathlineto{\pgfqpoint{3.664924in}{0.602459in}}%
\pgfpathlineto{\pgfqpoint{3.666154in}{0.594839in}}%
\pgfpathlineto{\pgfqpoint{3.668614in}{0.556979in}}%
\pgfpathlineto{\pgfqpoint{3.669434in}{0.566054in}}%
\pgfpathlineto{\pgfqpoint{3.671894in}{0.583078in}}%
\pgfpathlineto{\pgfqpoint{3.672714in}{0.581377in}}%
\pgfpathlineto{\pgfqpoint{3.674354in}{0.568520in}}%
\pgfpathlineto{\pgfqpoint{3.675584in}{0.554500in}}%
\pgfpathlineto{\pgfqpoint{3.676404in}{0.555684in}}%
\pgfpathlineto{\pgfqpoint{3.678044in}{0.551363in}}%
\pgfpathlineto{\pgfqpoint{3.678454in}{0.553477in}}%
\pgfpathlineto{\pgfqpoint{3.680504in}{0.561003in}}%
\pgfpathlineto{\pgfqpoint{3.682964in}{0.573904in}}%
\pgfpathlineto{\pgfqpoint{3.683784in}{0.572457in}}%
\pgfpathlineto{\pgfqpoint{3.685424in}{0.562205in}}%
\pgfpathlineto{\pgfqpoint{3.687064in}{0.547257in}}%
\pgfpathlineto{\pgfqpoint{3.687474in}{0.550887in}}%
\pgfpathlineto{\pgfqpoint{3.689934in}{0.562571in}}%
\pgfpathlineto{\pgfqpoint{3.690754in}{0.562136in}}%
\pgfpathlineto{\pgfqpoint{3.692394in}{0.555388in}}%
\pgfpathlineto{\pgfqpoint{3.696084in}{0.535968in}}%
\pgfpathlineto{\pgfqpoint{3.698544in}{0.549590in}}%
\pgfpathlineto{\pgfqpoint{3.701004in}{0.558285in}}%
\pgfpathlineto{\pgfqpoint{3.701824in}{0.557845in}}%
\pgfpathlineto{\pgfqpoint{3.703464in}{0.552427in}}%
\pgfpathlineto{\pgfqpoint{3.705514in}{0.542987in}}%
\pgfpathlineto{\pgfqpoint{3.705924in}{0.545049in}}%
\pgfpathlineto{\pgfqpoint{3.708384in}{0.551168in}}%
\pgfpathlineto{\pgfqpoint{3.709614in}{0.549808in}}%
\pgfpathlineto{\pgfqpoint{3.711664in}{0.542279in}}%
\pgfpathlineto{\pgfqpoint{3.714124in}{0.531780in}}%
\pgfpathlineto{\pgfqpoint{3.718634in}{0.548857in}}%
\pgfpathlineto{\pgfqpoint{3.719864in}{0.549205in}}%
\pgfpathlineto{\pgfqpoint{3.721504in}{0.546286in}}%
\pgfpathlineto{\pgfqpoint{3.723554in}{0.539083in}}%
\pgfpathlineto{\pgfqpoint{3.724374in}{0.541192in}}%
\pgfpathlineto{\pgfqpoint{3.726834in}{0.544364in}}%
\pgfpathlineto{\pgfqpoint{3.728474in}{0.542457in}}%
\pgfpathlineto{\pgfqpoint{3.732164in}{0.532729in}}%
\pgfpathlineto{\pgfqpoint{3.732984in}{0.534918in}}%
\pgfpathlineto{\pgfqpoint{3.736674in}{0.543444in}}%
\pgfpathlineto{\pgfqpoint{3.738314in}{0.543701in}}%
\pgfpathlineto{\pgfqpoint{3.740364in}{0.540760in}}%
\pgfpathlineto{\pgfqpoint{3.742004in}{0.537214in}}%
\pgfpathlineto{\pgfqpoint{3.742414in}{0.537865in}}%
\pgfpathlineto{\pgfqpoint{3.744874in}{0.540225in}}%
\pgfpathlineto{\pgfqpoint{3.746514in}{0.539225in}}%
\pgfpathlineto{\pgfqpoint{3.750614in}{0.533616in}}%
\pgfpathlineto{\pgfqpoint{3.751024in}{0.534410in}}%
\pgfpathlineto{\pgfqpoint{3.755124in}{0.540306in}}%
\pgfpathlineto{\pgfqpoint{3.757174in}{0.539975in}}%
\pgfpathlineto{\pgfqpoint{3.761684in}{0.537072in}}%
\pgfpathlineto{\pgfqpoint{3.764144in}{0.537320in}}%
\pgfpathlineto{\pgfqpoint{3.767834in}{0.533829in}}%
\pgfpathlineto{\pgfqpoint{3.769064in}{0.534089in}}%
\pgfpathlineto{\pgfqpoint{3.773574in}{0.538184in}}%
\pgfpathlineto{\pgfqpoint{3.776034in}{0.537524in}}%
\pgfpathlineto{\pgfqpoint{3.780134in}{0.535629in}}%
\pgfpathlineto{\pgfqpoint{3.783004in}{0.535416in}}%
\pgfpathlineto{\pgfqpoint{3.787104in}{0.533858in}}%
\pgfpathlineto{\pgfqpoint{3.792024in}{0.536712in}}%
\pgfpathlineto{\pgfqpoint{3.795304in}{0.535666in}}%
\pgfpathlineto{\pgfqpoint{3.798584in}{0.534807in}}%
\pgfpathlineto{\pgfqpoint{3.802684in}{0.534090in}}%
\pgfpathlineto{\pgfqpoint{3.805554in}{0.533892in}}%
\pgfpathlineto{\pgfqpoint{3.810884in}{0.535650in}}%
\pgfpathlineto{\pgfqpoint{3.821954in}{0.533415in}}%
\pgfpathlineto{\pgfqpoint{3.824004in}{0.533802in}}%
\pgfpathlineto{\pgfqpoint{3.829744in}{0.534836in}}%
\pgfpathlineto{\pgfqpoint{3.843684in}{0.533945in}}%
\pgfpathlineto{\pgfqpoint{3.849424in}{0.534089in}}%
\pgfpathlineto{\pgfqpoint{3.859673in}{0.533275in}}%
\pgfpathlineto{\pgfqpoint{3.868283in}{0.533637in}}%
\pgfpathlineto{\pgfqpoint{3.878533in}{0.533206in}}%
\pgfpathlineto{\pgfqpoint{3.887963in}{0.533205in}}%
\pgfpathlineto{\pgfqpoint{3.898213in}{0.533153in}}%
\pgfpathlineto{\pgfqpoint{3.909283in}{0.532810in}}%
\pgfpathlineto{\pgfqpoint{3.936343in}{0.532841in}}%
\pgfpathlineto{\pgfqpoint{3.982673in}{0.532373in}}%
\pgfpathlineto{\pgfqpoint{4.145033in}{0.531638in}}%
\pgfpathlineto{\pgfqpoint{4.634162in}{0.530578in}}%
\pgfpathlineto{\pgfqpoint{5.657521in}{0.529793in}}%
\pgfpathlineto{\pgfqpoint{5.657521in}{0.529793in}}%
\pgfusepath{stroke}%
\end{pgfscope}%
\begin{pgfscope}%
\pgfpathrectangle{\pgfqpoint{3.505455in}{0.528000in}}{\pgfqpoint{2.254545in}{1.680000in}}%
\pgfusepath{clip}%
\pgfsetrectcap%
\pgfsetroundjoin%
\pgfsetlinewidth{1.505625pt}%
\definecolor{currentstroke}{rgb}{0.549020,0.337255,0.294118}%
\pgfsetstrokecolor{currentstroke}%
\pgfsetdash{}{0pt}%
\pgfpathmoveto{\pgfqpoint{3.607934in}{1.461333in}}%
\pgfpathlineto{\pgfqpoint{3.617774in}{1.460249in}}%
\pgfpathlineto{\pgfqpoint{3.627204in}{1.456913in}}%
\pgfpathlineto{\pgfqpoint{3.635814in}{1.451520in}}%
\pgfpathlineto{\pgfqpoint{3.644014in}{1.443781in}}%
\pgfpathlineto{\pgfqpoint{3.652214in}{1.432841in}}%
\pgfpathlineto{\pgfqpoint{3.660414in}{1.417894in}}%
\pgfpathlineto{\pgfqpoint{3.668614in}{1.397993in}}%
\pgfpathlineto{\pgfqpoint{3.676814in}{1.372087in}}%
\pgfpathlineto{\pgfqpoint{3.685424in}{1.337221in}}%
\pgfpathlineto{\pgfqpoint{3.694444in}{1.290958in}}%
\pgfpathlineto{\pgfqpoint{3.703874in}{1.230672in}}%
\pgfpathlineto{\pgfqpoint{3.709614in}{1.190179in}}%
\pgfpathlineto{\pgfqpoint{3.726014in}{1.420559in}}%
\pgfpathlineto{\pgfqpoint{3.754304in}{1.857030in}}%
\pgfpathlineto{\pgfqpoint{3.774804in}{2.155686in}}%
\pgfpathlineto{\pgfqpoint{3.779585in}{2.218000in}}%
\pgfpathmoveto{\pgfqpoint{3.889312in}{2.218000in}}%
\pgfpathlineto{\pgfqpoint{3.897803in}{2.051334in}}%
\pgfpathlineto{\pgfqpoint{3.907233in}{1.819910in}}%
\pgfpathlineto{\pgfqpoint{3.917073in}{1.521837in}}%
\pgfpathlineto{\pgfqpoint{3.928553in}{1.101454in}}%
\pgfpathlineto{\pgfqpoint{3.933883in}{0.889013in}}%
\pgfpathlineto{\pgfqpoint{3.934703in}{0.892175in}}%
\pgfpathlineto{\pgfqpoint{3.965043in}{1.107027in}}%
\pgfpathlineto{\pgfqpoint{3.974473in}{1.156355in}}%
\pgfpathlineto{\pgfqpoint{3.981853in}{1.184430in}}%
\pgfpathlineto{\pgfqpoint{3.987593in}{1.198830in}}%
\pgfpathlineto{\pgfqpoint{3.992103in}{1.205058in}}%
\pgfpathlineto{\pgfqpoint{3.995383in}{1.206514in}}%
\pgfpathlineto{\pgfqpoint{3.998253in}{1.205489in}}%
\pgfpathlineto{\pgfqpoint{4.001533in}{1.201471in}}%
\pgfpathlineto{\pgfqpoint{4.005223in}{1.192961in}}%
\pgfpathlineto{\pgfqpoint{4.009733in}{1.176051in}}%
\pgfpathlineto{\pgfqpoint{4.014653in}{1.147958in}}%
\pgfpathlineto{\pgfqpoint{4.019983in}{1.103452in}}%
\pgfpathlineto{\pgfqpoint{4.025723in}{1.034256in}}%
\pgfpathlineto{\pgfqpoint{4.031463in}{0.935348in}}%
\pgfpathlineto{\pgfqpoint{4.035153in}{0.851193in}}%
\pgfpathlineto{\pgfqpoint{4.035973in}{0.860439in}}%
\pgfpathlineto{\pgfqpoint{4.042533in}{1.000537in}}%
\pgfpathlineto{\pgfqpoint{4.050323in}{1.222261in}}%
\pgfpathlineto{\pgfqpoint{4.060983in}{1.522857in}}%
\pgfpathlineto{\pgfqpoint{4.064263in}{1.565541in}}%
\pgfpathlineto{\pgfqpoint{4.065903in}{1.571110in}}%
\pgfpathlineto{\pgfqpoint{4.066723in}{1.569491in}}%
\pgfpathlineto{\pgfqpoint{4.068363in}{1.557008in}}%
\pgfpathlineto{\pgfqpoint{4.070823in}{1.514559in}}%
\pgfpathlineto{\pgfqpoint{4.074103in}{1.414714in}}%
\pgfpathlineto{\pgfqpoint{4.078203in}{1.249932in}}%
\pgfpathlineto{\pgfqpoint{4.078613in}{1.273887in}}%
\pgfpathlineto{\pgfqpoint{4.089683in}{2.067091in}}%
\pgfpathlineto{\pgfqpoint{4.092413in}{2.218000in}}%
\pgfpathmoveto{\pgfqpoint{4.141080in}{2.218000in}}%
\pgfpathlineto{\pgfqpoint{4.147493in}{1.883901in}}%
\pgfpathlineto{\pgfqpoint{4.154873in}{1.378096in}}%
\pgfpathlineto{\pgfqpoint{4.160613in}{0.906727in}}%
\pgfpathlineto{\pgfqpoint{4.161843in}{0.919511in}}%
\pgfpathlineto{\pgfqpoint{4.177423in}{1.138853in}}%
\pgfpathlineto{\pgfqpoint{4.184393in}{1.206468in}}%
\pgfpathlineto{\pgfqpoint{4.189723in}{1.239165in}}%
\pgfpathlineto{\pgfqpoint{4.193823in}{1.252088in}}%
\pgfpathlineto{\pgfqpoint{4.196283in}{1.254370in}}%
\pgfpathlineto{\pgfqpoint{4.197923in}{1.253453in}}%
\pgfpathlineto{\pgfqpoint{4.199973in}{1.249377in}}%
\pgfpathlineto{\pgfqpoint{4.202843in}{1.237690in}}%
\pgfpathlineto{\pgfqpoint{4.206533in}{1.210800in}}%
\pgfpathlineto{\pgfqpoint{4.210633in}{1.161289in}}%
\pgfpathlineto{\pgfqpoint{4.214733in}{1.083721in}}%
\pgfpathlineto{\pgfqpoint{4.219243in}{0.951206in}}%
\pgfpathlineto{\pgfqpoint{4.221703in}{0.879710in}}%
\pgfpathlineto{\pgfqpoint{4.225393in}{1.014862in}}%
\pgfpathlineto{\pgfqpoint{4.231133in}{1.311581in}}%
\pgfpathlineto{\pgfqpoint{4.238103in}{1.659297in}}%
\pgfpathlineto{\pgfqpoint{4.240563in}{1.697024in}}%
\pgfpathlineto{\pgfqpoint{4.240973in}{1.696420in}}%
\pgfpathlineto{\pgfqpoint{4.242203in}{1.681795in}}%
\pgfpathlineto{\pgfqpoint{4.244253in}{1.613815in}}%
\pgfpathlineto{\pgfqpoint{4.247533in}{1.400320in}}%
\pgfpathlineto{\pgfqpoint{4.248763in}{1.330505in}}%
\pgfpathlineto{\pgfqpoint{4.250813in}{1.573233in}}%
\pgfpathlineto{\pgfqpoint{4.256590in}{2.218000in}}%
\pgfpathmoveto{\pgfqpoint{4.296171in}{2.218000in}}%
\pgfpathlineto{\pgfqpoint{4.301653in}{1.735509in}}%
\pgfpathlineto{\pgfqpoint{4.309443in}{0.943634in}}%
\pgfpathlineto{\pgfqpoint{4.310263in}{0.960102in}}%
\pgfpathlineto{\pgfqpoint{4.321743in}{1.195234in}}%
\pgfpathlineto{\pgfqpoint{4.327483in}{1.270851in}}%
\pgfpathlineto{\pgfqpoint{4.331993in}{1.304983in}}%
\pgfpathlineto{\pgfqpoint{4.335273in}{1.315480in}}%
\pgfpathlineto{\pgfqpoint{4.336913in}{1.315992in}}%
\pgfpathlineto{\pgfqpoint{4.338553in}{1.313136in}}%
\pgfpathlineto{\pgfqpoint{4.341013in}{1.301952in}}%
\pgfpathlineto{\pgfqpoint{4.343883in}{1.276769in}}%
\pgfpathlineto{\pgfqpoint{4.347163in}{1.227578in}}%
\pgfpathlineto{\pgfqpoint{4.350853in}{1.135069in}}%
\pgfpathlineto{\pgfqpoint{4.354543in}{0.991296in}}%
\pgfpathlineto{\pgfqpoint{4.356183in}{0.992543in}}%
\pgfpathlineto{\pgfqpoint{4.357823in}{0.992469in}}%
\pgfpathlineto{\pgfqpoint{4.358643in}{1.023919in}}%
\pgfpathlineto{\pgfqpoint{4.363563in}{1.398795in}}%
\pgfpathlineto{\pgfqpoint{4.368483in}{1.737204in}}%
\pgfpathlineto{\pgfqpoint{4.370123in}{1.769915in}}%
\pgfpathlineto{\pgfqpoint{4.370533in}{1.769436in}}%
\pgfpathlineto{\pgfqpoint{4.371763in}{1.746618in}}%
\pgfpathlineto{\pgfqpoint{4.373813in}{1.656390in}}%
\pgfpathlineto{\pgfqpoint{4.374223in}{1.676343in}}%
\pgfpathlineto{\pgfqpoint{4.376683in}{1.745759in}}%
\pgfpathlineto{\pgfqpoint{4.377913in}{1.752500in}}%
\pgfpathlineto{\pgfqpoint{4.378323in}{1.751548in}}%
\pgfpathlineto{\pgfqpoint{4.379963in}{1.736436in}}%
\pgfpathlineto{\pgfqpoint{4.380373in}{1.743009in}}%
\pgfpathlineto{\pgfqpoint{4.384237in}{2.218000in}}%
\pgfpathmoveto{\pgfqpoint{4.413857in}{2.218000in}}%
\pgfpathlineto{\pgfqpoint{4.421783in}{1.663130in}}%
\pgfpathlineto{\pgfqpoint{4.426293in}{1.389287in}}%
\pgfpathlineto{\pgfqpoint{4.431623in}{0.984826in}}%
\pgfpathlineto{\pgfqpoint{4.432443in}{0.996953in}}%
\pgfpathlineto{\pgfqpoint{4.436953in}{1.046429in}}%
\pgfpathlineto{\pgfqpoint{4.439823in}{1.058473in}}%
\pgfpathlineto{\pgfqpoint{4.441053in}{1.058213in}}%
\pgfpathlineto{\pgfqpoint{4.442693in}{1.052502in}}%
\pgfpathlineto{\pgfqpoint{4.443513in}{1.047302in}}%
\pgfpathlineto{\pgfqpoint{4.443923in}{1.047625in}}%
\pgfpathlineto{\pgfqpoint{4.450073in}{1.133585in}}%
\pgfpathlineto{\pgfqpoint{4.452943in}{1.152803in}}%
\pgfpathlineto{\pgfqpoint{4.454173in}{1.154340in}}%
\pgfpathlineto{\pgfqpoint{4.455403in}{1.151056in}}%
\pgfpathlineto{\pgfqpoint{4.457453in}{1.134242in}}%
\pgfpathlineto{\pgfqpoint{4.460733in}{1.080870in}}%
\pgfpathlineto{\pgfqpoint{4.465653in}{0.968694in}}%
\pgfpathlineto{\pgfqpoint{4.466883in}{0.977048in}}%
\pgfpathlineto{\pgfqpoint{4.467703in}{0.979648in}}%
\pgfpathlineto{\pgfqpoint{4.468113in}{0.979524in}}%
\pgfpathlineto{\pgfqpoint{4.469343in}{0.974024in}}%
\pgfpathlineto{\pgfqpoint{4.471393in}{0.955110in}}%
\pgfpathlineto{\pgfqpoint{4.472213in}{0.956881in}}%
\pgfpathlineto{\pgfqpoint{4.473033in}{0.954977in}}%
\pgfpathlineto{\pgfqpoint{4.474673in}{0.941363in}}%
\pgfpathlineto{\pgfqpoint{4.476723in}{0.908746in}}%
\pgfpathlineto{\pgfqpoint{4.477133in}{0.923312in}}%
\pgfpathlineto{\pgfqpoint{4.480413in}{1.256072in}}%
\pgfpathlineto{\pgfqpoint{4.485743in}{1.931091in}}%
\pgfpathlineto{\pgfqpoint{4.486153in}{1.932996in}}%
\pgfpathlineto{\pgfqpoint{4.486973in}{1.899547in}}%
\pgfpathlineto{\pgfqpoint{4.487383in}{1.864798in}}%
\pgfpathlineto{\pgfqpoint{4.488768in}{2.218000in}}%
\pgfpathmoveto{\pgfqpoint{4.498400in}{2.218000in}}%
\pgfpathlineto{\pgfqpoint{4.499682in}{2.181746in}}%
\pgfpathlineto{\pgfqpoint{4.501732in}{2.089888in}}%
\pgfpathlineto{\pgfqpoint{4.502142in}{2.107109in}}%
\pgfpathlineto{\pgfqpoint{4.505422in}{2.197635in}}%
\pgfpathlineto{\pgfqpoint{4.507472in}{2.215788in}}%
\pgfpathlineto{\pgfqpoint{4.507882in}{2.215927in}}%
\pgfpathlineto{\pgfqpoint{4.508702in}{2.212605in}}%
\pgfpathlineto{\pgfqpoint{4.510342in}{2.190921in}}%
\pgfpathlineto{\pgfqpoint{4.512802in}{2.117508in}}%
\pgfpathlineto{\pgfqpoint{4.516082in}{1.936456in}}%
\pgfpathlineto{\pgfqpoint{4.524282in}{1.322861in}}%
\pgfpathlineto{\pgfqpoint{4.524692in}{1.371301in}}%
\pgfpathlineto{\pgfqpoint{4.527972in}{1.649186in}}%
\pgfpathlineto{\pgfqpoint{4.529612in}{1.681798in}}%
\pgfpathlineto{\pgfqpoint{4.530432in}{1.669674in}}%
\pgfpathlineto{\pgfqpoint{4.532072in}{1.592450in}}%
\pgfpathlineto{\pgfqpoint{4.533712in}{1.534972in}}%
\pgfpathlineto{\pgfqpoint{4.534122in}{1.531532in}}%
\pgfpathlineto{\pgfqpoint{4.534532in}{1.536655in}}%
\pgfpathlineto{\pgfqpoint{4.537402in}{1.630709in}}%
\pgfpathlineto{\pgfqpoint{4.538632in}{1.641517in}}%
\pgfpathlineto{\pgfqpoint{4.539042in}{1.641112in}}%
\pgfpathlineto{\pgfqpoint{4.540272in}{1.627821in}}%
\pgfpathlineto{\pgfqpoint{4.542322in}{1.565916in}}%
\pgfpathlineto{\pgfqpoint{4.545602in}{1.372460in}}%
\pgfpathlineto{\pgfqpoint{4.548472in}{1.163915in}}%
\pgfpathlineto{\pgfqpoint{4.548882in}{1.172166in}}%
\pgfpathlineto{\pgfqpoint{4.552982in}{1.231181in}}%
\pgfpathlineto{\pgfqpoint{4.555032in}{1.240295in}}%
\pgfpathlineto{\pgfqpoint{4.555852in}{1.238961in}}%
\pgfpathlineto{\pgfqpoint{4.557082in}{1.230512in}}%
\pgfpathlineto{\pgfqpoint{4.559132in}{1.194633in}}%
\pgfpathlineto{\pgfqpoint{4.561592in}{1.111165in}}%
\pgfpathlineto{\pgfqpoint{4.562002in}{1.129690in}}%
\pgfpathlineto{\pgfqpoint{4.564872in}{1.317906in}}%
\pgfpathlineto{\pgfqpoint{4.569792in}{1.710281in}}%
\pgfpathlineto{\pgfqpoint{4.570612in}{1.693375in}}%
\pgfpathlineto{\pgfqpoint{4.571022in}{1.674350in}}%
\pgfpathlineto{\pgfqpoint{4.572977in}{2.218000in}}%
\pgfpathmoveto{\pgfqpoint{4.577514in}{2.218000in}}%
\pgfpathlineto{\pgfqpoint{4.586602in}{1.449496in}}%
\pgfpathlineto{\pgfqpoint{4.587012in}{1.501545in}}%
\pgfpathlineto{\pgfqpoint{4.589472in}{1.874373in}}%
\pgfpathlineto{\pgfqpoint{4.590292in}{1.888511in}}%
\pgfpathlineto{\pgfqpoint{4.591522in}{1.842870in}}%
\pgfpathlineto{\pgfqpoint{4.592251in}{2.218000in}}%
\pgfpathmoveto{\pgfqpoint{4.621066in}{2.218000in}}%
\pgfpathlineto{\pgfqpoint{4.624322in}{1.910008in}}%
\pgfpathlineto{\pgfqpoint{4.629242in}{1.244092in}}%
\pgfpathlineto{\pgfqpoint{4.630062in}{1.309017in}}%
\pgfpathlineto{\pgfqpoint{4.633342in}{1.663884in}}%
\pgfpathlineto{\pgfqpoint{4.633752in}{1.665664in}}%
\pgfpathlineto{\pgfqpoint{4.634982in}{1.624023in}}%
\pgfpathlineto{\pgfqpoint{4.637442in}{2.129161in}}%
\pgfpathlineto{\pgfqpoint{4.638262in}{2.142418in}}%
\pgfpathlineto{\pgfqpoint{4.638672in}{2.136027in}}%
\pgfpathlineto{\pgfqpoint{4.640722in}{2.051519in}}%
\pgfpathlineto{\pgfqpoint{4.642362in}{1.986181in}}%
\pgfpathlineto{\pgfqpoint{4.644412in}{2.064999in}}%
\pgfpathlineto{\pgfqpoint{4.644822in}{2.067110in}}%
\pgfpathlineto{\pgfqpoint{4.645232in}{2.065339in}}%
\pgfpathlineto{\pgfqpoint{4.646462in}{2.038591in}}%
\pgfpathlineto{\pgfqpoint{4.648512in}{1.929674in}}%
\pgfpathlineto{\pgfqpoint{4.651792in}{1.609401in}}%
\pgfpathlineto{\pgfqpoint{4.654252in}{1.284973in}}%
\pgfpathlineto{\pgfqpoint{4.655072in}{1.296992in}}%
\pgfpathlineto{\pgfqpoint{4.658352in}{1.343780in}}%
\pgfpathlineto{\pgfqpoint{4.659582in}{1.348542in}}%
\pgfpathlineto{\pgfqpoint{4.659992in}{1.348052in}}%
\pgfpathlineto{\pgfqpoint{4.661222in}{1.339015in}}%
\pgfpathlineto{\pgfqpoint{4.662862in}{1.303685in}}%
\pgfpathlineto{\pgfqpoint{4.664912in}{1.199440in}}%
\pgfpathlineto{\pgfqpoint{4.665322in}{1.230063in}}%
\pgfpathlineto{\pgfqpoint{4.667372in}{1.487193in}}%
\pgfpathlineto{\pgfqpoint{4.670535in}{2.218000in}}%
\pgfpathmoveto{\pgfqpoint{4.688101in}{2.218000in}}%
\pgfpathlineto{\pgfqpoint{4.690332in}{1.693860in}}%
\pgfpathlineto{\pgfqpoint{4.690742in}{1.858164in}}%
\pgfpathlineto{\pgfqpoint{4.691748in}{2.218000in}}%
\pgfpathmoveto{\pgfqpoint{4.749942in}{2.218000in}}%
\pgfpathlineto{\pgfqpoint{4.753472in}{1.552206in}}%
\pgfpathlineto{\pgfqpoint{4.755522in}{1.177090in}}%
\pgfpathlineto{\pgfqpoint{4.755932in}{1.235280in}}%
\pgfpathlineto{\pgfqpoint{4.757982in}{1.703480in}}%
\pgfpathlineto{\pgfqpoint{4.759604in}{2.218000in}}%
\pgfpathmoveto{\pgfqpoint{4.778110in}{2.218000in}}%
\pgfpathlineto{\pgfqpoint{4.781352in}{1.685705in}}%
\pgfpathlineto{\pgfqpoint{4.782582in}{1.435293in}}%
\pgfpathlineto{\pgfqpoint{4.783402in}{1.469073in}}%
\pgfpathlineto{\pgfqpoint{4.788732in}{1.646776in}}%
\pgfpathlineto{\pgfqpoint{4.792422in}{1.704106in}}%
\pgfpathlineto{\pgfqpoint{4.795702in}{1.724657in}}%
\pgfpathlineto{\pgfqpoint{4.797342in}{1.726992in}}%
\pgfpathlineto{\pgfqpoint{4.798572in}{1.725402in}}%
\pgfpathlineto{\pgfqpoint{4.800212in}{1.718276in}}%
\pgfpathlineto{\pgfqpoint{4.802262in}{1.698746in}}%
\pgfpathlineto{\pgfqpoint{4.804722in}{1.647729in}}%
\pgfpathlineto{\pgfqpoint{4.806772in}{1.550028in}}%
\pgfpathlineto{\pgfqpoint{4.808002in}{1.451169in}}%
\pgfpathlineto{\pgfqpoint{4.808412in}{1.496958in}}%
\pgfpathlineto{\pgfqpoint{4.810052in}{1.843710in}}%
\pgfpathlineto{\pgfqpoint{4.811149in}{2.218000in}}%
\pgfpathmoveto{\pgfqpoint{4.826597in}{2.218000in}}%
\pgfpathlineto{\pgfqpoint{4.829322in}{1.770280in}}%
\pgfpathlineto{\pgfqpoint{4.830962in}{1.444220in}}%
\pgfpathlineto{\pgfqpoint{4.831782in}{1.479973in}}%
\pgfpathlineto{\pgfqpoint{4.836702in}{1.646964in}}%
\pgfpathlineto{\pgfqpoint{4.839982in}{1.691277in}}%
\pgfpathlineto{\pgfqpoint{4.841622in}{1.696874in}}%
\pgfpathlineto{\pgfqpoint{4.842032in}{1.696628in}}%
\pgfpathlineto{\pgfqpoint{4.843262in}{1.691580in}}%
\pgfpathlineto{\pgfqpoint{4.844902in}{1.672497in}}%
\pgfpathlineto{\pgfqpoint{4.846952in}{1.615415in}}%
\pgfpathlineto{\pgfqpoint{4.848592in}{1.509550in}}%
\pgfpathlineto{\pgfqpoint{4.849002in}{1.465552in}}%
\pgfpathlineto{\pgfqpoint{4.849412in}{1.468146in}}%
\pgfpathlineto{\pgfqpoint{4.851052in}{1.780778in}}%
\pgfpathlineto{\pgfqpoint{4.852452in}{2.218000in}}%
\pgfpathmoveto{\pgfqpoint{4.854292in}{2.218000in}}%
\pgfpathlineto{\pgfqpoint{4.854332in}{2.206820in}}%
\pgfpathlineto{\pgfqpoint{4.854433in}{2.218000in}}%
\pgfpathmoveto{\pgfqpoint{4.863434in}{2.218000in}}%
\pgfpathlineto{\pgfqpoint{4.865812in}{1.893673in}}%
\pgfpathlineto{\pgfqpoint{4.868272in}{1.436369in}}%
\pgfpathlineto{\pgfqpoint{4.869092in}{1.473588in}}%
\pgfpathlineto{\pgfqpoint{4.873602in}{1.627743in}}%
\pgfpathlineto{\pgfqpoint{4.876062in}{1.658652in}}%
\pgfpathlineto{\pgfqpoint{4.876882in}{1.660824in}}%
\pgfpathlineto{\pgfqpoint{4.877292in}{1.660323in}}%
\pgfpathlineto{\pgfqpoint{4.878522in}{1.651703in}}%
\pgfpathlineto{\pgfqpoint{4.880162in}{1.618190in}}%
\pgfpathlineto{\pgfqpoint{4.881802in}{1.538683in}}%
\pgfpathlineto{\pgfqpoint{4.883032in}{1.449894in}}%
\pgfpathlineto{\pgfqpoint{4.884672in}{1.743807in}}%
\pgfpathlineto{\pgfqpoint{4.886722in}{2.208413in}}%
\pgfpathlineto{\pgfqpoint{4.887132in}{2.172994in}}%
\pgfpathlineto{\pgfqpoint{4.887952in}{2.009124in}}%
\pgfpathlineto{\pgfqpoint{4.888409in}{2.218000in}}%
\pgfpathmoveto{\pgfqpoint{4.890220in}{2.218000in}}%
\pgfpathlineto{\pgfqpoint{4.891232in}{2.125431in}}%
\pgfpathlineto{\pgfqpoint{4.891642in}{2.149360in}}%
\pgfpathlineto{\pgfqpoint{4.892814in}{2.218000in}}%
\pgfpathmoveto{\pgfqpoint{4.893433in}{2.218000in}}%
\pgfpathlineto{\pgfqpoint{4.894102in}{2.197773in}}%
\pgfpathlineto{\pgfqpoint{4.895742in}{2.063767in}}%
\pgfpathlineto{\pgfqpoint{4.898202in}{1.652850in}}%
\pgfpathlineto{\pgfqpoint{4.899432in}{1.419039in}}%
\pgfpathlineto{\pgfqpoint{4.899842in}{1.438444in}}%
\pgfpathlineto{\pgfqpoint{4.904352in}{1.603926in}}%
\pgfpathlineto{\pgfqpoint{4.906812in}{1.631992in}}%
\pgfpathlineto{\pgfqpoint{4.907222in}{1.631924in}}%
\pgfpathlineto{\pgfqpoint{4.908042in}{1.627181in}}%
\pgfpathlineto{\pgfqpoint{4.909682in}{1.594118in}}%
\pgfpathlineto{\pgfqpoint{4.911322in}{1.505873in}}%
\pgfpathlineto{\pgfqpoint{4.912142in}{1.425697in}}%
\pgfpathlineto{\pgfqpoint{4.912552in}{1.472203in}}%
\pgfpathlineto{\pgfqpoint{4.914602in}{1.909196in}}%
\pgfpathlineto{\pgfqpoint{4.915832in}{2.085682in}}%
\pgfpathlineto{\pgfqpoint{4.916652in}{1.903969in}}%
\pgfpathlineto{\pgfqpoint{4.917062in}{1.935651in}}%
\pgfpathlineto{\pgfqpoint{4.918292in}{2.136199in}}%
\pgfpathlineto{\pgfqpoint{4.918702in}{2.119658in}}%
\pgfpathlineto{\pgfqpoint{4.919932in}{2.026917in}}%
\pgfpathlineto{\pgfqpoint{4.920342in}{2.053978in}}%
\pgfpathlineto{\pgfqpoint{4.921572in}{2.114413in}}%
\pgfpathlineto{\pgfqpoint{4.921982in}{2.109486in}}%
\pgfpathlineto{\pgfqpoint{4.923212in}{2.037282in}}%
\pgfpathlineto{\pgfqpoint{4.925262in}{1.748587in}}%
\pgfpathlineto{\pgfqpoint{4.926902in}{1.401097in}}%
\pgfpathlineto{\pgfqpoint{4.927722in}{1.441255in}}%
\pgfpathlineto{\pgfqpoint{4.931822in}{1.589540in}}%
\pgfpathlineto{\pgfqpoint{4.933872in}{1.609455in}}%
\pgfpathlineto{\pgfqpoint{4.934692in}{1.605223in}}%
\pgfpathlineto{\pgfqpoint{4.935922in}{1.581751in}}%
\pgfpathlineto{\pgfqpoint{4.937562in}{1.497989in}}%
\pgfpathlineto{\pgfqpoint{4.938382in}{1.414475in}}%
\pgfpathlineto{\pgfqpoint{4.938792in}{1.460453in}}%
\pgfpathlineto{\pgfqpoint{4.941662in}{1.999353in}}%
\pgfpathlineto{\pgfqpoint{4.942482in}{1.891529in}}%
\pgfpathlineto{\pgfqpoint{4.942892in}{1.753617in}}%
\pgfpathlineto{\pgfqpoint{4.943302in}{1.893207in}}%
\pgfpathlineto{\pgfqpoint{4.944532in}{2.021582in}}%
\pgfpathlineto{\pgfqpoint{4.945762in}{1.947294in}}%
\pgfpathlineto{\pgfqpoint{4.946172in}{1.974412in}}%
\pgfpathlineto{\pgfqpoint{4.947402in}{2.027192in}}%
\pgfpathlineto{\pgfqpoint{4.948222in}{1.995181in}}%
\pgfpathlineto{\pgfqpoint{4.949862in}{1.808478in}}%
\pgfpathlineto{\pgfqpoint{4.951912in}{1.389323in}}%
\pgfpathlineto{\pgfqpoint{4.952732in}{1.430967in}}%
\pgfpathlineto{\pgfqpoint{4.956832in}{1.578452in}}%
\pgfpathlineto{\pgfqpoint{4.958472in}{1.590789in}}%
\pgfpathlineto{\pgfqpoint{4.959292in}{1.584157in}}%
\pgfpathlineto{\pgfqpoint{4.960522in}{1.552402in}}%
\pgfpathlineto{\pgfqpoint{4.962572in}{1.421018in}}%
\pgfpathlineto{\pgfqpoint{4.962982in}{1.470587in}}%
\pgfpathlineto{\pgfqpoint{4.965442in}{1.926630in}}%
\pgfpathlineto{\pgfqpoint{4.966262in}{1.852299in}}%
\pgfpathlineto{\pgfqpoint{4.966672in}{1.727341in}}%
\pgfpathlineto{\pgfqpoint{4.967082in}{1.766071in}}%
\pgfpathlineto{\pgfqpoint{4.968312in}{1.937343in}}%
\pgfpathlineto{\pgfqpoint{4.968722in}{1.926617in}}%
\pgfpathlineto{\pgfqpoint{4.969542in}{1.873366in}}%
\pgfpathlineto{\pgfqpoint{4.969952in}{1.916681in}}%
\pgfpathlineto{\pgfqpoint{4.970772in}{1.955870in}}%
\pgfpathlineto{\pgfqpoint{4.971182in}{1.951089in}}%
\pgfpathlineto{\pgfqpoint{4.972412in}{1.860126in}}%
\pgfpathlineto{\pgfqpoint{4.975282in}{1.395638in}}%
\pgfpathlineto{\pgfqpoint{4.976512in}{1.458224in}}%
\pgfpathlineto{\pgfqpoint{4.979792in}{1.566898in}}%
\pgfpathlineto{\pgfqpoint{4.981022in}{1.575430in}}%
\pgfpathlineto{\pgfqpoint{4.981432in}{1.573479in}}%
\pgfpathlineto{\pgfqpoint{4.982662in}{1.549413in}}%
\pgfpathlineto{\pgfqpoint{4.984302in}{1.448750in}}%
\pgfpathlineto{\pgfqpoint{4.984712in}{1.401794in}}%
\pgfpathlineto{\pgfqpoint{4.985122in}{1.443054in}}%
\pgfpathlineto{\pgfqpoint{4.987582in}{1.872415in}}%
\pgfpathlineto{\pgfqpoint{4.988402in}{1.802434in}}%
\pgfpathlineto{\pgfqpoint{4.988812in}{1.681585in}}%
\pgfpathlineto{\pgfqpoint{4.989222in}{1.691262in}}%
\pgfpathlineto{\pgfqpoint{4.990452in}{1.867167in}}%
\pgfpathlineto{\pgfqpoint{4.990862in}{1.857880in}}%
\pgfpathlineto{\pgfqpoint{4.991682in}{1.830716in}}%
\pgfpathlineto{\pgfqpoint{4.992912in}{1.893474in}}%
\pgfpathlineto{\pgfqpoint{4.993732in}{1.850237in}}%
\pgfpathlineto{\pgfqpoint{4.995372in}{1.613510in}}%
\pgfpathlineto{\pgfqpoint{4.996602in}{1.378491in}}%
\pgfpathlineto{\pgfqpoint{4.997012in}{1.400921in}}%
\pgfpathlineto{\pgfqpoint{5.000702in}{1.548723in}}%
\pgfpathlineto{\pgfqpoint{5.001932in}{1.562120in}}%
\pgfpathlineto{\pgfqpoint{5.002342in}{1.561382in}}%
\pgfpathlineto{\pgfqpoint{5.003162in}{1.550555in}}%
\pgfpathlineto{\pgfqpoint{5.004392in}{1.502725in}}%
\pgfpathlineto{\pgfqpoint{5.005622in}{1.396835in}}%
\pgfpathlineto{\pgfqpoint{5.006032in}{1.444441in}}%
\pgfpathlineto{\pgfqpoint{5.008492in}{1.835501in}}%
\pgfpathlineto{\pgfqpoint{5.008902in}{1.802985in}}%
\pgfpathlineto{\pgfqpoint{5.009722in}{1.576550in}}%
\pgfpathlineto{\pgfqpoint{5.010132in}{1.696390in}}%
\pgfpathlineto{\pgfqpoint{5.011362in}{1.805740in}}%
\pgfpathlineto{\pgfqpoint{5.012182in}{1.770837in}}%
\pgfpathlineto{\pgfqpoint{5.013412in}{1.840210in}}%
\pgfpathlineto{\pgfqpoint{5.014232in}{1.792352in}}%
\pgfpathlineto{\pgfqpoint{5.016282in}{1.441292in}}%
\pgfpathlineto{\pgfqpoint{5.016692in}{1.367269in}}%
\pgfpathlineto{\pgfqpoint{5.017512in}{1.413210in}}%
\pgfpathlineto{\pgfqpoint{5.020792in}{1.540518in}}%
\pgfpathlineto{\pgfqpoint{5.022022in}{1.550312in}}%
\pgfpathlineto{\pgfqpoint{5.022842in}{1.541057in}}%
\pgfpathlineto{\pgfqpoint{5.024072in}{1.493335in}}%
\pgfpathlineto{\pgfqpoint{5.025302in}{1.398077in}}%
\pgfpathlineto{\pgfqpoint{5.025712in}{1.447674in}}%
\pgfpathlineto{\pgfqpoint{5.028172in}{1.792348in}}%
\pgfpathlineto{\pgfqpoint{5.028582in}{1.730664in}}%
\pgfpathlineto{\pgfqpoint{5.029402in}{1.575766in}}%
\pgfpathlineto{\pgfqpoint{5.029812in}{1.690460in}}%
\pgfpathlineto{\pgfqpoint{5.030632in}{1.756973in}}%
\pgfpathlineto{\pgfqpoint{5.031042in}{1.747708in}}%
\pgfpathlineto{\pgfqpoint{5.031452in}{1.722884in}}%
\pgfpathlineto{\pgfqpoint{5.031862in}{1.765158in}}%
\pgfpathlineto{\pgfqpoint{5.032682in}{1.793784in}}%
\pgfpathlineto{\pgfqpoint{5.033092in}{1.777391in}}%
\pgfpathlineto{\pgfqpoint{5.034732in}{1.555835in}}%
\pgfpathlineto{\pgfqpoint{5.035552in}{1.370143in}}%
\pgfpathlineto{\pgfqpoint{5.036372in}{1.399499in}}%
\pgfpathlineto{\pgfqpoint{5.039652in}{1.531478in}}%
\pgfpathlineto{\pgfqpoint{5.040472in}{1.539944in}}%
\pgfpathlineto{\pgfqpoint{5.040882in}{1.539323in}}%
\pgfpathlineto{\pgfqpoint{5.041702in}{1.526473in}}%
\pgfpathlineto{\pgfqpoint{5.042932in}{1.466685in}}%
\pgfpathlineto{\pgfqpoint{5.043752in}{1.383489in}}%
\pgfpathlineto{\pgfqpoint{5.044162in}{1.431465in}}%
\pgfpathlineto{\pgfqpoint{5.046212in}{1.760771in}}%
\pgfpathlineto{\pgfqpoint{5.047032in}{1.694368in}}%
\pgfpathlineto{\pgfqpoint{5.047852in}{1.545286in}}%
\pgfpathlineto{\pgfqpoint{5.048262in}{1.652722in}}%
\pgfpathlineto{\pgfqpoint{5.049082in}{1.711906in}}%
\pgfpathlineto{\pgfqpoint{5.049492in}{1.699726in}}%
\pgfpathlineto{\pgfqpoint{5.049902in}{1.694996in}}%
\pgfpathlineto{\pgfqpoint{5.050722in}{1.752995in}}%
\pgfpathlineto{\pgfqpoint{5.051132in}{1.745618in}}%
\pgfpathlineto{\pgfqpoint{5.052362in}{1.613594in}}%
\pgfpathlineto{\pgfqpoint{5.053592in}{1.352556in}}%
\pgfpathlineto{\pgfqpoint{5.054412in}{1.396828in}}%
\pgfpathlineto{\pgfqpoint{5.057692in}{1.525998in}}%
\pgfpathlineto{\pgfqpoint{5.058512in}{1.530487in}}%
\pgfpathlineto{\pgfqpoint{5.059332in}{1.519109in}}%
\pgfpathlineto{\pgfqpoint{5.060562in}{1.459083in}}%
\pgfpathlineto{\pgfqpoint{5.061382in}{1.383754in}}%
\pgfpathlineto{\pgfqpoint{5.061792in}{1.433378in}}%
\pgfpathlineto{\pgfqpoint{5.063842in}{1.740329in}}%
\pgfpathlineto{\pgfqpoint{5.064252in}{1.715900in}}%
\pgfpathlineto{\pgfqpoint{5.065072in}{1.506680in}}%
\pgfpathlineto{\pgfqpoint{5.065482in}{1.558350in}}%
\pgfpathlineto{\pgfqpoint{5.066302in}{1.668969in}}%
\pgfpathlineto{\pgfqpoint{5.066712in}{1.668387in}}%
\pgfpathlineto{\pgfqpoint{5.067122in}{1.647062in}}%
\pgfpathlineto{\pgfqpoint{5.067942in}{1.713558in}}%
\pgfpathlineto{\pgfqpoint{5.068352in}{1.711231in}}%
\pgfpathlineto{\pgfqpoint{5.069582in}{1.582549in}}%
\pgfpathlineto{\pgfqpoint{5.070812in}{1.349118in}}%
\pgfpathlineto{\pgfqpoint{5.071632in}{1.399435in}}%
\pgfpathlineto{\pgfqpoint{5.074502in}{1.515908in}}%
\pgfpathlineto{\pgfqpoint{5.075322in}{1.522128in}}%
\pgfpathlineto{\pgfqpoint{5.075732in}{1.519140in}}%
\pgfpathlineto{\pgfqpoint{5.076962in}{1.477674in}}%
\pgfpathlineto{\pgfqpoint{5.078192in}{1.388710in}}%
\pgfpathlineto{\pgfqpoint{5.080652in}{1.714051in}}%
\pgfpathlineto{\pgfqpoint{5.081062in}{1.661517in}}%
\pgfpathlineto{\pgfqpoint{5.081882in}{1.452977in}}%
\pgfpathlineto{\pgfqpoint{5.082292in}{1.569155in}}%
\pgfpathlineto{\pgfqpoint{5.083932in}{1.644671in}}%
\pgfpathlineto{\pgfqpoint{5.084752in}{1.679049in}}%
\pgfpathlineto{\pgfqpoint{5.085162in}{1.657539in}}%
\pgfpathlineto{\pgfqpoint{5.087212in}{1.351001in}}%
\pgfpathlineto{\pgfqpoint{5.088442in}{1.425476in}}%
\pgfpathlineto{\pgfqpoint{5.090902in}{1.511976in}}%
\pgfpathlineto{\pgfqpoint{5.091312in}{1.514356in}}%
\pgfpathlineto{\pgfqpoint{5.091722in}{1.512297in}}%
\pgfpathlineto{\pgfqpoint{5.092952in}{1.471848in}}%
\pgfpathlineto{\pgfqpoint{5.094182in}{1.389158in}}%
\pgfpathlineto{\pgfqpoint{5.096232in}{1.692160in}}%
\pgfpathlineto{\pgfqpoint{5.096642in}{1.683771in}}%
\pgfpathlineto{\pgfqpoint{5.097872in}{1.471378in}}%
\pgfpathlineto{\pgfqpoint{5.098282in}{1.562117in}}%
\pgfpathlineto{\pgfqpoint{5.100332in}{1.649472in}}%
\pgfpathlineto{\pgfqpoint{5.101152in}{1.597468in}}%
\pgfpathlineto{\pgfqpoint{5.102792in}{1.348053in}}%
\pgfpathlineto{\pgfqpoint{5.103612in}{1.400374in}}%
\pgfpathlineto{\pgfqpoint{5.106482in}{1.506399in}}%
\pgfpathlineto{\pgfqpoint{5.106892in}{1.506616in}}%
\pgfpathlineto{\pgfqpoint{5.107712in}{1.490824in}}%
\pgfpathlineto{\pgfqpoint{5.109352in}{1.376571in}}%
\pgfpathlineto{\pgfqpoint{5.109762in}{1.429103in}}%
\pgfpathlineto{\pgfqpoint{5.111402in}{1.673915in}}%
\pgfpathlineto{\pgfqpoint{5.111812in}{1.664248in}}%
\pgfpathlineto{\pgfqpoint{5.113042in}{1.450785in}}%
\pgfpathlineto{\pgfqpoint{5.113452in}{1.537038in}}%
\pgfpathlineto{\pgfqpoint{5.115502in}{1.618374in}}%
\pgfpathlineto{\pgfqpoint{5.116322in}{1.546685in}}%
\pgfpathlineto{\pgfqpoint{5.117552in}{1.335409in}}%
\pgfpathlineto{\pgfqpoint{5.117962in}{1.363446in}}%
\pgfpathlineto{\pgfqpoint{5.120832in}{1.495315in}}%
\pgfpathlineto{\pgfqpoint{5.121242in}{1.499885in}}%
\pgfpathlineto{\pgfqpoint{5.121652in}{1.499523in}}%
\pgfpathlineto{\pgfqpoint{5.122472in}{1.480904in}}%
\pgfpathlineto{\pgfqpoint{5.123702in}{1.385779in}}%
\pgfpathlineto{\pgfqpoint{5.124111in}{1.393031in}}%
\pgfpathlineto{\pgfqpoint{5.126161in}{1.659165in}}%
\pgfpathlineto{\pgfqpoint{5.126571in}{1.610934in}}%
\pgfpathlineto{\pgfqpoint{5.127391in}{1.384705in}}%
\pgfpathlineto{\pgfqpoint{5.127801in}{1.486971in}}%
\pgfpathlineto{\pgfqpoint{5.129441in}{1.584145in}}%
\pgfpathlineto{\pgfqpoint{5.129851in}{1.592376in}}%
\pgfpathlineto{\pgfqpoint{5.130671in}{1.522371in}}%
\pgfpathlineto{\pgfqpoint{5.131901in}{1.337674in}}%
\pgfpathlineto{\pgfqpoint{5.132311in}{1.366162in}}%
\pgfpathlineto{\pgfqpoint{5.135181in}{1.492177in}}%
\pgfpathlineto{\pgfqpoint{5.135591in}{1.493996in}}%
\pgfpathlineto{\pgfqpoint{5.136411in}{1.479485in}}%
\pgfpathlineto{\pgfqpoint{5.138051in}{1.382678in}}%
\pgfpathlineto{\pgfqpoint{5.140101in}{1.643315in}}%
\pgfpathlineto{\pgfqpoint{5.140511in}{1.591990in}}%
\pgfpathlineto{\pgfqpoint{5.141331in}{1.364546in}}%
\pgfpathlineto{\pgfqpoint{5.141741in}{1.469218in}}%
\pgfpathlineto{\pgfqpoint{5.143381in}{1.564973in}}%
\pgfpathlineto{\pgfqpoint{5.143791in}{1.563019in}}%
\pgfpathlineto{\pgfqpoint{5.145021in}{1.391187in}}%
\pgfpathlineto{\pgfqpoint{5.145431in}{1.322408in}}%
\pgfpathlineto{\pgfqpoint{5.145841in}{1.352295in}}%
\pgfpathlineto{\pgfqpoint{5.148711in}{1.485997in}}%
\pgfpathlineto{\pgfqpoint{5.149121in}{1.488102in}}%
\pgfpathlineto{\pgfqpoint{5.149941in}{1.472882in}}%
\pgfpathlineto{\pgfqpoint{5.151171in}{1.377104in}}%
\pgfpathlineto{\pgfqpoint{5.151581in}{1.391165in}}%
\pgfpathlineto{\pgfqpoint{5.153221in}{1.628327in}}%
\pgfpathlineto{\pgfqpoint{5.153631in}{1.617428in}}%
\pgfpathlineto{\pgfqpoint{5.154861in}{1.396477in}}%
\pgfpathlineto{\pgfqpoint{5.155271in}{1.471395in}}%
\pgfpathlineto{\pgfqpoint{5.156911in}{1.543929in}}%
\pgfpathlineto{\pgfqpoint{5.157731in}{1.467979in}}%
\pgfpathlineto{\pgfqpoint{5.158551in}{1.316153in}}%
\pgfpathlineto{\pgfqpoint{5.158961in}{1.346995in}}%
\pgfpathlineto{\pgfqpoint{5.161831in}{1.481409in}}%
\pgfpathlineto{\pgfqpoint{5.162241in}{1.482104in}}%
\pgfpathlineto{\pgfqpoint{5.163061in}{1.461754in}}%
\pgfpathlineto{\pgfqpoint{5.164291in}{1.361146in}}%
\pgfpathlineto{\pgfqpoint{5.164701in}{1.415973in}}%
\pgfpathlineto{\pgfqpoint{5.166341in}{1.616365in}}%
\pgfpathlineto{\pgfqpoint{5.167571in}{1.337245in}}%
\pgfpathlineto{\pgfqpoint{5.168391in}{1.468425in}}%
\pgfpathlineto{\pgfqpoint{5.168801in}{1.461420in}}%
\pgfpathlineto{\pgfqpoint{5.169621in}{1.522491in}}%
\pgfpathlineto{\pgfqpoint{5.170031in}{1.501618in}}%
\pgfpathlineto{\pgfqpoint{5.171261in}{1.316977in}}%
\pgfpathlineto{\pgfqpoint{5.172081in}{1.377621in}}%
\pgfpathlineto{\pgfqpoint{5.174541in}{1.477223in}}%
\pgfpathlineto{\pgfqpoint{5.174951in}{1.474866in}}%
\pgfpathlineto{\pgfqpoint{5.175771in}{1.444845in}}%
\pgfpathlineto{\pgfqpoint{5.176591in}{1.365599in}}%
\pgfpathlineto{\pgfqpoint{5.177001in}{1.393610in}}%
\pgfpathlineto{\pgfqpoint{5.178641in}{1.607279in}}%
\pgfpathlineto{\pgfqpoint{5.179051in}{1.565446in}}%
\pgfpathlineto{\pgfqpoint{5.179871in}{1.343994in}}%
\pgfpathlineto{\pgfqpoint{5.180281in}{1.405028in}}%
\pgfpathlineto{\pgfqpoint{5.181921in}{1.501562in}}%
\pgfpathlineto{\pgfqpoint{5.182741in}{1.422070in}}%
\pgfpathlineto{\pgfqpoint{5.183561in}{1.322718in}}%
\pgfpathlineto{\pgfqpoint{5.183971in}{1.354301in}}%
\pgfpathlineto{\pgfqpoint{5.186431in}{1.471036in}}%
\pgfpathlineto{\pgfqpoint{5.186841in}{1.471757in}}%
\pgfpathlineto{\pgfqpoint{5.187661in}{1.448218in}}%
\pgfpathlineto{\pgfqpoint{5.188481in}{1.376701in}}%
\pgfpathlineto{\pgfqpoint{5.188891in}{1.377874in}}%
\pgfpathlineto{\pgfqpoint{5.190531in}{1.596897in}}%
\pgfpathlineto{\pgfqpoint{5.190941in}{1.561012in}}%
\pgfpathlineto{\pgfqpoint{5.191761in}{1.342924in}}%
\pgfpathlineto{\pgfqpoint{5.192581in}{1.425500in}}%
\pgfpathlineto{\pgfqpoint{5.192991in}{1.420225in}}%
\pgfpathlineto{\pgfqpoint{5.193811in}{1.480955in}}%
\pgfpathlineto{\pgfqpoint{5.194221in}{1.451505in}}%
\pgfpathlineto{\pgfqpoint{5.195041in}{1.298227in}}%
\pgfpathlineto{\pgfqpoint{5.195861in}{1.362641in}}%
\pgfpathlineto{\pgfqpoint{5.198321in}{1.467591in}}%
\pgfpathlineto{\pgfqpoint{5.199141in}{1.449947in}}%
\pgfpathlineto{\pgfqpoint{5.200371in}{1.364581in}}%
\pgfpathlineto{\pgfqpoint{5.202011in}{1.586860in}}%
\pgfpathlineto{\pgfqpoint{5.202421in}{1.554838in}}%
\pgfpathlineto{\pgfqpoint{5.203241in}{1.338627in}}%
\pgfpathlineto{\pgfqpoint{5.204061in}{1.406850in}}%
\pgfpathlineto{\pgfqpoint{5.204471in}{1.401040in}}%
\pgfpathlineto{\pgfqpoint{5.205291in}{1.460265in}}%
\pgfpathlineto{\pgfqpoint{5.205701in}{1.423666in}}%
\pgfpathlineto{\pgfqpoint{5.206521in}{1.306313in}}%
\pgfpathlineto{\pgfqpoint{5.206931in}{1.340183in}}%
\pgfpathlineto{\pgfqpoint{5.209391in}{1.462471in}}%
\pgfpathlineto{\pgfqpoint{5.209801in}{1.461480in}}%
\pgfpathlineto{\pgfqpoint{5.210621in}{1.429751in}}%
\pgfpathlineto{\pgfqpoint{5.211441in}{1.350381in}}%
\pgfpathlineto{\pgfqpoint{5.211851in}{1.408305in}}%
\pgfpathlineto{\pgfqpoint{5.213081in}{1.577006in}}%
\pgfpathlineto{\pgfqpoint{5.213491in}{1.550988in}}%
\pgfpathlineto{\pgfqpoint{5.214311in}{1.338140in}}%
\pgfpathlineto{\pgfqpoint{5.215131in}{1.388781in}}%
\pgfpathlineto{\pgfqpoint{5.215541in}{1.389580in}}%
\pgfpathlineto{\pgfqpoint{5.215951in}{1.440864in}}%
\pgfpathlineto{\pgfqpoint{5.216361in}{1.440266in}}%
\pgfpathlineto{\pgfqpoint{5.217591in}{1.313993in}}%
\pgfpathlineto{\pgfqpoint{5.218001in}{1.347844in}}%
\pgfpathlineto{\pgfqpoint{5.220461in}{1.458665in}}%
\pgfpathlineto{\pgfqpoint{5.221281in}{1.435540in}}%
\pgfpathlineto{\pgfqpoint{5.222101in}{1.355946in}}%
\pgfpathlineto{\pgfqpoint{5.222511in}{1.388260in}}%
\pgfpathlineto{\pgfqpoint{5.223741in}{1.565690in}}%
\pgfpathlineto{\pgfqpoint{5.224151in}{1.551827in}}%
\pgfpathlineto{\pgfqpoint{5.225381in}{1.321542in}}%
\pgfpathlineto{\pgfqpoint{5.225791in}{1.370313in}}%
\pgfpathlineto{\pgfqpoint{5.226201in}{1.371737in}}%
\pgfpathlineto{\pgfqpoint{5.226611in}{1.424485in}}%
\pgfpathlineto{\pgfqpoint{5.227021in}{1.422829in}}%
\pgfpathlineto{\pgfqpoint{5.227841in}{1.295475in}}%
\pgfpathlineto{\pgfqpoint{5.228661in}{1.352413in}}%
\pgfpathlineto{\pgfqpoint{5.230711in}{1.453821in}}%
\pgfpathlineto{\pgfqpoint{5.231121in}{1.453076in}}%
\pgfpathlineto{\pgfqpoint{5.231941in}{1.418065in}}%
\pgfpathlineto{\pgfqpoint{5.232761in}{1.361460in}}%
\pgfpathlineto{\pgfqpoint{5.234401in}{1.555773in}}%
\pgfpathlineto{\pgfqpoint{5.235631in}{1.283224in}}%
\pgfpathlineto{\pgfqpoint{5.236451in}{1.352497in}}%
\pgfpathlineto{\pgfqpoint{5.237271in}{1.409492in}}%
\pgfpathlineto{\pgfqpoint{5.238091in}{1.285272in}}%
\pgfpathlineto{\pgfqpoint{5.238911in}{1.352290in}}%
\pgfpathlineto{\pgfqpoint{5.240961in}{1.450557in}}%
\pgfpathlineto{\pgfqpoint{5.241371in}{1.447132in}}%
\pgfpathlineto{\pgfqpoint{5.242191in}{1.402478in}}%
\pgfpathlineto{\pgfqpoint{5.242601in}{1.353826in}}%
\pgfpathlineto{\pgfqpoint{5.243011in}{1.384096in}}%
\pgfpathlineto{\pgfqpoint{5.244241in}{1.553206in}}%
\pgfpathlineto{\pgfqpoint{5.244651in}{1.522073in}}%
\pgfpathlineto{\pgfqpoint{5.245471in}{1.301675in}}%
\pgfpathlineto{\pgfqpoint{5.246291in}{1.341300in}}%
\pgfpathlineto{\pgfqpoint{5.247111in}{1.398572in}}%
\pgfpathlineto{\pgfqpoint{5.247521in}{1.367478in}}%
\pgfpathlineto{\pgfqpoint{5.247931in}{1.293718in}}%
\pgfpathlineto{\pgfqpoint{5.248751in}{1.345746in}}%
\pgfpathlineto{\pgfqpoint{5.250801in}{1.446878in}}%
\pgfpathlineto{\pgfqpoint{5.251211in}{1.443061in}}%
\pgfpathlineto{\pgfqpoint{5.252031in}{1.395962in}}%
\pgfpathlineto{\pgfqpoint{5.252441in}{1.344993in}}%
\pgfpathlineto{\pgfqpoint{5.252851in}{1.389355in}}%
\pgfpathlineto{\pgfqpoint{5.254081in}{1.545769in}}%
\pgfpathlineto{\pgfqpoint{5.254491in}{1.502636in}}%
\pgfpathlineto{\pgfqpoint{5.255311in}{1.280378in}}%
\pgfpathlineto{\pgfqpoint{5.256131in}{1.324830in}}%
\pgfpathlineto{\pgfqpoint{5.256951in}{1.380622in}}%
\pgfpathlineto{\pgfqpoint{5.257361in}{1.340683in}}%
\pgfpathlineto{\pgfqpoint{5.257771in}{1.279218in}}%
\pgfpathlineto{\pgfqpoint{5.258181in}{1.317289in}}%
\pgfpathlineto{\pgfqpoint{5.260641in}{1.443587in}}%
\pgfpathlineto{\pgfqpoint{5.261461in}{1.417353in}}%
\pgfpathlineto{\pgfqpoint{5.262281in}{1.344211in}}%
\pgfpathlineto{\pgfqpoint{5.262691in}{1.404056in}}%
\pgfpathlineto{\pgfqpoint{5.263921in}{1.534377in}}%
\pgfpathlineto{\pgfqpoint{5.265151in}{1.252742in}}%
\pgfpathlineto{\pgfqpoint{5.265971in}{1.307400in}}%
\pgfpathlineto{\pgfqpoint{5.266381in}{1.363296in}}%
\pgfpathlineto{\pgfqpoint{5.266791in}{1.360108in}}%
\pgfpathlineto{\pgfqpoint{5.267611in}{1.289255in}}%
\pgfpathlineto{\pgfqpoint{5.268021in}{1.326100in}}%
\pgfpathlineto{\pgfqpoint{5.270481in}{1.439522in}}%
\pgfpathlineto{\pgfqpoint{5.271301in}{1.405167in}}%
\pgfpathlineto{\pgfqpoint{5.272121in}{1.360458in}}%
\pgfpathlineto{\pgfqpoint{5.273351in}{1.527658in}}%
\pgfpathlineto{\pgfqpoint{5.273761in}{1.516307in}}%
\pgfpathlineto{\pgfqpoint{5.274991in}{1.253585in}}%
\pgfpathlineto{\pgfqpoint{5.275811in}{1.317456in}}%
\pgfpathlineto{\pgfqpoint{5.276221in}{1.353823in}}%
\pgfpathlineto{\pgfqpoint{5.276631in}{1.335520in}}%
\pgfpathlineto{\pgfqpoint{5.277041in}{1.273306in}}%
\pgfpathlineto{\pgfqpoint{5.277451in}{1.300676in}}%
\pgfpathlineto{\pgfqpoint{5.279911in}{1.437306in}}%
\pgfpathlineto{\pgfqpoint{5.280321in}{1.434167in}}%
\pgfpathlineto{\pgfqpoint{5.281141in}{1.389904in}}%
\pgfpathlineto{\pgfqpoint{5.281551in}{1.341891in}}%
\pgfpathlineto{\pgfqpoint{5.281961in}{1.379702in}}%
\pgfpathlineto{\pgfqpoint{5.283191in}{1.524807in}}%
\pgfpathlineto{\pgfqpoint{5.283601in}{1.489548in}}%
\pgfpathlineto{\pgfqpoint{5.284831in}{1.263247in}}%
\pgfpathlineto{\pgfqpoint{5.285241in}{1.285486in}}%
\pgfpathlineto{\pgfqpoint{5.286061in}{1.339666in}}%
\pgfpathlineto{\pgfqpoint{5.286471in}{1.305427in}}%
\pgfpathlineto{\pgfqpoint{5.286881in}{1.275813in}}%
\pgfpathlineto{\pgfqpoint{5.287291in}{1.313165in}}%
\pgfpathlineto{\pgfqpoint{5.289751in}{1.434437in}}%
\pgfpathlineto{\pgfqpoint{5.290571in}{1.406779in}}%
\pgfpathlineto{\pgfqpoint{5.291391in}{1.344514in}}%
\pgfpathlineto{\pgfqpoint{5.293031in}{1.512752in}}%
\pgfpathlineto{\pgfqpoint{5.294261in}{1.248404in}}%
\pgfpathlineto{\pgfqpoint{5.295081in}{1.271695in}}%
\pgfpathlineto{\pgfqpoint{5.295491in}{1.323814in}}%
\pgfpathlineto{\pgfqpoint{5.295901in}{1.319066in}}%
\pgfpathlineto{\pgfqpoint{5.296311in}{1.268603in}}%
\pgfpathlineto{\pgfqpoint{5.296721in}{1.290743in}}%
\pgfpathlineto{\pgfqpoint{5.299181in}{1.431287in}}%
\pgfpathlineto{\pgfqpoint{5.299591in}{1.429786in}}%
\pgfpathlineto{\pgfqpoint{5.300411in}{1.390628in}}%
\pgfpathlineto{\pgfqpoint{5.300821in}{1.346746in}}%
\pgfpathlineto{\pgfqpoint{5.301231in}{1.367208in}}%
\pgfpathlineto{\pgfqpoint{5.302461in}{1.512321in}}%
\pgfpathlineto{\pgfqpoint{5.302871in}{1.488807in}}%
\pgfpathlineto{\pgfqpoint{5.304101in}{1.228288in}}%
\pgfpathlineto{\pgfqpoint{5.304921in}{1.289775in}}%
\pgfpathlineto{\pgfqpoint{5.305331in}{1.315831in}}%
\pgfpathlineto{\pgfqpoint{5.305741in}{1.290583in}}%
\pgfpathlineto{\pgfqpoint{5.306151in}{1.268942in}}%
\pgfpathlineto{\pgfqpoint{5.309021in}{1.429208in}}%
\pgfpathlineto{\pgfqpoint{5.309841in}{1.403922in}}%
\pgfpathlineto{\pgfqpoint{5.310661in}{1.337894in}}%
\pgfpathlineto{\pgfqpoint{5.311071in}{1.392595in}}%
\pgfpathlineto{\pgfqpoint{5.312301in}{1.503498in}}%
\pgfpathlineto{\pgfqpoint{5.313941in}{1.240112in}}%
\pgfpathlineto{\pgfqpoint{5.314761in}{1.297368in}}%
\pgfpathlineto{\pgfqpoint{5.315171in}{1.298244in}}%
\pgfpathlineto{\pgfqpoint{5.315581in}{1.253135in}}%
\pgfpathlineto{\pgfqpoint{5.315991in}{1.286732in}}%
\pgfpathlineto{\pgfqpoint{5.318451in}{1.426133in}}%
\pgfpathlineto{\pgfqpoint{5.318861in}{1.424737in}}%
\pgfpathlineto{\pgfqpoint{5.319681in}{1.386406in}}%
\pgfpathlineto{\pgfqpoint{5.320091in}{1.343576in}}%
\pgfpathlineto{\pgfqpoint{5.320501in}{1.363573in}}%
\pgfpathlineto{\pgfqpoint{5.321731in}{1.501596in}}%
\pgfpathlineto{\pgfqpoint{5.322141in}{1.479732in}}%
\pgfpathlineto{\pgfqpoint{5.323371in}{1.213693in}}%
\pgfpathlineto{\pgfqpoint{5.324191in}{1.265337in}}%
\pgfpathlineto{\pgfqpoint{5.324601in}{1.293036in}}%
\pgfpathlineto{\pgfqpoint{5.325011in}{1.269958in}}%
\pgfpathlineto{\pgfqpoint{5.325421in}{1.267836in}}%
\pgfpathlineto{\pgfqpoint{5.327881in}{1.421259in}}%
\pgfpathlineto{\pgfqpoint{5.328291in}{1.424203in}}%
\pgfpathlineto{\pgfqpoint{5.329111in}{1.397849in}}%
\pgfpathlineto{\pgfqpoint{5.329931in}{1.338543in}}%
\pgfpathlineto{\pgfqpoint{5.331571in}{1.492129in}}%
\pgfpathlineto{\pgfqpoint{5.333211in}{1.220199in}}%
\pgfpathlineto{\pgfqpoint{5.334031in}{1.276414in}}%
\pgfpathlineto{\pgfqpoint{5.334441in}{1.275810in}}%
\pgfpathlineto{\pgfqpoint{5.334851in}{1.249622in}}%
\pgfpathlineto{\pgfqpoint{5.335261in}{1.287833in}}%
\pgfpathlineto{\pgfqpoint{5.337721in}{1.421762in}}%
\pgfpathlineto{\pgfqpoint{5.338131in}{1.419176in}}%
\pgfpathlineto{\pgfqpoint{5.338951in}{1.378136in}}%
\pgfpathlineto{\pgfqpoint{5.339361in}{1.333911in}}%
\pgfpathlineto{\pgfqpoint{5.339771in}{1.366990in}}%
\pgfpathlineto{\pgfqpoint{5.341001in}{1.492364in}}%
\pgfpathlineto{\pgfqpoint{5.341411in}{1.464882in}}%
\pgfpathlineto{\pgfqpoint{5.342641in}{1.206851in}}%
\pgfpathlineto{\pgfqpoint{5.343461in}{1.249828in}}%
\pgfpathlineto{\pgfqpoint{5.344281in}{1.245159in}}%
\pgfpathlineto{\pgfqpoint{5.347561in}{1.419188in}}%
\pgfpathlineto{\pgfqpoint{5.348381in}{1.388763in}}%
\pgfpathlineto{\pgfqpoint{5.349201in}{1.345255in}}%
\pgfpathlineto{\pgfqpoint{5.350431in}{1.484220in}}%
\pgfpathlineto{\pgfqpoint{5.350841in}{1.478114in}}%
\pgfpathlineto{\pgfqpoint{5.352481in}{1.205378in}}%
\pgfpathlineto{\pgfqpoint{5.353301in}{1.259585in}}%
\pgfpathlineto{\pgfqpoint{5.353711in}{1.251640in}}%
\pgfpathlineto{\pgfqpoint{5.354121in}{1.256129in}}%
\pgfpathlineto{\pgfqpoint{5.356991in}{1.417676in}}%
\pgfpathlineto{\pgfqpoint{5.357811in}{1.396273in}}%
\pgfpathlineto{\pgfqpoint{5.358631in}{1.326294in}}%
\pgfpathlineto{\pgfqpoint{5.359041in}{1.376259in}}%
\pgfpathlineto{\pgfqpoint{5.360271in}{1.482131in}}%
\pgfpathlineto{\pgfqpoint{5.360681in}{1.443825in}}%
\pgfpathlineto{\pgfqpoint{5.361911in}{1.192078in}}%
\pgfpathlineto{\pgfqpoint{5.362731in}{1.239953in}}%
\pgfpathlineto{\pgfqpoint{5.363961in}{1.279492in}}%
\pgfpathlineto{\pgfqpoint{5.366421in}{1.415040in}}%
\pgfpathlineto{\pgfqpoint{5.366831in}{1.413541in}}%
\pgfpathlineto{\pgfqpoint{5.367651in}{1.376202in}}%
\pgfpathlineto{\pgfqpoint{5.368061in}{1.335130in}}%
\pgfpathlineto{\pgfqpoint{5.368471in}{1.357239in}}%
\pgfpathlineto{\pgfqpoint{5.369701in}{1.479311in}}%
\pgfpathlineto{\pgfqpoint{5.370111in}{1.459052in}}%
\pgfpathlineto{\pgfqpoint{5.371751in}{1.192278in}}%
\pgfpathlineto{\pgfqpoint{5.372161in}{1.214222in}}%
\pgfpathlineto{\pgfqpoint{5.375851in}{1.411593in}}%
\pgfpathlineto{\pgfqpoint{5.376261in}{1.413092in}}%
\pgfpathlineto{\pgfqpoint{5.377081in}{1.383990in}}%
\pgfpathlineto{\pgfqpoint{5.377901in}{1.340436in}}%
\pgfpathlineto{\pgfqpoint{5.379131in}{1.471915in}}%
\pgfpathlineto{\pgfqpoint{5.379541in}{1.467277in}}%
\pgfpathlineto{\pgfqpoint{5.381181in}{1.177579in}}%
\pgfpathlineto{\pgfqpoint{5.382411in}{1.226565in}}%
\pgfpathlineto{\pgfqpoint{5.385691in}{1.411686in}}%
\pgfpathlineto{\pgfqpoint{5.386101in}{1.406354in}}%
\pgfpathlineto{\pgfqpoint{5.387331in}{1.325623in}}%
\pgfpathlineto{\pgfqpoint{5.387741in}{1.373787in}}%
\pgfpathlineto{\pgfqpoint{5.388971in}{1.469886in}}%
\pgfpathlineto{\pgfqpoint{5.390201in}{1.270232in}}%
\pgfpathlineto{\pgfqpoint{5.391021in}{1.176810in}}%
\pgfpathlineto{\pgfqpoint{5.391431in}{1.214161in}}%
\pgfpathlineto{\pgfqpoint{5.392251in}{1.242335in}}%
\pgfpathlineto{\pgfqpoint{5.395121in}{1.409574in}}%
\pgfpathlineto{\pgfqpoint{5.395531in}{1.407000in}}%
\pgfpathlineto{\pgfqpoint{5.396351in}{1.367502in}}%
\pgfpathlineto{\pgfqpoint{5.396761in}{1.325599in}}%
\pgfpathlineto{\pgfqpoint{5.397171in}{1.358952in}}%
\pgfpathlineto{\pgfqpoint{5.398401in}{1.468316in}}%
\pgfpathlineto{\pgfqpoint{5.398811in}{1.443984in}}%
\pgfpathlineto{\pgfqpoint{5.400451in}{1.168883in}}%
\pgfpathlineto{\pgfqpoint{5.401271in}{1.218354in}}%
\pgfpathlineto{\pgfqpoint{5.402091in}{1.269431in}}%
\pgfpathlineto{\pgfqpoint{5.404551in}{1.406958in}}%
\pgfpathlineto{\pgfqpoint{5.404961in}{1.406787in}}%
\pgfpathlineto{\pgfqpoint{5.405781in}{1.373900in}}%
\pgfpathlineto{\pgfqpoint{5.406191in}{1.336533in}}%
\pgfpathlineto{\pgfqpoint{5.406601in}{1.345734in}}%
\pgfpathlineto{\pgfqpoint{5.407831in}{1.463855in}}%
\pgfpathlineto{\pgfqpoint{5.408241in}{1.451367in}}%
\pgfpathlineto{\pgfqpoint{5.409881in}{1.165704in}}%
\pgfpathlineto{\pgfqpoint{5.410701in}{1.208959in}}%
\pgfpathlineto{\pgfqpoint{5.411521in}{1.259820in}}%
\pgfpathlineto{\pgfqpoint{5.413981in}{1.403998in}}%
\pgfpathlineto{\pgfqpoint{5.414391in}{1.405925in}}%
\pgfpathlineto{\pgfqpoint{5.415211in}{1.378784in}}%
\pgfpathlineto{\pgfqpoint{5.416031in}{1.333995in}}%
\pgfpathlineto{\pgfqpoint{5.417261in}{1.457552in}}%
\pgfpathlineto{\pgfqpoint{5.417671in}{1.455131in}}%
\pgfpathlineto{\pgfqpoint{5.419721in}{1.152527in}}%
\pgfpathlineto{\pgfqpoint{5.420541in}{1.211376in}}%
\pgfpathlineto{\pgfqpoint{5.423411in}{1.400821in}}%
\pgfpathlineto{\pgfqpoint{5.423821in}{1.404581in}}%
\pgfpathlineto{\pgfqpoint{5.424231in}{1.399061in}}%
\pgfpathlineto{\pgfqpoint{5.425461in}{1.323588in}}%
\pgfpathlineto{\pgfqpoint{5.425871in}{1.369471in}}%
\pgfpathlineto{\pgfqpoint{5.427101in}{1.456096in}}%
\pgfpathlineto{\pgfqpoint{5.428331in}{1.269108in}}%
\pgfpathlineto{\pgfqpoint{5.429151in}{1.147659in}}%
\pgfpathlineto{\pgfqpoint{5.429561in}{1.182925in}}%
\pgfpathlineto{\pgfqpoint{5.431201in}{1.311747in}}%
\pgfpathlineto{\pgfqpoint{5.433251in}{1.402891in}}%
\pgfpathlineto{\pgfqpoint{5.433661in}{1.399325in}}%
\pgfpathlineto{\pgfqpoint{5.434481in}{1.358124in}}%
\pgfpathlineto{\pgfqpoint{5.434891in}{1.315889in}}%
\pgfpathlineto{\pgfqpoint{5.435301in}{1.359099in}}%
\pgfpathlineto{\pgfqpoint{5.436531in}{1.454998in}}%
\pgfpathlineto{\pgfqpoint{5.436941in}{1.427947in}}%
\pgfpathlineto{\pgfqpoint{5.438581in}{1.150264in}}%
\pgfpathlineto{\pgfqpoint{5.439401in}{1.194236in}}%
\pgfpathlineto{\pgfqpoint{5.442271in}{1.394206in}}%
\pgfpathlineto{\pgfqpoint{5.442681in}{1.400963in}}%
\pgfpathlineto{\pgfqpoint{5.443091in}{1.399102in}}%
\pgfpathlineto{\pgfqpoint{5.443911in}{1.362608in}}%
\pgfpathlineto{\pgfqpoint{5.444321in}{1.323573in}}%
\pgfpathlineto{\pgfqpoint{5.444731in}{1.349846in}}%
\pgfpathlineto{\pgfqpoint{5.445961in}{1.452458in}}%
\pgfpathlineto{\pgfqpoint{5.446371in}{1.433231in}}%
\pgfpathlineto{\pgfqpoint{5.448421in}{1.151072in}}%
\pgfpathlineto{\pgfqpoint{5.448831in}{1.186714in}}%
\pgfpathlineto{\pgfqpoint{5.451701in}{1.390914in}}%
\pgfpathlineto{\pgfqpoint{5.452111in}{1.398886in}}%
\pgfpathlineto{\pgfqpoint{5.452521in}{1.398509in}}%
\pgfpathlineto{\pgfqpoint{5.453341in}{1.366112in}}%
\pgfpathlineto{\pgfqpoint{5.453751in}{1.329876in}}%
\pgfpathlineto{\pgfqpoint{5.454161in}{1.341628in}}%
\pgfpathlineto{\pgfqpoint{5.455391in}{1.448976in}}%
\pgfpathlineto{\pgfqpoint{5.455801in}{1.436528in}}%
\pgfpathlineto{\pgfqpoint{5.457851in}{1.134695in}}%
\pgfpathlineto{\pgfqpoint{5.458261in}{1.179885in}}%
\pgfpathlineto{\pgfqpoint{5.461131in}{1.387705in}}%
\pgfpathlineto{\pgfqpoint{5.461951in}{1.397641in}}%
\pgfpathlineto{\pgfqpoint{5.462771in}{1.368804in}}%
\pgfpathlineto{\pgfqpoint{5.463591in}{1.334361in}}%
\pgfpathlineto{\pgfqpoint{5.464821in}{1.444937in}}%
\pgfpathlineto{\pgfqpoint{5.465231in}{1.438253in}}%
\pgfpathlineto{\pgfqpoint{5.467281in}{1.130508in}}%
\pgfpathlineto{\pgfqpoint{5.468101in}{1.215421in}}%
\pgfpathlineto{\pgfqpoint{5.470971in}{1.394547in}}%
\pgfpathlineto{\pgfqpoint{5.471381in}{1.396574in}}%
\pgfpathlineto{\pgfqpoint{5.472201in}{1.370821in}}%
\pgfpathlineto{\pgfqpoint{5.473021in}{1.327966in}}%
\pgfpathlineto{\pgfqpoint{5.474251in}{1.440633in}}%
\pgfpathlineto{\pgfqpoint{5.474661in}{1.438772in}}%
\pgfpathlineto{\pgfqpoint{5.476711in}{1.137465in}}%
\pgfpathlineto{\pgfqpoint{5.477531in}{1.210180in}}%
\pgfpathlineto{\pgfqpoint{5.480401in}{1.392387in}}%
\pgfpathlineto{\pgfqpoint{5.480811in}{1.395371in}}%
\pgfpathlineto{\pgfqpoint{5.481631in}{1.372275in}}%
\pgfpathlineto{\pgfqpoint{5.482451in}{1.322369in}}%
\pgfpathlineto{\pgfqpoint{5.484091in}{1.438391in}}%
\pgfpathlineto{\pgfqpoint{5.485321in}{1.264725in}}%
\pgfpathlineto{\pgfqpoint{5.486141in}{1.143914in}}%
\pgfpathlineto{\pgfqpoint{5.486551in}{1.163453in}}%
\pgfpathlineto{\pgfqpoint{5.489831in}{1.390282in}}%
\pgfpathlineto{\pgfqpoint{5.490241in}{1.394080in}}%
\pgfpathlineto{\pgfqpoint{5.490651in}{1.388957in}}%
\pgfpathlineto{\pgfqpoint{5.491881in}{1.317504in}}%
\pgfpathlineto{\pgfqpoint{5.492291in}{1.359845in}}%
\pgfpathlineto{\pgfqpoint{5.493521in}{1.437363in}}%
\pgfpathlineto{\pgfqpoint{5.494751in}{1.272228in}}%
\pgfpathlineto{\pgfqpoint{5.495571in}{1.149807in}}%
\pgfpathlineto{\pgfqpoint{5.495981in}{1.159279in}}%
\pgfpathlineto{\pgfqpoint{5.499261in}{1.388260in}}%
\pgfpathlineto{\pgfqpoint{5.499671in}{1.392741in}}%
\pgfpathlineto{\pgfqpoint{5.500081in}{1.388473in}}%
\pgfpathlineto{\pgfqpoint{5.501311in}{1.313314in}}%
\pgfpathlineto{\pgfqpoint{5.501721in}{1.355055in}}%
\pgfpathlineto{\pgfqpoint{5.502951in}{1.435888in}}%
\pgfpathlineto{\pgfqpoint{5.503361in}{1.408839in}}%
\pgfpathlineto{\pgfqpoint{5.505001in}{1.155112in}}%
\pgfpathlineto{\pgfqpoint{5.505821in}{1.197941in}}%
\pgfpathlineto{\pgfqpoint{5.508691in}{1.386341in}}%
\pgfpathlineto{\pgfqpoint{5.509101in}{1.391386in}}%
\pgfpathlineto{\pgfqpoint{5.509511in}{1.387828in}}%
\pgfpathlineto{\pgfqpoint{5.510741in}{1.309746in}}%
\pgfpathlineto{\pgfqpoint{5.511151in}{1.350927in}}%
\pgfpathlineto{\pgfqpoint{5.512381in}{1.434123in}}%
\pgfpathlineto{\pgfqpoint{5.512791in}{1.410138in}}%
\pgfpathlineto{\pgfqpoint{5.514841in}{1.152772in}}%
\pgfpathlineto{\pgfqpoint{5.515251in}{1.194971in}}%
\pgfpathlineto{\pgfqpoint{5.518121in}{1.384539in}}%
\pgfpathlineto{\pgfqpoint{5.518531in}{1.390038in}}%
\pgfpathlineto{\pgfqpoint{5.518941in}{1.387060in}}%
\pgfpathlineto{\pgfqpoint{5.519761in}{1.349252in}}%
\pgfpathlineto{\pgfqpoint{5.520171in}{1.310659in}}%
\pgfpathlineto{\pgfqpoint{5.520581in}{1.347416in}}%
\pgfpathlineto{\pgfqpoint{5.521811in}{1.432187in}}%
\pgfpathlineto{\pgfqpoint{5.522221in}{1.410724in}}%
\pgfpathlineto{\pgfqpoint{5.524271in}{1.150393in}}%
\pgfpathlineto{\pgfqpoint{5.524681in}{1.192527in}}%
\pgfpathlineto{\pgfqpoint{5.527551in}{1.382865in}}%
\pgfpathlineto{\pgfqpoint{5.527961in}{1.388715in}}%
\pgfpathlineto{\pgfqpoint{5.528371in}{1.386194in}}%
\pgfpathlineto{\pgfqpoint{5.529191in}{1.349750in}}%
\pgfpathlineto{\pgfqpoint{5.529601in}{1.312145in}}%
\pgfpathlineto{\pgfqpoint{5.530011in}{1.344482in}}%
\pgfpathlineto{\pgfqpoint{5.531241in}{1.430167in}}%
\pgfpathlineto{\pgfqpoint{5.531651in}{1.410719in}}%
\pgfpathlineto{\pgfqpoint{5.533701in}{1.148566in}}%
\pgfpathlineto{\pgfqpoint{5.534111in}{1.190590in}}%
\pgfpathlineto{\pgfqpoint{5.536981in}{1.381323in}}%
\pgfpathlineto{\pgfqpoint{5.537391in}{1.387430in}}%
\pgfpathlineto{\pgfqpoint{5.537801in}{1.385253in}}%
\pgfpathlineto{\pgfqpoint{5.538621in}{1.349873in}}%
\pgfpathlineto{\pgfqpoint{5.539031in}{1.313065in}}%
\pgfpathlineto{\pgfqpoint{5.539441in}{1.342089in}}%
\pgfpathlineto{\pgfqpoint{5.540671in}{1.428126in}}%
\pgfpathlineto{\pgfqpoint{5.541081in}{1.410223in}}%
\pgfpathlineto{\pgfqpoint{5.543131in}{1.147266in}}%
\pgfpathlineto{\pgfqpoint{5.543541in}{1.189139in}}%
\pgfpathlineto{\pgfqpoint{5.546411in}{1.379916in}}%
\pgfpathlineto{\pgfqpoint{5.546821in}{1.386190in}}%
\pgfpathlineto{\pgfqpoint{5.547231in}{1.384253in}}%
\pgfpathlineto{\pgfqpoint{5.548051in}{1.349658in}}%
\pgfpathlineto{\pgfqpoint{5.548461in}{1.313470in}}%
\pgfpathlineto{\pgfqpoint{5.548871in}{1.340201in}}%
\pgfpathlineto{\pgfqpoint{5.550101in}{1.426104in}}%
\pgfpathlineto{\pgfqpoint{5.550511in}{1.409316in}}%
\pgfpathlineto{\pgfqpoint{5.552561in}{1.146468in}}%
\pgfpathlineto{\pgfqpoint{5.552971in}{1.188156in}}%
\pgfpathlineto{\pgfqpoint{5.555841in}{1.378645in}}%
\pgfpathlineto{\pgfqpoint{5.556251in}{1.385002in}}%
\pgfpathlineto{\pgfqpoint{5.556661in}{1.383203in}}%
\pgfpathlineto{\pgfqpoint{5.557481in}{1.349134in}}%
\pgfpathlineto{\pgfqpoint{5.557891in}{1.313400in}}%
\pgfpathlineto{\pgfqpoint{5.558301in}{1.338788in}}%
\pgfpathlineto{\pgfqpoint{5.559531in}{1.424128in}}%
\pgfpathlineto{\pgfqpoint{5.559941in}{1.408057in}}%
\pgfpathlineto{\pgfqpoint{5.561991in}{1.146147in}}%
\pgfpathlineto{\pgfqpoint{5.562401in}{1.187619in}}%
\pgfpathlineto{\pgfqpoint{5.565271in}{1.377505in}}%
\pgfpathlineto{\pgfqpoint{5.565681in}{1.383866in}}%
\pgfpathlineto{\pgfqpoint{5.566091in}{1.382111in}}%
\pgfpathlineto{\pgfqpoint{5.566911in}{1.348323in}}%
\pgfpathlineto{\pgfqpoint{5.567321in}{1.312889in}}%
\pgfpathlineto{\pgfqpoint{5.567731in}{1.337820in}}%
\pgfpathlineto{\pgfqpoint{5.568961in}{1.422207in}}%
\pgfpathlineto{\pgfqpoint{5.569371in}{1.406488in}}%
\pgfpathlineto{\pgfqpoint{5.571421in}{1.146279in}}%
\pgfpathlineto{\pgfqpoint{5.571831in}{1.187508in}}%
\pgfpathlineto{\pgfqpoint{5.574701in}{1.376493in}}%
\pgfpathlineto{\pgfqpoint{5.575111in}{1.382782in}}%
\pgfpathlineto{\pgfqpoint{5.575521in}{1.380981in}}%
\pgfpathlineto{\pgfqpoint{5.576341in}{1.347241in}}%
\pgfpathlineto{\pgfqpoint{5.576751in}{1.311962in}}%
\pgfpathlineto{\pgfqpoint{5.577161in}{1.337272in}}%
\pgfpathlineto{\pgfqpoint{5.578391in}{1.420342in}}%
\pgfpathlineto{\pgfqpoint{5.578801in}{1.404639in}}%
\pgfpathlineto{\pgfqpoint{5.580851in}{1.146840in}}%
\pgfpathlineto{\pgfqpoint{5.581261in}{1.187803in}}%
\pgfpathlineto{\pgfqpoint{5.584131in}{1.375602in}}%
\pgfpathlineto{\pgfqpoint{5.584541in}{1.381745in}}%
\pgfpathlineto{\pgfqpoint{5.584951in}{1.379813in}}%
\pgfpathlineto{\pgfqpoint{5.585771in}{1.345900in}}%
\pgfpathlineto{\pgfqpoint{5.586181in}{1.310640in}}%
\pgfpathlineto{\pgfqpoint{5.586591in}{1.337118in}}%
\pgfpathlineto{\pgfqpoint{5.587821in}{1.418523in}}%
\pgfpathlineto{\pgfqpoint{5.588231in}{1.402526in}}%
\pgfpathlineto{\pgfqpoint{5.590281in}{1.147806in}}%
\pgfpathlineto{\pgfqpoint{5.590691in}{1.188483in}}%
\pgfpathlineto{\pgfqpoint{5.593561in}{1.374822in}}%
\pgfpathlineto{\pgfqpoint{5.593971in}{1.380750in}}%
\pgfpathlineto{\pgfqpoint{5.594381in}{1.378603in}}%
\pgfpathlineto{\pgfqpoint{5.595201in}{1.344307in}}%
\pgfpathlineto{\pgfqpoint{5.595611in}{1.308938in}}%
\pgfpathlineto{\pgfqpoint{5.596021in}{1.337336in}}%
\pgfpathlineto{\pgfqpoint{5.597251in}{1.416735in}}%
\pgfpathlineto{\pgfqpoint{5.597661in}{1.400156in}}%
\pgfpathlineto{\pgfqpoint{5.599711in}{1.149157in}}%
\pgfpathlineto{\pgfqpoint{5.600121in}{1.189530in}}%
\pgfpathlineto{\pgfqpoint{5.602991in}{1.374145in}}%
\pgfpathlineto{\pgfqpoint{5.603401in}{1.379789in}}%
\pgfpathlineto{\pgfqpoint{5.603811in}{1.377349in}}%
\pgfpathlineto{\pgfqpoint{5.604631in}{1.342466in}}%
\pgfpathlineto{\pgfqpoint{5.605041in}{1.306865in}}%
\pgfpathlineto{\pgfqpoint{5.605451in}{1.337904in}}%
\pgfpathlineto{\pgfqpoint{5.606681in}{1.414954in}}%
\pgfpathlineto{\pgfqpoint{5.607091in}{1.397529in}}%
\pgfpathlineto{\pgfqpoint{5.609141in}{1.150871in}}%
\pgfpathlineto{\pgfqpoint{5.609551in}{1.190923in}}%
\pgfpathlineto{\pgfqpoint{5.612421in}{1.373559in}}%
\pgfpathlineto{\pgfqpoint{5.612831in}{1.378854in}}%
\pgfpathlineto{\pgfqpoint{5.613241in}{1.376041in}}%
\pgfpathlineto{\pgfqpoint{5.614061in}{1.340376in}}%
\pgfpathlineto{\pgfqpoint{5.614471in}{1.304429in}}%
\pgfpathlineto{\pgfqpoint{5.614881in}{1.338801in}}%
\pgfpathlineto{\pgfqpoint{5.616111in}{1.413152in}}%
\pgfpathlineto{\pgfqpoint{5.616521in}{1.394635in}}%
\pgfpathlineto{\pgfqpoint{5.618571in}{1.152927in}}%
\pgfpathlineto{\pgfqpoint{5.618981in}{1.192643in}}%
\pgfpathlineto{\pgfqpoint{5.621851in}{1.373052in}}%
\pgfpathlineto{\pgfqpoint{5.622261in}{1.377933in}}%
\pgfpathlineto{\pgfqpoint{5.622671in}{1.374672in}}%
\pgfpathlineto{\pgfqpoint{5.623901in}{1.302947in}}%
\pgfpathlineto{\pgfqpoint{5.624311in}{1.340008in}}%
\pgfpathlineto{\pgfqpoint{5.625541in}{1.411296in}}%
\pgfpathlineto{\pgfqpoint{5.625951in}{1.391463in}}%
\pgfpathlineto{\pgfqpoint{5.628001in}{1.155306in}}%
\pgfpathlineto{\pgfqpoint{5.628411in}{1.194673in}}%
\pgfpathlineto{\pgfqpoint{5.631281in}{1.372612in}}%
\pgfpathlineto{\pgfqpoint{5.631691in}{1.377014in}}%
\pgfpathlineto{\pgfqpoint{5.632101in}{1.373231in}}%
\pgfpathlineto{\pgfqpoint{5.633331in}{1.304689in}}%
\pgfpathlineto{\pgfqpoint{5.633741in}{1.341503in}}%
\pgfpathlineto{\pgfqpoint{5.634971in}{1.409351in}}%
\pgfpathlineto{\pgfqpoint{5.635381in}{1.387996in}}%
\pgfpathlineto{\pgfqpoint{5.637431in}{1.157990in}}%
\pgfpathlineto{\pgfqpoint{5.637841in}{1.196993in}}%
\pgfpathlineto{\pgfqpoint{5.640711in}{1.372223in}}%
\pgfpathlineto{\pgfqpoint{5.641121in}{1.376085in}}%
\pgfpathlineto{\pgfqpoint{5.641531in}{1.371705in}}%
\pgfpathlineto{\pgfqpoint{5.642761in}{1.306713in}}%
\pgfpathlineto{\pgfqpoint{5.643171in}{1.343267in}}%
\pgfpathlineto{\pgfqpoint{5.644401in}{1.407275in}}%
\pgfpathlineto{\pgfqpoint{5.645631in}{1.276019in}}%
\pgfpathlineto{\pgfqpoint{5.646861in}{1.160961in}}%
\pgfpathlineto{\pgfqpoint{5.647271in}{1.199588in}}%
\pgfpathlineto{\pgfqpoint{5.650141in}{1.371872in}}%
\pgfpathlineto{\pgfqpoint{5.650551in}{1.375130in}}%
\pgfpathlineto{\pgfqpoint{5.650961in}{1.370082in}}%
\pgfpathlineto{\pgfqpoint{5.652191in}{1.309005in}}%
\pgfpathlineto{\pgfqpoint{5.652601in}{1.345278in}}%
\pgfpathlineto{\pgfqpoint{5.653831in}{1.405028in}}%
\pgfpathlineto{\pgfqpoint{5.655061in}{1.271317in}}%
\pgfpathlineto{\pgfqpoint{5.656291in}{1.164201in}}%
\pgfpathlineto{\pgfqpoint{5.657521in}{1.270372in}}%
\pgfpathlineto{\pgfqpoint{5.657521in}{1.270372in}}%
\pgfusepath{stroke}%
\end{pgfscope}%
\begin{pgfscope}%
\pgfpathrectangle{\pgfqpoint{3.505455in}{0.528000in}}{\pgfqpoint{2.254545in}{1.680000in}}%
\pgfusepath{clip}%
\pgfsetrectcap%
\pgfsetroundjoin%
\pgfsetlinewidth{1.505625pt}%
\definecolor{currentstroke}{rgb}{0.890196,0.466667,0.760784}%
\pgfsetstrokecolor{currentstroke}%
\pgfsetdash{}{0pt}%
\pgfpathmoveto{\pgfqpoint{3.607934in}{1.461333in}}%
\pgfpathlineto{\pgfqpoint{3.618184in}{1.460228in}}%
\pgfpathlineto{\pgfqpoint{3.627614in}{1.457004in}}%
\pgfpathlineto{\pgfqpoint{3.636634in}{1.451510in}}%
\pgfpathlineto{\pgfqpoint{3.644834in}{1.443916in}}%
\pgfpathlineto{\pgfqpoint{3.653034in}{1.433160in}}%
\pgfpathlineto{\pgfqpoint{3.661234in}{1.418418in}}%
\pgfpathlineto{\pgfqpoint{3.669434in}{1.398723in}}%
\pgfpathlineto{\pgfqpoint{3.677634in}{1.373005in}}%
\pgfpathlineto{\pgfqpoint{3.686244in}{1.338287in}}%
\pgfpathlineto{\pgfqpoint{3.695264in}{1.292089in}}%
\pgfpathlineto{\pgfqpoint{3.704694in}{1.231731in}}%
\pgfpathlineto{\pgfqpoint{3.710024in}{1.191804in}}%
\pgfpathlineto{\pgfqpoint{3.710434in}{1.195462in}}%
\pgfpathlineto{\pgfqpoint{3.727244in}{1.430071in}}%
\pgfpathlineto{\pgfqpoint{3.781102in}{2.218000in}}%
\pgfpathmoveto{\pgfqpoint{3.885849in}{2.218000in}}%
\pgfpathlineto{\pgfqpoint{3.894113in}{2.066230in}}%
\pgfpathlineto{\pgfqpoint{3.903543in}{1.849859in}}%
\pgfpathlineto{\pgfqpoint{3.913383in}{1.570280in}}%
\pgfpathlineto{\pgfqpoint{3.924453in}{1.188704in}}%
\pgfpathlineto{\pgfqpoint{3.932243in}{0.882307in}}%
\pgfpathlineto{\pgfqpoint{3.933473in}{0.889454in}}%
\pgfpathlineto{\pgfqpoint{3.962583in}{1.089225in}}%
\pgfpathlineto{\pgfqpoint{3.972013in}{1.137648in}}%
\pgfpathlineto{\pgfqpoint{3.979393in}{1.165181in}}%
\pgfpathlineto{\pgfqpoint{3.985133in}{1.179169in}}%
\pgfpathlineto{\pgfqpoint{3.989643in}{1.185046in}}%
\pgfpathlineto{\pgfqpoint{3.992923in}{1.186221in}}%
\pgfpathlineto{\pgfqpoint{3.995793in}{1.184932in}}%
\pgfpathlineto{\pgfqpoint{3.999073in}{1.180591in}}%
\pgfpathlineto{\pgfqpoint{4.002763in}{1.171697in}}%
\pgfpathlineto{\pgfqpoint{4.007273in}{1.154313in}}%
\pgfpathlineto{\pgfqpoint{4.012193in}{1.125770in}}%
\pgfpathlineto{\pgfqpoint{4.017523in}{1.081038in}}%
\pgfpathlineto{\pgfqpoint{4.023263in}{1.012353in}}%
\pgfpathlineto{\pgfqpoint{4.029003in}{0.915680in}}%
\pgfpathlineto{\pgfqpoint{4.032693in}{0.842892in}}%
\pgfpathlineto{\pgfqpoint{4.033103in}{0.849728in}}%
\pgfpathlineto{\pgfqpoint{4.040073in}{0.985995in}}%
\pgfpathlineto{\pgfqpoint{4.048683in}{1.205588in}}%
\pgfpathlineto{\pgfqpoint{4.058113in}{1.435043in}}%
\pgfpathlineto{\pgfqpoint{4.061803in}{1.477862in}}%
\pgfpathlineto{\pgfqpoint{4.063443in}{1.482318in}}%
\pgfpathlineto{\pgfqpoint{4.064263in}{1.480748in}}%
\pgfpathlineto{\pgfqpoint{4.065903in}{1.469628in}}%
\pgfpathlineto{\pgfqpoint{4.068363in}{1.432445in}}%
\pgfpathlineto{\pgfqpoint{4.071643in}{1.345339in}}%
\pgfpathlineto{\pgfqpoint{4.075743in}{1.191740in}}%
\pgfpathlineto{\pgfqpoint{4.076153in}{1.221713in}}%
\pgfpathlineto{\pgfqpoint{4.087223in}{1.949204in}}%
\pgfpathlineto{\pgfqpoint{4.092919in}{2.218000in}}%
\pgfpathmoveto{\pgfqpoint{4.131546in}{2.218000in}}%
\pgfpathlineto{\pgfqpoint{4.137653in}{1.967288in}}%
\pgfpathlineto{\pgfqpoint{4.144623in}{1.585028in}}%
\pgfpathlineto{\pgfqpoint{4.153233in}{0.979704in}}%
\pgfpathlineto{\pgfqpoint{4.154873in}{0.882473in}}%
\pgfpathlineto{\pgfqpoint{4.155283in}{0.888048in}}%
\pgfpathlineto{\pgfqpoint{4.170453in}{1.080475in}}%
\pgfpathlineto{\pgfqpoint{4.177423in}{1.141162in}}%
\pgfpathlineto{\pgfqpoint{4.182343in}{1.167133in}}%
\pgfpathlineto{\pgfqpoint{4.185623in}{1.175584in}}%
\pgfpathlineto{\pgfqpoint{4.187673in}{1.176938in}}%
\pgfpathlineto{\pgfqpoint{4.189313in}{1.175692in}}%
\pgfpathlineto{\pgfqpoint{4.191363in}{1.171027in}}%
\pgfpathlineto{\pgfqpoint{4.194233in}{1.158161in}}%
\pgfpathlineto{\pgfqpoint{4.197923in}{1.129162in}}%
\pgfpathlineto{\pgfqpoint{4.202023in}{1.076793in}}%
\pgfpathlineto{\pgfqpoint{4.206533in}{0.986954in}}%
\pgfpathlineto{\pgfqpoint{4.211043in}{0.850767in}}%
\pgfpathlineto{\pgfqpoint{4.211453in}{0.835771in}}%
\pgfpathlineto{\pgfqpoint{4.211863in}{0.845975in}}%
\pgfpathlineto{\pgfqpoint{4.217603in}{1.028461in}}%
\pgfpathlineto{\pgfqpoint{4.229903in}{1.447969in}}%
\pgfpathlineto{\pgfqpoint{4.231133in}{1.454109in}}%
\pgfpathlineto{\pgfqpoint{4.231543in}{1.453347in}}%
\pgfpathlineto{\pgfqpoint{4.232773in}{1.442202in}}%
\pgfpathlineto{\pgfqpoint{4.234823in}{1.393445in}}%
\pgfpathlineto{\pgfqpoint{4.238103in}{1.242120in}}%
\pgfpathlineto{\pgfqpoint{4.239333in}{1.172314in}}%
\pgfpathlineto{\pgfqpoint{4.247943in}{1.983225in}}%
\pgfpathlineto{\pgfqpoint{4.251649in}{2.218000in}}%
\pgfpathmoveto{\pgfqpoint{4.276123in}{2.218000in}}%
\pgfpathlineto{\pgfqpoint{4.280743in}{1.967902in}}%
\pgfpathlineto{\pgfqpoint{4.286893in}{1.503119in}}%
\pgfpathlineto{\pgfqpoint{4.293453in}{0.872563in}}%
\pgfpathlineto{\pgfqpoint{4.294683in}{0.895179in}}%
\pgfpathlineto{\pgfqpoint{4.305753in}{1.078285in}}%
\pgfpathlineto{\pgfqpoint{4.311083in}{1.133548in}}%
\pgfpathlineto{\pgfqpoint{4.314773in}{1.153150in}}%
\pgfpathlineto{\pgfqpoint{4.316823in}{1.156563in}}%
\pgfpathlineto{\pgfqpoint{4.318053in}{1.155825in}}%
\pgfpathlineto{\pgfqpoint{4.319693in}{1.151358in}}%
\pgfpathlineto{\pgfqpoint{4.322153in}{1.136490in}}%
\pgfpathlineto{\pgfqpoint{4.325433in}{1.098938in}}%
\pgfpathlineto{\pgfqpoint{4.329123in}{1.026063in}}%
\pgfpathlineto{\pgfqpoint{4.333223in}{0.892744in}}%
\pgfpathlineto{\pgfqpoint{4.334863in}{0.831928in}}%
\pgfpathlineto{\pgfqpoint{4.335273in}{0.845857in}}%
\pgfpathlineto{\pgfqpoint{4.340603in}{1.069806in}}%
\pgfpathlineto{\pgfqpoint{4.347983in}{1.386813in}}%
\pgfpathlineto{\pgfqpoint{4.349623in}{1.404961in}}%
\pgfpathlineto{\pgfqpoint{4.350033in}{1.404182in}}%
\pgfpathlineto{\pgfqpoint{4.351263in}{1.388073in}}%
\pgfpathlineto{\pgfqpoint{4.353313in}{1.314761in}}%
\pgfpathlineto{\pgfqpoint{4.356183in}{1.155822in}}%
\pgfpathlineto{\pgfqpoint{4.363153in}{1.956343in}}%
\pgfpathlineto{\pgfqpoint{4.366818in}{2.218000in}}%
\pgfpathmoveto{\pgfqpoint{4.382904in}{2.218000in}}%
\pgfpathlineto{\pgfqpoint{4.386933in}{1.976830in}}%
\pgfpathlineto{\pgfqpoint{4.392263in}{1.506542in}}%
\pgfpathlineto{\pgfqpoint{4.398003in}{0.864278in}}%
\pgfpathlineto{\pgfqpoint{4.398823in}{0.882445in}}%
\pgfpathlineto{\pgfqpoint{4.407843in}{1.059063in}}%
\pgfpathlineto{\pgfqpoint{4.412353in}{1.110871in}}%
\pgfpathlineto{\pgfqpoint{4.415223in}{1.125098in}}%
\pgfpathlineto{\pgfqpoint{4.416453in}{1.126117in}}%
\pgfpathlineto{\pgfqpoint{4.416863in}{1.125730in}}%
\pgfpathlineto{\pgfqpoint{4.418503in}{1.120327in}}%
\pgfpathlineto{\pgfqpoint{4.420553in}{1.104131in}}%
\pgfpathlineto{\pgfqpoint{4.423423in}{1.060765in}}%
\pgfpathlineto{\pgfqpoint{4.426703in}{0.973524in}}%
\pgfpathlineto{\pgfqpoint{4.430393in}{0.817205in}}%
\pgfpathlineto{\pgfqpoint{4.431213in}{0.850148in}}%
\pgfpathlineto{\pgfqpoint{4.436953in}{1.140193in}}%
\pgfpathlineto{\pgfqpoint{4.441463in}{1.328305in}}%
\pgfpathlineto{\pgfqpoint{4.442693in}{1.338527in}}%
\pgfpathlineto{\pgfqpoint{4.443103in}{1.336265in}}%
\pgfpathlineto{\pgfqpoint{4.444333in}{1.311374in}}%
\pgfpathlineto{\pgfqpoint{4.446383in}{1.210608in}}%
\pgfpathlineto{\pgfqpoint{4.448023in}{1.115519in}}%
\pgfpathlineto{\pgfqpoint{4.454173in}{1.908733in}}%
\pgfpathlineto{\pgfqpoint{4.458273in}{2.198679in}}%
\pgfpathlineto{\pgfqpoint{4.458710in}{2.218000in}}%
\pgfpathmoveto{\pgfqpoint{4.467530in}{2.218000in}}%
\pgfpathlineto{\pgfqpoint{4.470573in}{2.064403in}}%
\pgfpathlineto{\pgfqpoint{4.474673in}{1.735439in}}%
\pgfpathlineto{\pgfqpoint{4.480003in}{1.108618in}}%
\pgfpathlineto{\pgfqpoint{4.482053in}{0.847948in}}%
\pgfpathlineto{\pgfqpoint{4.482873in}{0.868448in}}%
\pgfpathlineto{\pgfqpoint{4.490662in}{1.035334in}}%
\pgfpathlineto{\pgfqpoint{4.494352in}{1.077619in}}%
\pgfpathlineto{\pgfqpoint{4.496812in}{1.086482in}}%
\pgfpathlineto{\pgfqpoint{4.497632in}{1.085500in}}%
\pgfpathlineto{\pgfqpoint{4.499272in}{1.076982in}}%
\pgfpathlineto{\pgfqpoint{4.501732in}{1.045661in}}%
\pgfpathlineto{\pgfqpoint{4.504602in}{0.974505in}}%
\pgfpathlineto{\pgfqpoint{4.507882in}{0.832131in}}%
\pgfpathlineto{\pgfqpoint{4.508702in}{0.805840in}}%
\pgfpathlineto{\pgfqpoint{4.509112in}{0.823802in}}%
\pgfpathlineto{\pgfqpoint{4.518952in}{1.259058in}}%
\pgfpathlineto{\pgfqpoint{4.519362in}{1.257509in}}%
\pgfpathlineto{\pgfqpoint{4.520592in}{1.232009in}}%
\pgfpathlineto{\pgfqpoint{4.522642in}{1.121659in}}%
\pgfpathlineto{\pgfqpoint{4.523462in}{1.057416in}}%
\pgfpathlineto{\pgfqpoint{4.523872in}{1.088843in}}%
\pgfpathlineto{\pgfqpoint{4.529612in}{1.860071in}}%
\pgfpathlineto{\pgfqpoint{4.533302in}{2.105268in}}%
\pgfpathlineto{\pgfqpoint{4.535762in}{2.162645in}}%
\pgfpathlineto{\pgfqpoint{4.536582in}{2.164845in}}%
\pgfpathlineto{\pgfqpoint{4.537402in}{2.159050in}}%
\pgfpathlineto{\pgfqpoint{4.539042in}{2.124271in}}%
\pgfpathlineto{\pgfqpoint{4.541502in}{2.015986in}}%
\pgfpathlineto{\pgfqpoint{4.545192in}{1.730232in}}%
\pgfpathlineto{\pgfqpoint{4.550112in}{1.134405in}}%
\pgfpathlineto{\pgfqpoint{4.552162in}{0.830938in}}%
\pgfpathlineto{\pgfqpoint{4.552982in}{0.845586in}}%
\pgfpathlineto{\pgfqpoint{4.559542in}{0.994933in}}%
\pgfpathlineto{\pgfqpoint{4.562822in}{1.033062in}}%
\pgfpathlineto{\pgfqpoint{4.564462in}{1.038719in}}%
\pgfpathlineto{\pgfqpoint{4.564872in}{1.038542in}}%
\pgfpathlineto{\pgfqpoint{4.566102in}{1.033873in}}%
\pgfpathlineto{\pgfqpoint{4.568152in}{1.010834in}}%
\pgfpathlineto{\pgfqpoint{4.570612in}{0.953246in}}%
\pgfpathlineto{\pgfqpoint{4.573892in}{0.810607in}}%
\pgfpathlineto{\pgfqpoint{4.574302in}{0.786379in}}%
\pgfpathlineto{\pgfqpoint{4.574712in}{0.786532in}}%
\pgfpathlineto{\pgfqpoint{4.583732in}{1.171005in}}%
\pgfpathlineto{\pgfqpoint{4.584142in}{1.168567in}}%
\pgfpathlineto{\pgfqpoint{4.585372in}{1.139189in}}%
\pgfpathlineto{\pgfqpoint{4.587422in}{1.020227in}}%
\pgfpathlineto{\pgfqpoint{4.587832in}{0.987872in}}%
\pgfpathlineto{\pgfqpoint{4.593572in}{1.770572in}}%
\pgfpathlineto{\pgfqpoint{4.596852in}{1.970369in}}%
\pgfpathlineto{\pgfqpoint{4.598902in}{2.003733in}}%
\pgfpathlineto{\pgfqpoint{4.599722in}{1.998968in}}%
\pgfpathlineto{\pgfqpoint{4.601362in}{1.960041in}}%
\pgfpathlineto{\pgfqpoint{4.603822in}{1.831240in}}%
\pgfpathlineto{\pgfqpoint{4.607512in}{1.488322in}}%
\pgfpathlineto{\pgfqpoint{4.612432in}{0.801701in}}%
\pgfpathlineto{\pgfqpoint{4.613662in}{0.835199in}}%
\pgfpathlineto{\pgfqpoint{4.619402in}{0.959900in}}%
\pgfpathlineto{\pgfqpoint{4.622272in}{0.984195in}}%
\pgfpathlineto{\pgfqpoint{4.623092in}{0.984621in}}%
\pgfpathlineto{\pgfqpoint{4.624322in}{0.979007in}}%
\pgfpathlineto{\pgfqpoint{4.626372in}{0.950817in}}%
\pgfpathlineto{\pgfqpoint{4.628832in}{0.879816in}}%
\pgfpathlineto{\pgfqpoint{4.631292in}{0.758680in}}%
\pgfpathlineto{\pgfqpoint{4.632112in}{0.783568in}}%
\pgfpathlineto{\pgfqpoint{4.639082in}{1.077222in}}%
\pgfpathlineto{\pgfqpoint{4.639492in}{1.079805in}}%
\pgfpathlineto{\pgfqpoint{4.639902in}{1.079036in}}%
\pgfpathlineto{\pgfqpoint{4.641132in}{1.055377in}}%
\pgfpathlineto{\pgfqpoint{4.643592in}{0.943643in}}%
\pgfpathlineto{\pgfqpoint{4.648512in}{1.612938in}}%
\pgfpathlineto{\pgfqpoint{4.651792in}{1.807958in}}%
\pgfpathlineto{\pgfqpoint{4.653432in}{1.827834in}}%
\pgfpathlineto{\pgfqpoint{4.654252in}{1.819599in}}%
\pgfpathlineto{\pgfqpoint{4.655892in}{1.768528in}}%
\pgfpathlineto{\pgfqpoint{4.658352in}{1.610010in}}%
\pgfpathlineto{\pgfqpoint{4.662042in}{1.207210in}}%
\pgfpathlineto{\pgfqpoint{4.665322in}{0.782963in}}%
\pgfpathlineto{\pgfqpoint{4.665732in}{0.794313in}}%
\pgfpathlineto{\pgfqpoint{4.670652in}{0.903008in}}%
\pgfpathlineto{\pgfqpoint{4.673112in}{0.926087in}}%
\pgfpathlineto{\pgfqpoint{4.673932in}{0.927265in}}%
\pgfpathlineto{\pgfqpoint{4.674342in}{0.926474in}}%
\pgfpathlineto{\pgfqpoint{4.675572in}{0.918138in}}%
\pgfpathlineto{\pgfqpoint{4.677622in}{0.881964in}}%
\pgfpathlineto{\pgfqpoint{4.680082in}{0.795067in}}%
\pgfpathlineto{\pgfqpoint{4.681312in}{0.731256in}}%
\pgfpathlineto{\pgfqpoint{4.681722in}{0.737658in}}%
\pgfpathlineto{\pgfqpoint{4.688282in}{0.987995in}}%
\pgfpathlineto{\pgfqpoint{4.688692in}{0.990565in}}%
\pgfpathlineto{\pgfqpoint{4.689102in}{0.990079in}}%
\pgfpathlineto{\pgfqpoint{4.690332in}{0.969082in}}%
\pgfpathlineto{\pgfqpoint{4.692382in}{0.871060in}}%
\pgfpathlineto{\pgfqpoint{4.692792in}{0.908736in}}%
\pgfpathlineto{\pgfqpoint{4.697302in}{1.480376in}}%
\pgfpathlineto{\pgfqpoint{4.700172in}{1.633582in}}%
\pgfpathlineto{\pgfqpoint{4.701402in}{1.646330in}}%
\pgfpathlineto{\pgfqpoint{4.701812in}{1.643803in}}%
\pgfpathlineto{\pgfqpoint{4.703042in}{1.616593in}}%
\pgfpathlineto{\pgfqpoint{4.705092in}{1.508991in}}%
\pgfpathlineto{\pgfqpoint{4.708372in}{1.192931in}}%
\pgfpathlineto{\pgfqpoint{4.711652in}{0.746644in}}%
\pgfpathlineto{\pgfqpoint{4.712882in}{0.777540in}}%
\pgfpathlineto{\pgfqpoint{4.716982in}{0.856151in}}%
\pgfpathlineto{\pgfqpoint{4.719032in}{0.869013in}}%
\pgfpathlineto{\pgfqpoint{4.719852in}{0.867563in}}%
\pgfpathlineto{\pgfqpoint{4.721082in}{0.857278in}}%
\pgfpathlineto{\pgfqpoint{4.723132in}{0.815715in}}%
\pgfpathlineto{\pgfqpoint{4.726002in}{0.698604in}}%
\pgfpathlineto{\pgfqpoint{4.726822in}{0.729169in}}%
\pgfpathlineto{\pgfqpoint{4.731742in}{0.899117in}}%
\pgfpathlineto{\pgfqpoint{4.732972in}{0.907388in}}%
\pgfpathlineto{\pgfqpoint{4.733792in}{0.899336in}}%
\pgfpathlineto{\pgfqpoint{4.735432in}{0.848928in}}%
\pgfpathlineto{\pgfqpoint{4.736252in}{0.808232in}}%
\pgfpathlineto{\pgfqpoint{4.743222in}{1.454436in}}%
\pgfpathlineto{\pgfqpoint{4.744452in}{1.467509in}}%
\pgfpathlineto{\pgfqpoint{4.744862in}{1.464722in}}%
\pgfpathlineto{\pgfqpoint{4.746092in}{1.435659in}}%
\pgfpathlineto{\pgfqpoint{4.748142in}{1.322241in}}%
\pgfpathlineto{\pgfqpoint{4.751422in}{0.997843in}}%
\pgfpathlineto{\pgfqpoint{4.753882in}{0.722081in}}%
\pgfpathlineto{\pgfqpoint{4.754292in}{0.732263in}}%
\pgfpathlineto{\pgfqpoint{4.758392in}{0.804840in}}%
\pgfpathlineto{\pgfqpoint{4.760032in}{0.812764in}}%
\pgfpathlineto{\pgfqpoint{4.760852in}{0.810761in}}%
\pgfpathlineto{\pgfqpoint{4.762492in}{0.793128in}}%
\pgfpathlineto{\pgfqpoint{4.764952in}{0.728334in}}%
\pgfpathlineto{\pgfqpoint{4.766182in}{0.677552in}}%
\pgfpathlineto{\pgfqpoint{4.766592in}{0.677790in}}%
\pgfpathlineto{\pgfqpoint{4.771512in}{0.826079in}}%
\pgfpathlineto{\pgfqpoint{4.772742in}{0.833135in}}%
\pgfpathlineto{\pgfqpoint{4.773562in}{0.826361in}}%
\pgfpathlineto{\pgfqpoint{4.775202in}{0.783920in}}%
\pgfpathlineto{\pgfqpoint{4.776022in}{0.750005in}}%
\pgfpathlineto{\pgfqpoint{4.780532in}{1.212089in}}%
\pgfpathlineto{\pgfqpoint{4.782992in}{1.297830in}}%
\pgfpathlineto{\pgfqpoint{4.783402in}{1.299430in}}%
\pgfpathlineto{\pgfqpoint{4.783812in}{1.297469in}}%
\pgfpathlineto{\pgfqpoint{4.785042in}{1.270761in}}%
\pgfpathlineto{\pgfqpoint{4.787092in}{1.161496in}}%
\pgfpathlineto{\pgfqpoint{4.790782in}{0.803909in}}%
\pgfpathlineto{\pgfqpoint{4.792012in}{0.690473in}}%
\pgfpathlineto{\pgfqpoint{4.792422in}{0.699700in}}%
\pgfpathlineto{\pgfqpoint{4.796112in}{0.756027in}}%
\pgfpathlineto{\pgfqpoint{4.797342in}{0.760629in}}%
\pgfpathlineto{\pgfqpoint{4.797752in}{0.760202in}}%
\pgfpathlineto{\pgfqpoint{4.798982in}{0.752474in}}%
\pgfpathlineto{\pgfqpoint{4.801032in}{0.716176in}}%
\pgfpathlineto{\pgfqpoint{4.803082in}{0.649154in}}%
\pgfpathlineto{\pgfqpoint{4.803902in}{0.665167in}}%
\pgfpathlineto{\pgfqpoint{4.808002in}{0.764170in}}%
\pgfpathlineto{\pgfqpoint{4.808822in}{0.769056in}}%
\pgfpathlineto{\pgfqpoint{4.809232in}{0.768732in}}%
\pgfpathlineto{\pgfqpoint{4.810462in}{0.755876in}}%
\pgfpathlineto{\pgfqpoint{4.812102in}{0.711700in}}%
\pgfpathlineto{\pgfqpoint{4.812512in}{0.719332in}}%
\pgfpathlineto{\pgfqpoint{4.816612in}{1.079813in}}%
\pgfpathlineto{\pgfqpoint{4.819072in}{1.147348in}}%
\pgfpathlineto{\pgfqpoint{4.819482in}{1.146560in}}%
\pgfpathlineto{\pgfqpoint{4.820712in}{1.124167in}}%
\pgfpathlineto{\pgfqpoint{4.822762in}{1.024996in}}%
\pgfpathlineto{\pgfqpoint{4.826452in}{0.700327in}}%
\pgfpathlineto{\pgfqpoint{4.826862in}{0.657337in}}%
\pgfpathlineto{\pgfqpoint{4.827682in}{0.672679in}}%
\pgfpathlineto{\pgfqpoint{4.830962in}{0.712246in}}%
\pgfpathlineto{\pgfqpoint{4.831782in}{0.714104in}}%
\pgfpathlineto{\pgfqpoint{4.832192in}{0.713606in}}%
\pgfpathlineto{\pgfqpoint{4.833422in}{0.706006in}}%
\pgfpathlineto{\pgfqpoint{4.835472in}{0.671702in}}%
\pgfpathlineto{\pgfqpoint{4.837112in}{0.624850in}}%
\pgfpathlineto{\pgfqpoint{4.837522in}{0.627184in}}%
\pgfpathlineto{\pgfqpoint{4.841622in}{0.711205in}}%
\pgfpathlineto{\pgfqpoint{4.842442in}{0.715179in}}%
\pgfpathlineto{\pgfqpoint{4.842852in}{0.714887in}}%
\pgfpathlineto{\pgfqpoint{4.844082in}{0.704350in}}%
\pgfpathlineto{\pgfqpoint{4.845722in}{0.668521in}}%
\pgfpathlineto{\pgfqpoint{4.846132in}{0.682464in}}%
\pgfpathlineto{\pgfqpoint{4.850232in}{0.975086in}}%
\pgfpathlineto{\pgfqpoint{4.852282in}{1.015062in}}%
\pgfpathlineto{\pgfqpoint{4.853102in}{1.009049in}}%
\pgfpathlineto{\pgfqpoint{4.854742in}{0.961302in}}%
\pgfpathlineto{\pgfqpoint{4.857612in}{0.782111in}}%
\pgfpathlineto{\pgfqpoint{4.859662in}{0.635514in}}%
\pgfpathlineto{\pgfqpoint{4.860072in}{0.642337in}}%
\pgfpathlineto{\pgfqpoint{4.862942in}{0.672234in}}%
\pgfpathlineto{\pgfqpoint{4.863762in}{0.673895in}}%
\pgfpathlineto{\pgfqpoint{4.864172in}{0.673408in}}%
\pgfpathlineto{\pgfqpoint{4.865402in}{0.666378in}}%
\pgfpathlineto{\pgfqpoint{4.867452in}{0.635482in}}%
\pgfpathlineto{\pgfqpoint{4.869092in}{0.603754in}}%
\pgfpathlineto{\pgfqpoint{4.869502in}{0.613220in}}%
\pgfpathlineto{\pgfqpoint{4.872782in}{0.666813in}}%
\pgfpathlineto{\pgfqpoint{4.874012in}{0.670948in}}%
\pgfpathlineto{\pgfqpoint{4.874832in}{0.667393in}}%
\pgfpathlineto{\pgfqpoint{4.876472in}{0.644964in}}%
\pgfpathlineto{\pgfqpoint{4.876882in}{0.636441in}}%
\pgfpathlineto{\pgfqpoint{4.877292in}{0.640658in}}%
\pgfpathlineto{\pgfqpoint{4.881392in}{0.877948in}}%
\pgfpathlineto{\pgfqpoint{4.883032in}{0.903218in}}%
\pgfpathlineto{\pgfqpoint{4.883442in}{0.902551in}}%
\pgfpathlineto{\pgfqpoint{4.884672in}{0.884243in}}%
\pgfpathlineto{\pgfqpoint{4.886722in}{0.804497in}}%
\pgfpathlineto{\pgfqpoint{4.890002in}{0.612075in}}%
\pgfpathlineto{\pgfqpoint{4.890822in}{0.622808in}}%
\pgfpathlineto{\pgfqpoint{4.893282in}{0.639836in}}%
\pgfpathlineto{\pgfqpoint{4.894102in}{0.639624in}}%
\pgfpathlineto{\pgfqpoint{4.895332in}{0.633190in}}%
\pgfpathlineto{\pgfqpoint{4.897382in}{0.606061in}}%
\pgfpathlineto{\pgfqpoint{4.898612in}{0.584091in}}%
\pgfpathlineto{\pgfqpoint{4.899022in}{0.591853in}}%
\pgfpathlineto{\pgfqpoint{4.902302in}{0.633344in}}%
\pgfpathlineto{\pgfqpoint{4.903122in}{0.635505in}}%
\pgfpathlineto{\pgfqpoint{4.903532in}{0.635071in}}%
\pgfpathlineto{\pgfqpoint{4.904762in}{0.627559in}}%
\pgfpathlineto{\pgfqpoint{4.906402in}{0.604808in}}%
\pgfpathlineto{\pgfqpoint{4.910092in}{0.784946in}}%
\pgfpathlineto{\pgfqpoint{4.912142in}{0.811527in}}%
\pgfpathlineto{\pgfqpoint{4.912962in}{0.805455in}}%
\pgfpathlineto{\pgfqpoint{4.914602in}{0.766572in}}%
\pgfpathlineto{\pgfqpoint{4.917472in}{0.632129in}}%
\pgfpathlineto{\pgfqpoint{4.918292in}{0.590275in}}%
\pgfpathlineto{\pgfqpoint{4.919112in}{0.599260in}}%
\pgfpathlineto{\pgfqpoint{4.921572in}{0.612468in}}%
\pgfpathlineto{\pgfqpoint{4.922392in}{0.611627in}}%
\pgfpathlineto{\pgfqpoint{4.923622in}{0.605093in}}%
\pgfpathlineto{\pgfqpoint{4.926492in}{0.570509in}}%
\pgfpathlineto{\pgfqpoint{4.927312in}{0.582385in}}%
\pgfpathlineto{\pgfqpoint{4.930182in}{0.606813in}}%
\pgfpathlineto{\pgfqpoint{4.931002in}{0.607295in}}%
\pgfpathlineto{\pgfqpoint{4.932232in}{0.601893in}}%
\pgfpathlineto{\pgfqpoint{4.933872in}{0.584148in}}%
\pgfpathlineto{\pgfqpoint{4.937562in}{0.722259in}}%
\pgfpathlineto{\pgfqpoint{4.939202in}{0.738291in}}%
\pgfpathlineto{\pgfqpoint{4.940022in}{0.734316in}}%
\pgfpathlineto{\pgfqpoint{4.941662in}{0.703916in}}%
\pgfpathlineto{\pgfqpoint{4.944532in}{0.596127in}}%
\pgfpathlineto{\pgfqpoint{4.945352in}{0.576896in}}%
\pgfpathlineto{\pgfqpoint{4.945762in}{0.580360in}}%
\pgfpathlineto{\pgfqpoint{4.948222in}{0.590485in}}%
\pgfpathlineto{\pgfqpoint{4.949042in}{0.589360in}}%
\pgfpathlineto{\pgfqpoint{4.950682in}{0.580140in}}%
\pgfpathlineto{\pgfqpoint{4.952732in}{0.558798in}}%
\pgfpathlineto{\pgfqpoint{4.953142in}{0.563582in}}%
\pgfpathlineto{\pgfqpoint{4.956012in}{0.584818in}}%
\pgfpathlineto{\pgfqpoint{4.956832in}{0.585796in}}%
\pgfpathlineto{\pgfqpoint{4.957242in}{0.585323in}}%
\pgfpathlineto{\pgfqpoint{4.958472in}{0.580090in}}%
\pgfpathlineto{\pgfqpoint{4.959702in}{0.569642in}}%
\pgfpathlineto{\pgfqpoint{4.965032in}{0.680853in}}%
\pgfpathlineto{\pgfqpoint{4.965852in}{0.676232in}}%
\pgfpathlineto{\pgfqpoint{4.967492in}{0.649135in}}%
\pgfpathlineto{\pgfqpoint{4.970362in}{0.561549in}}%
\pgfpathlineto{\pgfqpoint{4.971592in}{0.569173in}}%
\pgfpathlineto{\pgfqpoint{4.973232in}{0.573299in}}%
\pgfpathlineto{\pgfqpoint{4.973642in}{0.573156in}}%
\pgfpathlineto{\pgfqpoint{4.974872in}{0.569863in}}%
\pgfpathlineto{\pgfqpoint{4.976922in}{0.555616in}}%
\pgfpathlineto{\pgfqpoint{4.977742in}{0.551488in}}%
\pgfpathlineto{\pgfqpoint{4.980612in}{0.568473in}}%
\pgfpathlineto{\pgfqpoint{4.981432in}{0.569328in}}%
\pgfpathlineto{\pgfqpoint{4.981842in}{0.569005in}}%
\pgfpathlineto{\pgfqpoint{4.983072in}{0.565096in}}%
\pgfpathlineto{\pgfqpoint{4.984302in}{0.557207in}}%
\pgfpathlineto{\pgfqpoint{4.988402in}{0.635158in}}%
\pgfpathlineto{\pgfqpoint{4.989222in}{0.637221in}}%
\pgfpathlineto{\pgfqpoint{4.989632in}{0.636306in}}%
\pgfpathlineto{\pgfqpoint{4.990862in}{0.626118in}}%
\pgfpathlineto{\pgfqpoint{4.993322in}{0.578419in}}%
\pgfpathlineto{\pgfqpoint{4.994552in}{0.553102in}}%
\pgfpathlineto{\pgfqpoint{4.994962in}{0.555157in}}%
\pgfpathlineto{\pgfqpoint{4.997012in}{0.560213in}}%
\pgfpathlineto{\pgfqpoint{4.998242in}{0.558734in}}%
\pgfpathlineto{\pgfqpoint{5.000292in}{0.549189in}}%
\pgfpathlineto{\pgfqpoint{5.001112in}{0.543405in}}%
\pgfpathlineto{\pgfqpoint{5.001522in}{0.545912in}}%
\pgfpathlineto{\pgfqpoint{5.004392in}{0.556885in}}%
\pgfpathlineto{\pgfqpoint{5.005622in}{0.556176in}}%
\pgfpathlineto{\pgfqpoint{5.007672in}{0.548858in}}%
\pgfpathlineto{\pgfqpoint{5.010952in}{0.600027in}}%
\pgfpathlineto{\pgfqpoint{5.012182in}{0.604537in}}%
\pgfpathlineto{\pgfqpoint{5.012592in}{0.604020in}}%
\pgfpathlineto{\pgfqpoint{5.013822in}{0.596669in}}%
\pgfpathlineto{\pgfqpoint{5.016282in}{0.560949in}}%
\pgfpathlineto{\pgfqpoint{5.017102in}{0.545264in}}%
\pgfpathlineto{\pgfqpoint{5.017922in}{0.547710in}}%
\pgfpathlineto{\pgfqpoint{5.019972in}{0.550401in}}%
\pgfpathlineto{\pgfqpoint{5.021202in}{0.548485in}}%
\pgfpathlineto{\pgfqpoint{5.023662in}{0.539170in}}%
\pgfpathlineto{\pgfqpoint{5.024072in}{0.541105in}}%
\pgfpathlineto{\pgfqpoint{5.026532in}{0.547855in}}%
\pgfpathlineto{\pgfqpoint{5.027762in}{0.547506in}}%
\pgfpathlineto{\pgfqpoint{5.029812in}{0.542573in}}%
\pgfpathlineto{\pgfqpoint{5.033092in}{0.578369in}}%
\pgfpathlineto{\pgfqpoint{5.034322in}{0.580461in}}%
\pgfpathlineto{\pgfqpoint{5.035552in}{0.575795in}}%
\pgfpathlineto{\pgfqpoint{5.038012in}{0.550318in}}%
\pgfpathlineto{\pgfqpoint{5.038832in}{0.539648in}}%
\pgfpathlineto{\pgfqpoint{5.039652in}{0.541740in}}%
\pgfpathlineto{\pgfqpoint{5.041702in}{0.543195in}}%
\pgfpathlineto{\pgfqpoint{5.043342in}{0.540405in}}%
\pgfpathlineto{\pgfqpoint{5.044982in}{0.535537in}}%
\pgfpathlineto{\pgfqpoint{5.045392in}{0.536919in}}%
\pgfpathlineto{\pgfqpoint{5.047852in}{0.541504in}}%
\pgfpathlineto{\pgfqpoint{5.049082in}{0.541010in}}%
\pgfpathlineto{\pgfqpoint{5.050722in}{0.537625in}}%
\pgfpathlineto{\pgfqpoint{5.054822in}{0.563420in}}%
\pgfpathlineto{\pgfqpoint{5.055232in}{0.563293in}}%
\pgfpathlineto{\pgfqpoint{5.056462in}{0.559635in}}%
\pgfpathlineto{\pgfqpoint{5.058922in}{0.540799in}}%
\pgfpathlineto{\pgfqpoint{5.059742in}{0.536378in}}%
\pgfpathlineto{\pgfqpoint{5.060152in}{0.537083in}}%
\pgfpathlineto{\pgfqpoint{5.062202in}{0.538171in}}%
\pgfpathlineto{\pgfqpoint{5.063842in}{0.536129in}}%
\pgfpathlineto{\pgfqpoint{5.065482in}{0.533407in}}%
\pgfpathlineto{\pgfqpoint{5.065892in}{0.534327in}}%
\pgfpathlineto{\pgfqpoint{5.068352in}{0.537004in}}%
\pgfpathlineto{\pgfqpoint{5.069992in}{0.535778in}}%
\pgfpathlineto{\pgfqpoint{5.070812in}{0.534369in}}%
\pgfpathlineto{\pgfqpoint{5.074912in}{0.551400in}}%
\pgfpathlineto{\pgfqpoint{5.076142in}{0.549348in}}%
\pgfpathlineto{\pgfqpoint{5.078192in}{0.539330in}}%
\pgfpathlineto{\pgfqpoint{5.079422in}{0.533518in}}%
\pgfpathlineto{\pgfqpoint{5.079832in}{0.533998in}}%
\pgfpathlineto{\pgfqpoint{5.081882in}{0.534626in}}%
\pgfpathlineto{\pgfqpoint{5.083932in}{0.532423in}}%
\pgfpathlineto{\pgfqpoint{5.084752in}{0.531297in}}%
\pgfpathlineto{\pgfqpoint{5.085162in}{0.531950in}}%
\pgfpathlineto{\pgfqpoint{5.087622in}{0.533861in}}%
\pgfpathlineto{\pgfqpoint{5.089672in}{0.532544in}}%
\pgfpathlineto{\pgfqpoint{5.090082in}{0.532011in}}%
\pgfpathlineto{\pgfqpoint{5.093362in}{0.542963in}}%
\pgfpathlineto{\pgfqpoint{5.094592in}{0.542639in}}%
\pgfpathlineto{\pgfqpoint{5.096232in}{0.538518in}}%
\pgfpathlineto{\pgfqpoint{5.098282in}{0.531610in}}%
\pgfpathlineto{\pgfqpoint{5.098692in}{0.531918in}}%
\pgfpathlineto{\pgfqpoint{5.100742in}{0.532198in}}%
\pgfpathlineto{\pgfqpoint{5.104842in}{0.531413in}}%
\pgfpathlineto{\pgfqpoint{5.107302in}{0.531380in}}%
\pgfpathlineto{\pgfqpoint{5.108532in}{0.530922in}}%
\pgfpathlineto{\pgfqpoint{5.111402in}{0.537393in}}%
\pgfpathlineto{\pgfqpoint{5.113042in}{0.536941in}}%
\pgfpathlineto{\pgfqpoint{5.115092in}{0.532575in}}%
\pgfpathlineto{\pgfqpoint{5.115912in}{0.530146in}}%
\pgfpathlineto{\pgfqpoint{5.116732in}{0.530504in}}%
\pgfpathlineto{\pgfqpoint{5.119192in}{0.530438in}}%
\pgfpathlineto{\pgfqpoint{5.122062in}{0.529935in}}%
\pgfpathlineto{\pgfqpoint{5.124931in}{0.530093in}}%
\pgfpathlineto{\pgfqpoint{5.126161in}{0.530021in}}%
\pgfpathlineto{\pgfqpoint{5.129031in}{0.533906in}}%
\pgfpathlineto{\pgfqpoint{5.130671in}{0.533288in}}%
\pgfpathlineto{\pgfqpoint{5.135181in}{0.529680in}}%
\pgfpathlineto{\pgfqpoint{5.143381in}{0.529797in}}%
\pgfpathlineto{\pgfqpoint{5.146251in}{0.531598in}}%
\pgfpathlineto{\pgfqpoint{5.148711in}{0.529921in}}%
\pgfpathlineto{\pgfqpoint{5.150761in}{0.528979in}}%
\pgfpathlineto{\pgfqpoint{5.164701in}{0.529057in}}%
\pgfpathlineto{\pgfqpoint{5.167161in}{0.528597in}}%
\pgfpathlineto{\pgfqpoint{5.264331in}{0.528003in}}%
\pgfpathlineto{\pgfqpoint{5.657521in}{0.528000in}}%
\pgfpathlineto{\pgfqpoint{5.657521in}{0.528000in}}%
\pgfusepath{stroke}%
\end{pgfscope}%
\begin{pgfscope}%
\pgfpathrectangle{\pgfqpoint{3.505455in}{0.528000in}}{\pgfqpoint{2.254545in}{1.680000in}}%
\pgfusepath{clip}%
\pgfsetrectcap%
\pgfsetroundjoin%
\pgfsetlinewidth{1.505625pt}%
\definecolor{currentstroke}{rgb}{0.498039,0.498039,0.498039}%
\pgfsetstrokecolor{currentstroke}%
\pgfsetdash{}{0pt}%
\pgfpathmoveto{\pgfqpoint{3.607934in}{1.461333in}}%
\pgfpathlineto{\pgfqpoint{3.613264in}{1.460236in}}%
\pgfpathlineto{\pgfqpoint{3.618594in}{1.456582in}}%
\pgfpathlineto{\pgfqpoint{3.623924in}{1.450048in}}%
\pgfpathlineto{\pgfqpoint{3.629664in}{1.439192in}}%
\pgfpathlineto{\pgfqpoint{3.635814in}{1.422167in}}%
\pgfpathlineto{\pgfqpoint{3.641964in}{1.398225in}}%
\pgfpathlineto{\pgfqpoint{3.648524in}{1.363315in}}%
\pgfpathlineto{\pgfqpoint{3.655494in}{1.313495in}}%
\pgfpathlineto{\pgfqpoint{3.662874in}{1.244193in}}%
\pgfpathlineto{\pgfqpoint{3.667794in}{1.190199in}}%
\pgfpathlineto{\pgfqpoint{3.668204in}{1.198896in}}%
\pgfpathlineto{\pgfqpoint{3.687474in}{1.632943in}}%
\pgfpathlineto{\pgfqpoint{3.702234in}{1.944113in}}%
\pgfpathlineto{\pgfqpoint{3.710844in}{2.084254in}}%
\pgfpathlineto{\pgfqpoint{3.717404in}{2.158522in}}%
\pgfpathlineto{\pgfqpoint{3.722324in}{2.191489in}}%
\pgfpathlineto{\pgfqpoint{3.725604in}{2.201273in}}%
\pgfpathlineto{\pgfqpoint{3.727654in}{2.202082in}}%
\pgfpathlineto{\pgfqpoint{3.729294in}{2.199666in}}%
\pgfpathlineto{\pgfqpoint{3.731754in}{2.190772in}}%
\pgfpathlineto{\pgfqpoint{3.735034in}{2.168727in}}%
\pgfpathlineto{\pgfqpoint{3.739134in}{2.124152in}}%
\pgfpathlineto{\pgfqpoint{3.744054in}{2.044763in}}%
\pgfpathlineto{\pgfqpoint{3.750204in}{1.904729in}}%
\pgfpathlineto{\pgfqpoint{3.757174in}{1.691235in}}%
\pgfpathlineto{\pgfqpoint{3.765784in}{1.351792in}}%
\pgfpathlineto{\pgfqpoint{3.774394in}{0.948603in}}%
\pgfpathlineto{\pgfqpoint{3.775624in}{0.959376in}}%
\pgfpathlineto{\pgfqpoint{3.786284in}{1.060495in}}%
\pgfpathlineto{\pgfqpoint{3.792844in}{1.103994in}}%
\pgfpathlineto{\pgfqpoint{3.796944in}{1.118823in}}%
\pgfpathlineto{\pgfqpoint{3.799404in}{1.121331in}}%
\pgfpathlineto{\pgfqpoint{3.801044in}{1.119707in}}%
\pgfpathlineto{\pgfqpoint{3.803094in}{1.113325in}}%
\pgfpathlineto{\pgfqpoint{3.805964in}{1.094757in}}%
\pgfpathlineto{\pgfqpoint{3.809244in}{1.056680in}}%
\pgfpathlineto{\pgfqpoint{3.812934in}{0.987108in}}%
\pgfpathlineto{\pgfqpoint{3.817444in}{0.855441in}}%
\pgfpathlineto{\pgfqpoint{3.819904in}{0.759994in}}%
\pgfpathlineto{\pgfqpoint{3.820314in}{0.763083in}}%
\pgfpathlineto{\pgfqpoint{3.835484in}{1.314944in}}%
\pgfpathlineto{\pgfqpoint{3.838354in}{1.352831in}}%
\pgfpathlineto{\pgfqpoint{3.839584in}{1.356834in}}%
\pgfpathlineto{\pgfqpoint{3.839994in}{1.356448in}}%
\pgfpathlineto{\pgfqpoint{3.841224in}{1.350052in}}%
\pgfpathlineto{\pgfqpoint{3.843274in}{1.321961in}}%
\pgfpathlineto{\pgfqpoint{3.846144in}{1.247873in}}%
\pgfpathlineto{\pgfqpoint{3.849834in}{1.103610in}}%
\pgfpathlineto{\pgfqpoint{3.850244in}{1.125029in}}%
\pgfpathlineto{\pgfqpoint{3.858033in}{1.679747in}}%
\pgfpathlineto{\pgfqpoint{3.862953in}{1.899391in}}%
\pgfpathlineto{\pgfqpoint{3.866643in}{1.983786in}}%
\pgfpathlineto{\pgfqpoint{3.869103in}{2.000536in}}%
\pgfpathlineto{\pgfqpoint{3.869923in}{1.999121in}}%
\pgfpathlineto{\pgfqpoint{3.871563in}{1.985922in}}%
\pgfpathlineto{\pgfqpoint{3.874023in}{1.940720in}}%
\pgfpathlineto{\pgfqpoint{3.877303in}{1.835049in}}%
\pgfpathlineto{\pgfqpoint{3.881813in}{1.612395in}}%
\pgfpathlineto{\pgfqpoint{3.888373in}{1.159015in}}%
\pgfpathlineto{\pgfqpoint{3.891653in}{0.913598in}}%
\pgfpathlineto{\pgfqpoint{3.892063in}{0.921093in}}%
\pgfpathlineto{\pgfqpoint{3.898623in}{1.021329in}}%
\pgfpathlineto{\pgfqpoint{3.902723in}{1.057296in}}%
\pgfpathlineto{\pgfqpoint{3.905183in}{1.063864in}}%
\pgfpathlineto{\pgfqpoint{3.906413in}{1.061663in}}%
\pgfpathlineto{\pgfqpoint{3.908053in}{1.051910in}}%
\pgfpathlineto{\pgfqpoint{3.910513in}{1.019855in}}%
\pgfpathlineto{\pgfqpoint{3.913383in}{0.949993in}}%
\pgfpathlineto{\pgfqpoint{3.917073in}{0.798277in}}%
\pgfpathlineto{\pgfqpoint{3.918303in}{0.734092in}}%
\pgfpathlineto{\pgfqpoint{3.918713in}{0.749203in}}%
\pgfpathlineto{\pgfqpoint{3.928553in}{1.216455in}}%
\pgfpathlineto{\pgfqpoint{3.931013in}{1.252172in}}%
\pgfpathlineto{\pgfqpoint{3.931423in}{1.252818in}}%
\pgfpathlineto{\pgfqpoint{3.931833in}{1.251888in}}%
\pgfpathlineto{\pgfqpoint{3.933063in}{1.239626in}}%
\pgfpathlineto{\pgfqpoint{3.935113in}{1.188580in}}%
\pgfpathlineto{\pgfqpoint{3.938393in}{1.039762in}}%
\pgfpathlineto{\pgfqpoint{3.938803in}{1.065582in}}%
\pgfpathlineto{\pgfqpoint{3.944953in}{1.609786in}}%
\pgfpathlineto{\pgfqpoint{3.949053in}{1.804268in}}%
\pgfpathlineto{\pgfqpoint{3.951513in}{1.846335in}}%
\pgfpathlineto{\pgfqpoint{3.952333in}{1.847872in}}%
\pgfpathlineto{\pgfqpoint{3.953153in}{1.843264in}}%
\pgfpathlineto{\pgfqpoint{3.954793in}{1.816003in}}%
\pgfpathlineto{\pgfqpoint{3.957253in}{1.731959in}}%
\pgfpathlineto{\pgfqpoint{3.960943in}{1.518282in}}%
\pgfpathlineto{\pgfqpoint{3.966273in}{1.064727in}}%
\pgfpathlineto{\pgfqpoint{3.968323in}{0.885337in}}%
\pgfpathlineto{\pgfqpoint{3.968733in}{0.894710in}}%
\pgfpathlineto{\pgfqpoint{3.974063in}{0.991222in}}%
\pgfpathlineto{\pgfqpoint{3.977343in}{1.017911in}}%
\pgfpathlineto{\pgfqpoint{3.978163in}{1.019056in}}%
\pgfpathlineto{\pgfqpoint{3.978573in}{1.018641in}}%
\pgfpathlineto{\pgfqpoint{3.979803in}{1.013046in}}%
\pgfpathlineto{\pgfqpoint{3.981853in}{0.987014in}}%
\pgfpathlineto{\pgfqpoint{3.984313in}{0.922219in}}%
\pgfpathlineto{\pgfqpoint{3.987593in}{0.767666in}}%
\pgfpathlineto{\pgfqpoint{3.988413in}{0.717115in}}%
\pgfpathlineto{\pgfqpoint{3.988823in}{0.731284in}}%
\pgfpathlineto{\pgfqpoint{3.996613in}{1.144922in}}%
\pgfpathlineto{\pgfqpoint{3.998663in}{1.175727in}}%
\pgfpathlineto{\pgfqpoint{3.999073in}{1.175607in}}%
\pgfpathlineto{\pgfqpoint{3.999893in}{1.168893in}}%
\pgfpathlineto{\pgfqpoint{4.001533in}{1.130159in}}%
\pgfpathlineto{\pgfqpoint{4.004403in}{0.992568in}}%
\pgfpathlineto{\pgfqpoint{4.004813in}{1.016772in}}%
\pgfpathlineto{\pgfqpoint{4.010143in}{1.540976in}}%
\pgfpathlineto{\pgfqpoint{4.013423in}{1.701272in}}%
\pgfpathlineto{\pgfqpoint{4.015473in}{1.729910in}}%
\pgfpathlineto{\pgfqpoint{4.016293in}{1.726051in}}%
\pgfpathlineto{\pgfqpoint{4.017933in}{1.692931in}}%
\pgfpathlineto{\pgfqpoint{4.020393in}{1.583479in}}%
\pgfpathlineto{\pgfqpoint{4.024083in}{1.303549in}}%
\pgfpathlineto{\pgfqpoint{4.028593in}{0.863032in}}%
\pgfpathlineto{\pgfqpoint{4.029413in}{0.884048in}}%
\pgfpathlineto{\pgfqpoint{4.033923in}{0.968234in}}%
\pgfpathlineto{\pgfqpoint{4.036383in}{0.983104in}}%
\pgfpathlineto{\pgfqpoint{4.037203in}{0.981326in}}%
\pgfpathlineto{\pgfqpoint{4.038433in}{0.971023in}}%
\pgfpathlineto{\pgfqpoint{4.040483in}{0.929875in}}%
\pgfpathlineto{\pgfqpoint{4.043353in}{0.811403in}}%
\pgfpathlineto{\pgfqpoint{4.044993in}{0.708991in}}%
\pgfpathlineto{\pgfqpoint{4.045403in}{0.712501in}}%
\pgfpathlineto{\pgfqpoint{4.051963in}{1.086577in}}%
\pgfpathlineto{\pgfqpoint{4.054013in}{1.116679in}}%
\pgfpathlineto{\pgfqpoint{4.054833in}{1.110509in}}%
\pgfpathlineto{\pgfqpoint{4.056473in}{1.067185in}}%
\pgfpathlineto{\pgfqpoint{4.058933in}{0.962472in}}%
\pgfpathlineto{\pgfqpoint{4.063853in}{1.480803in}}%
\pgfpathlineto{\pgfqpoint{4.066723in}{1.618447in}}%
\pgfpathlineto{\pgfqpoint{4.068363in}{1.635281in}}%
\pgfpathlineto{\pgfqpoint{4.069183in}{1.627050in}}%
\pgfpathlineto{\pgfqpoint{4.070823in}{1.578649in}}%
\pgfpathlineto{\pgfqpoint{4.073693in}{1.400543in}}%
\pgfpathlineto{\pgfqpoint{4.078203in}{0.937387in}}%
\pgfpathlineto{\pgfqpoint{4.079023in}{0.844742in}}%
\pgfpathlineto{\pgfqpoint{4.079843in}{0.859824in}}%
\pgfpathlineto{\pgfqpoint{4.083943in}{0.942326in}}%
\pgfpathlineto{\pgfqpoint{4.085993in}{0.953440in}}%
\pgfpathlineto{\pgfqpoint{4.086813in}{0.950199in}}%
\pgfpathlineto{\pgfqpoint{4.088453in}{0.927623in}}%
\pgfpathlineto{\pgfqpoint{4.090503in}{0.863528in}}%
\pgfpathlineto{\pgfqpoint{4.093373in}{0.696781in}}%
\pgfpathlineto{\pgfqpoint{4.094193in}{0.731682in}}%
\pgfpathlineto{\pgfqpoint{4.099523in}{1.042901in}}%
\pgfpathlineto{\pgfqpoint{4.101163in}{1.069488in}}%
\pgfpathlineto{\pgfqpoint{4.101573in}{1.068583in}}%
\pgfpathlineto{\pgfqpoint{4.102803in}{1.047445in}}%
\pgfpathlineto{\pgfqpoint{4.104853in}{0.956343in}}%
\pgfpathlineto{\pgfqpoint{4.105263in}{0.931211in}}%
\pgfpathlineto{\pgfqpoint{4.105673in}{0.933846in}}%
\pgfpathlineto{\pgfqpoint{4.110183in}{1.428698in}}%
\pgfpathlineto{\pgfqpoint{4.113053in}{1.553254in}}%
\pgfpathlineto{\pgfqpoint{4.113873in}{1.558325in}}%
\pgfpathlineto{\pgfqpoint{4.114283in}{1.555832in}}%
\pgfpathlineto{\pgfqpoint{4.115513in}{1.528760in}}%
\pgfpathlineto{\pgfqpoint{4.117563in}{1.422250in}}%
\pgfpathlineto{\pgfqpoint{4.120843in}{1.118449in}}%
\pgfpathlineto{\pgfqpoint{4.123303in}{0.826334in}}%
\pgfpathlineto{\pgfqpoint{4.124123in}{0.845308in}}%
\pgfpathlineto{\pgfqpoint{4.127813in}{0.921099in}}%
\pgfpathlineto{\pgfqpoint{4.129043in}{0.928520in}}%
\pgfpathlineto{\pgfqpoint{4.129453in}{0.928440in}}%
\pgfpathlineto{\pgfqpoint{4.130273in}{0.923969in}}%
\pgfpathlineto{\pgfqpoint{4.131913in}{0.895387in}}%
\pgfpathlineto{\pgfqpoint{4.134373in}{0.794004in}}%
\pgfpathlineto{\pgfqpoint{4.136013in}{0.684090in}}%
\pgfpathlineto{\pgfqpoint{4.136423in}{0.697076in}}%
\pgfpathlineto{\pgfqpoint{4.141753in}{1.013509in}}%
\pgfpathlineto{\pgfqpoint{4.142983in}{1.030938in}}%
\pgfpathlineto{\pgfqpoint{4.143393in}{1.029869in}}%
\pgfpathlineto{\pgfqpoint{4.144623in}{1.005903in}}%
\pgfpathlineto{\pgfqpoint{4.146673in}{0.904391in}}%
\pgfpathlineto{\pgfqpoint{4.147083in}{0.911546in}}%
\pgfpathlineto{\pgfqpoint{4.151593in}{1.405672in}}%
\pgfpathlineto{\pgfqpoint{4.154053in}{1.492988in}}%
\pgfpathlineto{\pgfqpoint{4.154463in}{1.493711in}}%
\pgfpathlineto{\pgfqpoint{4.155283in}{1.483532in}}%
\pgfpathlineto{\pgfqpoint{4.156923in}{1.418618in}}%
\pgfpathlineto{\pgfqpoint{4.159793in}{1.179994in}}%
\pgfpathlineto{\pgfqpoint{4.163073in}{0.812943in}}%
\pgfpathlineto{\pgfqpoint{4.163893in}{0.838572in}}%
\pgfpathlineto{\pgfqpoint{4.167173in}{0.903308in}}%
\pgfpathlineto{\pgfqpoint{4.167993in}{0.907320in}}%
\pgfpathlineto{\pgfqpoint{4.168403in}{0.907038in}}%
\pgfpathlineto{\pgfqpoint{4.169223in}{0.901407in}}%
\pgfpathlineto{\pgfqpoint{4.170863in}{0.867024in}}%
\pgfpathlineto{\pgfqpoint{4.173323in}{0.747603in}}%
\pgfpathlineto{\pgfqpoint{4.174553in}{0.677113in}}%
\pgfpathlineto{\pgfqpoint{4.174963in}{0.708082in}}%
\pgfpathlineto{\pgfqpoint{4.179473in}{0.980380in}}%
\pgfpathlineto{\pgfqpoint{4.180703in}{0.998791in}}%
\pgfpathlineto{\pgfqpoint{4.181113in}{0.997309in}}%
\pgfpathlineto{\pgfqpoint{4.182343in}{0.969973in}}%
\pgfpathlineto{\pgfqpoint{4.184393in}{0.883066in}}%
\pgfpathlineto{\pgfqpoint{4.188493in}{1.351974in}}%
\pgfpathlineto{\pgfqpoint{4.190953in}{1.438928in}}%
\pgfpathlineto{\pgfqpoint{4.191363in}{1.437713in}}%
\pgfpathlineto{\pgfqpoint{4.192593in}{1.407969in}}%
\pgfpathlineto{\pgfqpoint{4.194643in}{1.277845in}}%
\pgfpathlineto{\pgfqpoint{4.198333in}{0.850109in}}%
\pgfpathlineto{\pgfqpoint{4.198743in}{0.800806in}}%
\pgfpathlineto{\pgfqpoint{4.199563in}{0.819250in}}%
\pgfpathlineto{\pgfqpoint{4.202843in}{0.886418in}}%
\pgfpathlineto{\pgfqpoint{4.203663in}{0.888910in}}%
\pgfpathlineto{\pgfqpoint{4.204073in}{0.887446in}}%
\pgfpathlineto{\pgfqpoint{4.205303in}{0.870804in}}%
\pgfpathlineto{\pgfqpoint{4.207353in}{0.795646in}}%
\pgfpathlineto{\pgfqpoint{4.209403in}{0.668356in}}%
\pgfpathlineto{\pgfqpoint{4.209813in}{0.695732in}}%
\pgfpathlineto{\pgfqpoint{4.214323in}{0.962046in}}%
\pgfpathlineto{\pgfqpoint{4.215143in}{0.971444in}}%
\pgfpathlineto{\pgfqpoint{4.215553in}{0.969855in}}%
\pgfpathlineto{\pgfqpoint{4.216783in}{0.940165in}}%
\pgfpathlineto{\pgfqpoint{4.218423in}{0.857387in}}%
\pgfpathlineto{\pgfqpoint{4.223753in}{1.378347in}}%
\pgfpathlineto{\pgfqpoint{4.224573in}{1.391481in}}%
\pgfpathlineto{\pgfqpoint{4.224983in}{1.390532in}}%
\pgfpathlineto{\pgfqpoint{4.226213in}{1.358481in}}%
\pgfpathlineto{\pgfqpoint{4.228263in}{1.215688in}}%
\pgfpathlineto{\pgfqpoint{4.231953in}{0.787182in}}%
\pgfpathlineto{\pgfqpoint{4.233183in}{0.827151in}}%
\pgfpathlineto{\pgfqpoint{4.235643in}{0.871262in}}%
\pgfpathlineto{\pgfqpoint{4.236053in}{0.872760in}}%
\pgfpathlineto{\pgfqpoint{4.236463in}{0.872260in}}%
\pgfpathlineto{\pgfqpoint{4.237693in}{0.857418in}}%
\pgfpathlineto{\pgfqpoint{4.239333in}{0.801748in}}%
\pgfpathlineto{\pgfqpoint{4.241383in}{0.668498in}}%
\pgfpathlineto{\pgfqpoint{4.242203in}{0.704603in}}%
\pgfpathlineto{\pgfqpoint{4.246303in}{0.941395in}}%
\pgfpathlineto{\pgfqpoint{4.247123in}{0.947706in}}%
\pgfpathlineto{\pgfqpoint{4.247533in}{0.944033in}}%
\pgfpathlineto{\pgfqpoint{4.248763in}{0.906500in}}%
\pgfpathlineto{\pgfqpoint{4.249993in}{0.840577in}}%
\pgfpathlineto{\pgfqpoint{4.255323in}{1.345846in}}%
\pgfpathlineto{\pgfqpoint{4.255733in}{1.350182in}}%
\pgfpathlineto{\pgfqpoint{4.256143in}{1.348999in}}%
\pgfpathlineto{\pgfqpoint{4.257373in}{1.313251in}}%
\pgfpathlineto{\pgfqpoint{4.259423in}{1.156152in}}%
\pgfpathlineto{\pgfqpoint{4.262703in}{0.781936in}}%
\pgfpathlineto{\pgfqpoint{4.263523in}{0.809684in}}%
\pgfpathlineto{\pgfqpoint{4.265983in}{0.857094in}}%
\pgfpathlineto{\pgfqpoint{4.266393in}{0.858493in}}%
\pgfpathlineto{\pgfqpoint{4.266803in}{0.857659in}}%
\pgfpathlineto{\pgfqpoint{4.268033in}{0.840235in}}%
\pgfpathlineto{\pgfqpoint{4.270083in}{0.753547in}}%
\pgfpathlineto{\pgfqpoint{4.271313in}{0.664833in}}%
\pgfpathlineto{\pgfqpoint{4.271723in}{0.665489in}}%
\pgfpathlineto{\pgfqpoint{4.275823in}{0.918554in}}%
\pgfpathlineto{\pgfqpoint{4.276643in}{0.927347in}}%
\pgfpathlineto{\pgfqpoint{4.277053in}{0.924437in}}%
\pgfpathlineto{\pgfqpoint{4.278283in}{0.887198in}}%
\pgfpathlineto{\pgfqpoint{4.279513in}{0.831078in}}%
\pgfpathlineto{\pgfqpoint{4.283613in}{1.284031in}}%
\pgfpathlineto{\pgfqpoint{4.284843in}{1.313905in}}%
\pgfpathlineto{\pgfqpoint{4.285253in}{1.311755in}}%
\pgfpathlineto{\pgfqpoint{4.286483in}{1.270262in}}%
\pgfpathlineto{\pgfqpoint{4.288533in}{1.096384in}}%
\pgfpathlineto{\pgfqpoint{4.290993in}{0.771228in}}%
\pgfpathlineto{\pgfqpoint{4.291813in}{0.791144in}}%
\pgfpathlineto{\pgfqpoint{4.294273in}{0.843911in}}%
\pgfpathlineto{\pgfqpoint{4.294683in}{0.845659in}}%
\pgfpathlineto{\pgfqpoint{4.295093in}{0.844972in}}%
\pgfpathlineto{\pgfqpoint{4.296323in}{0.826598in}}%
\pgfpathlineto{\pgfqpoint{4.297963in}{0.758547in}}%
\pgfpathlineto{\pgfqpoint{4.299603in}{0.652265in}}%
\pgfpathlineto{\pgfqpoint{4.300013in}{0.681951in}}%
\pgfpathlineto{\pgfqpoint{4.303703in}{0.902947in}}%
\pgfpathlineto{\pgfqpoint{4.304523in}{0.908759in}}%
\pgfpathlineto{\pgfqpoint{4.305343in}{0.893816in}}%
\pgfpathlineto{\pgfqpoint{4.306983in}{0.813558in}}%
\pgfpathlineto{\pgfqpoint{4.311903in}{1.280243in}}%
\pgfpathlineto{\pgfqpoint{4.312313in}{1.281094in}}%
\pgfpathlineto{\pgfqpoint{4.313133in}{1.263450in}}%
\pgfpathlineto{\pgfqpoint{4.314773in}{1.156021in}}%
\pgfpathlineto{\pgfqpoint{4.318053in}{0.759775in}}%
\pgfpathlineto{\pgfqpoint{4.319283in}{0.802030in}}%
\pgfpathlineto{\pgfqpoint{4.321333in}{0.834152in}}%
\pgfpathlineto{\pgfqpoint{4.321743in}{0.833335in}}%
\pgfpathlineto{\pgfqpoint{4.322973in}{0.813073in}}%
\pgfpathlineto{\pgfqpoint{4.325023in}{0.711280in}}%
\pgfpathlineto{\pgfqpoint{4.325843in}{0.648186in}}%
\pgfpathlineto{\pgfqpoint{4.326253in}{0.666214in}}%
\pgfpathlineto{\pgfqpoint{4.329943in}{0.888633in}}%
\pgfpathlineto{\pgfqpoint{4.330353in}{0.892946in}}%
\pgfpathlineto{\pgfqpoint{4.330763in}{0.891739in}}%
\pgfpathlineto{\pgfqpoint{4.331993in}{0.855708in}}%
\pgfpathlineto{\pgfqpoint{4.332813in}{0.807848in}}%
\pgfpathlineto{\pgfqpoint{4.333223in}{0.826388in}}%
\pgfpathlineto{\pgfqpoint{4.336503in}{1.213090in}}%
\pgfpathlineto{\pgfqpoint{4.337733in}{1.252076in}}%
\pgfpathlineto{\pgfqpoint{4.338143in}{1.250907in}}%
\pgfpathlineto{\pgfqpoint{4.339373in}{1.206374in}}%
\pgfpathlineto{\pgfqpoint{4.341423in}{1.011610in}}%
\pgfpathlineto{\pgfqpoint{4.343473in}{0.754137in}}%
\pgfpathlineto{\pgfqpoint{4.343883in}{0.769956in}}%
\pgfpathlineto{\pgfqpoint{4.346343in}{0.823267in}}%
\pgfpathlineto{\pgfqpoint{4.346753in}{0.823529in}}%
\pgfpathlineto{\pgfqpoint{4.347573in}{0.814844in}}%
\pgfpathlineto{\pgfqpoint{4.349213in}{0.755472in}}%
\pgfpathlineto{\pgfqpoint{4.350853in}{0.644253in}}%
\pgfpathlineto{\pgfqpoint{4.351263in}{0.668229in}}%
\pgfpathlineto{\pgfqpoint{4.354543in}{0.871631in}}%
\pgfpathlineto{\pgfqpoint{4.355363in}{0.878196in}}%
\pgfpathlineto{\pgfqpoint{4.356183in}{0.861534in}}%
\pgfpathlineto{\pgfqpoint{4.357413in}{0.797243in}}%
\pgfpathlineto{\pgfqpoint{4.357823in}{0.818160in}}%
\pgfpathlineto{\pgfqpoint{4.361103in}{1.198182in}}%
\pgfpathlineto{\pgfqpoint{4.362333in}{1.225921in}}%
\pgfpathlineto{\pgfqpoint{4.362743in}{1.220055in}}%
\pgfpathlineto{\pgfqpoint{4.363973in}{1.159369in}}%
\pgfpathlineto{\pgfqpoint{4.366433in}{0.876624in}}%
\pgfpathlineto{\pgfqpoint{4.367663in}{0.752879in}}%
\pgfpathlineto{\pgfqpoint{4.368073in}{0.768388in}}%
\pgfpathlineto{\pgfqpoint{4.370533in}{0.814214in}}%
\pgfpathlineto{\pgfqpoint{4.371353in}{0.806444in}}%
\pgfpathlineto{\pgfqpoint{4.372993in}{0.746245in}}%
\pgfpathlineto{\pgfqpoint{4.374633in}{0.640568in}}%
\pgfpathlineto{\pgfqpoint{4.375043in}{0.674166in}}%
\pgfpathlineto{\pgfqpoint{4.378323in}{0.863730in}}%
\pgfpathlineto{\pgfqpoint{4.378733in}{0.865635in}}%
\pgfpathlineto{\pgfqpoint{4.379553in}{0.851042in}}%
\pgfpathlineto{\pgfqpoint{4.380783in}{0.787399in}}%
\pgfpathlineto{\pgfqpoint{4.381193in}{0.811377in}}%
\pgfpathlineto{\pgfqpoint{4.384473in}{1.183512in}}%
\pgfpathlineto{\pgfqpoint{4.385293in}{1.202291in}}%
\pgfpathlineto{\pgfqpoint{4.385703in}{1.199555in}}%
\pgfpathlineto{\pgfqpoint{4.386933in}{1.144821in}}%
\pgfpathlineto{\pgfqpoint{4.389393in}{0.860808in}}%
\pgfpathlineto{\pgfqpoint{4.390213in}{0.742411in}}%
\pgfpathlineto{\pgfqpoint{4.391033in}{0.765268in}}%
\pgfpathlineto{\pgfqpoint{4.393083in}{0.805455in}}%
\pgfpathlineto{\pgfqpoint{4.393493in}{0.804574in}}%
\pgfpathlineto{\pgfqpoint{4.394723in}{0.779720in}}%
\pgfpathlineto{\pgfqpoint{4.397183in}{0.637345in}}%
\pgfpathlineto{\pgfqpoint{4.397593in}{0.672172in}}%
\pgfpathlineto{\pgfqpoint{4.400463in}{0.848160in}}%
\pgfpathlineto{\pgfqpoint{4.400873in}{0.853573in}}%
\pgfpathlineto{\pgfqpoint{4.401283in}{0.852545in}}%
\pgfpathlineto{\pgfqpoint{4.402513in}{0.811841in}}%
\pgfpathlineto{\pgfqpoint{4.403333in}{0.783085in}}%
\pgfpathlineto{\pgfqpoint{4.406613in}{1.163932in}}%
\pgfpathlineto{\pgfqpoint{4.407433in}{1.180637in}}%
\pgfpathlineto{\pgfqpoint{4.407843in}{1.176127in}}%
\pgfpathlineto{\pgfqpoint{4.409073in}{1.113563in}}%
\pgfpathlineto{\pgfqpoint{4.411533in}{0.808844in}}%
\pgfpathlineto{\pgfqpoint{4.412353in}{0.740230in}}%
\pgfpathlineto{\pgfqpoint{4.412763in}{0.756259in}}%
\pgfpathlineto{\pgfqpoint{4.414813in}{0.797505in}}%
\pgfpathlineto{\pgfqpoint{4.415223in}{0.796189in}}%
\pgfpathlineto{\pgfqpoint{4.416453in}{0.768386in}}%
\pgfpathlineto{\pgfqpoint{4.418503in}{0.634817in}}%
\pgfpathlineto{\pgfqpoint{4.419323in}{0.687302in}}%
\pgfpathlineto{\pgfqpoint{4.422193in}{0.842550in}}%
\pgfpathlineto{\pgfqpoint{4.422603in}{0.842113in}}%
\pgfpathlineto{\pgfqpoint{4.423833in}{0.801354in}}%
\pgfpathlineto{\pgfqpoint{4.424243in}{0.776128in}}%
\pgfpathlineto{\pgfqpoint{4.424653in}{0.784757in}}%
\pgfpathlineto{\pgfqpoint{4.427523in}{1.134179in}}%
\pgfpathlineto{\pgfqpoint{4.428343in}{1.159923in}}%
\pgfpathlineto{\pgfqpoint{4.428753in}{1.159128in}}%
\pgfpathlineto{\pgfqpoint{4.429983in}{1.104131in}}%
\pgfpathlineto{\pgfqpoint{4.432033in}{0.864507in}}%
\pgfpathlineto{\pgfqpoint{4.433263in}{0.736202in}}%
\pgfpathlineto{\pgfqpoint{4.433673in}{0.752250in}}%
\pgfpathlineto{\pgfqpoint{4.435723in}{0.790027in}}%
\pgfpathlineto{\pgfqpoint{4.436543in}{0.780038in}}%
\pgfpathlineto{\pgfqpoint{4.438183in}{0.706037in}}%
\pgfpathlineto{\pgfqpoint{4.439413in}{0.640217in}}%
\pgfpathlineto{\pgfqpoint{4.442693in}{0.832970in}}%
\pgfpathlineto{\pgfqpoint{4.443103in}{0.831709in}}%
\pgfpathlineto{\pgfqpoint{4.444333in}{0.787012in}}%
\pgfpathlineto{\pgfqpoint{4.444743in}{0.760255in}}%
\pgfpathlineto{\pgfqpoint{4.445153in}{0.800771in}}%
\pgfpathlineto{\pgfqpoint{4.448023in}{1.128349in}}%
\pgfpathlineto{\pgfqpoint{4.448843in}{1.141904in}}%
\pgfpathlineto{\pgfqpoint{4.449253in}{1.134344in}}%
\pgfpathlineto{\pgfqpoint{4.450483in}{1.057690in}}%
\pgfpathlineto{\pgfqpoint{4.452943in}{0.727478in}}%
\pgfpathlineto{\pgfqpoint{4.454173in}{0.763135in}}%
\pgfpathlineto{\pgfqpoint{4.455403in}{0.783087in}}%
\pgfpathlineto{\pgfqpoint{4.455813in}{0.782286in}}%
\pgfpathlineto{\pgfqpoint{4.456633in}{0.767637in}}%
\pgfpathlineto{\pgfqpoint{4.458273in}{0.680513in}}%
\pgfpathlineto{\pgfqpoint{4.459093in}{0.630019in}}%
\pgfpathlineto{\pgfqpoint{4.459503in}{0.666476in}}%
\pgfpathlineto{\pgfqpoint{4.462373in}{0.824125in}}%
\pgfpathlineto{\pgfqpoint{4.462783in}{0.821892in}}%
\pgfpathlineto{\pgfqpoint{4.464013in}{0.772922in}}%
\pgfpathlineto{\pgfqpoint{4.464423in}{0.757289in}}%
\pgfpathlineto{\pgfqpoint{4.467703in}{1.119912in}}%
\pgfpathlineto{\pgfqpoint{4.468113in}{1.125337in}}%
\pgfpathlineto{\pgfqpoint{4.468523in}{1.120678in}}%
\pgfpathlineto{\pgfqpoint{4.469753in}{1.049195in}}%
\pgfpathlineto{\pgfqpoint{4.472213in}{0.720816in}}%
\pgfpathlineto{\pgfqpoint{4.473443in}{0.759709in}}%
\pgfpathlineto{\pgfqpoint{4.474673in}{0.776888in}}%
\pgfpathlineto{\pgfqpoint{4.475083in}{0.774514in}}%
\pgfpathlineto{\pgfqpoint{4.476313in}{0.738671in}}%
\pgfpathlineto{\pgfqpoint{4.477953in}{0.627073in}}%
\pgfpathlineto{\pgfqpoint{4.478363in}{0.651948in}}%
\pgfpathlineto{\pgfqpoint{4.481233in}{0.815685in}}%
\pgfpathlineto{\pgfqpoint{4.481643in}{0.813631in}}%
\pgfpathlineto{\pgfqpoint{4.483283in}{0.754480in}}%
\pgfpathlineto{\pgfqpoint{4.486563in}{1.107870in}}%
\pgfpathlineto{\pgfqpoint{4.486973in}{1.108613in}}%
\pgfpathlineto{\pgfqpoint{4.487793in}{1.078914in}}%
\pgfpathlineto{\pgfqpoint{4.489432in}{0.909285in}}%
\pgfpathlineto{\pgfqpoint{4.490662in}{0.720589in}}%
\pgfpathlineto{\pgfqpoint{4.491482in}{0.740834in}}%
\pgfpathlineto{\pgfqpoint{4.493122in}{0.770898in}}%
\pgfpathlineto{\pgfqpoint{4.493942in}{0.760169in}}%
\pgfpathlineto{\pgfqpoint{4.495582in}{0.675803in}}%
\pgfpathlineto{\pgfqpoint{4.496402in}{0.626651in}}%
\pgfpathlineto{\pgfqpoint{4.496812in}{0.663728in}}%
\pgfpathlineto{\pgfqpoint{4.499272in}{0.806806in}}%
\pgfpathlineto{\pgfqpoint{4.499682in}{0.807411in}}%
\pgfpathlineto{\pgfqpoint{4.500502in}{0.784915in}}%
\pgfpathlineto{\pgfqpoint{4.501322in}{0.744955in}}%
\pgfpathlineto{\pgfqpoint{4.504602in}{1.093825in}}%
\pgfpathlineto{\pgfqpoint{4.505012in}{1.093247in}}%
\pgfpathlineto{\pgfqpoint{4.506242in}{1.027339in}}%
\pgfpathlineto{\pgfqpoint{4.508702in}{0.711159in}}%
\pgfpathlineto{\pgfqpoint{4.509932in}{0.754848in}}%
\pgfpathlineto{\pgfqpoint{4.510752in}{0.765292in}}%
\pgfpathlineto{\pgfqpoint{4.511162in}{0.763720in}}%
\pgfpathlineto{\pgfqpoint{4.512392in}{0.727550in}}%
\pgfpathlineto{\pgfqpoint{4.514032in}{0.623103in}}%
\pgfpathlineto{\pgfqpoint{4.514442in}{0.660535in}}%
\pgfpathlineto{\pgfqpoint{4.516902in}{0.800404in}}%
\pgfpathlineto{\pgfqpoint{4.517312in}{0.799164in}}%
\pgfpathlineto{\pgfqpoint{4.518542in}{0.747917in}}%
\pgfpathlineto{\pgfqpoint{4.518952in}{0.751912in}}%
\pgfpathlineto{\pgfqpoint{4.521822in}{1.077590in}}%
\pgfpathlineto{\pgfqpoint{4.522232in}{1.081020in}}%
\pgfpathlineto{\pgfqpoint{4.523052in}{1.053469in}}%
\pgfpathlineto{\pgfqpoint{4.524692in}{0.877251in}}%
\pgfpathlineto{\pgfqpoint{4.525922in}{0.707058in}}%
\pgfpathlineto{\pgfqpoint{4.526332in}{0.725072in}}%
\pgfpathlineto{\pgfqpoint{4.527972in}{0.760139in}}%
\pgfpathlineto{\pgfqpoint{4.528792in}{0.749598in}}%
\pgfpathlineto{\pgfqpoint{4.530432in}{0.659808in}}%
\pgfpathlineto{\pgfqpoint{4.530842in}{0.623451in}}%
\pgfpathlineto{\pgfqpoint{4.531252in}{0.637565in}}%
\pgfpathlineto{\pgfqpoint{4.533712in}{0.792216in}}%
\pgfpathlineto{\pgfqpoint{4.534122in}{0.793739in}}%
\pgfpathlineto{\pgfqpoint{4.534942in}{0.771199in}}%
\pgfpathlineto{\pgfqpoint{4.535762in}{0.739482in}}%
\pgfpathlineto{\pgfqpoint{4.538632in}{1.065337in}}%
\pgfpathlineto{\pgfqpoint{4.539042in}{1.067994in}}%
\pgfpathlineto{\pgfqpoint{4.539862in}{1.037439in}}%
\pgfpathlineto{\pgfqpoint{4.541502in}{0.851480in}}%
\pgfpathlineto{\pgfqpoint{4.542732in}{0.710827in}}%
\pgfpathlineto{\pgfqpoint{4.543142in}{0.727728in}}%
\pgfpathlineto{\pgfqpoint{4.544372in}{0.754761in}}%
\pgfpathlineto{\pgfqpoint{4.544782in}{0.754359in}}%
\pgfpathlineto{\pgfqpoint{4.545602in}{0.736826in}}%
\pgfpathlineto{\pgfqpoint{4.547652in}{0.628151in}}%
\pgfpathlineto{\pgfqpoint{4.548062in}{0.665835in}}%
\pgfpathlineto{\pgfqpoint{4.550522in}{0.787353in}}%
\pgfpathlineto{\pgfqpoint{4.551342in}{0.763554in}}%
\pgfpathlineto{\pgfqpoint{4.551752in}{0.739657in}}%
\pgfpathlineto{\pgfqpoint{4.552162in}{0.740468in}}%
\pgfpathlineto{\pgfqpoint{4.554622in}{1.045433in}}%
\pgfpathlineto{\pgfqpoint{4.555032in}{1.055851in}}%
\pgfpathlineto{\pgfqpoint{4.555442in}{1.053534in}}%
\pgfpathlineto{\pgfqpoint{4.556672in}{0.974117in}}%
\pgfpathlineto{\pgfqpoint{4.558722in}{0.701491in}}%
\pgfpathlineto{\pgfqpoint{4.559542in}{0.733993in}}%
\pgfpathlineto{\pgfqpoint{4.560772in}{0.750092in}}%
\pgfpathlineto{\pgfqpoint{4.561592in}{0.733472in}}%
\pgfpathlineto{\pgfqpoint{4.563232in}{0.625276in}}%
\pgfpathlineto{\pgfqpoint{4.564052in}{0.667193in}}%
\pgfpathlineto{\pgfqpoint{4.566102in}{0.781410in}}%
\pgfpathlineto{\pgfqpoint{4.566512in}{0.780111in}}%
\pgfpathlineto{\pgfqpoint{4.567742in}{0.723775in}}%
\pgfpathlineto{\pgfqpoint{4.571022in}{1.044533in}}%
\pgfpathlineto{\pgfqpoint{4.571432in}{1.033764in}}%
\pgfpathlineto{\pgfqpoint{4.572662in}{0.928816in}}%
\pgfpathlineto{\pgfqpoint{4.574302in}{0.696418in}}%
\pgfpathlineto{\pgfqpoint{4.575122in}{0.729736in}}%
\pgfpathlineto{\pgfqpoint{4.576352in}{0.745530in}}%
\pgfpathlineto{\pgfqpoint{4.577172in}{0.727342in}}%
\pgfpathlineto{\pgfqpoint{4.578812in}{0.616114in}}%
\pgfpathlineto{\pgfqpoint{4.579222in}{0.638299in}}%
\pgfpathlineto{\pgfqpoint{4.581682in}{0.776428in}}%
\pgfpathlineto{\pgfqpoint{4.582502in}{0.755412in}}%
\pgfpathlineto{\pgfqpoint{4.582912in}{0.731857in}}%
\pgfpathlineto{\pgfqpoint{4.583322in}{0.731946in}}%
\pgfpathlineto{\pgfqpoint{4.585782in}{1.029275in}}%
\pgfpathlineto{\pgfqpoint{4.586192in}{1.034509in}}%
\pgfpathlineto{\pgfqpoint{4.586602in}{1.026017in}}%
\pgfpathlineto{\pgfqpoint{4.587832in}{0.924218in}}%
\pgfpathlineto{\pgfqpoint{4.589472in}{0.694403in}}%
\pgfpathlineto{\pgfqpoint{4.590292in}{0.727425in}}%
\pgfpathlineto{\pgfqpoint{4.591112in}{0.741834in}}%
\pgfpathlineto{\pgfqpoint{4.591522in}{0.740658in}}%
\pgfpathlineto{\pgfqpoint{4.592342in}{0.719050in}}%
\pgfpathlineto{\pgfqpoint{4.593982in}{0.614543in}}%
\pgfpathlineto{\pgfqpoint{4.594392in}{0.651937in}}%
\pgfpathlineto{\pgfqpoint{4.596442in}{0.770796in}}%
\pgfpathlineto{\pgfqpoint{4.596852in}{0.768917in}}%
\pgfpathlineto{\pgfqpoint{4.598082in}{0.719585in}}%
\pgfpathlineto{\pgfqpoint{4.600952in}{1.024603in}}%
\pgfpathlineto{\pgfqpoint{4.601772in}{0.996218in}}%
\pgfpathlineto{\pgfqpoint{4.603412in}{0.793643in}}%
\pgfpathlineto{\pgfqpoint{4.604232in}{0.694080in}}%
\pgfpathlineto{\pgfqpoint{4.604642in}{0.712277in}}%
\pgfpathlineto{\pgfqpoint{4.605872in}{0.738199in}}%
\pgfpathlineto{\pgfqpoint{4.606282in}{0.735300in}}%
\pgfpathlineto{\pgfqpoint{4.607512in}{0.685278in}}%
\pgfpathlineto{\pgfqpoint{4.608332in}{0.617282in}}%
\pgfpathlineto{\pgfqpoint{4.608742in}{0.629170in}}%
\pgfpathlineto{\pgfqpoint{4.611202in}{0.765876in}}%
\pgfpathlineto{\pgfqpoint{4.612022in}{0.739523in}}%
\pgfpathlineto{\pgfqpoint{4.612432in}{0.712935in}}%
\pgfpathlineto{\pgfqpoint{4.612842in}{0.757347in}}%
\pgfpathlineto{\pgfqpoint{4.615302in}{1.015107in}}%
\pgfpathlineto{\pgfqpoint{4.616122in}{0.988070in}}%
\pgfpathlineto{\pgfqpoint{4.617762in}{0.782497in}}%
\pgfpathlineto{\pgfqpoint{4.618582in}{0.694148in}}%
\pgfpathlineto{\pgfqpoint{4.618992in}{0.711863in}}%
\pgfpathlineto{\pgfqpoint{4.620222in}{0.734339in}}%
\pgfpathlineto{\pgfqpoint{4.620632in}{0.729511in}}%
\pgfpathlineto{\pgfqpoint{4.621862in}{0.671589in}}%
\pgfpathlineto{\pgfqpoint{4.622682in}{0.611983in}}%
\pgfpathlineto{\pgfqpoint{4.623092in}{0.645174in}}%
\pgfpathlineto{\pgfqpoint{4.625142in}{0.761710in}}%
\pgfpathlineto{\pgfqpoint{4.625962in}{0.742470in}}%
\pgfpathlineto{\pgfqpoint{4.626372in}{0.718378in}}%
\pgfpathlineto{\pgfqpoint{4.626782in}{0.730900in}}%
\pgfpathlineto{\pgfqpoint{4.629242in}{1.005922in}}%
\pgfpathlineto{\pgfqpoint{4.629652in}{1.001152in}}%
\pgfpathlineto{\pgfqpoint{4.630882in}{0.900404in}}%
\pgfpathlineto{\pgfqpoint{4.632522in}{0.693469in}}%
\pgfpathlineto{\pgfqpoint{4.632932in}{0.710832in}}%
\pgfpathlineto{\pgfqpoint{4.634162in}{0.730329in}}%
\pgfpathlineto{\pgfqpoint{4.634982in}{0.709739in}}%
\pgfpathlineto{\pgfqpoint{4.636622in}{0.619422in}}%
\pgfpathlineto{\pgfqpoint{4.637032in}{0.658649in}}%
\pgfpathlineto{\pgfqpoint{4.639082in}{0.756212in}}%
\pgfpathlineto{\pgfqpoint{4.640312in}{0.711302in}}%
\pgfpathlineto{\pgfqpoint{4.642772in}{0.996508in}}%
\pgfpathlineto{\pgfqpoint{4.643182in}{0.994736in}}%
\pgfpathlineto{\pgfqpoint{4.644412in}{0.898930in}}%
\pgfpathlineto{\pgfqpoint{4.646052in}{0.691044in}}%
\pgfpathlineto{\pgfqpoint{4.646462in}{0.708469in}}%
\pgfpathlineto{\pgfqpoint{4.647282in}{0.726946in}}%
\pgfpathlineto{\pgfqpoint{4.647692in}{0.726556in}}%
\pgfpathlineto{\pgfqpoint{4.648512in}{0.703402in}}%
\pgfpathlineto{\pgfqpoint{4.649742in}{0.610451in}}%
\pgfpathlineto{\pgfqpoint{4.650152in}{0.629117in}}%
\pgfpathlineto{\pgfqpoint{4.652202in}{0.753130in}}%
\pgfpathlineto{\pgfqpoint{4.653022in}{0.733589in}}%
\pgfpathlineto{\pgfqpoint{4.653432in}{0.708494in}}%
\pgfpathlineto{\pgfqpoint{4.653842in}{0.736165in}}%
\pgfpathlineto{\pgfqpoint{4.656302in}{0.989017in}}%
\pgfpathlineto{\pgfqpoint{4.657122in}{0.947719in}}%
\pgfpathlineto{\pgfqpoint{4.659172in}{0.685849in}}%
\pgfpathlineto{\pgfqpoint{4.659992in}{0.716892in}}%
\pgfpathlineto{\pgfqpoint{4.660402in}{0.723619in}}%
\pgfpathlineto{\pgfqpoint{4.660812in}{0.723468in}}%
\pgfpathlineto{\pgfqpoint{4.661632in}{0.699992in}}%
\pgfpathlineto{\pgfqpoint{4.662862in}{0.608556in}}%
\pgfpathlineto{\pgfqpoint{4.663272in}{0.632624in}}%
\pgfpathlineto{\pgfqpoint{4.665322in}{0.749176in}}%
\pgfpathlineto{\pgfqpoint{4.666142in}{0.723346in}}%
\pgfpathlineto{\pgfqpoint{4.666552in}{0.701592in}}%
\pgfpathlineto{\pgfqpoint{4.669012in}{0.981582in}}%
\pgfpathlineto{\pgfqpoint{4.669422in}{0.976912in}}%
\pgfpathlineto{\pgfqpoint{4.670652in}{0.867374in}}%
\pgfpathlineto{\pgfqpoint{4.671882in}{0.677116in}}%
\pgfpathlineto{\pgfqpoint{4.672702in}{0.711100in}}%
\pgfpathlineto{\pgfqpoint{4.673522in}{0.721094in}}%
\pgfpathlineto{\pgfqpoint{4.673932in}{0.714925in}}%
\pgfpathlineto{\pgfqpoint{4.675162in}{0.646212in}}%
\pgfpathlineto{\pgfqpoint{4.675572in}{0.607473in}}%
\pgfpathlineto{\pgfqpoint{4.675982in}{0.628209in}}%
\pgfpathlineto{\pgfqpoint{4.678032in}{0.745338in}}%
\pgfpathlineto{\pgfqpoint{4.679262in}{0.702436in}}%
\pgfpathlineto{\pgfqpoint{4.681722in}{0.974725in}}%
\pgfpathlineto{\pgfqpoint{4.682542in}{0.938337in}}%
\pgfpathlineto{\pgfqpoint{4.684592in}{0.683967in}}%
\pgfpathlineto{\pgfqpoint{4.685412in}{0.713446in}}%
\pgfpathlineto{\pgfqpoint{4.685822in}{0.718411in}}%
\pgfpathlineto{\pgfqpoint{4.686232in}{0.715744in}}%
\pgfpathlineto{\pgfqpoint{4.687462in}{0.656496in}}%
\pgfpathlineto{\pgfqpoint{4.688282in}{0.613950in}}%
\pgfpathlineto{\pgfqpoint{4.690332in}{0.741993in}}%
\pgfpathlineto{\pgfqpoint{4.691152in}{0.717936in}}%
\pgfpathlineto{\pgfqpoint{4.691562in}{0.696800in}}%
\pgfpathlineto{\pgfqpoint{4.694022in}{0.967770in}}%
\pgfpathlineto{\pgfqpoint{4.694432in}{0.957343in}}%
\pgfpathlineto{\pgfqpoint{4.695662in}{0.827010in}}%
\pgfpathlineto{\pgfqpoint{4.696482in}{0.684334in}}%
\pgfpathlineto{\pgfqpoint{4.697302in}{0.702349in}}%
\pgfpathlineto{\pgfqpoint{4.698122in}{0.715702in}}%
\pgfpathlineto{\pgfqpoint{4.698532in}{0.710619in}}%
\pgfpathlineto{\pgfqpoint{4.699762in}{0.641936in}}%
\pgfpathlineto{\pgfqpoint{4.700172in}{0.605611in}}%
\pgfpathlineto{\pgfqpoint{4.700582in}{0.629805in}}%
\pgfpathlineto{\pgfqpoint{4.702222in}{0.737154in}}%
\pgfpathlineto{\pgfqpoint{4.702632in}{0.736865in}}%
\pgfpathlineto{\pgfqpoint{4.703452in}{0.699714in}}%
\pgfpathlineto{\pgfqpoint{4.703862in}{0.722784in}}%
\pgfpathlineto{\pgfqpoint{4.705912in}{0.960867in}}%
\pgfpathlineto{\pgfqpoint{4.706322in}{0.955061in}}%
\pgfpathlineto{\pgfqpoint{4.707552in}{0.833344in}}%
\pgfpathlineto{\pgfqpoint{4.708782in}{0.681559in}}%
\pgfpathlineto{\pgfqpoint{4.709192in}{0.698916in}}%
\pgfpathlineto{\pgfqpoint{4.710012in}{0.713076in}}%
\pgfpathlineto{\pgfqpoint{4.710422in}{0.708024in}}%
\pgfpathlineto{\pgfqpoint{4.711652in}{0.637721in}}%
\pgfpathlineto{\pgfqpoint{4.712062in}{0.604764in}}%
\pgfpathlineto{\pgfqpoint{4.712472in}{0.633064in}}%
\pgfpathlineto{\pgfqpoint{4.714112in}{0.734948in}}%
\pgfpathlineto{\pgfqpoint{4.714522in}{0.731788in}}%
\pgfpathlineto{\pgfqpoint{4.715342in}{0.688931in}}%
\pgfpathlineto{\pgfqpoint{4.717802in}{0.954430in}}%
\pgfpathlineto{\pgfqpoint{4.718212in}{0.940465in}}%
\pgfpathlineto{\pgfqpoint{4.719442in}{0.795352in}}%
\pgfpathlineto{\pgfqpoint{4.720262in}{0.670239in}}%
\pgfpathlineto{\pgfqpoint{4.720672in}{0.690273in}}%
\pgfpathlineto{\pgfqpoint{4.721492in}{0.710269in}}%
\pgfpathlineto{\pgfqpoint{4.721902in}{0.708223in}}%
\pgfpathlineto{\pgfqpoint{4.723132in}{0.645720in}}%
\pgfpathlineto{\pgfqpoint{4.723542in}{0.606620in}}%
\pgfpathlineto{\pgfqpoint{4.723952in}{0.622439in}}%
\pgfpathlineto{\pgfqpoint{4.725592in}{0.731323in}}%
\pgfpathlineto{\pgfqpoint{4.726002in}{0.729749in}}%
\pgfpathlineto{\pgfqpoint{4.726822in}{0.688376in}}%
\pgfpathlineto{\pgfqpoint{4.729282in}{0.948070in}}%
\pgfpathlineto{\pgfqpoint{4.729692in}{0.932406in}}%
\pgfpathlineto{\pgfqpoint{4.731742in}{0.673029in}}%
\pgfpathlineto{\pgfqpoint{4.733382in}{0.703423in}}%
\pgfpathlineto{\pgfqpoint{4.734612in}{0.630615in}}%
\pgfpathlineto{\pgfqpoint{4.735022in}{0.603116in}}%
\pgfpathlineto{\pgfqpoint{4.735432in}{0.638166in}}%
\pgfpathlineto{\pgfqpoint{4.737072in}{0.729289in}}%
\pgfpathlineto{\pgfqpoint{4.737892in}{0.699113in}}%
\pgfpathlineto{\pgfqpoint{4.738302in}{0.699383in}}%
\pgfpathlineto{\pgfqpoint{4.740352in}{0.943141in}}%
\pgfpathlineto{\pgfqpoint{4.741172in}{0.903701in}}%
\pgfpathlineto{\pgfqpoint{4.742812in}{0.667265in}}%
\pgfpathlineto{\pgfqpoint{4.743632in}{0.700604in}}%
\pgfpathlineto{\pgfqpoint{4.744042in}{0.705800in}}%
\pgfpathlineto{\pgfqpoint{4.744452in}{0.701906in}}%
\pgfpathlineto{\pgfqpoint{4.745682in}{0.629923in}}%
\pgfpathlineto{\pgfqpoint{4.746092in}{0.602361in}}%
\pgfpathlineto{\pgfqpoint{4.746502in}{0.637746in}}%
\pgfpathlineto{\pgfqpoint{4.748142in}{0.726210in}}%
\pgfpathlineto{\pgfqpoint{4.748962in}{0.692374in}}%
\pgfpathlineto{\pgfqpoint{4.749372in}{0.712278in}}%
\pgfpathlineto{\pgfqpoint{4.751422in}{0.936952in}}%
\pgfpathlineto{\pgfqpoint{4.752242in}{0.884911in}}%
\pgfpathlineto{\pgfqpoint{4.753882in}{0.674702in}}%
\pgfpathlineto{\pgfqpoint{4.754702in}{0.701911in}}%
\pgfpathlineto{\pgfqpoint{4.755112in}{0.702863in}}%
\pgfpathlineto{\pgfqpoint{4.755932in}{0.674579in}}%
\pgfpathlineto{\pgfqpoint{4.756752in}{0.605163in}}%
\pgfpathlineto{\pgfqpoint{4.757162in}{0.619825in}}%
\pgfpathlineto{\pgfqpoint{4.758802in}{0.723872in}}%
\pgfpathlineto{\pgfqpoint{4.759212in}{0.717717in}}%
\pgfpathlineto{\pgfqpoint{4.760032in}{0.692843in}}%
\pgfpathlineto{\pgfqpoint{4.762082in}{0.932255in}}%
\pgfpathlineto{\pgfqpoint{4.762902in}{0.884710in}}%
\pgfpathlineto{\pgfqpoint{4.764542in}{0.672713in}}%
\pgfpathlineto{\pgfqpoint{4.765362in}{0.699985in}}%
\pgfpathlineto{\pgfqpoint{4.765772in}{0.700509in}}%
\pgfpathlineto{\pgfqpoint{4.766592in}{0.670301in}}%
\pgfpathlineto{\pgfqpoint{4.767412in}{0.600911in}}%
\pgfpathlineto{\pgfqpoint{4.767822in}{0.625511in}}%
\pgfpathlineto{\pgfqpoint{4.769462in}{0.721159in}}%
\pgfpathlineto{\pgfqpoint{4.770282in}{0.687404in}}%
\pgfpathlineto{\pgfqpoint{4.770692in}{0.713037in}}%
\pgfpathlineto{\pgfqpoint{4.772332in}{0.924640in}}%
\pgfpathlineto{\pgfqpoint{4.772742in}{0.924356in}}%
\pgfpathlineto{\pgfqpoint{4.773972in}{0.797184in}}%
\pgfpathlineto{\pgfqpoint{4.774792in}{0.662216in}}%
\pgfpathlineto{\pgfqpoint{4.775612in}{0.694862in}}%
\pgfpathlineto{\pgfqpoint{4.776022in}{0.699428in}}%
\pgfpathlineto{\pgfqpoint{4.776432in}{0.693879in}}%
\pgfpathlineto{\pgfqpoint{4.778072in}{0.611233in}}%
\pgfpathlineto{\pgfqpoint{4.778482in}{0.653044in}}%
\pgfpathlineto{\pgfqpoint{4.779712in}{0.718887in}}%
\pgfpathlineto{\pgfqpoint{4.780122in}{0.712052in}}%
\pgfpathlineto{\pgfqpoint{4.780532in}{0.690599in}}%
\pgfpathlineto{\pgfqpoint{4.780942in}{0.696691in}}%
\pgfpathlineto{\pgfqpoint{4.782992in}{0.920200in}}%
\pgfpathlineto{\pgfqpoint{4.783812in}{0.855342in}}%
\pgfpathlineto{\pgfqpoint{4.785042in}{0.659968in}}%
\pgfpathlineto{\pgfqpoint{4.785862in}{0.693746in}}%
\pgfpathlineto{\pgfqpoint{4.786272in}{0.697364in}}%
\pgfpathlineto{\pgfqpoint{4.787092in}{0.672111in}}%
\pgfpathlineto{\pgfqpoint{4.787912in}{0.601966in}}%
\pgfpathlineto{\pgfqpoint{4.788322in}{0.619543in}}%
\pgfpathlineto{\pgfqpoint{4.789962in}{0.716133in}}%
\pgfpathlineto{\pgfqpoint{4.790782in}{0.678633in}}%
\pgfpathlineto{\pgfqpoint{4.792832in}{0.917737in}}%
\pgfpathlineto{\pgfqpoint{4.793242in}{0.908502in}}%
\pgfpathlineto{\pgfqpoint{4.794472in}{0.752685in}}%
\pgfpathlineto{\pgfqpoint{4.795292in}{0.670995in}}%
\pgfpathlineto{\pgfqpoint{4.795702in}{0.687631in}}%
\pgfpathlineto{\pgfqpoint{4.796112in}{0.695250in}}%
\pgfpathlineto{\pgfqpoint{4.796522in}{0.692466in}}%
\pgfpathlineto{\pgfqpoint{4.798162in}{0.605657in}}%
\pgfpathlineto{\pgfqpoint{4.798572in}{0.648472in}}%
\pgfpathlineto{\pgfqpoint{4.799802in}{0.714322in}}%
\pgfpathlineto{\pgfqpoint{4.800212in}{0.705840in}}%
\pgfpathlineto{\pgfqpoint{4.800622in}{0.682185in}}%
\pgfpathlineto{\pgfqpoint{4.801032in}{0.710011in}}%
\pgfpathlineto{\pgfqpoint{4.802672in}{0.913034in}}%
\pgfpathlineto{\pgfqpoint{4.803082in}{0.905255in}}%
\pgfpathlineto{\pgfqpoint{4.804312in}{0.749571in}}%
\pgfpathlineto{\pgfqpoint{4.804722in}{0.670438in}}%
\pgfpathlineto{\pgfqpoint{4.805542in}{0.686741in}}%
\pgfpathlineto{\pgfqpoint{4.805952in}{0.693633in}}%
\pgfpathlineto{\pgfqpoint{4.806362in}{0.689690in}}%
\pgfpathlineto{\pgfqpoint{4.807592in}{0.606344in}}%
\pgfpathlineto{\pgfqpoint{4.808002in}{0.613222in}}%
\pgfpathlineto{\pgfqpoint{4.809642in}{0.711490in}}%
\pgfpathlineto{\pgfqpoint{4.810462in}{0.674073in}}%
\pgfpathlineto{\pgfqpoint{4.812512in}{0.908597in}}%
\pgfpathlineto{\pgfqpoint{4.812922in}{0.890544in}}%
\pgfpathlineto{\pgfqpoint{4.814562in}{0.658048in}}%
\pgfpathlineto{\pgfqpoint{4.815792in}{0.691068in}}%
\pgfpathlineto{\pgfqpoint{4.816612in}{0.656854in}}%
\pgfpathlineto{\pgfqpoint{4.817432in}{0.597905in}}%
\pgfpathlineto{\pgfqpoint{4.817842in}{0.640661in}}%
\pgfpathlineto{\pgfqpoint{4.819072in}{0.710110in}}%
\pgfpathlineto{\pgfqpoint{4.819482in}{0.701384in}}%
\pgfpathlineto{\pgfqpoint{4.819892in}{0.676740in}}%
\pgfpathlineto{\pgfqpoint{4.820302in}{0.715005in}}%
\pgfpathlineto{\pgfqpoint{4.821942in}{0.905043in}}%
\pgfpathlineto{\pgfqpoint{4.822762in}{0.850631in}}%
\pgfpathlineto{\pgfqpoint{4.823992in}{0.655687in}}%
\pgfpathlineto{\pgfqpoint{4.824812in}{0.688383in}}%
\pgfpathlineto{\pgfqpoint{4.825222in}{0.689315in}}%
\pgfpathlineto{\pgfqpoint{4.826042in}{0.653853in}}%
\pgfpathlineto{\pgfqpoint{4.826862in}{0.599968in}}%
\pgfpathlineto{\pgfqpoint{4.828502in}{0.707856in}}%
\pgfpathlineto{\pgfqpoint{4.829322in}{0.670730in}}%
\pgfpathlineto{\pgfqpoint{4.831372in}{0.899338in}}%
\pgfpathlineto{\pgfqpoint{4.831782in}{0.876290in}}%
\pgfpathlineto{\pgfqpoint{4.833422in}{0.663715in}}%
\pgfpathlineto{\pgfqpoint{4.834242in}{0.688608in}}%
\pgfpathlineto{\pgfqpoint{4.834652in}{0.684139in}}%
\pgfpathlineto{\pgfqpoint{4.835882in}{0.596948in}}%
\pgfpathlineto{\pgfqpoint{4.836292in}{0.623321in}}%
\pgfpathlineto{\pgfqpoint{4.837522in}{0.705762in}}%
\pgfpathlineto{\pgfqpoint{4.837932in}{0.700809in}}%
\pgfpathlineto{\pgfqpoint{4.838342in}{0.678626in}}%
\pgfpathlineto{\pgfqpoint{4.838752in}{0.699620in}}%
\pgfpathlineto{\pgfqpoint{4.840392in}{0.897312in}}%
\pgfpathlineto{\pgfqpoint{4.841212in}{0.839044in}}%
\pgfpathlineto{\pgfqpoint{4.842442in}{0.658317in}}%
\pgfpathlineto{\pgfqpoint{4.843262in}{0.686700in}}%
\pgfpathlineto{\pgfqpoint{4.843672in}{0.683854in}}%
\pgfpathlineto{\pgfqpoint{4.844902in}{0.596399in}}%
\pgfpathlineto{\pgfqpoint{4.845312in}{0.620218in}}%
\pgfpathlineto{\pgfqpoint{4.846542in}{0.703976in}}%
\pgfpathlineto{\pgfqpoint{4.846952in}{0.698685in}}%
\pgfpathlineto{\pgfqpoint{4.847362in}{0.675781in}}%
\pgfpathlineto{\pgfqpoint{4.847772in}{0.703554in}}%
\pgfpathlineto{\pgfqpoint{4.849412in}{0.893098in}}%
\pgfpathlineto{\pgfqpoint{4.850232in}{0.826414in}}%
\pgfpathlineto{\pgfqpoint{4.851462in}{0.662291in}}%
\pgfpathlineto{\pgfqpoint{4.851872in}{0.679463in}}%
\pgfpathlineto{\pgfqpoint{4.852282in}{0.685681in}}%
\pgfpathlineto{\pgfqpoint{4.852692in}{0.679189in}}%
\pgfpathlineto{\pgfqpoint{4.853922in}{0.596131in}}%
\pgfpathlineto{\pgfqpoint{4.854332in}{0.635010in}}%
\pgfpathlineto{\pgfqpoint{4.855562in}{0.702295in}}%
\pgfpathlineto{\pgfqpoint{4.855972in}{0.689146in}}%
\pgfpathlineto{\pgfqpoint{4.856382in}{0.671737in}}%
\pgfpathlineto{\pgfqpoint{4.858022in}{0.889447in}}%
\pgfpathlineto{\pgfqpoint{4.858432in}{0.882100in}}%
\pgfpathlineto{\pgfqpoint{4.862532in}{0.595635in}}%
\pgfpathlineto{\pgfqpoint{4.864172in}{0.700989in}}%
\pgfpathlineto{\pgfqpoint{4.864582in}{0.691305in}}%
\pgfpathlineto{\pgfqpoint{4.864992in}{0.665355in}}%
\pgfpathlineto{\pgfqpoint{4.866632in}{0.885329in}}%
\pgfpathlineto{\pgfqpoint{4.867042in}{0.880366in}}%
\pgfpathlineto{\pgfqpoint{4.868272in}{0.706719in}}%
\pgfpathlineto{\pgfqpoint{4.868682in}{0.650959in}}%
\pgfpathlineto{\pgfqpoint{4.869502in}{0.682233in}}%
\pgfpathlineto{\pgfqpoint{4.869912in}{0.679526in}}%
\pgfpathlineto{\pgfqpoint{4.871142in}{0.595335in}}%
\pgfpathlineto{\pgfqpoint{4.871552in}{0.627657in}}%
\pgfpathlineto{\pgfqpoint{4.872782in}{0.699076in}}%
\pgfpathlineto{\pgfqpoint{4.873192in}{0.685708in}}%
\pgfpathlineto{\pgfqpoint{4.873602in}{0.671328in}}%
\pgfpathlineto{\pgfqpoint{4.875242in}{0.883767in}}%
\pgfpathlineto{\pgfqpoint{4.875652in}{0.869960in}}%
\pgfpathlineto{\pgfqpoint{4.877292in}{0.658867in}}%
\pgfpathlineto{\pgfqpoint{4.878522in}{0.672738in}}%
\pgfpathlineto{\pgfqpoint{4.879752in}{0.602939in}}%
\pgfpathlineto{\pgfqpoint{4.880982in}{0.696982in}}%
\pgfpathlineto{\pgfqpoint{4.881392in}{0.692574in}}%
\pgfpathlineto{\pgfqpoint{4.881802in}{0.668659in}}%
\pgfpathlineto{\pgfqpoint{4.882212in}{0.708313in}}%
\pgfpathlineto{\pgfqpoint{4.883442in}{0.878174in}}%
\pgfpathlineto{\pgfqpoint{4.883852in}{0.875058in}}%
\pgfpathlineto{\pgfqpoint{4.887952in}{0.594442in}}%
\pgfpathlineto{\pgfqpoint{4.889182in}{0.694711in}}%
\pgfpathlineto{\pgfqpoint{4.889592in}{0.692433in}}%
\pgfpathlineto{\pgfqpoint{4.890002in}{0.669913in}}%
\pgfpathlineto{\pgfqpoint{4.890412in}{0.701066in}}%
\pgfpathlineto{\pgfqpoint{4.891642in}{0.875113in}}%
\pgfpathlineto{\pgfqpoint{4.892052in}{0.872009in}}%
\pgfpathlineto{\pgfqpoint{4.896152in}{0.602626in}}%
\pgfpathlineto{\pgfqpoint{4.897382in}{0.694661in}}%
\pgfpathlineto{\pgfqpoint{4.897792in}{0.687219in}}%
\pgfpathlineto{\pgfqpoint{4.898202in}{0.661022in}}%
\pgfpathlineto{\pgfqpoint{4.899842in}{0.874853in}}%
\pgfpathlineto{\pgfqpoint{4.900252in}{0.860565in}}%
\pgfpathlineto{\pgfqpoint{4.901482in}{0.659585in}}%
\pgfpathlineto{\pgfqpoint{4.903122in}{0.661589in}}%
\pgfpathlineto{\pgfqpoint{4.903942in}{0.593957in}}%
\pgfpathlineto{\pgfqpoint{4.904352in}{0.626733in}}%
\pgfpathlineto{\pgfqpoint{4.905582in}{0.691646in}}%
\pgfpathlineto{\pgfqpoint{4.905992in}{0.671915in}}%
\pgfpathlineto{\pgfqpoint{4.906402in}{0.687660in}}%
\pgfpathlineto{\pgfqpoint{4.907632in}{0.869454in}}%
\pgfpathlineto{\pgfqpoint{4.908042in}{0.866069in}}%
\pgfpathlineto{\pgfqpoint{4.911732in}{0.593426in}}%
\pgfpathlineto{\pgfqpoint{4.915422in}{0.866580in}}%
\pgfpathlineto{\pgfqpoint{4.916242in}{0.826733in}}%
\pgfpathlineto{\pgfqpoint{4.917472in}{0.654760in}}%
\pgfpathlineto{\pgfqpoint{4.917882in}{0.671852in}}%
\pgfpathlineto{\pgfqpoint{4.918292in}{0.674844in}}%
\pgfpathlineto{\pgfqpoint{4.919112in}{0.630875in}}%
\pgfpathlineto{\pgfqpoint{4.919522in}{0.593367in}}%
\pgfpathlineto{\pgfqpoint{4.919932in}{0.625874in}}%
\pgfpathlineto{\pgfqpoint{4.920752in}{0.688195in}}%
\pgfpathlineto{\pgfqpoint{4.921162in}{0.687911in}}%
\pgfpathlineto{\pgfqpoint{4.921572in}{0.665155in}}%
\pgfpathlineto{\pgfqpoint{4.921982in}{0.702664in}}%
\pgfpathlineto{\pgfqpoint{4.923212in}{0.866832in}}%
\pgfpathlineto{\pgfqpoint{4.923622in}{0.853116in}}%
\pgfpathlineto{\pgfqpoint{4.926492in}{0.649545in}}%
\pgfpathlineto{\pgfqpoint{4.927312in}{0.599149in}}%
\pgfpathlineto{\pgfqpoint{4.928542in}{0.689779in}}%
\pgfpathlineto{\pgfqpoint{4.928952in}{0.677369in}}%
\pgfpathlineto{\pgfqpoint{4.929362in}{0.668344in}}%
\pgfpathlineto{\pgfqpoint{4.931002in}{0.860300in}}%
\pgfpathlineto{\pgfqpoint{4.934692in}{0.592877in}}%
\pgfpathlineto{\pgfqpoint{4.935922in}{0.688280in}}%
\pgfpathlineto{\pgfqpoint{4.936332in}{0.680166in}}%
\pgfpathlineto{\pgfqpoint{4.936742in}{0.662257in}}%
\pgfpathlineto{\pgfqpoint{4.938382in}{0.859584in}}%
\pgfpathlineto{\pgfqpoint{4.939202in}{0.762090in}}%
\pgfpathlineto{\pgfqpoint{4.940022in}{0.652448in}}%
\pgfpathlineto{\pgfqpoint{4.940432in}{0.669508in}}%
\pgfpathlineto{\pgfqpoint{4.940842in}{0.671145in}}%
\pgfpathlineto{\pgfqpoint{4.942072in}{0.592604in}}%
\pgfpathlineto{\pgfqpoint{4.942482in}{0.637553in}}%
\pgfpathlineto{\pgfqpoint{4.943302in}{0.687459in}}%
\pgfpathlineto{\pgfqpoint{4.943712in}{0.676748in}}%
\pgfpathlineto{\pgfqpoint{4.944122in}{0.665949in}}%
\pgfpathlineto{\pgfqpoint{4.945352in}{0.856428in}}%
\pgfpathlineto{\pgfqpoint{4.945762in}{0.853983in}}%
\pgfpathlineto{\pgfqpoint{4.949042in}{0.604965in}}%
\pgfpathlineto{\pgfqpoint{4.949452in}{0.605187in}}%
\pgfpathlineto{\pgfqpoint{4.950682in}{0.685233in}}%
\pgfpathlineto{\pgfqpoint{4.951092in}{0.665006in}}%
\pgfpathlineto{\pgfqpoint{4.951912in}{0.781949in}}%
\pgfpathlineto{\pgfqpoint{4.952732in}{0.857988in}}%
\pgfpathlineto{\pgfqpoint{4.953142in}{0.837450in}}%
\pgfpathlineto{\pgfqpoint{4.954372in}{0.645197in}}%
\pgfpathlineto{\pgfqpoint{4.955192in}{0.670210in}}%
\pgfpathlineto{\pgfqpoint{4.956422in}{0.592267in}}%
\pgfpathlineto{\pgfqpoint{4.956832in}{0.634595in}}%
\pgfpathlineto{\pgfqpoint{4.957652in}{0.685425in}}%
\pgfpathlineto{\pgfqpoint{4.958062in}{0.673409in}}%
\pgfpathlineto{\pgfqpoint{4.958472in}{0.667697in}}%
\pgfpathlineto{\pgfqpoint{4.959702in}{0.854767in}}%
\pgfpathlineto{\pgfqpoint{4.960112in}{0.844379in}}%
\pgfpathlineto{\pgfqpoint{4.962982in}{0.628575in}}%
\pgfpathlineto{\pgfqpoint{4.963392in}{0.592088in}}%
\pgfpathlineto{\pgfqpoint{4.963802in}{0.628685in}}%
\pgfpathlineto{\pgfqpoint{4.964622in}{0.684272in}}%
\pgfpathlineto{\pgfqpoint{4.965032in}{0.674111in}}%
\pgfpathlineto{\pgfqpoint{4.965442in}{0.665440in}}%
\pgfpathlineto{\pgfqpoint{4.966672in}{0.852763in}}%
\pgfpathlineto{\pgfqpoint{4.967082in}{0.842232in}}%
\pgfpathlineto{\pgfqpoint{4.969952in}{0.623132in}}%
\pgfpathlineto{\pgfqpoint{4.970362in}{0.591924in}}%
\pgfpathlineto{\pgfqpoint{4.970772in}{0.634926in}}%
\pgfpathlineto{\pgfqpoint{4.971592in}{0.683503in}}%
\pgfpathlineto{\pgfqpoint{4.972002in}{0.668300in}}%
\pgfpathlineto{\pgfqpoint{4.972412in}{0.674150in}}%
\pgfpathlineto{\pgfqpoint{4.973642in}{0.852155in}}%
\pgfpathlineto{\pgfqpoint{4.974052in}{0.830691in}}%
\pgfpathlineto{\pgfqpoint{4.975282in}{0.647557in}}%
\pgfpathlineto{\pgfqpoint{4.976102in}{0.665955in}}%
\pgfpathlineto{\pgfqpoint{4.977332in}{0.604384in}}%
\pgfpathlineto{\pgfqpoint{4.978562in}{0.679249in}}%
\pgfpathlineto{\pgfqpoint{4.978972in}{0.652487in}}%
\pgfpathlineto{\pgfqpoint{4.980202in}{0.845250in}}%
\pgfpathlineto{\pgfqpoint{4.980612in}{0.844543in}}%
\pgfpathlineto{\pgfqpoint{4.983892in}{0.591626in}}%
\pgfpathlineto{\pgfqpoint{4.985122in}{0.681403in}}%
\pgfpathlineto{\pgfqpoint{4.985532in}{0.662409in}}%
\pgfpathlineto{\pgfqpoint{4.986352in}{0.782476in}}%
\pgfpathlineto{\pgfqpoint{4.987172in}{0.847052in}}%
\pgfpathlineto{\pgfqpoint{4.987582in}{0.813466in}}%
\pgfpathlineto{\pgfqpoint{4.988402in}{0.652344in}}%
\pgfpathlineto{\pgfqpoint{4.989222in}{0.666573in}}%
\pgfpathlineto{\pgfqpoint{4.990042in}{0.627706in}}%
\pgfpathlineto{\pgfqpoint{4.990452in}{0.591534in}}%
\pgfpathlineto{\pgfqpoint{4.990862in}{0.630577in}}%
\pgfpathlineto{\pgfqpoint{4.993732in}{0.845498in}}%
\pgfpathlineto{\pgfqpoint{4.994552in}{0.738161in}}%
\pgfpathlineto{\pgfqpoint{4.994962in}{0.650489in}}%
\pgfpathlineto{\pgfqpoint{4.995782in}{0.665971in}}%
\pgfpathlineto{\pgfqpoint{4.996602in}{0.623010in}}%
\pgfpathlineto{\pgfqpoint{4.997012in}{0.591412in}}%
\pgfpathlineto{\pgfqpoint{4.997422in}{0.636092in}}%
\pgfpathlineto{\pgfqpoint{5.000292in}{0.839980in}}%
\pgfpathlineto{\pgfqpoint{5.003572in}{0.602521in}}%
\pgfpathlineto{\pgfqpoint{5.004392in}{0.677340in}}%
\pgfpathlineto{\pgfqpoint{5.004802in}{0.673312in}}%
\pgfpathlineto{\pgfqpoint{5.005212in}{0.659621in}}%
\pgfpathlineto{\pgfqpoint{5.006442in}{0.843974in}}%
\pgfpathlineto{\pgfqpoint{5.006852in}{0.823116in}}%
\pgfpathlineto{\pgfqpoint{5.008082in}{0.649894in}}%
\pgfpathlineto{\pgfqpoint{5.008902in}{0.657461in}}%
\pgfpathlineto{\pgfqpoint{5.009722in}{0.591268in}}%
\pgfpathlineto{\pgfqpoint{5.010132in}{0.632414in}}%
\pgfpathlineto{\pgfqpoint{5.012592in}{0.839009in}}%
\pgfpathlineto{\pgfqpoint{5.013002in}{0.833374in}}%
\pgfpathlineto{\pgfqpoint{5.015872in}{0.590873in}}%
\pgfpathlineto{\pgfqpoint{5.018742in}{0.835244in}}%
\pgfpathlineto{\pgfqpoint{5.019152in}{0.834818in}}%
\pgfpathlineto{\pgfqpoint{5.022022in}{0.590822in}}%
\pgfpathlineto{\pgfqpoint{5.024892in}{0.836279in}}%
\pgfpathlineto{\pgfqpoint{5.025302in}{0.830350in}}%
\pgfpathlineto{\pgfqpoint{5.028172in}{0.591121in}}%
\pgfpathlineto{\pgfqpoint{5.031042in}{0.838815in}}%
\pgfpathlineto{\pgfqpoint{5.031452in}{0.816359in}}%
\pgfpathlineto{\pgfqpoint{5.033502in}{0.647921in}}%
\pgfpathlineto{\pgfqpoint{5.034322in}{0.602496in}}%
\pgfpathlineto{\pgfqpoint{5.037192in}{0.832388in}}%
\pgfpathlineto{\pgfqpoint{5.040062in}{0.591090in}}%
\pgfpathlineto{\pgfqpoint{5.042932in}{0.836040in}}%
\pgfpathlineto{\pgfqpoint{5.045802in}{0.590907in}}%
\pgfpathlineto{\pgfqpoint{5.046212in}{0.627295in}}%
\pgfpathlineto{\pgfqpoint{5.048672in}{0.835667in}}%
\pgfpathlineto{\pgfqpoint{5.049082in}{0.803613in}}%
\pgfpathlineto{\pgfqpoint{5.049902in}{0.639650in}}%
\pgfpathlineto{\pgfqpoint{5.050722in}{0.658807in}}%
\pgfpathlineto{\pgfqpoint{5.051542in}{0.590940in}}%
\pgfpathlineto{\pgfqpoint{5.051952in}{0.628130in}}%
\pgfpathlineto{\pgfqpoint{5.054412in}{0.834003in}}%
\pgfpathlineto{\pgfqpoint{5.057282in}{0.591135in}}%
\pgfpathlineto{\pgfqpoint{5.060152in}{0.827621in}}%
\pgfpathlineto{\pgfqpoint{5.062612in}{0.606630in}}%
\pgfpathlineto{\pgfqpoint{5.063022in}{0.604799in}}%
\pgfpathlineto{\pgfqpoint{5.065482in}{0.833212in}}%
\pgfpathlineto{\pgfqpoint{5.066302in}{0.727046in}}%
\pgfpathlineto{\pgfqpoint{5.066712in}{0.640441in}}%
\pgfpathlineto{\pgfqpoint{5.067532in}{0.656959in}}%
\pgfpathlineto{\pgfqpoint{5.068352in}{0.591250in}}%
\pgfpathlineto{\pgfqpoint{5.068762in}{0.636651in}}%
\pgfpathlineto{\pgfqpoint{5.070812in}{0.827251in}}%
\pgfpathlineto{\pgfqpoint{5.071222in}{0.821930in}}%
\pgfpathlineto{\pgfqpoint{5.073682in}{0.590700in}}%
\pgfpathlineto{\pgfqpoint{5.076552in}{0.826579in}}%
\pgfpathlineto{\pgfqpoint{5.076962in}{0.770974in}}%
\pgfpathlineto{\pgfqpoint{5.077782in}{0.646520in}}%
\pgfpathlineto{\pgfqpoint{5.078192in}{0.659192in}}%
\pgfpathlineto{\pgfqpoint{5.079012in}{0.594455in}}%
\pgfpathlineto{\pgfqpoint{5.079422in}{0.618472in}}%
\pgfpathlineto{\pgfqpoint{5.081882in}{0.825979in}}%
\pgfpathlineto{\pgfqpoint{5.084342in}{0.590814in}}%
\pgfpathlineto{\pgfqpoint{5.086802in}{0.824675in}}%
\pgfpathlineto{\pgfqpoint{5.087212in}{0.820013in}}%
\pgfpathlineto{\pgfqpoint{5.089672in}{0.591610in}}%
\pgfpathlineto{\pgfqpoint{5.092132in}{0.830090in}}%
\pgfpathlineto{\pgfqpoint{5.092542in}{0.803151in}}%
\pgfpathlineto{\pgfqpoint{5.093362in}{0.634050in}}%
\pgfpathlineto{\pgfqpoint{5.094182in}{0.650639in}}%
\pgfpathlineto{\pgfqpoint{5.095002in}{0.603424in}}%
\pgfpathlineto{\pgfqpoint{5.097462in}{0.824692in}}%
\pgfpathlineto{\pgfqpoint{5.099922in}{0.591859in}}%
\pgfpathlineto{\pgfqpoint{5.102382in}{0.829656in}}%
\pgfpathlineto{\pgfqpoint{5.102792in}{0.791891in}}%
\pgfpathlineto{\pgfqpoint{5.103612in}{0.640657in}}%
\pgfpathlineto{\pgfqpoint{5.104432in}{0.642931in}}%
\pgfpathlineto{\pgfqpoint{5.104842in}{0.592373in}}%
\pgfpathlineto{\pgfqpoint{5.105252in}{0.623070in}}%
\pgfpathlineto{\pgfqpoint{5.107302in}{0.828525in}}%
\pgfpathlineto{\pgfqpoint{5.107712in}{0.803719in}}%
\pgfpathlineto{\pgfqpoint{5.108532in}{0.635250in}}%
\pgfpathlineto{\pgfqpoint{5.109352in}{0.646943in}}%
\pgfpathlineto{\pgfqpoint{5.109762in}{0.600258in}}%
\pgfpathlineto{\pgfqpoint{5.110172in}{0.616194in}}%
\pgfpathlineto{\pgfqpoint{5.112222in}{0.827698in}}%
\pgfpathlineto{\pgfqpoint{5.112632in}{0.805513in}}%
\pgfpathlineto{\pgfqpoint{5.114682in}{0.598369in}}%
\pgfpathlineto{\pgfqpoint{5.116322in}{0.669830in}}%
\pgfpathlineto{\pgfqpoint{5.117142in}{0.828730in}}%
\pgfpathlineto{\pgfqpoint{5.117552in}{0.798106in}}%
\pgfpathlineto{\pgfqpoint{5.118372in}{0.639441in}}%
\pgfpathlineto{\pgfqpoint{5.119192in}{0.640897in}}%
\pgfpathlineto{\pgfqpoint{5.119602in}{0.592001in}}%
\pgfpathlineto{\pgfqpoint{5.120012in}{0.630402in}}%
\pgfpathlineto{\pgfqpoint{5.122062in}{0.827998in}}%
\pgfpathlineto{\pgfqpoint{5.122472in}{0.777861in}}%
\pgfpathlineto{\pgfqpoint{5.124111in}{0.627505in}}%
\pgfpathlineto{\pgfqpoint{5.124521in}{0.592705in}}%
\pgfpathlineto{\pgfqpoint{5.124931in}{0.648608in}}%
\pgfpathlineto{\pgfqpoint{5.126161in}{0.745849in}}%
\pgfpathlineto{\pgfqpoint{5.126571in}{0.820838in}}%
\pgfpathlineto{\pgfqpoint{5.126981in}{0.816595in}}%
\pgfpathlineto{\pgfqpoint{5.129031in}{0.600746in}}%
\pgfpathlineto{\pgfqpoint{5.130671in}{0.685353in}}%
\pgfpathlineto{\pgfqpoint{5.131491in}{0.828627in}}%
\pgfpathlineto{\pgfqpoint{5.131901in}{0.781153in}}%
\pgfpathlineto{\pgfqpoint{5.133541in}{0.624539in}}%
\pgfpathlineto{\pgfqpoint{5.133951in}{0.592931in}}%
\pgfpathlineto{\pgfqpoint{5.136001in}{0.826782in}}%
\pgfpathlineto{\pgfqpoint{5.136411in}{0.804371in}}%
\pgfpathlineto{\pgfqpoint{5.138051in}{0.637047in}}%
\pgfpathlineto{\pgfqpoint{5.138461in}{0.593413in}}%
\pgfpathlineto{\pgfqpoint{5.138871in}{0.641670in}}%
\pgfpathlineto{\pgfqpoint{5.140511in}{0.821541in}}%
\pgfpathlineto{\pgfqpoint{5.140921in}{0.814606in}}%
\pgfpathlineto{\pgfqpoint{5.142971in}{0.593241in}}%
\pgfpathlineto{\pgfqpoint{5.145021in}{0.819435in}}%
\pgfpathlineto{\pgfqpoint{5.145431in}{0.817037in}}%
\pgfpathlineto{\pgfqpoint{5.147481in}{0.593571in}}%
\pgfpathlineto{\pgfqpoint{5.149531in}{0.822568in}}%
\pgfpathlineto{\pgfqpoint{5.149941in}{0.813022in}}%
\pgfpathlineto{\pgfqpoint{5.151991in}{0.594316in}}%
\pgfpathlineto{\pgfqpoint{5.154041in}{0.828373in}}%
\pgfpathlineto{\pgfqpoint{5.154451in}{0.799614in}}%
\pgfpathlineto{\pgfqpoint{5.156091in}{0.624035in}}%
\pgfpathlineto{\pgfqpoint{5.156501in}{0.597144in}}%
\pgfpathlineto{\pgfqpoint{5.158551in}{0.829286in}}%
\pgfpathlineto{\pgfqpoint{5.160601in}{0.598458in}}%
\pgfpathlineto{\pgfqpoint{5.161011in}{0.627276in}}%
\pgfpathlineto{\pgfqpoint{5.162651in}{0.823208in}}%
\pgfpathlineto{\pgfqpoint{5.163061in}{0.812837in}}%
\pgfpathlineto{\pgfqpoint{5.164701in}{0.629623in}}%
\pgfpathlineto{\pgfqpoint{5.165111in}{0.595061in}}%
\pgfpathlineto{\pgfqpoint{5.167161in}{0.829441in}}%
\pgfpathlineto{\pgfqpoint{5.169211in}{0.594964in}}%
\pgfpathlineto{\pgfqpoint{5.171261in}{0.831218in}}%
\pgfpathlineto{\pgfqpoint{5.171671in}{0.791939in}}%
\pgfpathlineto{\pgfqpoint{5.173311in}{0.605031in}}%
\pgfpathlineto{\pgfqpoint{5.174131in}{0.672295in}}%
\pgfpathlineto{\pgfqpoint{5.174541in}{0.651936in}}%
\pgfpathlineto{\pgfqpoint{5.175361in}{0.828052in}}%
\pgfpathlineto{\pgfqpoint{5.175771in}{0.806029in}}%
\pgfpathlineto{\pgfqpoint{5.177411in}{0.614062in}}%
\pgfpathlineto{\pgfqpoint{5.177821in}{0.615777in}}%
\pgfpathlineto{\pgfqpoint{5.179461in}{0.826429in}}%
\pgfpathlineto{\pgfqpoint{5.179871in}{0.810485in}}%
\pgfpathlineto{\pgfqpoint{5.181511in}{0.614927in}}%
\pgfpathlineto{\pgfqpoint{5.181921in}{0.616074in}}%
\pgfpathlineto{\pgfqpoint{5.183561in}{0.828843in}}%
\pgfpathlineto{\pgfqpoint{5.183971in}{0.806874in}}%
\pgfpathlineto{\pgfqpoint{5.185611in}{0.607595in}}%
\pgfpathlineto{\pgfqpoint{5.186431in}{0.674019in}}%
\pgfpathlineto{\pgfqpoint{5.186841in}{0.656240in}}%
\pgfpathlineto{\pgfqpoint{5.187661in}{0.833508in}}%
\pgfpathlineto{\pgfqpoint{5.188071in}{0.793009in}}%
\pgfpathlineto{\pgfqpoint{5.189711in}{0.596553in}}%
\pgfpathlineto{\pgfqpoint{5.191761in}{0.834075in}}%
\pgfpathlineto{\pgfqpoint{5.192171in}{0.763113in}}%
\pgfpathlineto{\pgfqpoint{5.193811in}{0.598047in}}%
\pgfpathlineto{\pgfqpoint{5.195451in}{0.822835in}}%
\pgfpathlineto{\pgfqpoint{5.195861in}{0.819659in}}%
\pgfpathlineto{\pgfqpoint{5.197501in}{0.609884in}}%
\pgfpathlineto{\pgfqpoint{5.198321in}{0.675919in}}%
\pgfpathlineto{\pgfqpoint{5.198731in}{0.664163in}}%
\pgfpathlineto{\pgfqpoint{5.199551in}{0.837002in}}%
\pgfpathlineto{\pgfqpoint{5.199961in}{0.776184in}}%
\pgfpathlineto{\pgfqpoint{5.201601in}{0.599020in}}%
\pgfpathlineto{\pgfqpoint{5.203241in}{0.828839in}}%
\pgfpathlineto{\pgfqpoint{5.203651in}{0.814415in}}%
\pgfpathlineto{\pgfqpoint{5.205291in}{0.597877in}}%
\pgfpathlineto{\pgfqpoint{5.207341in}{0.832358in}}%
\pgfpathlineto{\pgfqpoint{5.208981in}{0.615588in}}%
\pgfpathlineto{\pgfqpoint{5.209391in}{0.625432in}}%
\pgfpathlineto{\pgfqpoint{5.211031in}{0.838929in}}%
\pgfpathlineto{\pgfqpoint{5.211441in}{0.761505in}}%
\pgfpathlineto{\pgfqpoint{5.212671in}{0.626776in}}%
\pgfpathlineto{\pgfqpoint{5.213081in}{0.613335in}}%
\pgfpathlineto{\pgfqpoint{5.214721in}{0.840963in}}%
\pgfpathlineto{\pgfqpoint{5.215131in}{0.773143in}}%
\pgfpathlineto{\pgfqpoint{5.216771in}{0.609796in}}%
\pgfpathlineto{\pgfqpoint{5.218411in}{0.842035in}}%
\pgfpathlineto{\pgfqpoint{5.218821in}{0.772110in}}%
\pgfpathlineto{\pgfqpoint{5.220051in}{0.627729in}}%
\pgfpathlineto{\pgfqpoint{5.220461in}{0.615539in}}%
\pgfpathlineto{\pgfqpoint{5.222101in}{0.841919in}}%
\pgfpathlineto{\pgfqpoint{5.223741in}{0.617191in}}%
\pgfpathlineto{\pgfqpoint{5.224151in}{0.630184in}}%
\pgfpathlineto{\pgfqpoint{5.225791in}{0.836336in}}%
\pgfpathlineto{\pgfqpoint{5.227431in}{0.600834in}}%
\pgfpathlineto{\pgfqpoint{5.229071in}{0.835216in}}%
\pgfpathlineto{\pgfqpoint{5.229481in}{0.817058in}}%
\pgfpathlineto{\pgfqpoint{5.231121in}{0.603316in}}%
\pgfpathlineto{\pgfqpoint{5.232761in}{0.846746in}}%
\pgfpathlineto{\pgfqpoint{5.233171in}{0.772973in}}%
\pgfpathlineto{\pgfqpoint{5.234401in}{0.617809in}}%
\pgfpathlineto{\pgfqpoint{5.235631in}{0.690301in}}%
\pgfpathlineto{\pgfqpoint{5.236451in}{0.829452in}}%
\pgfpathlineto{\pgfqpoint{5.238091in}{0.604557in}}%
\pgfpathlineto{\pgfqpoint{5.239731in}{0.848886in}}%
\pgfpathlineto{\pgfqpoint{5.240141in}{0.765067in}}%
\pgfpathlineto{\pgfqpoint{5.241371in}{0.606689in}}%
\pgfpathlineto{\pgfqpoint{5.243011in}{0.843902in}}%
\pgfpathlineto{\pgfqpoint{5.243421in}{0.809660in}}%
\pgfpathlineto{\pgfqpoint{5.245061in}{0.621743in}}%
\pgfpathlineto{\pgfqpoint{5.246701in}{0.833868in}}%
\pgfpathlineto{\pgfqpoint{5.248341in}{0.606100in}}%
\pgfpathlineto{\pgfqpoint{5.249981in}{0.845894in}}%
\pgfpathlineto{\pgfqpoint{5.251621in}{0.606484in}}%
\pgfpathlineto{\pgfqpoint{5.253261in}{0.852645in}}%
\pgfpathlineto{\pgfqpoint{5.254901in}{0.605242in}}%
\pgfpathlineto{\pgfqpoint{5.256541in}{0.854583in}}%
\pgfpathlineto{\pgfqpoint{5.256951in}{0.782881in}}%
\pgfpathlineto{\pgfqpoint{5.258181in}{0.610514in}}%
\pgfpathlineto{\pgfqpoint{5.259821in}{0.851551in}}%
\pgfpathlineto{\pgfqpoint{5.260231in}{0.806631in}}%
\pgfpathlineto{\pgfqpoint{5.261461in}{0.626429in}}%
\pgfpathlineto{\pgfqpoint{5.261871in}{0.634836in}}%
\pgfpathlineto{\pgfqpoint{5.263101in}{0.843455in}}%
\pgfpathlineto{\pgfqpoint{5.263511in}{0.826509in}}%
\pgfpathlineto{\pgfqpoint{5.265151in}{0.617021in}}%
\pgfpathlineto{\pgfqpoint{5.266791in}{0.842147in}}%
\pgfpathlineto{\pgfqpoint{5.268431in}{0.607660in}}%
\pgfpathlineto{\pgfqpoint{5.270071in}{0.853222in}}%
\pgfpathlineto{\pgfqpoint{5.271711in}{0.607768in}}%
\pgfpathlineto{\pgfqpoint{5.273351in}{0.859462in}}%
\pgfpathlineto{\pgfqpoint{5.273761in}{0.764876in}}%
\pgfpathlineto{\pgfqpoint{5.274991in}{0.606194in}}%
\pgfpathlineto{\pgfqpoint{5.276631in}{0.860649in}}%
\pgfpathlineto{\pgfqpoint{5.277041in}{0.792369in}}%
\pgfpathlineto{\pgfqpoint{5.278271in}{0.615354in}}%
\pgfpathlineto{\pgfqpoint{5.279501in}{0.728918in}}%
\pgfpathlineto{\pgfqpoint{5.279911in}{0.856625in}}%
\pgfpathlineto{\pgfqpoint{5.280321in}{0.816175in}}%
\pgfpathlineto{\pgfqpoint{5.281551in}{0.631293in}}%
\pgfpathlineto{\pgfqpoint{5.281961in}{0.633231in}}%
\pgfpathlineto{\pgfqpoint{5.283191in}{0.847298in}}%
\pgfpathlineto{\pgfqpoint{5.283601in}{0.835865in}}%
\pgfpathlineto{\pgfqpoint{5.285241in}{0.614416in}}%
\pgfpathlineto{\pgfqpoint{5.286881in}{0.851059in}}%
\pgfpathlineto{\pgfqpoint{5.288521in}{0.609308in}}%
\pgfpathlineto{\pgfqpoint{5.290161in}{0.861424in}}%
\pgfpathlineto{\pgfqpoint{5.291801in}{0.608970in}}%
\pgfpathlineto{\pgfqpoint{5.293441in}{0.866680in}}%
\pgfpathlineto{\pgfqpoint{5.293851in}{0.776844in}}%
\pgfpathlineto{\pgfqpoint{5.295081in}{0.606905in}}%
\pgfpathlineto{\pgfqpoint{5.296721in}{0.866609in}}%
\pgfpathlineto{\pgfqpoint{5.297131in}{0.804310in}}%
\pgfpathlineto{\pgfqpoint{5.298361in}{0.621766in}}%
\pgfpathlineto{\pgfqpoint{5.299591in}{0.723603in}}%
\pgfpathlineto{\pgfqpoint{5.300001in}{0.861057in}}%
\pgfpathlineto{\pgfqpoint{5.300411in}{0.827804in}}%
\pgfpathlineto{\pgfqpoint{5.302051in}{0.629809in}}%
\pgfpathlineto{\pgfqpoint{5.303281in}{0.849940in}}%
\pgfpathlineto{\pgfqpoint{5.303691in}{0.846887in}}%
\pgfpathlineto{\pgfqpoint{5.305331in}{0.609985in}}%
\pgfpathlineto{\pgfqpoint{5.306971in}{0.861164in}}%
\pgfpathlineto{\pgfqpoint{5.308611in}{0.610931in}}%
\pgfpathlineto{\pgfqpoint{5.310251in}{0.870297in}}%
\pgfpathlineto{\pgfqpoint{5.311891in}{0.609956in}}%
\pgfpathlineto{\pgfqpoint{5.313531in}{0.874006in}}%
\pgfpathlineto{\pgfqpoint{5.313941in}{0.791678in}}%
\pgfpathlineto{\pgfqpoint{5.315171in}{0.611740in}}%
\pgfpathlineto{\pgfqpoint{5.316811in}{0.872077in}}%
\pgfpathlineto{\pgfqpoint{5.317221in}{0.818722in}}%
\pgfpathlineto{\pgfqpoint{5.318451in}{0.629750in}}%
\pgfpathlineto{\pgfqpoint{5.318861in}{0.643804in}}%
\pgfpathlineto{\pgfqpoint{5.320091in}{0.864365in}}%
\pgfpathlineto{\pgfqpoint{5.320501in}{0.841448in}}%
\pgfpathlineto{\pgfqpoint{5.322141in}{0.624184in}}%
\pgfpathlineto{\pgfqpoint{5.323781in}{0.859401in}}%
\pgfpathlineto{\pgfqpoint{5.325421in}{0.612209in}}%
\pgfpathlineto{\pgfqpoint{5.327061in}{0.872184in}}%
\pgfpathlineto{\pgfqpoint{5.328701in}{0.612346in}}%
\pgfpathlineto{\pgfqpoint{5.330341in}{0.879456in}}%
\pgfpathlineto{\pgfqpoint{5.330751in}{0.779030in}}%
\pgfpathlineto{\pgfqpoint{5.331981in}{0.610526in}}%
\pgfpathlineto{\pgfqpoint{5.333621in}{0.880943in}}%
\pgfpathlineto{\pgfqpoint{5.334031in}{0.809372in}}%
\pgfpathlineto{\pgfqpoint{5.335261in}{0.621851in}}%
\pgfpathlineto{\pgfqpoint{5.336491in}{0.740305in}}%
\pgfpathlineto{\pgfqpoint{5.336901in}{0.876442in}}%
\pgfpathlineto{\pgfqpoint{5.337311in}{0.835488in}}%
\pgfpathlineto{\pgfqpoint{5.338541in}{0.639204in}}%
\pgfpathlineto{\pgfqpoint{5.338951in}{0.637201in}}%
\pgfpathlineto{\pgfqpoint{5.340181in}{0.865832in}}%
\pgfpathlineto{\pgfqpoint{5.340591in}{0.856864in}}%
\pgfpathlineto{\pgfqpoint{5.342231in}{0.615906in}}%
\pgfpathlineto{\pgfqpoint{5.343871in}{0.873041in}}%
\pgfpathlineto{\pgfqpoint{5.345511in}{0.614210in}}%
\pgfpathlineto{\pgfqpoint{5.347151in}{0.883620in}}%
\pgfpathlineto{\pgfqpoint{5.348791in}{0.613291in}}%
\pgfpathlineto{\pgfqpoint{5.350431in}{0.888267in}}%
\pgfpathlineto{\pgfqpoint{5.350841in}{0.800492in}}%
\pgfpathlineto{\pgfqpoint{5.352071in}{0.614227in}}%
\pgfpathlineto{\pgfqpoint{5.353711in}{0.886723in}}%
\pgfpathlineto{\pgfqpoint{5.354121in}{0.829745in}}%
\pgfpathlineto{\pgfqpoint{5.355351in}{0.633637in}}%
\pgfpathlineto{\pgfqpoint{5.356171in}{0.703468in}}%
\pgfpathlineto{\pgfqpoint{5.356581in}{0.723947in}}%
\pgfpathlineto{\pgfqpoint{5.356991in}{0.878808in}}%
\pgfpathlineto{\pgfqpoint{5.357401in}{0.854278in}}%
\pgfpathlineto{\pgfqpoint{5.359041in}{0.627554in}}%
\pgfpathlineto{\pgfqpoint{5.360681in}{0.873571in}}%
\pgfpathlineto{\pgfqpoint{5.362321in}{0.615706in}}%
\pgfpathlineto{\pgfqpoint{5.363961in}{0.887165in}}%
\pgfpathlineto{\pgfqpoint{5.365601in}{0.615642in}}%
\pgfpathlineto{\pgfqpoint{5.367241in}{0.894668in}}%
\pgfpathlineto{\pgfqpoint{5.367651in}{0.792807in}}%
\pgfpathlineto{\pgfqpoint{5.368881in}{0.613416in}}%
\pgfpathlineto{\pgfqpoint{5.370521in}{0.895766in}}%
\pgfpathlineto{\pgfqpoint{5.370931in}{0.824945in}}%
\pgfpathlineto{\pgfqpoint{5.372161in}{0.628668in}}%
\pgfpathlineto{\pgfqpoint{5.373391in}{0.744601in}}%
\pgfpathlineto{\pgfqpoint{5.373801in}{0.890224in}}%
\pgfpathlineto{\pgfqpoint{5.374211in}{0.852359in}}%
\pgfpathlineto{\pgfqpoint{5.375851in}{0.637699in}}%
\pgfpathlineto{\pgfqpoint{5.377081in}{0.877902in}}%
\pgfpathlineto{\pgfqpoint{5.377491in}{0.874465in}}%
\pgfpathlineto{\pgfqpoint{5.379131in}{0.617014in}}%
\pgfpathlineto{\pgfqpoint{5.380771in}{0.890746in}}%
\pgfpathlineto{\pgfqpoint{5.382411in}{0.617719in}}%
\pgfpathlineto{\pgfqpoint{5.384051in}{0.900753in}}%
\pgfpathlineto{\pgfqpoint{5.385691in}{0.616060in}}%
\pgfpathlineto{\pgfqpoint{5.387331in}{0.904115in}}%
\pgfpathlineto{\pgfqpoint{5.387741in}{0.821797in}}%
\pgfpathlineto{\pgfqpoint{5.388971in}{0.624740in}}%
\pgfpathlineto{\pgfqpoint{5.390611in}{0.900548in}}%
\pgfpathlineto{\pgfqpoint{5.391021in}{0.851801in}}%
\pgfpathlineto{\pgfqpoint{5.392251in}{0.644515in}}%
\pgfpathlineto{\pgfqpoint{5.392661in}{0.646233in}}%
\pgfpathlineto{\pgfqpoint{5.393891in}{0.889863in}}%
\pgfpathlineto{\pgfqpoint{5.394301in}{0.876392in}}%
\pgfpathlineto{\pgfqpoint{5.395941in}{0.622307in}}%
\pgfpathlineto{\pgfqpoint{5.397581in}{0.894991in}}%
\pgfpathlineto{\pgfqpoint{5.399221in}{0.619662in}}%
\pgfpathlineto{\pgfqpoint{5.400861in}{0.907092in}}%
\pgfpathlineto{\pgfqpoint{5.402501in}{0.618440in}}%
\pgfpathlineto{\pgfqpoint{5.404141in}{0.912273in}}%
\pgfpathlineto{\pgfqpoint{5.404551in}{0.820990in}}%
\pgfpathlineto{\pgfqpoint{5.405781in}{0.622298in}}%
\pgfpathlineto{\pgfqpoint{5.407421in}{0.910200in}}%
\pgfpathlineto{\pgfqpoint{5.407831in}{0.853273in}}%
\pgfpathlineto{\pgfqpoint{5.409061in}{0.643561in}}%
\pgfpathlineto{\pgfqpoint{5.409471in}{0.653033in}}%
\pgfpathlineto{\pgfqpoint{5.410701in}{0.900639in}}%
\pgfpathlineto{\pgfqpoint{5.411111in}{0.879983in}}%
\pgfpathlineto{\pgfqpoint{5.412751in}{0.628451in}}%
\pgfpathlineto{\pgfqpoint{5.414391in}{0.900478in}}%
\pgfpathlineto{\pgfqpoint{5.416031in}{0.621599in}}%
\pgfpathlineto{\pgfqpoint{5.417671in}{0.914198in}}%
\pgfpathlineto{\pgfqpoint{5.419311in}{0.620634in}}%
\pgfpathlineto{\pgfqpoint{5.420951in}{0.920669in}}%
\pgfpathlineto{\pgfqpoint{5.421361in}{0.823189in}}%
\pgfpathlineto{\pgfqpoint{5.422591in}{0.621778in}}%
\pgfpathlineto{\pgfqpoint{5.424231in}{0.919511in}}%
\pgfpathlineto{\pgfqpoint{5.424641in}{0.857406in}}%
\pgfpathlineto{\pgfqpoint{5.425871in}{0.644293in}}%
\pgfpathlineto{\pgfqpoint{5.426281in}{0.657930in}}%
\pgfpathlineto{\pgfqpoint{5.427511in}{0.910451in}}%
\pgfpathlineto{\pgfqpoint{5.427921in}{0.885815in}}%
\pgfpathlineto{\pgfqpoint{5.429561in}{0.632426in}}%
\pgfpathlineto{\pgfqpoint{5.431201in}{0.907714in}}%
\pgfpathlineto{\pgfqpoint{5.432841in}{0.623626in}}%
\pgfpathlineto{\pgfqpoint{5.434481in}{0.922489in}}%
\pgfpathlineto{\pgfqpoint{5.436121in}{0.622685in}}%
\pgfpathlineto{\pgfqpoint{5.437761in}{0.929617in}}%
\pgfpathlineto{\pgfqpoint{5.438171in}{0.829025in}}%
\pgfpathlineto{\pgfqpoint{5.439401in}{0.623596in}}%
\pgfpathlineto{\pgfqpoint{5.441041in}{0.928676in}}%
\pgfpathlineto{\pgfqpoint{5.441451in}{0.864773in}}%
\pgfpathlineto{\pgfqpoint{5.442681in}{0.647066in}}%
\pgfpathlineto{\pgfqpoint{5.443091in}{0.660659in}}%
\pgfpathlineto{\pgfqpoint{5.444321in}{0.919363in}}%
\pgfpathlineto{\pgfqpoint{5.444731in}{0.894384in}}%
\pgfpathlineto{\pgfqpoint{5.446371in}{0.633892in}}%
\pgfpathlineto{\pgfqpoint{5.448011in}{0.917097in}}%
\pgfpathlineto{\pgfqpoint{5.449651in}{0.625778in}}%
\pgfpathlineto{\pgfqpoint{5.451291in}{0.932246in}}%
\pgfpathlineto{\pgfqpoint{5.452931in}{0.624569in}}%
\pgfpathlineto{\pgfqpoint{5.454571in}{0.939263in}}%
\pgfpathlineto{\pgfqpoint{5.454981in}{0.839073in}}%
\pgfpathlineto{\pgfqpoint{5.456211in}{0.628122in}}%
\pgfpathlineto{\pgfqpoint{5.457851in}{0.937693in}}%
\pgfpathlineto{\pgfqpoint{5.458261in}{0.875858in}}%
\pgfpathlineto{\pgfqpoint{5.459491in}{0.652166in}}%
\pgfpathlineto{\pgfqpoint{5.459901in}{0.660819in}}%
\pgfpathlineto{\pgfqpoint{5.461131in}{0.927207in}}%
\pgfpathlineto{\pgfqpoint{5.461541in}{0.906059in}}%
\pgfpathlineto{\pgfqpoint{5.463181in}{0.632390in}}%
\pgfpathlineto{\pgfqpoint{5.464821in}{0.928860in}}%
\pgfpathlineto{\pgfqpoint{5.466461in}{0.628002in}}%
\pgfpathlineto{\pgfqpoint{5.468101in}{0.943544in}}%
\pgfpathlineto{\pgfqpoint{5.469741in}{0.626175in}}%
\pgfpathlineto{\pgfqpoint{5.471381in}{0.949508in}}%
\pgfpathlineto{\pgfqpoint{5.471791in}{0.853813in}}%
\pgfpathlineto{\pgfqpoint{5.473021in}{0.635647in}}%
\pgfpathlineto{\pgfqpoint{5.474661in}{0.946272in}}%
\pgfpathlineto{\pgfqpoint{5.475071in}{0.891001in}}%
\pgfpathlineto{\pgfqpoint{5.476301in}{0.659760in}}%
\pgfpathlineto{\pgfqpoint{5.476711in}{0.657823in}}%
\pgfpathlineto{\pgfqpoint{5.477941in}{0.933493in}}%
\pgfpathlineto{\pgfqpoint{5.478351in}{0.921016in}}%
\pgfpathlineto{\pgfqpoint{5.479991in}{0.629521in}}%
\pgfpathlineto{\pgfqpoint{5.481631in}{0.942988in}}%
\pgfpathlineto{\pgfqpoint{5.483271in}{0.630115in}}%
\pgfpathlineto{\pgfqpoint{5.484911in}{0.956161in}}%
\pgfpathlineto{\pgfqpoint{5.486551in}{0.627263in}}%
\pgfpathlineto{\pgfqpoint{5.488191in}{0.959903in}}%
\pgfpathlineto{\pgfqpoint{5.488601in}{0.873562in}}%
\pgfpathlineto{\pgfqpoint{5.489831in}{0.646310in}}%
\pgfpathlineto{\pgfqpoint{5.491061in}{0.786665in}}%
\pgfpathlineto{\pgfqpoint{5.491471in}{0.953721in}}%
\pgfpathlineto{\pgfqpoint{5.491881in}{0.910315in}}%
\pgfpathlineto{\pgfqpoint{5.493521in}{0.650872in}}%
\pgfpathlineto{\pgfqpoint{5.495161in}{0.939133in}}%
\pgfpathlineto{\pgfqpoint{5.496801in}{0.632578in}}%
\pgfpathlineto{\pgfqpoint{5.498441in}{0.959102in}}%
\pgfpathlineto{\pgfqpoint{5.500081in}{0.631763in}}%
\pgfpathlineto{\pgfqpoint{5.501721in}{0.969437in}}%
\pgfpathlineto{\pgfqpoint{5.502131in}{0.855353in}}%
\pgfpathlineto{\pgfqpoint{5.503361in}{0.632055in}}%
\pgfpathlineto{\pgfqpoint{5.505001in}{0.969492in}}%
\pgfpathlineto{\pgfqpoint{5.505411in}{0.898355in}}%
\pgfpathlineto{\pgfqpoint{5.506641in}{0.660000in}}%
\pgfpathlineto{\pgfqpoint{5.507051in}{0.672020in}}%
\pgfpathlineto{\pgfqpoint{5.508281in}{0.958782in}}%
\pgfpathlineto{\pgfqpoint{5.508691in}{0.933540in}}%
\pgfpathlineto{\pgfqpoint{5.510331in}{0.638971in}}%
\pgfpathlineto{\pgfqpoint{5.511971in}{0.959829in}}%
\pgfpathlineto{\pgfqpoint{5.513611in}{0.635166in}}%
\pgfpathlineto{\pgfqpoint{5.515251in}{0.976275in}}%
\pgfpathlineto{\pgfqpoint{5.516891in}{0.632390in}}%
\pgfpathlineto{\pgfqpoint{5.518531in}{0.982081in}}%
\pgfpathlineto{\pgfqpoint{5.518941in}{0.886709in}}%
\pgfpathlineto{\pgfqpoint{5.520171in}{0.650164in}}%
\pgfpathlineto{\pgfqpoint{5.521401in}{0.805992in}}%
\pgfpathlineto{\pgfqpoint{5.521811in}{0.976611in}}%
\pgfpathlineto{\pgfqpoint{5.522221in}{0.927761in}}%
\pgfpathlineto{\pgfqpoint{5.523861in}{0.657580in}}%
\pgfpathlineto{\pgfqpoint{5.525501in}{0.959838in}}%
\pgfpathlineto{\pgfqpoint{5.527141in}{0.637776in}}%
\pgfpathlineto{\pgfqpoint{5.528781in}{0.981830in}}%
\pgfpathlineto{\pgfqpoint{5.530421in}{0.636501in}}%
\pgfpathlineto{\pgfqpoint{5.532061in}{0.992780in}}%
\pgfpathlineto{\pgfqpoint{5.532471in}{0.876903in}}%
\pgfpathlineto{\pgfqpoint{5.533701in}{0.641298in}}%
\pgfpathlineto{\pgfqpoint{5.535341in}{0.991903in}}%
\pgfpathlineto{\pgfqpoint{5.535751in}{0.923308in}}%
\pgfpathlineto{\pgfqpoint{5.536981in}{0.671095in}}%
\pgfpathlineto{\pgfqpoint{5.537391in}{0.673457in}}%
\pgfpathlineto{\pgfqpoint{5.538621in}{0.978605in}}%
\pgfpathlineto{\pgfqpoint{5.539031in}{0.960594in}}%
\pgfpathlineto{\pgfqpoint{5.540671in}{0.639962in}}%
\pgfpathlineto{\pgfqpoint{5.542311in}{0.987477in}}%
\pgfpathlineto{\pgfqpoint{5.543951in}{0.640026in}}%
\pgfpathlineto{\pgfqpoint{5.545591in}{1.002836in}}%
\pgfpathlineto{\pgfqpoint{5.547231in}{0.635538in}}%
\pgfpathlineto{\pgfqpoint{5.548871in}{1.005736in}}%
\pgfpathlineto{\pgfqpoint{5.549281in}{0.921641in}}%
\pgfpathlineto{\pgfqpoint{5.550511in}{0.667610in}}%
\pgfpathlineto{\pgfqpoint{5.550921in}{0.686350in}}%
\pgfpathlineto{\pgfqpoint{5.552151in}{0.995447in}}%
\pgfpathlineto{\pgfqpoint{5.552561in}{0.963493in}}%
\pgfpathlineto{\pgfqpoint{5.554201in}{0.647553in}}%
\pgfpathlineto{\pgfqpoint{5.555841in}{0.994504in}}%
\pgfpathlineto{\pgfqpoint{5.557481in}{0.643221in}}%
\pgfpathlineto{\pgfqpoint{5.559121in}{1.013387in}}%
\pgfpathlineto{\pgfqpoint{5.560761in}{0.639224in}}%
\pgfpathlineto{\pgfqpoint{5.562401in}{1.019056in}}%
\pgfpathlineto{\pgfqpoint{5.562811in}{0.924143in}}%
\pgfpathlineto{\pgfqpoint{5.564041in}{0.666661in}}%
\pgfpathlineto{\pgfqpoint{5.565271in}{0.817494in}}%
\pgfpathlineto{\pgfqpoint{5.565681in}{1.010646in}}%
\pgfpathlineto{\pgfqpoint{5.566091in}{0.969825in}}%
\pgfpathlineto{\pgfqpoint{5.567731in}{0.655501in}}%
\pgfpathlineto{\pgfqpoint{5.569371in}{1.004057in}}%
\pgfpathlineto{\pgfqpoint{5.571011in}{0.646292in}}%
\pgfpathlineto{\pgfqpoint{5.572651in}{1.025381in}}%
\pgfpathlineto{\pgfqpoint{5.574291in}{0.642345in}}%
\pgfpathlineto{\pgfqpoint{5.575931in}{1.032564in}}%
\pgfpathlineto{\pgfqpoint{5.576341in}{0.932092in}}%
\pgfpathlineto{\pgfqpoint{5.577571in}{0.669080in}}%
\pgfpathlineto{\pgfqpoint{5.578801in}{0.829308in}}%
\pgfpathlineto{\pgfqpoint{5.579211in}{1.024612in}}%
\pgfpathlineto{\pgfqpoint{5.579621in}{0.980719in}}%
\pgfpathlineto{\pgfqpoint{5.581261in}{0.659065in}}%
\pgfpathlineto{\pgfqpoint{5.582901in}{1.017052in}}%
\pgfpathlineto{\pgfqpoint{5.584541in}{0.649326in}}%
\pgfpathlineto{\pgfqpoint{5.586181in}{1.039467in}}%
\pgfpathlineto{\pgfqpoint{5.587821in}{0.644839in}}%
\pgfpathlineto{\pgfqpoint{5.589461in}{1.046584in}}%
\pgfpathlineto{\pgfqpoint{5.589871in}{0.946600in}}%
\pgfpathlineto{\pgfqpoint{5.591101in}{0.675541in}}%
\pgfpathlineto{\pgfqpoint{5.592331in}{0.833049in}}%
\pgfpathlineto{\pgfqpoint{5.592741in}{1.037297in}}%
\pgfpathlineto{\pgfqpoint{5.593151in}{0.997047in}}%
\pgfpathlineto{\pgfqpoint{5.594791in}{0.657349in}}%
\pgfpathlineto{\pgfqpoint{5.596431in}{1.034054in}}%
\pgfpathlineto{\pgfqpoint{5.598071in}{0.652202in}}%
\pgfpathlineto{\pgfqpoint{5.599711in}{1.055830in}}%
\pgfpathlineto{\pgfqpoint{5.601351in}{0.646444in}}%
\pgfpathlineto{\pgfqpoint{5.602991in}{1.060869in}}%
\pgfpathlineto{\pgfqpoint{5.603401in}{0.968508in}}%
\pgfpathlineto{\pgfqpoint{5.604631in}{0.686436in}}%
\pgfpathlineto{\pgfqpoint{5.605041in}{0.699480in}}%
\pgfpathlineto{\pgfqpoint{5.606271in}{1.047969in}}%
\pgfpathlineto{\pgfqpoint{5.606681in}{1.019274in}}%
\pgfpathlineto{\pgfqpoint{5.608321in}{0.655686in}}%
\pgfpathlineto{\pgfqpoint{5.609961in}{1.055069in}}%
\pgfpathlineto{\pgfqpoint{5.611601in}{0.654494in}}%
\pgfpathlineto{\pgfqpoint{5.613241in}{1.073954in}}%
\pgfpathlineto{\pgfqpoint{5.613651in}{0.934602in}}%
\pgfpathlineto{\pgfqpoint{5.614881in}{0.662198in}}%
\pgfpathlineto{\pgfqpoint{5.616521in}{1.074317in}}%
\pgfpathlineto{\pgfqpoint{5.616931in}{0.998178in}}%
\pgfpathlineto{\pgfqpoint{5.618161in}{0.701671in}}%
\pgfpathlineto{\pgfqpoint{5.618571in}{0.688242in}}%
\pgfpathlineto{\pgfqpoint{5.619801in}{1.054895in}}%
\pgfpathlineto{\pgfqpoint{5.620211in}{1.047194in}}%
\pgfpathlineto{\pgfqpoint{5.621851in}{0.659624in}}%
\pgfpathlineto{\pgfqpoint{5.623491in}{1.079240in}}%
\pgfpathlineto{\pgfqpoint{5.625131in}{0.655387in}}%
\pgfpathlineto{\pgfqpoint{5.626771in}{1.092260in}}%
\pgfpathlineto{\pgfqpoint{5.627181in}{0.973808in}}%
\pgfpathlineto{\pgfqpoint{5.628411in}{0.684145in}}%
\pgfpathlineto{\pgfqpoint{5.629641in}{0.872541in}}%
\pgfpathlineto{\pgfqpoint{5.630051in}{1.084576in}}%
\pgfpathlineto{\pgfqpoint{5.630461in}{1.035155in}}%
\pgfpathlineto{\pgfqpoint{5.632101in}{0.667680in}}%
\pgfpathlineto{\pgfqpoint{5.633741in}{1.079525in}}%
\pgfpathlineto{\pgfqpoint{5.635381in}{0.662041in}}%
\pgfpathlineto{\pgfqpoint{5.637021in}{1.104376in}}%
\pgfpathlineto{\pgfqpoint{5.638661in}{0.665372in}}%
\pgfpathlineto{\pgfqpoint{5.640301in}{1.107589in}}%
\pgfpathlineto{\pgfqpoint{5.640711in}{1.021864in}}%
\pgfpathlineto{\pgfqpoint{5.641941in}{0.710344in}}%
\pgfpathlineto{\pgfqpoint{5.642351in}{0.698644in}}%
\pgfpathlineto{\pgfqpoint{5.643581in}{1.087479in}}%
\pgfpathlineto{\pgfqpoint{5.643991in}{1.077607in}}%
\pgfpathlineto{\pgfqpoint{5.645631in}{0.666982in}}%
\pgfpathlineto{\pgfqpoint{5.647271in}{1.113274in}}%
\pgfpathlineto{\pgfqpoint{5.648911in}{0.661141in}}%
\pgfpathlineto{\pgfqpoint{5.650551in}{1.126266in}}%
\pgfpathlineto{\pgfqpoint{5.650961in}{1.010136in}}%
\pgfpathlineto{\pgfqpoint{5.652191in}{0.700636in}}%
\pgfpathlineto{\pgfqpoint{5.652601in}{0.725124in}}%
\pgfpathlineto{\pgfqpoint{5.653831in}{1.114477in}}%
\pgfpathlineto{\pgfqpoint{5.654241in}{1.076208in}}%
\pgfpathlineto{\pgfqpoint{5.655881in}{0.670788in}}%
\pgfpathlineto{\pgfqpoint{5.657521in}{1.121482in}}%
\pgfpathlineto{\pgfqpoint{5.657521in}{1.121482in}}%
\pgfusepath{stroke}%
\end{pgfscope}%
\begin{pgfscope}%
\pgfpathrectangle{\pgfqpoint{3.505455in}{0.528000in}}{\pgfqpoint{2.254545in}{1.680000in}}%
\pgfusepath{clip}%
\pgfsetrectcap%
\pgfsetroundjoin%
\pgfsetlinewidth{1.505625pt}%
\definecolor{currentstroke}{rgb}{0.737255,0.741176,0.133333}%
\pgfsetstrokecolor{currentstroke}%
\pgfsetdash{}{0pt}%
\pgfpathmoveto{\pgfqpoint{3.607934in}{1.461333in}}%
\pgfpathlineto{\pgfqpoint{3.613674in}{1.460240in}}%
\pgfpathlineto{\pgfqpoint{3.619004in}{1.456995in}}%
\pgfpathlineto{\pgfqpoint{3.624744in}{1.450701in}}%
\pgfpathlineto{\pgfqpoint{3.630484in}{1.440825in}}%
\pgfpathlineto{\pgfqpoint{3.636224in}{1.426411in}}%
\pgfpathlineto{\pgfqpoint{3.642374in}{1.404565in}}%
\pgfpathlineto{\pgfqpoint{3.648934in}{1.372070in}}%
\pgfpathlineto{\pgfqpoint{3.655494in}{1.327892in}}%
\pgfpathlineto{\pgfqpoint{3.662464in}{1.265725in}}%
\pgfpathlineto{\pgfqpoint{3.668614in}{1.201041in}}%
\pgfpathlineto{\pgfqpoint{3.669024in}{1.209728in}}%
\pgfpathlineto{\pgfqpoint{3.689934in}{1.678173in}}%
\pgfpathlineto{\pgfqpoint{3.703054in}{1.945349in}}%
\pgfpathlineto{\pgfqpoint{3.711254in}{2.070570in}}%
\pgfpathlineto{\pgfqpoint{3.716994in}{2.129296in}}%
\pgfpathlineto{\pgfqpoint{3.721504in}{2.155201in}}%
\pgfpathlineto{\pgfqpoint{3.724374in}{2.161407in}}%
\pgfpathlineto{\pgfqpoint{3.726014in}{2.161150in}}%
\pgfpathlineto{\pgfqpoint{3.727654in}{2.158034in}}%
\pgfpathlineto{\pgfqpoint{3.730114in}{2.147849in}}%
\pgfpathlineto{\pgfqpoint{3.733394in}{2.123672in}}%
\pgfpathlineto{\pgfqpoint{3.737494in}{2.075875in}}%
\pgfpathlineto{\pgfqpoint{3.742824in}{1.983747in}}%
\pgfpathlineto{\pgfqpoint{3.748974in}{1.834796in}}%
\pgfpathlineto{\pgfqpoint{3.756354in}{1.596810in}}%
\pgfpathlineto{\pgfqpoint{3.765374in}{1.225078in}}%
\pgfpathlineto{\pgfqpoint{3.771114in}{0.956637in}}%
\pgfpathlineto{\pgfqpoint{3.771934in}{0.965170in}}%
\pgfpathlineto{\pgfqpoint{3.783004in}{1.068529in}}%
\pgfpathlineto{\pgfqpoint{3.789974in}{1.115524in}}%
\pgfpathlineto{\pgfqpoint{3.794484in}{1.133232in}}%
\pgfpathlineto{\pgfqpoint{3.797354in}{1.137184in}}%
\pgfpathlineto{\pgfqpoint{3.798994in}{1.136278in}}%
\pgfpathlineto{\pgfqpoint{3.801044in}{1.131370in}}%
\pgfpathlineto{\pgfqpoint{3.803504in}{1.119092in}}%
\pgfpathlineto{\pgfqpoint{3.806784in}{1.089807in}}%
\pgfpathlineto{\pgfqpoint{3.810474in}{1.035386in}}%
\pgfpathlineto{\pgfqpoint{3.814984in}{0.931006in}}%
\pgfpathlineto{\pgfqpoint{3.819494in}{0.778164in}}%
\pgfpathlineto{\pgfqpoint{3.820314in}{0.795279in}}%
\pgfpathlineto{\pgfqpoint{3.836714in}{1.352839in}}%
\pgfpathlineto{\pgfqpoint{3.839584in}{1.383855in}}%
\pgfpathlineto{\pgfqpoint{3.840404in}{1.385486in}}%
\pgfpathlineto{\pgfqpoint{3.840814in}{1.385021in}}%
\pgfpathlineto{\pgfqpoint{3.842044in}{1.378390in}}%
\pgfpathlineto{\pgfqpoint{3.844094in}{1.349634in}}%
\pgfpathlineto{\pgfqpoint{3.846964in}{1.273051in}}%
\pgfpathlineto{\pgfqpoint{3.850654in}{1.135973in}}%
\pgfpathlineto{\pgfqpoint{3.858443in}{1.696409in}}%
\pgfpathlineto{\pgfqpoint{3.863363in}{1.910155in}}%
\pgfpathlineto{\pgfqpoint{3.866643in}{1.978675in}}%
\pgfpathlineto{\pgfqpoint{3.868693in}{1.990513in}}%
\pgfpathlineto{\pgfqpoint{3.869513in}{1.988632in}}%
\pgfpathlineto{\pgfqpoint{3.871153in}{1.973726in}}%
\pgfpathlineto{\pgfqpoint{3.873613in}{1.924295in}}%
\pgfpathlineto{\pgfqpoint{3.877303in}{1.792944in}}%
\pgfpathlineto{\pgfqpoint{3.882223in}{1.523707in}}%
\pgfpathlineto{\pgfqpoint{3.889193in}{1.002564in}}%
\pgfpathlineto{\pgfqpoint{3.890423in}{0.926590in}}%
\pgfpathlineto{\pgfqpoint{3.890833in}{0.933982in}}%
\pgfpathlineto{\pgfqpoint{3.897393in}{1.033004in}}%
\pgfpathlineto{\pgfqpoint{3.901493in}{1.069880in}}%
\pgfpathlineto{\pgfqpoint{3.903953in}{1.078456in}}%
\pgfpathlineto{\pgfqpoint{3.905183in}{1.077914in}}%
\pgfpathlineto{\pgfqpoint{3.906823in}{1.071302in}}%
\pgfpathlineto{\pgfqpoint{3.908873in}{1.052083in}}%
\pgfpathlineto{\pgfqpoint{3.911743in}{1.000702in}}%
\pgfpathlineto{\pgfqpoint{3.915433in}{0.883815in}}%
\pgfpathlineto{\pgfqpoint{3.918303in}{0.748464in}}%
\pgfpathlineto{\pgfqpoint{3.918713in}{0.760565in}}%
\pgfpathlineto{\pgfqpoint{3.928963in}{1.220645in}}%
\pgfpathlineto{\pgfqpoint{3.931423in}{1.260157in}}%
\pgfpathlineto{\pgfqpoint{3.932243in}{1.262033in}}%
\pgfpathlineto{\pgfqpoint{3.932653in}{1.260710in}}%
\pgfpathlineto{\pgfqpoint{3.933883in}{1.247547in}}%
\pgfpathlineto{\pgfqpoint{3.935933in}{1.195322in}}%
\pgfpathlineto{\pgfqpoint{3.939213in}{1.047011in}}%
\pgfpathlineto{\pgfqpoint{3.939623in}{1.090347in}}%
\pgfpathlineto{\pgfqpoint{3.945363in}{1.586090in}}%
\pgfpathlineto{\pgfqpoint{3.949053in}{1.758190in}}%
\pgfpathlineto{\pgfqpoint{3.951513in}{1.799198in}}%
\pgfpathlineto{\pgfqpoint{3.951923in}{1.800270in}}%
\pgfpathlineto{\pgfqpoint{3.952333in}{1.799719in}}%
\pgfpathlineto{\pgfqpoint{3.953563in}{1.788467in}}%
\pgfpathlineto{\pgfqpoint{3.955613in}{1.738796in}}%
\pgfpathlineto{\pgfqpoint{3.958893in}{1.585536in}}%
\pgfpathlineto{\pgfqpoint{3.963403in}{1.252216in}}%
\pgfpathlineto{\pgfqpoint{3.967503in}{0.885777in}}%
\pgfpathlineto{\pgfqpoint{3.968323in}{0.903717in}}%
\pgfpathlineto{\pgfqpoint{3.973653in}{0.992182in}}%
\pgfpathlineto{\pgfqpoint{3.976523in}{1.013091in}}%
\pgfpathlineto{\pgfqpoint{3.977343in}{1.014470in}}%
\pgfpathlineto{\pgfqpoint{3.977753in}{1.014282in}}%
\pgfpathlineto{\pgfqpoint{3.978983in}{1.009904in}}%
\pgfpathlineto{\pgfqpoint{3.981033in}{0.988294in}}%
\pgfpathlineto{\pgfqpoint{3.983493in}{0.934485in}}%
\pgfpathlineto{\pgfqpoint{3.986773in}{0.807121in}}%
\pgfpathlineto{\pgfqpoint{3.988413in}{0.718340in}}%
\pgfpathlineto{\pgfqpoint{3.988823in}{0.734958in}}%
\pgfpathlineto{\pgfqpoint{3.996613in}{1.104754in}}%
\pgfpathlineto{\pgfqpoint{3.999073in}{1.142355in}}%
\pgfpathlineto{\pgfqpoint{3.999483in}{1.142366in}}%
\pgfpathlineto{\pgfqpoint{4.000303in}{1.136691in}}%
\pgfpathlineto{\pgfqpoint{4.001943in}{1.102586in}}%
\pgfpathlineto{\pgfqpoint{4.004813in}{0.976999in}}%
\pgfpathlineto{\pgfqpoint{4.005223in}{0.982947in}}%
\pgfpathlineto{\pgfqpoint{4.010553in}{1.462973in}}%
\pgfpathlineto{\pgfqpoint{4.013833in}{1.596880in}}%
\pgfpathlineto{\pgfqpoint{4.015473in}{1.611560in}}%
\pgfpathlineto{\pgfqpoint{4.016293in}{1.605984in}}%
\pgfpathlineto{\pgfqpoint{4.017933in}{1.569974in}}%
\pgfpathlineto{\pgfqpoint{4.020393in}{1.458294in}}%
\pgfpathlineto{\pgfqpoint{4.024493in}{1.145260in}}%
\pgfpathlineto{\pgfqpoint{4.027773in}{0.844058in}}%
\pgfpathlineto{\pgfqpoint{4.028593in}{0.863432in}}%
\pgfpathlineto{\pgfqpoint{4.032693in}{0.933841in}}%
\pgfpathlineto{\pgfqpoint{4.035153in}{0.949232in}}%
\pgfpathlineto{\pgfqpoint{4.035973in}{0.948650in}}%
\pgfpathlineto{\pgfqpoint{4.037203in}{0.941562in}}%
\pgfpathlineto{\pgfqpoint{4.039253in}{0.910961in}}%
\pgfpathlineto{\pgfqpoint{4.042123in}{0.822432in}}%
\pgfpathlineto{\pgfqpoint{4.044993in}{0.696799in}}%
\pgfpathlineto{\pgfqpoint{4.045403in}{0.718924in}}%
\pgfpathlineto{\pgfqpoint{4.051553in}{0.999937in}}%
\pgfpathlineto{\pgfqpoint{4.054013in}{1.033398in}}%
\pgfpathlineto{\pgfqpoint{4.054833in}{1.028849in}}%
\pgfpathlineto{\pgfqpoint{4.056473in}{0.995373in}}%
\pgfpathlineto{\pgfqpoint{4.058933in}{0.890630in}}%
\pgfpathlineto{\pgfqpoint{4.059343in}{0.929665in}}%
\pgfpathlineto{\pgfqpoint{4.063853in}{1.319818in}}%
\pgfpathlineto{\pgfqpoint{4.066723in}{1.425923in}}%
\pgfpathlineto{\pgfqpoint{4.067953in}{1.433522in}}%
\pgfpathlineto{\pgfqpoint{4.068773in}{1.426130in}}%
\pgfpathlineto{\pgfqpoint{4.070413in}{1.382704in}}%
\pgfpathlineto{\pgfqpoint{4.073283in}{1.224053in}}%
\pgfpathlineto{\pgfqpoint{4.078203in}{0.800007in}}%
\pgfpathlineto{\pgfqpoint{4.079433in}{0.828605in}}%
\pgfpathlineto{\pgfqpoint{4.082713in}{0.878565in}}%
\pgfpathlineto{\pgfqpoint{4.084353in}{0.886064in}}%
\pgfpathlineto{\pgfqpoint{4.084763in}{0.885779in}}%
\pgfpathlineto{\pgfqpoint{4.085993in}{0.879188in}}%
\pgfpathlineto{\pgfqpoint{4.088043in}{0.847055in}}%
\pgfpathlineto{\pgfqpoint{4.090913in}{0.752383in}}%
\pgfpathlineto{\pgfqpoint{4.092963in}{0.671320in}}%
\pgfpathlineto{\pgfqpoint{4.098703in}{0.912097in}}%
\pgfpathlineto{\pgfqpoint{4.100753in}{0.936968in}}%
\pgfpathlineto{\pgfqpoint{4.101573in}{0.933261in}}%
\pgfpathlineto{\pgfqpoint{4.103213in}{0.901863in}}%
\pgfpathlineto{\pgfqpoint{4.105263in}{0.822449in}}%
\pgfpathlineto{\pgfqpoint{4.110593in}{1.224184in}}%
\pgfpathlineto{\pgfqpoint{4.112643in}{1.270514in}}%
\pgfpathlineto{\pgfqpoint{4.113053in}{1.271683in}}%
\pgfpathlineto{\pgfqpoint{4.113463in}{1.270184in}}%
\pgfpathlineto{\pgfqpoint{4.114693in}{1.250029in}}%
\pgfpathlineto{\pgfqpoint{4.116743in}{1.167589in}}%
\pgfpathlineto{\pgfqpoint{4.120023in}{0.931642in}}%
\pgfpathlineto{\pgfqpoint{4.122073in}{0.756069in}}%
\pgfpathlineto{\pgfqpoint{4.122893in}{0.775011in}}%
\pgfpathlineto{\pgfqpoint{4.126173in}{0.822451in}}%
\pgfpathlineto{\pgfqpoint{4.127403in}{0.826446in}}%
\pgfpathlineto{\pgfqpoint{4.127813in}{0.825844in}}%
\pgfpathlineto{\pgfqpoint{4.129043in}{0.817776in}}%
\pgfpathlineto{\pgfqpoint{4.131093in}{0.781748in}}%
\pgfpathlineto{\pgfqpoint{4.133963in}{0.681436in}}%
\pgfpathlineto{\pgfqpoint{4.134783in}{0.642886in}}%
\pgfpathlineto{\pgfqpoint{4.135193in}{0.650030in}}%
\pgfpathlineto{\pgfqpoint{4.140113in}{0.832352in}}%
\pgfpathlineto{\pgfqpoint{4.142163in}{0.853838in}}%
\pgfpathlineto{\pgfqpoint{4.142983in}{0.849615in}}%
\pgfpathlineto{\pgfqpoint{4.144623in}{0.818982in}}%
\pgfpathlineto{\pgfqpoint{4.146263in}{0.764239in}}%
\pgfpathlineto{\pgfqpoint{4.150773in}{1.079503in}}%
\pgfpathlineto{\pgfqpoint{4.152823in}{1.126537in}}%
\pgfpathlineto{\pgfqpoint{4.153233in}{1.127831in}}%
\pgfpathlineto{\pgfqpoint{4.153643in}{1.126442in}}%
\pgfpathlineto{\pgfqpoint{4.154873in}{1.106548in}}%
\pgfpathlineto{\pgfqpoint{4.156923in}{1.024759in}}%
\pgfpathlineto{\pgfqpoint{4.160613in}{0.759836in}}%
\pgfpathlineto{\pgfqpoint{4.161433in}{0.719605in}}%
\pgfpathlineto{\pgfqpoint{4.161843in}{0.728392in}}%
\pgfpathlineto{\pgfqpoint{4.164713in}{0.768159in}}%
\pgfpathlineto{\pgfqpoint{4.165943in}{0.771856in}}%
\pgfpathlineto{\pgfqpoint{4.166353in}{0.771081in}}%
\pgfpathlineto{\pgfqpoint{4.167583in}{0.762348in}}%
\pgfpathlineto{\pgfqpoint{4.169633in}{0.725299in}}%
\pgfpathlineto{\pgfqpoint{4.172913in}{0.620875in}}%
\pgfpathlineto{\pgfqpoint{4.173323in}{0.637365in}}%
\pgfpathlineto{\pgfqpoint{4.177833in}{0.772443in}}%
\pgfpathlineto{\pgfqpoint{4.179473in}{0.783594in}}%
\pgfpathlineto{\pgfqpoint{4.180293in}{0.779194in}}%
\pgfpathlineto{\pgfqpoint{4.181933in}{0.750732in}}%
\pgfpathlineto{\pgfqpoint{4.183163in}{0.714177in}}%
\pgfpathlineto{\pgfqpoint{4.188493in}{0.994089in}}%
\pgfpathlineto{\pgfqpoint{4.189723in}{1.003154in}}%
\pgfpathlineto{\pgfqpoint{4.190543in}{0.996381in}}%
\pgfpathlineto{\pgfqpoint{4.192183in}{0.953693in}}%
\pgfpathlineto{\pgfqpoint{4.195053in}{0.800336in}}%
\pgfpathlineto{\pgfqpoint{4.196693in}{0.681522in}}%
\pgfpathlineto{\pgfqpoint{4.197513in}{0.691980in}}%
\pgfpathlineto{\pgfqpoint{4.200383in}{0.722585in}}%
\pgfpathlineto{\pgfqpoint{4.200793in}{0.723314in}}%
\pgfpathlineto{\pgfqpoint{4.201203in}{0.723051in}}%
\pgfpathlineto{\pgfqpoint{4.202433in}{0.716097in}}%
\pgfpathlineto{\pgfqpoint{4.204483in}{0.683331in}}%
\pgfpathlineto{\pgfqpoint{4.207353in}{0.597452in}}%
\pgfpathlineto{\pgfqpoint{4.207763in}{0.611524in}}%
\pgfpathlineto{\pgfqpoint{4.211863in}{0.715568in}}%
\pgfpathlineto{\pgfqpoint{4.213503in}{0.725488in}}%
\pgfpathlineto{\pgfqpoint{4.214323in}{0.721779in}}%
\pgfpathlineto{\pgfqpoint{4.215963in}{0.697417in}}%
\pgfpathlineto{\pgfqpoint{4.216783in}{0.677676in}}%
\pgfpathlineto{\pgfqpoint{4.217193in}{0.681146in}}%
\pgfpathlineto{\pgfqpoint{4.220883in}{0.868519in}}%
\pgfpathlineto{\pgfqpoint{4.222933in}{0.897977in}}%
\pgfpathlineto{\pgfqpoint{4.223753in}{0.893242in}}%
\pgfpathlineto{\pgfqpoint{4.225393in}{0.856898in}}%
\pgfpathlineto{\pgfqpoint{4.228263in}{0.721975in}}%
\pgfpathlineto{\pgfqpoint{4.229493in}{0.645730in}}%
\pgfpathlineto{\pgfqpoint{4.230313in}{0.659279in}}%
\pgfpathlineto{\pgfqpoint{4.232773in}{0.680819in}}%
\pgfpathlineto{\pgfqpoint{4.233183in}{0.681330in}}%
\pgfpathlineto{\pgfqpoint{4.233593in}{0.680903in}}%
\pgfpathlineto{\pgfqpoint{4.234823in}{0.673861in}}%
\pgfpathlineto{\pgfqpoint{4.236873in}{0.642960in}}%
\pgfpathlineto{\pgfqpoint{4.239333in}{0.584079in}}%
\pgfpathlineto{\pgfqpoint{4.239743in}{0.595647in}}%
\pgfpathlineto{\pgfqpoint{4.243433in}{0.670838in}}%
\pgfpathlineto{\pgfqpoint{4.245073in}{0.678113in}}%
\pgfpathlineto{\pgfqpoint{4.245893in}{0.674391in}}%
\pgfpathlineto{\pgfqpoint{4.247533in}{0.652845in}}%
\pgfpathlineto{\pgfqpoint{4.248353in}{0.637948in}}%
\pgfpathlineto{\pgfqpoint{4.252043in}{0.791383in}}%
\pgfpathlineto{\pgfqpoint{4.253683in}{0.810814in}}%
\pgfpathlineto{\pgfqpoint{4.254093in}{0.810431in}}%
\pgfpathlineto{\pgfqpoint{4.255323in}{0.797001in}}%
\pgfpathlineto{\pgfqpoint{4.257373in}{0.737549in}}%
\pgfpathlineto{\pgfqpoint{4.259833in}{0.621508in}}%
\pgfpathlineto{\pgfqpoint{4.260653in}{0.629505in}}%
\pgfpathlineto{\pgfqpoint{4.263113in}{0.645901in}}%
\pgfpathlineto{\pgfqpoint{4.263933in}{0.644799in}}%
\pgfpathlineto{\pgfqpoint{4.265163in}{0.636705in}}%
\pgfpathlineto{\pgfqpoint{4.267623in}{0.598648in}}%
\pgfpathlineto{\pgfqpoint{4.268853in}{0.570780in}}%
\pgfpathlineto{\pgfqpoint{4.269263in}{0.575922in}}%
\pgfpathlineto{\pgfqpoint{4.272953in}{0.635798in}}%
\pgfpathlineto{\pgfqpoint{4.274183in}{0.640527in}}%
\pgfpathlineto{\pgfqpoint{4.274593in}{0.640071in}}%
\pgfpathlineto{\pgfqpoint{4.275823in}{0.632652in}}%
\pgfpathlineto{\pgfqpoint{4.277463in}{0.609974in}}%
\pgfpathlineto{\pgfqpoint{4.281563in}{0.734221in}}%
\pgfpathlineto{\pgfqpoint{4.282793in}{0.740426in}}%
\pgfpathlineto{\pgfqpoint{4.283613in}{0.735623in}}%
\pgfpathlineto{\pgfqpoint{4.285253in}{0.706006in}}%
\pgfpathlineto{\pgfqpoint{4.288533in}{0.599410in}}%
\pgfpathlineto{\pgfqpoint{4.289763in}{0.611039in}}%
\pgfpathlineto{\pgfqpoint{4.291403in}{0.616948in}}%
\pgfpathlineto{\pgfqpoint{4.292223in}{0.615504in}}%
\pgfpathlineto{\pgfqpoint{4.293863in}{0.603814in}}%
\pgfpathlineto{\pgfqpoint{4.296733in}{0.559432in}}%
\pgfpathlineto{\pgfqpoint{4.297553in}{0.571729in}}%
\pgfpathlineto{\pgfqpoint{4.300833in}{0.608911in}}%
\pgfpathlineto{\pgfqpoint{4.301653in}{0.610916in}}%
\pgfpathlineto{\pgfqpoint{4.302063in}{0.610677in}}%
\pgfpathlineto{\pgfqpoint{4.303293in}{0.605062in}}%
\pgfpathlineto{\pgfqpoint{4.304933in}{0.589110in}}%
\pgfpathlineto{\pgfqpoint{4.308213in}{0.674456in}}%
\pgfpathlineto{\pgfqpoint{4.309853in}{0.684808in}}%
\pgfpathlineto{\pgfqpoint{4.310673in}{0.681039in}}%
\pgfpathlineto{\pgfqpoint{4.312313in}{0.656887in}}%
\pgfpathlineto{\pgfqpoint{4.315183in}{0.580140in}}%
\pgfpathlineto{\pgfqpoint{4.316413in}{0.589491in}}%
\pgfpathlineto{\pgfqpoint{4.318053in}{0.593749in}}%
\pgfpathlineto{\pgfqpoint{4.318873in}{0.592152in}}%
\pgfpathlineto{\pgfqpoint{4.320513in}{0.581709in}}%
\pgfpathlineto{\pgfqpoint{4.322973in}{0.551285in}}%
\pgfpathlineto{\pgfqpoint{4.323793in}{0.560345in}}%
\pgfpathlineto{\pgfqpoint{4.326663in}{0.585973in}}%
\pgfpathlineto{\pgfqpoint{4.327893in}{0.588088in}}%
\pgfpathlineto{\pgfqpoint{4.329123in}{0.584327in}}%
\pgfpathlineto{\pgfqpoint{4.330763in}{0.571868in}}%
\pgfpathlineto{\pgfqpoint{4.334043in}{0.635805in}}%
\pgfpathlineto{\pgfqpoint{4.335273in}{0.641745in}}%
\pgfpathlineto{\pgfqpoint{4.335683in}{0.641293in}}%
\pgfpathlineto{\pgfqpoint{4.336913in}{0.632885in}}%
\pgfpathlineto{\pgfqpoint{4.339373in}{0.589263in}}%
\pgfpathlineto{\pgfqpoint{4.340603in}{0.566894in}}%
\pgfpathlineto{\pgfqpoint{4.341013in}{0.569557in}}%
\pgfpathlineto{\pgfqpoint{4.343063in}{0.575708in}}%
\pgfpathlineto{\pgfqpoint{4.343883in}{0.574634in}}%
\pgfpathlineto{\pgfqpoint{4.345523in}{0.566626in}}%
\pgfpathlineto{\pgfqpoint{4.347983in}{0.544773in}}%
\pgfpathlineto{\pgfqpoint{4.348393in}{0.549036in}}%
\pgfpathlineto{\pgfqpoint{4.351673in}{0.570251in}}%
\pgfpathlineto{\pgfqpoint{4.352493in}{0.570880in}}%
\pgfpathlineto{\pgfqpoint{4.352903in}{0.570423in}}%
\pgfpathlineto{\pgfqpoint{4.354133in}{0.566098in}}%
\pgfpathlineto{\pgfqpoint{4.354953in}{0.560986in}}%
\pgfpathlineto{\pgfqpoint{4.355363in}{0.561452in}}%
\pgfpathlineto{\pgfqpoint{4.358643in}{0.606511in}}%
\pgfpathlineto{\pgfqpoint{4.359463in}{0.609018in}}%
\pgfpathlineto{\pgfqpoint{4.359873in}{0.608828in}}%
\pgfpathlineto{\pgfqpoint{4.361103in}{0.602667in}}%
\pgfpathlineto{\pgfqpoint{4.363153in}{0.576391in}}%
\pgfpathlineto{\pgfqpoint{4.364383in}{0.554568in}}%
\pgfpathlineto{\pgfqpoint{4.365203in}{0.558561in}}%
\pgfpathlineto{\pgfqpoint{4.366843in}{0.561957in}}%
\pgfpathlineto{\pgfqpoint{4.367253in}{0.561798in}}%
\pgfpathlineto{\pgfqpoint{4.368483in}{0.558938in}}%
\pgfpathlineto{\pgfqpoint{4.370943in}{0.544118in}}%
\pgfpathlineto{\pgfqpoint{4.371353in}{0.540814in}}%
\pgfpathlineto{\pgfqpoint{4.371763in}{0.541067in}}%
\pgfpathlineto{\pgfqpoint{4.375043in}{0.557552in}}%
\pgfpathlineto{\pgfqpoint{4.376273in}{0.557799in}}%
\pgfpathlineto{\pgfqpoint{4.377913in}{0.552857in}}%
\pgfpathlineto{\pgfqpoint{4.378323in}{0.550798in}}%
\pgfpathlineto{\pgfqpoint{4.378733in}{0.552613in}}%
\pgfpathlineto{\pgfqpoint{4.381603in}{0.582227in}}%
\pgfpathlineto{\pgfqpoint{4.382833in}{0.584713in}}%
\pgfpathlineto{\pgfqpoint{4.383653in}{0.582692in}}%
\pgfpathlineto{\pgfqpoint{4.385293in}{0.570669in}}%
\pgfpathlineto{\pgfqpoint{4.387343in}{0.547001in}}%
\pgfpathlineto{\pgfqpoint{4.388163in}{0.549772in}}%
\pgfpathlineto{\pgfqpoint{4.389803in}{0.551696in}}%
\pgfpathlineto{\pgfqpoint{4.391033in}{0.549882in}}%
\pgfpathlineto{\pgfqpoint{4.393083in}{0.541509in}}%
\pgfpathlineto{\pgfqpoint{4.393903in}{0.536827in}}%
\pgfpathlineto{\pgfqpoint{4.394313in}{0.537245in}}%
\pgfpathlineto{\pgfqpoint{4.397183in}{0.548099in}}%
\pgfpathlineto{\pgfqpoint{4.398413in}{0.548701in}}%
\pgfpathlineto{\pgfqpoint{4.399643in}{0.546704in}}%
\pgfpathlineto{\pgfqpoint{4.400873in}{0.544101in}}%
\pgfpathlineto{\pgfqpoint{4.403743in}{0.565502in}}%
\pgfpathlineto{\pgfqpoint{4.404973in}{0.566962in}}%
\pgfpathlineto{\pgfqpoint{4.406203in}{0.563498in}}%
\pgfpathlineto{\pgfqpoint{4.408663in}{0.544790in}}%
\pgfpathlineto{\pgfqpoint{4.409073in}{0.540751in}}%
\pgfpathlineto{\pgfqpoint{4.409893in}{0.542829in}}%
\pgfpathlineto{\pgfqpoint{4.411533in}{0.544255in}}%
\pgfpathlineto{\pgfqpoint{4.412763in}{0.542875in}}%
\pgfpathlineto{\pgfqpoint{4.415223in}{0.534984in}}%
\pgfpathlineto{\pgfqpoint{4.415633in}{0.533463in}}%
\pgfpathlineto{\pgfqpoint{4.416043in}{0.535093in}}%
\pgfpathlineto{\pgfqpoint{4.418913in}{0.541947in}}%
\pgfpathlineto{\pgfqpoint{4.420143in}{0.541805in}}%
\pgfpathlineto{\pgfqpoint{4.421783in}{0.538784in}}%
\pgfpathlineto{\pgfqpoint{4.422193in}{0.539378in}}%
\pgfpathlineto{\pgfqpoint{4.425063in}{0.553757in}}%
\pgfpathlineto{\pgfqpoint{4.426293in}{0.554145in}}%
\pgfpathlineto{\pgfqpoint{4.427523in}{0.550964in}}%
\pgfpathlineto{\pgfqpoint{4.429983in}{0.536604in}}%
\pgfpathlineto{\pgfqpoint{4.431213in}{0.538532in}}%
\pgfpathlineto{\pgfqpoint{4.432853in}{0.538713in}}%
\pgfpathlineto{\pgfqpoint{4.434903in}{0.535331in}}%
\pgfpathlineto{\pgfqpoint{4.436133in}{0.531939in}}%
\pgfpathlineto{\pgfqpoint{4.436543in}{0.532419in}}%
\pgfpathlineto{\pgfqpoint{4.439413in}{0.537314in}}%
\pgfpathlineto{\pgfqpoint{4.441053in}{0.536842in}}%
\pgfpathlineto{\pgfqpoint{4.442283in}{0.534996in}}%
\pgfpathlineto{\pgfqpoint{4.442693in}{0.536199in}}%
\pgfpathlineto{\pgfqpoint{4.445563in}{0.545381in}}%
\pgfpathlineto{\pgfqpoint{4.446793in}{0.545081in}}%
\pgfpathlineto{\pgfqpoint{4.448433in}{0.540874in}}%
\pgfpathlineto{\pgfqpoint{4.450073in}{0.533736in}}%
\pgfpathlineto{\pgfqpoint{4.450893in}{0.534714in}}%
\pgfpathlineto{\pgfqpoint{4.452943in}{0.535015in}}%
\pgfpathlineto{\pgfqpoint{4.454993in}{0.532470in}}%
\pgfpathlineto{\pgfqpoint{4.456223in}{0.530661in}}%
\pgfpathlineto{\pgfqpoint{4.456633in}{0.531417in}}%
\pgfpathlineto{\pgfqpoint{4.459503in}{0.534192in}}%
\pgfpathlineto{\pgfqpoint{4.461553in}{0.532991in}}%
\pgfpathlineto{\pgfqpoint{4.461963in}{0.532473in}}%
\pgfpathlineto{\pgfqpoint{4.462373in}{0.533721in}}%
\pgfpathlineto{\pgfqpoint{4.465243in}{0.539458in}}%
\pgfpathlineto{\pgfqpoint{4.466473in}{0.538892in}}%
\pgfpathlineto{\pgfqpoint{4.468523in}{0.534384in}}%
\pgfpathlineto{\pgfqpoint{4.469343in}{0.531759in}}%
\pgfpathlineto{\pgfqpoint{4.470163in}{0.532376in}}%
\pgfpathlineto{\pgfqpoint{4.472213in}{0.532527in}}%
\pgfpathlineto{\pgfqpoint{4.474673in}{0.530183in}}%
\pgfpathlineto{\pgfqpoint{4.475083in}{0.529624in}}%
\pgfpathlineto{\pgfqpoint{4.475493in}{0.529980in}}%
\pgfpathlineto{\pgfqpoint{4.478363in}{0.531994in}}%
\pgfpathlineto{\pgfqpoint{4.481643in}{0.532616in}}%
\pgfpathlineto{\pgfqpoint{4.484103in}{0.535382in}}%
\pgfpathlineto{\pgfqpoint{4.485743in}{0.534409in}}%
\pgfpathlineto{\pgfqpoint{4.489432in}{0.530997in}}%
\pgfpathlineto{\pgfqpoint{4.491892in}{0.530311in}}%
\pgfpathlineto{\pgfqpoint{4.494352in}{0.529635in}}%
\pgfpathlineto{\pgfqpoint{4.497222in}{0.530510in}}%
\pgfpathlineto{\pgfqpoint{4.499682in}{0.530764in}}%
\pgfpathlineto{\pgfqpoint{4.502142in}{0.532663in}}%
\pgfpathlineto{\pgfqpoint{4.504192in}{0.531648in}}%
\pgfpathlineto{\pgfqpoint{4.507062in}{0.529853in}}%
\pgfpathlineto{\pgfqpoint{4.510342in}{0.529140in}}%
\pgfpathlineto{\pgfqpoint{4.512392in}{0.529164in}}%
\pgfpathlineto{\pgfqpoint{4.515672in}{0.529424in}}%
\pgfpathlineto{\pgfqpoint{4.517312in}{0.529803in}}%
\pgfpathlineto{\pgfqpoint{4.520182in}{0.530856in}}%
\pgfpathlineto{\pgfqpoint{4.531662in}{0.528970in}}%
\pgfpathlineto{\pgfqpoint{4.551752in}{0.528951in}}%
\pgfpathlineto{\pgfqpoint{4.555442in}{0.528606in}}%
\pgfpathlineto{\pgfqpoint{4.559542in}{0.528337in}}%
\pgfpathlineto{\pgfqpoint{4.567332in}{0.528523in}}%
\pgfpathlineto{\pgfqpoint{4.573892in}{0.528248in}}%
\pgfpathlineto{\pgfqpoint{4.632112in}{0.528025in}}%
\pgfpathlineto{\pgfqpoint{5.277041in}{0.528000in}}%
\pgfpathlineto{\pgfqpoint{5.657521in}{0.528000in}}%
\pgfpathlineto{\pgfqpoint{5.657521in}{0.528000in}}%
\pgfusepath{stroke}%
\end{pgfscope}%
\begin{pgfscope}%
\pgfpathrectangle{\pgfqpoint{3.505455in}{0.528000in}}{\pgfqpoint{2.254545in}{1.680000in}}%
\pgfusepath{clip}%
\pgfsetrectcap%
\pgfsetroundjoin%
\pgfsetlinewidth{1.505625pt}%
\definecolor{currentstroke}{rgb}{0.090196,0.745098,0.811765}%
\pgfsetstrokecolor{currentstroke}%
\pgfsetdash{}{0pt}%
\pgfpathmoveto{\pgfqpoint{3.607934in}{1.461333in}}%
\pgfpathlineto{\pgfqpoint{3.613674in}{1.460244in}}%
\pgfpathlineto{\pgfqpoint{3.619004in}{1.457002in}}%
\pgfpathlineto{\pgfqpoint{3.624744in}{1.450711in}}%
\pgfpathlineto{\pgfqpoint{3.630484in}{1.440837in}}%
\pgfpathlineto{\pgfqpoint{3.636224in}{1.426422in}}%
\pgfpathlineto{\pgfqpoint{3.642374in}{1.404574in}}%
\pgfpathlineto{\pgfqpoint{3.648934in}{1.372075in}}%
\pgfpathlineto{\pgfqpoint{3.655494in}{1.327890in}}%
\pgfpathlineto{\pgfqpoint{3.662464in}{1.265715in}}%
\pgfpathlineto{\pgfqpoint{3.668614in}{1.200940in}}%
\pgfpathlineto{\pgfqpoint{3.669024in}{1.209625in}}%
\pgfpathlineto{\pgfqpoint{3.689934in}{1.677910in}}%
\pgfpathlineto{\pgfqpoint{3.703054in}{1.944907in}}%
\pgfpathlineto{\pgfqpoint{3.711254in}{2.069999in}}%
\pgfpathlineto{\pgfqpoint{3.716994in}{2.128641in}}%
\pgfpathlineto{\pgfqpoint{3.721504in}{2.154492in}}%
\pgfpathlineto{\pgfqpoint{3.724374in}{2.160673in}}%
\pgfpathlineto{\pgfqpoint{3.726014in}{2.160405in}}%
\pgfpathlineto{\pgfqpoint{3.727654in}{2.157281in}}%
\pgfpathlineto{\pgfqpoint{3.730114in}{2.147090in}}%
\pgfpathlineto{\pgfqpoint{3.733394in}{2.122916in}}%
\pgfpathlineto{\pgfqpoint{3.737494in}{2.075143in}}%
\pgfpathlineto{\pgfqpoint{3.742824in}{1.983081in}}%
\pgfpathlineto{\pgfqpoint{3.748974in}{1.834252in}}%
\pgfpathlineto{\pgfqpoint{3.756354in}{1.596472in}}%
\pgfpathlineto{\pgfqpoint{3.765374in}{1.225063in}}%
\pgfpathlineto{\pgfqpoint{3.771114in}{0.956186in}}%
\pgfpathlineto{\pgfqpoint{3.771934in}{0.964713in}}%
\pgfpathlineto{\pgfqpoint{3.783004in}{1.067952in}}%
\pgfpathlineto{\pgfqpoint{3.789974in}{1.114826in}}%
\pgfpathlineto{\pgfqpoint{3.794484in}{1.132437in}}%
\pgfpathlineto{\pgfqpoint{3.797354in}{1.136321in}}%
\pgfpathlineto{\pgfqpoint{3.798994in}{1.135376in}}%
\pgfpathlineto{\pgfqpoint{3.801044in}{1.130419in}}%
\pgfpathlineto{\pgfqpoint{3.803504in}{1.118085in}}%
\pgfpathlineto{\pgfqpoint{3.806784in}{1.088743in}}%
\pgfpathlineto{\pgfqpoint{3.810474in}{1.034301in}}%
\pgfpathlineto{\pgfqpoint{3.814984in}{0.930016in}}%
\pgfpathlineto{\pgfqpoint{3.819494in}{0.777495in}}%
\pgfpathlineto{\pgfqpoint{3.820314in}{0.794780in}}%
\pgfpathlineto{\pgfqpoint{3.836714in}{1.349149in}}%
\pgfpathlineto{\pgfqpoint{3.839584in}{1.379751in}}%
\pgfpathlineto{\pgfqpoint{3.840404in}{1.381307in}}%
\pgfpathlineto{\pgfqpoint{3.840814in}{1.380813in}}%
\pgfpathlineto{\pgfqpoint{3.842044in}{1.374131in}}%
\pgfpathlineto{\pgfqpoint{3.844094in}{1.345420in}}%
\pgfpathlineto{\pgfqpoint{3.846964in}{1.269187in}}%
\pgfpathlineto{\pgfqpoint{3.850654in}{1.132374in}}%
\pgfpathlineto{\pgfqpoint{3.858443in}{1.689607in}}%
\pgfpathlineto{\pgfqpoint{3.863363in}{1.901890in}}%
\pgfpathlineto{\pgfqpoint{3.866643in}{1.969811in}}%
\pgfpathlineto{\pgfqpoint{3.868693in}{1.981443in}}%
\pgfpathlineto{\pgfqpoint{3.869513in}{1.979515in}}%
\pgfpathlineto{\pgfqpoint{3.871153in}{1.964577in}}%
\pgfpathlineto{\pgfqpoint{3.873613in}{1.915247in}}%
\pgfpathlineto{\pgfqpoint{3.877303in}{1.784363in}}%
\pgfpathlineto{\pgfqpoint{3.882223in}{1.516285in}}%
\pgfpathlineto{\pgfqpoint{3.889193in}{0.997703in}}%
\pgfpathlineto{\pgfqpoint{3.890423in}{0.924948in}}%
\pgfpathlineto{\pgfqpoint{3.890833in}{0.932298in}}%
\pgfpathlineto{\pgfqpoint{3.897393in}{1.030448in}}%
\pgfpathlineto{\pgfqpoint{3.901493in}{1.066581in}}%
\pgfpathlineto{\pgfqpoint{3.903953in}{1.074641in}}%
\pgfpathlineto{\pgfqpoint{3.904773in}{1.074493in}}%
\pgfpathlineto{\pgfqpoint{3.906003in}{1.071227in}}%
\pgfpathlineto{\pgfqpoint{3.908053in}{1.056611in}}%
\pgfpathlineto{\pgfqpoint{3.910513in}{1.021289in}}%
\pgfpathlineto{\pgfqpoint{3.913793in}{0.937710in}}%
\pgfpathlineto{\pgfqpoint{3.917893in}{0.764761in}}%
\pgfpathlineto{\pgfqpoint{3.918303in}{0.743141in}}%
\pgfpathlineto{\pgfqpoint{3.918713in}{0.761444in}}%
\pgfpathlineto{\pgfqpoint{3.928963in}{1.212876in}}%
\pgfpathlineto{\pgfqpoint{3.931423in}{1.250288in}}%
\pgfpathlineto{\pgfqpoint{3.932243in}{1.251630in}}%
\pgfpathlineto{\pgfqpoint{3.933063in}{1.247033in}}%
\pgfpathlineto{\pgfqpoint{3.934703in}{1.219818in}}%
\pgfpathlineto{\pgfqpoint{3.937163in}{1.136007in}}%
\pgfpathlineto{\pgfqpoint{3.939213in}{1.044129in}}%
\pgfpathlineto{\pgfqpoint{3.945363in}{1.573440in}}%
\pgfpathlineto{\pgfqpoint{3.949053in}{1.741163in}}%
\pgfpathlineto{\pgfqpoint{3.951513in}{1.780217in}}%
\pgfpathlineto{\pgfqpoint{3.951923in}{1.781044in}}%
\pgfpathlineto{\pgfqpoint{3.952333in}{1.780271in}}%
\pgfpathlineto{\pgfqpoint{3.953563in}{1.768488in}}%
\pgfpathlineto{\pgfqpoint{3.955613in}{1.718386in}}%
\pgfpathlineto{\pgfqpoint{3.958893in}{1.565561in}}%
\pgfpathlineto{\pgfqpoint{3.963403in}{1.234908in}}%
\pgfpathlineto{\pgfqpoint{3.967503in}{0.883105in}}%
\pgfpathlineto{\pgfqpoint{3.968323in}{0.900733in}}%
\pgfpathlineto{\pgfqpoint{3.973243in}{0.982218in}}%
\pgfpathlineto{\pgfqpoint{3.976113in}{1.004774in}}%
\pgfpathlineto{\pgfqpoint{3.977343in}{1.006933in}}%
\pgfpathlineto{\pgfqpoint{3.977753in}{1.006492in}}%
\pgfpathlineto{\pgfqpoint{3.978983in}{1.001348in}}%
\pgfpathlineto{\pgfqpoint{3.981033in}{0.978482in}}%
\pgfpathlineto{\pgfqpoint{3.983903in}{0.910937in}}%
\pgfpathlineto{\pgfqpoint{3.987183in}{0.774838in}}%
\pgfpathlineto{\pgfqpoint{3.988413in}{0.716301in}}%
\pgfpathlineto{\pgfqpoint{3.988823in}{0.737889in}}%
\pgfpathlineto{\pgfqpoint{3.996613in}{1.093661in}}%
\pgfpathlineto{\pgfqpoint{3.999073in}{1.126709in}}%
\pgfpathlineto{\pgfqpoint{3.999893in}{1.123712in}}%
\pgfpathlineto{\pgfqpoint{4.001533in}{1.095720in}}%
\pgfpathlineto{\pgfqpoint{4.003993in}{1.002407in}}%
\pgfpathlineto{\pgfqpoint{4.004813in}{0.959763in}}%
\pgfpathlineto{\pgfqpoint{4.005223in}{0.982192in}}%
\pgfpathlineto{\pgfqpoint{4.010553in}{1.444546in}}%
\pgfpathlineto{\pgfqpoint{4.013833in}{1.570489in}}%
\pgfpathlineto{\pgfqpoint{4.015063in}{1.582638in}}%
\pgfpathlineto{\pgfqpoint{4.015473in}{1.582460in}}%
\pgfpathlineto{\pgfqpoint{4.016703in}{1.569506in}}%
\pgfpathlineto{\pgfqpoint{4.018753in}{1.508304in}}%
\pgfpathlineto{\pgfqpoint{4.022033in}{1.318880in}}%
\pgfpathlineto{\pgfqpoint{4.026953in}{0.879343in}}%
\pgfpathlineto{\pgfqpoint{4.027363in}{0.837284in}}%
\pgfpathlineto{\pgfqpoint{4.028183in}{0.849740in}}%
\pgfpathlineto{\pgfqpoint{4.032283in}{0.921098in}}%
\pgfpathlineto{\pgfqpoint{4.034743in}{0.937824in}}%
\pgfpathlineto{\pgfqpoint{4.035563in}{0.937893in}}%
\pgfpathlineto{\pgfqpoint{4.036793in}{0.932029in}}%
\pgfpathlineto{\pgfqpoint{4.038843in}{0.904280in}}%
\pgfpathlineto{\pgfqpoint{4.041713in}{0.821845in}}%
\pgfpathlineto{\pgfqpoint{4.044583in}{0.684908in}}%
\pgfpathlineto{\pgfqpoint{4.044993in}{0.701376in}}%
\pgfpathlineto{\pgfqpoint{4.051553in}{0.986251in}}%
\pgfpathlineto{\pgfqpoint{4.053603in}{1.013141in}}%
\pgfpathlineto{\pgfqpoint{4.054013in}{1.012990in}}%
\pgfpathlineto{\pgfqpoint{4.054833in}{1.006881in}}%
\pgfpathlineto{\pgfqpoint{4.056473in}{0.971711in}}%
\pgfpathlineto{\pgfqpoint{4.058933in}{0.885948in}}%
\pgfpathlineto{\pgfqpoint{4.063853in}{1.296397in}}%
\pgfpathlineto{\pgfqpoint{4.066723in}{1.391796in}}%
\pgfpathlineto{\pgfqpoint{4.067543in}{1.397180in}}%
\pgfpathlineto{\pgfqpoint{4.067953in}{1.396252in}}%
\pgfpathlineto{\pgfqpoint{4.069183in}{1.379312in}}%
\pgfpathlineto{\pgfqpoint{4.071233in}{1.306409in}}%
\pgfpathlineto{\pgfqpoint{4.074513in}{1.090529in}}%
\pgfpathlineto{\pgfqpoint{4.077793in}{0.788726in}}%
\pgfpathlineto{\pgfqpoint{4.078613in}{0.804189in}}%
\pgfpathlineto{\pgfqpoint{4.082303in}{0.863395in}}%
\pgfpathlineto{\pgfqpoint{4.084353in}{0.871653in}}%
\pgfpathlineto{\pgfqpoint{4.085173in}{0.868945in}}%
\pgfpathlineto{\pgfqpoint{4.086813in}{0.851843in}}%
\pgfpathlineto{\pgfqpoint{4.089273in}{0.793774in}}%
\pgfpathlineto{\pgfqpoint{4.092553in}{0.656067in}}%
\pgfpathlineto{\pgfqpoint{4.092963in}{0.675897in}}%
\pgfpathlineto{\pgfqpoint{4.098293in}{0.887097in}}%
\pgfpathlineto{\pgfqpoint{4.100343in}{0.913959in}}%
\pgfpathlineto{\pgfqpoint{4.100753in}{0.913988in}}%
\pgfpathlineto{\pgfqpoint{4.101573in}{0.908430in}}%
\pgfpathlineto{\pgfqpoint{4.103213in}{0.875247in}}%
\pgfpathlineto{\pgfqpoint{4.104853in}{0.815732in}}%
\pgfpathlineto{\pgfqpoint{4.105263in}{0.820504in}}%
\pgfpathlineto{\pgfqpoint{4.109773in}{1.158830in}}%
\pgfpathlineto{\pgfqpoint{4.112233in}{1.227004in}}%
\pgfpathlineto{\pgfqpoint{4.112643in}{1.229449in}}%
\pgfpathlineto{\pgfqpoint{4.113053in}{1.229357in}}%
\pgfpathlineto{\pgfqpoint{4.113873in}{1.221669in}}%
\pgfpathlineto{\pgfqpoint{4.115513in}{1.177483in}}%
\pgfpathlineto{\pgfqpoint{4.118383in}{1.018867in}}%
\pgfpathlineto{\pgfqpoint{4.121663in}{0.746710in}}%
\pgfpathlineto{\pgfqpoint{4.122893in}{0.766512in}}%
\pgfpathlineto{\pgfqpoint{4.126173in}{0.807613in}}%
\pgfpathlineto{\pgfqpoint{4.126993in}{0.809728in}}%
\pgfpathlineto{\pgfqpoint{4.127403in}{0.809399in}}%
\pgfpathlineto{\pgfqpoint{4.128633in}{0.802520in}}%
\pgfpathlineto{\pgfqpoint{4.130683in}{0.769948in}}%
\pgfpathlineto{\pgfqpoint{4.133553in}{0.678142in}}%
\pgfpathlineto{\pgfqpoint{4.134783in}{0.635548in}}%
\pgfpathlineto{\pgfqpoint{4.135193in}{0.653134in}}%
\pgfpathlineto{\pgfqpoint{4.140113in}{0.814940in}}%
\pgfpathlineto{\pgfqpoint{4.141753in}{0.829900in}}%
\pgfpathlineto{\pgfqpoint{4.142163in}{0.829490in}}%
\pgfpathlineto{\pgfqpoint{4.143393in}{0.818018in}}%
\pgfpathlineto{\pgfqpoint{4.145443in}{0.766845in}}%
\pgfpathlineto{\pgfqpoint{4.145853in}{0.752455in}}%
\pgfpathlineto{\pgfqpoint{4.146263in}{0.763410in}}%
\pgfpathlineto{\pgfqpoint{4.150363in}{1.030809in}}%
\pgfpathlineto{\pgfqpoint{4.152823in}{1.082908in}}%
\pgfpathlineto{\pgfqpoint{4.153233in}{1.082831in}}%
\pgfpathlineto{\pgfqpoint{4.154053in}{1.075286in}}%
\pgfpathlineto{\pgfqpoint{4.155693in}{1.031940in}}%
\pgfpathlineto{\pgfqpoint{4.158563in}{0.878244in}}%
\pgfpathlineto{\pgfqpoint{4.161023in}{0.702231in}}%
\pgfpathlineto{\pgfqpoint{4.161843in}{0.718515in}}%
\pgfpathlineto{\pgfqpoint{4.164713in}{0.751885in}}%
\pgfpathlineto{\pgfqpoint{4.165533in}{0.753752in}}%
\pgfpathlineto{\pgfqpoint{4.165943in}{0.753273in}}%
\pgfpathlineto{\pgfqpoint{4.167173in}{0.745912in}}%
\pgfpathlineto{\pgfqpoint{4.169223in}{0.713009in}}%
\pgfpathlineto{\pgfqpoint{4.172503in}{0.609031in}}%
\pgfpathlineto{\pgfqpoint{4.173323in}{0.637485in}}%
\pgfpathlineto{\pgfqpoint{4.177423in}{0.747931in}}%
\pgfpathlineto{\pgfqpoint{4.179063in}{0.759996in}}%
\pgfpathlineto{\pgfqpoint{4.179883in}{0.757205in}}%
\pgfpathlineto{\pgfqpoint{4.181523in}{0.734050in}}%
\pgfpathlineto{\pgfqpoint{4.182753in}{0.702872in}}%
\pgfpathlineto{\pgfqpoint{4.183163in}{0.706315in}}%
\pgfpathlineto{\pgfqpoint{4.187263in}{0.927349in}}%
\pgfpathlineto{\pgfqpoint{4.189313in}{0.958462in}}%
\pgfpathlineto{\pgfqpoint{4.190133in}{0.954562in}}%
\pgfpathlineto{\pgfqpoint{4.191773in}{0.920003in}}%
\pgfpathlineto{\pgfqpoint{4.194643in}{0.786406in}}%
\pgfpathlineto{\pgfqpoint{4.196693in}{0.666630in}}%
\pgfpathlineto{\pgfqpoint{4.197103in}{0.673907in}}%
\pgfpathlineto{\pgfqpoint{4.199973in}{0.703753in}}%
\pgfpathlineto{\pgfqpoint{4.200793in}{0.704712in}}%
\pgfpathlineto{\pgfqpoint{4.201203in}{0.703818in}}%
\pgfpathlineto{\pgfqpoint{4.202433in}{0.695476in}}%
\pgfpathlineto{\pgfqpoint{4.204483in}{0.662548in}}%
\pgfpathlineto{\pgfqpoint{4.206943in}{0.594888in}}%
\pgfpathlineto{\pgfqpoint{4.207763in}{0.610073in}}%
\pgfpathlineto{\pgfqpoint{4.211453in}{0.692542in}}%
\pgfpathlineto{\pgfqpoint{4.213093in}{0.703148in}}%
\pgfpathlineto{\pgfqpoint{4.213913in}{0.700954in}}%
\pgfpathlineto{\pgfqpoint{4.215553in}{0.681746in}}%
\pgfpathlineto{\pgfqpoint{4.216783in}{0.655829in}}%
\pgfpathlineto{\pgfqpoint{4.221703in}{0.848554in}}%
\pgfpathlineto{\pgfqpoint{4.222933in}{0.854914in}}%
\pgfpathlineto{\pgfqpoint{4.223753in}{0.848759in}}%
\pgfpathlineto{\pgfqpoint{4.225393in}{0.812978in}}%
\pgfpathlineto{\pgfqpoint{4.228263in}{0.689091in}}%
\pgfpathlineto{\pgfqpoint{4.229493in}{0.635512in}}%
\pgfpathlineto{\pgfqpoint{4.229903in}{0.641530in}}%
\pgfpathlineto{\pgfqpoint{4.232363in}{0.662530in}}%
\pgfpathlineto{\pgfqpoint{4.233183in}{0.663215in}}%
\pgfpathlineto{\pgfqpoint{4.234413in}{0.657903in}}%
\pgfpathlineto{\pgfqpoint{4.236463in}{0.632201in}}%
\pgfpathlineto{\pgfqpoint{4.238923in}{0.577375in}}%
\pgfpathlineto{\pgfqpoint{4.239743in}{0.592188in}}%
\pgfpathlineto{\pgfqpoint{4.243433in}{0.653391in}}%
\pgfpathlineto{\pgfqpoint{4.244663in}{0.657936in}}%
\pgfpathlineto{\pgfqpoint{4.245073in}{0.657369in}}%
\pgfpathlineto{\pgfqpoint{4.246303in}{0.649464in}}%
\pgfpathlineto{\pgfqpoint{4.247943in}{0.625706in}}%
\pgfpathlineto{\pgfqpoint{4.248353in}{0.631046in}}%
\pgfpathlineto{\pgfqpoint{4.252043in}{0.757987in}}%
\pgfpathlineto{\pgfqpoint{4.253683in}{0.771700in}}%
\pgfpathlineto{\pgfqpoint{4.254503in}{0.767757in}}%
\pgfpathlineto{\pgfqpoint{4.256143in}{0.739504in}}%
\pgfpathlineto{\pgfqpoint{4.259013in}{0.637750in}}%
\pgfpathlineto{\pgfqpoint{4.259833in}{0.607907in}}%
\pgfpathlineto{\pgfqpoint{4.260243in}{0.612882in}}%
\pgfpathlineto{\pgfqpoint{4.262703in}{0.629067in}}%
\pgfpathlineto{\pgfqpoint{4.263523in}{0.628793in}}%
\pgfpathlineto{\pgfqpoint{4.264753in}{0.622837in}}%
\pgfpathlineto{\pgfqpoint{4.266803in}{0.598811in}}%
\pgfpathlineto{\pgfqpoint{4.268853in}{0.564236in}}%
\pgfpathlineto{\pgfqpoint{4.269263in}{0.572113in}}%
\pgfpathlineto{\pgfqpoint{4.272953in}{0.619836in}}%
\pgfpathlineto{\pgfqpoint{4.274183in}{0.622624in}}%
\pgfpathlineto{\pgfqpoint{4.275003in}{0.620275in}}%
\pgfpathlineto{\pgfqpoint{4.276643in}{0.606028in}}%
\pgfpathlineto{\pgfqpoint{4.277463in}{0.599667in}}%
\pgfpathlineto{\pgfqpoint{4.281153in}{0.697735in}}%
\pgfpathlineto{\pgfqpoint{4.282383in}{0.705894in}}%
\pgfpathlineto{\pgfqpoint{4.282793in}{0.705599in}}%
\pgfpathlineto{\pgfqpoint{4.284023in}{0.695914in}}%
\pgfpathlineto{\pgfqpoint{4.286073in}{0.653536in}}%
\pgfpathlineto{\pgfqpoint{4.288123in}{0.587560in}}%
\pgfpathlineto{\pgfqpoint{4.288943in}{0.592643in}}%
\pgfpathlineto{\pgfqpoint{4.290993in}{0.602050in}}%
\pgfpathlineto{\pgfqpoint{4.291813in}{0.601549in}}%
\pgfpathlineto{\pgfqpoint{4.293043in}{0.596158in}}%
\pgfpathlineto{\pgfqpoint{4.295503in}{0.570327in}}%
\pgfpathlineto{\pgfqpoint{4.296733in}{0.554142in}}%
\pgfpathlineto{\pgfqpoint{4.297143in}{0.560168in}}%
\pgfpathlineto{\pgfqpoint{4.300423in}{0.593057in}}%
\pgfpathlineto{\pgfqpoint{4.301653in}{0.595770in}}%
\pgfpathlineto{\pgfqpoint{4.302063in}{0.595344in}}%
\pgfpathlineto{\pgfqpoint{4.303293in}{0.590171in}}%
\pgfpathlineto{\pgfqpoint{4.304523in}{0.579685in}}%
\pgfpathlineto{\pgfqpoint{4.304933in}{0.580615in}}%
\pgfpathlineto{\pgfqpoint{4.308213in}{0.648068in}}%
\pgfpathlineto{\pgfqpoint{4.309853in}{0.655238in}}%
\pgfpathlineto{\pgfqpoint{4.311083in}{0.648012in}}%
\pgfpathlineto{\pgfqpoint{4.313133in}{0.614896in}}%
\pgfpathlineto{\pgfqpoint{4.315183in}{0.570969in}}%
\pgfpathlineto{\pgfqpoint{4.315593in}{0.573830in}}%
\pgfpathlineto{\pgfqpoint{4.317643in}{0.581077in}}%
\pgfpathlineto{\pgfqpoint{4.318463in}{0.580477in}}%
\pgfpathlineto{\pgfqpoint{4.320103in}{0.573352in}}%
\pgfpathlineto{\pgfqpoint{4.322973in}{0.545926in}}%
\pgfpathlineto{\pgfqpoint{4.323793in}{0.554730in}}%
\pgfpathlineto{\pgfqpoint{4.326663in}{0.574242in}}%
\pgfpathlineto{\pgfqpoint{4.327893in}{0.575525in}}%
\pgfpathlineto{\pgfqpoint{4.329123in}{0.572236in}}%
\pgfpathlineto{\pgfqpoint{4.330763in}{0.563837in}}%
\pgfpathlineto{\pgfqpoint{4.334043in}{0.613220in}}%
\pgfpathlineto{\pgfqpoint{4.335273in}{0.617438in}}%
\pgfpathlineto{\pgfqpoint{4.335683in}{0.616935in}}%
\pgfpathlineto{\pgfqpoint{4.336913in}{0.609920in}}%
\pgfpathlineto{\pgfqpoint{4.339373in}{0.575146in}}%
\pgfpathlineto{\pgfqpoint{4.340603in}{0.558736in}}%
\pgfpathlineto{\pgfqpoint{4.341013in}{0.560770in}}%
\pgfpathlineto{\pgfqpoint{4.343063in}{0.565235in}}%
\pgfpathlineto{\pgfqpoint{4.344293in}{0.563159in}}%
\pgfpathlineto{\pgfqpoint{4.346343in}{0.552509in}}%
\pgfpathlineto{\pgfqpoint{4.347983in}{0.541315in}}%
\pgfpathlineto{\pgfqpoint{4.348393in}{0.544570in}}%
\pgfpathlineto{\pgfqpoint{4.351263in}{0.559678in}}%
\pgfpathlineto{\pgfqpoint{4.352493in}{0.560827in}}%
\pgfpathlineto{\pgfqpoint{4.352903in}{0.560433in}}%
\pgfpathlineto{\pgfqpoint{4.354543in}{0.555216in}}%
\pgfpathlineto{\pgfqpoint{4.354953in}{0.553099in}}%
\pgfpathlineto{\pgfqpoint{4.355363in}{0.553712in}}%
\pgfpathlineto{\pgfqpoint{4.358643in}{0.587725in}}%
\pgfpathlineto{\pgfqpoint{4.359463in}{0.589549in}}%
\pgfpathlineto{\pgfqpoint{4.359873in}{0.589367in}}%
\pgfpathlineto{\pgfqpoint{4.361103in}{0.584594in}}%
\pgfpathlineto{\pgfqpoint{4.363563in}{0.559239in}}%
\pgfpathlineto{\pgfqpoint{4.364383in}{0.548098in}}%
\pgfpathlineto{\pgfqpoint{4.365203in}{0.551092in}}%
\pgfpathlineto{\pgfqpoint{4.367253in}{0.553438in}}%
\pgfpathlineto{\pgfqpoint{4.368483in}{0.551225in}}%
\pgfpathlineto{\pgfqpoint{4.370943in}{0.540067in}}%
\pgfpathlineto{\pgfqpoint{4.371353in}{0.537602in}}%
\pgfpathlineto{\pgfqpoint{4.371763in}{0.537705in}}%
\pgfpathlineto{\pgfqpoint{4.375043in}{0.549860in}}%
\pgfpathlineto{\pgfqpoint{4.376273in}{0.550034in}}%
\pgfpathlineto{\pgfqpoint{4.377913in}{0.546418in}}%
\pgfpathlineto{\pgfqpoint{4.378323in}{0.544914in}}%
\pgfpathlineto{\pgfqpoint{4.378733in}{0.545752in}}%
\pgfpathlineto{\pgfqpoint{4.382013in}{0.568816in}}%
\pgfpathlineto{\pgfqpoint{4.382833in}{0.569545in}}%
\pgfpathlineto{\pgfqpoint{4.383243in}{0.569099in}}%
\pgfpathlineto{\pgfqpoint{4.384473in}{0.564691in}}%
\pgfpathlineto{\pgfqpoint{4.387343in}{0.541701in}}%
\pgfpathlineto{\pgfqpoint{4.388573in}{0.544447in}}%
\pgfpathlineto{\pgfqpoint{4.390213in}{0.544987in}}%
\pgfpathlineto{\pgfqpoint{4.391853in}{0.542028in}}%
\pgfpathlineto{\pgfqpoint{4.394313in}{0.534414in}}%
\pgfpathlineto{\pgfqpoint{4.394723in}{0.536008in}}%
\pgfpathlineto{\pgfqpoint{4.397593in}{0.542643in}}%
\pgfpathlineto{\pgfqpoint{4.398823in}{0.542512in}}%
\pgfpathlineto{\pgfqpoint{4.400873in}{0.538797in}}%
\pgfpathlineto{\pgfqpoint{4.403743in}{0.554269in}}%
\pgfpathlineto{\pgfqpoint{4.404973in}{0.555493in}}%
\pgfpathlineto{\pgfqpoint{4.405383in}{0.555117in}}%
\pgfpathlineto{\pgfqpoint{4.407023in}{0.550013in}}%
\pgfpathlineto{\pgfqpoint{4.409483in}{0.537554in}}%
\pgfpathlineto{\pgfqpoint{4.410303in}{0.538757in}}%
\pgfpathlineto{\pgfqpoint{4.411943in}{0.539211in}}%
\pgfpathlineto{\pgfqpoint{4.413583in}{0.537141in}}%
\pgfpathlineto{\pgfqpoint{4.415633in}{0.531932in}}%
\pgfpathlineto{\pgfqpoint{4.416453in}{0.533666in}}%
\pgfpathlineto{\pgfqpoint{4.419323in}{0.537652in}}%
\pgfpathlineto{\pgfqpoint{4.420963in}{0.536796in}}%
\pgfpathlineto{\pgfqpoint{4.422193in}{0.535084in}}%
\pgfpathlineto{\pgfqpoint{4.425063in}{0.545280in}}%
\pgfpathlineto{\pgfqpoint{4.426293in}{0.545776in}}%
\pgfpathlineto{\pgfqpoint{4.427933in}{0.542721in}}%
\pgfpathlineto{\pgfqpoint{4.430393in}{0.534141in}}%
\pgfpathlineto{\pgfqpoint{4.431213in}{0.534961in}}%
\pgfpathlineto{\pgfqpoint{4.433263in}{0.534999in}}%
\pgfpathlineto{\pgfqpoint{4.435723in}{0.531701in}}%
\pgfpathlineto{\pgfqpoint{4.436543in}{0.530640in}}%
\pgfpathlineto{\pgfqpoint{4.436953in}{0.531371in}}%
\pgfpathlineto{\pgfqpoint{4.439823in}{0.534179in}}%
\pgfpathlineto{\pgfqpoint{4.441873in}{0.533241in}}%
\pgfpathlineto{\pgfqpoint{4.442693in}{0.532752in}}%
\pgfpathlineto{\pgfqpoint{4.445563in}{0.539171in}}%
\pgfpathlineto{\pgfqpoint{4.447203in}{0.538840in}}%
\pgfpathlineto{\pgfqpoint{4.449253in}{0.534668in}}%
\pgfpathlineto{\pgfqpoint{4.450483in}{0.531885in}}%
\pgfpathlineto{\pgfqpoint{4.450893in}{0.532194in}}%
\pgfpathlineto{\pgfqpoint{4.453353in}{0.532390in}}%
\pgfpathlineto{\pgfqpoint{4.457453in}{0.530787in}}%
\pgfpathlineto{\pgfqpoint{4.460323in}{0.531836in}}%
\pgfpathlineto{\pgfqpoint{4.462783in}{0.531963in}}%
\pgfpathlineto{\pgfqpoint{4.465653in}{0.535131in}}%
\pgfpathlineto{\pgfqpoint{4.467293in}{0.534188in}}%
\pgfpathlineto{\pgfqpoint{4.470983in}{0.530845in}}%
\pgfpathlineto{\pgfqpoint{4.473853in}{0.530110in}}%
\pgfpathlineto{\pgfqpoint{4.476313in}{0.529588in}}%
\pgfpathlineto{\pgfqpoint{4.479593in}{0.530312in}}%
\pgfpathlineto{\pgfqpoint{4.481643in}{0.530401in}}%
\pgfpathlineto{\pgfqpoint{4.484513in}{0.532375in}}%
\pgfpathlineto{\pgfqpoint{4.486563in}{0.531336in}}%
\pgfpathlineto{\pgfqpoint{4.489432in}{0.529719in}}%
\pgfpathlineto{\pgfqpoint{4.492712in}{0.529126in}}%
\pgfpathlineto{\pgfqpoint{4.495172in}{0.529131in}}%
\pgfpathlineto{\pgfqpoint{4.498862in}{0.529188in}}%
\pgfpathlineto{\pgfqpoint{4.500092in}{0.529635in}}%
\pgfpathlineto{\pgfqpoint{4.502962in}{0.530610in}}%
\pgfpathlineto{\pgfqpoint{4.505832in}{0.529154in}}%
\pgfpathlineto{\pgfqpoint{4.507472in}{0.529037in}}%
\pgfpathlineto{\pgfqpoint{4.523052in}{0.528762in}}%
\pgfpathlineto{\pgfqpoint{4.525922in}{0.528607in}}%
\pgfpathlineto{\pgfqpoint{4.548062in}{0.528285in}}%
\pgfpathlineto{\pgfqpoint{4.685412in}{0.528000in}}%
\pgfpathlineto{\pgfqpoint{5.657521in}{0.528000in}}%
\pgfpathlineto{\pgfqpoint{5.657521in}{0.528000in}}%
\pgfusepath{stroke}%
\end{pgfscope}%
\begin{pgfscope}%
\pgfsetrectcap%
\pgfsetmiterjoin%
\pgfsetlinewidth{0.803000pt}%
\definecolor{currentstroke}{rgb}{0.000000,0.000000,0.000000}%
\pgfsetstrokecolor{currentstroke}%
\pgfsetdash{}{0pt}%
\pgfpathmoveto{\pgfqpoint{3.505455in}{0.528000in}}%
\pgfpathlineto{\pgfqpoint{3.505455in}{2.208000in}}%
\pgfusepath{stroke}%
\end{pgfscope}%
\begin{pgfscope}%
\pgfsetrectcap%
\pgfsetmiterjoin%
\pgfsetlinewidth{0.803000pt}%
\definecolor{currentstroke}{rgb}{0.000000,0.000000,0.000000}%
\pgfsetstrokecolor{currentstroke}%
\pgfsetdash{}{0pt}%
\pgfpathmoveto{\pgfqpoint{5.760000in}{0.528000in}}%
\pgfpathlineto{\pgfqpoint{5.760000in}{2.208000in}}%
\pgfusepath{stroke}%
\end{pgfscope}%
\begin{pgfscope}%
\pgfsetrectcap%
\pgfsetmiterjoin%
\pgfsetlinewidth{0.803000pt}%
\definecolor{currentstroke}{rgb}{0.000000,0.000000,0.000000}%
\pgfsetstrokecolor{currentstroke}%
\pgfsetdash{}{0pt}%
\pgfpathmoveto{\pgfqpoint{3.505455in}{0.528000in}}%
\pgfpathlineto{\pgfqpoint{5.760000in}{0.528000in}}%
\pgfusepath{stroke}%
\end{pgfscope}%
\begin{pgfscope}%
\pgfsetrectcap%
\pgfsetmiterjoin%
\pgfsetlinewidth{0.803000pt}%
\definecolor{currentstroke}{rgb}{0.000000,0.000000,0.000000}%
\pgfsetstrokecolor{currentstroke}%
\pgfsetdash{}{0pt}%
\pgfpathmoveto{\pgfqpoint{3.505455in}{2.208000in}}%
\pgfpathlineto{\pgfqpoint{5.760000in}{2.208000in}}%
\pgfusepath{stroke}%
\end{pgfscope}%
\begin{pgfscope}%
\pgfsetbuttcap%
\pgfsetmiterjoin%
\definecolor{currentfill}{rgb}{1.000000,1.000000,1.000000}%
\pgfsetfillcolor{currentfill}%
\pgfsetfillopacity{0.800000}%
\pgfsetlinewidth{1.003750pt}%
\definecolor{currentstroke}{rgb}{0.800000,0.800000,0.800000}%
\pgfsetstrokecolor{currentstroke}%
\pgfsetstrokeopacity{0.800000}%
\pgfsetdash{}{0pt}%
\pgfpathmoveto{\pgfqpoint{-2.001925in}{0.324016in}}%
\pgfpathlineto{\pgfqpoint{11.267379in}{0.324016in}}%
\pgfpathquadraticcurveto{\pgfqpoint{11.295157in}{0.324016in}}{\pgfqpoint{11.295157in}{0.351794in}}%
\pgfpathlineto{\pgfqpoint{11.295157in}{0.531578in}}%
\pgfpathquadraticcurveto{\pgfqpoint{11.295157in}{0.559356in}}{\pgfqpoint{11.267379in}{0.559356in}}%
\pgfpathlineto{\pgfqpoint{-2.001925in}{0.559356in}}%
\pgfpathquadraticcurveto{\pgfqpoint{-2.029702in}{0.559356in}}{\pgfqpoint{-2.029702in}{0.531578in}}%
\pgfpathlineto{\pgfqpoint{-2.029702in}{0.351794in}}%
\pgfpathquadraticcurveto{\pgfqpoint{-2.029702in}{0.324016in}}{\pgfqpoint{-2.001925in}{0.324016in}}%
\pgfpathlineto{\pgfqpoint{-2.001925in}{0.324016in}}%
\pgfpathclose%
\pgfusepath{stroke,fill}%
\end{pgfscope}%
\begin{pgfscope}%
\pgfsetrectcap%
\pgfsetroundjoin%
\pgfsetlinewidth{1.505625pt}%
\definecolor{currentstroke}{rgb}{0.121569,0.466667,0.705882}%
\pgfsetstrokecolor{currentstroke}%
\pgfsetdash{}{0pt}%
\pgfpathmoveto{\pgfqpoint{-1.974147in}{0.455189in}}%
\pgfpathlineto{\pgfqpoint{-1.835258in}{0.455189in}}%
\pgfpathlineto{\pgfqpoint{-1.696369in}{0.455189in}}%
\pgfusepath{stroke}%
\end{pgfscope}%
\begin{pgfscope}%
\definecolor{textcolor}{rgb}{0.000000,0.000000,0.000000}%
\pgfsetstrokecolor{textcolor}%
\pgfsetfillcolor{textcolor}%
\pgftext[x=-1.585258in,y=0.406578in,left,base]{\color{textcolor}\rmfamily\fontsize{10.000000}{12.000000}\selectfont MMD}%
\end{pgfscope}%
\begin{pgfscope}%
\pgfsetrectcap%
\pgfsetroundjoin%
\pgfsetlinewidth{1.505625pt}%
\definecolor{currentstroke}{rgb}{1.000000,0.498039,0.054902}%
\pgfsetstrokecolor{currentstroke}%
\pgfsetdash{}{0pt}%
\pgfpathmoveto{\pgfqpoint{-0.946754in}{0.455189in}}%
\pgfpathlineto{\pgfqpoint{-0.807865in}{0.455189in}}%
\pgfpathlineto{\pgfqpoint{-0.668976in}{0.455189in}}%
\pgfusepath{stroke}%
\end{pgfscope}%
\begin{pgfscope}%
\definecolor{textcolor}{rgb}{0.000000,0.000000,0.000000}%
\pgfsetstrokecolor{textcolor}%
\pgfsetfillcolor{textcolor}%
\pgftext[x=-0.557865in,y=0.406578in,left,base]{\color{textcolor}\rmfamily\fontsize{10.000000}{12.000000}\selectfont FMMD}%
\end{pgfscope}%
\begin{pgfscope}%
\pgfsetrectcap%
\pgfsetroundjoin%
\pgfsetlinewidth{1.505625pt}%
\definecolor{currentstroke}{rgb}{0.172549,0.627451,0.172549}%
\pgfsetstrokecolor{currentstroke}%
\pgfsetdash{}{0pt}%
\pgfpathmoveto{\pgfqpoint{0.171302in}{0.455189in}}%
\pgfpathlineto{\pgfqpoint{0.310191in}{0.455189in}}%
\pgfpathlineto{\pgfqpoint{0.449080in}{0.455189in}}%
\pgfusepath{stroke}%
\end{pgfscope}%
\begin{pgfscope}%
\definecolor{textcolor}{rgb}{0.000000,0.000000,0.000000}%
\pgfsetstrokecolor{textcolor}%
\pgfsetfillcolor{textcolor}%
\pgftext[x=0.560191in,y=0.406578in,left,base]{\color{textcolor}\rmfamily\fontsize{10.000000}{12.000000}\selectfont FMMD-eg}%
\end{pgfscope}%
\begin{pgfscope}%
\pgfsetrectcap%
\pgfsetroundjoin%
\pgfsetlinewidth{1.505625pt}%
\definecolor{currentstroke}{rgb}{0.839216,0.152941,0.156863}%
\pgfsetstrokecolor{currentstroke}%
\pgfsetdash{}{0pt}%
\pgfpathmoveto{\pgfqpoint{1.466828in}{0.455189in}}%
\pgfpathlineto{\pgfqpoint{1.605717in}{0.455189in}}%
\pgfpathlineto{\pgfqpoint{1.744606in}{0.455189in}}%
\pgfusepath{stroke}%
\end{pgfscope}%
\begin{pgfscope}%
\definecolor{textcolor}{rgb}{0.000000,0.000000,0.000000}%
\pgfsetstrokecolor{textcolor}%
\pgfsetfillcolor{textcolor}%
\pgftext[x=1.855717in,y=0.406578in,left,base]{\color{textcolor}\rmfamily\fontsize{10.000000}{12.000000}\selectfont FMMD-nrd}%
\end{pgfscope}%
\begin{pgfscope}%
\pgfsetrectcap%
\pgfsetroundjoin%
\pgfsetlinewidth{1.505625pt}%
\definecolor{currentstroke}{rgb}{0.580392,0.403922,0.741176}%
\pgfsetstrokecolor{currentstroke}%
\pgfsetdash{}{0pt}%
\pgfpathmoveto{\pgfqpoint{2.839901in}{0.455189in}}%
\pgfpathlineto{\pgfqpoint{2.978790in}{0.455189in}}%
\pgfpathlineto{\pgfqpoint{3.117679in}{0.455189in}}%
\pgfusepath{stroke}%
\end{pgfscope}%
\begin{pgfscope}%
\definecolor{textcolor}{rgb}{0.000000,0.000000,0.000000}%
\pgfsetstrokecolor{textcolor}%
\pgfsetfillcolor{textcolor}%
\pgftext[x=3.228790in,y=0.406578in,left,base]{\color{textcolor}\rmfamily\fontsize{10.000000}{12.000000}\selectfont FMMD-nrd-eg}%
\end{pgfscope}%
\begin{pgfscope}%
\pgfsetrectcap%
\pgfsetroundjoin%
\pgfsetlinewidth{1.505625pt}%
\definecolor{currentstroke}{rgb}{0.549020,0.337255,0.294118}%
\pgfsetstrokecolor{currentstroke}%
\pgfsetdash{}{0pt}%
\pgfpathmoveto{\pgfqpoint{4.390442in}{0.455189in}}%
\pgfpathlineto{\pgfqpoint{4.529331in}{0.455189in}}%
\pgfpathlineto{\pgfqpoint{4.668220in}{0.455189in}}%
\pgfusepath{stroke}%
\end{pgfscope}%
\begin{pgfscope}%
\definecolor{textcolor}{rgb}{0.000000,0.000000,0.000000}%
\pgfsetstrokecolor{textcolor}%
\pgfsetfillcolor{textcolor}%
\pgftext[x=4.779331in,y=0.406578in,left,base]{\color{textcolor}\rmfamily\fontsize{10.000000}{12.000000}\selectfont MDPO}%
\end{pgfscope}%
\begin{pgfscope}%
\pgfsetrectcap%
\pgfsetroundjoin%
\pgfsetlinewidth{1.505625pt}%
\definecolor{currentstroke}{rgb}{0.890196,0.466667,0.760784}%
\pgfsetstrokecolor{currentstroke}%
\pgfsetdash{}{0pt}%
\pgfpathmoveto{\pgfqpoint{5.493067in}{0.455189in}}%
\pgfpathlineto{\pgfqpoint{5.631956in}{0.455189in}}%
\pgfpathlineto{\pgfqpoint{5.770845in}{0.455189in}}%
\pgfusepath{stroke}%
\end{pgfscope}%
\begin{pgfscope}%
\definecolor{textcolor}{rgb}{0.000000,0.000000,0.000000}%
\pgfsetstrokecolor{textcolor}%
\pgfsetfillcolor{textcolor}%
\pgftext[x=5.881956in,y=0.406578in,left,base]{\color{textcolor}\rmfamily\fontsize{10.000000}{12.000000}\selectfont MDPO-eg}%
\end{pgfscope}%
\begin{pgfscope}%
\pgfsetrectcap%
\pgfsetroundjoin%
\pgfsetlinewidth{1.505625pt}%
\definecolor{currentstroke}{rgb}{0.498039,0.498039,0.498039}%
\pgfsetstrokecolor{currentstroke}%
\pgfsetdash{}{0pt}%
\pgfpathmoveto{\pgfqpoint{6.773161in}{0.455189in}}%
\pgfpathlineto{\pgfqpoint{6.912050in}{0.455189in}}%
\pgfpathlineto{\pgfqpoint{7.050938in}{0.455189in}}%
\pgfusepath{stroke}%
\end{pgfscope}%
\begin{pgfscope}%
\definecolor{textcolor}{rgb}{0.000000,0.000000,0.000000}%
\pgfsetstrokecolor{textcolor}%
\pgfsetfillcolor{textcolor}%
\pgftext[x=7.162050in,y=0.406578in,left,base]{\color{textcolor}\rmfamily\fontsize{10.000000}{12.000000}\selectfont MDPO-nrd}%
\end{pgfscope}%
\begin{pgfscope}%
\pgfsetrectcap%
\pgfsetroundjoin%
\pgfsetlinewidth{1.505625pt}%
\definecolor{currentstroke}{rgb}{0.737255,0.741176,0.133333}%
\pgfsetstrokecolor{currentstroke}%
\pgfsetdash{}{0pt}%
\pgfpathmoveto{\pgfqpoint{8.130801in}{0.455189in}}%
\pgfpathlineto{\pgfqpoint{8.269690in}{0.455189in}}%
\pgfpathlineto{\pgfqpoint{8.408579in}{0.455189in}}%
\pgfusepath{stroke}%
\end{pgfscope}%
\begin{pgfscope}%
\definecolor{textcolor}{rgb}{0.000000,0.000000,0.000000}%
\pgfsetstrokecolor{textcolor}%
\pgfsetfillcolor{textcolor}%
\pgftext[x=8.519690in,y=0.406578in,left,base]{\color{textcolor}\rmfamily\fontsize{10.000000}{12.000000}\selectfont MDPO-nrd-eg}%
\end{pgfscope}%
\begin{pgfscope}%
\pgfsetrectcap%
\pgfsetroundjoin%
\pgfsetlinewidth{1.505625pt}%
\definecolor{currentstroke}{rgb}{0.090196,0.745098,0.811765}%
\pgfsetstrokecolor{currentstroke}%
\pgfsetdash{}{0pt}%
\pgfpathmoveto{\pgfqpoint{9.665911in}{0.455189in}}%
\pgfpathlineto{\pgfqpoint{9.804800in}{0.455189in}}%
\pgfpathlineto{\pgfqpoint{9.943688in}{0.455189in}}%
\pgfusepath{stroke}%
\end{pgfscope}%
\begin{pgfscope}%
\definecolor{textcolor}{rgb}{0.000000,0.000000,0.000000}%
\pgfsetstrokecolor{textcolor}%
\pgfsetfillcolor{textcolor}%
\pgftext[x=10.054800in,y=0.406578in,left,base]{\color{textcolor}\rmfamily\fontsize{10.000000}{12.000000}\selectfont MDPO-nrd-eg-opt2}%
\end{pgfscope}%
\end{pgfpicture}%
\makeatother%
\endgroup%
}
	\caption{Last-iterate converegence plots in PerturbedRPS.}
	\label{fig:tabne}
\end{figure}

\subsection{Observations}
From the above experimental results, we note the following:

\begin{itemize}
	\item {NeuRD-fix speeds up convergence in all the algorithms with or without EG and Optimism.
	      }
	\item {Most notably, Optimistic-NeuRD (PG-nrd-opt2) exhibits faster last-iterate convergence.
	      }
	      % \item {EG updates provide the most improvement in terms of convergence speeds.}
	\item {MMD shows no performance improvements with the addition of Extragradient or Optimistic updates.
	      }
	      % \item {Entropy annealing (MMD) shows slower convergence compared to EG, or Optimism as a last-iterate
	      %       inducing mechanism, however it is easier to implement in practice compared to the latter.}
	\item {In the presence of EG/Optimism to enable last-iterate convergence, trust-region constraints and entropy
	      regularization slow-down the learning, as evidenced by the better performacne of SPG variants compared to the MDPO, and MMD variants.}
	\item {Combining Optimism with EG updates provides minimal improvements as they fulfill a similar role.
	      }
\end{itemize}

The tabular experiments, and the above observations provide some answers to questions~\ref{qn1},
and~\ref{qn2}.
