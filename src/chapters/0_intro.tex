\chapter{Introduction}

% Setup
Multi-agent Systems (MAS)~\cite{tuylsMultiagent2012} are frameworks of problems of distributed
nature with independent actors called agents that cooperate or compete to achieve some outcome.
Due to the complexity of such problems, designing efficient algorithms for them can be challenging.
Machine learning and Reinforcement learning present opportunities for creating learning algorithms
for agents in Multi-agent settings.
Multi-agent Reinforcement Learning (MARL) as a research area deals with designing efficient and
performant RL algorithms for general multi-agent problems.
Due to the inherent issues in multi-agent settings such as the violation of non-stationarity assumption that underlies most RL algorithms, their direct adaptation to the multi-agent setting
can be limiting.
Also, the multi-agent setting presents new complications in the form of large state and action
spaces, high-dimensional problem representations, and credit assignment to name a few.
There have been continued efforts to apply popular policy gradient and value-based single-agent RL
methods to multi-agent setting through novel adaptions that account for some of the above
limitations.
On the other hand, various works have adopted ideas from areas like online learning, game theory,
and evolutionary biology to design novel RL algorithms that circumvent the above limitations by
providing new theoretical guarantees, and displaying strong empirical performances.
Of these, a vast majority of methods rely on regret minimization and self-play learning where
theoretical convergence guarantees necessitate maintaining an average of past strategies that the
agents use (also called as the average policy in RL terms).
While this is easier in tabular settings, maintaining such averages of past policies can become
burdening for large-scale problems especially when utilizing function approximation methods like
neural networks to represent the policy parameters.
Hence, there is a strong incentive in designing algorithms that have convergence guarantees for the
last-iterate policy as opposed to having to maintain an average policy.

In terms of the problem setting that concerns our work, we focus on the game-theoretic construct of
two-player zero-sum games (2p0s).
There are various reasons for the interesting nature of two-player zero-sum games.
Firstly, within various game-theoretic problem formulations, these games present a simple enough
structure while still retaining the multi-agent setting.
This presents an opportunity to study and design algorithms under the dynamics of two-player
zero-sum games, with an aim to extend these algorithms to general multi-agent settings.
Second, two-player zero-sum games are saddle-point problems, and provide an opportunity to connect
with optimization algorithms that have been studied extensively for solving saddle-point problems.
Finally, two-player zero-sum games also model other learning problems such as generative
adversarial networks (GANs)~\cite{goodfellowGenerative2014} and as such any improvements in methods
for the games can provide insights into designing algorithms for applications like GANs.

Our work primarily revolves around algorithms that were proposed or can be interpreted as
extensions of mirror descent as a reinforcement learning update rule.
Mirror Descent is a popular first-order optimization algorithm that has seen wide application in
various problem settings such as online learning, convex optimization, optimal transport, etc. In
this work, we study adaptations of mirror descent to reinforcement learning algorithm in single,
and multi-agent setting.
Specifically, we study Mirror Descent Policy Optimization (MDPO), and Magnetic Mirror Descent
(MMD), two recent algorithms that extend mirror descent as RL algorithms, and approximate
equilibrium solvers.
While there have been many works studying mirror descent with the aim of designing RL algorithms,
MDPO presents a practical deep RL algorithm and studies its similarities in structure, and
connections to existing work in the literature.
MDPO does not directly apply to two-player zero-sum games as it is a single-agent RL algorithm, and
as such has no convergence guarantees.
However, in our work, we study its performance in this setting and also propose some modifications
to the update that MDPO uses to make it more amenable to the multi-agent setting.
MMD, on the other hand, approaches a different problem of entropy regularized equilibrium (QREs)
solving in two-player zero-sum games.
Finding such equilibrium is cast as variational inequality problems and an algorithm from the
variational inequality literature is applied to this problem setting under the name of MMD.
MMD enjoys novel linear last-iterate convergence by taking advantage of problem-specific
assumptions in finding QREs.
Although there are no theoretical guarantees, MMD is also empirically studied as an algorithm for
finding approximate Nash equilibria by annealing the strength of the regularization.
MMD shows strong empirical performance as both tabular and deep reinforcement learning algorithms
in self-play to converge to the Nash equilibrium.
In this work, we use MMD as a baseline method and empirically study the effect of the same proposed
modifications to understand how it improves MMD's performance.
Given the already strong baseline, any improvements observed can guide further algorithmic design
choices, and direct more studies to understand the effect of the proposed modification in this
setting.

The modifications that we apply in this work are two-fold - one that modifies the loss being used
by the algorithm, and the other that modifies the update structure of the algorithms, thereby
affecting the problem's geometry in an effort to improve, or induce last-iterate convergence.
The first modification is the Neural Replicator Dynamics (NeuRD), an extension of the Replicator
dynamics framework from evolutionary game theory to function approximation, and deep reinforcement
learning.
NeuRD can be interpreted as a modification of softmax policy gradients (SPGs) to handle the
plateauing effect of the nonlinearity in adapting to the non-stationary rewards in multi-agent
settings.
Due to this structure, NeuRD can be incorporated into other algorithms that are extensions of SPGs.
The second proposed modification is the incorporation of Extragradient and Optimistic updates to
the above algorithms.
The idea of using extra-gradients, or reusing past gradients is independent of the loss function
being used, and is therefore extendable to the algorithms discussed in this work.
However, there can be inherent limitations in applying these algorithms due to the theoretical
assumptions behind them when applied to specific problem settings.
We only study the effect of the modifications empirically and do not provide any theoretical
convergence guarantees for the proposed updates.
However, we take the potential theoretical limitations into consideration when discussing the
results of the empirical evaluations.

We evaluate the proposed algorithms on the classic game-theoretic normal-form game of biased
rock-paper-scissor to understand their empirical performance.
% Results
We show that in the presence of certain modifications for last-iterate convergence policy gradient
methods outperform the newer methods in this basic setting.
Through this comprehensive study, we then present these algorithms as extensions of policy gradient
methods with various design choices.
This presentation mainly aims at identifying the components of these extensions that contribute
more towards the performance and potential components that detract from the performance.
Based on our findings, we proceed to implement and evaluate a subset of the modifications as deep
multi-agent reinforcement learning algorithms by training the players in self-play.
We use standard extensive-form games like Kuhn poker and large-scale benchmarks like Abrupt Dark
Hex to test the applicability and scalability of the modifications in two-player zero-sum extensive
form games.
We summarize our findings regarding the effectiveness of mirror-descent-based reinforcement
learning algorithms in the multi-agent setting and the effect of the modifications we apply.
Through this study, we provide some recommendations for designing MARL algorithms and close with
some remarks about future research directions.
