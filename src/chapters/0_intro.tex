\chapter{Introduction}

Multiagent Systems (MAS)~\cite{tuylsMultiagent2012} are frameworks of problems of distributed
nature with independent actors called agents that cooperate or compete to achieve some outcome.
Due to the complexity of such problems, designing efficient algorithms for them can be challenging.
Machine learning and Reinforcement learning present opportunities for creating learning algorithms
for agents in Multi-agent settings.
Multiagent Reinforcement Learning (MARL) as a research area deals with designing efficient and
performant RL algorithms for general multiagent problems.
Two-player zero-sum games are instances of multiagent systems that are purely competitive in
nature.
Due to their unique structure, two-player zero-sum games can also be represented as saddle-point
problems that provide certain analytical properties.
This presents an opportunity to study and design optimization algorithms under the dynamics of
two-player zero-sum games, with an aim to extend these algorithms to general multiagent settings.
Two-player zero-sum games also model other learning problems~\cite{goodfellowGenerative2014} and as
such these advances can also be equivalently applied or adapted to solve them.

Mirror Descent is a popular first-order optimization algorithm that has wide applications.
In this work, we study mirror-descent-based reinforcement learning algorithms in the context of
two-player zero-sum games.
Specifically, we study Mirror Descent Policy Optimization, and Magnetic Mirror Descent, two recent
algorithms that extend Mirror Descent as RL algorithms, and approximate equilibrium solvers.
We propose novel improvements to these algorithms by incorporating existing techniques and study
their convergence behaviors in normal-form games.
We also evaluate the performance of these algorithms in benchmark extensive-form games under
function approximation.
We summarize our findings regarding the effectiveness of mirror-descent-based reinforcement
learning algorithms in the multiagent setting and the effect of the modifications we apply.
Through this study, we provide some recommendations in designing MARL algorithms and close with some
remarks about future research directions.
