\chapter{Modified Updates for Last-iterate Convergence}
\label{chp:updates}

\old{Our main contribution is adapting existing techniques for inducing convergence in multiagent
	settings with the algorithms discussed in the previous section.}
(\revdone{Too brief.
	Slow down and explain to the reader what the problem is that these modified updates are intended to
	solve.
})
Last-iterate convergence of algorithms is an attractive property that has been studied widely in the context
of optimization, and equilibirium learning algorithms.
While there have been numerous methods proposed for equilibrium solving in games, many of them only
guarantee convergence in the ergodic sense, i.e., only the average of their iterates converge to an
equilibrium.
However, in cases where the strategies are represented using function approximators, this presents
a problem as it becomes complicated to maintain a history of past strategies or in terms of
averaging them.

The main contribution of our work is in studying a few modifications to the algorithms discussed in
the previous section, with the objective of either inducing and speeding up last-iterate
convergence in two-player zero-sum settings.
More specifically, we study the following three techniques, namely - Neural Replicator
Dynamics~\cite{hennesNeural2020}, Extragradient updates~\cite{korpelevichextragradient1976}, and
Optimistic gradient updates~\cite{popovmodification1980}, that have been previously studied in the
context of learning in games, and solving saddle point problems.
In this section we describe these techniques in more detail and provide our proposed modifications
to the algorithms from the previous section.

\section{Neural Replicator Dynamics (NeuRD)}

% The continuous-time single-population replicator dynamics is defined by the following system of
% differential equations:
% \begin{equation}
% 	\label{eqn:rd} \dot{\pi}(a) = \pi(a)[u(a, \pi) -
% 		\bar{u}(\pi)], \forall a \in \mathcal{A}
% \end{equation}

Consider the problem of learning a parameterized policy $\pi_{\theta}$ in a single-state all-action
problem setting.
In such a setting, SPG employs the following update at each iteration
$t$~\cite[Section~A.1]{hennesNeural2020}: \[y_{\theta_t}(a) = y_{\theta_{t-1}} (a) + \eta_t
	\pi_{\theta_{t-1}}(a) [r_t(a) - \bar{r_t}], \forall a \in A\] where $y$ represents the logits,
$r_t(a)$ is the immediate reward associated with action $a$, and $\bar{r}_t$ is the average reward
of the state.

\subsection{Replicator Dynamics and No-regret Learning}
The above SPG update is equivalent to the instant regret scaled by the current policy.
This scaling makes it difficult for SPG to adapt to sudden changes in rewards associated with
actions that are already less likely under the current policy.
This problem is more evident in multiagent settings where the opponent's policy can change
arbitrarily affecting rewards associated with actions even in single-state settings.

Motivated by this observation, Hennes et.
al~\cite{hennesNeural2020} propose to make modifications
to SPG using the idea of Replicator dynamics from Evolutionary game theory.
Replicator Dynamics defiens operators that describe the dynamics of a population's evolution when
attempting to maximize some arbitrary utility function (\note{cite}).
Replicator dynamics also have a strong connection to no-regret algorithms, and
in~\cite[Statement~1]{hennesNeural2020}, the authors establish the equivalence between Hedge, and
discrete time RD.
Due to this equivalence, RD also inherits the property of no-regret algorithms that their
time-average policy converges to the Nash equilibrium (\note{cite}).

\subsection{NeuRD}
Neural Replicator Dynamics (NeuRD) is an adaptation of discrete-time Replicator Dynamics to
reinforcement learning for function approximation settings (\revdone{Include the math, not just the
	english}).
For a parameterized policy $\pi_\theta$, the NeuRD-update rule is given by:

\begin{equation}
	\label{eqn:nrd} \theta_t \leftarrow \theta_{t-1} + \eta_t \sum_{a'}
	\nabla_{\theta} \text{Y}_{\theta_{t-1}}(s, a') A(s, a')
\end{equation}

where
$Y_{\theta_{t-1}}$ represent the logits of the parameterized policy at timestep $t-1$, and $A$
represents the advantage value.

The only difference between the NeuRD update, and the SPG update is that the gradient of the
parameters are computed directly with respect to the logits.
This can be viewed as a modification to SPG making it more adaptive in non-stationary settings.
To prevent numerical issues stemming from accumulating the advantages into the logits resulting in
unstable gradients, a logit gap based thresholding is proposed in~\cite{hennesNeural2020}.
This thresholding operator is: $\nabla_{\theta}(f(\theta), \eta, \beta) \doteq [\eta\nabla_\theta
		f(\theta)] \mathds{I}\{f(\theta + \eta\nabla_{\theta}f(\theta)) \in [-\beta, \beta]\}$, where
$\beta$ is the \textit{NeuRD-threshold} that determines how arbitrarily close to 0 or 1 the
probabilities can get to.
% While NeuRD has average iterate convergence,  also induce last iterate convergence in
% imperfect information settings for NeuRD using reward transformation based regularization~\cite{perolatPoincare2021}.
Since most PG methods, including the ones discussed in the previous chapter utilize softmax
parameterization, the NeuRD fix is easily extendable to the policy loss component of their
respective loss functions without any major changes.
In our work, we also apply the NeuRD fix to the policy loss component of the MMD and MDPO loss
functions (\fillin{rewrite this sentence}).

\subsection{Relation to Natural Policy Gradients}

\subsection{Alternating vs Simultaneous gradient updates}
As an aside, we wish to discuss two possible update schemes for discrete learning dynamics, namely
- Simultaneous and Alternating updates.
We also observe that while alternating updates shows average-iterate convergence, simultaneous
updates do not converge.
\begin{figure}[H]
	\centering
	\scalebox{0.6}[0.6]{%% Creator: Matplotlib, PGF backend
%%
%% To include the figure in your LaTeX document, write
%%   \input{<filename>.pgf}
%%
%% Make sure the required packages are loaded in your preamble
%%   \usepackage{pgf}
%%
%% Also ensure that all the required font packages are loaded; for instance,
%% the lmodern package is sometimes necessary when using math font.
%%   \usepackage{lmodern}
%%
%% Figures using additional raster images can only be included by \input if
%% they are in the same directory as the main LaTeX file. For loading figures
%% from other directories you can use the `import` package
%%   \usepackage{import}
%%
%% and then include the figures with
%%   \import{<path to file>}{<filename>.pgf}
%%
%% Matplotlib used the following preamble
%%   
%%   \usepackage{fontspec}
%%   \setmainfont{DejaVuSerif.ttf}[Path=\detokenize{/home/slurp/miniconda3/envs/thesis_results/lib/python3.9/site-packages/matplotlib/mpl-data/fonts/ttf/}]
%%   \setsansfont{DejaVuSans.ttf}[Path=\detokenize{/home/slurp/miniconda3/envs/thesis_results/lib/python3.9/site-packages/matplotlib/mpl-data/fonts/ttf/}]
%%   \setmonofont{DejaVuSansMono.ttf}[Path=\detokenize{/home/slurp/miniconda3/envs/thesis_results/lib/python3.9/site-packages/matplotlib/mpl-data/fonts/ttf/}]
%%   \makeatletter\@ifpackageloaded{underscore}{}{\usepackage[strings]{underscore}}\makeatother
%%
\begingroup%
\makeatletter%
\begin{pgfpicture}%
\pgfpathrectangle{\pgfpointorigin}{\pgfqpoint{9.000000in}{6.000000in}}%
\pgfusepath{use as bounding box, clip}%
\begin{pgfscope}%
\pgfsetbuttcap%
\pgfsetmiterjoin%
\definecolor{currentfill}{rgb}{1.000000,1.000000,1.000000}%
\pgfsetfillcolor{currentfill}%
\pgfsetlinewidth{0.000000pt}%
\definecolor{currentstroke}{rgb}{1.000000,1.000000,1.000000}%
\pgfsetstrokecolor{currentstroke}%
\pgfsetdash{}{0pt}%
\pgfpathmoveto{\pgfqpoint{0.000000in}{0.000000in}}%
\pgfpathlineto{\pgfqpoint{9.000000in}{0.000000in}}%
\pgfpathlineto{\pgfqpoint{9.000000in}{6.000000in}}%
\pgfpathlineto{\pgfqpoint{0.000000in}{6.000000in}}%
\pgfpathlineto{\pgfqpoint{0.000000in}{0.000000in}}%
\pgfpathclose%
\pgfusepath{fill}%
\end{pgfscope}%
\begin{pgfscope}%
\pgfsetbuttcap%
\pgfsetmiterjoin%
\definecolor{currentfill}{rgb}{1.000000,1.000000,1.000000}%
\pgfsetfillcolor{currentfill}%
\pgfsetlinewidth{0.000000pt}%
\definecolor{currentstroke}{rgb}{0.000000,0.000000,0.000000}%
\pgfsetstrokecolor{currentstroke}%
\pgfsetstrokeopacity{0.000000}%
\pgfsetdash{}{0pt}%
\pgfpathmoveto{\pgfqpoint{0.152333in}{3.075000in}}%
\pgfpathlineto{\pgfqpoint{4.376551in}{3.075000in}}%
\pgfpathlineto{\pgfqpoint{4.376551in}{5.640000in}}%
\pgfpathlineto{\pgfqpoint{0.152333in}{5.640000in}}%
\pgfpathlineto{\pgfqpoint{0.152333in}{3.075000in}}%
\pgfpathclose%
\pgfusepath{fill}%
\end{pgfscope}%
\begin{pgfscope}%
\pgfpathrectangle{\pgfqpoint{0.152333in}{3.075000in}}{\pgfqpoint{4.224218in}{2.565000in}}%
\pgfusepath{clip}%
\pgfsetbuttcap%
\pgfsetmiterjoin%
\definecolor{currentfill}{rgb}{0.960784,0.960784,0.960784}%
\pgfsetfillcolor{currentfill}%
\pgfsetfillopacity{0.750000}%
\pgfsetlinewidth{1.003750pt}%
\definecolor{currentstroke}{rgb}{0.960784,0.960784,0.960784}%
\pgfsetstrokecolor{currentstroke}%
\pgfsetstrokeopacity{0.750000}%
\pgfsetdash{}{0pt}%
\pgfpathmoveto{\pgfqpoint{4.184541in}{3.331500in}}%
\pgfpathlineto{\pgfqpoint{2.264442in}{5.552855in}}%
\pgfpathlineto{\pgfqpoint{0.344343in}{3.331500in}}%
\pgfpathlineto{\pgfqpoint{4.184541in}{3.331500in}}%
\pgfpathclose%
\pgfusepath{stroke,fill}%
\end{pgfscope}%
\begin{pgfscope}%
\pgfpathrectangle{\pgfqpoint{0.152333in}{3.075000in}}{\pgfqpoint{4.224218in}{2.565000in}}%
\pgfusepath{clip}%
\pgfsetrectcap%
\pgfsetroundjoin%
\pgfsetlinewidth{0.803000pt}%
\definecolor{currentstroke}{rgb}{0.000000,0.000000,0.000000}%
\pgfsetstrokecolor{currentstroke}%
\pgfsetdash{}{0pt}%
\pgfpathmoveto{\pgfqpoint{0.344343in}{3.331500in}}%
\pgfpathlineto{\pgfqpoint{4.184541in}{3.331500in}}%
\pgfusepath{stroke}%
\end{pgfscope}%
\begin{pgfscope}%
\pgfpathrectangle{\pgfqpoint{0.152333in}{3.075000in}}{\pgfqpoint{4.224218in}{2.565000in}}%
\pgfusepath{clip}%
\pgfsetrectcap%
\pgfsetroundjoin%
\pgfsetlinewidth{0.803000pt}%
\definecolor{currentstroke}{rgb}{0.000000,0.000000,0.000000}%
\pgfsetstrokecolor{currentstroke}%
\pgfsetdash{}{0pt}%
\pgfpathmoveto{\pgfqpoint{2.264442in}{5.552855in}}%
\pgfpathlineto{\pgfqpoint{0.344343in}{3.331500in}}%
\pgfusepath{stroke}%
\end{pgfscope}%
\begin{pgfscope}%
\pgfpathrectangle{\pgfqpoint{0.152333in}{3.075000in}}{\pgfqpoint{4.224218in}{2.565000in}}%
\pgfusepath{clip}%
\pgfsetrectcap%
\pgfsetroundjoin%
\pgfsetlinewidth{0.803000pt}%
\definecolor{currentstroke}{rgb}{0.000000,0.000000,0.000000}%
\pgfsetstrokecolor{currentstroke}%
\pgfsetdash{}{0pt}%
\pgfpathmoveto{\pgfqpoint{2.264442in}{5.552855in}}%
\pgfpathlineto{\pgfqpoint{4.184541in}{3.331500in}}%
\pgfusepath{stroke}%
\end{pgfscope}%
\begin{pgfscope}%
\pgfpathrectangle{\pgfqpoint{0.152333in}{3.075000in}}{\pgfqpoint{4.224218in}{2.565000in}}%
\pgfusepath{clip}%
\pgfsetbuttcap%
\pgfsetroundjoin%
\pgfsetlinewidth{0.501875pt}%
\definecolor{currentstroke}{rgb}{0.000000,0.000000,0.000000}%
\pgfsetstrokecolor{currentstroke}%
\pgfsetdash{{0.500000pt}{0.825000pt}}{0.000000pt}%
\pgfpathmoveto{\pgfqpoint{0.344343in}{3.331500in}}%
\pgfpathlineto{\pgfqpoint{4.184541in}{3.331500in}}%
\pgfusepath{stroke}%
\end{pgfscope}%
\begin{pgfscope}%
\pgfpathrectangle{\pgfqpoint{0.152333in}{3.075000in}}{\pgfqpoint{4.224218in}{2.565000in}}%
\pgfusepath{clip}%
\pgfsetbuttcap%
\pgfsetroundjoin%
\pgfsetlinewidth{0.501875pt}%
\definecolor{currentstroke}{rgb}{0.000000,0.000000,0.000000}%
\pgfsetstrokecolor{currentstroke}%
\pgfsetdash{{0.500000pt}{0.825000pt}}{0.000000pt}%
\pgfpathmoveto{\pgfqpoint{1.304393in}{4.442178in}}%
\pgfpathlineto{\pgfqpoint{3.224492in}{4.442178in}}%
\pgfusepath{stroke}%
\end{pgfscope}%
\begin{pgfscope}%
\pgfpathrectangle{\pgfqpoint{0.152333in}{3.075000in}}{\pgfqpoint{4.224218in}{2.565000in}}%
\pgfusepath{clip}%
\pgfsetbuttcap%
\pgfsetroundjoin%
\pgfsetlinewidth{0.501875pt}%
\definecolor{currentstroke}{rgb}{0.000000,0.000000,0.000000}%
\pgfsetstrokecolor{currentstroke}%
\pgfsetdash{{0.500000pt}{0.825000pt}}{0.000000pt}%
\pgfpathmoveto{\pgfqpoint{2.264442in}{5.552855in}}%
\pgfpathlineto{\pgfqpoint{0.344343in}{3.331500in}}%
\pgfusepath{stroke}%
\end{pgfscope}%
\begin{pgfscope}%
\pgfpathrectangle{\pgfqpoint{0.152333in}{3.075000in}}{\pgfqpoint{4.224218in}{2.565000in}}%
\pgfusepath{clip}%
\pgfsetbuttcap%
\pgfsetroundjoin%
\pgfsetlinewidth{0.501875pt}%
\definecolor{currentstroke}{rgb}{0.000000,0.000000,0.000000}%
\pgfsetstrokecolor{currentstroke}%
\pgfsetdash{{0.500000pt}{0.825000pt}}{0.000000pt}%
\pgfpathmoveto{\pgfqpoint{2.264442in}{5.552855in}}%
\pgfpathlineto{\pgfqpoint{4.184541in}{3.331500in}}%
\pgfusepath{stroke}%
\end{pgfscope}%
\begin{pgfscope}%
\pgfpathrectangle{\pgfqpoint{0.152333in}{3.075000in}}{\pgfqpoint{4.224218in}{2.565000in}}%
\pgfusepath{clip}%
\pgfsetbuttcap%
\pgfsetroundjoin%
\pgfsetlinewidth{0.501875pt}%
\definecolor{currentstroke}{rgb}{0.000000,0.000000,0.000000}%
\pgfsetstrokecolor{currentstroke}%
\pgfsetdash{{0.500000pt}{0.825000pt}}{0.000000pt}%
\pgfpathmoveto{\pgfqpoint{3.224492in}{4.442178in}}%
\pgfpathlineto{\pgfqpoint{2.264442in}{3.331500in}}%
\pgfusepath{stroke}%
\end{pgfscope}%
\begin{pgfscope}%
\pgfpathrectangle{\pgfqpoint{0.152333in}{3.075000in}}{\pgfqpoint{4.224218in}{2.565000in}}%
\pgfusepath{clip}%
\pgfsetbuttcap%
\pgfsetroundjoin%
\pgfsetlinewidth{0.501875pt}%
\definecolor{currentstroke}{rgb}{0.000000,0.000000,0.000000}%
\pgfsetstrokecolor{currentstroke}%
\pgfsetdash{{0.500000pt}{0.825000pt}}{0.000000pt}%
\pgfpathmoveto{\pgfqpoint{1.304393in}{4.442178in}}%
\pgfpathlineto{\pgfqpoint{2.264442in}{3.331500in}}%
\pgfusepath{stroke}%
\end{pgfscope}%
\begin{pgfscope}%
\pgfpathrectangle{\pgfqpoint{0.152333in}{3.075000in}}{\pgfqpoint{4.224218in}{2.565000in}}%
\pgfusepath{clip}%
\pgfsetbuttcap%
\pgfsetroundjoin%
\pgfsetlinewidth{0.501875pt}%
\definecolor{currentstroke}{rgb}{0.000000,0.000000,0.000000}%
\pgfsetstrokecolor{currentstroke}%
\pgfsetdash{{0.500000pt}{0.825000pt}}{0.000000pt}%
\pgfpathmoveto{\pgfqpoint{4.184541in}{3.331500in}}%
\pgfpathlineto{\pgfqpoint{4.184541in}{3.331500in}}%
\pgfusepath{stroke}%
\end{pgfscope}%
\begin{pgfscope}%
\pgfpathrectangle{\pgfqpoint{0.152333in}{3.075000in}}{\pgfqpoint{4.224218in}{2.565000in}}%
\pgfusepath{clip}%
\pgfsetbuttcap%
\pgfsetroundjoin%
\pgfsetlinewidth{0.501875pt}%
\definecolor{currentstroke}{rgb}{0.000000,0.000000,0.000000}%
\pgfsetstrokecolor{currentstroke}%
\pgfsetdash{{0.500000pt}{0.825000pt}}{0.000000pt}%
\pgfpathmoveto{\pgfqpoint{0.344343in}{3.331500in}}%
\pgfpathlineto{\pgfqpoint{0.344343in}{3.331500in}}%
\pgfusepath{stroke}%
\end{pgfscope}%
\begin{pgfscope}%
\pgfpathrectangle{\pgfqpoint{0.152333in}{3.075000in}}{\pgfqpoint{4.224218in}{2.565000in}}%
\pgfusepath{clip}%
\pgfsetbuttcap%
\pgfsetroundjoin%
\pgfsetlinewidth{0.501875pt}%
\definecolor{currentstroke}{rgb}{0.000000,0.000000,1.000000}%
\pgfsetstrokecolor{currentstroke}%
\pgfsetdash{{0.500000pt}{0.825000pt}}{0.000000pt}%
\pgfpathmoveto{\pgfqpoint{0.344343in}{3.331500in}}%
\pgfpathlineto{\pgfqpoint{4.184541in}{3.331500in}}%
\pgfusepath{stroke}%
\end{pgfscope}%
\begin{pgfscope}%
\pgfpathrectangle{\pgfqpoint{0.152333in}{3.075000in}}{\pgfqpoint{4.224218in}{2.565000in}}%
\pgfusepath{clip}%
\pgfsetbuttcap%
\pgfsetroundjoin%
\pgfsetlinewidth{0.501875pt}%
\definecolor{currentstroke}{rgb}{0.000000,0.000000,1.000000}%
\pgfsetstrokecolor{currentstroke}%
\pgfsetdash{{0.500000pt}{0.825000pt}}{0.000000pt}%
\pgfpathmoveto{\pgfqpoint{0.536353in}{3.553636in}}%
\pgfpathlineto{\pgfqpoint{3.992531in}{3.553636in}}%
\pgfusepath{stroke}%
\end{pgfscope}%
\begin{pgfscope}%
\pgfpathrectangle{\pgfqpoint{0.152333in}{3.075000in}}{\pgfqpoint{4.224218in}{2.565000in}}%
\pgfusepath{clip}%
\pgfsetbuttcap%
\pgfsetroundjoin%
\pgfsetlinewidth{0.501875pt}%
\definecolor{currentstroke}{rgb}{0.000000,0.000000,1.000000}%
\pgfsetstrokecolor{currentstroke}%
\pgfsetdash{{0.500000pt}{0.825000pt}}{0.000000pt}%
\pgfpathmoveto{\pgfqpoint{0.728363in}{3.775771in}}%
\pgfpathlineto{\pgfqpoint{3.800522in}{3.775771in}}%
\pgfusepath{stroke}%
\end{pgfscope}%
\begin{pgfscope}%
\pgfpathrectangle{\pgfqpoint{0.152333in}{3.075000in}}{\pgfqpoint{4.224218in}{2.565000in}}%
\pgfusepath{clip}%
\pgfsetbuttcap%
\pgfsetroundjoin%
\pgfsetlinewidth{0.501875pt}%
\definecolor{currentstroke}{rgb}{0.000000,0.000000,1.000000}%
\pgfsetstrokecolor{currentstroke}%
\pgfsetdash{{0.500000pt}{0.825000pt}}{0.000000pt}%
\pgfpathmoveto{\pgfqpoint{0.920373in}{3.997907in}}%
\pgfpathlineto{\pgfqpoint{3.608512in}{3.997907in}}%
\pgfusepath{stroke}%
\end{pgfscope}%
\begin{pgfscope}%
\pgfpathrectangle{\pgfqpoint{0.152333in}{3.075000in}}{\pgfqpoint{4.224218in}{2.565000in}}%
\pgfusepath{clip}%
\pgfsetbuttcap%
\pgfsetroundjoin%
\pgfsetlinewidth{0.501875pt}%
\definecolor{currentstroke}{rgb}{0.000000,0.000000,1.000000}%
\pgfsetstrokecolor{currentstroke}%
\pgfsetdash{{0.500000pt}{0.825000pt}}{0.000000pt}%
\pgfpathmoveto{\pgfqpoint{1.112383in}{4.220042in}}%
\pgfpathlineto{\pgfqpoint{3.416502in}{4.220042in}}%
\pgfusepath{stroke}%
\end{pgfscope}%
\begin{pgfscope}%
\pgfpathrectangle{\pgfqpoint{0.152333in}{3.075000in}}{\pgfqpoint{4.224218in}{2.565000in}}%
\pgfusepath{clip}%
\pgfsetbuttcap%
\pgfsetroundjoin%
\pgfsetlinewidth{0.501875pt}%
\definecolor{currentstroke}{rgb}{0.000000,0.000000,1.000000}%
\pgfsetstrokecolor{currentstroke}%
\pgfsetdash{{0.500000pt}{0.825000pt}}{0.000000pt}%
\pgfpathmoveto{\pgfqpoint{1.304393in}{4.442178in}}%
\pgfpathlineto{\pgfqpoint{3.224492in}{4.442178in}}%
\pgfusepath{stroke}%
\end{pgfscope}%
\begin{pgfscope}%
\pgfpathrectangle{\pgfqpoint{0.152333in}{3.075000in}}{\pgfqpoint{4.224218in}{2.565000in}}%
\pgfusepath{clip}%
\pgfsetbuttcap%
\pgfsetroundjoin%
\pgfsetlinewidth{0.501875pt}%
\definecolor{currentstroke}{rgb}{0.000000,0.000000,1.000000}%
\pgfsetstrokecolor{currentstroke}%
\pgfsetdash{{0.500000pt}{0.825000pt}}{0.000000pt}%
\pgfpathmoveto{\pgfqpoint{1.496402in}{4.664313in}}%
\pgfpathlineto{\pgfqpoint{3.032482in}{4.664313in}}%
\pgfusepath{stroke}%
\end{pgfscope}%
\begin{pgfscope}%
\pgfpathrectangle{\pgfqpoint{0.152333in}{3.075000in}}{\pgfqpoint{4.224218in}{2.565000in}}%
\pgfusepath{clip}%
\pgfsetbuttcap%
\pgfsetroundjoin%
\pgfsetlinewidth{0.501875pt}%
\definecolor{currentstroke}{rgb}{0.000000,0.000000,1.000000}%
\pgfsetstrokecolor{currentstroke}%
\pgfsetdash{{0.500000pt}{0.825000pt}}{0.000000pt}%
\pgfpathmoveto{\pgfqpoint{1.688412in}{4.886449in}}%
\pgfpathlineto{\pgfqpoint{2.840472in}{4.886449in}}%
\pgfusepath{stroke}%
\end{pgfscope}%
\begin{pgfscope}%
\pgfpathrectangle{\pgfqpoint{0.152333in}{3.075000in}}{\pgfqpoint{4.224218in}{2.565000in}}%
\pgfusepath{clip}%
\pgfsetbuttcap%
\pgfsetroundjoin%
\pgfsetlinewidth{0.501875pt}%
\definecolor{currentstroke}{rgb}{0.000000,0.000000,1.000000}%
\pgfsetstrokecolor{currentstroke}%
\pgfsetdash{{0.500000pt}{0.825000pt}}{0.000000pt}%
\pgfpathmoveto{\pgfqpoint{1.880422in}{5.108584in}}%
\pgfpathlineto{\pgfqpoint{2.648462in}{5.108584in}}%
\pgfusepath{stroke}%
\end{pgfscope}%
\begin{pgfscope}%
\pgfpathrectangle{\pgfqpoint{0.152333in}{3.075000in}}{\pgfqpoint{4.224218in}{2.565000in}}%
\pgfusepath{clip}%
\pgfsetbuttcap%
\pgfsetroundjoin%
\pgfsetlinewidth{0.501875pt}%
\definecolor{currentstroke}{rgb}{0.000000,0.000000,1.000000}%
\pgfsetstrokecolor{currentstroke}%
\pgfsetdash{{0.500000pt}{0.825000pt}}{0.000000pt}%
\pgfpathmoveto{\pgfqpoint{2.072432in}{5.330720in}}%
\pgfpathlineto{\pgfqpoint{2.456452in}{5.330720in}}%
\pgfusepath{stroke}%
\end{pgfscope}%
\begin{pgfscope}%
\pgfpathrectangle{\pgfqpoint{0.152333in}{3.075000in}}{\pgfqpoint{4.224218in}{2.565000in}}%
\pgfusepath{clip}%
\pgfsetbuttcap%
\pgfsetroundjoin%
\pgfsetlinewidth{0.501875pt}%
\definecolor{currentstroke}{rgb}{0.000000,0.000000,1.000000}%
\pgfsetstrokecolor{currentstroke}%
\pgfsetdash{{0.500000pt}{0.825000pt}}{0.000000pt}%
\pgfpathmoveto{\pgfqpoint{2.264442in}{5.552855in}}%
\pgfpathlineto{\pgfqpoint{0.344343in}{3.331500in}}%
\pgfusepath{stroke}%
\end{pgfscope}%
\begin{pgfscope}%
\pgfpathrectangle{\pgfqpoint{0.152333in}{3.075000in}}{\pgfqpoint{4.224218in}{2.565000in}}%
\pgfusepath{clip}%
\pgfsetbuttcap%
\pgfsetroundjoin%
\pgfsetlinewidth{0.501875pt}%
\definecolor{currentstroke}{rgb}{0.000000,0.000000,1.000000}%
\pgfsetstrokecolor{currentstroke}%
\pgfsetdash{{0.500000pt}{0.825000pt}}{0.000000pt}%
\pgfpathmoveto{\pgfqpoint{2.264442in}{5.552855in}}%
\pgfpathlineto{\pgfqpoint{4.184541in}{3.331500in}}%
\pgfusepath{stroke}%
\end{pgfscope}%
\begin{pgfscope}%
\pgfpathrectangle{\pgfqpoint{0.152333in}{3.075000in}}{\pgfqpoint{4.224218in}{2.565000in}}%
\pgfusepath{clip}%
\pgfsetbuttcap%
\pgfsetroundjoin%
\pgfsetlinewidth{0.501875pt}%
\definecolor{currentstroke}{rgb}{0.000000,0.000000,1.000000}%
\pgfsetstrokecolor{currentstroke}%
\pgfsetdash{{0.500000pt}{0.825000pt}}{0.000000pt}%
\pgfpathmoveto{\pgfqpoint{2.456452in}{5.330720in}}%
\pgfpathlineto{\pgfqpoint{0.728363in}{3.331500in}}%
\pgfusepath{stroke}%
\end{pgfscope}%
\begin{pgfscope}%
\pgfpathrectangle{\pgfqpoint{0.152333in}{3.075000in}}{\pgfqpoint{4.224218in}{2.565000in}}%
\pgfusepath{clip}%
\pgfsetbuttcap%
\pgfsetroundjoin%
\pgfsetlinewidth{0.501875pt}%
\definecolor{currentstroke}{rgb}{0.000000,0.000000,1.000000}%
\pgfsetstrokecolor{currentstroke}%
\pgfsetdash{{0.500000pt}{0.825000pt}}{0.000000pt}%
\pgfpathmoveto{\pgfqpoint{2.072432in}{5.330720in}}%
\pgfpathlineto{\pgfqpoint{3.800522in}{3.331500in}}%
\pgfusepath{stroke}%
\end{pgfscope}%
\begin{pgfscope}%
\pgfpathrectangle{\pgfqpoint{0.152333in}{3.075000in}}{\pgfqpoint{4.224218in}{2.565000in}}%
\pgfusepath{clip}%
\pgfsetbuttcap%
\pgfsetroundjoin%
\pgfsetlinewidth{0.501875pt}%
\definecolor{currentstroke}{rgb}{0.000000,0.000000,1.000000}%
\pgfsetstrokecolor{currentstroke}%
\pgfsetdash{{0.500000pt}{0.825000pt}}{0.000000pt}%
\pgfpathmoveto{\pgfqpoint{2.648462in}{5.108584in}}%
\pgfpathlineto{\pgfqpoint{1.112383in}{3.331500in}}%
\pgfusepath{stroke}%
\end{pgfscope}%
\begin{pgfscope}%
\pgfpathrectangle{\pgfqpoint{0.152333in}{3.075000in}}{\pgfqpoint{4.224218in}{2.565000in}}%
\pgfusepath{clip}%
\pgfsetbuttcap%
\pgfsetroundjoin%
\pgfsetlinewidth{0.501875pt}%
\definecolor{currentstroke}{rgb}{0.000000,0.000000,1.000000}%
\pgfsetstrokecolor{currentstroke}%
\pgfsetdash{{0.500000pt}{0.825000pt}}{0.000000pt}%
\pgfpathmoveto{\pgfqpoint{1.880422in}{5.108584in}}%
\pgfpathlineto{\pgfqpoint{3.416502in}{3.331500in}}%
\pgfusepath{stroke}%
\end{pgfscope}%
\begin{pgfscope}%
\pgfpathrectangle{\pgfqpoint{0.152333in}{3.075000in}}{\pgfqpoint{4.224218in}{2.565000in}}%
\pgfusepath{clip}%
\pgfsetbuttcap%
\pgfsetroundjoin%
\pgfsetlinewidth{0.501875pt}%
\definecolor{currentstroke}{rgb}{0.000000,0.000000,1.000000}%
\pgfsetstrokecolor{currentstroke}%
\pgfsetdash{{0.500000pt}{0.825000pt}}{0.000000pt}%
\pgfpathmoveto{\pgfqpoint{2.840472in}{4.886449in}}%
\pgfpathlineto{\pgfqpoint{1.496402in}{3.331500in}}%
\pgfusepath{stroke}%
\end{pgfscope}%
\begin{pgfscope}%
\pgfpathrectangle{\pgfqpoint{0.152333in}{3.075000in}}{\pgfqpoint{4.224218in}{2.565000in}}%
\pgfusepath{clip}%
\pgfsetbuttcap%
\pgfsetroundjoin%
\pgfsetlinewidth{0.501875pt}%
\definecolor{currentstroke}{rgb}{0.000000,0.000000,1.000000}%
\pgfsetstrokecolor{currentstroke}%
\pgfsetdash{{0.500000pt}{0.825000pt}}{0.000000pt}%
\pgfpathmoveto{\pgfqpoint{1.688412in}{4.886449in}}%
\pgfpathlineto{\pgfqpoint{3.032482in}{3.331500in}}%
\pgfusepath{stroke}%
\end{pgfscope}%
\begin{pgfscope}%
\pgfpathrectangle{\pgfqpoint{0.152333in}{3.075000in}}{\pgfqpoint{4.224218in}{2.565000in}}%
\pgfusepath{clip}%
\pgfsetbuttcap%
\pgfsetroundjoin%
\pgfsetlinewidth{0.501875pt}%
\definecolor{currentstroke}{rgb}{0.000000,0.000000,1.000000}%
\pgfsetstrokecolor{currentstroke}%
\pgfsetdash{{0.500000pt}{0.825000pt}}{0.000000pt}%
\pgfpathmoveto{\pgfqpoint{3.032482in}{4.664313in}}%
\pgfpathlineto{\pgfqpoint{1.880422in}{3.331500in}}%
\pgfusepath{stroke}%
\end{pgfscope}%
\begin{pgfscope}%
\pgfpathrectangle{\pgfqpoint{0.152333in}{3.075000in}}{\pgfqpoint{4.224218in}{2.565000in}}%
\pgfusepath{clip}%
\pgfsetbuttcap%
\pgfsetroundjoin%
\pgfsetlinewidth{0.501875pt}%
\definecolor{currentstroke}{rgb}{0.000000,0.000000,1.000000}%
\pgfsetstrokecolor{currentstroke}%
\pgfsetdash{{0.500000pt}{0.825000pt}}{0.000000pt}%
\pgfpathmoveto{\pgfqpoint{1.496402in}{4.664313in}}%
\pgfpathlineto{\pgfqpoint{2.648462in}{3.331500in}}%
\pgfusepath{stroke}%
\end{pgfscope}%
\begin{pgfscope}%
\pgfpathrectangle{\pgfqpoint{0.152333in}{3.075000in}}{\pgfqpoint{4.224218in}{2.565000in}}%
\pgfusepath{clip}%
\pgfsetbuttcap%
\pgfsetroundjoin%
\pgfsetlinewidth{0.501875pt}%
\definecolor{currentstroke}{rgb}{0.000000,0.000000,1.000000}%
\pgfsetstrokecolor{currentstroke}%
\pgfsetdash{{0.500000pt}{0.825000pt}}{0.000000pt}%
\pgfpathmoveto{\pgfqpoint{3.224492in}{4.442178in}}%
\pgfpathlineto{\pgfqpoint{2.264442in}{3.331500in}}%
\pgfusepath{stroke}%
\end{pgfscope}%
\begin{pgfscope}%
\pgfpathrectangle{\pgfqpoint{0.152333in}{3.075000in}}{\pgfqpoint{4.224218in}{2.565000in}}%
\pgfusepath{clip}%
\pgfsetbuttcap%
\pgfsetroundjoin%
\pgfsetlinewidth{0.501875pt}%
\definecolor{currentstroke}{rgb}{0.000000,0.000000,1.000000}%
\pgfsetstrokecolor{currentstroke}%
\pgfsetdash{{0.500000pt}{0.825000pt}}{0.000000pt}%
\pgfpathmoveto{\pgfqpoint{1.304393in}{4.442178in}}%
\pgfpathlineto{\pgfqpoint{2.264442in}{3.331500in}}%
\pgfusepath{stroke}%
\end{pgfscope}%
\begin{pgfscope}%
\pgfpathrectangle{\pgfqpoint{0.152333in}{3.075000in}}{\pgfqpoint{4.224218in}{2.565000in}}%
\pgfusepath{clip}%
\pgfsetbuttcap%
\pgfsetroundjoin%
\pgfsetlinewidth{0.501875pt}%
\definecolor{currentstroke}{rgb}{0.000000,0.000000,1.000000}%
\pgfsetstrokecolor{currentstroke}%
\pgfsetdash{{0.500000pt}{0.825000pt}}{0.000000pt}%
\pgfpathmoveto{\pgfqpoint{3.416502in}{4.220042in}}%
\pgfpathlineto{\pgfqpoint{2.648462in}{3.331500in}}%
\pgfusepath{stroke}%
\end{pgfscope}%
\begin{pgfscope}%
\pgfpathrectangle{\pgfqpoint{0.152333in}{3.075000in}}{\pgfqpoint{4.224218in}{2.565000in}}%
\pgfusepath{clip}%
\pgfsetbuttcap%
\pgfsetroundjoin%
\pgfsetlinewidth{0.501875pt}%
\definecolor{currentstroke}{rgb}{0.000000,0.000000,1.000000}%
\pgfsetstrokecolor{currentstroke}%
\pgfsetdash{{0.500000pt}{0.825000pt}}{0.000000pt}%
\pgfpathmoveto{\pgfqpoint{1.112383in}{4.220042in}}%
\pgfpathlineto{\pgfqpoint{1.880422in}{3.331500in}}%
\pgfusepath{stroke}%
\end{pgfscope}%
\begin{pgfscope}%
\pgfpathrectangle{\pgfqpoint{0.152333in}{3.075000in}}{\pgfqpoint{4.224218in}{2.565000in}}%
\pgfusepath{clip}%
\pgfsetbuttcap%
\pgfsetroundjoin%
\pgfsetlinewidth{0.501875pt}%
\definecolor{currentstroke}{rgb}{0.000000,0.000000,1.000000}%
\pgfsetstrokecolor{currentstroke}%
\pgfsetdash{{0.500000pt}{0.825000pt}}{0.000000pt}%
\pgfpathmoveto{\pgfqpoint{3.608512in}{3.997907in}}%
\pgfpathlineto{\pgfqpoint{3.032482in}{3.331500in}}%
\pgfusepath{stroke}%
\end{pgfscope}%
\begin{pgfscope}%
\pgfpathrectangle{\pgfqpoint{0.152333in}{3.075000in}}{\pgfqpoint{4.224218in}{2.565000in}}%
\pgfusepath{clip}%
\pgfsetbuttcap%
\pgfsetroundjoin%
\pgfsetlinewidth{0.501875pt}%
\definecolor{currentstroke}{rgb}{0.000000,0.000000,1.000000}%
\pgfsetstrokecolor{currentstroke}%
\pgfsetdash{{0.500000pt}{0.825000pt}}{0.000000pt}%
\pgfpathmoveto{\pgfqpoint{0.920373in}{3.997907in}}%
\pgfpathlineto{\pgfqpoint{1.496402in}{3.331500in}}%
\pgfusepath{stroke}%
\end{pgfscope}%
\begin{pgfscope}%
\pgfpathrectangle{\pgfqpoint{0.152333in}{3.075000in}}{\pgfqpoint{4.224218in}{2.565000in}}%
\pgfusepath{clip}%
\pgfsetbuttcap%
\pgfsetroundjoin%
\pgfsetlinewidth{0.501875pt}%
\definecolor{currentstroke}{rgb}{0.000000,0.000000,1.000000}%
\pgfsetstrokecolor{currentstroke}%
\pgfsetdash{{0.500000pt}{0.825000pt}}{0.000000pt}%
\pgfpathmoveto{\pgfqpoint{3.800522in}{3.775771in}}%
\pgfpathlineto{\pgfqpoint{3.416502in}{3.331500in}}%
\pgfusepath{stroke}%
\end{pgfscope}%
\begin{pgfscope}%
\pgfpathrectangle{\pgfqpoint{0.152333in}{3.075000in}}{\pgfqpoint{4.224218in}{2.565000in}}%
\pgfusepath{clip}%
\pgfsetbuttcap%
\pgfsetroundjoin%
\pgfsetlinewidth{0.501875pt}%
\definecolor{currentstroke}{rgb}{0.000000,0.000000,1.000000}%
\pgfsetstrokecolor{currentstroke}%
\pgfsetdash{{0.500000pt}{0.825000pt}}{0.000000pt}%
\pgfpathmoveto{\pgfqpoint{0.728363in}{3.775771in}}%
\pgfpathlineto{\pgfqpoint{1.112383in}{3.331500in}}%
\pgfusepath{stroke}%
\end{pgfscope}%
\begin{pgfscope}%
\pgfpathrectangle{\pgfqpoint{0.152333in}{3.075000in}}{\pgfqpoint{4.224218in}{2.565000in}}%
\pgfusepath{clip}%
\pgfsetbuttcap%
\pgfsetroundjoin%
\pgfsetlinewidth{0.501875pt}%
\definecolor{currentstroke}{rgb}{0.000000,0.000000,1.000000}%
\pgfsetstrokecolor{currentstroke}%
\pgfsetdash{{0.500000pt}{0.825000pt}}{0.000000pt}%
\pgfpathmoveto{\pgfqpoint{3.992531in}{3.553636in}}%
\pgfpathlineto{\pgfqpoint{3.800522in}{3.331500in}}%
\pgfusepath{stroke}%
\end{pgfscope}%
\begin{pgfscope}%
\pgfpathrectangle{\pgfqpoint{0.152333in}{3.075000in}}{\pgfqpoint{4.224218in}{2.565000in}}%
\pgfusepath{clip}%
\pgfsetbuttcap%
\pgfsetroundjoin%
\pgfsetlinewidth{0.501875pt}%
\definecolor{currentstroke}{rgb}{0.000000,0.000000,1.000000}%
\pgfsetstrokecolor{currentstroke}%
\pgfsetdash{{0.500000pt}{0.825000pt}}{0.000000pt}%
\pgfpathmoveto{\pgfqpoint{0.536353in}{3.553636in}}%
\pgfpathlineto{\pgfqpoint{0.728363in}{3.331500in}}%
\pgfusepath{stroke}%
\end{pgfscope}%
\begin{pgfscope}%
\pgfpathrectangle{\pgfqpoint{0.152333in}{3.075000in}}{\pgfqpoint{4.224218in}{2.565000in}}%
\pgfusepath{clip}%
\pgfsetbuttcap%
\pgfsetroundjoin%
\pgfsetlinewidth{0.501875pt}%
\definecolor{currentstroke}{rgb}{0.000000,0.000000,1.000000}%
\pgfsetstrokecolor{currentstroke}%
\pgfsetdash{{0.500000pt}{0.825000pt}}{0.000000pt}%
\pgfpathmoveto{\pgfqpoint{4.184541in}{3.331500in}}%
\pgfpathlineto{\pgfqpoint{4.184541in}{3.331500in}}%
\pgfusepath{stroke}%
\end{pgfscope}%
\begin{pgfscope}%
\pgfpathrectangle{\pgfqpoint{0.152333in}{3.075000in}}{\pgfqpoint{4.224218in}{2.565000in}}%
\pgfusepath{clip}%
\pgfsetbuttcap%
\pgfsetroundjoin%
\pgfsetlinewidth{0.501875pt}%
\definecolor{currentstroke}{rgb}{0.000000,0.000000,1.000000}%
\pgfsetstrokecolor{currentstroke}%
\pgfsetdash{{0.500000pt}{0.825000pt}}{0.000000pt}%
\pgfpathmoveto{\pgfqpoint{0.344343in}{3.331500in}}%
\pgfpathlineto{\pgfqpoint{0.344343in}{3.331500in}}%
\pgfusepath{stroke}%
\end{pgfscope}%
\begin{pgfscope}%
\pgfpathrectangle{\pgfqpoint{0.152333in}{3.075000in}}{\pgfqpoint{4.224218in}{2.565000in}}%
\pgfusepath{clip}%
\pgfsetrectcap%
\pgfsetroundjoin%
\pgfsetlinewidth{1.003750pt}%
\definecolor{currentstroke}{rgb}{0.000000,0.000000,0.000000}%
\pgfsetstrokecolor{currentstroke}%
\pgfsetdash{}{0pt}%
\pgfpathmoveto{\pgfqpoint{4.184541in}{3.331500in}}%
\pgfpathlineto{\pgfqpoint{4.222943in}{3.331500in}}%
\pgfusepath{stroke}%
\end{pgfscope}%
\begin{pgfscope}%
\pgfpathrectangle{\pgfqpoint{0.152333in}{3.075000in}}{\pgfqpoint{4.224218in}{2.565000in}}%
\pgfusepath{clip}%
\pgfsetrectcap%
\pgfsetroundjoin%
\pgfsetlinewidth{1.003750pt}%
\definecolor{currentstroke}{rgb}{0.000000,0.000000,0.000000}%
\pgfsetstrokecolor{currentstroke}%
\pgfsetdash{}{0pt}%
\pgfpathmoveto{\pgfqpoint{3.992531in}{3.553636in}}%
\pgfpathlineto{\pgfqpoint{4.030933in}{3.553636in}}%
\pgfusepath{stroke}%
\end{pgfscope}%
\begin{pgfscope}%
\pgfpathrectangle{\pgfqpoint{0.152333in}{3.075000in}}{\pgfqpoint{4.224218in}{2.565000in}}%
\pgfusepath{clip}%
\pgfsetrectcap%
\pgfsetroundjoin%
\pgfsetlinewidth{1.003750pt}%
\definecolor{currentstroke}{rgb}{0.000000,0.000000,0.000000}%
\pgfsetstrokecolor{currentstroke}%
\pgfsetdash{}{0pt}%
\pgfpathmoveto{\pgfqpoint{3.800522in}{3.775771in}}%
\pgfpathlineto{\pgfqpoint{3.838924in}{3.775771in}}%
\pgfusepath{stroke}%
\end{pgfscope}%
\begin{pgfscope}%
\pgfpathrectangle{\pgfqpoint{0.152333in}{3.075000in}}{\pgfqpoint{4.224218in}{2.565000in}}%
\pgfusepath{clip}%
\pgfsetrectcap%
\pgfsetroundjoin%
\pgfsetlinewidth{1.003750pt}%
\definecolor{currentstroke}{rgb}{0.000000,0.000000,0.000000}%
\pgfsetstrokecolor{currentstroke}%
\pgfsetdash{}{0pt}%
\pgfpathmoveto{\pgfqpoint{3.608512in}{3.997907in}}%
\pgfpathlineto{\pgfqpoint{3.646914in}{3.997907in}}%
\pgfusepath{stroke}%
\end{pgfscope}%
\begin{pgfscope}%
\pgfpathrectangle{\pgfqpoint{0.152333in}{3.075000in}}{\pgfqpoint{4.224218in}{2.565000in}}%
\pgfusepath{clip}%
\pgfsetrectcap%
\pgfsetroundjoin%
\pgfsetlinewidth{1.003750pt}%
\definecolor{currentstroke}{rgb}{0.000000,0.000000,0.000000}%
\pgfsetstrokecolor{currentstroke}%
\pgfsetdash{}{0pt}%
\pgfpathmoveto{\pgfqpoint{3.416502in}{4.220042in}}%
\pgfpathlineto{\pgfqpoint{3.454904in}{4.220042in}}%
\pgfusepath{stroke}%
\end{pgfscope}%
\begin{pgfscope}%
\pgfpathrectangle{\pgfqpoint{0.152333in}{3.075000in}}{\pgfqpoint{4.224218in}{2.565000in}}%
\pgfusepath{clip}%
\pgfsetrectcap%
\pgfsetroundjoin%
\pgfsetlinewidth{1.003750pt}%
\definecolor{currentstroke}{rgb}{0.000000,0.000000,0.000000}%
\pgfsetstrokecolor{currentstroke}%
\pgfsetdash{}{0pt}%
\pgfpathmoveto{\pgfqpoint{3.224492in}{4.442178in}}%
\pgfpathlineto{\pgfqpoint{3.262894in}{4.442178in}}%
\pgfusepath{stroke}%
\end{pgfscope}%
\begin{pgfscope}%
\pgfpathrectangle{\pgfqpoint{0.152333in}{3.075000in}}{\pgfqpoint{4.224218in}{2.565000in}}%
\pgfusepath{clip}%
\pgfsetrectcap%
\pgfsetroundjoin%
\pgfsetlinewidth{1.003750pt}%
\definecolor{currentstroke}{rgb}{0.000000,0.000000,0.000000}%
\pgfsetstrokecolor{currentstroke}%
\pgfsetdash{}{0pt}%
\pgfpathmoveto{\pgfqpoint{3.032482in}{4.664313in}}%
\pgfpathlineto{\pgfqpoint{3.070884in}{4.664313in}}%
\pgfusepath{stroke}%
\end{pgfscope}%
\begin{pgfscope}%
\pgfpathrectangle{\pgfqpoint{0.152333in}{3.075000in}}{\pgfqpoint{4.224218in}{2.565000in}}%
\pgfusepath{clip}%
\pgfsetrectcap%
\pgfsetroundjoin%
\pgfsetlinewidth{1.003750pt}%
\definecolor{currentstroke}{rgb}{0.000000,0.000000,0.000000}%
\pgfsetstrokecolor{currentstroke}%
\pgfsetdash{}{0pt}%
\pgfpathmoveto{\pgfqpoint{2.840472in}{4.886449in}}%
\pgfpathlineto{\pgfqpoint{2.878874in}{4.886449in}}%
\pgfusepath{stroke}%
\end{pgfscope}%
\begin{pgfscope}%
\pgfpathrectangle{\pgfqpoint{0.152333in}{3.075000in}}{\pgfqpoint{4.224218in}{2.565000in}}%
\pgfusepath{clip}%
\pgfsetrectcap%
\pgfsetroundjoin%
\pgfsetlinewidth{1.003750pt}%
\definecolor{currentstroke}{rgb}{0.000000,0.000000,0.000000}%
\pgfsetstrokecolor{currentstroke}%
\pgfsetdash{}{0pt}%
\pgfpathmoveto{\pgfqpoint{2.648462in}{5.108584in}}%
\pgfpathlineto{\pgfqpoint{2.686864in}{5.108584in}}%
\pgfusepath{stroke}%
\end{pgfscope}%
\begin{pgfscope}%
\pgfpathrectangle{\pgfqpoint{0.152333in}{3.075000in}}{\pgfqpoint{4.224218in}{2.565000in}}%
\pgfusepath{clip}%
\pgfsetrectcap%
\pgfsetroundjoin%
\pgfsetlinewidth{1.003750pt}%
\definecolor{currentstroke}{rgb}{0.000000,0.000000,0.000000}%
\pgfsetstrokecolor{currentstroke}%
\pgfsetdash{}{0pt}%
\pgfpathmoveto{\pgfqpoint{2.456452in}{5.330720in}}%
\pgfpathlineto{\pgfqpoint{2.494854in}{5.330720in}}%
\pgfusepath{stroke}%
\end{pgfscope}%
\begin{pgfscope}%
\pgfpathrectangle{\pgfqpoint{0.152333in}{3.075000in}}{\pgfqpoint{4.224218in}{2.565000in}}%
\pgfusepath{clip}%
\pgfsetrectcap%
\pgfsetroundjoin%
\pgfsetlinewidth{1.003750pt}%
\definecolor{currentstroke}{rgb}{0.000000,0.000000,0.000000}%
\pgfsetstrokecolor{currentstroke}%
\pgfsetdash{}{0pt}%
\pgfpathmoveto{\pgfqpoint{2.264442in}{5.552855in}}%
\pgfpathlineto{\pgfqpoint{2.302844in}{5.552855in}}%
\pgfusepath{stroke}%
\end{pgfscope}%
\begin{pgfscope}%
\pgfpathrectangle{\pgfqpoint{0.152333in}{3.075000in}}{\pgfqpoint{4.224218in}{2.565000in}}%
\pgfusepath{clip}%
\pgfsetrectcap%
\pgfsetroundjoin%
\pgfsetlinewidth{1.003750pt}%
\definecolor{currentstroke}{rgb}{0.000000,0.000000,0.000000}%
\pgfsetstrokecolor{currentstroke}%
\pgfsetdash{}{0pt}%
\pgfpathmoveto{\pgfqpoint{0.344343in}{3.331500in}}%
\pgfpathlineto{\pgfqpoint{0.325142in}{3.353714in}}%
\pgfusepath{stroke}%
\end{pgfscope}%
\begin{pgfscope}%
\pgfpathrectangle{\pgfqpoint{0.152333in}{3.075000in}}{\pgfqpoint{4.224218in}{2.565000in}}%
\pgfusepath{clip}%
\pgfsetrectcap%
\pgfsetroundjoin%
\pgfsetlinewidth{1.003750pt}%
\definecolor{currentstroke}{rgb}{0.000000,0.000000,0.000000}%
\pgfsetstrokecolor{currentstroke}%
\pgfsetdash{}{0pt}%
\pgfpathmoveto{\pgfqpoint{0.536353in}{3.553636in}}%
\pgfpathlineto{\pgfqpoint{0.517152in}{3.575849in}}%
\pgfusepath{stroke}%
\end{pgfscope}%
\begin{pgfscope}%
\pgfpathrectangle{\pgfqpoint{0.152333in}{3.075000in}}{\pgfqpoint{4.224218in}{2.565000in}}%
\pgfusepath{clip}%
\pgfsetrectcap%
\pgfsetroundjoin%
\pgfsetlinewidth{1.003750pt}%
\definecolor{currentstroke}{rgb}{0.000000,0.000000,0.000000}%
\pgfsetstrokecolor{currentstroke}%
\pgfsetdash{}{0pt}%
\pgfpathmoveto{\pgfqpoint{0.728363in}{3.775771in}}%
\pgfpathlineto{\pgfqpoint{0.709162in}{3.797985in}}%
\pgfusepath{stroke}%
\end{pgfscope}%
\begin{pgfscope}%
\pgfpathrectangle{\pgfqpoint{0.152333in}{3.075000in}}{\pgfqpoint{4.224218in}{2.565000in}}%
\pgfusepath{clip}%
\pgfsetrectcap%
\pgfsetroundjoin%
\pgfsetlinewidth{1.003750pt}%
\definecolor{currentstroke}{rgb}{0.000000,0.000000,0.000000}%
\pgfsetstrokecolor{currentstroke}%
\pgfsetdash{}{0pt}%
\pgfpathmoveto{\pgfqpoint{0.920373in}{3.997907in}}%
\pgfpathlineto{\pgfqpoint{0.901172in}{4.020120in}}%
\pgfusepath{stroke}%
\end{pgfscope}%
\begin{pgfscope}%
\pgfpathrectangle{\pgfqpoint{0.152333in}{3.075000in}}{\pgfqpoint{4.224218in}{2.565000in}}%
\pgfusepath{clip}%
\pgfsetrectcap%
\pgfsetroundjoin%
\pgfsetlinewidth{1.003750pt}%
\definecolor{currentstroke}{rgb}{0.000000,0.000000,0.000000}%
\pgfsetstrokecolor{currentstroke}%
\pgfsetdash{}{0pt}%
\pgfpathmoveto{\pgfqpoint{1.112383in}{4.220042in}}%
\pgfpathlineto{\pgfqpoint{1.093182in}{4.242256in}}%
\pgfusepath{stroke}%
\end{pgfscope}%
\begin{pgfscope}%
\pgfpathrectangle{\pgfqpoint{0.152333in}{3.075000in}}{\pgfqpoint{4.224218in}{2.565000in}}%
\pgfusepath{clip}%
\pgfsetrectcap%
\pgfsetroundjoin%
\pgfsetlinewidth{1.003750pt}%
\definecolor{currentstroke}{rgb}{0.000000,0.000000,0.000000}%
\pgfsetstrokecolor{currentstroke}%
\pgfsetdash{}{0pt}%
\pgfpathmoveto{\pgfqpoint{1.304393in}{4.442178in}}%
\pgfpathlineto{\pgfqpoint{1.285192in}{4.464391in}}%
\pgfusepath{stroke}%
\end{pgfscope}%
\begin{pgfscope}%
\pgfpathrectangle{\pgfqpoint{0.152333in}{3.075000in}}{\pgfqpoint{4.224218in}{2.565000in}}%
\pgfusepath{clip}%
\pgfsetrectcap%
\pgfsetroundjoin%
\pgfsetlinewidth{1.003750pt}%
\definecolor{currentstroke}{rgb}{0.000000,0.000000,0.000000}%
\pgfsetstrokecolor{currentstroke}%
\pgfsetdash{}{0pt}%
\pgfpathmoveto{\pgfqpoint{1.496402in}{4.664313in}}%
\pgfpathlineto{\pgfqpoint{1.477201in}{4.686527in}}%
\pgfusepath{stroke}%
\end{pgfscope}%
\begin{pgfscope}%
\pgfpathrectangle{\pgfqpoint{0.152333in}{3.075000in}}{\pgfqpoint{4.224218in}{2.565000in}}%
\pgfusepath{clip}%
\pgfsetrectcap%
\pgfsetroundjoin%
\pgfsetlinewidth{1.003750pt}%
\definecolor{currentstroke}{rgb}{0.000000,0.000000,0.000000}%
\pgfsetstrokecolor{currentstroke}%
\pgfsetdash{}{0pt}%
\pgfpathmoveto{\pgfqpoint{1.688412in}{4.886449in}}%
\pgfpathlineto{\pgfqpoint{1.669211in}{4.908662in}}%
\pgfusepath{stroke}%
\end{pgfscope}%
\begin{pgfscope}%
\pgfpathrectangle{\pgfqpoint{0.152333in}{3.075000in}}{\pgfqpoint{4.224218in}{2.565000in}}%
\pgfusepath{clip}%
\pgfsetrectcap%
\pgfsetroundjoin%
\pgfsetlinewidth{1.003750pt}%
\definecolor{currentstroke}{rgb}{0.000000,0.000000,0.000000}%
\pgfsetstrokecolor{currentstroke}%
\pgfsetdash{}{0pt}%
\pgfpathmoveto{\pgfqpoint{1.880422in}{5.108584in}}%
\pgfpathlineto{\pgfqpoint{1.861221in}{5.130798in}}%
\pgfusepath{stroke}%
\end{pgfscope}%
\begin{pgfscope}%
\pgfpathrectangle{\pgfqpoint{0.152333in}{3.075000in}}{\pgfqpoint{4.224218in}{2.565000in}}%
\pgfusepath{clip}%
\pgfsetrectcap%
\pgfsetroundjoin%
\pgfsetlinewidth{1.003750pt}%
\definecolor{currentstroke}{rgb}{0.000000,0.000000,0.000000}%
\pgfsetstrokecolor{currentstroke}%
\pgfsetdash{}{0pt}%
\pgfpathmoveto{\pgfqpoint{2.072432in}{5.330720in}}%
\pgfpathlineto{\pgfqpoint{2.053231in}{5.352933in}}%
\pgfusepath{stroke}%
\end{pgfscope}%
\begin{pgfscope}%
\pgfpathrectangle{\pgfqpoint{0.152333in}{3.075000in}}{\pgfqpoint{4.224218in}{2.565000in}}%
\pgfusepath{clip}%
\pgfsetrectcap%
\pgfsetroundjoin%
\pgfsetlinewidth{1.003750pt}%
\definecolor{currentstroke}{rgb}{0.000000,0.000000,0.000000}%
\pgfsetstrokecolor{currentstroke}%
\pgfsetdash{}{0pt}%
\pgfpathmoveto{\pgfqpoint{2.264442in}{5.552855in}}%
\pgfpathlineto{\pgfqpoint{2.245241in}{5.575069in}}%
\pgfusepath{stroke}%
\end{pgfscope}%
\begin{pgfscope}%
\pgfpathrectangle{\pgfqpoint{0.152333in}{3.075000in}}{\pgfqpoint{4.224218in}{2.565000in}}%
\pgfusepath{clip}%
\pgfsetrectcap%
\pgfsetroundjoin%
\pgfsetlinewidth{1.003750pt}%
\definecolor{currentstroke}{rgb}{0.000000,0.000000,0.000000}%
\pgfsetstrokecolor{currentstroke}%
\pgfsetdash{}{0pt}%
\pgfpathmoveto{\pgfqpoint{0.344343in}{3.331500in}}%
\pgfpathlineto{\pgfqpoint{0.325142in}{3.309286in}}%
\pgfusepath{stroke}%
\end{pgfscope}%
\begin{pgfscope}%
\pgfpathrectangle{\pgfqpoint{0.152333in}{3.075000in}}{\pgfqpoint{4.224218in}{2.565000in}}%
\pgfusepath{clip}%
\pgfsetrectcap%
\pgfsetroundjoin%
\pgfsetlinewidth{1.003750pt}%
\definecolor{currentstroke}{rgb}{0.000000,0.000000,0.000000}%
\pgfsetstrokecolor{currentstroke}%
\pgfsetdash{}{0pt}%
\pgfpathmoveto{\pgfqpoint{0.728363in}{3.331500in}}%
\pgfpathlineto{\pgfqpoint{0.709162in}{3.309286in}}%
\pgfusepath{stroke}%
\end{pgfscope}%
\begin{pgfscope}%
\pgfpathrectangle{\pgfqpoint{0.152333in}{3.075000in}}{\pgfqpoint{4.224218in}{2.565000in}}%
\pgfusepath{clip}%
\pgfsetrectcap%
\pgfsetroundjoin%
\pgfsetlinewidth{1.003750pt}%
\definecolor{currentstroke}{rgb}{0.000000,0.000000,0.000000}%
\pgfsetstrokecolor{currentstroke}%
\pgfsetdash{}{0pt}%
\pgfpathmoveto{\pgfqpoint{1.112383in}{3.331500in}}%
\pgfpathlineto{\pgfqpoint{1.093182in}{3.309286in}}%
\pgfusepath{stroke}%
\end{pgfscope}%
\begin{pgfscope}%
\pgfpathrectangle{\pgfqpoint{0.152333in}{3.075000in}}{\pgfqpoint{4.224218in}{2.565000in}}%
\pgfusepath{clip}%
\pgfsetrectcap%
\pgfsetroundjoin%
\pgfsetlinewidth{1.003750pt}%
\definecolor{currentstroke}{rgb}{0.000000,0.000000,0.000000}%
\pgfsetstrokecolor{currentstroke}%
\pgfsetdash{}{0pt}%
\pgfpathmoveto{\pgfqpoint{1.496402in}{3.331500in}}%
\pgfpathlineto{\pgfqpoint{1.477201in}{3.309286in}}%
\pgfusepath{stroke}%
\end{pgfscope}%
\begin{pgfscope}%
\pgfpathrectangle{\pgfqpoint{0.152333in}{3.075000in}}{\pgfqpoint{4.224218in}{2.565000in}}%
\pgfusepath{clip}%
\pgfsetrectcap%
\pgfsetroundjoin%
\pgfsetlinewidth{1.003750pt}%
\definecolor{currentstroke}{rgb}{0.000000,0.000000,0.000000}%
\pgfsetstrokecolor{currentstroke}%
\pgfsetdash{}{0pt}%
\pgfpathmoveto{\pgfqpoint{1.880422in}{3.331500in}}%
\pgfpathlineto{\pgfqpoint{1.861221in}{3.309286in}}%
\pgfusepath{stroke}%
\end{pgfscope}%
\begin{pgfscope}%
\pgfpathrectangle{\pgfqpoint{0.152333in}{3.075000in}}{\pgfqpoint{4.224218in}{2.565000in}}%
\pgfusepath{clip}%
\pgfsetrectcap%
\pgfsetroundjoin%
\pgfsetlinewidth{1.003750pt}%
\definecolor{currentstroke}{rgb}{0.000000,0.000000,0.000000}%
\pgfsetstrokecolor{currentstroke}%
\pgfsetdash{}{0pt}%
\pgfpathmoveto{\pgfqpoint{2.264442in}{3.331500in}}%
\pgfpathlineto{\pgfqpoint{2.245241in}{3.309286in}}%
\pgfusepath{stroke}%
\end{pgfscope}%
\begin{pgfscope}%
\pgfpathrectangle{\pgfqpoint{0.152333in}{3.075000in}}{\pgfqpoint{4.224218in}{2.565000in}}%
\pgfusepath{clip}%
\pgfsetrectcap%
\pgfsetroundjoin%
\pgfsetlinewidth{1.003750pt}%
\definecolor{currentstroke}{rgb}{0.000000,0.000000,0.000000}%
\pgfsetstrokecolor{currentstroke}%
\pgfsetdash{}{0pt}%
\pgfpathmoveto{\pgfqpoint{2.648462in}{3.331500in}}%
\pgfpathlineto{\pgfqpoint{2.629261in}{3.309286in}}%
\pgfusepath{stroke}%
\end{pgfscope}%
\begin{pgfscope}%
\pgfpathrectangle{\pgfqpoint{0.152333in}{3.075000in}}{\pgfqpoint{4.224218in}{2.565000in}}%
\pgfusepath{clip}%
\pgfsetrectcap%
\pgfsetroundjoin%
\pgfsetlinewidth{1.003750pt}%
\definecolor{currentstroke}{rgb}{0.000000,0.000000,0.000000}%
\pgfsetstrokecolor{currentstroke}%
\pgfsetdash{}{0pt}%
\pgfpathmoveto{\pgfqpoint{3.032482in}{3.331500in}}%
\pgfpathlineto{\pgfqpoint{3.013281in}{3.309286in}}%
\pgfusepath{stroke}%
\end{pgfscope}%
\begin{pgfscope}%
\pgfpathrectangle{\pgfqpoint{0.152333in}{3.075000in}}{\pgfqpoint{4.224218in}{2.565000in}}%
\pgfusepath{clip}%
\pgfsetrectcap%
\pgfsetroundjoin%
\pgfsetlinewidth{1.003750pt}%
\definecolor{currentstroke}{rgb}{0.000000,0.000000,0.000000}%
\pgfsetstrokecolor{currentstroke}%
\pgfsetdash{}{0pt}%
\pgfpathmoveto{\pgfqpoint{3.416502in}{3.331500in}}%
\pgfpathlineto{\pgfqpoint{3.397301in}{3.309286in}}%
\pgfusepath{stroke}%
\end{pgfscope}%
\begin{pgfscope}%
\pgfpathrectangle{\pgfqpoint{0.152333in}{3.075000in}}{\pgfqpoint{4.224218in}{2.565000in}}%
\pgfusepath{clip}%
\pgfsetrectcap%
\pgfsetroundjoin%
\pgfsetlinewidth{1.003750pt}%
\definecolor{currentstroke}{rgb}{0.000000,0.000000,0.000000}%
\pgfsetstrokecolor{currentstroke}%
\pgfsetdash{}{0pt}%
\pgfpathmoveto{\pgfqpoint{3.800522in}{3.331500in}}%
\pgfpathlineto{\pgfqpoint{3.781321in}{3.309286in}}%
\pgfusepath{stroke}%
\end{pgfscope}%
\begin{pgfscope}%
\pgfpathrectangle{\pgfqpoint{0.152333in}{3.075000in}}{\pgfqpoint{4.224218in}{2.565000in}}%
\pgfusepath{clip}%
\pgfsetrectcap%
\pgfsetroundjoin%
\pgfsetlinewidth{1.003750pt}%
\definecolor{currentstroke}{rgb}{0.000000,0.000000,0.000000}%
\pgfsetstrokecolor{currentstroke}%
\pgfsetdash{}{0pt}%
\pgfpathmoveto{\pgfqpoint{4.184541in}{3.331500in}}%
\pgfpathlineto{\pgfqpoint{4.165340in}{3.309286in}}%
\pgfusepath{stroke}%
\end{pgfscope}%
\begin{pgfscope}%
\pgfpathrectangle{\pgfqpoint{0.152333in}{3.075000in}}{\pgfqpoint{4.224218in}{2.565000in}}%
\pgfusepath{clip}%
\pgfsetrectcap%
\pgfsetroundjoin%
\pgfsetlinewidth{1.003750pt}%
\definecolor{currentstroke}{rgb}{0.000000,0.000000,1.000000}%
\pgfsetstrokecolor{currentstroke}%
\pgfsetdash{}{0pt}%
\pgfpathmoveto{\pgfqpoint{2.327865in}{4.157031in}}%
\pgfpathlineto{\pgfqpoint{2.351885in}{4.260695in}}%
\pgfpathlineto{\pgfqpoint{2.343362in}{4.377889in}}%
\pgfpathlineto{\pgfqpoint{2.313997in}{4.498639in}}%
\pgfpathlineto{\pgfqpoint{2.275765in}{4.611668in}}%
\pgfpathlineto{\pgfqpoint{2.237106in}{4.708991in}}%
\pgfpathlineto{\pgfqpoint{2.202008in}{4.787666in}}%
\pgfpathlineto{\pgfqpoint{2.171328in}{4.848474in}}%
\pgfpathlineto{\pgfqpoint{2.144439in}{4.893834in}}%
\pgfpathlineto{\pgfqpoint{2.120267in}{4.926394in}}%
\pgfpathlineto{\pgfqpoint{2.097742in}{4.948422in}}%
\pgfpathlineto{\pgfqpoint{2.075929in}{4.961652in}}%
\pgfpathlineto{\pgfqpoint{2.054030in}{4.967314in}}%
\pgfpathlineto{\pgfqpoint{2.031348in}{4.966218in}}%
\pgfpathlineto{\pgfqpoint{2.007248in}{4.958835in}}%
\pgfpathlineto{\pgfqpoint{1.981116in}{4.945363in}}%
\pgfpathlineto{\pgfqpoint{1.952341in}{4.925779in}}%
\pgfpathlineto{\pgfqpoint{1.920299in}{4.899889in}}%
\pgfpathlineto{\pgfqpoint{1.884353in}{4.867370in}}%
\pgfpathlineto{\pgfqpoint{1.843878in}{4.827831in}}%
\pgfpathlineto{\pgfqpoint{1.798306in}{4.780895in}}%
\pgfpathlineto{\pgfqpoint{1.747214in}{4.726316in}}%
\pgfpathlineto{\pgfqpoint{1.690439in}{4.664124in}}%
\pgfpathlineto{\pgfqpoint{1.561326in}{4.519387in}}%
\pgfpathlineto{\pgfqpoint{1.419444in}{4.357687in}}%
\pgfpathlineto{\pgfqpoint{1.280378in}{4.197400in}}%
\pgfpathlineto{\pgfqpoint{1.159497in}{4.055684in}}%
\pgfpathlineto{\pgfqpoint{1.108793in}{3.994799in}}%
\pgfpathlineto{\pgfqpoint{1.065119in}{3.940955in}}%
\pgfpathlineto{\pgfqpoint{1.028424in}{3.893913in}}%
\pgfpathlineto{\pgfqpoint{0.998470in}{3.853196in}}%
\pgfpathlineto{\pgfqpoint{0.974954in}{3.818227in}}%
\pgfpathlineto{\pgfqpoint{0.957607in}{3.788413in}}%
\pgfpathlineto{\pgfqpoint{0.946259in}{3.763205in}}%
\pgfpathlineto{\pgfqpoint{0.940893in}{3.742124in}}%
\pgfpathlineto{\pgfqpoint{0.941698in}{3.724783in}}%
\pgfpathlineto{\pgfqpoint{0.949125in}{3.710888in}}%
\pgfpathlineto{\pgfqpoint{0.963965in}{3.700241in}}%
\pgfpathlineto{\pgfqpoint{0.987464in}{3.692749in}}%
\pgfpathlineto{\pgfqpoint{1.021484in}{3.688427in}}%
\pgfpathlineto{\pgfqpoint{1.068734in}{3.687398in}}%
\pgfpathlineto{\pgfqpoint{1.133069in}{3.689902in}}%
\pgfpathlineto{\pgfqpoint{1.219801in}{3.696273in}}%
\pgfpathlineto{\pgfqpoint{1.335754in}{3.706882in}}%
\pgfpathlineto{\pgfqpoint{1.682261in}{3.741705in}}%
\pgfpathlineto{\pgfqpoint{1.912534in}{3.765939in}}%
\pgfpathlineto{\pgfqpoint{2.157628in}{3.795604in}}%
\pgfpathlineto{\pgfqpoint{2.382617in}{3.834140in}}%
\pgfpathlineto{\pgfqpoint{2.555237in}{3.888187in}}%
\pgfpathlineto{\pgfqpoint{2.658693in}{3.966467in}}%
\pgfpathlineto{\pgfqpoint{2.691145in}{4.077457in}}%
\pgfpathlineto{\pgfqpoint{2.661366in}{4.224747in}}%
\pgfpathlineto{\pgfqpoint{2.588441in}{4.399465in}}%
\pgfpathlineto{\pgfqpoint{2.499209in}{4.578090in}}%
\pgfpathlineto{\pgfqpoint{2.302485in}{4.952759in}}%
\pgfpathlineto{\pgfqpoint{2.266357in}{5.020688in}}%
\pgfpathlineto{\pgfqpoint{2.239009in}{5.070286in}}%
\pgfpathlineto{\pgfqpoint{2.217519in}{5.106618in}}%
\pgfpathlineto{\pgfqpoint{2.199885in}{5.133238in}}%
\pgfpathlineto{\pgfqpoint{2.184758in}{5.152568in}}%
\pgfpathlineto{\pgfqpoint{2.171213in}{5.166252in}}%
\pgfpathlineto{\pgfqpoint{2.158596in}{5.175404in}}%
\pgfpathlineto{\pgfqpoint{2.146421in}{5.180779in}}%
\pgfpathlineto{\pgfqpoint{2.134303in}{5.182872in}}%
\pgfpathlineto{\pgfqpoint{2.121921in}{5.181997in}}%
\pgfpathlineto{\pgfqpoint{2.108983in}{5.178323in}}%
\pgfpathlineto{\pgfqpoint{2.095205in}{5.171902in}}%
\pgfpathlineto{\pgfqpoint{2.080295in}{5.162686in}}%
\pgfpathlineto{\pgfqpoint{2.063937in}{5.150530in}}%
\pgfpathlineto{\pgfqpoint{2.045775in}{5.135197in}}%
\pgfpathlineto{\pgfqpoint{2.025401in}{5.116345in}}%
\pgfpathlineto{\pgfqpoint{2.002336in}{5.093531in}}%
\pgfpathlineto{\pgfqpoint{1.976018in}{5.066191in}}%
\pgfpathlineto{\pgfqpoint{1.910908in}{4.995077in}}%
\pgfpathlineto{\pgfqpoint{1.823765in}{4.896266in}}%
\pgfpathlineto{\pgfqpoint{1.638357in}{4.682237in}}%
\pgfpathlineto{\pgfqpoint{1.392299in}{4.398127in}}%
\pgfpathlineto{\pgfqpoint{1.074573in}{4.034105in}}%
\pgfpathlineto{\pgfqpoint{0.912355in}{3.847701in}}%
\pgfpathlineto{\pgfqpoint{0.840116in}{3.762718in}}%
\pgfpathlineto{\pgfqpoint{0.789501in}{3.700448in}}%
\pgfpathlineto{\pgfqpoint{0.770627in}{3.675787in}}%
\pgfpathlineto{\pgfqpoint{0.755361in}{3.654584in}}%
\pgfpathlineto{\pgfqpoint{0.743322in}{3.636331in}}%
\pgfpathlineto{\pgfqpoint{0.734226in}{3.620612in}}%
\pgfpathlineto{\pgfqpoint{0.727879in}{3.607090in}}%
\pgfpathlineto{\pgfqpoint{0.724173in}{3.595496in}}%
\pgfpathlineto{\pgfqpoint{0.723085in}{3.585617in}}%
\pgfpathlineto{\pgfqpoint{0.724682in}{3.577288in}}%
\pgfpathlineto{\pgfqpoint{0.729137in}{3.570390in}}%
\pgfpathlineto{\pgfqpoint{0.736744in}{3.564841in}}%
\pgfpathlineto{\pgfqpoint{0.747957in}{3.560601in}}%
\pgfpathlineto{\pgfqpoint{0.763446in}{3.557671in}}%
\pgfpathlineto{\pgfqpoint{0.784175in}{3.556095in}}%
\pgfpathlineto{\pgfqpoint{0.811544in}{3.555971in}}%
\pgfpathlineto{\pgfqpoint{0.847602in}{3.557458in}}%
\pgfpathlineto{\pgfqpoint{0.895381in}{3.560798in}}%
\pgfpathlineto{\pgfqpoint{1.046739in}{3.574443in}}%
\pgfpathlineto{\pgfqpoint{1.167648in}{3.585626in}}%
\pgfpathlineto{\pgfqpoint{1.336773in}{3.600149in}}%
\pgfpathlineto{\pgfqpoint{1.570584in}{3.617623in}}%
\pgfpathlineto{\pgfqpoint{1.875639in}{3.636436in}}%
\pgfpathlineto{\pgfqpoint{2.225958in}{3.654589in}}%
\pgfpathlineto{\pgfqpoint{2.559044in}{3.673057in}}%
\pgfpathlineto{\pgfqpoint{2.818217in}{3.697505in}}%
\pgfpathlineto{\pgfqpoint{2.986029in}{3.735827in}}%
\pgfpathlineto{\pgfqpoint{3.070101in}{3.796824in}}%
\pgfpathlineto{\pgfqpoint{3.080457in}{3.891536in}}%
\pgfpathlineto{\pgfqpoint{3.022339in}{4.033264in}}%
\pgfpathlineto{\pgfqpoint{2.903368in}{4.229617in}}%
\pgfpathlineto{\pgfqpoint{2.600632in}{4.688506in}}%
\pgfpathlineto{\pgfqpoint{2.486556in}{4.867857in}}%
\pgfpathlineto{\pgfqpoint{2.408118in}{4.996429in}}%
\pgfpathlineto{\pgfqpoint{2.355522in}{5.085978in}}%
\pgfpathlineto{\pgfqpoint{2.239652in}{5.288343in}}%
\pgfpathlineto{\pgfqpoint{2.224518in}{5.309759in}}%
\pgfpathlineto{\pgfqpoint{2.212352in}{5.322872in}}%
\pgfpathlineto{\pgfqpoint{2.206873in}{5.327187in}}%
\pgfpathlineto{\pgfqpoint{2.201620in}{5.330292in}}%
\pgfpathlineto{\pgfqpoint{2.196493in}{5.332331in}}%
\pgfpathlineto{\pgfqpoint{2.191406in}{5.333409in}}%
\pgfpathlineto{\pgfqpoint{2.186286in}{5.333598in}}%
\pgfpathlineto{\pgfqpoint{2.181068in}{5.332947in}}%
\pgfpathlineto{\pgfqpoint{2.170078in}{5.329209in}}%
\pgfpathlineto{\pgfqpoint{2.157923in}{5.322196in}}%
\pgfpathlineto{\pgfqpoint{2.144024in}{5.311633in}}%
\pgfpathlineto{\pgfqpoint{2.127661in}{5.296949in}}%
\pgfpathlineto{\pgfqpoint{2.107874in}{5.277202in}}%
\pgfpathlineto{\pgfqpoint{2.083313in}{5.250934in}}%
\pgfpathlineto{\pgfqpoint{2.032984in}{5.194150in}}%
\pgfpathlineto{\pgfqpoint{1.956104in}{5.104543in}}%
\pgfpathlineto{\pgfqpoint{1.551729in}{4.629759in}}%
\pgfpathlineto{\pgfqpoint{1.363783in}{4.411893in}}%
\pgfpathlineto{\pgfqpoint{1.174632in}{4.194671in}}%
\pgfpathlineto{\pgfqpoint{0.945405in}{3.934019in}}%
\pgfpathlineto{\pgfqpoint{0.665531in}{3.616574in}}%
\pgfpathlineto{\pgfqpoint{0.623817in}{3.567014in}}%
\pgfpathlineto{\pgfqpoint{0.597369in}{3.533180in}}%
\pgfpathlineto{\pgfqpoint{0.585722in}{3.516413in}}%
\pgfpathlineto{\pgfqpoint{0.577869in}{3.503015in}}%
\pgfpathlineto{\pgfqpoint{0.573344in}{3.492280in}}%
\pgfpathlineto{\pgfqpoint{0.571940in}{3.483718in}}%
\pgfpathlineto{\pgfqpoint{0.572411in}{3.480143in}}%
\pgfpathlineto{\pgfqpoint{0.573696in}{3.476999in}}%
\pgfpathlineto{\pgfqpoint{0.575840in}{3.474262in}}%
\pgfpathlineto{\pgfqpoint{0.578912in}{3.471917in}}%
\pgfpathlineto{\pgfqpoint{0.583003in}{3.469954in}}%
\pgfpathlineto{\pgfqpoint{0.588236in}{3.468372in}}%
\pgfpathlineto{\pgfqpoint{0.602836in}{3.466375in}}%
\pgfpathlineto{\pgfqpoint{0.624756in}{3.466069in}}%
\pgfpathlineto{\pgfqpoint{0.657674in}{3.467788in}}%
\pgfpathlineto{\pgfqpoint{0.708617in}{3.472186in}}%
\pgfpathlineto{\pgfqpoint{0.942982in}{3.494938in}}%
\pgfpathlineto{\pgfqpoint{1.067106in}{3.505522in}}%
\pgfpathlineto{\pgfqpoint{1.248335in}{3.518809in}}%
\pgfpathlineto{\pgfqpoint{1.514115in}{3.534269in}}%
\pgfpathlineto{\pgfqpoint{1.883124in}{3.549355in}}%
\pgfpathlineto{\pgfqpoint{2.321411in}{3.560023in}}%
\pgfpathlineto{\pgfqpoint{2.730542in}{3.565829in}}%
\pgfpathlineto{\pgfqpoint{3.035530in}{3.571832in}}%
\pgfpathlineto{\pgfqpoint{3.233485in}{3.583391in}}%
\pgfpathlineto{\pgfqpoint{3.351122in}{3.604158in}}%
\pgfpathlineto{\pgfqpoint{3.410963in}{3.637799in}}%
\pgfpathlineto{\pgfqpoint{3.424519in}{3.690018in}}%
\pgfpathlineto{\pgfqpoint{3.393401in}{3.770600in}}%
\pgfpathlineto{\pgfqpoint{3.311011in}{3.895378in}}%
\pgfpathlineto{\pgfqpoint{3.167742in}{4.083887in}}%
\pgfpathlineto{\pgfqpoint{2.969733in}{4.339338in}}%
\pgfpathlineto{\pgfqpoint{2.761525in}{4.615888in}}%
\pgfpathlineto{\pgfqpoint{2.596017in}{4.846239in}}%
\pgfpathlineto{\pgfqpoint{2.485455in}{5.008025in}}%
\pgfpathlineto{\pgfqpoint{2.415148in}{5.115873in}}%
\pgfpathlineto{\pgfqpoint{2.369585in}{5.188731in}}%
\pgfpathlineto{\pgfqpoint{2.317090in}{5.276661in}}%
\pgfpathlineto{\pgfqpoint{2.248006in}{5.395828in}}%
\pgfpathlineto{\pgfqpoint{2.239214in}{5.407629in}}%
\pgfpathlineto{\pgfqpoint{2.232059in}{5.414740in}}%
\pgfpathlineto{\pgfqpoint{2.225672in}{5.418624in}}%
\pgfpathlineto{\pgfqpoint{2.219531in}{5.420044in}}%
\pgfpathlineto{\pgfqpoint{2.213264in}{5.419379in}}%
\pgfpathlineto{\pgfqpoint{2.206553in}{5.416769in}}%
\pgfpathlineto{\pgfqpoint{2.199076in}{5.412166in}}%
\pgfpathlineto{\pgfqpoint{2.190457in}{5.405349in}}%
\pgfpathlineto{\pgfqpoint{2.176311in}{5.392030in}}%
\pgfpathlineto{\pgfqpoint{2.157464in}{5.372094in}}%
\pgfpathlineto{\pgfqpoint{2.122315in}{5.332127in}}%
\pgfpathlineto{\pgfqpoint{2.061341in}{5.260152in}}%
\pgfpathlineto{\pgfqpoint{1.799487in}{4.948852in}}%
\pgfpathlineto{\pgfqpoint{1.574532in}{4.684656in}}%
\pgfpathlineto{\pgfqpoint{1.378318in}{4.456617in}}%
\pgfpathlineto{\pgfqpoint{1.175226in}{4.222750in}}%
\pgfpathlineto{\pgfqpoint{0.926796in}{3.939680in}}%
\pgfpathlineto{\pgfqpoint{0.698289in}{3.682383in}}%
\pgfpathlineto{\pgfqpoint{0.543223in}{3.507823in}}%
\pgfpathlineto{\pgfqpoint{0.509725in}{3.467956in}}%
\pgfpathlineto{\pgfqpoint{0.494505in}{3.447657in}}%
\pgfpathlineto{\pgfqpoint{0.487611in}{3.436302in}}%
\pgfpathlineto{\pgfqpoint{0.484888in}{3.429778in}}%
\pgfpathlineto{\pgfqpoint{0.484087in}{3.424591in}}%
\pgfpathlineto{\pgfqpoint{0.485220in}{3.420531in}}%
\pgfpathlineto{\pgfqpoint{0.487133in}{3.418383in}}%
\pgfpathlineto{\pgfqpoint{0.492009in}{3.415947in}}%
\pgfpathlineto{\pgfqpoint{0.499860in}{3.414449in}}%
\pgfpathlineto{\pgfqpoint{0.511784in}{3.413958in}}%
\pgfpathlineto{\pgfqpoint{0.529776in}{3.414665in}}%
\pgfpathlineto{\pgfqpoint{0.570476in}{3.418209in}}%
\pgfpathlineto{\pgfqpoint{0.800120in}{3.440874in}}%
\pgfpathlineto{\pgfqpoint{0.887956in}{3.448157in}}%
\pgfpathlineto{\pgfqpoint{1.015494in}{3.457433in}}%
\pgfpathlineto{\pgfqpoint{1.207659in}{3.469003in}}%
\pgfpathlineto{\pgfqpoint{1.501066in}{3.482284in}}%
\pgfpathlineto{\pgfqpoint{1.923251in}{3.494285in}}%
\pgfpathlineto{\pgfqpoint{2.425491in}{3.500140in}}%
\pgfpathlineto{\pgfqpoint{2.873635in}{3.499495in}}%
\pgfpathlineto{\pgfqpoint{3.189683in}{3.497474in}}%
\pgfpathlineto{\pgfqpoint{3.390724in}{3.497996in}}%
\pgfpathlineto{\pgfqpoint{3.516025in}{3.502478in}}%
\pgfpathlineto{\pgfqpoint{3.593589in}{3.511524in}}%
\pgfpathlineto{\pgfqpoint{3.639343in}{3.525930in}}%
\pgfpathlineto{\pgfqpoint{3.661416in}{3.547170in}}%
\pgfpathlineto{\pgfqpoint{3.662916in}{3.577885in}}%
\pgfpathlineto{\pgfqpoint{3.642966in}{3.622725in}}%
\pgfpathlineto{\pgfqpoint{3.596374in}{3.689922in}}%
\pgfpathlineto{\pgfqpoint{3.512237in}{3.793921in}}%
\pgfpathlineto{\pgfqpoint{3.373515in}{3.957499in}}%
\pgfpathlineto{\pgfqpoint{3.167528in}{4.202122in}}%
\pgfpathlineto{\pgfqpoint{2.920724in}{4.505889in}}%
\pgfpathlineto{\pgfqpoint{2.704107in}{4.785464in}}%
\pgfpathlineto{\pgfqpoint{2.556256in}{4.985904in}}%
\pgfpathlineto{\pgfqpoint{2.464650in}{5.115986in}}%
\pgfpathlineto{\pgfqpoint{2.407453in}{5.200668in}}%
\pgfpathlineto{\pgfqpoint{2.344545in}{5.298575in}}%
\pgfpathlineto{\pgfqpoint{2.302113in}{5.369026in}}%
\pgfpathlineto{\pgfqpoint{2.250671in}{5.456728in}}%
\pgfpathlineto{\pgfqpoint{2.245140in}{5.462975in}}%
\pgfpathlineto{\pgfqpoint{2.240432in}{5.466289in}}%
\pgfpathlineto{\pgfqpoint{2.235999in}{5.467514in}}%
\pgfpathlineto{\pgfqpoint{2.231485in}{5.467038in}}%
\pgfpathlineto{\pgfqpoint{2.226596in}{5.464985in}}%
\pgfpathlineto{\pgfqpoint{2.219804in}{5.460331in}}%
\pgfpathlineto{\pgfqpoint{2.209720in}{5.451201in}}%
\pgfpathlineto{\pgfqpoint{2.195588in}{5.436287in}}%
\pgfpathlineto{\pgfqpoint{2.169720in}{5.406623in}}%
\pgfpathlineto{\pgfqpoint{2.109642in}{5.334985in}}%
\pgfpathlineto{\pgfqpoint{1.924936in}{5.114134in}}%
\pgfpathlineto{\pgfqpoint{1.772882in}{4.934225in}}%
\pgfpathlineto{\pgfqpoint{1.620171in}{4.755001in}}%
\pgfpathlineto{\pgfqpoint{1.620171in}{4.755001in}}%
\pgfusepath{stroke}%
\end{pgfscope}%
\begin{pgfscope}%
\pgfpathrectangle{\pgfqpoint{0.152333in}{3.075000in}}{\pgfqpoint{4.224218in}{2.565000in}}%
\pgfusepath{clip}%
\pgfsetbuttcap%
\pgfsetroundjoin%
\pgfsetlinewidth{1.003750pt}%
\definecolor{currentstroke}{rgb}{0.501961,0.501961,0.501961}%
\pgfsetstrokecolor{currentstroke}%
\pgfsetdash{{3.700000pt}{1.600000pt}}{0.000000pt}%
\pgfpathmoveto{\pgfqpoint{2.327865in}{4.157031in}}%
\pgfpathlineto{\pgfqpoint{2.351885in}{4.260695in}}%
\pgfpathlineto{\pgfqpoint{2.347623in}{4.319292in}}%
\pgfpathlineto{\pgfqpoint{2.336415in}{4.379074in}}%
\pgfpathlineto{\pgfqpoint{2.321252in}{4.437223in}}%
\pgfpathlineto{\pgfqpoint{2.304423in}{4.491576in}}%
\pgfpathlineto{\pgfqpoint{2.287354in}{4.540925in}}%
\pgfpathlineto{\pgfqpoint{2.270778in}{4.584860in}}%
\pgfpathlineto{\pgfqpoint{2.254986in}{4.623482in}}%
\pgfpathlineto{\pgfqpoint{2.240017in}{4.657139in}}%
\pgfpathlineto{\pgfqpoint{2.225790in}{4.686267in}}%
\pgfpathlineto{\pgfqpoint{2.212166in}{4.711302in}}%
\pgfpathlineto{\pgfqpoint{2.198988in}{4.732637in}}%
\pgfpathlineto{\pgfqpoint{2.186093in}{4.750604in}}%
\pgfpathlineto{\pgfqpoint{2.173318in}{4.765478in}}%
\pgfpathlineto{\pgfqpoint{2.160505in}{4.777470in}}%
\pgfpathlineto{\pgfqpoint{2.147494in}{4.786740in}}%
\pgfpathlineto{\pgfqpoint{2.134130in}{4.793395in}}%
\pgfpathlineto{\pgfqpoint{2.120253in}{4.797505in}}%
\pgfpathlineto{\pgfqpoint{2.105707in}{4.799101in}}%
\pgfpathlineto{\pgfqpoint{2.090337in}{4.798191in}}%
\pgfpathlineto{\pgfqpoint{2.073998in}{4.794768in}}%
\pgfpathlineto{\pgfqpoint{2.056564in}{4.788830in}}%
\pgfpathlineto{\pgfqpoint{2.037940in}{4.780394in}}%
\pgfpathlineto{\pgfqpoint{2.018081in}{4.769518in}}%
\pgfpathlineto{\pgfqpoint{1.997002in}{4.756322in}}%
\pgfpathlineto{\pgfqpoint{1.974788in}{4.740990in}}%
\pgfpathlineto{\pgfqpoint{1.951592in}{4.723773in}}%
\pgfpathlineto{\pgfqpoint{1.903114in}{4.684922in}}%
\pgfpathlineto{\pgfqpoint{1.853503in}{4.642362in}}%
\pgfpathlineto{\pgfqpoint{1.804610in}{4.598427in}}%
\pgfpathlineto{\pgfqpoint{1.757873in}{4.554843in}}%
\pgfpathlineto{\pgfqpoint{1.714309in}{4.512733in}}%
\pgfpathlineto{\pgfqpoint{1.674667in}{4.472770in}}%
\pgfpathlineto{\pgfqpoint{1.639637in}{4.435345in}}%
\pgfpathlineto{\pgfqpoint{1.610094in}{4.400705in}}%
\pgfpathlineto{\pgfqpoint{1.597791in}{4.384494in}}%
\pgfpathlineto{\pgfqpoint{1.587464in}{4.369059in}}%
\pgfpathlineto{\pgfqpoint{1.579471in}{4.354433in}}%
\pgfpathlineto{\pgfqpoint{1.574285in}{4.340655in}}%
\pgfpathlineto{\pgfqpoint{1.572495in}{4.327767in}}%
\pgfpathlineto{\pgfqpoint{1.574735in}{4.315806in}}%
\pgfpathlineto{\pgfqpoint{1.581491in}{4.304809in}}%
\pgfpathlineto{\pgfqpoint{1.592788in}{4.294824in}}%
\pgfpathlineto{\pgfqpoint{1.607977in}{4.285965in}}%
\pgfpathlineto{\pgfqpoint{1.625850in}{4.278460in}}%
\pgfpathlineto{\pgfqpoint{1.644977in}{4.272682in}}%
\pgfpathlineto{\pgfqpoint{1.663998in}{4.269133in}}%
\pgfpathlineto{\pgfqpoint{1.681808in}{4.268340in}}%
\pgfpathlineto{\pgfqpoint{1.697714in}{4.270641in}}%
\pgfpathlineto{\pgfqpoint{1.711533in}{4.275941in}}%
\pgfpathlineto{\pgfqpoint{1.723486in}{4.283730in}}%
\pgfpathlineto{\pgfqpoint{1.733950in}{4.293333in}}%
\pgfpathlineto{\pgfqpoint{1.743270in}{4.304143in}}%
\pgfpathlineto{\pgfqpoint{1.751707in}{4.315700in}}%
\pgfpathlineto{\pgfqpoint{1.766600in}{4.339849in}}%
\pgfpathlineto{\pgfqpoint{1.779500in}{4.364184in}}%
\pgfpathlineto{\pgfqpoint{1.795989in}{4.399400in}}%
\pgfpathlineto{\pgfqpoint{1.809562in}{4.431969in}}%
\pgfpathlineto{\pgfqpoint{1.820372in}{4.461158in}}%
\pgfpathlineto{\pgfqpoint{1.828223in}{4.486307in}}%
\pgfpathlineto{\pgfqpoint{1.831540in}{4.500397in}}%
\pgfpathlineto{\pgfqpoint{1.832983in}{4.511908in}}%
\pgfpathlineto{\pgfqpoint{1.832122in}{4.520321in}}%
\pgfpathlineto{\pgfqpoint{1.830662in}{4.523175in}}%
\pgfpathlineto{\pgfqpoint{1.828426in}{4.525024in}}%
\pgfpathlineto{\pgfqpoint{1.825356in}{4.525807in}}%
\pgfpathlineto{\pgfqpoint{1.821414in}{4.525484in}}%
\pgfpathlineto{\pgfqpoint{1.816592in}{4.524053in}}%
\pgfpathlineto{\pgfqpoint{1.804453in}{4.518055in}}%
\pgfpathlineto{\pgfqpoint{1.789513in}{4.508512in}}%
\pgfpathlineto{\pgfqpoint{1.763625in}{4.489619in}}%
\pgfpathlineto{\pgfqpoint{1.726233in}{4.459976in}}%
\pgfpathlineto{\pgfqpoint{1.679569in}{4.420877in}}%
\pgfpathlineto{\pgfqpoint{1.636296in}{4.382597in}}%
\pgfpathlineto{\pgfqpoint{1.613169in}{4.360708in}}%
\pgfpathlineto{\pgfqpoint{1.593471in}{4.339951in}}%
\pgfpathlineto{\pgfqpoint{1.583464in}{4.326907in}}%
\pgfpathlineto{\pgfqpoint{1.579970in}{4.320678in}}%
\pgfpathlineto{\pgfqpoint{1.577944in}{4.314673in}}%
\pgfpathlineto{\pgfqpoint{1.577883in}{4.308913in}}%
\pgfpathlineto{\pgfqpoint{1.580323in}{4.303401in}}%
\pgfpathlineto{\pgfqpoint{1.585572in}{4.298126in}}%
\pgfpathlineto{\pgfqpoint{1.593423in}{4.293085in}}%
\pgfpathlineto{\pgfqpoint{1.603221in}{4.288320in}}%
\pgfpathlineto{\pgfqpoint{1.625660in}{4.280100in}}%
\pgfpathlineto{\pgfqpoint{1.637025in}{4.277064in}}%
\pgfpathlineto{\pgfqpoint{1.647764in}{4.275174in}}%
\pgfpathlineto{\pgfqpoint{1.657423in}{4.274824in}}%
\pgfpathlineto{\pgfqpoint{1.665755in}{4.276263in}}%
\pgfpathlineto{\pgfqpoint{1.672837in}{4.279386in}}%
\pgfpathlineto{\pgfqpoint{1.678955in}{4.283810in}}%
\pgfpathlineto{\pgfqpoint{1.684397in}{4.289129in}}%
\pgfpathlineto{\pgfqpoint{1.694002in}{4.301310in}}%
\pgfpathlineto{\pgfqpoint{1.706576in}{4.321254in}}%
\pgfpathlineto{\pgfqpoint{1.721392in}{4.348432in}}%
\pgfpathlineto{\pgfqpoint{1.740845in}{4.387846in}}%
\pgfpathlineto{\pgfqpoint{1.760079in}{4.430013in}}%
\pgfpathlineto{\pgfqpoint{1.775374in}{4.466726in}}%
\pgfpathlineto{\pgfqpoint{1.782955in}{4.488269in}}%
\pgfpathlineto{\pgfqpoint{1.785349in}{4.498323in}}%
\pgfpathlineto{\pgfqpoint{1.785574in}{4.503270in}}%
\pgfpathlineto{\pgfqpoint{1.784294in}{4.506344in}}%
\pgfpathlineto{\pgfqpoint{1.782973in}{4.507045in}}%
\pgfpathlineto{\pgfqpoint{1.778803in}{4.506601in}}%
\pgfpathlineto{\pgfqpoint{1.772605in}{4.503725in}}%
\pgfpathlineto{\pgfqpoint{1.760195in}{4.495690in}}%
\pgfpathlineto{\pgfqpoint{1.740111in}{4.480676in}}%
\pgfpathlineto{\pgfqpoint{1.701532in}{4.449706in}}%
\pgfpathlineto{\pgfqpoint{1.646911in}{4.403942in}}%
\pgfpathlineto{\pgfqpoint{1.601932in}{4.364615in}}%
\pgfpathlineto{\pgfqpoint{1.575959in}{4.340051in}}%
\pgfpathlineto{\pgfqpoint{1.565441in}{4.328443in}}%
\pgfpathlineto{\pgfqpoint{1.560437in}{4.321048in}}%
\pgfpathlineto{\pgfqpoint{1.559050in}{4.317482in}}%
\pgfpathlineto{\pgfqpoint{1.558851in}{4.314017in}}%
\pgfpathlineto{\pgfqpoint{1.560280in}{4.310648in}}%
\pgfpathlineto{\pgfqpoint{1.563618in}{4.307356in}}%
\pgfpathlineto{\pgfqpoint{1.575091in}{4.300934in}}%
\pgfpathlineto{\pgfqpoint{1.589895in}{4.294838in}}%
\pgfpathlineto{\pgfqpoint{1.613125in}{4.287237in}}%
\pgfpathlineto{\pgfqpoint{1.626849in}{4.284726in}}%
\pgfpathlineto{\pgfqpoint{1.632491in}{4.284955in}}%
\pgfpathlineto{\pgfqpoint{1.637215in}{4.286340in}}%
\pgfpathlineto{\pgfqpoint{1.641210in}{4.288673in}}%
\pgfpathlineto{\pgfqpoint{1.647897in}{4.295064in}}%
\pgfpathlineto{\pgfqpoint{1.656394in}{4.306573in}}%
\pgfpathlineto{\pgfqpoint{1.668894in}{4.327227in}}%
\pgfpathlineto{\pgfqpoint{1.686827in}{4.360372in}}%
\pgfpathlineto{\pgfqpoint{1.712479in}{4.411112in}}%
\pgfpathlineto{\pgfqpoint{1.738721in}{4.466095in}}%
\pgfpathlineto{\pgfqpoint{1.750966in}{4.494460in}}%
\pgfpathlineto{\pgfqpoint{1.754536in}{4.505267in}}%
\pgfpathlineto{\pgfqpoint{1.754946in}{4.509593in}}%
\pgfpathlineto{\pgfqpoint{1.754049in}{4.511060in}}%
\pgfpathlineto{\pgfqpoint{1.751945in}{4.511088in}}%
\pgfpathlineto{\pgfqpoint{1.746423in}{4.508343in}}%
\pgfpathlineto{\pgfqpoint{1.735718in}{4.500773in}}%
\pgfpathlineto{\pgfqpoint{1.709370in}{4.479747in}}%
\pgfpathlineto{\pgfqpoint{1.653750in}{4.433099in}}%
\pgfpathlineto{\pgfqpoint{1.585964in}{4.374479in}}%
\pgfpathlineto{\pgfqpoint{1.554902in}{4.346125in}}%
\pgfpathlineto{\pgfqpoint{1.543067in}{4.333944in}}%
\pgfpathlineto{\pgfqpoint{1.538000in}{4.326956in}}%
\pgfpathlineto{\pgfqpoint{1.537039in}{4.322499in}}%
\pgfpathlineto{\pgfqpoint{1.538047in}{4.320336in}}%
\pgfpathlineto{\pgfqpoint{1.540358in}{4.318200in}}%
\pgfpathlineto{\pgfqpoint{1.548085in}{4.313953in}}%
\pgfpathlineto{\pgfqpoint{1.568463in}{4.305702in}}%
\pgfpathlineto{\pgfqpoint{1.589503in}{4.298633in}}%
\pgfpathlineto{\pgfqpoint{1.598863in}{4.296497in}}%
\pgfpathlineto{\pgfqpoint{1.606126in}{4.296786in}}%
\pgfpathlineto{\pgfqpoint{1.611246in}{4.299731in}}%
\pgfpathlineto{\pgfqpoint{1.617223in}{4.306370in}}%
\pgfpathlineto{\pgfqpoint{1.625784in}{4.319112in}}%
\pgfpathlineto{\pgfqpoint{1.641197in}{4.345571in}}%
\pgfpathlineto{\pgfqpoint{1.668776in}{4.396526in}}%
\pgfpathlineto{\pgfqpoint{1.706102in}{4.468670in}}%
\pgfpathlineto{\pgfqpoint{1.722947in}{4.503718in}}%
\pgfpathlineto{\pgfqpoint{1.726139in}{4.512741in}}%
\pgfpathlineto{\pgfqpoint{1.725973in}{4.514750in}}%
\pgfpathlineto{\pgfqpoint{1.725973in}{4.514750in}}%
\pgfusepath{stroke}%
\end{pgfscope}%
\begin{pgfscope}%
\pgfpathrectangle{\pgfqpoint{0.152333in}{3.075000in}}{\pgfqpoint{4.224218in}{2.565000in}}%
\pgfusepath{clip}%
\pgfsetbuttcap%
\pgfsetroundjoin%
\definecolor{currentfill}{rgb}{0.501961,0.000000,0.501961}%
\pgfsetfillcolor{currentfill}%
\pgfsetlinewidth{1.505625pt}%
\definecolor{currentstroke}{rgb}{0.501961,0.000000,0.501961}%
\pgfsetstrokecolor{currentstroke}%
\pgfsetdash{}{0pt}%
\pgfsys@defobject{currentmarker}{\pgfqpoint{-0.017010in}{-0.017010in}}{\pgfqpoint{0.017010in}{0.017010in}}{%
\pgfpathmoveto{\pgfqpoint{-0.017010in}{0.000000in}}%
\pgfpathlineto{\pgfqpoint{0.017010in}{0.000000in}}%
\pgfpathmoveto{\pgfqpoint{0.000000in}{-0.017010in}}%
\pgfpathlineto{\pgfqpoint{0.000000in}{0.017010in}}%
\pgfusepath{stroke,fill}%
}%
\begin{pgfscope}%
\pgfsys@transformshift{2.327865in}{4.157031in}%
\pgfsys@useobject{currentmarker}{}%
\end{pgfscope}%
\end{pgfscope}%
\begin{pgfscope}%
\pgfpathrectangle{\pgfqpoint{0.152333in}{3.075000in}}{\pgfqpoint{4.224218in}{2.565000in}}%
\pgfusepath{clip}%
\pgfsetbuttcap%
\pgfsetroundjoin%
\definecolor{currentfill}{rgb}{0.501961,0.000000,0.501961}%
\pgfsetfillcolor{currentfill}%
\pgfsetlinewidth{1.003750pt}%
\definecolor{currentstroke}{rgb}{0.501961,0.000000,0.501961}%
\pgfsetstrokecolor{currentstroke}%
\pgfsetdash{}{0pt}%
\pgfsys@defobject{currentmarker}{\pgfqpoint{-0.016178in}{-0.013762in}}{\pgfqpoint{0.016178in}{0.017010in}}{%
\pgfpathmoveto{\pgfqpoint{0.000000in}{0.017010in}}%
\pgfpathlineto{\pgfqpoint{-0.003819in}{0.005256in}}%
\pgfpathlineto{\pgfqpoint{-0.016178in}{0.005256in}}%
\pgfpathlineto{\pgfqpoint{-0.006179in}{-0.002008in}}%
\pgfpathlineto{\pgfqpoint{-0.009998in}{-0.013762in}}%
\pgfpathlineto{\pgfqpoint{-0.000000in}{-0.006497in}}%
\pgfpathlineto{\pgfqpoint{0.009998in}{-0.013762in}}%
\pgfpathlineto{\pgfqpoint{0.006179in}{-0.002008in}}%
\pgfpathlineto{\pgfqpoint{0.016178in}{0.005256in}}%
\pgfpathlineto{\pgfqpoint{0.003819in}{0.005256in}}%
\pgfpathlineto{\pgfqpoint{0.000000in}{0.017010in}}%
\pgfpathclose%
\pgfusepath{stroke,fill}%
}%
\begin{pgfscope}%
\pgfsys@transformshift{1.620171in}{4.755001in}%
\pgfsys@useobject{currentmarker}{}%
\end{pgfscope}%
\end{pgfscope}%
\begin{pgfscope}%
\pgfpathrectangle{\pgfqpoint{0.152333in}{3.075000in}}{\pgfqpoint{4.224218in}{2.565000in}}%
\pgfusepath{clip}%
\pgfsetbuttcap%
\pgfsetroundjoin%
\definecolor{currentfill}{rgb}{0.000000,0.000000,0.000000}%
\pgfsetfillcolor{currentfill}%
\pgfsetlinewidth{1.505625pt}%
\definecolor{currentstroke}{rgb}{0.000000,0.000000,0.000000}%
\pgfsetstrokecolor{currentstroke}%
\pgfsetdash{}{0pt}%
\pgfsys@defobject{currentmarker}{\pgfqpoint{-0.017010in}{-0.017010in}}{\pgfqpoint{0.017010in}{0.017010in}}{%
\pgfpathmoveto{\pgfqpoint{-0.017010in}{0.000000in}}%
\pgfpathlineto{\pgfqpoint{0.017010in}{0.000000in}}%
\pgfpathmoveto{\pgfqpoint{0.000000in}{-0.017010in}}%
\pgfpathlineto{\pgfqpoint{0.000000in}{0.017010in}}%
\pgfusepath{stroke,fill}%
}%
\begin{pgfscope}%
\pgfsys@transformshift{2.327865in}{4.157031in}%
\pgfsys@useobject{currentmarker}{}%
\end{pgfscope}%
\end{pgfscope}%
\begin{pgfscope}%
\pgfpathrectangle{\pgfqpoint{0.152333in}{3.075000in}}{\pgfqpoint{4.224218in}{2.565000in}}%
\pgfusepath{clip}%
\pgfsetbuttcap%
\pgfsetroundjoin%
\definecolor{currentfill}{rgb}{0.000000,0.000000,0.000000}%
\pgfsetfillcolor{currentfill}%
\pgfsetlinewidth{1.003750pt}%
\definecolor{currentstroke}{rgb}{0.000000,0.000000,0.000000}%
\pgfsetstrokecolor{currentstroke}%
\pgfsetdash{}{0pt}%
\pgfsys@defobject{currentmarker}{\pgfqpoint{-0.016178in}{-0.013762in}}{\pgfqpoint{0.016178in}{0.017010in}}{%
\pgfpathmoveto{\pgfqpoint{0.000000in}{0.017010in}}%
\pgfpathlineto{\pgfqpoint{-0.003819in}{0.005256in}}%
\pgfpathlineto{\pgfqpoint{-0.016178in}{0.005256in}}%
\pgfpathlineto{\pgfqpoint{-0.006179in}{-0.002008in}}%
\pgfpathlineto{\pgfqpoint{-0.009998in}{-0.013762in}}%
\pgfpathlineto{\pgfqpoint{-0.000000in}{-0.006497in}}%
\pgfpathlineto{\pgfqpoint{0.009998in}{-0.013762in}}%
\pgfpathlineto{\pgfqpoint{0.006179in}{-0.002008in}}%
\pgfpathlineto{\pgfqpoint{0.016178in}{0.005256in}}%
\pgfpathlineto{\pgfqpoint{0.003819in}{0.005256in}}%
\pgfpathlineto{\pgfqpoint{0.000000in}{0.017010in}}%
\pgfpathclose%
\pgfusepath{stroke,fill}%
}%
\begin{pgfscope}%
\pgfsys@transformshift{1.725973in}{4.514750in}%
\pgfsys@useobject{currentmarker}{}%
\end{pgfscope}%
\end{pgfscope}%
\begin{pgfscope}%
\pgfsetrectcap%
\pgfsetmiterjoin%
\pgfsetlinewidth{0.803000pt}%
\definecolor{currentstroke}{rgb}{0.000000,0.000000,0.000000}%
\pgfsetstrokecolor{currentstroke}%
\pgfsetdash{}{0pt}%
\pgfpathmoveto{\pgfqpoint{0.152333in}{3.075000in}}%
\pgfpathlineto{\pgfqpoint{0.152333in}{5.640000in}}%
\pgfusepath{stroke}%
\end{pgfscope}%
\begin{pgfscope}%
\pgfsetrectcap%
\pgfsetmiterjoin%
\pgfsetlinewidth{0.803000pt}%
\definecolor{currentstroke}{rgb}{0.000000,0.000000,0.000000}%
\pgfsetstrokecolor{currentstroke}%
\pgfsetdash{}{0pt}%
\pgfpathmoveto{\pgfqpoint{4.376551in}{3.075000in}}%
\pgfpathlineto{\pgfqpoint{4.376551in}{5.640000in}}%
\pgfusepath{stroke}%
\end{pgfscope}%
\begin{pgfscope}%
\pgfsetrectcap%
\pgfsetmiterjoin%
\pgfsetlinewidth{0.803000pt}%
\definecolor{currentstroke}{rgb}{0.000000,0.000000,0.000000}%
\pgfsetstrokecolor{currentstroke}%
\pgfsetdash{}{0pt}%
\pgfpathmoveto{\pgfqpoint{0.152333in}{3.075000in}}%
\pgfpathlineto{\pgfqpoint{4.376551in}{3.075000in}}%
\pgfusepath{stroke}%
\end{pgfscope}%
\begin{pgfscope}%
\pgfsetrectcap%
\pgfsetmiterjoin%
\pgfsetlinewidth{0.803000pt}%
\definecolor{currentstroke}{rgb}{0.000000,0.000000,0.000000}%
\pgfsetstrokecolor{currentstroke}%
\pgfsetdash{}{0pt}%
\pgfpathmoveto{\pgfqpoint{0.152333in}{5.640000in}}%
\pgfpathlineto{\pgfqpoint{4.376551in}{5.640000in}}%
\pgfusepath{stroke}%
\end{pgfscope}%
\begin{pgfscope}%
\definecolor{textcolor}{rgb}{0.000000,0.000000,0.000000}%
\pgfsetstrokecolor{textcolor}%
\pgfsetfillcolor{textcolor}%
\pgftext[x=4.293987in,y=3.320393in,,base]{\color{textcolor}\sffamily\fontsize{10.000000}{12.000000}\selectfont 0.0}%
\end{pgfscope}%
\begin{pgfscope}%
\definecolor{textcolor}{rgb}{0.000000,0.000000,0.000000}%
\pgfsetstrokecolor{textcolor}%
\pgfsetfillcolor{textcolor}%
\pgftext[x=4.101977in,y=3.542529in,,base]{\color{textcolor}\sffamily\fontsize{10.000000}{12.000000}\selectfont 0.1}%
\end{pgfscope}%
\begin{pgfscope}%
\definecolor{textcolor}{rgb}{0.000000,0.000000,0.000000}%
\pgfsetstrokecolor{textcolor}%
\pgfsetfillcolor{textcolor}%
\pgftext[x=3.909967in,y=3.764664in,,base]{\color{textcolor}\sffamily\fontsize{10.000000}{12.000000}\selectfont 0.2}%
\end{pgfscope}%
\begin{pgfscope}%
\definecolor{textcolor}{rgb}{0.000000,0.000000,0.000000}%
\pgfsetstrokecolor{textcolor}%
\pgfsetfillcolor{textcolor}%
\pgftext[x=3.717957in,y=3.986800in,,base]{\color{textcolor}\sffamily\fontsize{10.000000}{12.000000}\selectfont 0.3}%
\end{pgfscope}%
\begin{pgfscope}%
\definecolor{textcolor}{rgb}{0.000000,0.000000,0.000000}%
\pgfsetstrokecolor{textcolor}%
\pgfsetfillcolor{textcolor}%
\pgftext[x=3.525947in,y=4.208935in,,base]{\color{textcolor}\sffamily\fontsize{10.000000}{12.000000}\selectfont 0.4}%
\end{pgfscope}%
\begin{pgfscope}%
\definecolor{textcolor}{rgb}{0.000000,0.000000,0.000000}%
\pgfsetstrokecolor{textcolor}%
\pgfsetfillcolor{textcolor}%
\pgftext[x=3.333937in,y=4.431071in,,base]{\color{textcolor}\sffamily\fontsize{10.000000}{12.000000}\selectfont 0.5}%
\end{pgfscope}%
\begin{pgfscope}%
\definecolor{textcolor}{rgb}{0.000000,0.000000,0.000000}%
\pgfsetstrokecolor{textcolor}%
\pgfsetfillcolor{textcolor}%
\pgftext[x=3.141927in,y=4.653206in,,base]{\color{textcolor}\sffamily\fontsize{10.000000}{12.000000}\selectfont 0.6}%
\end{pgfscope}%
\begin{pgfscope}%
\definecolor{textcolor}{rgb}{0.000000,0.000000,0.000000}%
\pgfsetstrokecolor{textcolor}%
\pgfsetfillcolor{textcolor}%
\pgftext[x=2.949918in,y=4.875342in,,base]{\color{textcolor}\sffamily\fontsize{10.000000}{12.000000}\selectfont 0.7}%
\end{pgfscope}%
\begin{pgfscope}%
\definecolor{textcolor}{rgb}{0.000000,0.000000,0.000000}%
\pgfsetstrokecolor{textcolor}%
\pgfsetfillcolor{textcolor}%
\pgftext[x=2.757908in,y=5.097477in,,base]{\color{textcolor}\sffamily\fontsize{10.000000}{12.000000}\selectfont 0.8}%
\end{pgfscope}%
\begin{pgfscope}%
\definecolor{textcolor}{rgb}{0.000000,0.000000,0.000000}%
\pgfsetstrokecolor{textcolor}%
\pgfsetfillcolor{textcolor}%
\pgftext[x=2.565898in,y=5.319613in,,base]{\color{textcolor}\sffamily\fontsize{10.000000}{12.000000}\selectfont 0.9}%
\end{pgfscope}%
\begin{pgfscope}%
\definecolor{textcolor}{rgb}{0.000000,0.000000,0.000000}%
\pgfsetstrokecolor{textcolor}%
\pgfsetfillcolor{textcolor}%
\pgftext[x=2.373888in,y=5.541748in,,base]{\color{textcolor}\sffamily\fontsize{10.000000}{12.000000}\selectfont 1.0}%
\end{pgfscope}%
\begin{pgfscope}%
\definecolor{textcolor}{rgb}{0.000000,0.000000,0.000000}%
\pgfsetstrokecolor{textcolor}%
\pgfsetfillcolor{textcolor}%
\pgftext[x=0.296340in,y=3.364820in,,base]{\color{textcolor}\sffamily\fontsize{10.000000}{12.000000}\selectfont 1.0}%
\end{pgfscope}%
\begin{pgfscope}%
\definecolor{textcolor}{rgb}{0.000000,0.000000,0.000000}%
\pgfsetstrokecolor{textcolor}%
\pgfsetfillcolor{textcolor}%
\pgftext[x=0.488350in,y=3.586956in,,base]{\color{textcolor}\sffamily\fontsize{10.000000}{12.000000}\selectfont 0.9}%
\end{pgfscope}%
\begin{pgfscope}%
\definecolor{textcolor}{rgb}{0.000000,0.000000,0.000000}%
\pgfsetstrokecolor{textcolor}%
\pgfsetfillcolor{textcolor}%
\pgftext[x=0.680360in,y=3.809091in,,base]{\color{textcolor}\sffamily\fontsize{10.000000}{12.000000}\selectfont 0.8}%
\end{pgfscope}%
\begin{pgfscope}%
\definecolor{textcolor}{rgb}{0.000000,0.000000,0.000000}%
\pgfsetstrokecolor{textcolor}%
\pgfsetfillcolor{textcolor}%
\pgftext[x=0.872370in,y=4.031227in,,base]{\color{textcolor}\sffamily\fontsize{10.000000}{12.000000}\selectfont 0.7}%
\end{pgfscope}%
\begin{pgfscope}%
\definecolor{textcolor}{rgb}{0.000000,0.000000,0.000000}%
\pgfsetstrokecolor{textcolor}%
\pgfsetfillcolor{textcolor}%
\pgftext[x=1.064380in,y=4.253362in,,base]{\color{textcolor}\sffamily\fontsize{10.000000}{12.000000}\selectfont 0.6}%
\end{pgfscope}%
\begin{pgfscope}%
\definecolor{textcolor}{rgb}{0.000000,0.000000,0.000000}%
\pgfsetstrokecolor{textcolor}%
\pgfsetfillcolor{textcolor}%
\pgftext[x=1.256390in,y=4.475498in,,base]{\color{textcolor}\sffamily\fontsize{10.000000}{12.000000}\selectfont 0.5}%
\end{pgfscope}%
\begin{pgfscope}%
\definecolor{textcolor}{rgb}{0.000000,0.000000,0.000000}%
\pgfsetstrokecolor{textcolor}%
\pgfsetfillcolor{textcolor}%
\pgftext[x=1.448400in,y=4.697633in,,base]{\color{textcolor}\sffamily\fontsize{10.000000}{12.000000}\selectfont 0.4}%
\end{pgfscope}%
\begin{pgfscope}%
\definecolor{textcolor}{rgb}{0.000000,0.000000,0.000000}%
\pgfsetstrokecolor{textcolor}%
\pgfsetfillcolor{textcolor}%
\pgftext[x=1.640410in,y=4.919769in,,base]{\color{textcolor}\sffamily\fontsize{10.000000}{12.000000}\selectfont 0.3}%
\end{pgfscope}%
\begin{pgfscope}%
\definecolor{textcolor}{rgb}{0.000000,0.000000,0.000000}%
\pgfsetstrokecolor{textcolor}%
\pgfsetfillcolor{textcolor}%
\pgftext[x=1.832420in,y=5.141904in,,base]{\color{textcolor}\sffamily\fontsize{10.000000}{12.000000}\selectfont 0.2}%
\end{pgfscope}%
\begin{pgfscope}%
\definecolor{textcolor}{rgb}{0.000000,0.000000,0.000000}%
\pgfsetstrokecolor{textcolor}%
\pgfsetfillcolor{textcolor}%
\pgftext[x=2.024430in,y=5.364040in,,base]{\color{textcolor}\sffamily\fontsize{10.000000}{12.000000}\selectfont 0.1}%
\end{pgfscope}%
\begin{pgfscope}%
\definecolor{textcolor}{rgb}{0.000000,0.000000,0.000000}%
\pgfsetstrokecolor{textcolor}%
\pgfsetfillcolor{textcolor}%
\pgftext[x=2.216440in,y=5.586175in,,base]{\color{textcolor}\sffamily\fontsize{10.000000}{12.000000}\selectfont 0.0}%
\end{pgfscope}%
\begin{pgfscope}%
\definecolor{textcolor}{rgb}{0.000000,0.000000,0.000000}%
\pgfsetstrokecolor{textcolor}%
\pgfsetfillcolor{textcolor}%
\pgftext[x=0.296340in,y=3.253753in,,base]{\color{textcolor}\sffamily\fontsize{10.000000}{12.000000}\selectfont 0.0}%
\end{pgfscope}%
\begin{pgfscope}%
\definecolor{textcolor}{rgb}{0.000000,0.000000,0.000000}%
\pgfsetstrokecolor{textcolor}%
\pgfsetfillcolor{textcolor}%
\pgftext[x=0.680360in,y=3.253753in,,base]{\color{textcolor}\sffamily\fontsize{10.000000}{12.000000}\selectfont 0.1}%
\end{pgfscope}%
\begin{pgfscope}%
\definecolor{textcolor}{rgb}{0.000000,0.000000,0.000000}%
\pgfsetstrokecolor{textcolor}%
\pgfsetfillcolor{textcolor}%
\pgftext[x=1.064380in,y=3.253753in,,base]{\color{textcolor}\sffamily\fontsize{10.000000}{12.000000}\selectfont 0.2}%
\end{pgfscope}%
\begin{pgfscope}%
\definecolor{textcolor}{rgb}{0.000000,0.000000,0.000000}%
\pgfsetstrokecolor{textcolor}%
\pgfsetfillcolor{textcolor}%
\pgftext[x=1.448400in,y=3.253753in,,base]{\color{textcolor}\sffamily\fontsize{10.000000}{12.000000}\selectfont 0.3}%
\end{pgfscope}%
\begin{pgfscope}%
\definecolor{textcolor}{rgb}{0.000000,0.000000,0.000000}%
\pgfsetstrokecolor{textcolor}%
\pgfsetfillcolor{textcolor}%
\pgftext[x=1.832420in,y=3.253753in,,base]{\color{textcolor}\sffamily\fontsize{10.000000}{12.000000}\selectfont 0.4}%
\end{pgfscope}%
\begin{pgfscope}%
\definecolor{textcolor}{rgb}{0.000000,0.000000,0.000000}%
\pgfsetstrokecolor{textcolor}%
\pgfsetfillcolor{textcolor}%
\pgftext[x=2.216440in,y=3.253753in,,base]{\color{textcolor}\sffamily\fontsize{10.000000}{12.000000}\selectfont 0.5}%
\end{pgfscope}%
\begin{pgfscope}%
\definecolor{textcolor}{rgb}{0.000000,0.000000,0.000000}%
\pgfsetstrokecolor{textcolor}%
\pgfsetfillcolor{textcolor}%
\pgftext[x=2.600460in,y=3.253753in,,base]{\color{textcolor}\sffamily\fontsize{10.000000}{12.000000}\selectfont 0.6}%
\end{pgfscope}%
\begin{pgfscope}%
\definecolor{textcolor}{rgb}{0.000000,0.000000,0.000000}%
\pgfsetstrokecolor{textcolor}%
\pgfsetfillcolor{textcolor}%
\pgftext[x=2.984479in,y=3.253753in,,base]{\color{textcolor}\sffamily\fontsize{10.000000}{12.000000}\selectfont 0.7}%
\end{pgfscope}%
\begin{pgfscope}%
\definecolor{textcolor}{rgb}{0.000000,0.000000,0.000000}%
\pgfsetstrokecolor{textcolor}%
\pgfsetfillcolor{textcolor}%
\pgftext[x=3.368499in,y=3.253753in,,base]{\color{textcolor}\sffamily\fontsize{10.000000}{12.000000}\selectfont 0.8}%
\end{pgfscope}%
\begin{pgfscope}%
\definecolor{textcolor}{rgb}{0.000000,0.000000,0.000000}%
\pgfsetstrokecolor{textcolor}%
\pgfsetfillcolor{textcolor}%
\pgftext[x=3.752519in,y=3.253753in,,base]{\color{textcolor}\sffamily\fontsize{10.000000}{12.000000}\selectfont 0.9}%
\end{pgfscope}%
\begin{pgfscope}%
\definecolor{textcolor}{rgb}{0.000000,0.000000,0.000000}%
\pgfsetstrokecolor{textcolor}%
\pgfsetfillcolor{textcolor}%
\pgftext[x=4.136539in,y=3.253753in,,base]{\color{textcolor}\sffamily\fontsize{10.000000}{12.000000}\selectfont 1.0}%
\end{pgfscope}%
\begin{pgfscope}%
\pgfpathrectangle{\pgfqpoint{0.152333in}{3.075000in}}{\pgfqpoint{4.224218in}{2.565000in}}%
\pgfusepath{clip}%
\pgfsetbuttcap%
\pgfsetroundjoin%
\definecolor{currentfill}{rgb}{1.000000,0.000000,0.000000}%
\pgfsetfillcolor{currentfill}%
\pgfsetlinewidth{1.505625pt}%
\definecolor{currentstroke}{rgb}{1.000000,0.000000,0.000000}%
\pgfsetstrokecolor{currentstroke}%
\pgfsetdash{}{0pt}%
\pgfsys@defobject{currentmarker}{\pgfqpoint{-0.013608in}{-0.017010in}}{\pgfqpoint{0.013608in}{0.008505in}}{%
\pgfpathmoveto{\pgfqpoint{0.000000in}{0.000000in}}%
\pgfpathlineto{\pgfqpoint{0.000000in}{-0.017010in}}%
\pgfpathmoveto{\pgfqpoint{0.000000in}{0.000000in}}%
\pgfpathlineto{\pgfqpoint{0.013608in}{0.008505in}}%
\pgfpathmoveto{\pgfqpoint{0.000000in}{0.000000in}}%
\pgfpathlineto{\pgfqpoint{-0.013608in}{0.008505in}}%
\pgfusepath{stroke,fill}%
}%
\begin{pgfscope}%
\pgfsys@transformshift{1.715842in}{4.283509in}%
\pgfsys@useobject{currentmarker}{}%
\end{pgfscope}%
\end{pgfscope}%
\begin{pgfscope}%
\definecolor{textcolor}{rgb}{0.000000,0.000000,0.000000}%
\pgfsetstrokecolor{textcolor}%
\pgfsetfillcolor{textcolor}%
\pgftext[x=2.264442in,y=5.723333in,,base]{\color{textcolor}\sffamily\fontsize{12.000000}{14.400000}\selectfont PG - Simultaneous}%
\end{pgfscope}%
\begin{pgfscope}%
\pgfsetbuttcap%
\pgfsetmiterjoin%
\definecolor{currentfill}{rgb}{1.000000,1.000000,1.000000}%
\pgfsetfillcolor{currentfill}%
\pgfsetlinewidth{0.000000pt}%
\definecolor{currentstroke}{rgb}{0.000000,0.000000,0.000000}%
\pgfsetstrokecolor{currentstroke}%
\pgfsetstrokeopacity{0.000000}%
\pgfsetdash{}{0pt}%
\pgfpathmoveto{\pgfqpoint{4.577333in}{3.075000in}}%
\pgfpathlineto{\pgfqpoint{8.801551in}{3.075000in}}%
\pgfpathlineto{\pgfqpoint{8.801551in}{5.640000in}}%
\pgfpathlineto{\pgfqpoint{4.577333in}{5.640000in}}%
\pgfpathlineto{\pgfqpoint{4.577333in}{3.075000in}}%
\pgfpathclose%
\pgfusepath{fill}%
\end{pgfscope}%
\begin{pgfscope}%
\pgfpathrectangle{\pgfqpoint{4.577333in}{3.075000in}}{\pgfqpoint{4.224218in}{2.565000in}}%
\pgfusepath{clip}%
\pgfsetbuttcap%
\pgfsetmiterjoin%
\definecolor{currentfill}{rgb}{0.960784,0.960784,0.960784}%
\pgfsetfillcolor{currentfill}%
\pgfsetfillopacity{0.750000}%
\pgfsetlinewidth{1.003750pt}%
\definecolor{currentstroke}{rgb}{0.960784,0.960784,0.960784}%
\pgfsetstrokecolor{currentstroke}%
\pgfsetstrokeopacity{0.750000}%
\pgfsetdash{}{0pt}%
\pgfpathmoveto{\pgfqpoint{8.609541in}{3.331500in}}%
\pgfpathlineto{\pgfqpoint{6.689442in}{5.552855in}}%
\pgfpathlineto{\pgfqpoint{4.769343in}{3.331500in}}%
\pgfpathlineto{\pgfqpoint{8.609541in}{3.331500in}}%
\pgfpathclose%
\pgfusepath{stroke,fill}%
\end{pgfscope}%
\begin{pgfscope}%
\pgfpathrectangle{\pgfqpoint{4.577333in}{3.075000in}}{\pgfqpoint{4.224218in}{2.565000in}}%
\pgfusepath{clip}%
\pgfsetrectcap%
\pgfsetroundjoin%
\pgfsetlinewidth{0.803000pt}%
\definecolor{currentstroke}{rgb}{0.000000,0.000000,0.000000}%
\pgfsetstrokecolor{currentstroke}%
\pgfsetdash{}{0pt}%
\pgfpathmoveto{\pgfqpoint{4.769343in}{3.331500in}}%
\pgfpathlineto{\pgfqpoint{8.609541in}{3.331500in}}%
\pgfusepath{stroke}%
\end{pgfscope}%
\begin{pgfscope}%
\pgfpathrectangle{\pgfqpoint{4.577333in}{3.075000in}}{\pgfqpoint{4.224218in}{2.565000in}}%
\pgfusepath{clip}%
\pgfsetrectcap%
\pgfsetroundjoin%
\pgfsetlinewidth{0.803000pt}%
\definecolor{currentstroke}{rgb}{0.000000,0.000000,0.000000}%
\pgfsetstrokecolor{currentstroke}%
\pgfsetdash{}{0pt}%
\pgfpathmoveto{\pgfqpoint{6.689442in}{5.552855in}}%
\pgfpathlineto{\pgfqpoint{4.769343in}{3.331500in}}%
\pgfusepath{stroke}%
\end{pgfscope}%
\begin{pgfscope}%
\pgfpathrectangle{\pgfqpoint{4.577333in}{3.075000in}}{\pgfqpoint{4.224218in}{2.565000in}}%
\pgfusepath{clip}%
\pgfsetrectcap%
\pgfsetroundjoin%
\pgfsetlinewidth{0.803000pt}%
\definecolor{currentstroke}{rgb}{0.000000,0.000000,0.000000}%
\pgfsetstrokecolor{currentstroke}%
\pgfsetdash{}{0pt}%
\pgfpathmoveto{\pgfqpoint{6.689442in}{5.552855in}}%
\pgfpathlineto{\pgfqpoint{8.609541in}{3.331500in}}%
\pgfusepath{stroke}%
\end{pgfscope}%
\begin{pgfscope}%
\pgfpathrectangle{\pgfqpoint{4.577333in}{3.075000in}}{\pgfqpoint{4.224218in}{2.565000in}}%
\pgfusepath{clip}%
\pgfsetbuttcap%
\pgfsetroundjoin%
\pgfsetlinewidth{0.501875pt}%
\definecolor{currentstroke}{rgb}{0.000000,0.000000,0.000000}%
\pgfsetstrokecolor{currentstroke}%
\pgfsetdash{{0.500000pt}{0.825000pt}}{0.000000pt}%
\pgfpathmoveto{\pgfqpoint{4.769343in}{3.331500in}}%
\pgfpathlineto{\pgfqpoint{8.609541in}{3.331500in}}%
\pgfusepath{stroke}%
\end{pgfscope}%
\begin{pgfscope}%
\pgfpathrectangle{\pgfqpoint{4.577333in}{3.075000in}}{\pgfqpoint{4.224218in}{2.565000in}}%
\pgfusepath{clip}%
\pgfsetbuttcap%
\pgfsetroundjoin%
\pgfsetlinewidth{0.501875pt}%
\definecolor{currentstroke}{rgb}{0.000000,0.000000,0.000000}%
\pgfsetstrokecolor{currentstroke}%
\pgfsetdash{{0.500000pt}{0.825000pt}}{0.000000pt}%
\pgfpathmoveto{\pgfqpoint{5.729393in}{4.442178in}}%
\pgfpathlineto{\pgfqpoint{7.649492in}{4.442178in}}%
\pgfusepath{stroke}%
\end{pgfscope}%
\begin{pgfscope}%
\pgfpathrectangle{\pgfqpoint{4.577333in}{3.075000in}}{\pgfqpoint{4.224218in}{2.565000in}}%
\pgfusepath{clip}%
\pgfsetbuttcap%
\pgfsetroundjoin%
\pgfsetlinewidth{0.501875pt}%
\definecolor{currentstroke}{rgb}{0.000000,0.000000,0.000000}%
\pgfsetstrokecolor{currentstroke}%
\pgfsetdash{{0.500000pt}{0.825000pt}}{0.000000pt}%
\pgfpathmoveto{\pgfqpoint{6.689442in}{5.552855in}}%
\pgfpathlineto{\pgfqpoint{4.769343in}{3.331500in}}%
\pgfusepath{stroke}%
\end{pgfscope}%
\begin{pgfscope}%
\pgfpathrectangle{\pgfqpoint{4.577333in}{3.075000in}}{\pgfqpoint{4.224218in}{2.565000in}}%
\pgfusepath{clip}%
\pgfsetbuttcap%
\pgfsetroundjoin%
\pgfsetlinewidth{0.501875pt}%
\definecolor{currentstroke}{rgb}{0.000000,0.000000,0.000000}%
\pgfsetstrokecolor{currentstroke}%
\pgfsetdash{{0.500000pt}{0.825000pt}}{0.000000pt}%
\pgfpathmoveto{\pgfqpoint{6.689442in}{5.552855in}}%
\pgfpathlineto{\pgfqpoint{8.609541in}{3.331500in}}%
\pgfusepath{stroke}%
\end{pgfscope}%
\begin{pgfscope}%
\pgfpathrectangle{\pgfqpoint{4.577333in}{3.075000in}}{\pgfqpoint{4.224218in}{2.565000in}}%
\pgfusepath{clip}%
\pgfsetbuttcap%
\pgfsetroundjoin%
\pgfsetlinewidth{0.501875pt}%
\definecolor{currentstroke}{rgb}{0.000000,0.000000,0.000000}%
\pgfsetstrokecolor{currentstroke}%
\pgfsetdash{{0.500000pt}{0.825000pt}}{0.000000pt}%
\pgfpathmoveto{\pgfqpoint{7.649492in}{4.442178in}}%
\pgfpathlineto{\pgfqpoint{6.689442in}{3.331500in}}%
\pgfusepath{stroke}%
\end{pgfscope}%
\begin{pgfscope}%
\pgfpathrectangle{\pgfqpoint{4.577333in}{3.075000in}}{\pgfqpoint{4.224218in}{2.565000in}}%
\pgfusepath{clip}%
\pgfsetbuttcap%
\pgfsetroundjoin%
\pgfsetlinewidth{0.501875pt}%
\definecolor{currentstroke}{rgb}{0.000000,0.000000,0.000000}%
\pgfsetstrokecolor{currentstroke}%
\pgfsetdash{{0.500000pt}{0.825000pt}}{0.000000pt}%
\pgfpathmoveto{\pgfqpoint{5.729393in}{4.442178in}}%
\pgfpathlineto{\pgfqpoint{6.689442in}{3.331500in}}%
\pgfusepath{stroke}%
\end{pgfscope}%
\begin{pgfscope}%
\pgfpathrectangle{\pgfqpoint{4.577333in}{3.075000in}}{\pgfqpoint{4.224218in}{2.565000in}}%
\pgfusepath{clip}%
\pgfsetbuttcap%
\pgfsetroundjoin%
\pgfsetlinewidth{0.501875pt}%
\definecolor{currentstroke}{rgb}{0.000000,0.000000,0.000000}%
\pgfsetstrokecolor{currentstroke}%
\pgfsetdash{{0.500000pt}{0.825000pt}}{0.000000pt}%
\pgfpathmoveto{\pgfqpoint{8.609541in}{3.331500in}}%
\pgfpathlineto{\pgfqpoint{8.609541in}{3.331500in}}%
\pgfusepath{stroke}%
\end{pgfscope}%
\begin{pgfscope}%
\pgfpathrectangle{\pgfqpoint{4.577333in}{3.075000in}}{\pgfqpoint{4.224218in}{2.565000in}}%
\pgfusepath{clip}%
\pgfsetbuttcap%
\pgfsetroundjoin%
\pgfsetlinewidth{0.501875pt}%
\definecolor{currentstroke}{rgb}{0.000000,0.000000,0.000000}%
\pgfsetstrokecolor{currentstroke}%
\pgfsetdash{{0.500000pt}{0.825000pt}}{0.000000pt}%
\pgfpathmoveto{\pgfqpoint{4.769343in}{3.331500in}}%
\pgfpathlineto{\pgfqpoint{4.769343in}{3.331500in}}%
\pgfusepath{stroke}%
\end{pgfscope}%
\begin{pgfscope}%
\pgfpathrectangle{\pgfqpoint{4.577333in}{3.075000in}}{\pgfqpoint{4.224218in}{2.565000in}}%
\pgfusepath{clip}%
\pgfsetbuttcap%
\pgfsetroundjoin%
\pgfsetlinewidth{0.501875pt}%
\definecolor{currentstroke}{rgb}{0.000000,0.000000,1.000000}%
\pgfsetstrokecolor{currentstroke}%
\pgfsetdash{{0.500000pt}{0.825000pt}}{0.000000pt}%
\pgfpathmoveto{\pgfqpoint{4.769343in}{3.331500in}}%
\pgfpathlineto{\pgfqpoint{8.609541in}{3.331500in}}%
\pgfusepath{stroke}%
\end{pgfscope}%
\begin{pgfscope}%
\pgfpathrectangle{\pgfqpoint{4.577333in}{3.075000in}}{\pgfqpoint{4.224218in}{2.565000in}}%
\pgfusepath{clip}%
\pgfsetbuttcap%
\pgfsetroundjoin%
\pgfsetlinewidth{0.501875pt}%
\definecolor{currentstroke}{rgb}{0.000000,0.000000,1.000000}%
\pgfsetstrokecolor{currentstroke}%
\pgfsetdash{{0.500000pt}{0.825000pt}}{0.000000pt}%
\pgfpathmoveto{\pgfqpoint{4.961353in}{3.553636in}}%
\pgfpathlineto{\pgfqpoint{8.417531in}{3.553636in}}%
\pgfusepath{stroke}%
\end{pgfscope}%
\begin{pgfscope}%
\pgfpathrectangle{\pgfqpoint{4.577333in}{3.075000in}}{\pgfqpoint{4.224218in}{2.565000in}}%
\pgfusepath{clip}%
\pgfsetbuttcap%
\pgfsetroundjoin%
\pgfsetlinewidth{0.501875pt}%
\definecolor{currentstroke}{rgb}{0.000000,0.000000,1.000000}%
\pgfsetstrokecolor{currentstroke}%
\pgfsetdash{{0.500000pt}{0.825000pt}}{0.000000pt}%
\pgfpathmoveto{\pgfqpoint{5.153363in}{3.775771in}}%
\pgfpathlineto{\pgfqpoint{8.225522in}{3.775771in}}%
\pgfusepath{stroke}%
\end{pgfscope}%
\begin{pgfscope}%
\pgfpathrectangle{\pgfqpoint{4.577333in}{3.075000in}}{\pgfqpoint{4.224218in}{2.565000in}}%
\pgfusepath{clip}%
\pgfsetbuttcap%
\pgfsetroundjoin%
\pgfsetlinewidth{0.501875pt}%
\definecolor{currentstroke}{rgb}{0.000000,0.000000,1.000000}%
\pgfsetstrokecolor{currentstroke}%
\pgfsetdash{{0.500000pt}{0.825000pt}}{0.000000pt}%
\pgfpathmoveto{\pgfqpoint{5.345373in}{3.997907in}}%
\pgfpathlineto{\pgfqpoint{8.033512in}{3.997907in}}%
\pgfusepath{stroke}%
\end{pgfscope}%
\begin{pgfscope}%
\pgfpathrectangle{\pgfqpoint{4.577333in}{3.075000in}}{\pgfqpoint{4.224218in}{2.565000in}}%
\pgfusepath{clip}%
\pgfsetbuttcap%
\pgfsetroundjoin%
\pgfsetlinewidth{0.501875pt}%
\definecolor{currentstroke}{rgb}{0.000000,0.000000,1.000000}%
\pgfsetstrokecolor{currentstroke}%
\pgfsetdash{{0.500000pt}{0.825000pt}}{0.000000pt}%
\pgfpathmoveto{\pgfqpoint{5.537383in}{4.220042in}}%
\pgfpathlineto{\pgfqpoint{7.841502in}{4.220042in}}%
\pgfusepath{stroke}%
\end{pgfscope}%
\begin{pgfscope}%
\pgfpathrectangle{\pgfqpoint{4.577333in}{3.075000in}}{\pgfqpoint{4.224218in}{2.565000in}}%
\pgfusepath{clip}%
\pgfsetbuttcap%
\pgfsetroundjoin%
\pgfsetlinewidth{0.501875pt}%
\definecolor{currentstroke}{rgb}{0.000000,0.000000,1.000000}%
\pgfsetstrokecolor{currentstroke}%
\pgfsetdash{{0.500000pt}{0.825000pt}}{0.000000pt}%
\pgfpathmoveto{\pgfqpoint{5.729393in}{4.442178in}}%
\pgfpathlineto{\pgfqpoint{7.649492in}{4.442178in}}%
\pgfusepath{stroke}%
\end{pgfscope}%
\begin{pgfscope}%
\pgfpathrectangle{\pgfqpoint{4.577333in}{3.075000in}}{\pgfqpoint{4.224218in}{2.565000in}}%
\pgfusepath{clip}%
\pgfsetbuttcap%
\pgfsetroundjoin%
\pgfsetlinewidth{0.501875pt}%
\definecolor{currentstroke}{rgb}{0.000000,0.000000,1.000000}%
\pgfsetstrokecolor{currentstroke}%
\pgfsetdash{{0.500000pt}{0.825000pt}}{0.000000pt}%
\pgfpathmoveto{\pgfqpoint{5.921402in}{4.664313in}}%
\pgfpathlineto{\pgfqpoint{7.457482in}{4.664313in}}%
\pgfusepath{stroke}%
\end{pgfscope}%
\begin{pgfscope}%
\pgfpathrectangle{\pgfqpoint{4.577333in}{3.075000in}}{\pgfqpoint{4.224218in}{2.565000in}}%
\pgfusepath{clip}%
\pgfsetbuttcap%
\pgfsetroundjoin%
\pgfsetlinewidth{0.501875pt}%
\definecolor{currentstroke}{rgb}{0.000000,0.000000,1.000000}%
\pgfsetstrokecolor{currentstroke}%
\pgfsetdash{{0.500000pt}{0.825000pt}}{0.000000pt}%
\pgfpathmoveto{\pgfqpoint{6.113412in}{4.886449in}}%
\pgfpathlineto{\pgfqpoint{7.265472in}{4.886449in}}%
\pgfusepath{stroke}%
\end{pgfscope}%
\begin{pgfscope}%
\pgfpathrectangle{\pgfqpoint{4.577333in}{3.075000in}}{\pgfqpoint{4.224218in}{2.565000in}}%
\pgfusepath{clip}%
\pgfsetbuttcap%
\pgfsetroundjoin%
\pgfsetlinewidth{0.501875pt}%
\definecolor{currentstroke}{rgb}{0.000000,0.000000,1.000000}%
\pgfsetstrokecolor{currentstroke}%
\pgfsetdash{{0.500000pt}{0.825000pt}}{0.000000pt}%
\pgfpathmoveto{\pgfqpoint{6.305422in}{5.108584in}}%
\pgfpathlineto{\pgfqpoint{7.073462in}{5.108584in}}%
\pgfusepath{stroke}%
\end{pgfscope}%
\begin{pgfscope}%
\pgfpathrectangle{\pgfqpoint{4.577333in}{3.075000in}}{\pgfqpoint{4.224218in}{2.565000in}}%
\pgfusepath{clip}%
\pgfsetbuttcap%
\pgfsetroundjoin%
\pgfsetlinewidth{0.501875pt}%
\definecolor{currentstroke}{rgb}{0.000000,0.000000,1.000000}%
\pgfsetstrokecolor{currentstroke}%
\pgfsetdash{{0.500000pt}{0.825000pt}}{0.000000pt}%
\pgfpathmoveto{\pgfqpoint{6.497432in}{5.330720in}}%
\pgfpathlineto{\pgfqpoint{6.881452in}{5.330720in}}%
\pgfusepath{stroke}%
\end{pgfscope}%
\begin{pgfscope}%
\pgfpathrectangle{\pgfqpoint{4.577333in}{3.075000in}}{\pgfqpoint{4.224218in}{2.565000in}}%
\pgfusepath{clip}%
\pgfsetbuttcap%
\pgfsetroundjoin%
\pgfsetlinewidth{0.501875pt}%
\definecolor{currentstroke}{rgb}{0.000000,0.000000,1.000000}%
\pgfsetstrokecolor{currentstroke}%
\pgfsetdash{{0.500000pt}{0.825000pt}}{0.000000pt}%
\pgfpathmoveto{\pgfqpoint{6.689442in}{5.552855in}}%
\pgfpathlineto{\pgfqpoint{4.769343in}{3.331500in}}%
\pgfusepath{stroke}%
\end{pgfscope}%
\begin{pgfscope}%
\pgfpathrectangle{\pgfqpoint{4.577333in}{3.075000in}}{\pgfqpoint{4.224218in}{2.565000in}}%
\pgfusepath{clip}%
\pgfsetbuttcap%
\pgfsetroundjoin%
\pgfsetlinewidth{0.501875pt}%
\definecolor{currentstroke}{rgb}{0.000000,0.000000,1.000000}%
\pgfsetstrokecolor{currentstroke}%
\pgfsetdash{{0.500000pt}{0.825000pt}}{0.000000pt}%
\pgfpathmoveto{\pgfqpoint{6.689442in}{5.552855in}}%
\pgfpathlineto{\pgfqpoint{8.609541in}{3.331500in}}%
\pgfusepath{stroke}%
\end{pgfscope}%
\begin{pgfscope}%
\pgfpathrectangle{\pgfqpoint{4.577333in}{3.075000in}}{\pgfqpoint{4.224218in}{2.565000in}}%
\pgfusepath{clip}%
\pgfsetbuttcap%
\pgfsetroundjoin%
\pgfsetlinewidth{0.501875pt}%
\definecolor{currentstroke}{rgb}{0.000000,0.000000,1.000000}%
\pgfsetstrokecolor{currentstroke}%
\pgfsetdash{{0.500000pt}{0.825000pt}}{0.000000pt}%
\pgfpathmoveto{\pgfqpoint{6.881452in}{5.330720in}}%
\pgfpathlineto{\pgfqpoint{5.153363in}{3.331500in}}%
\pgfusepath{stroke}%
\end{pgfscope}%
\begin{pgfscope}%
\pgfpathrectangle{\pgfqpoint{4.577333in}{3.075000in}}{\pgfqpoint{4.224218in}{2.565000in}}%
\pgfusepath{clip}%
\pgfsetbuttcap%
\pgfsetroundjoin%
\pgfsetlinewidth{0.501875pt}%
\definecolor{currentstroke}{rgb}{0.000000,0.000000,1.000000}%
\pgfsetstrokecolor{currentstroke}%
\pgfsetdash{{0.500000pt}{0.825000pt}}{0.000000pt}%
\pgfpathmoveto{\pgfqpoint{6.497432in}{5.330720in}}%
\pgfpathlineto{\pgfqpoint{8.225522in}{3.331500in}}%
\pgfusepath{stroke}%
\end{pgfscope}%
\begin{pgfscope}%
\pgfpathrectangle{\pgfqpoint{4.577333in}{3.075000in}}{\pgfqpoint{4.224218in}{2.565000in}}%
\pgfusepath{clip}%
\pgfsetbuttcap%
\pgfsetroundjoin%
\pgfsetlinewidth{0.501875pt}%
\definecolor{currentstroke}{rgb}{0.000000,0.000000,1.000000}%
\pgfsetstrokecolor{currentstroke}%
\pgfsetdash{{0.500000pt}{0.825000pt}}{0.000000pt}%
\pgfpathmoveto{\pgfqpoint{7.073462in}{5.108584in}}%
\pgfpathlineto{\pgfqpoint{5.537383in}{3.331500in}}%
\pgfusepath{stroke}%
\end{pgfscope}%
\begin{pgfscope}%
\pgfpathrectangle{\pgfqpoint{4.577333in}{3.075000in}}{\pgfqpoint{4.224218in}{2.565000in}}%
\pgfusepath{clip}%
\pgfsetbuttcap%
\pgfsetroundjoin%
\pgfsetlinewidth{0.501875pt}%
\definecolor{currentstroke}{rgb}{0.000000,0.000000,1.000000}%
\pgfsetstrokecolor{currentstroke}%
\pgfsetdash{{0.500000pt}{0.825000pt}}{0.000000pt}%
\pgfpathmoveto{\pgfqpoint{6.305422in}{5.108584in}}%
\pgfpathlineto{\pgfqpoint{7.841502in}{3.331500in}}%
\pgfusepath{stroke}%
\end{pgfscope}%
\begin{pgfscope}%
\pgfpathrectangle{\pgfqpoint{4.577333in}{3.075000in}}{\pgfqpoint{4.224218in}{2.565000in}}%
\pgfusepath{clip}%
\pgfsetbuttcap%
\pgfsetroundjoin%
\pgfsetlinewidth{0.501875pt}%
\definecolor{currentstroke}{rgb}{0.000000,0.000000,1.000000}%
\pgfsetstrokecolor{currentstroke}%
\pgfsetdash{{0.500000pt}{0.825000pt}}{0.000000pt}%
\pgfpathmoveto{\pgfqpoint{7.265472in}{4.886449in}}%
\pgfpathlineto{\pgfqpoint{5.921402in}{3.331500in}}%
\pgfusepath{stroke}%
\end{pgfscope}%
\begin{pgfscope}%
\pgfpathrectangle{\pgfqpoint{4.577333in}{3.075000in}}{\pgfqpoint{4.224218in}{2.565000in}}%
\pgfusepath{clip}%
\pgfsetbuttcap%
\pgfsetroundjoin%
\pgfsetlinewidth{0.501875pt}%
\definecolor{currentstroke}{rgb}{0.000000,0.000000,1.000000}%
\pgfsetstrokecolor{currentstroke}%
\pgfsetdash{{0.500000pt}{0.825000pt}}{0.000000pt}%
\pgfpathmoveto{\pgfqpoint{6.113412in}{4.886449in}}%
\pgfpathlineto{\pgfqpoint{7.457482in}{3.331500in}}%
\pgfusepath{stroke}%
\end{pgfscope}%
\begin{pgfscope}%
\pgfpathrectangle{\pgfqpoint{4.577333in}{3.075000in}}{\pgfqpoint{4.224218in}{2.565000in}}%
\pgfusepath{clip}%
\pgfsetbuttcap%
\pgfsetroundjoin%
\pgfsetlinewidth{0.501875pt}%
\definecolor{currentstroke}{rgb}{0.000000,0.000000,1.000000}%
\pgfsetstrokecolor{currentstroke}%
\pgfsetdash{{0.500000pt}{0.825000pt}}{0.000000pt}%
\pgfpathmoveto{\pgfqpoint{7.457482in}{4.664313in}}%
\pgfpathlineto{\pgfqpoint{6.305422in}{3.331500in}}%
\pgfusepath{stroke}%
\end{pgfscope}%
\begin{pgfscope}%
\pgfpathrectangle{\pgfqpoint{4.577333in}{3.075000in}}{\pgfqpoint{4.224218in}{2.565000in}}%
\pgfusepath{clip}%
\pgfsetbuttcap%
\pgfsetroundjoin%
\pgfsetlinewidth{0.501875pt}%
\definecolor{currentstroke}{rgb}{0.000000,0.000000,1.000000}%
\pgfsetstrokecolor{currentstroke}%
\pgfsetdash{{0.500000pt}{0.825000pt}}{0.000000pt}%
\pgfpathmoveto{\pgfqpoint{5.921402in}{4.664313in}}%
\pgfpathlineto{\pgfqpoint{7.073462in}{3.331500in}}%
\pgfusepath{stroke}%
\end{pgfscope}%
\begin{pgfscope}%
\pgfpathrectangle{\pgfqpoint{4.577333in}{3.075000in}}{\pgfqpoint{4.224218in}{2.565000in}}%
\pgfusepath{clip}%
\pgfsetbuttcap%
\pgfsetroundjoin%
\pgfsetlinewidth{0.501875pt}%
\definecolor{currentstroke}{rgb}{0.000000,0.000000,1.000000}%
\pgfsetstrokecolor{currentstroke}%
\pgfsetdash{{0.500000pt}{0.825000pt}}{0.000000pt}%
\pgfpathmoveto{\pgfqpoint{7.649492in}{4.442178in}}%
\pgfpathlineto{\pgfqpoint{6.689442in}{3.331500in}}%
\pgfusepath{stroke}%
\end{pgfscope}%
\begin{pgfscope}%
\pgfpathrectangle{\pgfqpoint{4.577333in}{3.075000in}}{\pgfqpoint{4.224218in}{2.565000in}}%
\pgfusepath{clip}%
\pgfsetbuttcap%
\pgfsetroundjoin%
\pgfsetlinewidth{0.501875pt}%
\definecolor{currentstroke}{rgb}{0.000000,0.000000,1.000000}%
\pgfsetstrokecolor{currentstroke}%
\pgfsetdash{{0.500000pt}{0.825000pt}}{0.000000pt}%
\pgfpathmoveto{\pgfqpoint{5.729393in}{4.442178in}}%
\pgfpathlineto{\pgfqpoint{6.689442in}{3.331500in}}%
\pgfusepath{stroke}%
\end{pgfscope}%
\begin{pgfscope}%
\pgfpathrectangle{\pgfqpoint{4.577333in}{3.075000in}}{\pgfqpoint{4.224218in}{2.565000in}}%
\pgfusepath{clip}%
\pgfsetbuttcap%
\pgfsetroundjoin%
\pgfsetlinewidth{0.501875pt}%
\definecolor{currentstroke}{rgb}{0.000000,0.000000,1.000000}%
\pgfsetstrokecolor{currentstroke}%
\pgfsetdash{{0.500000pt}{0.825000pt}}{0.000000pt}%
\pgfpathmoveto{\pgfqpoint{7.841502in}{4.220042in}}%
\pgfpathlineto{\pgfqpoint{7.073462in}{3.331500in}}%
\pgfusepath{stroke}%
\end{pgfscope}%
\begin{pgfscope}%
\pgfpathrectangle{\pgfqpoint{4.577333in}{3.075000in}}{\pgfqpoint{4.224218in}{2.565000in}}%
\pgfusepath{clip}%
\pgfsetbuttcap%
\pgfsetroundjoin%
\pgfsetlinewidth{0.501875pt}%
\definecolor{currentstroke}{rgb}{0.000000,0.000000,1.000000}%
\pgfsetstrokecolor{currentstroke}%
\pgfsetdash{{0.500000pt}{0.825000pt}}{0.000000pt}%
\pgfpathmoveto{\pgfqpoint{5.537383in}{4.220042in}}%
\pgfpathlineto{\pgfqpoint{6.305422in}{3.331500in}}%
\pgfusepath{stroke}%
\end{pgfscope}%
\begin{pgfscope}%
\pgfpathrectangle{\pgfqpoint{4.577333in}{3.075000in}}{\pgfqpoint{4.224218in}{2.565000in}}%
\pgfusepath{clip}%
\pgfsetbuttcap%
\pgfsetroundjoin%
\pgfsetlinewidth{0.501875pt}%
\definecolor{currentstroke}{rgb}{0.000000,0.000000,1.000000}%
\pgfsetstrokecolor{currentstroke}%
\pgfsetdash{{0.500000pt}{0.825000pt}}{0.000000pt}%
\pgfpathmoveto{\pgfqpoint{8.033512in}{3.997907in}}%
\pgfpathlineto{\pgfqpoint{7.457482in}{3.331500in}}%
\pgfusepath{stroke}%
\end{pgfscope}%
\begin{pgfscope}%
\pgfpathrectangle{\pgfqpoint{4.577333in}{3.075000in}}{\pgfqpoint{4.224218in}{2.565000in}}%
\pgfusepath{clip}%
\pgfsetbuttcap%
\pgfsetroundjoin%
\pgfsetlinewidth{0.501875pt}%
\definecolor{currentstroke}{rgb}{0.000000,0.000000,1.000000}%
\pgfsetstrokecolor{currentstroke}%
\pgfsetdash{{0.500000pt}{0.825000pt}}{0.000000pt}%
\pgfpathmoveto{\pgfqpoint{5.345373in}{3.997907in}}%
\pgfpathlineto{\pgfqpoint{5.921402in}{3.331500in}}%
\pgfusepath{stroke}%
\end{pgfscope}%
\begin{pgfscope}%
\pgfpathrectangle{\pgfqpoint{4.577333in}{3.075000in}}{\pgfqpoint{4.224218in}{2.565000in}}%
\pgfusepath{clip}%
\pgfsetbuttcap%
\pgfsetroundjoin%
\pgfsetlinewidth{0.501875pt}%
\definecolor{currentstroke}{rgb}{0.000000,0.000000,1.000000}%
\pgfsetstrokecolor{currentstroke}%
\pgfsetdash{{0.500000pt}{0.825000pt}}{0.000000pt}%
\pgfpathmoveto{\pgfqpoint{8.225522in}{3.775771in}}%
\pgfpathlineto{\pgfqpoint{7.841502in}{3.331500in}}%
\pgfusepath{stroke}%
\end{pgfscope}%
\begin{pgfscope}%
\pgfpathrectangle{\pgfqpoint{4.577333in}{3.075000in}}{\pgfqpoint{4.224218in}{2.565000in}}%
\pgfusepath{clip}%
\pgfsetbuttcap%
\pgfsetroundjoin%
\pgfsetlinewidth{0.501875pt}%
\definecolor{currentstroke}{rgb}{0.000000,0.000000,1.000000}%
\pgfsetstrokecolor{currentstroke}%
\pgfsetdash{{0.500000pt}{0.825000pt}}{0.000000pt}%
\pgfpathmoveto{\pgfqpoint{5.153363in}{3.775771in}}%
\pgfpathlineto{\pgfqpoint{5.537383in}{3.331500in}}%
\pgfusepath{stroke}%
\end{pgfscope}%
\begin{pgfscope}%
\pgfpathrectangle{\pgfqpoint{4.577333in}{3.075000in}}{\pgfqpoint{4.224218in}{2.565000in}}%
\pgfusepath{clip}%
\pgfsetbuttcap%
\pgfsetroundjoin%
\pgfsetlinewidth{0.501875pt}%
\definecolor{currentstroke}{rgb}{0.000000,0.000000,1.000000}%
\pgfsetstrokecolor{currentstroke}%
\pgfsetdash{{0.500000pt}{0.825000pt}}{0.000000pt}%
\pgfpathmoveto{\pgfqpoint{8.417531in}{3.553636in}}%
\pgfpathlineto{\pgfqpoint{8.225522in}{3.331500in}}%
\pgfusepath{stroke}%
\end{pgfscope}%
\begin{pgfscope}%
\pgfpathrectangle{\pgfqpoint{4.577333in}{3.075000in}}{\pgfqpoint{4.224218in}{2.565000in}}%
\pgfusepath{clip}%
\pgfsetbuttcap%
\pgfsetroundjoin%
\pgfsetlinewidth{0.501875pt}%
\definecolor{currentstroke}{rgb}{0.000000,0.000000,1.000000}%
\pgfsetstrokecolor{currentstroke}%
\pgfsetdash{{0.500000pt}{0.825000pt}}{0.000000pt}%
\pgfpathmoveto{\pgfqpoint{4.961353in}{3.553636in}}%
\pgfpathlineto{\pgfqpoint{5.153363in}{3.331500in}}%
\pgfusepath{stroke}%
\end{pgfscope}%
\begin{pgfscope}%
\pgfpathrectangle{\pgfqpoint{4.577333in}{3.075000in}}{\pgfqpoint{4.224218in}{2.565000in}}%
\pgfusepath{clip}%
\pgfsetbuttcap%
\pgfsetroundjoin%
\pgfsetlinewidth{0.501875pt}%
\definecolor{currentstroke}{rgb}{0.000000,0.000000,1.000000}%
\pgfsetstrokecolor{currentstroke}%
\pgfsetdash{{0.500000pt}{0.825000pt}}{0.000000pt}%
\pgfpathmoveto{\pgfqpoint{8.609541in}{3.331500in}}%
\pgfpathlineto{\pgfqpoint{8.609541in}{3.331500in}}%
\pgfusepath{stroke}%
\end{pgfscope}%
\begin{pgfscope}%
\pgfpathrectangle{\pgfqpoint{4.577333in}{3.075000in}}{\pgfqpoint{4.224218in}{2.565000in}}%
\pgfusepath{clip}%
\pgfsetbuttcap%
\pgfsetroundjoin%
\pgfsetlinewidth{0.501875pt}%
\definecolor{currentstroke}{rgb}{0.000000,0.000000,1.000000}%
\pgfsetstrokecolor{currentstroke}%
\pgfsetdash{{0.500000pt}{0.825000pt}}{0.000000pt}%
\pgfpathmoveto{\pgfqpoint{4.769343in}{3.331500in}}%
\pgfpathlineto{\pgfqpoint{4.769343in}{3.331500in}}%
\pgfusepath{stroke}%
\end{pgfscope}%
\begin{pgfscope}%
\pgfpathrectangle{\pgfqpoint{4.577333in}{3.075000in}}{\pgfqpoint{4.224218in}{2.565000in}}%
\pgfusepath{clip}%
\pgfsetrectcap%
\pgfsetroundjoin%
\pgfsetlinewidth{1.003750pt}%
\definecolor{currentstroke}{rgb}{0.000000,0.000000,0.000000}%
\pgfsetstrokecolor{currentstroke}%
\pgfsetdash{}{0pt}%
\pgfpathmoveto{\pgfqpoint{8.609541in}{3.331500in}}%
\pgfpathlineto{\pgfqpoint{8.647943in}{3.331500in}}%
\pgfusepath{stroke}%
\end{pgfscope}%
\begin{pgfscope}%
\pgfpathrectangle{\pgfqpoint{4.577333in}{3.075000in}}{\pgfqpoint{4.224218in}{2.565000in}}%
\pgfusepath{clip}%
\pgfsetrectcap%
\pgfsetroundjoin%
\pgfsetlinewidth{1.003750pt}%
\definecolor{currentstroke}{rgb}{0.000000,0.000000,0.000000}%
\pgfsetstrokecolor{currentstroke}%
\pgfsetdash{}{0pt}%
\pgfpathmoveto{\pgfqpoint{8.417531in}{3.553636in}}%
\pgfpathlineto{\pgfqpoint{8.455933in}{3.553636in}}%
\pgfusepath{stroke}%
\end{pgfscope}%
\begin{pgfscope}%
\pgfpathrectangle{\pgfqpoint{4.577333in}{3.075000in}}{\pgfqpoint{4.224218in}{2.565000in}}%
\pgfusepath{clip}%
\pgfsetrectcap%
\pgfsetroundjoin%
\pgfsetlinewidth{1.003750pt}%
\definecolor{currentstroke}{rgb}{0.000000,0.000000,0.000000}%
\pgfsetstrokecolor{currentstroke}%
\pgfsetdash{}{0pt}%
\pgfpathmoveto{\pgfqpoint{8.225522in}{3.775771in}}%
\pgfpathlineto{\pgfqpoint{8.263924in}{3.775771in}}%
\pgfusepath{stroke}%
\end{pgfscope}%
\begin{pgfscope}%
\pgfpathrectangle{\pgfqpoint{4.577333in}{3.075000in}}{\pgfqpoint{4.224218in}{2.565000in}}%
\pgfusepath{clip}%
\pgfsetrectcap%
\pgfsetroundjoin%
\pgfsetlinewidth{1.003750pt}%
\definecolor{currentstroke}{rgb}{0.000000,0.000000,0.000000}%
\pgfsetstrokecolor{currentstroke}%
\pgfsetdash{}{0pt}%
\pgfpathmoveto{\pgfqpoint{8.033512in}{3.997907in}}%
\pgfpathlineto{\pgfqpoint{8.071914in}{3.997907in}}%
\pgfusepath{stroke}%
\end{pgfscope}%
\begin{pgfscope}%
\pgfpathrectangle{\pgfqpoint{4.577333in}{3.075000in}}{\pgfqpoint{4.224218in}{2.565000in}}%
\pgfusepath{clip}%
\pgfsetrectcap%
\pgfsetroundjoin%
\pgfsetlinewidth{1.003750pt}%
\definecolor{currentstroke}{rgb}{0.000000,0.000000,0.000000}%
\pgfsetstrokecolor{currentstroke}%
\pgfsetdash{}{0pt}%
\pgfpathmoveto{\pgfqpoint{7.841502in}{4.220042in}}%
\pgfpathlineto{\pgfqpoint{7.879904in}{4.220042in}}%
\pgfusepath{stroke}%
\end{pgfscope}%
\begin{pgfscope}%
\pgfpathrectangle{\pgfqpoint{4.577333in}{3.075000in}}{\pgfqpoint{4.224218in}{2.565000in}}%
\pgfusepath{clip}%
\pgfsetrectcap%
\pgfsetroundjoin%
\pgfsetlinewidth{1.003750pt}%
\definecolor{currentstroke}{rgb}{0.000000,0.000000,0.000000}%
\pgfsetstrokecolor{currentstroke}%
\pgfsetdash{}{0pt}%
\pgfpathmoveto{\pgfqpoint{7.649492in}{4.442178in}}%
\pgfpathlineto{\pgfqpoint{7.687894in}{4.442178in}}%
\pgfusepath{stroke}%
\end{pgfscope}%
\begin{pgfscope}%
\pgfpathrectangle{\pgfqpoint{4.577333in}{3.075000in}}{\pgfqpoint{4.224218in}{2.565000in}}%
\pgfusepath{clip}%
\pgfsetrectcap%
\pgfsetroundjoin%
\pgfsetlinewidth{1.003750pt}%
\definecolor{currentstroke}{rgb}{0.000000,0.000000,0.000000}%
\pgfsetstrokecolor{currentstroke}%
\pgfsetdash{}{0pt}%
\pgfpathmoveto{\pgfqpoint{7.457482in}{4.664313in}}%
\pgfpathlineto{\pgfqpoint{7.495884in}{4.664313in}}%
\pgfusepath{stroke}%
\end{pgfscope}%
\begin{pgfscope}%
\pgfpathrectangle{\pgfqpoint{4.577333in}{3.075000in}}{\pgfqpoint{4.224218in}{2.565000in}}%
\pgfusepath{clip}%
\pgfsetrectcap%
\pgfsetroundjoin%
\pgfsetlinewidth{1.003750pt}%
\definecolor{currentstroke}{rgb}{0.000000,0.000000,0.000000}%
\pgfsetstrokecolor{currentstroke}%
\pgfsetdash{}{0pt}%
\pgfpathmoveto{\pgfqpoint{7.265472in}{4.886449in}}%
\pgfpathlineto{\pgfqpoint{7.303874in}{4.886449in}}%
\pgfusepath{stroke}%
\end{pgfscope}%
\begin{pgfscope}%
\pgfpathrectangle{\pgfqpoint{4.577333in}{3.075000in}}{\pgfqpoint{4.224218in}{2.565000in}}%
\pgfusepath{clip}%
\pgfsetrectcap%
\pgfsetroundjoin%
\pgfsetlinewidth{1.003750pt}%
\definecolor{currentstroke}{rgb}{0.000000,0.000000,0.000000}%
\pgfsetstrokecolor{currentstroke}%
\pgfsetdash{}{0pt}%
\pgfpathmoveto{\pgfqpoint{7.073462in}{5.108584in}}%
\pgfpathlineto{\pgfqpoint{7.111864in}{5.108584in}}%
\pgfusepath{stroke}%
\end{pgfscope}%
\begin{pgfscope}%
\pgfpathrectangle{\pgfqpoint{4.577333in}{3.075000in}}{\pgfqpoint{4.224218in}{2.565000in}}%
\pgfusepath{clip}%
\pgfsetrectcap%
\pgfsetroundjoin%
\pgfsetlinewidth{1.003750pt}%
\definecolor{currentstroke}{rgb}{0.000000,0.000000,0.000000}%
\pgfsetstrokecolor{currentstroke}%
\pgfsetdash{}{0pt}%
\pgfpathmoveto{\pgfqpoint{6.881452in}{5.330720in}}%
\pgfpathlineto{\pgfqpoint{6.919854in}{5.330720in}}%
\pgfusepath{stroke}%
\end{pgfscope}%
\begin{pgfscope}%
\pgfpathrectangle{\pgfqpoint{4.577333in}{3.075000in}}{\pgfqpoint{4.224218in}{2.565000in}}%
\pgfusepath{clip}%
\pgfsetrectcap%
\pgfsetroundjoin%
\pgfsetlinewidth{1.003750pt}%
\definecolor{currentstroke}{rgb}{0.000000,0.000000,0.000000}%
\pgfsetstrokecolor{currentstroke}%
\pgfsetdash{}{0pt}%
\pgfpathmoveto{\pgfqpoint{6.689442in}{5.552855in}}%
\pgfpathlineto{\pgfqpoint{6.727844in}{5.552855in}}%
\pgfusepath{stroke}%
\end{pgfscope}%
\begin{pgfscope}%
\pgfpathrectangle{\pgfqpoint{4.577333in}{3.075000in}}{\pgfqpoint{4.224218in}{2.565000in}}%
\pgfusepath{clip}%
\pgfsetrectcap%
\pgfsetroundjoin%
\pgfsetlinewidth{1.003750pt}%
\definecolor{currentstroke}{rgb}{0.000000,0.000000,0.000000}%
\pgfsetstrokecolor{currentstroke}%
\pgfsetdash{}{0pt}%
\pgfpathmoveto{\pgfqpoint{4.769343in}{3.331500in}}%
\pgfpathlineto{\pgfqpoint{4.750142in}{3.353714in}}%
\pgfusepath{stroke}%
\end{pgfscope}%
\begin{pgfscope}%
\pgfpathrectangle{\pgfqpoint{4.577333in}{3.075000in}}{\pgfqpoint{4.224218in}{2.565000in}}%
\pgfusepath{clip}%
\pgfsetrectcap%
\pgfsetroundjoin%
\pgfsetlinewidth{1.003750pt}%
\definecolor{currentstroke}{rgb}{0.000000,0.000000,0.000000}%
\pgfsetstrokecolor{currentstroke}%
\pgfsetdash{}{0pt}%
\pgfpathmoveto{\pgfqpoint{4.961353in}{3.553636in}}%
\pgfpathlineto{\pgfqpoint{4.942152in}{3.575849in}}%
\pgfusepath{stroke}%
\end{pgfscope}%
\begin{pgfscope}%
\pgfpathrectangle{\pgfqpoint{4.577333in}{3.075000in}}{\pgfqpoint{4.224218in}{2.565000in}}%
\pgfusepath{clip}%
\pgfsetrectcap%
\pgfsetroundjoin%
\pgfsetlinewidth{1.003750pt}%
\definecolor{currentstroke}{rgb}{0.000000,0.000000,0.000000}%
\pgfsetstrokecolor{currentstroke}%
\pgfsetdash{}{0pt}%
\pgfpathmoveto{\pgfqpoint{5.153363in}{3.775771in}}%
\pgfpathlineto{\pgfqpoint{5.134162in}{3.797985in}}%
\pgfusepath{stroke}%
\end{pgfscope}%
\begin{pgfscope}%
\pgfpathrectangle{\pgfqpoint{4.577333in}{3.075000in}}{\pgfqpoint{4.224218in}{2.565000in}}%
\pgfusepath{clip}%
\pgfsetrectcap%
\pgfsetroundjoin%
\pgfsetlinewidth{1.003750pt}%
\definecolor{currentstroke}{rgb}{0.000000,0.000000,0.000000}%
\pgfsetstrokecolor{currentstroke}%
\pgfsetdash{}{0pt}%
\pgfpathmoveto{\pgfqpoint{5.345373in}{3.997907in}}%
\pgfpathlineto{\pgfqpoint{5.326172in}{4.020120in}}%
\pgfusepath{stroke}%
\end{pgfscope}%
\begin{pgfscope}%
\pgfpathrectangle{\pgfqpoint{4.577333in}{3.075000in}}{\pgfqpoint{4.224218in}{2.565000in}}%
\pgfusepath{clip}%
\pgfsetrectcap%
\pgfsetroundjoin%
\pgfsetlinewidth{1.003750pt}%
\definecolor{currentstroke}{rgb}{0.000000,0.000000,0.000000}%
\pgfsetstrokecolor{currentstroke}%
\pgfsetdash{}{0pt}%
\pgfpathmoveto{\pgfqpoint{5.537383in}{4.220042in}}%
\pgfpathlineto{\pgfqpoint{5.518182in}{4.242256in}}%
\pgfusepath{stroke}%
\end{pgfscope}%
\begin{pgfscope}%
\pgfpathrectangle{\pgfqpoint{4.577333in}{3.075000in}}{\pgfqpoint{4.224218in}{2.565000in}}%
\pgfusepath{clip}%
\pgfsetrectcap%
\pgfsetroundjoin%
\pgfsetlinewidth{1.003750pt}%
\definecolor{currentstroke}{rgb}{0.000000,0.000000,0.000000}%
\pgfsetstrokecolor{currentstroke}%
\pgfsetdash{}{0pt}%
\pgfpathmoveto{\pgfqpoint{5.729393in}{4.442178in}}%
\pgfpathlineto{\pgfqpoint{5.710192in}{4.464391in}}%
\pgfusepath{stroke}%
\end{pgfscope}%
\begin{pgfscope}%
\pgfpathrectangle{\pgfqpoint{4.577333in}{3.075000in}}{\pgfqpoint{4.224218in}{2.565000in}}%
\pgfusepath{clip}%
\pgfsetrectcap%
\pgfsetroundjoin%
\pgfsetlinewidth{1.003750pt}%
\definecolor{currentstroke}{rgb}{0.000000,0.000000,0.000000}%
\pgfsetstrokecolor{currentstroke}%
\pgfsetdash{}{0pt}%
\pgfpathmoveto{\pgfqpoint{5.921402in}{4.664313in}}%
\pgfpathlineto{\pgfqpoint{5.902201in}{4.686527in}}%
\pgfusepath{stroke}%
\end{pgfscope}%
\begin{pgfscope}%
\pgfpathrectangle{\pgfqpoint{4.577333in}{3.075000in}}{\pgfqpoint{4.224218in}{2.565000in}}%
\pgfusepath{clip}%
\pgfsetrectcap%
\pgfsetroundjoin%
\pgfsetlinewidth{1.003750pt}%
\definecolor{currentstroke}{rgb}{0.000000,0.000000,0.000000}%
\pgfsetstrokecolor{currentstroke}%
\pgfsetdash{}{0pt}%
\pgfpathmoveto{\pgfqpoint{6.113412in}{4.886449in}}%
\pgfpathlineto{\pgfqpoint{6.094211in}{4.908662in}}%
\pgfusepath{stroke}%
\end{pgfscope}%
\begin{pgfscope}%
\pgfpathrectangle{\pgfqpoint{4.577333in}{3.075000in}}{\pgfqpoint{4.224218in}{2.565000in}}%
\pgfusepath{clip}%
\pgfsetrectcap%
\pgfsetroundjoin%
\pgfsetlinewidth{1.003750pt}%
\definecolor{currentstroke}{rgb}{0.000000,0.000000,0.000000}%
\pgfsetstrokecolor{currentstroke}%
\pgfsetdash{}{0pt}%
\pgfpathmoveto{\pgfqpoint{6.305422in}{5.108584in}}%
\pgfpathlineto{\pgfqpoint{6.286221in}{5.130798in}}%
\pgfusepath{stroke}%
\end{pgfscope}%
\begin{pgfscope}%
\pgfpathrectangle{\pgfqpoint{4.577333in}{3.075000in}}{\pgfqpoint{4.224218in}{2.565000in}}%
\pgfusepath{clip}%
\pgfsetrectcap%
\pgfsetroundjoin%
\pgfsetlinewidth{1.003750pt}%
\definecolor{currentstroke}{rgb}{0.000000,0.000000,0.000000}%
\pgfsetstrokecolor{currentstroke}%
\pgfsetdash{}{0pt}%
\pgfpathmoveto{\pgfqpoint{6.497432in}{5.330720in}}%
\pgfpathlineto{\pgfqpoint{6.478231in}{5.352933in}}%
\pgfusepath{stroke}%
\end{pgfscope}%
\begin{pgfscope}%
\pgfpathrectangle{\pgfqpoint{4.577333in}{3.075000in}}{\pgfqpoint{4.224218in}{2.565000in}}%
\pgfusepath{clip}%
\pgfsetrectcap%
\pgfsetroundjoin%
\pgfsetlinewidth{1.003750pt}%
\definecolor{currentstroke}{rgb}{0.000000,0.000000,0.000000}%
\pgfsetstrokecolor{currentstroke}%
\pgfsetdash{}{0pt}%
\pgfpathmoveto{\pgfqpoint{6.689442in}{5.552855in}}%
\pgfpathlineto{\pgfqpoint{6.670241in}{5.575069in}}%
\pgfusepath{stroke}%
\end{pgfscope}%
\begin{pgfscope}%
\pgfpathrectangle{\pgfqpoint{4.577333in}{3.075000in}}{\pgfqpoint{4.224218in}{2.565000in}}%
\pgfusepath{clip}%
\pgfsetrectcap%
\pgfsetroundjoin%
\pgfsetlinewidth{1.003750pt}%
\definecolor{currentstroke}{rgb}{0.000000,0.000000,0.000000}%
\pgfsetstrokecolor{currentstroke}%
\pgfsetdash{}{0pt}%
\pgfpathmoveto{\pgfqpoint{4.769343in}{3.331500in}}%
\pgfpathlineto{\pgfqpoint{4.750142in}{3.309286in}}%
\pgfusepath{stroke}%
\end{pgfscope}%
\begin{pgfscope}%
\pgfpathrectangle{\pgfqpoint{4.577333in}{3.075000in}}{\pgfqpoint{4.224218in}{2.565000in}}%
\pgfusepath{clip}%
\pgfsetrectcap%
\pgfsetroundjoin%
\pgfsetlinewidth{1.003750pt}%
\definecolor{currentstroke}{rgb}{0.000000,0.000000,0.000000}%
\pgfsetstrokecolor{currentstroke}%
\pgfsetdash{}{0pt}%
\pgfpathmoveto{\pgfqpoint{5.153363in}{3.331500in}}%
\pgfpathlineto{\pgfqpoint{5.134162in}{3.309286in}}%
\pgfusepath{stroke}%
\end{pgfscope}%
\begin{pgfscope}%
\pgfpathrectangle{\pgfqpoint{4.577333in}{3.075000in}}{\pgfqpoint{4.224218in}{2.565000in}}%
\pgfusepath{clip}%
\pgfsetrectcap%
\pgfsetroundjoin%
\pgfsetlinewidth{1.003750pt}%
\definecolor{currentstroke}{rgb}{0.000000,0.000000,0.000000}%
\pgfsetstrokecolor{currentstroke}%
\pgfsetdash{}{0pt}%
\pgfpathmoveto{\pgfqpoint{5.537383in}{3.331500in}}%
\pgfpathlineto{\pgfqpoint{5.518182in}{3.309286in}}%
\pgfusepath{stroke}%
\end{pgfscope}%
\begin{pgfscope}%
\pgfpathrectangle{\pgfqpoint{4.577333in}{3.075000in}}{\pgfqpoint{4.224218in}{2.565000in}}%
\pgfusepath{clip}%
\pgfsetrectcap%
\pgfsetroundjoin%
\pgfsetlinewidth{1.003750pt}%
\definecolor{currentstroke}{rgb}{0.000000,0.000000,0.000000}%
\pgfsetstrokecolor{currentstroke}%
\pgfsetdash{}{0pt}%
\pgfpathmoveto{\pgfqpoint{5.921402in}{3.331500in}}%
\pgfpathlineto{\pgfqpoint{5.902201in}{3.309286in}}%
\pgfusepath{stroke}%
\end{pgfscope}%
\begin{pgfscope}%
\pgfpathrectangle{\pgfqpoint{4.577333in}{3.075000in}}{\pgfqpoint{4.224218in}{2.565000in}}%
\pgfusepath{clip}%
\pgfsetrectcap%
\pgfsetroundjoin%
\pgfsetlinewidth{1.003750pt}%
\definecolor{currentstroke}{rgb}{0.000000,0.000000,0.000000}%
\pgfsetstrokecolor{currentstroke}%
\pgfsetdash{}{0pt}%
\pgfpathmoveto{\pgfqpoint{6.305422in}{3.331500in}}%
\pgfpathlineto{\pgfqpoint{6.286221in}{3.309286in}}%
\pgfusepath{stroke}%
\end{pgfscope}%
\begin{pgfscope}%
\pgfpathrectangle{\pgfqpoint{4.577333in}{3.075000in}}{\pgfqpoint{4.224218in}{2.565000in}}%
\pgfusepath{clip}%
\pgfsetrectcap%
\pgfsetroundjoin%
\pgfsetlinewidth{1.003750pt}%
\definecolor{currentstroke}{rgb}{0.000000,0.000000,0.000000}%
\pgfsetstrokecolor{currentstroke}%
\pgfsetdash{}{0pt}%
\pgfpathmoveto{\pgfqpoint{6.689442in}{3.331500in}}%
\pgfpathlineto{\pgfqpoint{6.670241in}{3.309286in}}%
\pgfusepath{stroke}%
\end{pgfscope}%
\begin{pgfscope}%
\pgfpathrectangle{\pgfqpoint{4.577333in}{3.075000in}}{\pgfqpoint{4.224218in}{2.565000in}}%
\pgfusepath{clip}%
\pgfsetrectcap%
\pgfsetroundjoin%
\pgfsetlinewidth{1.003750pt}%
\definecolor{currentstroke}{rgb}{0.000000,0.000000,0.000000}%
\pgfsetstrokecolor{currentstroke}%
\pgfsetdash{}{0pt}%
\pgfpathmoveto{\pgfqpoint{7.073462in}{3.331500in}}%
\pgfpathlineto{\pgfqpoint{7.054261in}{3.309286in}}%
\pgfusepath{stroke}%
\end{pgfscope}%
\begin{pgfscope}%
\pgfpathrectangle{\pgfqpoint{4.577333in}{3.075000in}}{\pgfqpoint{4.224218in}{2.565000in}}%
\pgfusepath{clip}%
\pgfsetrectcap%
\pgfsetroundjoin%
\pgfsetlinewidth{1.003750pt}%
\definecolor{currentstroke}{rgb}{0.000000,0.000000,0.000000}%
\pgfsetstrokecolor{currentstroke}%
\pgfsetdash{}{0pt}%
\pgfpathmoveto{\pgfqpoint{7.457482in}{3.331500in}}%
\pgfpathlineto{\pgfqpoint{7.438281in}{3.309286in}}%
\pgfusepath{stroke}%
\end{pgfscope}%
\begin{pgfscope}%
\pgfpathrectangle{\pgfqpoint{4.577333in}{3.075000in}}{\pgfqpoint{4.224218in}{2.565000in}}%
\pgfusepath{clip}%
\pgfsetrectcap%
\pgfsetroundjoin%
\pgfsetlinewidth{1.003750pt}%
\definecolor{currentstroke}{rgb}{0.000000,0.000000,0.000000}%
\pgfsetstrokecolor{currentstroke}%
\pgfsetdash{}{0pt}%
\pgfpathmoveto{\pgfqpoint{7.841502in}{3.331500in}}%
\pgfpathlineto{\pgfqpoint{7.822301in}{3.309286in}}%
\pgfusepath{stroke}%
\end{pgfscope}%
\begin{pgfscope}%
\pgfpathrectangle{\pgfqpoint{4.577333in}{3.075000in}}{\pgfqpoint{4.224218in}{2.565000in}}%
\pgfusepath{clip}%
\pgfsetrectcap%
\pgfsetroundjoin%
\pgfsetlinewidth{1.003750pt}%
\definecolor{currentstroke}{rgb}{0.000000,0.000000,0.000000}%
\pgfsetstrokecolor{currentstroke}%
\pgfsetdash{}{0pt}%
\pgfpathmoveto{\pgfqpoint{8.225522in}{3.331500in}}%
\pgfpathlineto{\pgfqpoint{8.206321in}{3.309286in}}%
\pgfusepath{stroke}%
\end{pgfscope}%
\begin{pgfscope}%
\pgfpathrectangle{\pgfqpoint{4.577333in}{3.075000in}}{\pgfqpoint{4.224218in}{2.565000in}}%
\pgfusepath{clip}%
\pgfsetrectcap%
\pgfsetroundjoin%
\pgfsetlinewidth{1.003750pt}%
\definecolor{currentstroke}{rgb}{0.000000,0.000000,0.000000}%
\pgfsetstrokecolor{currentstroke}%
\pgfsetdash{}{0pt}%
\pgfpathmoveto{\pgfqpoint{8.609541in}{3.331500in}}%
\pgfpathlineto{\pgfqpoint{8.590340in}{3.309286in}}%
\pgfusepath{stroke}%
\end{pgfscope}%
\begin{pgfscope}%
\pgfpathrectangle{\pgfqpoint{4.577333in}{3.075000in}}{\pgfqpoint{4.224218in}{2.565000in}}%
\pgfusepath{clip}%
\pgfsetrectcap%
\pgfsetroundjoin%
\pgfsetlinewidth{1.003750pt}%
\definecolor{currentstroke}{rgb}{0.000000,0.000000,1.000000}%
\pgfsetstrokecolor{currentstroke}%
\pgfsetdash{}{0pt}%
\pgfpathmoveto{\pgfqpoint{6.752865in}{4.157031in}}%
\pgfpathlineto{\pgfqpoint{6.774472in}{4.256270in}}%
\pgfpathlineto{\pgfqpoint{6.763390in}{4.364501in}}%
\pgfpathlineto{\pgfqpoint{6.732113in}{4.473342in}}%
\pgfpathlineto{\pgfqpoint{6.692222in}{4.574176in}}%
\pgfpathlineto{\pgfqpoint{6.651557in}{4.661110in}}%
\pgfpathlineto{\pgfqpoint{6.613877in}{4.731857in}}%
\pgfpathlineto{\pgfqpoint{6.580110in}{4.786804in}}%
\pgfpathlineto{\pgfqpoint{6.549768in}{4.827620in}}%
\pgfpathlineto{\pgfqpoint{6.521867in}{4.856262in}}%
\pgfpathlineto{\pgfqpoint{6.495354in}{4.874481in}}%
\pgfpathlineto{\pgfqpoint{6.469265in}{4.883652in}}%
\pgfpathlineto{\pgfqpoint{6.442744in}{4.884759in}}%
\pgfpathlineto{\pgfqpoint{6.415027in}{4.878441in}}%
\pgfpathlineto{\pgfqpoint{6.385415in}{4.865056in}}%
\pgfpathlineto{\pgfqpoint{6.353254in}{4.844736in}}%
\pgfpathlineto{\pgfqpoint{6.317928in}{4.817459in}}%
\pgfpathlineto{\pgfqpoint{6.278875in}{4.783113in}}%
\pgfpathlineto{\pgfqpoint{6.235627in}{4.741586in}}%
\pgfpathlineto{\pgfqpoint{6.187869in}{4.692868in}}%
\pgfpathlineto{\pgfqpoint{6.135531in}{4.637175in}}%
\pgfpathlineto{\pgfqpoint{6.078888in}{4.575076in}}%
\pgfpathlineto{\pgfqpoint{6.018641in}{4.507591in}}%
\pgfpathlineto{\pgfqpoint{5.892362in}{4.362846in}}%
\pgfpathlineto{\pgfqpoint{5.769672in}{4.218430in}}%
\pgfpathlineto{\pgfqpoint{5.713970in}{4.151155in}}%
\pgfpathlineto{\pgfqpoint{5.663817in}{4.089017in}}%
\pgfpathlineto{\pgfqpoint{5.620068in}{4.032781in}}%
\pgfpathlineto{\pgfqpoint{5.583232in}{3.982762in}}%
\pgfpathlineto{\pgfqpoint{5.553562in}{3.938933in}}%
\pgfpathlineto{\pgfqpoint{5.531176in}{3.901047in}}%
\pgfpathlineto{\pgfqpoint{5.516173in}{3.868745in}}%
\pgfpathlineto{\pgfqpoint{5.508729in}{3.841638in}}%
\pgfpathlineto{\pgfqpoint{5.509189in}{3.819366in}}%
\pgfpathlineto{\pgfqpoint{5.518152in}{3.801633in}}%
\pgfpathlineto{\pgfqpoint{5.536552in}{3.788236in}}%
\pgfpathlineto{\pgfqpoint{5.565752in}{3.779077in}}%
\pgfpathlineto{\pgfqpoint{5.607628in}{3.774178in}}%
\pgfpathlineto{\pgfqpoint{5.664603in}{3.773689in}}%
\pgfpathlineto{\pgfqpoint{5.739544in}{3.777893in}}%
\pgfpathlineto{\pgfqpoint{5.835303in}{3.787207in}}%
\pgfpathlineto{\pgfqpoint{5.953593in}{3.802196in}}%
\pgfpathlineto{\pgfqpoint{6.092955in}{3.823643in}}%
\pgfpathlineto{\pgfqpoint{6.246325in}{3.852742in}}%
\pgfpathlineto{\pgfqpoint{6.400049in}{3.891383in}}%
\pgfpathlineto{\pgfqpoint{6.536619in}{3.942315in}}%
\pgfpathlineto{\pgfqpoint{6.640526in}{4.008733in}}%
\pgfpathlineto{\pgfqpoint{6.703464in}{4.093003in}}%
\pgfpathlineto{\pgfqpoint{6.726109in}{4.194644in}}%
\pgfpathlineto{\pgfqpoint{6.716739in}{4.308559in}}%
\pgfpathlineto{\pgfqpoint{6.687612in}{4.425378in}}%
\pgfpathlineto{\pgfqpoint{6.650342in}{4.534781in}}%
\pgfpathlineto{\pgfqpoint{6.612666in}{4.629403in}}%
\pgfpathlineto{\pgfqpoint{6.578068in}{4.706283in}}%
\pgfpathlineto{\pgfqpoint{6.547207in}{4.765797in}}%
\pgfpathlineto{\pgfqpoint{6.519435in}{4.809900in}}%
\pgfpathlineto{\pgfqpoint{6.493709in}{4.840888in}}%
\pgfpathlineto{\pgfqpoint{6.468994in}{4.860798in}}%
\pgfpathlineto{\pgfqpoint{6.444371in}{4.871224in}}%
\pgfpathlineto{\pgfqpoint{6.419047in}{4.873314in}}%
\pgfpathlineto{\pgfqpoint{6.392325in}{4.867832in}}%
\pgfpathlineto{\pgfqpoint{6.363572in}{4.855239in}}%
\pgfpathlineto{\pgfqpoint{6.332202in}{4.835758in}}%
\pgfpathlineto{\pgfqpoint{6.297674in}{4.809454in}}%
\pgfpathlineto{\pgfqpoint{6.259501in}{4.776302in}}%
\pgfpathlineto{\pgfqpoint{6.217291in}{4.736270in}}%
\pgfpathlineto{\pgfqpoint{6.170803in}{4.689419in}}%
\pgfpathlineto{\pgfqpoint{6.120021in}{4.636008in}}%
\pgfpathlineto{\pgfqpoint{6.065238in}{4.576603in}}%
\pgfpathlineto{\pgfqpoint{6.007122in}{4.512155in}}%
\pgfpathlineto{\pgfqpoint{5.885471in}{4.373908in}}%
\pgfpathlineto{\pgfqpoint{5.766765in}{4.235213in}}%
\pgfpathlineto{\pgfqpoint{5.712417in}{4.170099in}}%
\pgfpathlineto{\pgfqpoint{5.663075in}{4.109580in}}%
\pgfpathlineto{\pgfqpoint{5.619582in}{4.054456in}}%
\pgfpathlineto{\pgfqpoint{5.582466in}{4.005128in}}%
\pgfpathlineto{\pgfqpoint{5.552024in}{3.961677in}}%
\pgfpathlineto{\pgfqpoint{5.528412in}{3.923970in}}%
\pgfpathlineto{\pgfqpoint{5.511752in}{3.891754in}}%
\pgfpathlineto{\pgfqpoint{5.502224in}{3.864738in}}%
\pgfpathlineto{\pgfqpoint{5.500149in}{3.842649in}}%
\pgfpathlineto{\pgfqpoint{5.506078in}{3.825284in}}%
\pgfpathlineto{\pgfqpoint{5.520878in}{3.812535in}}%
\pgfpathlineto{\pgfqpoint{5.545837in}{3.804424in}}%
\pgfpathlineto{\pgfqpoint{5.582778in}{3.801129in}}%
\pgfpathlineto{\pgfqpoint{5.634164in}{3.803004in}}%
\pgfpathlineto{\pgfqpoint{5.703126in}{3.810598in}}%
\pgfpathlineto{\pgfqpoint{5.793237in}{3.824628in}}%
\pgfpathlineto{\pgfqpoint{5.907673in}{3.845904in}}%
\pgfpathlineto{\pgfqpoint{6.047289in}{3.875155in}}%
\pgfpathlineto{\pgfqpoint{6.207572in}{3.912859in}}%
\pgfpathlineto{\pgfqpoint{6.376085in}{3.959317in}}%
\pgfpathlineto{\pgfqpoint{6.533833in}{4.015150in}}%
\pgfpathlineto{\pgfqpoint{6.661800in}{4.081696in}}%
\pgfpathlineto{\pgfqpoint{6.748173in}{4.160415in}}%
\pgfpathlineto{\pgfqpoint{6.791030in}{4.251271in}}%
\pgfpathlineto{\pgfqpoint{6.796618in}{4.351244in}}%
\pgfpathlineto{\pgfqpoint{6.775836in}{4.454270in}}%
\pgfpathlineto{\pgfqpoint{6.740442in}{4.552990in}}%
\pgfpathlineto{\pgfqpoint{6.699926in}{4.641233in}}%
\pgfpathlineto{\pgfqpoint{6.660090in}{4.715544in}}%
\pgfpathlineto{\pgfqpoint{6.623440in}{4.775115in}}%
\pgfpathlineto{\pgfqpoint{6.590394in}{4.820817in}}%
\pgfpathlineto{\pgfqpoint{6.560366in}{4.854212in}}%
\pgfpathlineto{\pgfqpoint{6.532423in}{4.876918in}}%
\pgfpathlineto{\pgfqpoint{6.505595in}{4.890317in}}%
\pgfpathlineto{\pgfqpoint{6.478979in}{4.895456in}}%
\pgfpathlineto{\pgfqpoint{6.451755in}{4.893050in}}%
\pgfpathlineto{\pgfqpoint{6.423167in}{4.883526in}}%
\pgfpathlineto{\pgfqpoint{6.392508in}{4.867065in}}%
\pgfpathlineto{\pgfqpoint{6.359100in}{4.843658in}}%
\pgfpathlineto{\pgfqpoint{6.322301in}{4.813168in}}%
\pgfpathlineto{\pgfqpoint{6.281523in}{4.775398in}}%
\pgfpathlineto{\pgfqpoint{6.236286in}{4.730189in}}%
\pgfpathlineto{\pgfqpoint{6.186295in}{4.677537in}}%
\pgfpathlineto{\pgfqpoint{6.131551in}{4.617735in}}%
\pgfpathlineto{\pgfqpoint{6.072459in}{4.551510in}}%
\pgfpathlineto{\pgfqpoint{5.945311in}{4.405331in}}%
\pgfpathlineto{\pgfqpoint{5.817247in}{4.254295in}}%
\pgfpathlineto{\pgfqpoint{5.757639in}{4.182401in}}%
\pgfpathlineto{\pgfqpoint{5.703195in}{4.115272in}}%
\pgfpathlineto{\pgfqpoint{5.655081in}{4.054007in}}%
\pgfpathlineto{\pgfqpoint{5.614036in}{3.999160in}}%
\pgfpathlineto{\pgfqpoint{5.580458in}{3.950837in}}%
\pgfpathlineto{\pgfqpoint{5.554526in}{3.908834in}}%
\pgfpathlineto{\pgfqpoint{5.536335in}{3.872768in}}%
\pgfpathlineto{\pgfqpoint{5.526008in}{3.842184in}}%
\pgfpathlineto{\pgfqpoint{5.523800in}{3.816621in}}%
\pgfpathlineto{\pgfqpoint{5.530178in}{3.795666in}}%
\pgfpathlineto{\pgfqpoint{5.545904in}{3.778975in}}%
\pgfpathlineto{\pgfqpoint{5.572095in}{3.766285in}}%
\pgfpathlineto{\pgfqpoint{5.610276in}{3.757421in}}%
\pgfpathlineto{\pgfqpoint{5.662367in}{3.752293in}}%
\pgfpathlineto{\pgfqpoint{5.730524in}{3.750896in}}%
\pgfpathlineto{\pgfqpoint{5.816685in}{3.753325in}}%
\pgfpathlineto{\pgfqpoint{5.921606in}{3.759833in}}%
\pgfpathlineto{\pgfqpoint{6.043353in}{3.770978in}}%
\pgfpathlineto{\pgfqpoint{6.175741in}{3.787897in}}%
\pgfpathlineto{\pgfqpoint{6.308071in}{3.812600in}}%
\pgfpathlineto{\pgfqpoint{6.427306in}{3.848047in}}%
\pgfpathlineto{\pgfqpoint{6.521909in}{3.897793in}}%
\pgfpathlineto{\pgfqpoint{6.584985in}{3.965221in}}%
\pgfpathlineto{\pgfqpoint{6.615522in}{4.052304in}}%
\pgfpathlineto{\pgfqpoint{6.618111in}{4.157816in}}%
\pgfpathlineto{\pgfqpoint{6.601356in}{4.275777in}}%
\pgfpathlineto{\pgfqpoint{6.574889in}{4.396055in}}%
\pgfpathlineto{\pgfqpoint{6.546187in}{4.507920in}}%
\pgfpathlineto{\pgfqpoint{6.519165in}{4.603979in}}%
\pgfpathlineto{\pgfqpoint{6.494858in}{4.681470in}}%
\pgfpathlineto{\pgfqpoint{6.472825in}{4.741011in}}%
\pgfpathlineto{\pgfqpoint{6.452159in}{4.784758in}}%
\pgfpathlineto{\pgfqpoint{6.431949in}{4.815146in}}%
\pgfpathlineto{\pgfqpoint{6.411413in}{4.834304in}}%
\pgfpathlineto{\pgfqpoint{6.389898in}{4.843891in}}%
\pgfpathlineto{\pgfqpoint{6.366849in}{4.845113in}}%
\pgfpathlineto{\pgfqpoint{6.341774in}{4.838792in}}%
\pgfpathlineto{\pgfqpoint{6.314224in}{4.825459in}}%
\pgfpathlineto{\pgfqpoint{6.283786in}{4.805428in}}%
\pgfpathlineto{\pgfqpoint{6.250093in}{4.778879in}}%
\pgfpathlineto{\pgfqpoint{6.212842in}{4.745926in}}%
\pgfpathlineto{\pgfqpoint{6.171834in}{4.706704in}}%
\pgfpathlineto{\pgfqpoint{6.127024in}{4.661445in}}%
\pgfpathlineto{\pgfqpoint{6.078578in}{4.610560in}}%
\pgfpathlineto{\pgfqpoint{6.026921in}{4.554714in}}%
\pgfpathlineto{\pgfqpoint{5.972767in}{4.494856in}}%
\pgfpathlineto{\pgfqpoint{5.861134in}{4.368220in}}%
\pgfpathlineto{\pgfqpoint{5.753510in}{4.242240in}}%
\pgfpathlineto{\pgfqpoint{5.704319in}{4.183015in}}%
\pgfpathlineto{\pgfqpoint{5.659555in}{4.127748in}}%
\pgfpathlineto{\pgfqpoint{5.619940in}{4.077147in}}%
\pgfpathlineto{\pgfqpoint{5.585978in}{4.031630in}}%
\pgfpathlineto{\pgfqpoint{5.558009in}{3.991375in}}%
\pgfpathlineto{\pgfqpoint{5.536277in}{3.956394in}}%
\pgfpathlineto{\pgfqpoint{5.521011in}{3.926603in}}%
\pgfpathlineto{\pgfqpoint{5.512497in}{3.901887in}}%
\pgfpathlineto{\pgfqpoint{5.511166in}{3.882153in}}%
\pgfpathlineto{\pgfqpoint{5.517668in}{3.867386in}}%
\pgfpathlineto{\pgfqpoint{5.532968in}{3.857684in}}%
\pgfpathlineto{\pgfqpoint{5.558455in}{3.853299in}}%
\pgfpathlineto{\pgfqpoint{5.596060in}{3.854680in}}%
\pgfpathlineto{\pgfqpoint{5.648353in}{3.862492in}}%
\pgfpathlineto{\pgfqpoint{5.718547in}{3.877597in}}%
\pgfpathlineto{\pgfqpoint{5.810186in}{3.900937in}}%
\pgfpathlineto{\pgfqpoint{5.926164in}{3.933208in}}%
\pgfpathlineto{\pgfqpoint{6.066685in}{3.974297in}}%
\pgfpathlineto{\pgfqpoint{6.226483in}{4.022712in}}%
\pgfpathlineto{\pgfqpoint{6.393288in}{4.075703in}}%
\pgfpathlineto{\pgfqpoint{6.550260in}{4.130566in}}%
\pgfpathlineto{\pgfqpoint{6.681877in}{4.186273in}}%
\pgfpathlineto{\pgfqpoint{6.778703in}{4.243867in}}%
\pgfpathlineto{\pgfqpoint{6.838084in}{4.305415in}}%
\pgfpathlineto{\pgfqpoint{6.862362in}{4.372539in}}%
\pgfpathlineto{\pgfqpoint{6.857128in}{4.445261in}}%
\pgfpathlineto{\pgfqpoint{6.830150in}{4.521393in}}%
\pgfpathlineto{\pgfqpoint{6.790122in}{4.596856in}}%
\pgfpathlineto{\pgfqpoint{6.744857in}{4.667038in}}%
\pgfpathlineto{\pgfqpoint{6.699808in}{4.728332in}}%
\pgfpathlineto{\pgfqpoint{6.657750in}{4.778866in}}%
\pgfpathlineto{\pgfqpoint{6.619434in}{4.818281in}}%
\pgfpathlineto{\pgfqpoint{6.584467in}{4.847120in}}%
\pgfpathlineto{\pgfqpoint{6.551957in}{4.866268in}}%
\pgfpathlineto{\pgfqpoint{6.520883in}{4.876606in}}%
\pgfpathlineto{\pgfqpoint{6.490239in}{4.878847in}}%
\pgfpathlineto{\pgfqpoint{6.459086in}{4.873474in}}%
\pgfpathlineto{\pgfqpoint{6.426549in}{4.860748in}}%
\pgfpathlineto{\pgfqpoint{6.391806in}{4.840727in}}%
\pgfpathlineto{\pgfqpoint{6.354081in}{4.813322in}}%
\pgfpathlineto{\pgfqpoint{6.312658in}{4.778353in}}%
\pgfpathlineto{\pgfqpoint{6.266922in}{4.735643in}}%
\pgfpathlineto{\pgfqpoint{6.216433in}{4.685135in}}%
\pgfpathlineto{\pgfqpoint{6.161029in}{4.627039in}}%
\pgfpathlineto{\pgfqpoint{6.100954in}{4.561980in}}%
\pgfpathlineto{\pgfqpoint{6.036962in}{4.491122in}}%
\pgfpathlineto{\pgfqpoint{5.902913in}{4.339375in}}%
\pgfpathlineto{\pgfqpoint{5.773631in}{4.189446in}}%
\pgfpathlineto{\pgfqpoint{5.715553in}{4.120423in}}%
\pgfpathlineto{\pgfqpoint{5.663727in}{4.057234in}}%
\pgfpathlineto{\pgfqpoint{5.618977in}{4.000543in}}%
\pgfpathlineto{\pgfqpoint{5.581722in}{3.950519in}}%
\pgfpathlineto{\pgfqpoint{5.552108in}{3.906978in}}%
\pgfpathlineto{\pgfqpoint{5.530145in}{3.869526in}}%
\pgfpathlineto{\pgfqpoint{5.515834in}{3.837678in}}%
\pgfpathlineto{\pgfqpoint{5.509264in}{3.810939in}}%
\pgfpathlineto{\pgfqpoint{5.510696in}{3.788860in}}%
\pgfpathlineto{\pgfqpoint{5.520631in}{3.771059in}}%
\pgfpathlineto{\pgfqpoint{5.539875in}{3.757249in}}%
\pgfpathlineto{\pgfqpoint{5.569593in}{3.747235in}}%
\pgfpathlineto{\pgfqpoint{5.611330in}{3.740928in}}%
\pgfpathlineto{\pgfqpoint{5.666975in}{3.738347in}}%
\pgfpathlineto{\pgfqpoint{5.738555in}{3.739636in}}%
\pgfpathlineto{\pgfqpoint{5.827729in}{3.745102in}}%
\pgfpathlineto{\pgfqpoint{5.934821in}{3.755306in}}%
\pgfpathlineto{\pgfqpoint{6.057456in}{3.771225in}}%
\pgfpathlineto{\pgfqpoint{6.189338in}{3.794489in}}%
\pgfpathlineto{\pgfqpoint{6.320276in}{3.827559in}}%
\pgfpathlineto{\pgfqpoint{6.438121in}{3.873662in}}%
\pgfpathlineto{\pgfqpoint{6.531918in}{3.936340in}}%
\pgfpathlineto{\pgfqpoint{6.594774in}{4.018496in}}%
\pgfpathlineto{\pgfqpoint{6.625569in}{4.120691in}}%
\pgfpathlineto{\pgfqpoint{6.629104in}{4.239005in}}%
\pgfpathlineto{\pgfqpoint{6.614240in}{4.364267in}}%
\pgfpathlineto{\pgfqpoint{6.590346in}{4.484711in}}%
\pgfpathlineto{\pgfqpoint{6.564185in}{4.590761in}}%
\pgfpathlineto{\pgfqpoint{6.539082in}{4.677827in}}%
\pgfpathlineto{\pgfqpoint{6.515917in}{4.745695in}}%
\pgfpathlineto{\pgfqpoint{6.494383in}{4.796431in}}%
\pgfpathlineto{\pgfqpoint{6.473788in}{4.832710in}}%
\pgfpathlineto{\pgfqpoint{6.453400in}{4.856969in}}%
\pgfpathlineto{\pgfqpoint{6.432552in}{4.871142in}}%
\pgfpathlineto{\pgfqpoint{6.410653in}{4.876638in}}%
\pgfpathlineto{\pgfqpoint{6.387166in}{4.874426in}}%
\pgfpathlineto{\pgfqpoint{6.361591in}{4.865116in}}%
\pgfpathlineto{\pgfqpoint{6.333450in}{4.849051in}}%
\pgfpathlineto{\pgfqpoint{6.302282in}{4.826381in}}%
\pgfpathlineto{\pgfqpoint{6.267658in}{4.797129in}}%
\pgfpathlineto{\pgfqpoint{6.229205in}{4.761267in}}%
\pgfpathlineto{\pgfqpoint{6.186650in}{4.718802in}}%
\pgfpathlineto{\pgfqpoint{6.139888in}{4.669867in}}%
\pgfpathlineto{\pgfqpoint{6.089049in}{4.614823in}}%
\pgfpathlineto{\pgfqpoint{6.034573in}{4.554346in}}%
\pgfpathlineto{\pgfqpoint{5.918236in}{4.421662in}}%
\pgfpathlineto{\pgfqpoint{5.800879in}{4.283956in}}%
\pgfpathlineto{\pgfqpoint{5.745600in}{4.217574in}}%
\pgfpathlineto{\pgfqpoint{5.694417in}{4.154847in}}%
\pgfpathlineto{\pgfqpoint{5.648372in}{4.096841in}}%
\pgfpathlineto{\pgfqpoint{5.608203in}{4.044223in}}%
\pgfpathlineto{\pgfqpoint{5.574380in}{3.997303in}}%
\pgfpathlineto{\pgfqpoint{5.547186in}{3.956114in}}%
\pgfpathlineto{\pgfqpoint{5.526808in}{3.920515in}}%
\pgfpathlineto{\pgfqpoint{5.513437in}{3.890268in}}%
\pgfpathlineto{\pgfqpoint{5.507355in}{3.865116in}}%
\pgfpathlineto{\pgfqpoint{5.509025in}{3.844836in}}%
\pgfpathlineto{\pgfqpoint{5.519180in}{3.829277in}}%
\pgfpathlineto{\pgfqpoint{5.538920in}{3.818393in}}%
\pgfpathlineto{\pgfqpoint{5.569835in}{3.812273in}}%
\pgfpathlineto{\pgfqpoint{5.614114in}{3.811157in}}%
\pgfpathlineto{\pgfqpoint{5.674638in}{3.815454in}}%
\pgfpathlineto{\pgfqpoint{5.754898in}{3.825719in}}%
\pgfpathlineto{\pgfqpoint{5.858495in}{3.842591in}}%
\pgfpathlineto{\pgfqpoint{5.987755in}{3.866641in}}%
\pgfpathlineto{\pgfqpoint{6.141085in}{3.898197in}}%
\pgfpathlineto{\pgfqpoint{6.309784in}{3.937359in}}%
\pgfpathlineto{\pgfqpoint{6.477132in}{3.984479in}}%
\pgfpathlineto{\pgfqpoint{6.622779in}{4.040864in}}%
\pgfpathlineto{\pgfqpoint{6.730492in}{4.108723in}}%
\pgfpathlineto{\pgfqpoint{6.793323in}{4.189812in}}%
\pgfpathlineto{\pgfqpoint{6.813740in}{4.283487in}}%
\pgfpathlineto{\pgfqpoint{6.800934in}{4.385500in}}%
\pgfpathlineto{\pgfqpoint{6.767219in}{4.488618in}}%
\pgfpathlineto{\pgfqpoint{6.724233in}{4.585060in}}%
\pgfpathlineto{\pgfqpoint{6.680188in}{4.669129in}}%
\pgfpathlineto{\pgfqpoint{6.639263in}{4.738267in}}%
\pgfpathlineto{\pgfqpoint{6.602656in}{4.792457in}}%
\pgfpathlineto{\pgfqpoint{6.569976in}{4.833030in}}%
\pgfpathlineto{\pgfqpoint{6.540220in}{4.861720in}}%
\pgfpathlineto{\pgfqpoint{6.512275in}{4.880142in}}%
\pgfpathlineto{\pgfqpoint{6.485100in}{4.889596in}}%
\pgfpathlineto{\pgfqpoint{6.457769in}{4.891023in}}%
\pgfpathlineto{\pgfqpoint{6.429458in}{4.885032in}}%
\pgfpathlineto{\pgfqpoint{6.399412in}{4.871957in}}%
\pgfpathlineto{\pgfqpoint{6.366927in}{4.851907in}}%
\pgfpathlineto{\pgfqpoint{6.331337in}{4.824830in}}%
\pgfpathlineto{\pgfqpoint{6.292027in}{4.790579in}}%
\pgfpathlineto{\pgfqpoint{6.248473in}{4.748996in}}%
\pgfpathlineto{\pgfqpoint{6.200297in}{4.700023in}}%
\pgfpathlineto{\pgfqpoint{6.147373in}{4.643827in}}%
\pgfpathlineto{\pgfqpoint{6.089922in}{4.580940in}}%
\pgfpathlineto{\pgfqpoint{6.028621in}{4.512374in}}%
\pgfpathlineto{\pgfqpoint{5.899559in}{4.364741in}}%
\pgfpathlineto{\pgfqpoint{5.773655in}{4.217071in}}%
\pgfpathlineto{\pgfqpoint{5.716413in}{4.148290in}}%
\pgfpathlineto{\pgfqpoint{5.664857in}{4.084819in}}%
\pgfpathlineto{\pgfqpoint{5.619876in}{4.027448in}}%
\pgfpathlineto{\pgfqpoint{5.581977in}{3.976484in}}%
\pgfpathlineto{\pgfqpoint{5.551393in}{3.931870in}}%
\pgfpathlineto{\pgfqpoint{5.528207in}{3.893318in}}%
\pgfpathlineto{\pgfqpoint{5.512473in}{3.860426in}}%
\pgfpathlineto{\pgfqpoint{5.504322in}{3.832758in}}%
\pgfpathlineto{\pgfqpoint{5.504044in}{3.809909in}}%
\pgfpathlineto{\pgfqpoint{5.512178in}{3.791540in}}%
\pgfpathlineto{\pgfqpoint{5.529584in}{3.777397in}}%
\pgfpathlineto{\pgfqpoint{5.557534in}{3.767330in}}%
\pgfpathlineto{\pgfqpoint{5.597778in}{3.761298in}}%
\pgfpathlineto{\pgfqpoint{5.652588in}{3.759374in}}%
\pgfpathlineto{\pgfqpoint{5.724649in}{3.761755in}}%
\pgfpathlineto{\pgfqpoint{5.816646in}{3.768767in}}%
\pgfpathlineto{\pgfqpoint{5.930257in}{3.780899in}}%
\pgfpathlineto{\pgfqpoint{6.064331in}{3.798909in}}%
\pgfpathlineto{\pgfqpoint{6.212671in}{3.824040in}}%
\pgfpathlineto{\pgfqpoint{6.362979in}{3.858311in}}%
\pgfpathlineto{\pgfqpoint{6.498983in}{3.904656in}}%
\pgfpathlineto{\pgfqpoint{6.605496in}{3.966565in}}%
\pgfpathlineto{\pgfqpoint{6.673363in}{4.046997in}}%
\pgfpathlineto{\pgfqpoint{6.701702in}{4.146512in}}%
\pgfpathlineto{\pgfqpoint{6.697319in}{4.261166in}}%
\pgfpathlineto{\pgfqpoint{6.671757in}{4.381997in}}%
\pgfpathlineto{\pgfqpoint{6.636790in}{4.497810in}}%
\pgfpathlineto{\pgfqpoint{6.600702in}{4.599670in}}%
\pgfpathlineto{\pgfqpoint{6.567414in}{4.683338in}}%
\pgfpathlineto{\pgfqpoint{6.537777in}{4.748605in}}%
\pgfpathlineto{\pgfqpoint{6.511192in}{4.797368in}}%
\pgfpathlineto{\pgfqpoint{6.486625in}{4.832098in}}%
\pgfpathlineto{\pgfqpoint{6.463042in}{4.855062in}}%
\pgfpathlineto{\pgfqpoint{6.439537in}{4.868065in}}%
\pgfpathlineto{\pgfqpoint{6.415338in}{4.872429in}}%
\pgfpathlineto{\pgfqpoint{6.389774in}{4.869054in}}%
\pgfpathlineto{\pgfqpoint{6.362240in}{4.858504in}}%
\pgfpathlineto{\pgfqpoint{6.332178in}{4.841089in}}%
\pgfpathlineto{\pgfqpoint{6.299072in}{4.816933in}}%
\pgfpathlineto{\pgfqpoint{6.262453in}{4.786054in}}%
\pgfpathlineto{\pgfqpoint{6.221930in}{4.748437in}}%
\pgfpathlineto{\pgfqpoint{6.177242in}{4.704126in}}%
\pgfpathlineto{\pgfqpoint{6.128323in}{4.653326in}}%
\pgfpathlineto{\pgfqpoint{6.075381in}{4.596502in}}%
\pgfpathlineto{\pgfqpoint{6.018961in}{4.534465in}}%
\pgfpathlineto{\pgfqpoint{5.899696in}{4.399875in}}%
\pgfpathlineto{\pgfqpoint{5.781261in}{4.262467in}}%
\pgfpathlineto{\pgfqpoint{5.726151in}{4.197043in}}%
\pgfpathlineto{\pgfqpoint{5.675507in}{4.135682in}}%
\pgfpathlineto{\pgfqpoint{5.630255in}{4.079315in}}%
\pgfpathlineto{\pgfqpoint{5.591011in}{4.028481in}}%
\pgfpathlineto{\pgfqpoint{5.558141in}{3.983385in}}%
\pgfpathlineto{\pgfqpoint{5.531843in}{3.943989in}}%
\pgfpathlineto{\pgfqpoint{5.512245in}{3.910110in}}%
\pgfpathlineto{\pgfqpoint{5.499493in}{3.881499in}}%
\pgfpathlineto{\pgfqpoint{5.493841in}{3.857908in}}%
\pgfpathlineto{\pgfqpoint{5.495729in}{3.839143in}}%
\pgfpathlineto{\pgfqpoint{5.505870in}{3.825098in}}%
\pgfpathlineto{\pgfqpoint{5.525347in}{3.815796in}}%
\pgfpathlineto{\pgfqpoint{5.555736in}{3.811417in}}%
\pgfpathlineto{\pgfqpoint{5.599222in}{3.812334in}}%
\pgfpathlineto{\pgfqpoint{5.658711in}{3.819137in}}%
\pgfpathlineto{\pgfqpoint{5.737777in}{3.832629in}}%
\pgfpathlineto{\pgfqpoint{5.840214in}{3.853751in}}%
\pgfpathlineto{\pgfqpoint{5.968679in}{3.883355in}}%
\pgfpathlineto{\pgfqpoint{6.122017in}{3.921846in}}%
\pgfpathlineto{\pgfqpoint{6.291909in}{3.968890in}}%
\pgfpathlineto{\pgfqpoint{6.461827in}{4.023717in}}%
\pgfpathlineto{\pgfqpoint{6.611523in}{4.086035in}}%
\pgfpathlineto{\pgfqpoint{6.725007in}{4.156515in}}%
\pgfpathlineto{\pgfqpoint{6.795553in}{4.235942in}}%
\pgfpathlineto{\pgfqpoint{6.825284in}{4.323702in}}%
\pgfpathlineto{\pgfqpoint{6.822100in}{4.416862in}}%
\pgfpathlineto{\pgfqpoint{6.796521in}{4.510473in}}%
\pgfpathlineto{\pgfqpoint{6.758929in}{4.598917in}}%
\pgfpathlineto{\pgfqpoint{6.717441in}{4.677576in}}%
\pgfpathlineto{\pgfqpoint{6.676996in}{4.743840in}}%
\pgfpathlineto{\pgfqpoint{6.639734in}{4.797074in}}%
\pgfpathlineto{\pgfqpoint{6.605995in}{4.837941in}}%
\pgfpathlineto{\pgfqpoint{6.575218in}{4.867673in}}%
\pgfpathlineto{\pgfqpoint{6.546513in}{4.887570in}}%
\pgfpathlineto{\pgfqpoint{6.518937in}{4.898750in}}%
\pgfpathlineto{\pgfqpoint{6.491591in}{4.902061in}}%
\pgfpathlineto{\pgfqpoint{6.463650in}{4.898066in}}%
\pgfpathlineto{\pgfqpoint{6.434337in}{4.887069in}}%
\pgfpathlineto{\pgfqpoint{6.402918in}{4.869156in}}%
\pgfpathlineto{\pgfqpoint{6.368683in}{4.844237in}}%
\pgfpathlineto{\pgfqpoint{6.330950in}{4.812105in}}%
\pgfpathlineto{\pgfqpoint{6.289098in}{4.772510in}}%
\pgfpathlineto{\pgfqpoint{6.242619in}{4.725257in}}%
\pgfpathlineto{\pgfqpoint{6.191211in}{4.670341in}}%
\pgfpathlineto{\pgfqpoint{6.134896in}{4.608098in}}%
\pgfpathlineto{\pgfqpoint{6.074147in}{4.539354in}}%
\pgfpathlineto{\pgfqpoint{5.943907in}{4.388519in}}%
\pgfpathlineto{\pgfqpoint{5.813993in}{4.234432in}}%
\pgfpathlineto{\pgfqpoint{5.754174in}{4.161841in}}%
\pgfpathlineto{\pgfqpoint{5.700017in}{4.094536in}}%
\pgfpathlineto{\pgfqpoint{5.652641in}{4.033515in}}%
\pgfpathlineto{\pgfqpoint{5.612721in}{3.979209in}}%
\pgfpathlineto{\pgfqpoint{5.580589in}{3.931603in}}%
\pgfpathlineto{\pgfqpoint{5.556377in}{3.890388in}}%
\pgfpathlineto{\pgfqpoint{5.540152in}{3.855103in}}%
\pgfpathlineto{\pgfqpoint{5.532034in}{3.825227in}}%
\pgfpathlineto{\pgfqpoint{5.532289in}{3.800255in}}%
\pgfpathlineto{\pgfqpoint{5.541409in}{3.779732in}}%
\pgfpathlineto{\pgfqpoint{5.560172in}{3.763273in}}%
\pgfpathlineto{\pgfqpoint{5.589681in}{3.750569in}}%
\pgfpathlineto{\pgfqpoint{5.631367in}{3.741385in}}%
\pgfpathlineto{\pgfqpoint{5.686882in}{3.735556in}}%
\pgfpathlineto{\pgfqpoint{5.757805in}{3.732991in}}%
\pgfpathlineto{\pgfqpoint{5.844997in}{3.733709in}}%
\pgfpathlineto{\pgfqpoint{5.947532in}{3.737931in}}%
\pgfpathlineto{\pgfqpoint{6.061432in}{3.746287in}}%
\pgfpathlineto{\pgfqpoint{6.179059in}{3.760080in}}%
\pgfpathlineto{\pgfqpoint{6.290289in}{3.781478in}}%
\pgfpathlineto{\pgfqpoint{6.385432in}{3.813423in}}%
\pgfpathlineto{\pgfqpoint{6.458036in}{3.859264in}}%
\pgfpathlineto{\pgfqpoint{6.505841in}{3.922248in}}%
\pgfpathlineto{\pgfqpoint{6.530271in}{4.004706in}}%
\pgfpathlineto{\pgfqpoint{6.535579in}{4.106611in}}%
\pgfpathlineto{\pgfqpoint{6.527817in}{4.223773in}}%
\pgfpathlineto{\pgfqpoint{6.513265in}{4.347289in}}%
\pgfpathlineto{\pgfqpoint{6.496680in}{4.465988in}}%
\pgfpathlineto{\pgfqpoint{6.480484in}{4.570706in}}%
\pgfpathlineto{\pgfqpoint{6.465236in}{4.656892in}}%
\pgfpathlineto{\pgfqpoint{6.450576in}{4.724166in}}%
\pgfpathlineto{\pgfqpoint{6.435900in}{4.774424in}}%
\pgfpathlineto{\pgfqpoint{6.420636in}{4.810235in}}%
\pgfpathlineto{\pgfqpoint{6.404306in}{4.833993in}}%
\pgfpathlineto{\pgfqpoint{6.386502in}{4.847621in}}%
\pgfpathlineto{\pgfqpoint{6.366860in}{4.852545in}}%
\pgfpathlineto{\pgfqpoint{6.345036in}{4.849763in}}%
\pgfpathlineto{\pgfqpoint{6.320691in}{4.839932in}}%
\pgfpathlineto{\pgfqpoint{6.293494in}{4.823459in}}%
\pgfpathlineto{\pgfqpoint{6.263126in}{4.800573in}}%
\pgfpathlineto{\pgfqpoint{6.229299in}{4.771401in}}%
\pgfpathlineto{\pgfqpoint{6.191790in}{4.736036in}}%
\pgfpathlineto{\pgfqpoint{6.150480in}{4.694618in}}%
\pgfpathlineto{\pgfqpoint{6.105410in}{4.647407in}}%
\pgfpathlineto{\pgfqpoint{6.056833in}{4.594865in}}%
\pgfpathlineto{\pgfqpoint{6.005256in}{4.537705in}}%
\pgfpathlineto{\pgfqpoint{5.896490in}{4.413787in}}%
\pgfpathlineto{\pgfqpoint{5.787950in}{4.286241in}}%
\pgfpathlineto{\pgfqpoint{5.736939in}{4.224756in}}%
\pgfpathlineto{\pgfqpoint{5.689635in}{4.166489in}}%
\pgfpathlineto{\pgfqpoint{5.646944in}{4.112378in}}%
\pgfpathlineto{\pgfqpoint{5.609543in}{4.063059in}}%
\pgfpathlineto{\pgfqpoint{5.577911in}{4.018885in}}%
\pgfpathlineto{\pgfqpoint{5.552380in}{3.979987in}}%
\pgfpathlineto{\pgfqpoint{5.533216in}{3.946346in}}%
\pgfpathlineto{\pgfqpoint{5.520690in}{3.917861in}}%
\pgfpathlineto{\pgfqpoint{5.515168in}{3.894412in}}%
\pgfpathlineto{\pgfqpoint{5.517184in}{3.875916in}}%
\pgfpathlineto{\pgfqpoint{5.527539in}{3.862367in}}%
\pgfpathlineto{\pgfqpoint{5.547396in}{3.853879in}}%
\pgfpathlineto{\pgfqpoint{5.578403in}{3.850720in}}%
\pgfpathlineto{\pgfqpoint{5.622815in}{3.853339in}}%
\pgfpathlineto{\pgfqpoint{5.683568in}{3.862369in}}%
\pgfpathlineto{\pgfqpoint{5.764188in}{3.878572in}}%
\pgfpathlineto{\pgfqpoint{5.868245in}{3.902658in}}%
\pgfpathlineto{\pgfqpoint{5.997911in}{3.934916in}}%
\pgfpathlineto{\pgfqpoint{6.151369in}{3.974730in}}%
\pgfpathlineto{\pgfqpoint{6.319994in}{4.020402in}}%
\pgfpathlineto{\pgfqpoint{6.488057in}{4.069909in}}%
\pgfpathlineto{\pgfqpoint{6.637144in}{4.122407in}}%
\pgfpathlineto{\pgfqpoint{6.752761in}{4.179037in}}%
\pgfpathlineto{\pgfqpoint{6.828034in}{4.242110in}}%
\pgfpathlineto{\pgfqpoint{6.863262in}{4.313420in}}%
\pgfpathlineto{\pgfqpoint{6.863877in}{4.392790in}}%
\pgfpathlineto{\pgfqpoint{6.838621in}{4.477387in}}%
\pgfpathlineto{\pgfqpoint{6.838621in}{4.477387in}}%
\pgfusepath{stroke}%
\end{pgfscope}%
\begin{pgfscope}%
\pgfpathrectangle{\pgfqpoint{4.577333in}{3.075000in}}{\pgfqpoint{4.224218in}{2.565000in}}%
\pgfusepath{clip}%
\pgfsetbuttcap%
\pgfsetroundjoin%
\pgfsetlinewidth{1.003750pt}%
\definecolor{currentstroke}{rgb}{0.501961,0.501961,0.501961}%
\pgfsetstrokecolor{currentstroke}%
\pgfsetdash{{3.700000pt}{1.600000pt}}{0.000000pt}%
\pgfpathmoveto{\pgfqpoint{6.752865in}{4.157031in}}%
\pgfpathlineto{\pgfqpoint{6.774472in}{4.256270in}}%
\pgfpathlineto{\pgfqpoint{6.768931in}{4.310385in}}%
\pgfpathlineto{\pgfqpoint{6.756658in}{4.364704in}}%
\pgfpathlineto{\pgfqpoint{6.740549in}{4.417072in}}%
\pgfpathlineto{\pgfqpoint{6.722751in}{4.465880in}}%
\pgfpathlineto{\pgfqpoint{6.704605in}{4.510209in}}%
\pgfpathlineto{\pgfqpoint{6.686820in}{4.549723in}}%
\pgfpathlineto{\pgfqpoint{6.669689in}{4.584460in}}%
\pgfpathlineto{\pgfqpoint{6.653264in}{4.614660in}}%
\pgfpathlineto{\pgfqpoint{6.637473in}{4.640642in}}%
\pgfpathlineto{\pgfqpoint{6.622181in}{4.662734in}}%
\pgfpathlineto{\pgfqpoint{6.607228in}{4.681236in}}%
\pgfpathlineto{\pgfqpoint{6.592444in}{4.696406in}}%
\pgfpathlineto{\pgfqpoint{6.577656in}{4.708452in}}%
\pgfpathlineto{\pgfqpoint{6.562696in}{4.717538in}}%
\pgfpathlineto{\pgfqpoint{6.547398in}{4.723783in}}%
\pgfpathlineto{\pgfqpoint{6.531602in}{4.727273in}}%
\pgfpathlineto{\pgfqpoint{6.515159in}{4.728068in}}%
\pgfpathlineto{\pgfqpoint{6.497933in}{4.726215in}}%
\pgfpathlineto{\pgfqpoint{6.479813in}{4.721763in}}%
\pgfpathlineto{\pgfqpoint{6.460722in}{4.714778in}}%
\pgfpathlineto{\pgfqpoint{6.440627in}{4.705361in}}%
\pgfpathlineto{\pgfqpoint{6.419554in}{4.693659in}}%
\pgfpathlineto{\pgfqpoint{6.397588in}{4.679875in}}%
\pgfpathlineto{\pgfqpoint{6.374871in}{4.664263in}}%
\pgfpathlineto{\pgfqpoint{6.351594in}{4.647115in}}%
\pgfpathlineto{\pgfqpoint{6.304258in}{4.609470in}}%
\pgfpathlineto{\pgfqpoint{6.257418in}{4.569357in}}%
\pgfpathlineto{\pgfqpoint{6.212727in}{4.528772in}}%
\pgfpathlineto{\pgfqpoint{6.171535in}{4.489149in}}%
\pgfpathlineto{\pgfqpoint{6.134987in}{4.451446in}}%
\pgfpathlineto{\pgfqpoint{6.104258in}{4.416300in}}%
\pgfpathlineto{\pgfqpoint{6.091524in}{4.399835in}}%
\pgfpathlineto{\pgfqpoint{6.080851in}{4.384181in}}%
\pgfpathlineto{\pgfqpoint{6.072527in}{4.369394in}}%
\pgfpathlineto{\pgfqpoint{6.066878in}{4.355532in}}%
\pgfpathlineto{\pgfqpoint{6.064244in}{4.342664in}}%
\pgfpathlineto{\pgfqpoint{6.064896in}{4.330868in}}%
\pgfpathlineto{\pgfqpoint{6.068928in}{4.320243in}}%
\pgfpathlineto{\pgfqpoint{6.076126in}{4.310920in}}%
\pgfpathlineto{\pgfqpoint{6.085924in}{4.303077in}}%
\pgfpathlineto{\pgfqpoint{6.097478in}{4.296945in}}%
\pgfpathlineto{\pgfqpoint{6.109845in}{4.292783in}}%
\pgfpathlineto{\pgfqpoint{6.122171in}{4.290820in}}%
\pgfpathlineto{\pgfqpoint{6.133829in}{4.291168in}}%
\pgfpathlineto{\pgfqpoint{6.144479in}{4.293749in}}%
\pgfpathlineto{\pgfqpoint{6.154023in}{4.298297in}}%
\pgfpathlineto{\pgfqpoint{6.162517in}{4.304428in}}%
\pgfpathlineto{\pgfqpoint{6.170072in}{4.311735in}}%
\pgfpathlineto{\pgfqpoint{6.176807in}{4.319843in}}%
\pgfpathlineto{\pgfqpoint{6.188178in}{4.337276in}}%
\pgfpathlineto{\pgfqpoint{6.197128in}{4.354900in}}%
\pgfpathlineto{\pgfqpoint{6.203856in}{4.371535in}}%
\pgfpathlineto{\pgfqpoint{6.208357in}{4.386346in}}%
\pgfpathlineto{\pgfqpoint{6.210485in}{4.398665in}}%
\pgfpathlineto{\pgfqpoint{6.210001in}{4.407906in}}%
\pgfpathlineto{\pgfqpoint{6.208697in}{4.411212in}}%
\pgfpathlineto{\pgfqpoint{6.206648in}{4.413575in}}%
\pgfpathlineto{\pgfqpoint{6.203838in}{4.414963in}}%
\pgfpathlineto{\pgfqpoint{6.200267in}{4.415367in}}%
\pgfpathlineto{\pgfqpoint{6.195955in}{4.414799in}}%
\pgfpathlineto{\pgfqpoint{6.185285in}{4.410923in}}%
\pgfpathlineto{\pgfqpoint{6.172362in}{4.403882in}}%
\pgfpathlineto{\pgfqpoint{6.157898in}{4.394411in}}%
\pgfpathlineto{\pgfqpoint{6.134953in}{4.377267in}}%
\pgfpathlineto{\pgfqpoint{6.112642in}{4.358453in}}%
\pgfpathlineto{\pgfqpoint{6.099325in}{4.345810in}}%
\pgfpathlineto{\pgfqpoint{6.088295in}{4.333591in}}%
\pgfpathlineto{\pgfqpoint{6.080820in}{4.322251in}}%
\pgfpathlineto{\pgfqpoint{6.078938in}{4.317074in}}%
\pgfpathlineto{\pgfqpoint{6.078597in}{4.312322in}}%
\pgfpathlineto{\pgfqpoint{6.079969in}{4.308072in}}%
\pgfpathlineto{\pgfqpoint{6.083086in}{4.304401in}}%
\pgfpathlineto{\pgfqpoint{6.087782in}{4.301388in}}%
\pgfpathlineto{\pgfqpoint{6.093699in}{4.299123in}}%
\pgfpathlineto{\pgfqpoint{6.100378in}{4.297708in}}%
\pgfpathlineto{\pgfqpoint{6.107354in}{4.297239in}}%
\pgfpathlineto{\pgfqpoint{6.114247in}{4.297779in}}%
\pgfpathlineto{\pgfqpoint{6.120797in}{4.299328in}}%
\pgfpathlineto{\pgfqpoint{6.126872in}{4.301815in}}%
\pgfpathlineto{\pgfqpoint{6.132436in}{4.305110in}}%
\pgfpathlineto{\pgfqpoint{6.142137in}{4.313495in}}%
\pgfpathlineto{\pgfqpoint{6.150235in}{4.323290in}}%
\pgfpathlineto{\pgfqpoint{6.157002in}{4.333571in}}%
\pgfpathlineto{\pgfqpoint{6.164885in}{4.348494in}}%
\pgfpathlineto{\pgfqpoint{6.169922in}{4.361350in}}%
\pgfpathlineto{\pgfqpoint{6.171443in}{4.368041in}}%
\pgfpathlineto{\pgfqpoint{6.171232in}{4.372740in}}%
\pgfpathlineto{\pgfqpoint{6.169083in}{4.375105in}}%
\pgfpathlineto{\pgfqpoint{6.164917in}{4.374978in}}%
\pgfpathlineto{\pgfqpoint{6.158877in}{4.372472in}}%
\pgfpathlineto{\pgfqpoint{6.147125in}{4.365058in}}%
\pgfpathlineto{\pgfqpoint{6.133613in}{4.354656in}}%
\pgfpathlineto{\pgfqpoint{6.120024in}{4.342645in}}%
\pgfpathlineto{\pgfqpoint{6.108113in}{4.330052in}}%
\pgfpathlineto{\pgfqpoint{6.102190in}{4.321728in}}%
\pgfpathlineto{\pgfqpoint{6.098884in}{4.313712in}}%
\pgfpathlineto{\pgfqpoint{6.098493in}{4.309890in}}%
\pgfpathlineto{\pgfqpoint{6.099034in}{4.306240in}}%
\pgfpathlineto{\pgfqpoint{6.100485in}{4.302812in}}%
\pgfpathlineto{\pgfqpoint{6.105610in}{4.296923in}}%
\pgfpathlineto{\pgfqpoint{6.112295in}{4.293029in}}%
\pgfpathlineto{\pgfqpoint{6.118927in}{4.292012in}}%
\pgfpathlineto{\pgfqpoint{6.121947in}{4.292701in}}%
\pgfpathlineto{\pgfqpoint{6.127316in}{4.296142in}}%
\pgfpathlineto{\pgfqpoint{6.131916in}{4.301499in}}%
\pgfpathlineto{\pgfqpoint{6.137611in}{4.311180in}}%
\pgfpathlineto{\pgfqpoint{6.141870in}{4.321082in}}%
\pgfpathlineto{\pgfqpoint{6.144446in}{4.329903in}}%
\pgfpathlineto{\pgfqpoint{6.144915in}{4.336635in}}%
\pgfpathlineto{\pgfqpoint{6.143824in}{4.339547in}}%
\pgfpathlineto{\pgfqpoint{6.141498in}{4.340997in}}%
\pgfpathlineto{\pgfqpoint{6.137939in}{4.340943in}}%
\pgfpathlineto{\pgfqpoint{6.133265in}{4.339476in}}%
\pgfpathlineto{\pgfqpoint{6.124645in}{4.335084in}}%
\pgfpathlineto{\pgfqpoint{6.111619in}{4.326490in}}%
\pgfpathlineto{\pgfqpoint{6.099112in}{4.316568in}}%
\pgfpathlineto{\pgfqpoint{6.092052in}{4.309399in}}%
\pgfpathlineto{\pgfqpoint{6.089708in}{4.305291in}}%
\pgfpathlineto{\pgfqpoint{6.089588in}{4.303567in}}%
\pgfpathlineto{\pgfqpoint{6.090297in}{4.302112in}}%
\pgfpathlineto{\pgfqpoint{6.094210in}{4.300071in}}%
\pgfpathlineto{\pgfqpoint{6.100667in}{4.299208in}}%
\pgfpathlineto{\pgfqpoint{6.108201in}{4.299608in}}%
\pgfpathlineto{\pgfqpoint{6.115518in}{4.301436in}}%
\pgfpathlineto{\pgfqpoint{6.121942in}{4.304692in}}%
\pgfpathlineto{\pgfqpoint{6.127374in}{4.309072in}}%
\pgfpathlineto{\pgfqpoint{6.133979in}{4.316785in}}%
\pgfpathlineto{\pgfqpoint{6.140398in}{4.327269in}}%
\pgfpathlineto{\pgfqpoint{6.143934in}{4.335875in}}%
\pgfpathlineto{\pgfqpoint{6.144147in}{4.339832in}}%
\pgfpathlineto{\pgfqpoint{6.142876in}{4.340862in}}%
\pgfpathlineto{\pgfqpoint{6.140427in}{4.340507in}}%
\pgfpathlineto{\pgfqpoint{6.134814in}{4.337611in}}%
\pgfpathlineto{\pgfqpoint{6.125004in}{4.330555in}}%
\pgfpathlineto{\pgfqpoint{6.114562in}{4.321522in}}%
\pgfpathlineto{\pgfqpoint{6.107714in}{4.314268in}}%
\pgfpathlineto{\pgfqpoint{6.104325in}{4.309454in}}%
\pgfpathlineto{\pgfqpoint{6.102474in}{4.304813in}}%
\pgfpathlineto{\pgfqpoint{6.102646in}{4.300517in}}%
\pgfpathlineto{\pgfqpoint{6.104904in}{4.296844in}}%
\pgfpathlineto{\pgfqpoint{6.108616in}{4.294258in}}%
\pgfpathlineto{\pgfqpoint{6.112782in}{4.293339in}}%
\pgfpathlineto{\pgfqpoint{6.116683in}{4.294384in}}%
\pgfpathlineto{\pgfqpoint{6.120121in}{4.297071in}}%
\pgfpathlineto{\pgfqpoint{6.124511in}{4.302867in}}%
\pgfpathlineto{\pgfqpoint{6.129071in}{4.311588in}}%
\pgfpathlineto{\pgfqpoint{6.131907in}{4.319510in}}%
\pgfpathlineto{\pgfqpoint{6.132507in}{4.323972in}}%
\pgfpathlineto{\pgfqpoint{6.131408in}{4.326516in}}%
\pgfpathlineto{\pgfqpoint{6.129622in}{4.326961in}}%
\pgfpathlineto{\pgfqpoint{6.125448in}{4.325781in}}%
\pgfpathlineto{\pgfqpoint{6.117836in}{4.321449in}}%
\pgfpathlineto{\pgfqpoint{6.107226in}{4.313501in}}%
\pgfpathlineto{\pgfqpoint{6.100172in}{4.306568in}}%
\pgfpathlineto{\pgfqpoint{6.098148in}{4.303310in}}%
\pgfpathlineto{\pgfqpoint{6.097917in}{4.300427in}}%
\pgfpathlineto{\pgfqpoint{6.099928in}{4.298118in}}%
\pgfpathlineto{\pgfqpoint{6.103825in}{4.296609in}}%
\pgfpathlineto{\pgfqpoint{6.108521in}{4.296206in}}%
\pgfpathlineto{\pgfqpoint{6.113024in}{4.297145in}}%
\pgfpathlineto{\pgfqpoint{6.116926in}{4.299331in}}%
\pgfpathlineto{\pgfqpoint{6.121717in}{4.304136in}}%
\pgfpathlineto{\pgfqpoint{6.126599in}{4.311599in}}%
\pgfpathlineto{\pgfqpoint{6.130376in}{4.320090in}}%
\pgfpathlineto{\pgfqpoint{6.130894in}{4.324472in}}%
\pgfpathlineto{\pgfqpoint{6.130055in}{4.325419in}}%
\pgfpathlineto{\pgfqpoint{6.128423in}{4.325431in}}%
\pgfpathlineto{\pgfqpoint{6.124638in}{4.323812in}}%
\pgfpathlineto{\pgfqpoint{6.116064in}{4.317965in}}%
\pgfpathlineto{\pgfqpoint{6.107190in}{4.310171in}}%
\pgfpathlineto{\pgfqpoint{6.103207in}{4.305295in}}%
\pgfpathlineto{\pgfqpoint{6.101855in}{4.302175in}}%
\pgfpathlineto{\pgfqpoint{6.102069in}{4.299305in}}%
\pgfpathlineto{\pgfqpoint{6.103982in}{4.296876in}}%
\pgfpathlineto{\pgfqpoint{6.107077in}{4.295199in}}%
\pgfpathlineto{\pgfqpoint{6.110482in}{4.294674in}}%
\pgfpathlineto{\pgfqpoint{6.113589in}{4.295504in}}%
\pgfpathlineto{\pgfqpoint{6.117456in}{4.298748in}}%
\pgfpathlineto{\pgfqpoint{6.121463in}{4.304792in}}%
\pgfpathlineto{\pgfqpoint{6.124754in}{4.312344in}}%
\pgfpathlineto{\pgfqpoint{6.125549in}{4.316832in}}%
\pgfpathlineto{\pgfqpoint{6.124676in}{4.318591in}}%
\pgfpathlineto{\pgfqpoint{6.122381in}{4.318691in}}%
\pgfpathlineto{\pgfqpoint{6.117417in}{4.316472in}}%
\pgfpathlineto{\pgfqpoint{6.109565in}{4.311208in}}%
\pgfpathlineto{\pgfqpoint{6.102140in}{4.304855in}}%
\pgfpathlineto{\pgfqpoint{6.099391in}{4.301233in}}%
\pgfpathlineto{\pgfqpoint{6.099114in}{4.299194in}}%
\pgfpathlineto{\pgfqpoint{6.100528in}{4.297653in}}%
\pgfpathlineto{\pgfqpoint{6.103402in}{4.296760in}}%
\pgfpathlineto{\pgfqpoint{6.108761in}{4.296976in}}%
\pgfpathlineto{\pgfqpoint{6.113615in}{4.299211in}}%
\pgfpathlineto{\pgfqpoint{6.117531in}{4.302877in}}%
\pgfpathlineto{\pgfqpoint{6.122441in}{4.310065in}}%
\pgfpathlineto{\pgfqpoint{6.125655in}{4.317414in}}%
\pgfpathlineto{\pgfqpoint{6.125435in}{4.319833in}}%
\pgfpathlineto{\pgfqpoint{6.123680in}{4.319772in}}%
\pgfpathlineto{\pgfqpoint{6.119513in}{4.317401in}}%
\pgfpathlineto{\pgfqpoint{6.111563in}{4.310868in}}%
\pgfpathlineto{\pgfqpoint{6.107026in}{4.305702in}}%
\pgfpathlineto{\pgfqpoint{6.105271in}{4.301801in}}%
\pgfpathlineto{\pgfqpoint{6.105339in}{4.299313in}}%
\pgfpathlineto{\pgfqpoint{6.107180in}{4.296061in}}%
\pgfpathlineto{\pgfqpoint{6.109977in}{4.294149in}}%
\pgfpathlineto{\pgfqpoint{6.111806in}{4.294111in}}%
\pgfpathlineto{\pgfqpoint{6.114255in}{4.295906in}}%
\pgfpathlineto{\pgfqpoint{6.117006in}{4.300202in}}%
\pgfpathlineto{\pgfqpoint{6.119763in}{4.307156in}}%
\pgfpathlineto{\pgfqpoint{6.120052in}{4.311243in}}%
\pgfpathlineto{\pgfqpoint{6.118973in}{4.312295in}}%
\pgfpathlineto{\pgfqpoint{6.116879in}{4.312134in}}%
\pgfpathlineto{\pgfqpoint{6.111632in}{4.309582in}}%
\pgfpathlineto{\pgfqpoint{6.104383in}{4.304403in}}%
\pgfpathlineto{\pgfqpoint{6.100284in}{4.299957in}}%
\pgfpathlineto{\pgfqpoint{6.100179in}{4.298551in}}%
\pgfpathlineto{\pgfqpoint{6.101412in}{4.297523in}}%
\pgfpathlineto{\pgfqpoint{6.105270in}{4.296819in}}%
\pgfpathlineto{\pgfqpoint{6.109779in}{4.297407in}}%
\pgfpathlineto{\pgfqpoint{6.109779in}{4.297407in}}%
\pgfusepath{stroke}%
\end{pgfscope}%
\begin{pgfscope}%
\pgfpathrectangle{\pgfqpoint{4.577333in}{3.075000in}}{\pgfqpoint{4.224218in}{2.565000in}}%
\pgfusepath{clip}%
\pgfsetbuttcap%
\pgfsetroundjoin%
\definecolor{currentfill}{rgb}{0.501961,0.000000,0.501961}%
\pgfsetfillcolor{currentfill}%
\pgfsetlinewidth{1.505625pt}%
\definecolor{currentstroke}{rgb}{0.501961,0.000000,0.501961}%
\pgfsetstrokecolor{currentstroke}%
\pgfsetdash{}{0pt}%
\pgfsys@defobject{currentmarker}{\pgfqpoint{-0.017010in}{-0.017010in}}{\pgfqpoint{0.017010in}{0.017010in}}{%
\pgfpathmoveto{\pgfqpoint{-0.017010in}{0.000000in}}%
\pgfpathlineto{\pgfqpoint{0.017010in}{0.000000in}}%
\pgfpathmoveto{\pgfqpoint{0.000000in}{-0.017010in}}%
\pgfpathlineto{\pgfqpoint{0.000000in}{0.017010in}}%
\pgfusepath{stroke,fill}%
}%
\begin{pgfscope}%
\pgfsys@transformshift{6.752865in}{4.157031in}%
\pgfsys@useobject{currentmarker}{}%
\end{pgfscope}%
\end{pgfscope}%
\begin{pgfscope}%
\pgfpathrectangle{\pgfqpoint{4.577333in}{3.075000in}}{\pgfqpoint{4.224218in}{2.565000in}}%
\pgfusepath{clip}%
\pgfsetbuttcap%
\pgfsetroundjoin%
\definecolor{currentfill}{rgb}{0.501961,0.000000,0.501961}%
\pgfsetfillcolor{currentfill}%
\pgfsetlinewidth{1.003750pt}%
\definecolor{currentstroke}{rgb}{0.501961,0.000000,0.501961}%
\pgfsetstrokecolor{currentstroke}%
\pgfsetdash{}{0pt}%
\pgfsys@defobject{currentmarker}{\pgfqpoint{-0.016178in}{-0.013762in}}{\pgfqpoint{0.016178in}{0.017010in}}{%
\pgfpathmoveto{\pgfqpoint{0.000000in}{0.017010in}}%
\pgfpathlineto{\pgfqpoint{-0.003819in}{0.005256in}}%
\pgfpathlineto{\pgfqpoint{-0.016178in}{0.005256in}}%
\pgfpathlineto{\pgfqpoint{-0.006179in}{-0.002008in}}%
\pgfpathlineto{\pgfqpoint{-0.009998in}{-0.013762in}}%
\pgfpathlineto{\pgfqpoint{-0.000000in}{-0.006497in}}%
\pgfpathlineto{\pgfqpoint{0.009998in}{-0.013762in}}%
\pgfpathlineto{\pgfqpoint{0.006179in}{-0.002008in}}%
\pgfpathlineto{\pgfqpoint{0.016178in}{0.005256in}}%
\pgfpathlineto{\pgfqpoint{0.003819in}{0.005256in}}%
\pgfpathlineto{\pgfqpoint{0.000000in}{0.017010in}}%
\pgfpathclose%
\pgfusepath{stroke,fill}%
}%
\begin{pgfscope}%
\pgfsys@transformshift{6.838621in}{4.477387in}%
\pgfsys@useobject{currentmarker}{}%
\end{pgfscope}%
\end{pgfscope}%
\begin{pgfscope}%
\pgfpathrectangle{\pgfqpoint{4.577333in}{3.075000in}}{\pgfqpoint{4.224218in}{2.565000in}}%
\pgfusepath{clip}%
\pgfsetbuttcap%
\pgfsetroundjoin%
\definecolor{currentfill}{rgb}{0.000000,0.000000,0.000000}%
\pgfsetfillcolor{currentfill}%
\pgfsetlinewidth{1.505625pt}%
\definecolor{currentstroke}{rgb}{0.000000,0.000000,0.000000}%
\pgfsetstrokecolor{currentstroke}%
\pgfsetdash{}{0pt}%
\pgfsys@defobject{currentmarker}{\pgfqpoint{-0.017010in}{-0.017010in}}{\pgfqpoint{0.017010in}{0.017010in}}{%
\pgfpathmoveto{\pgfqpoint{-0.017010in}{0.000000in}}%
\pgfpathlineto{\pgfqpoint{0.017010in}{0.000000in}}%
\pgfpathmoveto{\pgfqpoint{0.000000in}{-0.017010in}}%
\pgfpathlineto{\pgfqpoint{0.000000in}{0.017010in}}%
\pgfusepath{stroke,fill}%
}%
\begin{pgfscope}%
\pgfsys@transformshift{6.752865in}{4.157031in}%
\pgfsys@useobject{currentmarker}{}%
\end{pgfscope}%
\end{pgfscope}%
\begin{pgfscope}%
\pgfpathrectangle{\pgfqpoint{4.577333in}{3.075000in}}{\pgfqpoint{4.224218in}{2.565000in}}%
\pgfusepath{clip}%
\pgfsetbuttcap%
\pgfsetroundjoin%
\definecolor{currentfill}{rgb}{0.000000,0.000000,0.000000}%
\pgfsetfillcolor{currentfill}%
\pgfsetlinewidth{1.003750pt}%
\definecolor{currentstroke}{rgb}{0.000000,0.000000,0.000000}%
\pgfsetstrokecolor{currentstroke}%
\pgfsetdash{}{0pt}%
\pgfsys@defobject{currentmarker}{\pgfqpoint{-0.016178in}{-0.013762in}}{\pgfqpoint{0.016178in}{0.017010in}}{%
\pgfpathmoveto{\pgfqpoint{0.000000in}{0.017010in}}%
\pgfpathlineto{\pgfqpoint{-0.003819in}{0.005256in}}%
\pgfpathlineto{\pgfqpoint{-0.016178in}{0.005256in}}%
\pgfpathlineto{\pgfqpoint{-0.006179in}{-0.002008in}}%
\pgfpathlineto{\pgfqpoint{-0.009998in}{-0.013762in}}%
\pgfpathlineto{\pgfqpoint{-0.000000in}{-0.006497in}}%
\pgfpathlineto{\pgfqpoint{0.009998in}{-0.013762in}}%
\pgfpathlineto{\pgfqpoint{0.006179in}{-0.002008in}}%
\pgfpathlineto{\pgfqpoint{0.016178in}{0.005256in}}%
\pgfpathlineto{\pgfqpoint{0.003819in}{0.005256in}}%
\pgfpathlineto{\pgfqpoint{0.000000in}{0.017010in}}%
\pgfpathclose%
\pgfusepath{stroke,fill}%
}%
\begin{pgfscope}%
\pgfsys@transformshift{6.109779in}{4.297407in}%
\pgfsys@useobject{currentmarker}{}%
\end{pgfscope}%
\end{pgfscope}%
\begin{pgfscope}%
\pgfsetrectcap%
\pgfsetmiterjoin%
\pgfsetlinewidth{0.803000pt}%
\definecolor{currentstroke}{rgb}{0.000000,0.000000,0.000000}%
\pgfsetstrokecolor{currentstroke}%
\pgfsetdash{}{0pt}%
\pgfpathmoveto{\pgfqpoint{4.577333in}{3.075000in}}%
\pgfpathlineto{\pgfqpoint{4.577333in}{5.640000in}}%
\pgfusepath{stroke}%
\end{pgfscope}%
\begin{pgfscope}%
\pgfsetrectcap%
\pgfsetmiterjoin%
\pgfsetlinewidth{0.803000pt}%
\definecolor{currentstroke}{rgb}{0.000000,0.000000,0.000000}%
\pgfsetstrokecolor{currentstroke}%
\pgfsetdash{}{0pt}%
\pgfpathmoveto{\pgfqpoint{8.801551in}{3.075000in}}%
\pgfpathlineto{\pgfqpoint{8.801551in}{5.640000in}}%
\pgfusepath{stroke}%
\end{pgfscope}%
\begin{pgfscope}%
\pgfsetrectcap%
\pgfsetmiterjoin%
\pgfsetlinewidth{0.803000pt}%
\definecolor{currentstroke}{rgb}{0.000000,0.000000,0.000000}%
\pgfsetstrokecolor{currentstroke}%
\pgfsetdash{}{0pt}%
\pgfpathmoveto{\pgfqpoint{4.577333in}{3.075000in}}%
\pgfpathlineto{\pgfqpoint{8.801551in}{3.075000in}}%
\pgfusepath{stroke}%
\end{pgfscope}%
\begin{pgfscope}%
\pgfsetrectcap%
\pgfsetmiterjoin%
\pgfsetlinewidth{0.803000pt}%
\definecolor{currentstroke}{rgb}{0.000000,0.000000,0.000000}%
\pgfsetstrokecolor{currentstroke}%
\pgfsetdash{}{0pt}%
\pgfpathmoveto{\pgfqpoint{4.577333in}{5.640000in}}%
\pgfpathlineto{\pgfqpoint{8.801551in}{5.640000in}}%
\pgfusepath{stroke}%
\end{pgfscope}%
\begin{pgfscope}%
\definecolor{textcolor}{rgb}{0.000000,0.000000,0.000000}%
\pgfsetstrokecolor{textcolor}%
\pgfsetfillcolor{textcolor}%
\pgftext[x=8.718987in,y=3.320393in,,base]{\color{textcolor}\sffamily\fontsize{10.000000}{12.000000}\selectfont 0.0}%
\end{pgfscope}%
\begin{pgfscope}%
\definecolor{textcolor}{rgb}{0.000000,0.000000,0.000000}%
\pgfsetstrokecolor{textcolor}%
\pgfsetfillcolor{textcolor}%
\pgftext[x=8.526977in,y=3.542529in,,base]{\color{textcolor}\sffamily\fontsize{10.000000}{12.000000}\selectfont 0.1}%
\end{pgfscope}%
\begin{pgfscope}%
\definecolor{textcolor}{rgb}{0.000000,0.000000,0.000000}%
\pgfsetstrokecolor{textcolor}%
\pgfsetfillcolor{textcolor}%
\pgftext[x=8.334967in,y=3.764664in,,base]{\color{textcolor}\sffamily\fontsize{10.000000}{12.000000}\selectfont 0.2}%
\end{pgfscope}%
\begin{pgfscope}%
\definecolor{textcolor}{rgb}{0.000000,0.000000,0.000000}%
\pgfsetstrokecolor{textcolor}%
\pgfsetfillcolor{textcolor}%
\pgftext[x=8.142957in,y=3.986800in,,base]{\color{textcolor}\sffamily\fontsize{10.000000}{12.000000}\selectfont 0.3}%
\end{pgfscope}%
\begin{pgfscope}%
\definecolor{textcolor}{rgb}{0.000000,0.000000,0.000000}%
\pgfsetstrokecolor{textcolor}%
\pgfsetfillcolor{textcolor}%
\pgftext[x=7.950947in,y=4.208935in,,base]{\color{textcolor}\sffamily\fontsize{10.000000}{12.000000}\selectfont 0.4}%
\end{pgfscope}%
\begin{pgfscope}%
\definecolor{textcolor}{rgb}{0.000000,0.000000,0.000000}%
\pgfsetstrokecolor{textcolor}%
\pgfsetfillcolor{textcolor}%
\pgftext[x=7.758937in,y=4.431071in,,base]{\color{textcolor}\sffamily\fontsize{10.000000}{12.000000}\selectfont 0.5}%
\end{pgfscope}%
\begin{pgfscope}%
\definecolor{textcolor}{rgb}{0.000000,0.000000,0.000000}%
\pgfsetstrokecolor{textcolor}%
\pgfsetfillcolor{textcolor}%
\pgftext[x=7.566927in,y=4.653206in,,base]{\color{textcolor}\sffamily\fontsize{10.000000}{12.000000}\selectfont 0.6}%
\end{pgfscope}%
\begin{pgfscope}%
\definecolor{textcolor}{rgb}{0.000000,0.000000,0.000000}%
\pgfsetstrokecolor{textcolor}%
\pgfsetfillcolor{textcolor}%
\pgftext[x=7.374918in,y=4.875342in,,base]{\color{textcolor}\sffamily\fontsize{10.000000}{12.000000}\selectfont 0.7}%
\end{pgfscope}%
\begin{pgfscope}%
\definecolor{textcolor}{rgb}{0.000000,0.000000,0.000000}%
\pgfsetstrokecolor{textcolor}%
\pgfsetfillcolor{textcolor}%
\pgftext[x=7.182908in,y=5.097477in,,base]{\color{textcolor}\sffamily\fontsize{10.000000}{12.000000}\selectfont 0.8}%
\end{pgfscope}%
\begin{pgfscope}%
\definecolor{textcolor}{rgb}{0.000000,0.000000,0.000000}%
\pgfsetstrokecolor{textcolor}%
\pgfsetfillcolor{textcolor}%
\pgftext[x=6.990898in,y=5.319613in,,base]{\color{textcolor}\sffamily\fontsize{10.000000}{12.000000}\selectfont 0.9}%
\end{pgfscope}%
\begin{pgfscope}%
\definecolor{textcolor}{rgb}{0.000000,0.000000,0.000000}%
\pgfsetstrokecolor{textcolor}%
\pgfsetfillcolor{textcolor}%
\pgftext[x=6.798888in,y=5.541748in,,base]{\color{textcolor}\sffamily\fontsize{10.000000}{12.000000}\selectfont 1.0}%
\end{pgfscope}%
\begin{pgfscope}%
\definecolor{textcolor}{rgb}{0.000000,0.000000,0.000000}%
\pgfsetstrokecolor{textcolor}%
\pgfsetfillcolor{textcolor}%
\pgftext[x=4.721340in,y=3.364820in,,base]{\color{textcolor}\sffamily\fontsize{10.000000}{12.000000}\selectfont 1.0}%
\end{pgfscope}%
\begin{pgfscope}%
\definecolor{textcolor}{rgb}{0.000000,0.000000,0.000000}%
\pgfsetstrokecolor{textcolor}%
\pgfsetfillcolor{textcolor}%
\pgftext[x=4.913350in,y=3.586956in,,base]{\color{textcolor}\sffamily\fontsize{10.000000}{12.000000}\selectfont 0.9}%
\end{pgfscope}%
\begin{pgfscope}%
\definecolor{textcolor}{rgb}{0.000000,0.000000,0.000000}%
\pgfsetstrokecolor{textcolor}%
\pgfsetfillcolor{textcolor}%
\pgftext[x=5.105360in,y=3.809091in,,base]{\color{textcolor}\sffamily\fontsize{10.000000}{12.000000}\selectfont 0.8}%
\end{pgfscope}%
\begin{pgfscope}%
\definecolor{textcolor}{rgb}{0.000000,0.000000,0.000000}%
\pgfsetstrokecolor{textcolor}%
\pgfsetfillcolor{textcolor}%
\pgftext[x=5.297370in,y=4.031227in,,base]{\color{textcolor}\sffamily\fontsize{10.000000}{12.000000}\selectfont 0.7}%
\end{pgfscope}%
\begin{pgfscope}%
\definecolor{textcolor}{rgb}{0.000000,0.000000,0.000000}%
\pgfsetstrokecolor{textcolor}%
\pgfsetfillcolor{textcolor}%
\pgftext[x=5.489380in,y=4.253362in,,base]{\color{textcolor}\sffamily\fontsize{10.000000}{12.000000}\selectfont 0.6}%
\end{pgfscope}%
\begin{pgfscope}%
\definecolor{textcolor}{rgb}{0.000000,0.000000,0.000000}%
\pgfsetstrokecolor{textcolor}%
\pgfsetfillcolor{textcolor}%
\pgftext[x=5.681390in,y=4.475498in,,base]{\color{textcolor}\sffamily\fontsize{10.000000}{12.000000}\selectfont 0.5}%
\end{pgfscope}%
\begin{pgfscope}%
\definecolor{textcolor}{rgb}{0.000000,0.000000,0.000000}%
\pgfsetstrokecolor{textcolor}%
\pgfsetfillcolor{textcolor}%
\pgftext[x=5.873400in,y=4.697633in,,base]{\color{textcolor}\sffamily\fontsize{10.000000}{12.000000}\selectfont 0.4}%
\end{pgfscope}%
\begin{pgfscope}%
\definecolor{textcolor}{rgb}{0.000000,0.000000,0.000000}%
\pgfsetstrokecolor{textcolor}%
\pgfsetfillcolor{textcolor}%
\pgftext[x=6.065410in,y=4.919769in,,base]{\color{textcolor}\sffamily\fontsize{10.000000}{12.000000}\selectfont 0.3}%
\end{pgfscope}%
\begin{pgfscope}%
\definecolor{textcolor}{rgb}{0.000000,0.000000,0.000000}%
\pgfsetstrokecolor{textcolor}%
\pgfsetfillcolor{textcolor}%
\pgftext[x=6.257420in,y=5.141904in,,base]{\color{textcolor}\sffamily\fontsize{10.000000}{12.000000}\selectfont 0.2}%
\end{pgfscope}%
\begin{pgfscope}%
\definecolor{textcolor}{rgb}{0.000000,0.000000,0.000000}%
\pgfsetstrokecolor{textcolor}%
\pgfsetfillcolor{textcolor}%
\pgftext[x=6.449430in,y=5.364040in,,base]{\color{textcolor}\sffamily\fontsize{10.000000}{12.000000}\selectfont 0.1}%
\end{pgfscope}%
\begin{pgfscope}%
\definecolor{textcolor}{rgb}{0.000000,0.000000,0.000000}%
\pgfsetstrokecolor{textcolor}%
\pgfsetfillcolor{textcolor}%
\pgftext[x=6.641440in,y=5.586175in,,base]{\color{textcolor}\sffamily\fontsize{10.000000}{12.000000}\selectfont 0.0}%
\end{pgfscope}%
\begin{pgfscope}%
\definecolor{textcolor}{rgb}{0.000000,0.000000,0.000000}%
\pgfsetstrokecolor{textcolor}%
\pgfsetfillcolor{textcolor}%
\pgftext[x=4.721340in,y=3.253753in,,base]{\color{textcolor}\sffamily\fontsize{10.000000}{12.000000}\selectfont 0.0}%
\end{pgfscope}%
\begin{pgfscope}%
\definecolor{textcolor}{rgb}{0.000000,0.000000,0.000000}%
\pgfsetstrokecolor{textcolor}%
\pgfsetfillcolor{textcolor}%
\pgftext[x=5.105360in,y=3.253753in,,base]{\color{textcolor}\sffamily\fontsize{10.000000}{12.000000}\selectfont 0.1}%
\end{pgfscope}%
\begin{pgfscope}%
\definecolor{textcolor}{rgb}{0.000000,0.000000,0.000000}%
\pgfsetstrokecolor{textcolor}%
\pgfsetfillcolor{textcolor}%
\pgftext[x=5.489380in,y=3.253753in,,base]{\color{textcolor}\sffamily\fontsize{10.000000}{12.000000}\selectfont 0.2}%
\end{pgfscope}%
\begin{pgfscope}%
\definecolor{textcolor}{rgb}{0.000000,0.000000,0.000000}%
\pgfsetstrokecolor{textcolor}%
\pgfsetfillcolor{textcolor}%
\pgftext[x=5.873400in,y=3.253753in,,base]{\color{textcolor}\sffamily\fontsize{10.000000}{12.000000}\selectfont 0.3}%
\end{pgfscope}%
\begin{pgfscope}%
\definecolor{textcolor}{rgb}{0.000000,0.000000,0.000000}%
\pgfsetstrokecolor{textcolor}%
\pgfsetfillcolor{textcolor}%
\pgftext[x=6.257420in,y=3.253753in,,base]{\color{textcolor}\sffamily\fontsize{10.000000}{12.000000}\selectfont 0.4}%
\end{pgfscope}%
\begin{pgfscope}%
\definecolor{textcolor}{rgb}{0.000000,0.000000,0.000000}%
\pgfsetstrokecolor{textcolor}%
\pgfsetfillcolor{textcolor}%
\pgftext[x=6.641440in,y=3.253753in,,base]{\color{textcolor}\sffamily\fontsize{10.000000}{12.000000}\selectfont 0.5}%
\end{pgfscope}%
\begin{pgfscope}%
\definecolor{textcolor}{rgb}{0.000000,0.000000,0.000000}%
\pgfsetstrokecolor{textcolor}%
\pgfsetfillcolor{textcolor}%
\pgftext[x=7.025460in,y=3.253753in,,base]{\color{textcolor}\sffamily\fontsize{10.000000}{12.000000}\selectfont 0.6}%
\end{pgfscope}%
\begin{pgfscope}%
\definecolor{textcolor}{rgb}{0.000000,0.000000,0.000000}%
\pgfsetstrokecolor{textcolor}%
\pgfsetfillcolor{textcolor}%
\pgftext[x=7.409479in,y=3.253753in,,base]{\color{textcolor}\sffamily\fontsize{10.000000}{12.000000}\selectfont 0.7}%
\end{pgfscope}%
\begin{pgfscope}%
\definecolor{textcolor}{rgb}{0.000000,0.000000,0.000000}%
\pgfsetstrokecolor{textcolor}%
\pgfsetfillcolor{textcolor}%
\pgftext[x=7.793499in,y=3.253753in,,base]{\color{textcolor}\sffamily\fontsize{10.000000}{12.000000}\selectfont 0.8}%
\end{pgfscope}%
\begin{pgfscope}%
\definecolor{textcolor}{rgb}{0.000000,0.000000,0.000000}%
\pgfsetstrokecolor{textcolor}%
\pgfsetfillcolor{textcolor}%
\pgftext[x=8.177519in,y=3.253753in,,base]{\color{textcolor}\sffamily\fontsize{10.000000}{12.000000}\selectfont 0.9}%
\end{pgfscope}%
\begin{pgfscope}%
\definecolor{textcolor}{rgb}{0.000000,0.000000,0.000000}%
\pgfsetstrokecolor{textcolor}%
\pgfsetfillcolor{textcolor}%
\pgftext[x=8.561539in,y=3.253753in,,base]{\color{textcolor}\sffamily\fontsize{10.000000}{12.000000}\selectfont 1.0}%
\end{pgfscope}%
\begin{pgfscope}%
\pgfpathrectangle{\pgfqpoint{4.577333in}{3.075000in}}{\pgfqpoint{4.224218in}{2.565000in}}%
\pgfusepath{clip}%
\pgfsetbuttcap%
\pgfsetroundjoin%
\definecolor{currentfill}{rgb}{1.000000,0.000000,0.000000}%
\pgfsetfillcolor{currentfill}%
\pgfsetlinewidth{1.505625pt}%
\definecolor{currentstroke}{rgb}{1.000000,0.000000,0.000000}%
\pgfsetstrokecolor{currentstroke}%
\pgfsetdash{}{0pt}%
\pgfsys@defobject{currentmarker}{\pgfqpoint{-0.013608in}{-0.017010in}}{\pgfqpoint{0.013608in}{0.008505in}}{%
\pgfpathmoveto{\pgfqpoint{0.000000in}{0.000000in}}%
\pgfpathlineto{\pgfqpoint{0.000000in}{-0.017010in}}%
\pgfpathmoveto{\pgfqpoint{0.000000in}{0.000000in}}%
\pgfpathlineto{\pgfqpoint{0.013608in}{0.008505in}}%
\pgfpathmoveto{\pgfqpoint{0.000000in}{0.000000in}}%
\pgfpathlineto{\pgfqpoint{-0.013608in}{0.008505in}}%
\pgfusepath{stroke,fill}%
}%
\begin{pgfscope}%
\pgfsys@transformshift{6.140842in}{4.283509in}%
\pgfsys@useobject{currentmarker}{}%
\end{pgfscope}%
\end{pgfscope}%
\begin{pgfscope}%
\definecolor{textcolor}{rgb}{0.000000,0.000000,0.000000}%
\pgfsetstrokecolor{textcolor}%
\pgfsetfillcolor{textcolor}%
\pgftext[x=6.689442in,y=5.723333in,,base]{\color{textcolor}\sffamily\fontsize{12.000000}{14.400000}\selectfont PG - Alternate}%
\end{pgfscope}%
\begin{pgfscope}%
\pgfsetbuttcap%
\pgfsetmiterjoin%
\definecolor{currentfill}{rgb}{1.000000,1.000000,1.000000}%
\pgfsetfillcolor{currentfill}%
\pgfsetlinewidth{0.000000pt}%
\definecolor{currentstroke}{rgb}{0.000000,0.000000,0.000000}%
\pgfsetstrokecolor{currentstroke}%
\pgfsetstrokeopacity{0.000000}%
\pgfsetdash{}{0pt}%
\pgfpathmoveto{\pgfqpoint{0.152333in}{0.150000in}}%
\pgfpathlineto{\pgfqpoint{4.376551in}{0.150000in}}%
\pgfpathlineto{\pgfqpoint{4.376551in}{2.715000in}}%
\pgfpathlineto{\pgfqpoint{0.152333in}{2.715000in}}%
\pgfpathlineto{\pgfqpoint{0.152333in}{0.150000in}}%
\pgfpathclose%
\pgfusepath{fill}%
\end{pgfscope}%
\begin{pgfscope}%
\pgfpathrectangle{\pgfqpoint{0.152333in}{0.150000in}}{\pgfqpoint{4.224218in}{2.565000in}}%
\pgfusepath{clip}%
\pgfsetbuttcap%
\pgfsetmiterjoin%
\definecolor{currentfill}{rgb}{0.960784,0.960784,0.960784}%
\pgfsetfillcolor{currentfill}%
\pgfsetfillopacity{0.750000}%
\pgfsetlinewidth{1.003750pt}%
\definecolor{currentstroke}{rgb}{0.960784,0.960784,0.960784}%
\pgfsetstrokecolor{currentstroke}%
\pgfsetstrokeopacity{0.750000}%
\pgfsetdash{}{0pt}%
\pgfpathmoveto{\pgfqpoint{4.184541in}{0.406500in}}%
\pgfpathlineto{\pgfqpoint{2.264442in}{2.627855in}}%
\pgfpathlineto{\pgfqpoint{0.344343in}{0.406500in}}%
\pgfpathlineto{\pgfqpoint{4.184541in}{0.406500in}}%
\pgfpathclose%
\pgfusepath{stroke,fill}%
\end{pgfscope}%
\begin{pgfscope}%
\pgfpathrectangle{\pgfqpoint{0.152333in}{0.150000in}}{\pgfqpoint{4.224218in}{2.565000in}}%
\pgfusepath{clip}%
\pgfsetrectcap%
\pgfsetroundjoin%
\pgfsetlinewidth{0.803000pt}%
\definecolor{currentstroke}{rgb}{0.000000,0.000000,0.000000}%
\pgfsetstrokecolor{currentstroke}%
\pgfsetdash{}{0pt}%
\pgfpathmoveto{\pgfqpoint{0.344343in}{0.406500in}}%
\pgfpathlineto{\pgfqpoint{4.184541in}{0.406500in}}%
\pgfusepath{stroke}%
\end{pgfscope}%
\begin{pgfscope}%
\pgfpathrectangle{\pgfqpoint{0.152333in}{0.150000in}}{\pgfqpoint{4.224218in}{2.565000in}}%
\pgfusepath{clip}%
\pgfsetrectcap%
\pgfsetroundjoin%
\pgfsetlinewidth{0.803000pt}%
\definecolor{currentstroke}{rgb}{0.000000,0.000000,0.000000}%
\pgfsetstrokecolor{currentstroke}%
\pgfsetdash{}{0pt}%
\pgfpathmoveto{\pgfqpoint{2.264442in}{2.627855in}}%
\pgfpathlineto{\pgfqpoint{0.344343in}{0.406500in}}%
\pgfusepath{stroke}%
\end{pgfscope}%
\begin{pgfscope}%
\pgfpathrectangle{\pgfqpoint{0.152333in}{0.150000in}}{\pgfqpoint{4.224218in}{2.565000in}}%
\pgfusepath{clip}%
\pgfsetrectcap%
\pgfsetroundjoin%
\pgfsetlinewidth{0.803000pt}%
\definecolor{currentstroke}{rgb}{0.000000,0.000000,0.000000}%
\pgfsetstrokecolor{currentstroke}%
\pgfsetdash{}{0pt}%
\pgfpathmoveto{\pgfqpoint{2.264442in}{2.627855in}}%
\pgfpathlineto{\pgfqpoint{4.184541in}{0.406500in}}%
\pgfusepath{stroke}%
\end{pgfscope}%
\begin{pgfscope}%
\pgfpathrectangle{\pgfqpoint{0.152333in}{0.150000in}}{\pgfqpoint{4.224218in}{2.565000in}}%
\pgfusepath{clip}%
\pgfsetbuttcap%
\pgfsetroundjoin%
\pgfsetlinewidth{0.501875pt}%
\definecolor{currentstroke}{rgb}{0.000000,0.000000,0.000000}%
\pgfsetstrokecolor{currentstroke}%
\pgfsetdash{{0.500000pt}{0.825000pt}}{0.000000pt}%
\pgfpathmoveto{\pgfqpoint{0.344343in}{0.406500in}}%
\pgfpathlineto{\pgfqpoint{4.184541in}{0.406500in}}%
\pgfusepath{stroke}%
\end{pgfscope}%
\begin{pgfscope}%
\pgfpathrectangle{\pgfqpoint{0.152333in}{0.150000in}}{\pgfqpoint{4.224218in}{2.565000in}}%
\pgfusepath{clip}%
\pgfsetbuttcap%
\pgfsetroundjoin%
\pgfsetlinewidth{0.501875pt}%
\definecolor{currentstroke}{rgb}{0.000000,0.000000,0.000000}%
\pgfsetstrokecolor{currentstroke}%
\pgfsetdash{{0.500000pt}{0.825000pt}}{0.000000pt}%
\pgfpathmoveto{\pgfqpoint{1.304393in}{1.517178in}}%
\pgfpathlineto{\pgfqpoint{3.224492in}{1.517178in}}%
\pgfusepath{stroke}%
\end{pgfscope}%
\begin{pgfscope}%
\pgfpathrectangle{\pgfqpoint{0.152333in}{0.150000in}}{\pgfqpoint{4.224218in}{2.565000in}}%
\pgfusepath{clip}%
\pgfsetbuttcap%
\pgfsetroundjoin%
\pgfsetlinewidth{0.501875pt}%
\definecolor{currentstroke}{rgb}{0.000000,0.000000,0.000000}%
\pgfsetstrokecolor{currentstroke}%
\pgfsetdash{{0.500000pt}{0.825000pt}}{0.000000pt}%
\pgfpathmoveto{\pgfqpoint{2.264442in}{2.627855in}}%
\pgfpathlineto{\pgfqpoint{0.344343in}{0.406500in}}%
\pgfusepath{stroke}%
\end{pgfscope}%
\begin{pgfscope}%
\pgfpathrectangle{\pgfqpoint{0.152333in}{0.150000in}}{\pgfqpoint{4.224218in}{2.565000in}}%
\pgfusepath{clip}%
\pgfsetbuttcap%
\pgfsetroundjoin%
\pgfsetlinewidth{0.501875pt}%
\definecolor{currentstroke}{rgb}{0.000000,0.000000,0.000000}%
\pgfsetstrokecolor{currentstroke}%
\pgfsetdash{{0.500000pt}{0.825000pt}}{0.000000pt}%
\pgfpathmoveto{\pgfqpoint{2.264442in}{2.627855in}}%
\pgfpathlineto{\pgfqpoint{4.184541in}{0.406500in}}%
\pgfusepath{stroke}%
\end{pgfscope}%
\begin{pgfscope}%
\pgfpathrectangle{\pgfqpoint{0.152333in}{0.150000in}}{\pgfqpoint{4.224218in}{2.565000in}}%
\pgfusepath{clip}%
\pgfsetbuttcap%
\pgfsetroundjoin%
\pgfsetlinewidth{0.501875pt}%
\definecolor{currentstroke}{rgb}{0.000000,0.000000,0.000000}%
\pgfsetstrokecolor{currentstroke}%
\pgfsetdash{{0.500000pt}{0.825000pt}}{0.000000pt}%
\pgfpathmoveto{\pgfqpoint{3.224492in}{1.517178in}}%
\pgfpathlineto{\pgfqpoint{2.264442in}{0.406500in}}%
\pgfusepath{stroke}%
\end{pgfscope}%
\begin{pgfscope}%
\pgfpathrectangle{\pgfqpoint{0.152333in}{0.150000in}}{\pgfqpoint{4.224218in}{2.565000in}}%
\pgfusepath{clip}%
\pgfsetbuttcap%
\pgfsetroundjoin%
\pgfsetlinewidth{0.501875pt}%
\definecolor{currentstroke}{rgb}{0.000000,0.000000,0.000000}%
\pgfsetstrokecolor{currentstroke}%
\pgfsetdash{{0.500000pt}{0.825000pt}}{0.000000pt}%
\pgfpathmoveto{\pgfqpoint{1.304393in}{1.517178in}}%
\pgfpathlineto{\pgfqpoint{2.264442in}{0.406500in}}%
\pgfusepath{stroke}%
\end{pgfscope}%
\begin{pgfscope}%
\pgfpathrectangle{\pgfqpoint{0.152333in}{0.150000in}}{\pgfqpoint{4.224218in}{2.565000in}}%
\pgfusepath{clip}%
\pgfsetbuttcap%
\pgfsetroundjoin%
\pgfsetlinewidth{0.501875pt}%
\definecolor{currentstroke}{rgb}{0.000000,0.000000,0.000000}%
\pgfsetstrokecolor{currentstroke}%
\pgfsetdash{{0.500000pt}{0.825000pt}}{0.000000pt}%
\pgfpathmoveto{\pgfqpoint{4.184541in}{0.406500in}}%
\pgfpathlineto{\pgfqpoint{4.184541in}{0.406500in}}%
\pgfusepath{stroke}%
\end{pgfscope}%
\begin{pgfscope}%
\pgfpathrectangle{\pgfqpoint{0.152333in}{0.150000in}}{\pgfqpoint{4.224218in}{2.565000in}}%
\pgfusepath{clip}%
\pgfsetbuttcap%
\pgfsetroundjoin%
\pgfsetlinewidth{0.501875pt}%
\definecolor{currentstroke}{rgb}{0.000000,0.000000,0.000000}%
\pgfsetstrokecolor{currentstroke}%
\pgfsetdash{{0.500000pt}{0.825000pt}}{0.000000pt}%
\pgfpathmoveto{\pgfqpoint{0.344343in}{0.406500in}}%
\pgfpathlineto{\pgfqpoint{0.344343in}{0.406500in}}%
\pgfusepath{stroke}%
\end{pgfscope}%
\begin{pgfscope}%
\pgfpathrectangle{\pgfqpoint{0.152333in}{0.150000in}}{\pgfqpoint{4.224218in}{2.565000in}}%
\pgfusepath{clip}%
\pgfsetbuttcap%
\pgfsetroundjoin%
\pgfsetlinewidth{0.501875pt}%
\definecolor{currentstroke}{rgb}{0.000000,0.000000,1.000000}%
\pgfsetstrokecolor{currentstroke}%
\pgfsetdash{{0.500000pt}{0.825000pt}}{0.000000pt}%
\pgfpathmoveto{\pgfqpoint{0.344343in}{0.406500in}}%
\pgfpathlineto{\pgfqpoint{4.184541in}{0.406500in}}%
\pgfusepath{stroke}%
\end{pgfscope}%
\begin{pgfscope}%
\pgfpathrectangle{\pgfqpoint{0.152333in}{0.150000in}}{\pgfqpoint{4.224218in}{2.565000in}}%
\pgfusepath{clip}%
\pgfsetbuttcap%
\pgfsetroundjoin%
\pgfsetlinewidth{0.501875pt}%
\definecolor{currentstroke}{rgb}{0.000000,0.000000,1.000000}%
\pgfsetstrokecolor{currentstroke}%
\pgfsetdash{{0.500000pt}{0.825000pt}}{0.000000pt}%
\pgfpathmoveto{\pgfqpoint{0.536353in}{0.628636in}}%
\pgfpathlineto{\pgfqpoint{3.992531in}{0.628636in}}%
\pgfusepath{stroke}%
\end{pgfscope}%
\begin{pgfscope}%
\pgfpathrectangle{\pgfqpoint{0.152333in}{0.150000in}}{\pgfqpoint{4.224218in}{2.565000in}}%
\pgfusepath{clip}%
\pgfsetbuttcap%
\pgfsetroundjoin%
\pgfsetlinewidth{0.501875pt}%
\definecolor{currentstroke}{rgb}{0.000000,0.000000,1.000000}%
\pgfsetstrokecolor{currentstroke}%
\pgfsetdash{{0.500000pt}{0.825000pt}}{0.000000pt}%
\pgfpathmoveto{\pgfqpoint{0.728363in}{0.850771in}}%
\pgfpathlineto{\pgfqpoint{3.800522in}{0.850771in}}%
\pgfusepath{stroke}%
\end{pgfscope}%
\begin{pgfscope}%
\pgfpathrectangle{\pgfqpoint{0.152333in}{0.150000in}}{\pgfqpoint{4.224218in}{2.565000in}}%
\pgfusepath{clip}%
\pgfsetbuttcap%
\pgfsetroundjoin%
\pgfsetlinewidth{0.501875pt}%
\definecolor{currentstroke}{rgb}{0.000000,0.000000,1.000000}%
\pgfsetstrokecolor{currentstroke}%
\pgfsetdash{{0.500000pt}{0.825000pt}}{0.000000pt}%
\pgfpathmoveto{\pgfqpoint{0.920373in}{1.072907in}}%
\pgfpathlineto{\pgfqpoint{3.608512in}{1.072907in}}%
\pgfusepath{stroke}%
\end{pgfscope}%
\begin{pgfscope}%
\pgfpathrectangle{\pgfqpoint{0.152333in}{0.150000in}}{\pgfqpoint{4.224218in}{2.565000in}}%
\pgfusepath{clip}%
\pgfsetbuttcap%
\pgfsetroundjoin%
\pgfsetlinewidth{0.501875pt}%
\definecolor{currentstroke}{rgb}{0.000000,0.000000,1.000000}%
\pgfsetstrokecolor{currentstroke}%
\pgfsetdash{{0.500000pt}{0.825000pt}}{0.000000pt}%
\pgfpathmoveto{\pgfqpoint{1.112383in}{1.295042in}}%
\pgfpathlineto{\pgfqpoint{3.416502in}{1.295042in}}%
\pgfusepath{stroke}%
\end{pgfscope}%
\begin{pgfscope}%
\pgfpathrectangle{\pgfqpoint{0.152333in}{0.150000in}}{\pgfqpoint{4.224218in}{2.565000in}}%
\pgfusepath{clip}%
\pgfsetbuttcap%
\pgfsetroundjoin%
\pgfsetlinewidth{0.501875pt}%
\definecolor{currentstroke}{rgb}{0.000000,0.000000,1.000000}%
\pgfsetstrokecolor{currentstroke}%
\pgfsetdash{{0.500000pt}{0.825000pt}}{0.000000pt}%
\pgfpathmoveto{\pgfqpoint{1.304393in}{1.517178in}}%
\pgfpathlineto{\pgfqpoint{3.224492in}{1.517178in}}%
\pgfusepath{stroke}%
\end{pgfscope}%
\begin{pgfscope}%
\pgfpathrectangle{\pgfqpoint{0.152333in}{0.150000in}}{\pgfqpoint{4.224218in}{2.565000in}}%
\pgfusepath{clip}%
\pgfsetbuttcap%
\pgfsetroundjoin%
\pgfsetlinewidth{0.501875pt}%
\definecolor{currentstroke}{rgb}{0.000000,0.000000,1.000000}%
\pgfsetstrokecolor{currentstroke}%
\pgfsetdash{{0.500000pt}{0.825000pt}}{0.000000pt}%
\pgfpathmoveto{\pgfqpoint{1.496402in}{1.739313in}}%
\pgfpathlineto{\pgfqpoint{3.032482in}{1.739313in}}%
\pgfusepath{stroke}%
\end{pgfscope}%
\begin{pgfscope}%
\pgfpathrectangle{\pgfqpoint{0.152333in}{0.150000in}}{\pgfqpoint{4.224218in}{2.565000in}}%
\pgfusepath{clip}%
\pgfsetbuttcap%
\pgfsetroundjoin%
\pgfsetlinewidth{0.501875pt}%
\definecolor{currentstroke}{rgb}{0.000000,0.000000,1.000000}%
\pgfsetstrokecolor{currentstroke}%
\pgfsetdash{{0.500000pt}{0.825000pt}}{0.000000pt}%
\pgfpathmoveto{\pgfqpoint{1.688412in}{1.961449in}}%
\pgfpathlineto{\pgfqpoint{2.840472in}{1.961449in}}%
\pgfusepath{stroke}%
\end{pgfscope}%
\begin{pgfscope}%
\pgfpathrectangle{\pgfqpoint{0.152333in}{0.150000in}}{\pgfqpoint{4.224218in}{2.565000in}}%
\pgfusepath{clip}%
\pgfsetbuttcap%
\pgfsetroundjoin%
\pgfsetlinewidth{0.501875pt}%
\definecolor{currentstroke}{rgb}{0.000000,0.000000,1.000000}%
\pgfsetstrokecolor{currentstroke}%
\pgfsetdash{{0.500000pt}{0.825000pt}}{0.000000pt}%
\pgfpathmoveto{\pgfqpoint{1.880422in}{2.183584in}}%
\pgfpathlineto{\pgfqpoint{2.648462in}{2.183584in}}%
\pgfusepath{stroke}%
\end{pgfscope}%
\begin{pgfscope}%
\pgfpathrectangle{\pgfqpoint{0.152333in}{0.150000in}}{\pgfqpoint{4.224218in}{2.565000in}}%
\pgfusepath{clip}%
\pgfsetbuttcap%
\pgfsetroundjoin%
\pgfsetlinewidth{0.501875pt}%
\definecolor{currentstroke}{rgb}{0.000000,0.000000,1.000000}%
\pgfsetstrokecolor{currentstroke}%
\pgfsetdash{{0.500000pt}{0.825000pt}}{0.000000pt}%
\pgfpathmoveto{\pgfqpoint{2.072432in}{2.405720in}}%
\pgfpathlineto{\pgfqpoint{2.456452in}{2.405720in}}%
\pgfusepath{stroke}%
\end{pgfscope}%
\begin{pgfscope}%
\pgfpathrectangle{\pgfqpoint{0.152333in}{0.150000in}}{\pgfqpoint{4.224218in}{2.565000in}}%
\pgfusepath{clip}%
\pgfsetbuttcap%
\pgfsetroundjoin%
\pgfsetlinewidth{0.501875pt}%
\definecolor{currentstroke}{rgb}{0.000000,0.000000,1.000000}%
\pgfsetstrokecolor{currentstroke}%
\pgfsetdash{{0.500000pt}{0.825000pt}}{0.000000pt}%
\pgfpathmoveto{\pgfqpoint{2.264442in}{2.627855in}}%
\pgfpathlineto{\pgfqpoint{0.344343in}{0.406500in}}%
\pgfusepath{stroke}%
\end{pgfscope}%
\begin{pgfscope}%
\pgfpathrectangle{\pgfqpoint{0.152333in}{0.150000in}}{\pgfqpoint{4.224218in}{2.565000in}}%
\pgfusepath{clip}%
\pgfsetbuttcap%
\pgfsetroundjoin%
\pgfsetlinewidth{0.501875pt}%
\definecolor{currentstroke}{rgb}{0.000000,0.000000,1.000000}%
\pgfsetstrokecolor{currentstroke}%
\pgfsetdash{{0.500000pt}{0.825000pt}}{0.000000pt}%
\pgfpathmoveto{\pgfqpoint{2.264442in}{2.627855in}}%
\pgfpathlineto{\pgfqpoint{4.184541in}{0.406500in}}%
\pgfusepath{stroke}%
\end{pgfscope}%
\begin{pgfscope}%
\pgfpathrectangle{\pgfqpoint{0.152333in}{0.150000in}}{\pgfqpoint{4.224218in}{2.565000in}}%
\pgfusepath{clip}%
\pgfsetbuttcap%
\pgfsetroundjoin%
\pgfsetlinewidth{0.501875pt}%
\definecolor{currentstroke}{rgb}{0.000000,0.000000,1.000000}%
\pgfsetstrokecolor{currentstroke}%
\pgfsetdash{{0.500000pt}{0.825000pt}}{0.000000pt}%
\pgfpathmoveto{\pgfqpoint{2.456452in}{2.405720in}}%
\pgfpathlineto{\pgfqpoint{0.728363in}{0.406500in}}%
\pgfusepath{stroke}%
\end{pgfscope}%
\begin{pgfscope}%
\pgfpathrectangle{\pgfqpoint{0.152333in}{0.150000in}}{\pgfqpoint{4.224218in}{2.565000in}}%
\pgfusepath{clip}%
\pgfsetbuttcap%
\pgfsetroundjoin%
\pgfsetlinewidth{0.501875pt}%
\definecolor{currentstroke}{rgb}{0.000000,0.000000,1.000000}%
\pgfsetstrokecolor{currentstroke}%
\pgfsetdash{{0.500000pt}{0.825000pt}}{0.000000pt}%
\pgfpathmoveto{\pgfqpoint{2.072432in}{2.405720in}}%
\pgfpathlineto{\pgfqpoint{3.800522in}{0.406500in}}%
\pgfusepath{stroke}%
\end{pgfscope}%
\begin{pgfscope}%
\pgfpathrectangle{\pgfqpoint{0.152333in}{0.150000in}}{\pgfqpoint{4.224218in}{2.565000in}}%
\pgfusepath{clip}%
\pgfsetbuttcap%
\pgfsetroundjoin%
\pgfsetlinewidth{0.501875pt}%
\definecolor{currentstroke}{rgb}{0.000000,0.000000,1.000000}%
\pgfsetstrokecolor{currentstroke}%
\pgfsetdash{{0.500000pt}{0.825000pt}}{0.000000pt}%
\pgfpathmoveto{\pgfqpoint{2.648462in}{2.183584in}}%
\pgfpathlineto{\pgfqpoint{1.112383in}{0.406500in}}%
\pgfusepath{stroke}%
\end{pgfscope}%
\begin{pgfscope}%
\pgfpathrectangle{\pgfqpoint{0.152333in}{0.150000in}}{\pgfqpoint{4.224218in}{2.565000in}}%
\pgfusepath{clip}%
\pgfsetbuttcap%
\pgfsetroundjoin%
\pgfsetlinewidth{0.501875pt}%
\definecolor{currentstroke}{rgb}{0.000000,0.000000,1.000000}%
\pgfsetstrokecolor{currentstroke}%
\pgfsetdash{{0.500000pt}{0.825000pt}}{0.000000pt}%
\pgfpathmoveto{\pgfqpoint{1.880422in}{2.183584in}}%
\pgfpathlineto{\pgfqpoint{3.416502in}{0.406500in}}%
\pgfusepath{stroke}%
\end{pgfscope}%
\begin{pgfscope}%
\pgfpathrectangle{\pgfqpoint{0.152333in}{0.150000in}}{\pgfqpoint{4.224218in}{2.565000in}}%
\pgfusepath{clip}%
\pgfsetbuttcap%
\pgfsetroundjoin%
\pgfsetlinewidth{0.501875pt}%
\definecolor{currentstroke}{rgb}{0.000000,0.000000,1.000000}%
\pgfsetstrokecolor{currentstroke}%
\pgfsetdash{{0.500000pt}{0.825000pt}}{0.000000pt}%
\pgfpathmoveto{\pgfqpoint{2.840472in}{1.961449in}}%
\pgfpathlineto{\pgfqpoint{1.496402in}{0.406500in}}%
\pgfusepath{stroke}%
\end{pgfscope}%
\begin{pgfscope}%
\pgfpathrectangle{\pgfqpoint{0.152333in}{0.150000in}}{\pgfqpoint{4.224218in}{2.565000in}}%
\pgfusepath{clip}%
\pgfsetbuttcap%
\pgfsetroundjoin%
\pgfsetlinewidth{0.501875pt}%
\definecolor{currentstroke}{rgb}{0.000000,0.000000,1.000000}%
\pgfsetstrokecolor{currentstroke}%
\pgfsetdash{{0.500000pt}{0.825000pt}}{0.000000pt}%
\pgfpathmoveto{\pgfqpoint{1.688412in}{1.961449in}}%
\pgfpathlineto{\pgfqpoint{3.032482in}{0.406500in}}%
\pgfusepath{stroke}%
\end{pgfscope}%
\begin{pgfscope}%
\pgfpathrectangle{\pgfqpoint{0.152333in}{0.150000in}}{\pgfqpoint{4.224218in}{2.565000in}}%
\pgfusepath{clip}%
\pgfsetbuttcap%
\pgfsetroundjoin%
\pgfsetlinewidth{0.501875pt}%
\definecolor{currentstroke}{rgb}{0.000000,0.000000,1.000000}%
\pgfsetstrokecolor{currentstroke}%
\pgfsetdash{{0.500000pt}{0.825000pt}}{0.000000pt}%
\pgfpathmoveto{\pgfqpoint{3.032482in}{1.739313in}}%
\pgfpathlineto{\pgfqpoint{1.880422in}{0.406500in}}%
\pgfusepath{stroke}%
\end{pgfscope}%
\begin{pgfscope}%
\pgfpathrectangle{\pgfqpoint{0.152333in}{0.150000in}}{\pgfqpoint{4.224218in}{2.565000in}}%
\pgfusepath{clip}%
\pgfsetbuttcap%
\pgfsetroundjoin%
\pgfsetlinewidth{0.501875pt}%
\definecolor{currentstroke}{rgb}{0.000000,0.000000,1.000000}%
\pgfsetstrokecolor{currentstroke}%
\pgfsetdash{{0.500000pt}{0.825000pt}}{0.000000pt}%
\pgfpathmoveto{\pgfqpoint{1.496402in}{1.739313in}}%
\pgfpathlineto{\pgfqpoint{2.648462in}{0.406500in}}%
\pgfusepath{stroke}%
\end{pgfscope}%
\begin{pgfscope}%
\pgfpathrectangle{\pgfqpoint{0.152333in}{0.150000in}}{\pgfqpoint{4.224218in}{2.565000in}}%
\pgfusepath{clip}%
\pgfsetbuttcap%
\pgfsetroundjoin%
\pgfsetlinewidth{0.501875pt}%
\definecolor{currentstroke}{rgb}{0.000000,0.000000,1.000000}%
\pgfsetstrokecolor{currentstroke}%
\pgfsetdash{{0.500000pt}{0.825000pt}}{0.000000pt}%
\pgfpathmoveto{\pgfqpoint{3.224492in}{1.517178in}}%
\pgfpathlineto{\pgfqpoint{2.264442in}{0.406500in}}%
\pgfusepath{stroke}%
\end{pgfscope}%
\begin{pgfscope}%
\pgfpathrectangle{\pgfqpoint{0.152333in}{0.150000in}}{\pgfqpoint{4.224218in}{2.565000in}}%
\pgfusepath{clip}%
\pgfsetbuttcap%
\pgfsetroundjoin%
\pgfsetlinewidth{0.501875pt}%
\definecolor{currentstroke}{rgb}{0.000000,0.000000,1.000000}%
\pgfsetstrokecolor{currentstroke}%
\pgfsetdash{{0.500000pt}{0.825000pt}}{0.000000pt}%
\pgfpathmoveto{\pgfqpoint{1.304393in}{1.517178in}}%
\pgfpathlineto{\pgfqpoint{2.264442in}{0.406500in}}%
\pgfusepath{stroke}%
\end{pgfscope}%
\begin{pgfscope}%
\pgfpathrectangle{\pgfqpoint{0.152333in}{0.150000in}}{\pgfqpoint{4.224218in}{2.565000in}}%
\pgfusepath{clip}%
\pgfsetbuttcap%
\pgfsetroundjoin%
\pgfsetlinewidth{0.501875pt}%
\definecolor{currentstroke}{rgb}{0.000000,0.000000,1.000000}%
\pgfsetstrokecolor{currentstroke}%
\pgfsetdash{{0.500000pt}{0.825000pt}}{0.000000pt}%
\pgfpathmoveto{\pgfqpoint{3.416502in}{1.295042in}}%
\pgfpathlineto{\pgfqpoint{2.648462in}{0.406500in}}%
\pgfusepath{stroke}%
\end{pgfscope}%
\begin{pgfscope}%
\pgfpathrectangle{\pgfqpoint{0.152333in}{0.150000in}}{\pgfqpoint{4.224218in}{2.565000in}}%
\pgfusepath{clip}%
\pgfsetbuttcap%
\pgfsetroundjoin%
\pgfsetlinewidth{0.501875pt}%
\definecolor{currentstroke}{rgb}{0.000000,0.000000,1.000000}%
\pgfsetstrokecolor{currentstroke}%
\pgfsetdash{{0.500000pt}{0.825000pt}}{0.000000pt}%
\pgfpathmoveto{\pgfqpoint{1.112383in}{1.295042in}}%
\pgfpathlineto{\pgfqpoint{1.880422in}{0.406500in}}%
\pgfusepath{stroke}%
\end{pgfscope}%
\begin{pgfscope}%
\pgfpathrectangle{\pgfqpoint{0.152333in}{0.150000in}}{\pgfqpoint{4.224218in}{2.565000in}}%
\pgfusepath{clip}%
\pgfsetbuttcap%
\pgfsetroundjoin%
\pgfsetlinewidth{0.501875pt}%
\definecolor{currentstroke}{rgb}{0.000000,0.000000,1.000000}%
\pgfsetstrokecolor{currentstroke}%
\pgfsetdash{{0.500000pt}{0.825000pt}}{0.000000pt}%
\pgfpathmoveto{\pgfqpoint{3.608512in}{1.072907in}}%
\pgfpathlineto{\pgfqpoint{3.032482in}{0.406500in}}%
\pgfusepath{stroke}%
\end{pgfscope}%
\begin{pgfscope}%
\pgfpathrectangle{\pgfqpoint{0.152333in}{0.150000in}}{\pgfqpoint{4.224218in}{2.565000in}}%
\pgfusepath{clip}%
\pgfsetbuttcap%
\pgfsetroundjoin%
\pgfsetlinewidth{0.501875pt}%
\definecolor{currentstroke}{rgb}{0.000000,0.000000,1.000000}%
\pgfsetstrokecolor{currentstroke}%
\pgfsetdash{{0.500000pt}{0.825000pt}}{0.000000pt}%
\pgfpathmoveto{\pgfqpoint{0.920373in}{1.072907in}}%
\pgfpathlineto{\pgfqpoint{1.496402in}{0.406500in}}%
\pgfusepath{stroke}%
\end{pgfscope}%
\begin{pgfscope}%
\pgfpathrectangle{\pgfqpoint{0.152333in}{0.150000in}}{\pgfqpoint{4.224218in}{2.565000in}}%
\pgfusepath{clip}%
\pgfsetbuttcap%
\pgfsetroundjoin%
\pgfsetlinewidth{0.501875pt}%
\definecolor{currentstroke}{rgb}{0.000000,0.000000,1.000000}%
\pgfsetstrokecolor{currentstroke}%
\pgfsetdash{{0.500000pt}{0.825000pt}}{0.000000pt}%
\pgfpathmoveto{\pgfqpoint{3.800522in}{0.850771in}}%
\pgfpathlineto{\pgfqpoint{3.416502in}{0.406500in}}%
\pgfusepath{stroke}%
\end{pgfscope}%
\begin{pgfscope}%
\pgfpathrectangle{\pgfqpoint{0.152333in}{0.150000in}}{\pgfqpoint{4.224218in}{2.565000in}}%
\pgfusepath{clip}%
\pgfsetbuttcap%
\pgfsetroundjoin%
\pgfsetlinewidth{0.501875pt}%
\definecolor{currentstroke}{rgb}{0.000000,0.000000,1.000000}%
\pgfsetstrokecolor{currentstroke}%
\pgfsetdash{{0.500000pt}{0.825000pt}}{0.000000pt}%
\pgfpathmoveto{\pgfqpoint{0.728363in}{0.850771in}}%
\pgfpathlineto{\pgfqpoint{1.112383in}{0.406500in}}%
\pgfusepath{stroke}%
\end{pgfscope}%
\begin{pgfscope}%
\pgfpathrectangle{\pgfqpoint{0.152333in}{0.150000in}}{\pgfqpoint{4.224218in}{2.565000in}}%
\pgfusepath{clip}%
\pgfsetbuttcap%
\pgfsetroundjoin%
\pgfsetlinewidth{0.501875pt}%
\definecolor{currentstroke}{rgb}{0.000000,0.000000,1.000000}%
\pgfsetstrokecolor{currentstroke}%
\pgfsetdash{{0.500000pt}{0.825000pt}}{0.000000pt}%
\pgfpathmoveto{\pgfqpoint{3.992531in}{0.628636in}}%
\pgfpathlineto{\pgfqpoint{3.800522in}{0.406500in}}%
\pgfusepath{stroke}%
\end{pgfscope}%
\begin{pgfscope}%
\pgfpathrectangle{\pgfqpoint{0.152333in}{0.150000in}}{\pgfqpoint{4.224218in}{2.565000in}}%
\pgfusepath{clip}%
\pgfsetbuttcap%
\pgfsetroundjoin%
\pgfsetlinewidth{0.501875pt}%
\definecolor{currentstroke}{rgb}{0.000000,0.000000,1.000000}%
\pgfsetstrokecolor{currentstroke}%
\pgfsetdash{{0.500000pt}{0.825000pt}}{0.000000pt}%
\pgfpathmoveto{\pgfqpoint{0.536353in}{0.628636in}}%
\pgfpathlineto{\pgfqpoint{0.728363in}{0.406500in}}%
\pgfusepath{stroke}%
\end{pgfscope}%
\begin{pgfscope}%
\pgfpathrectangle{\pgfqpoint{0.152333in}{0.150000in}}{\pgfqpoint{4.224218in}{2.565000in}}%
\pgfusepath{clip}%
\pgfsetbuttcap%
\pgfsetroundjoin%
\pgfsetlinewidth{0.501875pt}%
\definecolor{currentstroke}{rgb}{0.000000,0.000000,1.000000}%
\pgfsetstrokecolor{currentstroke}%
\pgfsetdash{{0.500000pt}{0.825000pt}}{0.000000pt}%
\pgfpathmoveto{\pgfqpoint{4.184541in}{0.406500in}}%
\pgfpathlineto{\pgfqpoint{4.184541in}{0.406500in}}%
\pgfusepath{stroke}%
\end{pgfscope}%
\begin{pgfscope}%
\pgfpathrectangle{\pgfqpoint{0.152333in}{0.150000in}}{\pgfqpoint{4.224218in}{2.565000in}}%
\pgfusepath{clip}%
\pgfsetbuttcap%
\pgfsetroundjoin%
\pgfsetlinewidth{0.501875pt}%
\definecolor{currentstroke}{rgb}{0.000000,0.000000,1.000000}%
\pgfsetstrokecolor{currentstroke}%
\pgfsetdash{{0.500000pt}{0.825000pt}}{0.000000pt}%
\pgfpathmoveto{\pgfqpoint{0.344343in}{0.406500in}}%
\pgfpathlineto{\pgfqpoint{0.344343in}{0.406500in}}%
\pgfusepath{stroke}%
\end{pgfscope}%
\begin{pgfscope}%
\pgfpathrectangle{\pgfqpoint{0.152333in}{0.150000in}}{\pgfqpoint{4.224218in}{2.565000in}}%
\pgfusepath{clip}%
\pgfsetrectcap%
\pgfsetroundjoin%
\pgfsetlinewidth{1.003750pt}%
\definecolor{currentstroke}{rgb}{0.000000,0.000000,0.000000}%
\pgfsetstrokecolor{currentstroke}%
\pgfsetdash{}{0pt}%
\pgfpathmoveto{\pgfqpoint{4.184541in}{0.406500in}}%
\pgfpathlineto{\pgfqpoint{4.222943in}{0.406500in}}%
\pgfusepath{stroke}%
\end{pgfscope}%
\begin{pgfscope}%
\pgfpathrectangle{\pgfqpoint{0.152333in}{0.150000in}}{\pgfqpoint{4.224218in}{2.565000in}}%
\pgfusepath{clip}%
\pgfsetrectcap%
\pgfsetroundjoin%
\pgfsetlinewidth{1.003750pt}%
\definecolor{currentstroke}{rgb}{0.000000,0.000000,0.000000}%
\pgfsetstrokecolor{currentstroke}%
\pgfsetdash{}{0pt}%
\pgfpathmoveto{\pgfqpoint{3.992531in}{0.628636in}}%
\pgfpathlineto{\pgfqpoint{4.030933in}{0.628636in}}%
\pgfusepath{stroke}%
\end{pgfscope}%
\begin{pgfscope}%
\pgfpathrectangle{\pgfqpoint{0.152333in}{0.150000in}}{\pgfqpoint{4.224218in}{2.565000in}}%
\pgfusepath{clip}%
\pgfsetrectcap%
\pgfsetroundjoin%
\pgfsetlinewidth{1.003750pt}%
\definecolor{currentstroke}{rgb}{0.000000,0.000000,0.000000}%
\pgfsetstrokecolor{currentstroke}%
\pgfsetdash{}{0pt}%
\pgfpathmoveto{\pgfqpoint{3.800522in}{0.850771in}}%
\pgfpathlineto{\pgfqpoint{3.838924in}{0.850771in}}%
\pgfusepath{stroke}%
\end{pgfscope}%
\begin{pgfscope}%
\pgfpathrectangle{\pgfqpoint{0.152333in}{0.150000in}}{\pgfqpoint{4.224218in}{2.565000in}}%
\pgfusepath{clip}%
\pgfsetrectcap%
\pgfsetroundjoin%
\pgfsetlinewidth{1.003750pt}%
\definecolor{currentstroke}{rgb}{0.000000,0.000000,0.000000}%
\pgfsetstrokecolor{currentstroke}%
\pgfsetdash{}{0pt}%
\pgfpathmoveto{\pgfqpoint{3.608512in}{1.072907in}}%
\pgfpathlineto{\pgfqpoint{3.646914in}{1.072907in}}%
\pgfusepath{stroke}%
\end{pgfscope}%
\begin{pgfscope}%
\pgfpathrectangle{\pgfqpoint{0.152333in}{0.150000in}}{\pgfqpoint{4.224218in}{2.565000in}}%
\pgfusepath{clip}%
\pgfsetrectcap%
\pgfsetroundjoin%
\pgfsetlinewidth{1.003750pt}%
\definecolor{currentstroke}{rgb}{0.000000,0.000000,0.000000}%
\pgfsetstrokecolor{currentstroke}%
\pgfsetdash{}{0pt}%
\pgfpathmoveto{\pgfqpoint{3.416502in}{1.295042in}}%
\pgfpathlineto{\pgfqpoint{3.454904in}{1.295042in}}%
\pgfusepath{stroke}%
\end{pgfscope}%
\begin{pgfscope}%
\pgfpathrectangle{\pgfqpoint{0.152333in}{0.150000in}}{\pgfqpoint{4.224218in}{2.565000in}}%
\pgfusepath{clip}%
\pgfsetrectcap%
\pgfsetroundjoin%
\pgfsetlinewidth{1.003750pt}%
\definecolor{currentstroke}{rgb}{0.000000,0.000000,0.000000}%
\pgfsetstrokecolor{currentstroke}%
\pgfsetdash{}{0pt}%
\pgfpathmoveto{\pgfqpoint{3.224492in}{1.517178in}}%
\pgfpathlineto{\pgfqpoint{3.262894in}{1.517178in}}%
\pgfusepath{stroke}%
\end{pgfscope}%
\begin{pgfscope}%
\pgfpathrectangle{\pgfqpoint{0.152333in}{0.150000in}}{\pgfqpoint{4.224218in}{2.565000in}}%
\pgfusepath{clip}%
\pgfsetrectcap%
\pgfsetroundjoin%
\pgfsetlinewidth{1.003750pt}%
\definecolor{currentstroke}{rgb}{0.000000,0.000000,0.000000}%
\pgfsetstrokecolor{currentstroke}%
\pgfsetdash{}{0pt}%
\pgfpathmoveto{\pgfqpoint{3.032482in}{1.739313in}}%
\pgfpathlineto{\pgfqpoint{3.070884in}{1.739313in}}%
\pgfusepath{stroke}%
\end{pgfscope}%
\begin{pgfscope}%
\pgfpathrectangle{\pgfqpoint{0.152333in}{0.150000in}}{\pgfqpoint{4.224218in}{2.565000in}}%
\pgfusepath{clip}%
\pgfsetrectcap%
\pgfsetroundjoin%
\pgfsetlinewidth{1.003750pt}%
\definecolor{currentstroke}{rgb}{0.000000,0.000000,0.000000}%
\pgfsetstrokecolor{currentstroke}%
\pgfsetdash{}{0pt}%
\pgfpathmoveto{\pgfqpoint{2.840472in}{1.961449in}}%
\pgfpathlineto{\pgfqpoint{2.878874in}{1.961449in}}%
\pgfusepath{stroke}%
\end{pgfscope}%
\begin{pgfscope}%
\pgfpathrectangle{\pgfqpoint{0.152333in}{0.150000in}}{\pgfqpoint{4.224218in}{2.565000in}}%
\pgfusepath{clip}%
\pgfsetrectcap%
\pgfsetroundjoin%
\pgfsetlinewidth{1.003750pt}%
\definecolor{currentstroke}{rgb}{0.000000,0.000000,0.000000}%
\pgfsetstrokecolor{currentstroke}%
\pgfsetdash{}{0pt}%
\pgfpathmoveto{\pgfqpoint{2.648462in}{2.183584in}}%
\pgfpathlineto{\pgfqpoint{2.686864in}{2.183584in}}%
\pgfusepath{stroke}%
\end{pgfscope}%
\begin{pgfscope}%
\pgfpathrectangle{\pgfqpoint{0.152333in}{0.150000in}}{\pgfqpoint{4.224218in}{2.565000in}}%
\pgfusepath{clip}%
\pgfsetrectcap%
\pgfsetroundjoin%
\pgfsetlinewidth{1.003750pt}%
\definecolor{currentstroke}{rgb}{0.000000,0.000000,0.000000}%
\pgfsetstrokecolor{currentstroke}%
\pgfsetdash{}{0pt}%
\pgfpathmoveto{\pgfqpoint{2.456452in}{2.405720in}}%
\pgfpathlineto{\pgfqpoint{2.494854in}{2.405720in}}%
\pgfusepath{stroke}%
\end{pgfscope}%
\begin{pgfscope}%
\pgfpathrectangle{\pgfqpoint{0.152333in}{0.150000in}}{\pgfqpoint{4.224218in}{2.565000in}}%
\pgfusepath{clip}%
\pgfsetrectcap%
\pgfsetroundjoin%
\pgfsetlinewidth{1.003750pt}%
\definecolor{currentstroke}{rgb}{0.000000,0.000000,0.000000}%
\pgfsetstrokecolor{currentstroke}%
\pgfsetdash{}{0pt}%
\pgfpathmoveto{\pgfqpoint{2.264442in}{2.627855in}}%
\pgfpathlineto{\pgfqpoint{2.302844in}{2.627855in}}%
\pgfusepath{stroke}%
\end{pgfscope}%
\begin{pgfscope}%
\pgfpathrectangle{\pgfqpoint{0.152333in}{0.150000in}}{\pgfqpoint{4.224218in}{2.565000in}}%
\pgfusepath{clip}%
\pgfsetrectcap%
\pgfsetroundjoin%
\pgfsetlinewidth{1.003750pt}%
\definecolor{currentstroke}{rgb}{0.000000,0.000000,0.000000}%
\pgfsetstrokecolor{currentstroke}%
\pgfsetdash{}{0pt}%
\pgfpathmoveto{\pgfqpoint{0.344343in}{0.406500in}}%
\pgfpathlineto{\pgfqpoint{0.325142in}{0.428714in}}%
\pgfusepath{stroke}%
\end{pgfscope}%
\begin{pgfscope}%
\pgfpathrectangle{\pgfqpoint{0.152333in}{0.150000in}}{\pgfqpoint{4.224218in}{2.565000in}}%
\pgfusepath{clip}%
\pgfsetrectcap%
\pgfsetroundjoin%
\pgfsetlinewidth{1.003750pt}%
\definecolor{currentstroke}{rgb}{0.000000,0.000000,0.000000}%
\pgfsetstrokecolor{currentstroke}%
\pgfsetdash{}{0pt}%
\pgfpathmoveto{\pgfqpoint{0.536353in}{0.628636in}}%
\pgfpathlineto{\pgfqpoint{0.517152in}{0.650849in}}%
\pgfusepath{stroke}%
\end{pgfscope}%
\begin{pgfscope}%
\pgfpathrectangle{\pgfqpoint{0.152333in}{0.150000in}}{\pgfqpoint{4.224218in}{2.565000in}}%
\pgfusepath{clip}%
\pgfsetrectcap%
\pgfsetroundjoin%
\pgfsetlinewidth{1.003750pt}%
\definecolor{currentstroke}{rgb}{0.000000,0.000000,0.000000}%
\pgfsetstrokecolor{currentstroke}%
\pgfsetdash{}{0pt}%
\pgfpathmoveto{\pgfqpoint{0.728363in}{0.850771in}}%
\pgfpathlineto{\pgfqpoint{0.709162in}{0.872985in}}%
\pgfusepath{stroke}%
\end{pgfscope}%
\begin{pgfscope}%
\pgfpathrectangle{\pgfqpoint{0.152333in}{0.150000in}}{\pgfqpoint{4.224218in}{2.565000in}}%
\pgfusepath{clip}%
\pgfsetrectcap%
\pgfsetroundjoin%
\pgfsetlinewidth{1.003750pt}%
\definecolor{currentstroke}{rgb}{0.000000,0.000000,0.000000}%
\pgfsetstrokecolor{currentstroke}%
\pgfsetdash{}{0pt}%
\pgfpathmoveto{\pgfqpoint{0.920373in}{1.072907in}}%
\pgfpathlineto{\pgfqpoint{0.901172in}{1.095120in}}%
\pgfusepath{stroke}%
\end{pgfscope}%
\begin{pgfscope}%
\pgfpathrectangle{\pgfqpoint{0.152333in}{0.150000in}}{\pgfqpoint{4.224218in}{2.565000in}}%
\pgfusepath{clip}%
\pgfsetrectcap%
\pgfsetroundjoin%
\pgfsetlinewidth{1.003750pt}%
\definecolor{currentstroke}{rgb}{0.000000,0.000000,0.000000}%
\pgfsetstrokecolor{currentstroke}%
\pgfsetdash{}{0pt}%
\pgfpathmoveto{\pgfqpoint{1.112383in}{1.295042in}}%
\pgfpathlineto{\pgfqpoint{1.093182in}{1.317256in}}%
\pgfusepath{stroke}%
\end{pgfscope}%
\begin{pgfscope}%
\pgfpathrectangle{\pgfqpoint{0.152333in}{0.150000in}}{\pgfqpoint{4.224218in}{2.565000in}}%
\pgfusepath{clip}%
\pgfsetrectcap%
\pgfsetroundjoin%
\pgfsetlinewidth{1.003750pt}%
\definecolor{currentstroke}{rgb}{0.000000,0.000000,0.000000}%
\pgfsetstrokecolor{currentstroke}%
\pgfsetdash{}{0pt}%
\pgfpathmoveto{\pgfqpoint{1.304393in}{1.517178in}}%
\pgfpathlineto{\pgfqpoint{1.285192in}{1.539391in}}%
\pgfusepath{stroke}%
\end{pgfscope}%
\begin{pgfscope}%
\pgfpathrectangle{\pgfqpoint{0.152333in}{0.150000in}}{\pgfqpoint{4.224218in}{2.565000in}}%
\pgfusepath{clip}%
\pgfsetrectcap%
\pgfsetroundjoin%
\pgfsetlinewidth{1.003750pt}%
\definecolor{currentstroke}{rgb}{0.000000,0.000000,0.000000}%
\pgfsetstrokecolor{currentstroke}%
\pgfsetdash{}{0pt}%
\pgfpathmoveto{\pgfqpoint{1.496402in}{1.739313in}}%
\pgfpathlineto{\pgfqpoint{1.477201in}{1.761527in}}%
\pgfusepath{stroke}%
\end{pgfscope}%
\begin{pgfscope}%
\pgfpathrectangle{\pgfqpoint{0.152333in}{0.150000in}}{\pgfqpoint{4.224218in}{2.565000in}}%
\pgfusepath{clip}%
\pgfsetrectcap%
\pgfsetroundjoin%
\pgfsetlinewidth{1.003750pt}%
\definecolor{currentstroke}{rgb}{0.000000,0.000000,0.000000}%
\pgfsetstrokecolor{currentstroke}%
\pgfsetdash{}{0pt}%
\pgfpathmoveto{\pgfqpoint{1.688412in}{1.961449in}}%
\pgfpathlineto{\pgfqpoint{1.669211in}{1.983662in}}%
\pgfusepath{stroke}%
\end{pgfscope}%
\begin{pgfscope}%
\pgfpathrectangle{\pgfqpoint{0.152333in}{0.150000in}}{\pgfqpoint{4.224218in}{2.565000in}}%
\pgfusepath{clip}%
\pgfsetrectcap%
\pgfsetroundjoin%
\pgfsetlinewidth{1.003750pt}%
\definecolor{currentstroke}{rgb}{0.000000,0.000000,0.000000}%
\pgfsetstrokecolor{currentstroke}%
\pgfsetdash{}{0pt}%
\pgfpathmoveto{\pgfqpoint{1.880422in}{2.183584in}}%
\pgfpathlineto{\pgfqpoint{1.861221in}{2.205798in}}%
\pgfusepath{stroke}%
\end{pgfscope}%
\begin{pgfscope}%
\pgfpathrectangle{\pgfqpoint{0.152333in}{0.150000in}}{\pgfqpoint{4.224218in}{2.565000in}}%
\pgfusepath{clip}%
\pgfsetrectcap%
\pgfsetroundjoin%
\pgfsetlinewidth{1.003750pt}%
\definecolor{currentstroke}{rgb}{0.000000,0.000000,0.000000}%
\pgfsetstrokecolor{currentstroke}%
\pgfsetdash{}{0pt}%
\pgfpathmoveto{\pgfqpoint{2.072432in}{2.405720in}}%
\pgfpathlineto{\pgfqpoint{2.053231in}{2.427933in}}%
\pgfusepath{stroke}%
\end{pgfscope}%
\begin{pgfscope}%
\pgfpathrectangle{\pgfqpoint{0.152333in}{0.150000in}}{\pgfqpoint{4.224218in}{2.565000in}}%
\pgfusepath{clip}%
\pgfsetrectcap%
\pgfsetroundjoin%
\pgfsetlinewidth{1.003750pt}%
\definecolor{currentstroke}{rgb}{0.000000,0.000000,0.000000}%
\pgfsetstrokecolor{currentstroke}%
\pgfsetdash{}{0pt}%
\pgfpathmoveto{\pgfqpoint{2.264442in}{2.627855in}}%
\pgfpathlineto{\pgfqpoint{2.245241in}{2.650069in}}%
\pgfusepath{stroke}%
\end{pgfscope}%
\begin{pgfscope}%
\pgfpathrectangle{\pgfqpoint{0.152333in}{0.150000in}}{\pgfqpoint{4.224218in}{2.565000in}}%
\pgfusepath{clip}%
\pgfsetrectcap%
\pgfsetroundjoin%
\pgfsetlinewidth{1.003750pt}%
\definecolor{currentstroke}{rgb}{0.000000,0.000000,0.000000}%
\pgfsetstrokecolor{currentstroke}%
\pgfsetdash{}{0pt}%
\pgfpathmoveto{\pgfqpoint{0.344343in}{0.406500in}}%
\pgfpathlineto{\pgfqpoint{0.325142in}{0.384286in}}%
\pgfusepath{stroke}%
\end{pgfscope}%
\begin{pgfscope}%
\pgfpathrectangle{\pgfqpoint{0.152333in}{0.150000in}}{\pgfqpoint{4.224218in}{2.565000in}}%
\pgfusepath{clip}%
\pgfsetrectcap%
\pgfsetroundjoin%
\pgfsetlinewidth{1.003750pt}%
\definecolor{currentstroke}{rgb}{0.000000,0.000000,0.000000}%
\pgfsetstrokecolor{currentstroke}%
\pgfsetdash{}{0pt}%
\pgfpathmoveto{\pgfqpoint{0.728363in}{0.406500in}}%
\pgfpathlineto{\pgfqpoint{0.709162in}{0.384286in}}%
\pgfusepath{stroke}%
\end{pgfscope}%
\begin{pgfscope}%
\pgfpathrectangle{\pgfqpoint{0.152333in}{0.150000in}}{\pgfqpoint{4.224218in}{2.565000in}}%
\pgfusepath{clip}%
\pgfsetrectcap%
\pgfsetroundjoin%
\pgfsetlinewidth{1.003750pt}%
\definecolor{currentstroke}{rgb}{0.000000,0.000000,0.000000}%
\pgfsetstrokecolor{currentstroke}%
\pgfsetdash{}{0pt}%
\pgfpathmoveto{\pgfqpoint{1.112383in}{0.406500in}}%
\pgfpathlineto{\pgfqpoint{1.093182in}{0.384286in}}%
\pgfusepath{stroke}%
\end{pgfscope}%
\begin{pgfscope}%
\pgfpathrectangle{\pgfqpoint{0.152333in}{0.150000in}}{\pgfqpoint{4.224218in}{2.565000in}}%
\pgfusepath{clip}%
\pgfsetrectcap%
\pgfsetroundjoin%
\pgfsetlinewidth{1.003750pt}%
\definecolor{currentstroke}{rgb}{0.000000,0.000000,0.000000}%
\pgfsetstrokecolor{currentstroke}%
\pgfsetdash{}{0pt}%
\pgfpathmoveto{\pgfqpoint{1.496402in}{0.406500in}}%
\pgfpathlineto{\pgfqpoint{1.477201in}{0.384286in}}%
\pgfusepath{stroke}%
\end{pgfscope}%
\begin{pgfscope}%
\pgfpathrectangle{\pgfqpoint{0.152333in}{0.150000in}}{\pgfqpoint{4.224218in}{2.565000in}}%
\pgfusepath{clip}%
\pgfsetrectcap%
\pgfsetroundjoin%
\pgfsetlinewidth{1.003750pt}%
\definecolor{currentstroke}{rgb}{0.000000,0.000000,0.000000}%
\pgfsetstrokecolor{currentstroke}%
\pgfsetdash{}{0pt}%
\pgfpathmoveto{\pgfqpoint{1.880422in}{0.406500in}}%
\pgfpathlineto{\pgfqpoint{1.861221in}{0.384286in}}%
\pgfusepath{stroke}%
\end{pgfscope}%
\begin{pgfscope}%
\pgfpathrectangle{\pgfqpoint{0.152333in}{0.150000in}}{\pgfqpoint{4.224218in}{2.565000in}}%
\pgfusepath{clip}%
\pgfsetrectcap%
\pgfsetroundjoin%
\pgfsetlinewidth{1.003750pt}%
\definecolor{currentstroke}{rgb}{0.000000,0.000000,0.000000}%
\pgfsetstrokecolor{currentstroke}%
\pgfsetdash{}{0pt}%
\pgfpathmoveto{\pgfqpoint{2.264442in}{0.406500in}}%
\pgfpathlineto{\pgfqpoint{2.245241in}{0.384286in}}%
\pgfusepath{stroke}%
\end{pgfscope}%
\begin{pgfscope}%
\pgfpathrectangle{\pgfqpoint{0.152333in}{0.150000in}}{\pgfqpoint{4.224218in}{2.565000in}}%
\pgfusepath{clip}%
\pgfsetrectcap%
\pgfsetroundjoin%
\pgfsetlinewidth{1.003750pt}%
\definecolor{currentstroke}{rgb}{0.000000,0.000000,0.000000}%
\pgfsetstrokecolor{currentstroke}%
\pgfsetdash{}{0pt}%
\pgfpathmoveto{\pgfqpoint{2.648462in}{0.406500in}}%
\pgfpathlineto{\pgfqpoint{2.629261in}{0.384286in}}%
\pgfusepath{stroke}%
\end{pgfscope}%
\begin{pgfscope}%
\pgfpathrectangle{\pgfqpoint{0.152333in}{0.150000in}}{\pgfqpoint{4.224218in}{2.565000in}}%
\pgfusepath{clip}%
\pgfsetrectcap%
\pgfsetroundjoin%
\pgfsetlinewidth{1.003750pt}%
\definecolor{currentstroke}{rgb}{0.000000,0.000000,0.000000}%
\pgfsetstrokecolor{currentstroke}%
\pgfsetdash{}{0pt}%
\pgfpathmoveto{\pgfqpoint{3.032482in}{0.406500in}}%
\pgfpathlineto{\pgfqpoint{3.013281in}{0.384286in}}%
\pgfusepath{stroke}%
\end{pgfscope}%
\begin{pgfscope}%
\pgfpathrectangle{\pgfqpoint{0.152333in}{0.150000in}}{\pgfqpoint{4.224218in}{2.565000in}}%
\pgfusepath{clip}%
\pgfsetrectcap%
\pgfsetroundjoin%
\pgfsetlinewidth{1.003750pt}%
\definecolor{currentstroke}{rgb}{0.000000,0.000000,0.000000}%
\pgfsetstrokecolor{currentstroke}%
\pgfsetdash{}{0pt}%
\pgfpathmoveto{\pgfqpoint{3.416502in}{0.406500in}}%
\pgfpathlineto{\pgfqpoint{3.397301in}{0.384286in}}%
\pgfusepath{stroke}%
\end{pgfscope}%
\begin{pgfscope}%
\pgfpathrectangle{\pgfqpoint{0.152333in}{0.150000in}}{\pgfqpoint{4.224218in}{2.565000in}}%
\pgfusepath{clip}%
\pgfsetrectcap%
\pgfsetroundjoin%
\pgfsetlinewidth{1.003750pt}%
\definecolor{currentstroke}{rgb}{0.000000,0.000000,0.000000}%
\pgfsetstrokecolor{currentstroke}%
\pgfsetdash{}{0pt}%
\pgfpathmoveto{\pgfqpoint{3.800522in}{0.406500in}}%
\pgfpathlineto{\pgfqpoint{3.781321in}{0.384286in}}%
\pgfusepath{stroke}%
\end{pgfscope}%
\begin{pgfscope}%
\pgfpathrectangle{\pgfqpoint{0.152333in}{0.150000in}}{\pgfqpoint{4.224218in}{2.565000in}}%
\pgfusepath{clip}%
\pgfsetrectcap%
\pgfsetroundjoin%
\pgfsetlinewidth{1.003750pt}%
\definecolor{currentstroke}{rgb}{0.000000,0.000000,0.000000}%
\pgfsetstrokecolor{currentstroke}%
\pgfsetdash{}{0pt}%
\pgfpathmoveto{\pgfqpoint{4.184541in}{0.406500in}}%
\pgfpathlineto{\pgfqpoint{4.165340in}{0.384286in}}%
\pgfusepath{stroke}%
\end{pgfscope}%
\begin{pgfscope}%
\pgfpathrectangle{\pgfqpoint{0.152333in}{0.150000in}}{\pgfqpoint{4.224218in}{2.565000in}}%
\pgfusepath{clip}%
\pgfsetrectcap%
\pgfsetroundjoin%
\pgfsetlinewidth{1.003750pt}%
\definecolor{currentstroke}{rgb}{0.000000,0.000000,1.000000}%
\pgfsetstrokecolor{currentstroke}%
\pgfsetdash{}{0pt}%
\pgfpathmoveto{\pgfqpoint{2.439334in}{1.402646in}}%
\pgfpathlineto{\pgfqpoint{2.362027in}{1.766946in}}%
\pgfpathlineto{\pgfqpoint{2.170171in}{2.046578in}}%
\pgfpathlineto{\pgfqpoint{1.991734in}{2.122456in}}%
\pgfpathlineto{\pgfqpoint{1.827568in}{2.030418in}}%
\pgfpathlineto{\pgfqpoint{1.467288in}{1.616124in}}%
\pgfpathlineto{\pgfqpoint{1.265743in}{1.383422in}}%
\pgfpathlineto{\pgfqpoint{1.081594in}{1.158172in}}%
\pgfpathlineto{\pgfqpoint{0.957110in}{0.959830in}}%
\pgfpathlineto{\pgfqpoint{0.953349in}{0.804875in}}%
\pgfpathlineto{\pgfqpoint{1.196428in}{0.697988in}}%
\pgfpathlineto{\pgfqpoint{1.863268in}{0.642868in}}%
\pgfpathlineto{\pgfqpoint{2.797494in}{0.667217in}}%
\pgfpathlineto{\pgfqpoint{3.327006in}{0.875409in}}%
\pgfpathlineto{\pgfqpoint{3.139064in}{1.435340in}}%
\pgfpathlineto{\pgfqpoint{2.633535in}{2.113798in}}%
\pgfpathlineto{\pgfqpoint{2.348528in}{2.445560in}}%
\pgfpathlineto{\pgfqpoint{2.242879in}{2.530176in}}%
\pgfpathlineto{\pgfqpoint{2.198664in}{2.514085in}}%
\pgfpathlineto{\pgfqpoint{1.983055in}{2.265902in}}%
\pgfpathlineto{\pgfqpoint{1.233103in}{1.399276in}}%
\pgfpathlineto{\pgfqpoint{1.022700in}{1.151325in}}%
\pgfpathlineto{\pgfqpoint{0.857826in}{0.937461in}}%
\pgfpathlineto{\pgfqpoint{0.774272in}{0.771723in}}%
\pgfpathlineto{\pgfqpoint{0.849087in}{0.654436in}}%
\pgfpathlineto{\pgfqpoint{1.273366in}{0.579863in}}%
\pgfpathlineto{\pgfqpoint{2.229947in}{0.548410in}}%
\pgfpathlineto{\pgfqpoint{3.233140in}{0.591868in}}%
\pgfpathlineto{\pgfqpoint{3.570693in}{0.829945in}}%
\pgfpathlineto{\pgfqpoint{3.199642in}{1.443445in}}%
\pgfpathlineto{\pgfqpoint{2.619622in}{2.128422in}}%
\pgfpathlineto{\pgfqpoint{2.342545in}{2.449643in}}%
\pgfpathlineto{\pgfqpoint{2.239587in}{2.529423in}}%
\pgfpathlineto{\pgfqpoint{2.196435in}{2.509573in}}%
\pgfpathlineto{\pgfqpoint{1.975253in}{2.255063in}}%
\pgfpathlineto{\pgfqpoint{1.220751in}{1.383239in}}%
\pgfpathlineto{\pgfqpoint{1.013172in}{1.137426in}}%
\pgfpathlineto{\pgfqpoint{0.853556in}{0.926748in}}%
\pgfpathlineto{\pgfqpoint{0.779932in}{0.764377in}}%
\pgfpathlineto{\pgfqpoint{0.877062in}{0.650159in}}%
\pgfpathlineto{\pgfqpoint{1.346904in}{0.578599in}}%
\pgfpathlineto{\pgfqpoint{2.337662in}{0.551987in}}%
\pgfpathlineto{\pgfqpoint{3.292749in}{0.608264in}}%
\pgfpathlineto{\pgfqpoint{3.552617in}{0.880857in}}%
\pgfpathlineto{\pgfqpoint{3.125242in}{1.529793in}}%
\pgfpathlineto{\pgfqpoint{2.573842in}{2.184334in}}%
\pgfpathlineto{\pgfqpoint{2.325671in}{2.465581in}}%
\pgfpathlineto{\pgfqpoint{2.232308in}{2.530935in}}%
\pgfpathlineto{\pgfqpoint{2.190445in}{2.504988in}}%
\pgfpathlineto{\pgfqpoint{1.957254in}{2.236427in}}%
\pgfpathlineto{\pgfqpoint{1.186090in}{1.345185in}}%
\pgfpathlineto{\pgfqpoint{0.982736in}{1.102403in}}%
\pgfpathlineto{\pgfqpoint{0.831399in}{0.898066in}}%
\pgfpathlineto{\pgfqpoint{0.771989in}{0.742882in}}%
\pgfpathlineto{\pgfqpoint{0.899114in}{0.635120in}}%
\pgfpathlineto{\pgfqpoint{1.430031in}{0.569052in}}%
\pgfpathlineto{\pgfqpoint{2.467202in}{0.548811in}}%
\pgfpathlineto{\pgfqpoint{3.368890in}{0.618407in}}%
\pgfpathlineto{\pgfqpoint{3.546290in}{0.923856in}}%
\pgfpathlineto{\pgfqpoint{3.062519in}{1.599484in}}%
\pgfpathlineto{\pgfqpoint{2.540374in}{2.225703in}}%
\pgfpathlineto{\pgfqpoint{2.313906in}{2.476704in}}%
\pgfpathlineto{\pgfqpoint{2.227257in}{2.531882in}}%
\pgfpathlineto{\pgfqpoint{2.186451in}{2.501401in}}%
\pgfpathlineto{\pgfqpoint{1.945153in}{2.223392in}}%
\pgfpathlineto{\pgfqpoint{1.164281in}{1.320853in}}%
\pgfpathlineto{\pgfqpoint{0.964063in}{1.080483in}}%
\pgfpathlineto{\pgfqpoint{0.818500in}{0.880422in}}%
\pgfpathlineto{\pgfqpoint{0.769178in}{0.729861in}}%
\pgfpathlineto{\pgfqpoint{0.918128in}{0.626169in}}%
\pgfpathlineto{\pgfqpoint{1.491770in}{0.563624in}}%
\pgfpathlineto{\pgfqpoint{2.554203in}{0.547832in}}%
\pgfpathlineto{\pgfqpoint{3.413854in}{0.627414in}}%
\pgfpathlineto{\pgfqpoint{3.535754in}{0.957132in}}%
\pgfpathlineto{\pgfqpoint{3.020138in}{1.649621in}}%
\pgfpathlineto{\pgfqpoint{2.517837in}{2.254192in}}%
\pgfpathlineto{\pgfqpoint{2.306081in}{2.485183in}}%
\pgfpathlineto{\pgfqpoint{2.224421in}{2.533382in}}%
\pgfpathlineto{\pgfqpoint{2.185005in}{2.499822in}}%
\pgfpathlineto{\pgfqpoint{1.940711in}{2.218339in}}%
\pgfpathlineto{\pgfqpoint{1.156784in}{1.312252in}}%
\pgfpathlineto{\pgfqpoint{0.957848in}{1.072879in}}%
\pgfpathlineto{\pgfqpoint{0.814588in}{0.874405in}}%
\pgfpathlineto{\pgfqpoint{0.769437in}{0.725504in}}%
\pgfpathlineto{\pgfqpoint{0.927626in}{0.623270in}}%
\pgfpathlineto{\pgfqpoint{1.518698in}{0.562042in}}%
\pgfpathlineto{\pgfqpoint{2.589173in}{0.548127in}}%
\pgfpathlineto{\pgfqpoint{3.429898in}{0.632282in}}%
\pgfpathlineto{\pgfqpoint{3.528728in}{0.973037in}}%
\pgfpathlineto{\pgfqpoint{3.002896in}{1.672987in}}%
\pgfpathlineto{\pgfqpoint{2.508397in}{2.267875in}}%
\pgfpathlineto{\pgfqpoint{2.303576in}{2.489341in}}%
\pgfpathlineto{\pgfqpoint{2.224211in}{2.535084in}}%
\pgfpathlineto{\pgfqpoint{2.186721in}{2.499950in}}%
\pgfpathlineto{\pgfqpoint{1.945481in}{2.222116in}}%
\pgfpathlineto{\pgfqpoint{1.166871in}{1.322263in}}%
\pgfpathlineto{\pgfqpoint{0.967264in}{1.082299in}}%
\pgfpathlineto{\pgfqpoint{0.822766in}{0.882316in}}%
\pgfpathlineto{\pgfqpoint{0.776126in}{0.731695in}}%
\pgfpathlineto{\pgfqpoint{0.931831in}{0.627998in}}%
\pgfpathlineto{\pgfqpoint{1.516785in}{0.565798in}}%
\pgfpathlineto{\pgfqpoint{2.579627in}{0.551586in}}%
\pgfpathlineto{\pgfqpoint{3.419893in}{0.636680in}}%
\pgfpathlineto{\pgfqpoint{3.519616in}{0.980279in}}%
\pgfpathlineto{\pgfqpoint{2.994365in}{1.680691in}}%
\pgfpathlineto{\pgfqpoint{2.505262in}{2.271888in}}%
\pgfpathlineto{\pgfqpoint{2.302468in}{2.490764in}}%
\pgfpathlineto{\pgfqpoint{2.223921in}{2.535481in}}%
\pgfpathlineto{\pgfqpoint{2.186454in}{2.500239in}}%
\pgfpathlineto{\pgfqpoint{1.944828in}{2.221921in}}%
\pgfpathlineto{\pgfqpoint{1.165088in}{1.320735in}}%
\pgfpathlineto{\pgfqpoint{0.965435in}{1.080756in}}%
\pgfpathlineto{\pgfqpoint{0.820871in}{0.880942in}}%
\pgfpathlineto{\pgfqpoint{0.773869in}{0.730546in}}%
\pgfpathlineto{\pgfqpoint{0.928512in}{0.627031in}}%
\pgfpathlineto{\pgfqpoint{1.511960in}{0.564881in}}%
\pgfpathlineto{\pgfqpoint{2.575739in}{0.550380in}}%
\pgfpathlineto{\pgfqpoint{3.420000in}{0.634227in}}%
\pgfpathlineto{\pgfqpoint{3.524074in}{0.974655in}}%
\pgfpathlineto{\pgfqpoint{3.000616in}{1.673491in}}%
\pgfpathlineto{\pgfqpoint{2.508688in}{2.268342in}}%
\pgfpathlineto{\pgfqpoint{2.303922in}{2.489894in}}%
\pgfpathlineto{\pgfqpoint{2.224782in}{2.535811in}}%
\pgfpathlineto{\pgfqpoint{2.187612in}{2.501039in}}%
\pgfpathlineto{\pgfqpoint{1.948248in}{2.225373in}}%
\pgfpathlineto{\pgfqpoint{1.171452in}{1.327622in}}%
\pgfpathlineto{\pgfqpoint{0.970999in}{1.087020in}}%
\pgfpathlineto{\pgfqpoint{0.824963in}{0.886034in}}%
\pgfpathlineto{\pgfqpoint{0.775474in}{0.734361in}}%
\pgfpathlineto{\pgfqpoint{0.924831in}{0.629731in}}%
\pgfpathlineto{\pgfqpoint{1.497955in}{0.566657in}}%
\pgfpathlineto{\pgfqpoint{2.555657in}{0.551116in}}%
\pgfpathlineto{\pgfqpoint{3.409290in}{0.632854in}}%
\pgfpathlineto{\pgfqpoint{3.525318in}{0.968384in}}%
\pgfpathlineto{\pgfqpoint{3.007130in}{1.663642in}}%
\pgfpathlineto{\pgfqpoint{2.512067in}{2.262022in}}%
\pgfpathlineto{\pgfqpoint{2.304154in}{2.488132in}}%
\pgfpathlineto{\pgfqpoint{2.224138in}{2.534478in}}%
\pgfpathlineto{\pgfqpoint{2.186483in}{2.499247in}}%
\pgfpathlineto{\pgfqpoint{1.944627in}{2.220726in}}%
\pgfpathlineto{\pgfqpoint{1.165938in}{1.320796in}}%
\pgfpathlineto{\pgfqpoint{0.966758in}{1.081142in}}%
\pgfpathlineto{\pgfqpoint{0.823023in}{0.881513in}}%
\pgfpathlineto{\pgfqpoint{0.777963in}{0.731231in}}%
\pgfpathlineto{\pgfqpoint{0.937355in}{0.627843in}}%
\pgfpathlineto{\pgfqpoint{1.528797in}{0.566015in}}%
\pgfpathlineto{\pgfqpoint{2.593226in}{0.552616in}}%
\pgfpathlineto{\pgfqpoint{3.424568in}{0.640118in}}%
\pgfpathlineto{\pgfqpoint{3.513916in}{0.989565in}}%
\pgfpathlineto{\pgfqpoint{2.984390in}{1.693227in}}%
\pgfpathlineto{\pgfqpoint{2.499786in}{2.278482in}}%
\pgfpathlineto{\pgfqpoint{2.300422in}{2.492893in}}%
\pgfpathlineto{\pgfqpoint{2.223143in}{2.535786in}}%
\pgfpathlineto{\pgfqpoint{2.185554in}{2.500178in}}%
\pgfpathlineto{\pgfqpoint{1.942324in}{2.219944in}}%
\pgfpathlineto{\pgfqpoint{1.159774in}{1.315465in}}%
\pgfpathlineto{\pgfqpoint{0.960460in}{1.075796in}}%
\pgfpathlineto{\pgfqpoint{0.816517in}{0.876776in}}%
\pgfpathlineto{\pgfqpoint{0.770208in}{0.727291in}}%
\pgfpathlineto{\pgfqpoint{0.925898in}{0.624559in}}%
\pgfpathlineto{\pgfqpoint{1.512056in}{0.562944in}}%
\pgfpathlineto{\pgfqpoint{2.579692in}{0.548634in}}%
\pgfpathlineto{\pgfqpoint{3.424715in}{0.632047in}}%
\pgfpathlineto{\pgfqpoint{3.528624in}{0.971077in}}%
\pgfpathlineto{\pgfqpoint{3.004975in}{1.669651in}}%
\pgfpathlineto{\pgfqpoint{2.509540in}{2.265644in}}%
\pgfpathlineto{\pgfqpoint{2.303431in}{2.489596in}}%
\pgfpathlineto{\pgfqpoint{2.224231in}{2.535207in}}%
\pgfpathlineto{\pgfqpoint{2.186783in}{2.500100in}}%
\pgfpathlineto{\pgfqpoint{1.945735in}{2.222483in}}%
\pgfpathlineto{\pgfqpoint{1.167291in}{1.322814in}}%
\pgfpathlineto{\pgfqpoint{0.967734in}{1.082749in}}%
\pgfpathlineto{\pgfqpoint{0.823393in}{0.882616in}}%
\pgfpathlineto{\pgfqpoint{0.777417in}{0.731797in}}%
\pgfpathlineto{\pgfqpoint{0.935282in}{0.627817in}}%
\pgfpathlineto{\pgfqpoint{1.524622in}{0.565165in}}%
\pgfpathlineto{\pgfqpoint{2.589077in}{0.550194in}}%
\pgfpathlineto{\pgfqpoint{3.425708in}{0.633603in}}%
\pgfpathlineto{\pgfqpoint{3.525093in}{0.973614in}}%
\pgfpathlineto{\pgfqpoint{3.000139in}{1.672691in}}%
\pgfpathlineto{\pgfqpoint{2.508665in}{2.268681in}}%
\pgfpathlineto{\pgfqpoint{2.303507in}{2.489465in}}%
\pgfpathlineto{\pgfqpoint{2.223923in}{2.534819in}}%
\pgfpathlineto{\pgfqpoint{2.186177in}{2.499378in}}%
\pgfpathlineto{\pgfqpoint{1.942949in}{2.219242in}}%
\pgfpathlineto{\pgfqpoint{1.160764in}{1.315255in}}%
\pgfpathlineto{\pgfqpoint{0.961404in}{1.075738in}}%
\pgfpathlineto{\pgfqpoint{0.817510in}{0.876828in}}%
\pgfpathlineto{\pgfqpoint{0.771047in}{0.727478in}}%
\pgfpathlineto{\pgfqpoint{0.925272in}{0.625011in}}%
\pgfpathlineto{\pgfqpoint{1.506550in}{0.563984in}}%
\pgfpathlineto{\pgfqpoint{2.569414in}{0.550948in}}%
\pgfpathlineto{\pgfqpoint{3.415424in}{0.637603in}}%
\pgfpathlineto{\pgfqpoint{3.516937in}{0.983977in}}%
\pgfpathlineto{\pgfqpoint{2.991568in}{1.685715in}}%
\pgfpathlineto{\pgfqpoint{2.503236in}{2.273805in}}%
\pgfpathlineto{\pgfqpoint{2.302329in}{2.491751in}}%
\pgfpathlineto{\pgfqpoint{2.224550in}{2.536437in}}%
\pgfpathlineto{\pgfqpoint{2.187571in}{2.501703in}}%
\pgfpathlineto{\pgfqpoint{1.948968in}{2.226865in}}%
\pgfpathlineto{\pgfqpoint{1.173206in}{1.330253in}}%
\pgfpathlineto{\pgfqpoint{0.973433in}{1.089212in}}%
\pgfpathlineto{\pgfqpoint{0.828983in}{0.887493in}}%
\pgfpathlineto{\pgfqpoint{0.785047in}{0.734754in}}%
\pgfpathlineto{\pgfqpoint{0.951724in}{0.628457in}}%
\pgfpathlineto{\pgfqpoint{1.559794in}{0.562636in}}%
\pgfpathlineto{\pgfqpoint{2.630341in}{0.542775in}}%
\pgfpathlineto{\pgfqpoint{3.452547in}{0.615469in}}%
\pgfpathlineto{\pgfqpoint{3.557518in}{0.931546in}}%
\pgfpathlineto{\pgfqpoint{3.043600in}{1.620665in}}%
\pgfpathlineto{\pgfqpoint{2.530082in}{2.239711in}}%
\pgfpathlineto{\pgfqpoint{2.305646in}{2.478234in}}%
\pgfpathlineto{\pgfqpoint{2.218766in}{2.525605in}}%
\pgfpathlineto{\pgfqpoint{2.176959in}{2.486142in}}%
\pgfpathlineto{\pgfqpoint{1.910793in}{2.179678in}}%
\pgfpathlineto{\pgfqpoint{1.106072in}{1.249856in}}%
\pgfpathlineto{\pgfqpoint{0.910520in}{1.019528in}}%
\pgfpathlineto{\pgfqpoint{0.770604in}{0.834345in}}%
\pgfpathlineto{\pgfqpoint{0.715677in}{0.700606in}}%
\pgfpathlineto{\pgfqpoint{0.822275in}{0.616065in}}%
\pgfpathlineto{\pgfqpoint{1.291825in}{0.581155in}}%
\pgfpathlineto{\pgfqpoint{2.296375in}{0.620397in}}%
\pgfpathlineto{\pgfqpoint{3.192329in}{0.833583in}}%
\pgfpathlineto{\pgfqpoint{3.202832in}{1.384014in}}%
\pgfpathlineto{\pgfqpoint{2.688431in}{2.049943in}}%
\pgfpathlineto{\pgfqpoint{2.291711in}{2.549400in}}%
\pgfpathlineto{\pgfqpoint{2.252067in}{2.579826in}}%
\pgfpathlineto{\pgfqpoint{2.233671in}{2.566947in}}%
\pgfpathlineto{\pgfqpoint{2.123692in}{2.439968in}}%
\pgfpathlineto{\pgfqpoint{2.049640in}{2.349467in}}%
\pgfpathlineto{\pgfqpoint{1.953288in}{2.215666in}}%
\pgfpathlineto{\pgfqpoint{1.849370in}{2.021915in}}%
\pgfpathlineto{\pgfqpoint{1.815500in}{1.736955in}}%
\pgfpathlineto{\pgfqpoint{2.084821in}{1.315584in}}%
\pgfpathlineto{\pgfqpoint{2.827494in}{0.838399in}}%
\pgfpathlineto{\pgfqpoint{3.493779in}{0.561682in}}%
\pgfpathlineto{\pgfqpoint{3.496330in}{0.514201in}}%
\pgfpathlineto{\pgfqpoint{2.338378in}{0.601485in}}%
\pgfpathlineto{\pgfqpoint{1.007944in}{0.623228in}}%
\pgfpathlineto{\pgfqpoint{0.561818in}{0.566093in}}%
\pgfpathlineto{\pgfqpoint{0.484099in}{0.522434in}}%
\pgfpathlineto{\pgfqpoint{0.482650in}{0.520721in}}%
\pgfpathlineto{\pgfqpoint{0.668040in}{0.735560in}}%
\pgfpathlineto{\pgfqpoint{1.644655in}{1.858469in}}%
\pgfpathlineto{\pgfqpoint{2.215254in}{2.530114in}}%
\pgfpathlineto{\pgfqpoint{2.264548in}{2.616431in}}%
\pgfpathlineto{\pgfqpoint{2.264411in}{2.617971in}}%
\pgfpathlineto{\pgfqpoint{2.266117in}{2.606880in}}%
\pgfpathlineto{\pgfqpoint{2.279633in}{2.581451in}}%
\pgfpathlineto{\pgfqpoint{2.345799in}{2.490955in}}%
\pgfpathlineto{\pgfqpoint{2.626120in}{2.153434in}}%
\pgfpathlineto{\pgfqpoint{3.307922in}{1.364721in}}%
\pgfpathlineto{\pgfqpoint{3.869816in}{0.715961in}}%
\pgfpathlineto{\pgfqpoint{3.982847in}{0.581349in}}%
\pgfpathlineto{\pgfqpoint{3.634675in}{0.823196in}}%
\pgfpathlineto{\pgfqpoint{2.686514in}{1.342755in}}%
\pgfpathlineto{\pgfqpoint{1.778466in}{1.570058in}}%
\pgfpathlineto{\pgfqpoint{1.306491in}{1.422321in}}%
\pgfpathlineto{\pgfqpoint{0.887869in}{0.945321in}}%
\pgfpathlineto{\pgfqpoint{0.730698in}{0.776392in}}%
\pgfpathlineto{\pgfqpoint{0.624802in}{0.663951in}}%
\pgfpathlineto{\pgfqpoint{0.581888in}{0.619894in}}%
\pgfpathlineto{\pgfqpoint{0.653952in}{0.704210in}}%
\pgfpathlineto{\pgfqpoint{1.261147in}{1.407881in}}%
\pgfpathlineto{\pgfqpoint{2.118935in}{2.399768in}}%
\pgfpathlineto{\pgfqpoint{2.258386in}{2.600139in}}%
\pgfpathlineto{\pgfqpoint{2.264468in}{2.617088in}}%
\pgfpathlineto{\pgfqpoint{2.263374in}{2.615824in}}%
\pgfpathlineto{\pgfqpoint{2.263411in}{2.608829in}}%
\pgfpathlineto{\pgfqpoint{2.269800in}{2.590906in}}%
\pgfpathlineto{\pgfqpoint{2.305375in}{2.534443in}}%
\pgfpathlineto{\pgfqpoint{2.469864in}{2.325708in}}%
\pgfpathlineto{\pgfqpoint{3.004776in}{1.701177in}}%
\pgfpathlineto{\pgfqpoint{3.697918in}{0.901143in}}%
\pgfpathlineto{\pgfqpoint{3.971748in}{0.590702in}}%
\pgfpathlineto{\pgfqpoint{3.814025in}{0.686860in}}%
\pgfpathlineto{\pgfqpoint{3.034626in}{1.127408in}}%
\pgfpathlineto{\pgfqpoint{1.970482in}{1.482821in}}%
\pgfpathlineto{\pgfqpoint{1.358267in}{1.410623in}}%
\pgfpathlineto{\pgfqpoint{1.085561in}{1.171016in}}%
\pgfpathlineto{\pgfqpoint{0.891057in}{0.949505in}}%
\pgfpathlineto{\pgfqpoint{0.733113in}{0.779644in}}%
\pgfpathlineto{\pgfqpoint{0.626604in}{0.666462in}}%
\pgfpathlineto{\pgfqpoint{0.583560in}{0.622235in}}%
\pgfpathlineto{\pgfqpoint{0.656327in}{0.707359in}}%
\pgfpathlineto{\pgfqpoint{1.266388in}{1.414364in}}%
\pgfpathlineto{\pgfqpoint{2.120177in}{2.401646in}}%
\pgfpathlineto{\pgfqpoint{2.258348in}{2.600428in}}%
\pgfpathlineto{\pgfqpoint{2.264356in}{2.617289in}}%
\pgfpathlineto{\pgfqpoint{2.262421in}{2.610330in}}%
\pgfpathlineto{\pgfqpoint{2.265581in}{2.596369in}}%
\pgfpathlineto{\pgfqpoint{2.287355in}{2.555955in}}%
\pgfpathlineto{\pgfqpoint{2.394197in}{2.412143in}}%
\pgfpathlineto{\pgfqpoint{2.799198in}{1.931352in}}%
\pgfpathlineto{\pgfqpoint{3.521957in}{1.097166in}}%
\pgfpathlineto{\pgfqpoint{3.932109in}{0.630722in}}%
\pgfpathlineto{\pgfqpoint{3.920307in}{0.613778in}}%
\pgfpathlineto{\pgfqpoint{3.349163in}{0.958886in}}%
\pgfpathlineto{\pgfqpoint{2.257794in}{1.410779in}}%
\pgfpathlineto{\pgfqpoint{1.488760in}{1.448192in}}%
\pgfpathlineto{\pgfqpoint{1.145144in}{1.231919in}}%
\pgfpathlineto{\pgfqpoint{0.933027in}{1.001373in}}%
\pgfpathlineto{\pgfqpoint{0.770948in}{0.816914in}}%
\pgfpathlineto{\pgfqpoint{0.588603in}{0.624345in}}%
\pgfpathlineto{\pgfqpoint{0.612838in}{0.653320in}}%
\pgfpathlineto{\pgfqpoint{1.990086in}{2.249411in}}%
\pgfpathlineto{\pgfqpoint{2.248713in}{2.581218in}}%
\pgfpathlineto{\pgfqpoint{2.264599in}{2.617915in}}%
\pgfpathlineto{\pgfqpoint{2.264261in}{2.617525in}}%
\pgfpathlineto{\pgfqpoint{2.263735in}{2.612723in}}%
\pgfpathlineto{\pgfqpoint{2.267064in}{2.600677in}}%
\pgfpathlineto{\pgfqpoint{2.287363in}{2.564772in}}%
\pgfpathlineto{\pgfqpoint{2.385110in}{2.434823in}}%
\pgfpathlineto{\pgfqpoint{2.763682in}{1.985143in}}%
\pgfpathlineto{\pgfqpoint{3.920787in}{0.647946in}}%
\pgfpathlineto{\pgfqpoint{3.937957in}{0.608024in}}%
\pgfpathlineto{\pgfqpoint{3.422516in}{0.939493in}}%
\pgfpathlineto{\pgfqpoint{2.367434in}{1.425582in}}%
\pgfpathlineto{\pgfqpoint{1.575360in}{1.508993in}}%
\pgfpathlineto{\pgfqpoint{1.209198in}{1.302469in}}%
\pgfpathlineto{\pgfqpoint{0.987681in}{1.061489in}}%
\pgfpathlineto{\pgfqpoint{0.812775in}{0.862400in}}%
\pgfpathlineto{\pgfqpoint{0.600348in}{0.636259in}}%
\pgfpathlineto{\pgfqpoint{0.587171in}{0.626934in}}%
\pgfpathlineto{\pgfqpoint{0.803843in}{0.878349in}}%
\pgfpathlineto{\pgfqpoint{1.730653in}{1.950236in}}%
\pgfpathlineto{\pgfqpoint{2.219964in}{2.537140in}}%
\pgfpathlineto{\pgfqpoint{2.264191in}{2.614896in}}%
\pgfpathlineto{\pgfqpoint{2.264571in}{2.617670in}}%
\pgfpathlineto{\pgfqpoint{2.264037in}{2.613909in}}%
\pgfpathlineto{\pgfqpoint{2.266707in}{2.603295in}}%
\pgfpathlineto{\pgfqpoint{2.283559in}{2.572556in}}%
\pgfpathlineto{\pgfqpoint{2.364876in}{2.462841in}}%
\pgfpathlineto{\pgfqpoint{2.693516in}{2.069746in}}%
\pgfpathlineto{\pgfqpoint{3.896028in}{0.680164in}}%
\pgfpathlineto{\pgfqpoint{3.964084in}{0.596066in}}%
\pgfpathlineto{\pgfqpoint{3.537685in}{0.891354in}}%
\pgfpathlineto{\pgfqpoint{2.543812in}{1.415291in}}%
\pgfpathlineto{\pgfqpoint{1.706692in}{1.580581in}}%
\pgfpathlineto{\pgfqpoint{1.289978in}{1.403963in}}%
\pgfpathlineto{\pgfqpoint{0.875976in}{0.932207in}}%
\pgfpathlineto{\pgfqpoint{0.722369in}{0.767279in}}%
\pgfpathlineto{\pgfqpoint{0.621656in}{0.660384in}}%
\pgfpathlineto{\pgfqpoint{0.584333in}{0.622848in}}%
\pgfpathlineto{\pgfqpoint{0.682190in}{0.736945in}}%
\pgfpathlineto{\pgfqpoint{1.374939in}{1.539205in}}%
\pgfpathlineto{\pgfqpoint{2.153619in}{2.443509in}}%
\pgfpathlineto{\pgfqpoint{2.260137in}{2.605632in}}%
\pgfpathlineto{\pgfqpoint{2.264311in}{2.617721in}}%
\pgfpathlineto{\pgfqpoint{2.263418in}{2.615658in}}%
\pgfpathlineto{\pgfqpoint{2.264212in}{2.607745in}}%
\pgfpathlineto{\pgfqpoint{2.273405in}{2.586555in}}%
\pgfpathlineto{\pgfqpoint{2.321528in}{2.515462in}}%
\pgfpathlineto{\pgfqpoint{2.536953in}{2.249295in}}%
\pgfpathlineto{\pgfqpoint{3.151979in}{1.535480in}}%
\pgfpathlineto{\pgfqpoint{3.790754in}{0.798174in}}%
\pgfpathlineto{\pgfqpoint{3.983721in}{0.579208in}}%
\pgfpathlineto{\pgfqpoint{3.731171in}{0.746415in}}%
\pgfpathlineto{\pgfqpoint{2.849754in}{1.230512in}}%
\pgfpathlineto{\pgfqpoint{1.849606in}{1.518206in}}%
\pgfpathlineto{\pgfqpoint{1.313627in}{1.401165in}}%
\pgfpathlineto{\pgfqpoint{1.073051in}{1.152274in}}%
\pgfpathlineto{\pgfqpoint{0.873333in}{0.935509in}}%
\pgfpathlineto{\pgfqpoint{0.726331in}{0.768280in}}%
\pgfpathlineto{\pgfqpoint{0.622639in}{0.658724in}}%
\pgfpathlineto{\pgfqpoint{0.582117in}{0.617716in}}%
\pgfpathlineto{\pgfqpoint{0.663925in}{0.713358in}}%
\pgfpathlineto{\pgfqpoint{1.307518in}{1.459082in}}%
\pgfpathlineto{\pgfqpoint{2.136976in}{2.418414in}}%
\pgfpathlineto{\pgfqpoint{2.259443in}{2.602663in}}%
\pgfpathlineto{\pgfqpoint{2.264552in}{2.617933in}}%
\pgfpathlineto{\pgfqpoint{2.263503in}{2.616721in}}%
\pgfpathlineto{\pgfqpoint{2.263019in}{2.610988in}}%
\pgfpathlineto{\pgfqpoint{2.267087in}{2.596762in}}%
\pgfpathlineto{\pgfqpoint{2.292007in}{2.553607in}}%
\pgfpathlineto{\pgfqpoint{2.412149in}{2.395895in}}%
\pgfpathlineto{\pgfqpoint{2.852132in}{1.877080in}}%
\pgfpathlineto{\pgfqpoint{3.577582in}{1.039634in}}%
\pgfpathlineto{\pgfqpoint{3.945235in}{0.616981in}}%
\pgfpathlineto{\pgfqpoint{3.898061in}{0.627967in}}%
\pgfpathlineto{\pgfqpoint{3.272070in}{0.997656in}}%
\pgfpathlineto{\pgfqpoint{2.173271in}{1.427292in}}%
\pgfpathlineto{\pgfqpoint{1.444120in}{1.431606in}}%
\pgfpathlineto{\pgfqpoint{1.121693in}{1.207625in}}%
\pgfpathlineto{\pgfqpoint{0.921090in}{0.979435in}}%
\pgfpathlineto{\pgfqpoint{0.755494in}{0.801353in}}%
\pgfpathlineto{\pgfqpoint{0.583734in}{0.620106in}}%
\pgfpathlineto{\pgfqpoint{0.618234in}{0.661147in}}%
\pgfpathlineto{\pgfqpoint{2.035375in}{2.303858in}}%
\pgfpathlineto{\pgfqpoint{2.251658in}{2.587415in}}%
\pgfpathlineto{\pgfqpoint{2.264416in}{2.617842in}}%
\pgfpathlineto{\pgfqpoint{2.263901in}{2.617248in}}%
\pgfpathlineto{\pgfqpoint{2.263060in}{2.612327in}}%
\pgfpathlineto{\pgfqpoint{2.265725in}{2.600268in}}%
\pgfpathlineto{\pgfqpoint{2.284143in}{2.565584in}}%
\pgfpathlineto{\pgfqpoint{2.374965in}{2.442210in}}%
\pgfpathlineto{\pgfqpoint{2.734166in}{2.013105in}}%
\pgfpathlineto{\pgfqpoint{3.911805in}{0.653813in}}%
\pgfpathlineto{\pgfqpoint{3.946261in}{0.598730in}}%
\pgfpathlineto{\pgfqpoint{3.451677in}{0.910316in}}%
\pgfpathlineto{\pgfqpoint{2.389483in}{1.390961in}}%
\pgfpathlineto{\pgfqpoint{1.567401in}{1.482009in}}%
\pgfpathlineto{\pgfqpoint{1.185125in}{1.280606in}}%
\pgfpathlineto{\pgfqpoint{0.974268in}{1.040564in}}%
\pgfpathlineto{\pgfqpoint{0.795208in}{0.847115in}}%
\pgfpathlineto{\pgfqpoint{0.663619in}{0.706386in}}%
\pgfpathlineto{\pgfqpoint{0.663619in}{0.706386in}}%
\pgfusepath{stroke}%
\end{pgfscope}%
\begin{pgfscope}%
\pgfpathrectangle{\pgfqpoint{0.152333in}{0.150000in}}{\pgfqpoint{4.224218in}{2.565000in}}%
\pgfusepath{clip}%
\pgfsetbuttcap%
\pgfsetroundjoin%
\pgfsetlinewidth{1.003750pt}%
\definecolor{currentstroke}{rgb}{0.501961,0.501961,0.501961}%
\pgfsetstrokecolor{currentstroke}%
\pgfsetdash{{3.700000pt}{1.600000pt}}{0.000000pt}%
\pgfpathmoveto{\pgfqpoint{2.439334in}{1.402646in}}%
\pgfpathlineto{\pgfqpoint{2.362027in}{1.766946in}}%
\pgfpathlineto{\pgfqpoint{2.266099in}{1.906762in}}%
\pgfpathlineto{\pgfqpoint{2.174644in}{1.978660in}}%
\pgfpathlineto{\pgfqpoint{2.087875in}{1.991600in}}%
\pgfpathlineto{\pgfqpoint{2.002200in}{1.960776in}}%
\pgfpathlineto{\pgfqpoint{1.913048in}{1.903334in}}%
\pgfpathlineto{\pgfqpoint{1.820576in}{1.829061in}}%
\pgfpathlineto{\pgfqpoint{1.728203in}{1.745200in}}%
\pgfpathlineto{\pgfqpoint{1.642526in}{1.657937in}}%
\pgfpathlineto{\pgfqpoint{1.573608in}{1.572631in}}%
\pgfpathlineto{\pgfqpoint{1.539319in}{1.493118in}}%
\pgfpathlineto{\pgfqpoint{1.566315in}{1.422263in}}%
\pgfpathlineto{\pgfqpoint{1.661021in}{1.364183in}}%
\pgfpathlineto{\pgfqpoint{1.780020in}{1.329270in}}%
\pgfpathlineto{\pgfqpoint{1.870623in}{1.336342in}}%
\pgfpathlineto{\pgfqpoint{1.918305in}{1.384933in}}%
\pgfpathlineto{\pgfqpoint{1.943612in}{1.447323in}}%
\pgfpathlineto{\pgfqpoint{1.960238in}{1.507481in}}%
\pgfpathlineto{\pgfqpoint{1.972787in}{1.560460in}}%
\pgfpathlineto{\pgfqpoint{1.981772in}{1.605493in}}%
\pgfpathlineto{\pgfqpoint{1.986607in}{1.642445in}}%
\pgfpathlineto{\pgfqpoint{1.986445in}{1.670784in}}%
\pgfpathlineto{\pgfqpoint{1.980289in}{1.689722in}}%
\pgfpathlineto{\pgfqpoint{1.967230in}{1.698515in}}%
\pgfpathlineto{\pgfqpoint{1.946814in}{1.696893in}}%
\pgfpathlineto{\pgfqpoint{1.919363in}{1.685446in}}%
\pgfpathlineto{\pgfqpoint{1.886154in}{1.665664in}}%
\pgfpathlineto{\pgfqpoint{1.849428in}{1.639657in}}%
\pgfpathlineto{\pgfqpoint{1.812353in}{1.609728in}}%
\pgfpathlineto{\pgfqpoint{1.780245in}{1.577885in}}%
\pgfpathlineto{\pgfqpoint{1.763894in}{1.545691in}}%
\pgfpathlineto{\pgfqpoint{1.778458in}{1.514526in}}%
\pgfpathlineto{\pgfqpoint{1.822539in}{1.486566in}}%
\pgfpathlineto{\pgfqpoint{1.873955in}{1.467254in}}%
\pgfpathlineto{\pgfqpoint{1.911832in}{1.466574in}}%
\pgfpathlineto{\pgfqpoint{1.931493in}{1.484958in}}%
\pgfpathlineto{\pgfqpoint{1.942602in}{1.511031in}}%
\pgfpathlineto{\pgfqpoint{1.950418in}{1.537831in}}%
\pgfpathlineto{\pgfqpoint{1.956726in}{1.562747in}}%
\pgfpathlineto{\pgfqpoint{1.961529in}{1.585054in}}%
\pgfpathlineto{\pgfqpoint{1.964364in}{1.604278in}}%
\pgfpathlineto{\pgfqpoint{1.964623in}{1.619773in}}%
\pgfpathlineto{\pgfqpoint{1.961601in}{1.630773in}}%
\pgfpathlineto{\pgfqpoint{1.954626in}{1.636549in}}%
\pgfpathlineto{\pgfqpoint{1.943275in}{1.636652in}}%
\pgfpathlineto{\pgfqpoint{1.927568in}{1.631143in}}%
\pgfpathlineto{\pgfqpoint{1.908113in}{1.620638in}}%
\pgfpathlineto{\pgfqpoint{1.886143in}{1.606182in}}%
\pgfpathlineto{\pgfqpoint{1.863567in}{1.589002in}}%
\pgfpathlineto{\pgfqpoint{1.843837in}{1.570226in}}%
\pgfpathlineto{\pgfqpoint{1.834093in}{1.550782in}}%
\pgfpathlineto{\pgfqpoint{1.843777in}{1.531574in}}%
\pgfpathlineto{\pgfqpoint{1.871116in}{1.514153in}}%
\pgfpathlineto{\pgfqpoint{1.902255in}{1.502426in}}%
\pgfpathlineto{\pgfqpoint{1.924491in}{1.502923in}}%
\pgfpathlineto{\pgfqpoint{1.936087in}{1.515091in}}%
\pgfpathlineto{\pgfqpoint{1.942922in}{1.531767in}}%
\pgfpathlineto{\pgfqpoint{1.952022in}{1.565197in}}%
\pgfpathlineto{\pgfqpoint{1.955156in}{1.579896in}}%
\pgfpathlineto{\pgfqpoint{1.956957in}{1.592697in}}%
\pgfpathlineto{\pgfqpoint{1.956962in}{1.603080in}}%
\pgfpathlineto{\pgfqpoint{1.954637in}{1.610443in}}%
\pgfpathlineto{\pgfqpoint{1.949482in}{1.614222in}}%
\pgfpathlineto{\pgfqpoint{1.941179in}{1.614063in}}%
\pgfpathlineto{\pgfqpoint{1.929738in}{1.609989in}}%
\pgfpathlineto{\pgfqpoint{1.915604in}{1.602413in}}%
\pgfpathlineto{\pgfqpoint{1.899660in}{1.592055in}}%
\pgfpathlineto{\pgfqpoint{1.883317in}{1.579748in}}%
\pgfpathlineto{\pgfqpoint{1.869257in}{1.566254in}}%
\pgfpathlineto{\pgfqpoint{1.863070in}{1.552208in}}%
\pgfpathlineto{\pgfqpoint{1.871461in}{1.538272in}}%
\pgfpathlineto{\pgfqpoint{1.891974in}{1.525671in}}%
\pgfpathlineto{\pgfqpoint{1.914329in}{1.517539in}}%
\pgfpathlineto{\pgfqpoint{1.929639in}{1.518631in}}%
\pgfpathlineto{\pgfqpoint{1.937675in}{1.527935in}}%
\pgfpathlineto{\pgfqpoint{1.942561in}{1.540257in}}%
\pgfpathlineto{\pgfqpoint{1.949252in}{1.564975in}}%
\pgfpathlineto{\pgfqpoint{1.952858in}{1.585508in}}%
\pgfpathlineto{\pgfqpoint{1.952764in}{1.593287in}}%
\pgfpathlineto{\pgfqpoint{1.950856in}{1.598782in}}%
\pgfpathlineto{\pgfqpoint{1.946739in}{1.601540in}}%
\pgfpathlineto{\pgfqpoint{1.940165in}{1.601281in}}%
\pgfpathlineto{\pgfqpoint{1.931143in}{1.598020in}}%
\pgfpathlineto{\pgfqpoint{1.920027in}{1.592071in}}%
\pgfpathlineto{\pgfqpoint{1.907510in}{1.583984in}}%
\pgfpathlineto{\pgfqpoint{1.894720in}{1.574387in}}%
\pgfpathlineto{\pgfqpoint{1.883869in}{1.563851in}}%
\pgfpathlineto{\pgfqpoint{1.879560in}{1.552860in}}%
\pgfpathlineto{\pgfqpoint{1.886893in}{1.541936in}}%
\pgfpathlineto{\pgfqpoint{1.903312in}{1.532102in}}%
\pgfpathlineto{\pgfqpoint{1.920678in}{1.525985in}}%
\pgfpathlineto{\pgfqpoint{1.932251in}{1.527287in}}%
\pgfpathlineto{\pgfqpoint{1.938351in}{1.534859in}}%
\pgfpathlineto{\pgfqpoint{1.942142in}{1.544656in}}%
\pgfpathlineto{\pgfqpoint{1.947447in}{1.564291in}}%
\pgfpathlineto{\pgfqpoint{1.950322in}{1.580712in}}%
\pgfpathlineto{\pgfqpoint{1.950227in}{1.586963in}}%
\pgfpathlineto{\pgfqpoint{1.948658in}{1.591388in}}%
\pgfpathlineto{\pgfqpoint{1.945286in}{1.593609in}}%
\pgfpathlineto{\pgfqpoint{1.939905in}{1.593391in}}%
\pgfpathlineto{\pgfqpoint{1.932517in}{1.590739in}}%
\pgfpathlineto{\pgfqpoint{1.923408in}{1.585899in}}%
\pgfpathlineto{\pgfqpoint{1.913141in}{1.579311in}}%
\pgfpathlineto{\pgfqpoint{1.902649in}{1.571478in}}%
\pgfpathlineto{\pgfqpoint{1.893785in}{1.562858in}}%
\pgfpathlineto{\pgfqpoint{1.890406in}{1.553841in}}%
\pgfpathlineto{\pgfqpoint{1.896645in}{1.544862in}}%
\pgfpathlineto{\pgfqpoint{1.910213in}{1.536786in}}%
\pgfpathlineto{\pgfqpoint{1.924411in}{1.531841in}}%
\pgfpathlineto{\pgfqpoint{1.933789in}{1.533068in}}%
\pgfpathlineto{\pgfqpoint{1.938742in}{1.539403in}}%
\pgfpathlineto{\pgfqpoint{1.941861in}{1.547522in}}%
\pgfpathlineto{\pgfqpoint{1.947858in}{1.571124in}}%
\pgfpathlineto{\pgfqpoint{1.948757in}{1.577564in}}%
\pgfpathlineto{\pgfqpoint{1.948730in}{1.582847in}}%
\pgfpathlineto{\pgfqpoint{1.947481in}{1.586633in}}%
\pgfpathlineto{\pgfqpoint{1.944730in}{1.588598in}}%
\pgfpathlineto{\pgfqpoint{1.940291in}{1.588530in}}%
\pgfpathlineto{\pgfqpoint{1.934153in}{1.586417in}}%
\pgfpathlineto{\pgfqpoint{1.926539in}{1.582448in}}%
\pgfpathlineto{\pgfqpoint{1.917916in}{1.576978in}}%
\pgfpathlineto{\pgfqpoint{1.909065in}{1.570425in}}%
\pgfpathlineto{\pgfqpoint{1.901548in}{1.563176in}}%
\pgfpathlineto{\pgfqpoint{1.898611in}{1.555562in}}%
\pgfpathlineto{\pgfqpoint{1.903770in}{1.547957in}}%
\pgfpathlineto{\pgfqpoint{1.915169in}{1.541105in}}%
\pgfpathlineto{\pgfqpoint{1.927143in}{1.536920in}}%
\pgfpathlineto{\pgfqpoint{1.935048in}{1.537985in}}%
\pgfpathlineto{\pgfqpoint{1.939241in}{1.543381in}}%
\pgfpathlineto{\pgfqpoint{1.943936in}{1.557435in}}%
\pgfpathlineto{\pgfqpoint{1.947026in}{1.570474in}}%
\pgfpathlineto{\pgfqpoint{1.947779in}{1.580553in}}%
\pgfpathlineto{\pgfqpoint{1.946705in}{1.583822in}}%
\pgfpathlineto{\pgfqpoint{1.944332in}{1.585527in}}%
\pgfpathlineto{\pgfqpoint{1.940497in}{1.585481in}}%
\pgfpathlineto{\pgfqpoint{1.935186in}{1.583668in}}%
\pgfpathlineto{\pgfqpoint{1.928589in}{1.580247in}}%
\pgfpathlineto{\pgfqpoint{1.913405in}{1.569851in}}%
\pgfpathlineto{\pgfqpoint{1.906839in}{1.563565in}}%
\pgfpathlineto{\pgfqpoint{1.904224in}{1.556951in}}%
\pgfpathlineto{\pgfqpoint{1.908642in}{1.550329in}}%
\pgfpathlineto{\pgfqpoint{1.918520in}{1.544342in}}%
\pgfpathlineto{\pgfqpoint{1.928946in}{1.540642in}}%
\pgfpathlineto{\pgfqpoint{1.935860in}{1.541500in}}%
\pgfpathlineto{\pgfqpoint{1.939532in}{1.546159in}}%
\pgfpathlineto{\pgfqpoint{1.943643in}{1.558395in}}%
\pgfpathlineto{\pgfqpoint{1.946369in}{1.569806in}}%
\pgfpathlineto{\pgfqpoint{1.947073in}{1.578685in}}%
\pgfpathlineto{\pgfqpoint{1.946163in}{1.581593in}}%
\pgfpathlineto{\pgfqpoint{1.944117in}{1.583141in}}%
\pgfpathlineto{\pgfqpoint{1.940787in}{1.583156in}}%
\pgfpathlineto{\pgfqpoint{1.936152in}{1.581617in}}%
\pgfpathlineto{\pgfqpoint{1.923793in}{1.574533in}}%
\pgfpathlineto{\pgfqpoint{1.916998in}{1.569561in}}%
\pgfpathlineto{\pgfqpoint{1.911162in}{1.564033in}}%
\pgfpathlineto{\pgfqpoint{1.908746in}{1.558200in}}%
\pgfpathlineto{\pgfqpoint{1.912507in}{1.552345in}}%
\pgfpathlineto{\pgfqpoint{1.921159in}{1.547030in}}%
\pgfpathlineto{\pgfqpoint{1.930378in}{1.543705in}}%
\pgfpathlineto{\pgfqpoint{1.936531in}{1.544390in}}%
\pgfpathlineto{\pgfqpoint{1.939801in}{1.548467in}}%
\pgfpathlineto{\pgfqpoint{1.943445in}{1.559286in}}%
\pgfpathlineto{\pgfqpoint{1.946462in}{1.573703in}}%
\pgfpathlineto{\pgfqpoint{1.946452in}{1.577258in}}%
\pgfpathlineto{\pgfqpoint{1.945619in}{1.579824in}}%
\pgfpathlineto{\pgfqpoint{1.943769in}{1.581176in}}%
\pgfpathlineto{\pgfqpoint{1.940769in}{1.581161in}}%
\pgfpathlineto{\pgfqpoint{1.936603in}{1.579762in}}%
\pgfpathlineto{\pgfqpoint{1.925521in}{1.573395in}}%
\pgfpathlineto{\pgfqpoint{1.919449in}{1.568939in}}%
\pgfpathlineto{\pgfqpoint{1.914280in}{1.563986in}}%
\pgfpathlineto{\pgfqpoint{1.912262in}{1.558761in}}%
\pgfpathlineto{\pgfqpoint{1.915809in}{1.553521in}}%
\pgfpathlineto{\pgfqpoint{1.923626in}{1.548788in}}%
\pgfpathlineto{\pgfqpoint{1.931824in}{1.545906in}}%
\pgfpathlineto{\pgfqpoint{1.937221in}{1.546661in}}%
\pgfpathlineto{\pgfqpoint{1.940092in}{1.550395in}}%
\pgfpathlineto{\pgfqpoint{1.943341in}{1.560132in}}%
\pgfpathlineto{\pgfqpoint{1.946041in}{1.573110in}}%
\pgfpathlineto{\pgfqpoint{1.946023in}{1.576312in}}%
\pgfpathlineto{\pgfqpoint{1.945258in}{1.578620in}}%
\pgfpathlineto{\pgfqpoint{1.943568in}{1.579829in}}%
\pgfpathlineto{\pgfqpoint{1.940835in}{1.579800in}}%
\pgfpathlineto{\pgfqpoint{1.937044in}{1.578517in}}%
\pgfpathlineto{\pgfqpoint{1.926961in}{1.572726in}}%
\pgfpathlineto{\pgfqpoint{1.921427in}{1.568681in}}%
\pgfpathlineto{\pgfqpoint{1.916686in}{1.564185in}}%
\pgfpathlineto{\pgfqpoint{1.914768in}{1.559440in}}%
\pgfpathlineto{\pgfqpoint{1.917905in}{1.554672in}}%
\pgfpathlineto{\pgfqpoint{1.924979in}{1.550341in}}%
\pgfpathlineto{\pgfqpoint{1.932473in}{1.547634in}}%
\pgfpathlineto{\pgfqpoint{1.937461in}{1.548201in}}%
\pgfpathlineto{\pgfqpoint{1.940110in}{1.551523in}}%
\pgfpathlineto{\pgfqpoint{1.943079in}{1.560338in}}%
\pgfpathlineto{\pgfqpoint{1.945563in}{1.572147in}}%
\pgfpathlineto{\pgfqpoint{1.944890in}{1.577201in}}%
\pgfpathlineto{\pgfqpoint{1.943380in}{1.578333in}}%
\pgfpathlineto{\pgfqpoint{1.940922in}{1.578344in}}%
\pgfpathlineto{\pgfqpoint{1.933227in}{1.575035in}}%
\pgfpathlineto{\pgfqpoint{1.923334in}{1.568329in}}%
\pgfpathlineto{\pgfqpoint{1.919038in}{1.564240in}}%
\pgfpathlineto{\pgfqpoint{1.917330in}{1.559915in}}%
\pgfpathlineto{\pgfqpoint{1.920226in}{1.555563in}}%
\pgfpathlineto{\pgfqpoint{1.926687in}{1.551606in}}%
\pgfpathlineto{\pgfqpoint{1.933518in}{1.549136in}}%
\pgfpathlineto{\pgfqpoint{1.938057in}{1.549662in}}%
\pgfpathlineto{\pgfqpoint{1.940474in}{1.552708in}}%
\pgfpathlineto{\pgfqpoint{1.943191in}{1.560771in}}%
\pgfpathlineto{\pgfqpoint{1.945452in}{1.571577in}}%
\pgfpathlineto{\pgfqpoint{1.944808in}{1.576188in}}%
\pgfpathlineto{\pgfqpoint{1.943401in}{1.577206in}}%
\pgfpathlineto{\pgfqpoint{1.941118in}{1.577191in}}%
\pgfpathlineto{\pgfqpoint{1.933992in}{1.574101in}}%
\pgfpathlineto{\pgfqpoint{1.924838in}{1.567900in}}%
\pgfpathlineto{\pgfqpoint{1.920839in}{1.564129in}}%
\pgfpathlineto{\pgfqpoint{1.919189in}{1.560144in}}%
\pgfpathlineto{\pgfqpoint{1.921769in}{1.556139in}}%
\pgfpathlineto{\pgfqpoint{1.927673in}{1.552509in}}%
\pgfpathlineto{\pgfqpoint{1.933930in}{1.550271in}}%
\pgfpathlineto{\pgfqpoint{1.938077in}{1.550802in}}%
\pgfpathlineto{\pgfqpoint{1.940285in}{1.553626in}}%
\pgfpathlineto{\pgfqpoint{1.942790in}{1.561071in}}%
\pgfpathlineto{\pgfqpoint{1.944913in}{1.571089in}}%
\pgfpathlineto{\pgfqpoint{1.944377in}{1.575421in}}%
\pgfpathlineto{\pgfqpoint{1.943118in}{1.576415in}}%
\pgfpathlineto{\pgfqpoint{1.941055in}{1.576460in}}%
\pgfpathlineto{\pgfqpoint{1.934555in}{1.573713in}}%
\pgfpathlineto{\pgfqpoint{1.926172in}{1.568043in}}%
\pgfpathlineto{\pgfqpoint{1.922563in}{1.564563in}}%
\pgfpathlineto{\pgfqpoint{1.921224in}{1.560866in}}%
\pgfpathlineto{\pgfqpoint{1.923831in}{1.557123in}}%
\pgfpathlineto{\pgfqpoint{1.929431in}{1.553673in}}%
\pgfpathlineto{\pgfqpoint{1.935373in}{1.551403in}}%
\pgfpathlineto{\pgfqpoint{1.939403in}{1.551655in}}%
\pgfpathlineto{\pgfqpoint{1.941543in}{1.554148in}}%
\pgfpathlineto{\pgfqpoint{1.943850in}{1.560966in}}%
\pgfpathlineto{\pgfqpoint{1.945617in}{1.570014in}}%
\pgfpathlineto{\pgfqpoint{1.944798in}{1.573662in}}%
\pgfpathlineto{\pgfqpoint{1.943407in}{1.574327in}}%
\pgfpathlineto{\pgfqpoint{1.941247in}{1.574088in}}%
\pgfpathlineto{\pgfqpoint{1.934745in}{1.571026in}}%
\pgfpathlineto{\pgfqpoint{1.926499in}{1.565465in}}%
\pgfpathlineto{\pgfqpoint{1.922691in}{1.562191in}}%
\pgfpathlineto{\pgfqpoint{1.920523in}{1.558820in}}%
\pgfpathlineto{\pgfqpoint{1.921810in}{1.555606in}}%
\pgfpathlineto{\pgfqpoint{1.926147in}{1.553142in}}%
\pgfpathlineto{\pgfqpoint{1.930489in}{1.552567in}}%
\pgfpathlineto{\pgfqpoint{1.933058in}{1.554253in}}%
\pgfpathlineto{\pgfqpoint{1.935817in}{1.560519in}}%
\pgfpathlineto{\pgfqpoint{1.940178in}{1.576496in}}%
\pgfpathlineto{\pgfqpoint{1.940581in}{1.581141in}}%
\pgfpathlineto{\pgfqpoint{1.939874in}{1.583091in}}%
\pgfpathlineto{\pgfqpoint{1.940346in}{1.582219in}}%
\pgfpathlineto{\pgfqpoint{1.948245in}{1.576509in}}%
\pgfpathlineto{\pgfqpoint{1.953238in}{1.573083in}}%
\pgfpathlineto{\pgfqpoint{1.954477in}{1.569959in}}%
\pgfpathlineto{\pgfqpoint{1.951443in}{1.566924in}}%
\pgfpathlineto{\pgfqpoint{1.933692in}{1.554510in}}%
\pgfpathlineto{\pgfqpoint{1.932781in}{1.555469in}}%
\pgfpathlineto{\pgfqpoint{1.939845in}{1.577994in}}%
\pgfpathlineto{\pgfqpoint{1.941095in}{1.580804in}}%
\pgfpathlineto{\pgfqpoint{1.943196in}{1.582560in}}%
\pgfpathlineto{\pgfqpoint{1.947369in}{1.581894in}}%
\pgfpathlineto{\pgfqpoint{1.966657in}{1.573240in}}%
\pgfpathlineto{\pgfqpoint{1.966091in}{1.573231in}}%
\pgfpathlineto{\pgfqpoint{1.961468in}{1.571560in}}%
\pgfpathlineto{\pgfqpoint{1.950664in}{1.564649in}}%
\pgfpathlineto{\pgfqpoint{1.942801in}{1.559324in}}%
\pgfpathlineto{\pgfqpoint{1.940796in}{1.558878in}}%
\pgfpathlineto{\pgfqpoint{1.943185in}{1.567451in}}%
\pgfpathlineto{\pgfqpoint{1.947928in}{1.582192in}}%
\pgfpathlineto{\pgfqpoint{1.949423in}{1.584322in}}%
\pgfpathlineto{\pgfqpoint{1.952438in}{1.584656in}}%
\pgfpathlineto{\pgfqpoint{1.963134in}{1.579890in}}%
\pgfpathlineto{\pgfqpoint{1.971387in}{1.575827in}}%
\pgfpathlineto{\pgfqpoint{1.967188in}{1.574230in}}%
\pgfpathlineto{\pgfqpoint{1.957047in}{1.567766in}}%
\pgfpathlineto{\pgfqpoint{1.949660in}{1.562777in}}%
\pgfpathlineto{\pgfqpoint{1.947777in}{1.562368in}}%
\pgfpathlineto{\pgfqpoint{1.949962in}{1.570380in}}%
\pgfpathlineto{\pgfqpoint{1.955439in}{1.586471in}}%
\pgfpathlineto{\pgfqpoint{1.957701in}{1.587396in}}%
\pgfpathlineto{\pgfqpoint{1.961883in}{1.586085in}}%
\pgfpathlineto{\pgfqpoint{1.976730in}{1.578862in}}%
\pgfpathlineto{\pgfqpoint{1.973257in}{1.577605in}}%
\pgfpathlineto{\pgfqpoint{1.963944in}{1.571791in}}%
\pgfpathlineto{\pgfqpoint{1.954411in}{1.565771in}}%
\pgfpathlineto{\pgfqpoint{1.955261in}{1.570150in}}%
\pgfpathlineto{\pgfqpoint{1.961109in}{1.588063in}}%
\pgfpathlineto{\pgfqpoint{1.963136in}{1.589065in}}%
\pgfpathlineto{\pgfqpoint{1.966974in}{1.587959in}}%
\pgfpathlineto{\pgfqpoint{1.981390in}{1.581149in}}%
\pgfpathlineto{\pgfqpoint{1.980380in}{1.580970in}}%
\pgfpathlineto{\pgfqpoint{1.976014in}{1.578994in}}%
\pgfpathlineto{\pgfqpoint{1.960369in}{1.568790in}}%
\pgfpathlineto{\pgfqpoint{1.959809in}{1.569721in}}%
\pgfpathlineto{\pgfqpoint{1.964866in}{1.586961in}}%
\pgfpathlineto{\pgfqpoint{1.965823in}{1.589056in}}%
\pgfpathlineto{\pgfqpoint{1.967560in}{1.590203in}}%
\pgfpathlineto{\pgfqpoint{1.970972in}{1.589399in}}%
\pgfpathlineto{\pgfqpoint{1.985259in}{1.582855in}}%
\pgfpathlineto{\pgfqpoint{1.984604in}{1.582850in}}%
\pgfpathlineto{\pgfqpoint{1.980827in}{1.581416in}}%
\pgfpathlineto{\pgfqpoint{1.968951in}{1.573660in}}%
\pgfpathlineto{\pgfqpoint{1.964607in}{1.571648in}}%
\pgfpathlineto{\pgfqpoint{1.967764in}{1.583159in}}%
\pgfpathlineto{\pgfqpoint{1.969931in}{1.589872in}}%
\pgfpathlineto{\pgfqpoint{1.971214in}{1.591364in}}%
\pgfpathlineto{\pgfqpoint{1.973880in}{1.591238in}}%
\pgfpathlineto{\pgfqpoint{1.988331in}{1.584503in}}%
\pgfpathlineto{\pgfqpoint{1.988022in}{1.584355in}}%
\pgfpathlineto{\pgfqpoint{1.984490in}{1.582987in}}%
\pgfpathlineto{\pgfqpoint{1.973182in}{1.575605in}}%
\pgfpathlineto{\pgfqpoint{1.968851in}{1.573458in}}%
\pgfpathlineto{\pgfqpoint{1.971134in}{1.582074in}}%
\pgfpathlineto{\pgfqpoint{1.974666in}{1.592533in}}%
\pgfpathlineto{\pgfqpoint{1.976549in}{1.593144in}}%
\pgfpathlineto{\pgfqpoint{1.984176in}{1.589875in}}%
\pgfpathlineto{\pgfqpoint{1.991375in}{1.586230in}}%
\pgfpathlineto{\pgfqpoint{1.988380in}{1.585103in}}%
\pgfpathlineto{\pgfqpoint{1.977782in}{1.578263in}}%
\pgfpathlineto{\pgfqpoint{1.973048in}{1.575526in}}%
\pgfpathlineto{\pgfqpoint{1.973757in}{1.579144in}}%
\pgfpathlineto{\pgfqpoint{1.978192in}{1.593528in}}%
\pgfpathlineto{\pgfqpoint{1.979738in}{1.594386in}}%
\pgfpathlineto{\pgfqpoint{1.982742in}{1.593549in}}%
\pgfpathlineto{\pgfqpoint{1.994418in}{1.587840in}}%
\pgfpathlineto{\pgfqpoint{1.993556in}{1.587626in}}%
\pgfpathlineto{\pgfqpoint{1.989878in}{1.585908in}}%
\pgfpathlineto{\pgfqpoint{1.984826in}{1.582664in}}%
\pgfpathlineto{\pgfqpoint{1.984826in}{1.582664in}}%
\pgfusepath{stroke}%
\end{pgfscope}%
\begin{pgfscope}%
\pgfpathrectangle{\pgfqpoint{0.152333in}{0.150000in}}{\pgfqpoint{4.224218in}{2.565000in}}%
\pgfusepath{clip}%
\pgfsetbuttcap%
\pgfsetroundjoin%
\definecolor{currentfill}{rgb}{0.501961,0.000000,0.501961}%
\pgfsetfillcolor{currentfill}%
\pgfsetlinewidth{1.505625pt}%
\definecolor{currentstroke}{rgb}{0.501961,0.000000,0.501961}%
\pgfsetstrokecolor{currentstroke}%
\pgfsetdash{}{0pt}%
\pgfsys@defobject{currentmarker}{\pgfqpoint{-0.017010in}{-0.017010in}}{\pgfqpoint{0.017010in}{0.017010in}}{%
\pgfpathmoveto{\pgfqpoint{-0.017010in}{0.000000in}}%
\pgfpathlineto{\pgfqpoint{0.017010in}{0.000000in}}%
\pgfpathmoveto{\pgfqpoint{0.000000in}{-0.017010in}}%
\pgfpathlineto{\pgfqpoint{0.000000in}{0.017010in}}%
\pgfusepath{stroke,fill}%
}%
\begin{pgfscope}%
\pgfsys@transformshift{2.439334in}{1.402646in}%
\pgfsys@useobject{currentmarker}{}%
\end{pgfscope}%
\end{pgfscope}%
\begin{pgfscope}%
\pgfpathrectangle{\pgfqpoint{0.152333in}{0.150000in}}{\pgfqpoint{4.224218in}{2.565000in}}%
\pgfusepath{clip}%
\pgfsetbuttcap%
\pgfsetroundjoin%
\definecolor{currentfill}{rgb}{0.501961,0.000000,0.501961}%
\pgfsetfillcolor{currentfill}%
\pgfsetlinewidth{1.003750pt}%
\definecolor{currentstroke}{rgb}{0.501961,0.000000,0.501961}%
\pgfsetstrokecolor{currentstroke}%
\pgfsetdash{}{0pt}%
\pgfsys@defobject{currentmarker}{\pgfqpoint{-0.016178in}{-0.013762in}}{\pgfqpoint{0.016178in}{0.017010in}}{%
\pgfpathmoveto{\pgfqpoint{0.000000in}{0.017010in}}%
\pgfpathlineto{\pgfqpoint{-0.003819in}{0.005256in}}%
\pgfpathlineto{\pgfqpoint{-0.016178in}{0.005256in}}%
\pgfpathlineto{\pgfqpoint{-0.006179in}{-0.002008in}}%
\pgfpathlineto{\pgfqpoint{-0.009998in}{-0.013762in}}%
\pgfpathlineto{\pgfqpoint{-0.000000in}{-0.006497in}}%
\pgfpathlineto{\pgfqpoint{0.009998in}{-0.013762in}}%
\pgfpathlineto{\pgfqpoint{0.006179in}{-0.002008in}}%
\pgfpathlineto{\pgfqpoint{0.016178in}{0.005256in}}%
\pgfpathlineto{\pgfqpoint{0.003819in}{0.005256in}}%
\pgfpathlineto{\pgfqpoint{0.000000in}{0.017010in}}%
\pgfpathclose%
\pgfusepath{stroke,fill}%
}%
\begin{pgfscope}%
\pgfsys@transformshift{0.663619in}{0.706386in}%
\pgfsys@useobject{currentmarker}{}%
\end{pgfscope}%
\end{pgfscope}%
\begin{pgfscope}%
\pgfpathrectangle{\pgfqpoint{0.152333in}{0.150000in}}{\pgfqpoint{4.224218in}{2.565000in}}%
\pgfusepath{clip}%
\pgfsetbuttcap%
\pgfsetroundjoin%
\definecolor{currentfill}{rgb}{0.000000,0.000000,0.000000}%
\pgfsetfillcolor{currentfill}%
\pgfsetlinewidth{1.505625pt}%
\definecolor{currentstroke}{rgb}{0.000000,0.000000,0.000000}%
\pgfsetstrokecolor{currentstroke}%
\pgfsetdash{}{0pt}%
\pgfsys@defobject{currentmarker}{\pgfqpoint{-0.017010in}{-0.017010in}}{\pgfqpoint{0.017010in}{0.017010in}}{%
\pgfpathmoveto{\pgfqpoint{-0.017010in}{0.000000in}}%
\pgfpathlineto{\pgfqpoint{0.017010in}{0.000000in}}%
\pgfpathmoveto{\pgfqpoint{0.000000in}{-0.017010in}}%
\pgfpathlineto{\pgfqpoint{0.000000in}{0.017010in}}%
\pgfusepath{stroke,fill}%
}%
\begin{pgfscope}%
\pgfsys@transformshift{2.439334in}{1.402646in}%
\pgfsys@useobject{currentmarker}{}%
\end{pgfscope}%
\end{pgfscope}%
\begin{pgfscope}%
\pgfpathrectangle{\pgfqpoint{0.152333in}{0.150000in}}{\pgfqpoint{4.224218in}{2.565000in}}%
\pgfusepath{clip}%
\pgfsetbuttcap%
\pgfsetroundjoin%
\definecolor{currentfill}{rgb}{0.000000,0.000000,0.000000}%
\pgfsetfillcolor{currentfill}%
\pgfsetlinewidth{1.003750pt}%
\definecolor{currentstroke}{rgb}{0.000000,0.000000,0.000000}%
\pgfsetstrokecolor{currentstroke}%
\pgfsetdash{}{0pt}%
\pgfsys@defobject{currentmarker}{\pgfqpoint{-0.016178in}{-0.013762in}}{\pgfqpoint{0.016178in}{0.017010in}}{%
\pgfpathmoveto{\pgfqpoint{0.000000in}{0.017010in}}%
\pgfpathlineto{\pgfqpoint{-0.003819in}{0.005256in}}%
\pgfpathlineto{\pgfqpoint{-0.016178in}{0.005256in}}%
\pgfpathlineto{\pgfqpoint{-0.006179in}{-0.002008in}}%
\pgfpathlineto{\pgfqpoint{-0.009998in}{-0.013762in}}%
\pgfpathlineto{\pgfqpoint{-0.000000in}{-0.006497in}}%
\pgfpathlineto{\pgfqpoint{0.009998in}{-0.013762in}}%
\pgfpathlineto{\pgfqpoint{0.006179in}{-0.002008in}}%
\pgfpathlineto{\pgfqpoint{0.016178in}{0.005256in}}%
\pgfpathlineto{\pgfqpoint{0.003819in}{0.005256in}}%
\pgfpathlineto{\pgfqpoint{0.000000in}{0.017010in}}%
\pgfpathclose%
\pgfusepath{stroke,fill}%
}%
\begin{pgfscope}%
\pgfsys@transformshift{1.984826in}{1.582664in}%
\pgfsys@useobject{currentmarker}{}%
\end{pgfscope}%
\end{pgfscope}%
\begin{pgfscope}%
\pgfsetrectcap%
\pgfsetmiterjoin%
\pgfsetlinewidth{0.803000pt}%
\definecolor{currentstroke}{rgb}{0.000000,0.000000,0.000000}%
\pgfsetstrokecolor{currentstroke}%
\pgfsetdash{}{0pt}%
\pgfpathmoveto{\pgfqpoint{0.152333in}{0.150000in}}%
\pgfpathlineto{\pgfqpoint{0.152333in}{2.715000in}}%
\pgfusepath{stroke}%
\end{pgfscope}%
\begin{pgfscope}%
\pgfsetrectcap%
\pgfsetmiterjoin%
\pgfsetlinewidth{0.803000pt}%
\definecolor{currentstroke}{rgb}{0.000000,0.000000,0.000000}%
\pgfsetstrokecolor{currentstroke}%
\pgfsetdash{}{0pt}%
\pgfpathmoveto{\pgfqpoint{4.376551in}{0.150000in}}%
\pgfpathlineto{\pgfqpoint{4.376551in}{2.715000in}}%
\pgfusepath{stroke}%
\end{pgfscope}%
\begin{pgfscope}%
\pgfsetrectcap%
\pgfsetmiterjoin%
\pgfsetlinewidth{0.803000pt}%
\definecolor{currentstroke}{rgb}{0.000000,0.000000,0.000000}%
\pgfsetstrokecolor{currentstroke}%
\pgfsetdash{}{0pt}%
\pgfpathmoveto{\pgfqpoint{0.152333in}{0.150000in}}%
\pgfpathlineto{\pgfqpoint{4.376551in}{0.150000in}}%
\pgfusepath{stroke}%
\end{pgfscope}%
\begin{pgfscope}%
\pgfsetrectcap%
\pgfsetmiterjoin%
\pgfsetlinewidth{0.803000pt}%
\definecolor{currentstroke}{rgb}{0.000000,0.000000,0.000000}%
\pgfsetstrokecolor{currentstroke}%
\pgfsetdash{}{0pt}%
\pgfpathmoveto{\pgfqpoint{0.152333in}{2.715000in}}%
\pgfpathlineto{\pgfqpoint{4.376551in}{2.715000in}}%
\pgfusepath{stroke}%
\end{pgfscope}%
\begin{pgfscope}%
\definecolor{textcolor}{rgb}{0.000000,0.000000,0.000000}%
\pgfsetstrokecolor{textcolor}%
\pgfsetfillcolor{textcolor}%
\pgftext[x=4.293987in,y=0.395393in,,base]{\color{textcolor}\sffamily\fontsize{10.000000}{12.000000}\selectfont 0.0}%
\end{pgfscope}%
\begin{pgfscope}%
\definecolor{textcolor}{rgb}{0.000000,0.000000,0.000000}%
\pgfsetstrokecolor{textcolor}%
\pgfsetfillcolor{textcolor}%
\pgftext[x=4.101977in,y=0.617529in,,base]{\color{textcolor}\sffamily\fontsize{10.000000}{12.000000}\selectfont 0.1}%
\end{pgfscope}%
\begin{pgfscope}%
\definecolor{textcolor}{rgb}{0.000000,0.000000,0.000000}%
\pgfsetstrokecolor{textcolor}%
\pgfsetfillcolor{textcolor}%
\pgftext[x=3.909967in,y=0.839664in,,base]{\color{textcolor}\sffamily\fontsize{10.000000}{12.000000}\selectfont 0.2}%
\end{pgfscope}%
\begin{pgfscope}%
\definecolor{textcolor}{rgb}{0.000000,0.000000,0.000000}%
\pgfsetstrokecolor{textcolor}%
\pgfsetfillcolor{textcolor}%
\pgftext[x=3.717957in,y=1.061800in,,base]{\color{textcolor}\sffamily\fontsize{10.000000}{12.000000}\selectfont 0.3}%
\end{pgfscope}%
\begin{pgfscope}%
\definecolor{textcolor}{rgb}{0.000000,0.000000,0.000000}%
\pgfsetstrokecolor{textcolor}%
\pgfsetfillcolor{textcolor}%
\pgftext[x=3.525947in,y=1.283935in,,base]{\color{textcolor}\sffamily\fontsize{10.000000}{12.000000}\selectfont 0.4}%
\end{pgfscope}%
\begin{pgfscope}%
\definecolor{textcolor}{rgb}{0.000000,0.000000,0.000000}%
\pgfsetstrokecolor{textcolor}%
\pgfsetfillcolor{textcolor}%
\pgftext[x=3.333937in,y=1.506071in,,base]{\color{textcolor}\sffamily\fontsize{10.000000}{12.000000}\selectfont 0.5}%
\end{pgfscope}%
\begin{pgfscope}%
\definecolor{textcolor}{rgb}{0.000000,0.000000,0.000000}%
\pgfsetstrokecolor{textcolor}%
\pgfsetfillcolor{textcolor}%
\pgftext[x=3.141927in,y=1.728206in,,base]{\color{textcolor}\sffamily\fontsize{10.000000}{12.000000}\selectfont 0.6}%
\end{pgfscope}%
\begin{pgfscope}%
\definecolor{textcolor}{rgb}{0.000000,0.000000,0.000000}%
\pgfsetstrokecolor{textcolor}%
\pgfsetfillcolor{textcolor}%
\pgftext[x=2.949918in,y=1.950342in,,base]{\color{textcolor}\sffamily\fontsize{10.000000}{12.000000}\selectfont 0.7}%
\end{pgfscope}%
\begin{pgfscope}%
\definecolor{textcolor}{rgb}{0.000000,0.000000,0.000000}%
\pgfsetstrokecolor{textcolor}%
\pgfsetfillcolor{textcolor}%
\pgftext[x=2.757908in,y=2.172477in,,base]{\color{textcolor}\sffamily\fontsize{10.000000}{12.000000}\selectfont 0.8}%
\end{pgfscope}%
\begin{pgfscope}%
\definecolor{textcolor}{rgb}{0.000000,0.000000,0.000000}%
\pgfsetstrokecolor{textcolor}%
\pgfsetfillcolor{textcolor}%
\pgftext[x=2.565898in,y=2.394613in,,base]{\color{textcolor}\sffamily\fontsize{10.000000}{12.000000}\selectfont 0.9}%
\end{pgfscope}%
\begin{pgfscope}%
\definecolor{textcolor}{rgb}{0.000000,0.000000,0.000000}%
\pgfsetstrokecolor{textcolor}%
\pgfsetfillcolor{textcolor}%
\pgftext[x=2.373888in,y=2.616748in,,base]{\color{textcolor}\sffamily\fontsize{10.000000}{12.000000}\selectfont 1.0}%
\end{pgfscope}%
\begin{pgfscope}%
\definecolor{textcolor}{rgb}{0.000000,0.000000,0.000000}%
\pgfsetstrokecolor{textcolor}%
\pgfsetfillcolor{textcolor}%
\pgftext[x=0.296340in,y=0.439820in,,base]{\color{textcolor}\sffamily\fontsize{10.000000}{12.000000}\selectfont 1.0}%
\end{pgfscope}%
\begin{pgfscope}%
\definecolor{textcolor}{rgb}{0.000000,0.000000,0.000000}%
\pgfsetstrokecolor{textcolor}%
\pgfsetfillcolor{textcolor}%
\pgftext[x=0.488350in,y=0.661956in,,base]{\color{textcolor}\sffamily\fontsize{10.000000}{12.000000}\selectfont 0.9}%
\end{pgfscope}%
\begin{pgfscope}%
\definecolor{textcolor}{rgb}{0.000000,0.000000,0.000000}%
\pgfsetstrokecolor{textcolor}%
\pgfsetfillcolor{textcolor}%
\pgftext[x=0.680360in,y=0.884091in,,base]{\color{textcolor}\sffamily\fontsize{10.000000}{12.000000}\selectfont 0.8}%
\end{pgfscope}%
\begin{pgfscope}%
\definecolor{textcolor}{rgb}{0.000000,0.000000,0.000000}%
\pgfsetstrokecolor{textcolor}%
\pgfsetfillcolor{textcolor}%
\pgftext[x=0.872370in,y=1.106227in,,base]{\color{textcolor}\sffamily\fontsize{10.000000}{12.000000}\selectfont 0.7}%
\end{pgfscope}%
\begin{pgfscope}%
\definecolor{textcolor}{rgb}{0.000000,0.000000,0.000000}%
\pgfsetstrokecolor{textcolor}%
\pgfsetfillcolor{textcolor}%
\pgftext[x=1.064380in,y=1.328362in,,base]{\color{textcolor}\sffamily\fontsize{10.000000}{12.000000}\selectfont 0.6}%
\end{pgfscope}%
\begin{pgfscope}%
\definecolor{textcolor}{rgb}{0.000000,0.000000,0.000000}%
\pgfsetstrokecolor{textcolor}%
\pgfsetfillcolor{textcolor}%
\pgftext[x=1.256390in,y=1.550498in,,base]{\color{textcolor}\sffamily\fontsize{10.000000}{12.000000}\selectfont 0.5}%
\end{pgfscope}%
\begin{pgfscope}%
\definecolor{textcolor}{rgb}{0.000000,0.000000,0.000000}%
\pgfsetstrokecolor{textcolor}%
\pgfsetfillcolor{textcolor}%
\pgftext[x=1.448400in,y=1.772633in,,base]{\color{textcolor}\sffamily\fontsize{10.000000}{12.000000}\selectfont 0.4}%
\end{pgfscope}%
\begin{pgfscope}%
\definecolor{textcolor}{rgb}{0.000000,0.000000,0.000000}%
\pgfsetstrokecolor{textcolor}%
\pgfsetfillcolor{textcolor}%
\pgftext[x=1.640410in,y=1.994769in,,base]{\color{textcolor}\sffamily\fontsize{10.000000}{12.000000}\selectfont 0.3}%
\end{pgfscope}%
\begin{pgfscope}%
\definecolor{textcolor}{rgb}{0.000000,0.000000,0.000000}%
\pgfsetstrokecolor{textcolor}%
\pgfsetfillcolor{textcolor}%
\pgftext[x=1.832420in,y=2.216904in,,base]{\color{textcolor}\sffamily\fontsize{10.000000}{12.000000}\selectfont 0.2}%
\end{pgfscope}%
\begin{pgfscope}%
\definecolor{textcolor}{rgb}{0.000000,0.000000,0.000000}%
\pgfsetstrokecolor{textcolor}%
\pgfsetfillcolor{textcolor}%
\pgftext[x=2.024430in,y=2.439040in,,base]{\color{textcolor}\sffamily\fontsize{10.000000}{12.000000}\selectfont 0.1}%
\end{pgfscope}%
\begin{pgfscope}%
\definecolor{textcolor}{rgb}{0.000000,0.000000,0.000000}%
\pgfsetstrokecolor{textcolor}%
\pgfsetfillcolor{textcolor}%
\pgftext[x=2.216440in,y=2.661175in,,base]{\color{textcolor}\sffamily\fontsize{10.000000}{12.000000}\selectfont 0.0}%
\end{pgfscope}%
\begin{pgfscope}%
\definecolor{textcolor}{rgb}{0.000000,0.000000,0.000000}%
\pgfsetstrokecolor{textcolor}%
\pgfsetfillcolor{textcolor}%
\pgftext[x=0.296340in,y=0.328753in,,base]{\color{textcolor}\sffamily\fontsize{10.000000}{12.000000}\selectfont 0.0}%
\end{pgfscope}%
\begin{pgfscope}%
\definecolor{textcolor}{rgb}{0.000000,0.000000,0.000000}%
\pgfsetstrokecolor{textcolor}%
\pgfsetfillcolor{textcolor}%
\pgftext[x=0.680360in,y=0.328753in,,base]{\color{textcolor}\sffamily\fontsize{10.000000}{12.000000}\selectfont 0.1}%
\end{pgfscope}%
\begin{pgfscope}%
\definecolor{textcolor}{rgb}{0.000000,0.000000,0.000000}%
\pgfsetstrokecolor{textcolor}%
\pgfsetfillcolor{textcolor}%
\pgftext[x=1.064380in,y=0.328753in,,base]{\color{textcolor}\sffamily\fontsize{10.000000}{12.000000}\selectfont 0.2}%
\end{pgfscope}%
\begin{pgfscope}%
\definecolor{textcolor}{rgb}{0.000000,0.000000,0.000000}%
\pgfsetstrokecolor{textcolor}%
\pgfsetfillcolor{textcolor}%
\pgftext[x=1.448400in,y=0.328753in,,base]{\color{textcolor}\sffamily\fontsize{10.000000}{12.000000}\selectfont 0.3}%
\end{pgfscope}%
\begin{pgfscope}%
\definecolor{textcolor}{rgb}{0.000000,0.000000,0.000000}%
\pgfsetstrokecolor{textcolor}%
\pgfsetfillcolor{textcolor}%
\pgftext[x=1.832420in,y=0.328753in,,base]{\color{textcolor}\sffamily\fontsize{10.000000}{12.000000}\selectfont 0.4}%
\end{pgfscope}%
\begin{pgfscope}%
\definecolor{textcolor}{rgb}{0.000000,0.000000,0.000000}%
\pgfsetstrokecolor{textcolor}%
\pgfsetfillcolor{textcolor}%
\pgftext[x=2.216440in,y=0.328753in,,base]{\color{textcolor}\sffamily\fontsize{10.000000}{12.000000}\selectfont 0.5}%
\end{pgfscope}%
\begin{pgfscope}%
\definecolor{textcolor}{rgb}{0.000000,0.000000,0.000000}%
\pgfsetstrokecolor{textcolor}%
\pgfsetfillcolor{textcolor}%
\pgftext[x=2.600460in,y=0.328753in,,base]{\color{textcolor}\sffamily\fontsize{10.000000}{12.000000}\selectfont 0.6}%
\end{pgfscope}%
\begin{pgfscope}%
\definecolor{textcolor}{rgb}{0.000000,0.000000,0.000000}%
\pgfsetstrokecolor{textcolor}%
\pgfsetfillcolor{textcolor}%
\pgftext[x=2.984479in,y=0.328753in,,base]{\color{textcolor}\sffamily\fontsize{10.000000}{12.000000}\selectfont 0.7}%
\end{pgfscope}%
\begin{pgfscope}%
\definecolor{textcolor}{rgb}{0.000000,0.000000,0.000000}%
\pgfsetstrokecolor{textcolor}%
\pgfsetfillcolor{textcolor}%
\pgftext[x=3.368499in,y=0.328753in,,base]{\color{textcolor}\sffamily\fontsize{10.000000}{12.000000}\selectfont 0.8}%
\end{pgfscope}%
\begin{pgfscope}%
\definecolor{textcolor}{rgb}{0.000000,0.000000,0.000000}%
\pgfsetstrokecolor{textcolor}%
\pgfsetfillcolor{textcolor}%
\pgftext[x=3.752519in,y=0.328753in,,base]{\color{textcolor}\sffamily\fontsize{10.000000}{12.000000}\selectfont 0.9}%
\end{pgfscope}%
\begin{pgfscope}%
\definecolor{textcolor}{rgb}{0.000000,0.000000,0.000000}%
\pgfsetstrokecolor{textcolor}%
\pgfsetfillcolor{textcolor}%
\pgftext[x=4.136539in,y=0.328753in,,base]{\color{textcolor}\sffamily\fontsize{10.000000}{12.000000}\selectfont 1.0}%
\end{pgfscope}%
\begin{pgfscope}%
\pgfpathrectangle{\pgfqpoint{0.152333in}{0.150000in}}{\pgfqpoint{4.224218in}{2.565000in}}%
\pgfusepath{clip}%
\pgfsetbuttcap%
\pgfsetroundjoin%
\definecolor{currentfill}{rgb}{1.000000,0.000000,0.000000}%
\pgfsetfillcolor{currentfill}%
\pgfsetlinewidth{1.505625pt}%
\definecolor{currentstroke}{rgb}{1.000000,0.000000,0.000000}%
\pgfsetstrokecolor{currentstroke}%
\pgfsetdash{}{0pt}%
\pgfsys@defobject{currentmarker}{\pgfqpoint{-0.013608in}{-0.017010in}}{\pgfqpoint{0.013608in}{0.008505in}}{%
\pgfpathmoveto{\pgfqpoint{0.000000in}{0.000000in}}%
\pgfpathlineto{\pgfqpoint{0.000000in}{-0.017010in}}%
\pgfpathmoveto{\pgfqpoint{0.000000in}{0.000000in}}%
\pgfpathlineto{\pgfqpoint{0.013608in}{0.008505in}}%
\pgfpathmoveto{\pgfqpoint{0.000000in}{0.000000in}}%
\pgfpathlineto{\pgfqpoint{-0.013608in}{0.008505in}}%
\pgfusepath{stroke,fill}%
}%
\begin{pgfscope}%
\pgfsys@transformshift{1.715842in}{1.358509in}%
\pgfsys@useobject{currentmarker}{}%
\end{pgfscope}%
\end{pgfscope}%
\begin{pgfscope}%
\definecolor{textcolor}{rgb}{0.000000,0.000000,0.000000}%
\pgfsetstrokecolor{textcolor}%
\pgfsetfillcolor{textcolor}%
\pgftext[x=2.264442in,y=2.798333in,,base]{\color{textcolor}\sffamily\fontsize{12.000000}{14.400000}\selectfont NeuRD - Simultaneous}%
\end{pgfscope}%
\begin{pgfscope}%
\pgfsetbuttcap%
\pgfsetmiterjoin%
\definecolor{currentfill}{rgb}{1.000000,1.000000,1.000000}%
\pgfsetfillcolor{currentfill}%
\pgfsetlinewidth{0.000000pt}%
\definecolor{currentstroke}{rgb}{0.000000,0.000000,0.000000}%
\pgfsetstrokecolor{currentstroke}%
\pgfsetstrokeopacity{0.000000}%
\pgfsetdash{}{0pt}%
\pgfpathmoveto{\pgfqpoint{4.577333in}{0.150000in}}%
\pgfpathlineto{\pgfqpoint{8.801551in}{0.150000in}}%
\pgfpathlineto{\pgfqpoint{8.801551in}{2.715000in}}%
\pgfpathlineto{\pgfqpoint{4.577333in}{2.715000in}}%
\pgfpathlineto{\pgfqpoint{4.577333in}{0.150000in}}%
\pgfpathclose%
\pgfusepath{fill}%
\end{pgfscope}%
\begin{pgfscope}%
\pgfpathrectangle{\pgfqpoint{4.577333in}{0.150000in}}{\pgfqpoint{4.224218in}{2.565000in}}%
\pgfusepath{clip}%
\pgfsetbuttcap%
\pgfsetmiterjoin%
\definecolor{currentfill}{rgb}{0.960784,0.960784,0.960784}%
\pgfsetfillcolor{currentfill}%
\pgfsetfillopacity{0.750000}%
\pgfsetlinewidth{1.003750pt}%
\definecolor{currentstroke}{rgb}{0.960784,0.960784,0.960784}%
\pgfsetstrokecolor{currentstroke}%
\pgfsetstrokeopacity{0.750000}%
\pgfsetdash{}{0pt}%
\pgfpathmoveto{\pgfqpoint{8.609541in}{0.406500in}}%
\pgfpathlineto{\pgfqpoint{6.689442in}{2.627855in}}%
\pgfpathlineto{\pgfqpoint{4.769343in}{0.406500in}}%
\pgfpathlineto{\pgfqpoint{8.609541in}{0.406500in}}%
\pgfpathclose%
\pgfusepath{stroke,fill}%
\end{pgfscope}%
\begin{pgfscope}%
\pgfpathrectangle{\pgfqpoint{4.577333in}{0.150000in}}{\pgfqpoint{4.224218in}{2.565000in}}%
\pgfusepath{clip}%
\pgfsetrectcap%
\pgfsetroundjoin%
\pgfsetlinewidth{0.803000pt}%
\definecolor{currentstroke}{rgb}{0.000000,0.000000,0.000000}%
\pgfsetstrokecolor{currentstroke}%
\pgfsetdash{}{0pt}%
\pgfpathmoveto{\pgfqpoint{4.769343in}{0.406500in}}%
\pgfpathlineto{\pgfqpoint{8.609541in}{0.406500in}}%
\pgfusepath{stroke}%
\end{pgfscope}%
\begin{pgfscope}%
\pgfpathrectangle{\pgfqpoint{4.577333in}{0.150000in}}{\pgfqpoint{4.224218in}{2.565000in}}%
\pgfusepath{clip}%
\pgfsetrectcap%
\pgfsetroundjoin%
\pgfsetlinewidth{0.803000pt}%
\definecolor{currentstroke}{rgb}{0.000000,0.000000,0.000000}%
\pgfsetstrokecolor{currentstroke}%
\pgfsetdash{}{0pt}%
\pgfpathmoveto{\pgfqpoint{6.689442in}{2.627855in}}%
\pgfpathlineto{\pgfqpoint{4.769343in}{0.406500in}}%
\pgfusepath{stroke}%
\end{pgfscope}%
\begin{pgfscope}%
\pgfpathrectangle{\pgfqpoint{4.577333in}{0.150000in}}{\pgfqpoint{4.224218in}{2.565000in}}%
\pgfusepath{clip}%
\pgfsetrectcap%
\pgfsetroundjoin%
\pgfsetlinewidth{0.803000pt}%
\definecolor{currentstroke}{rgb}{0.000000,0.000000,0.000000}%
\pgfsetstrokecolor{currentstroke}%
\pgfsetdash{}{0pt}%
\pgfpathmoveto{\pgfqpoint{6.689442in}{2.627855in}}%
\pgfpathlineto{\pgfqpoint{8.609541in}{0.406500in}}%
\pgfusepath{stroke}%
\end{pgfscope}%
\begin{pgfscope}%
\pgfpathrectangle{\pgfqpoint{4.577333in}{0.150000in}}{\pgfqpoint{4.224218in}{2.565000in}}%
\pgfusepath{clip}%
\pgfsetbuttcap%
\pgfsetroundjoin%
\pgfsetlinewidth{0.501875pt}%
\definecolor{currentstroke}{rgb}{0.000000,0.000000,0.000000}%
\pgfsetstrokecolor{currentstroke}%
\pgfsetdash{{0.500000pt}{0.825000pt}}{0.000000pt}%
\pgfpathmoveto{\pgfqpoint{4.769343in}{0.406500in}}%
\pgfpathlineto{\pgfqpoint{8.609541in}{0.406500in}}%
\pgfusepath{stroke}%
\end{pgfscope}%
\begin{pgfscope}%
\pgfpathrectangle{\pgfqpoint{4.577333in}{0.150000in}}{\pgfqpoint{4.224218in}{2.565000in}}%
\pgfusepath{clip}%
\pgfsetbuttcap%
\pgfsetroundjoin%
\pgfsetlinewidth{0.501875pt}%
\definecolor{currentstroke}{rgb}{0.000000,0.000000,0.000000}%
\pgfsetstrokecolor{currentstroke}%
\pgfsetdash{{0.500000pt}{0.825000pt}}{0.000000pt}%
\pgfpathmoveto{\pgfqpoint{5.729393in}{1.517178in}}%
\pgfpathlineto{\pgfqpoint{7.649492in}{1.517178in}}%
\pgfusepath{stroke}%
\end{pgfscope}%
\begin{pgfscope}%
\pgfpathrectangle{\pgfqpoint{4.577333in}{0.150000in}}{\pgfqpoint{4.224218in}{2.565000in}}%
\pgfusepath{clip}%
\pgfsetbuttcap%
\pgfsetroundjoin%
\pgfsetlinewidth{0.501875pt}%
\definecolor{currentstroke}{rgb}{0.000000,0.000000,0.000000}%
\pgfsetstrokecolor{currentstroke}%
\pgfsetdash{{0.500000pt}{0.825000pt}}{0.000000pt}%
\pgfpathmoveto{\pgfqpoint{6.689442in}{2.627855in}}%
\pgfpathlineto{\pgfqpoint{4.769343in}{0.406500in}}%
\pgfusepath{stroke}%
\end{pgfscope}%
\begin{pgfscope}%
\pgfpathrectangle{\pgfqpoint{4.577333in}{0.150000in}}{\pgfqpoint{4.224218in}{2.565000in}}%
\pgfusepath{clip}%
\pgfsetbuttcap%
\pgfsetroundjoin%
\pgfsetlinewidth{0.501875pt}%
\definecolor{currentstroke}{rgb}{0.000000,0.000000,0.000000}%
\pgfsetstrokecolor{currentstroke}%
\pgfsetdash{{0.500000pt}{0.825000pt}}{0.000000pt}%
\pgfpathmoveto{\pgfqpoint{6.689442in}{2.627855in}}%
\pgfpathlineto{\pgfqpoint{8.609541in}{0.406500in}}%
\pgfusepath{stroke}%
\end{pgfscope}%
\begin{pgfscope}%
\pgfpathrectangle{\pgfqpoint{4.577333in}{0.150000in}}{\pgfqpoint{4.224218in}{2.565000in}}%
\pgfusepath{clip}%
\pgfsetbuttcap%
\pgfsetroundjoin%
\pgfsetlinewidth{0.501875pt}%
\definecolor{currentstroke}{rgb}{0.000000,0.000000,0.000000}%
\pgfsetstrokecolor{currentstroke}%
\pgfsetdash{{0.500000pt}{0.825000pt}}{0.000000pt}%
\pgfpathmoveto{\pgfqpoint{7.649492in}{1.517178in}}%
\pgfpathlineto{\pgfqpoint{6.689442in}{0.406500in}}%
\pgfusepath{stroke}%
\end{pgfscope}%
\begin{pgfscope}%
\pgfpathrectangle{\pgfqpoint{4.577333in}{0.150000in}}{\pgfqpoint{4.224218in}{2.565000in}}%
\pgfusepath{clip}%
\pgfsetbuttcap%
\pgfsetroundjoin%
\pgfsetlinewidth{0.501875pt}%
\definecolor{currentstroke}{rgb}{0.000000,0.000000,0.000000}%
\pgfsetstrokecolor{currentstroke}%
\pgfsetdash{{0.500000pt}{0.825000pt}}{0.000000pt}%
\pgfpathmoveto{\pgfqpoint{5.729393in}{1.517178in}}%
\pgfpathlineto{\pgfqpoint{6.689442in}{0.406500in}}%
\pgfusepath{stroke}%
\end{pgfscope}%
\begin{pgfscope}%
\pgfpathrectangle{\pgfqpoint{4.577333in}{0.150000in}}{\pgfqpoint{4.224218in}{2.565000in}}%
\pgfusepath{clip}%
\pgfsetbuttcap%
\pgfsetroundjoin%
\pgfsetlinewidth{0.501875pt}%
\definecolor{currentstroke}{rgb}{0.000000,0.000000,0.000000}%
\pgfsetstrokecolor{currentstroke}%
\pgfsetdash{{0.500000pt}{0.825000pt}}{0.000000pt}%
\pgfpathmoveto{\pgfqpoint{8.609541in}{0.406500in}}%
\pgfpathlineto{\pgfqpoint{8.609541in}{0.406500in}}%
\pgfusepath{stroke}%
\end{pgfscope}%
\begin{pgfscope}%
\pgfpathrectangle{\pgfqpoint{4.577333in}{0.150000in}}{\pgfqpoint{4.224218in}{2.565000in}}%
\pgfusepath{clip}%
\pgfsetbuttcap%
\pgfsetroundjoin%
\pgfsetlinewidth{0.501875pt}%
\definecolor{currentstroke}{rgb}{0.000000,0.000000,0.000000}%
\pgfsetstrokecolor{currentstroke}%
\pgfsetdash{{0.500000pt}{0.825000pt}}{0.000000pt}%
\pgfpathmoveto{\pgfqpoint{4.769343in}{0.406500in}}%
\pgfpathlineto{\pgfqpoint{4.769343in}{0.406500in}}%
\pgfusepath{stroke}%
\end{pgfscope}%
\begin{pgfscope}%
\pgfpathrectangle{\pgfqpoint{4.577333in}{0.150000in}}{\pgfqpoint{4.224218in}{2.565000in}}%
\pgfusepath{clip}%
\pgfsetbuttcap%
\pgfsetroundjoin%
\pgfsetlinewidth{0.501875pt}%
\definecolor{currentstroke}{rgb}{0.000000,0.000000,1.000000}%
\pgfsetstrokecolor{currentstroke}%
\pgfsetdash{{0.500000pt}{0.825000pt}}{0.000000pt}%
\pgfpathmoveto{\pgfqpoint{4.769343in}{0.406500in}}%
\pgfpathlineto{\pgfqpoint{8.609541in}{0.406500in}}%
\pgfusepath{stroke}%
\end{pgfscope}%
\begin{pgfscope}%
\pgfpathrectangle{\pgfqpoint{4.577333in}{0.150000in}}{\pgfqpoint{4.224218in}{2.565000in}}%
\pgfusepath{clip}%
\pgfsetbuttcap%
\pgfsetroundjoin%
\pgfsetlinewidth{0.501875pt}%
\definecolor{currentstroke}{rgb}{0.000000,0.000000,1.000000}%
\pgfsetstrokecolor{currentstroke}%
\pgfsetdash{{0.500000pt}{0.825000pt}}{0.000000pt}%
\pgfpathmoveto{\pgfqpoint{4.961353in}{0.628636in}}%
\pgfpathlineto{\pgfqpoint{8.417531in}{0.628636in}}%
\pgfusepath{stroke}%
\end{pgfscope}%
\begin{pgfscope}%
\pgfpathrectangle{\pgfqpoint{4.577333in}{0.150000in}}{\pgfqpoint{4.224218in}{2.565000in}}%
\pgfusepath{clip}%
\pgfsetbuttcap%
\pgfsetroundjoin%
\pgfsetlinewidth{0.501875pt}%
\definecolor{currentstroke}{rgb}{0.000000,0.000000,1.000000}%
\pgfsetstrokecolor{currentstroke}%
\pgfsetdash{{0.500000pt}{0.825000pt}}{0.000000pt}%
\pgfpathmoveto{\pgfqpoint{5.153363in}{0.850771in}}%
\pgfpathlineto{\pgfqpoint{8.225522in}{0.850771in}}%
\pgfusepath{stroke}%
\end{pgfscope}%
\begin{pgfscope}%
\pgfpathrectangle{\pgfqpoint{4.577333in}{0.150000in}}{\pgfqpoint{4.224218in}{2.565000in}}%
\pgfusepath{clip}%
\pgfsetbuttcap%
\pgfsetroundjoin%
\pgfsetlinewidth{0.501875pt}%
\definecolor{currentstroke}{rgb}{0.000000,0.000000,1.000000}%
\pgfsetstrokecolor{currentstroke}%
\pgfsetdash{{0.500000pt}{0.825000pt}}{0.000000pt}%
\pgfpathmoveto{\pgfqpoint{5.345373in}{1.072907in}}%
\pgfpathlineto{\pgfqpoint{8.033512in}{1.072907in}}%
\pgfusepath{stroke}%
\end{pgfscope}%
\begin{pgfscope}%
\pgfpathrectangle{\pgfqpoint{4.577333in}{0.150000in}}{\pgfqpoint{4.224218in}{2.565000in}}%
\pgfusepath{clip}%
\pgfsetbuttcap%
\pgfsetroundjoin%
\pgfsetlinewidth{0.501875pt}%
\definecolor{currentstroke}{rgb}{0.000000,0.000000,1.000000}%
\pgfsetstrokecolor{currentstroke}%
\pgfsetdash{{0.500000pt}{0.825000pt}}{0.000000pt}%
\pgfpathmoveto{\pgfqpoint{5.537383in}{1.295042in}}%
\pgfpathlineto{\pgfqpoint{7.841502in}{1.295042in}}%
\pgfusepath{stroke}%
\end{pgfscope}%
\begin{pgfscope}%
\pgfpathrectangle{\pgfqpoint{4.577333in}{0.150000in}}{\pgfqpoint{4.224218in}{2.565000in}}%
\pgfusepath{clip}%
\pgfsetbuttcap%
\pgfsetroundjoin%
\pgfsetlinewidth{0.501875pt}%
\definecolor{currentstroke}{rgb}{0.000000,0.000000,1.000000}%
\pgfsetstrokecolor{currentstroke}%
\pgfsetdash{{0.500000pt}{0.825000pt}}{0.000000pt}%
\pgfpathmoveto{\pgfqpoint{5.729393in}{1.517178in}}%
\pgfpathlineto{\pgfqpoint{7.649492in}{1.517178in}}%
\pgfusepath{stroke}%
\end{pgfscope}%
\begin{pgfscope}%
\pgfpathrectangle{\pgfqpoint{4.577333in}{0.150000in}}{\pgfqpoint{4.224218in}{2.565000in}}%
\pgfusepath{clip}%
\pgfsetbuttcap%
\pgfsetroundjoin%
\pgfsetlinewidth{0.501875pt}%
\definecolor{currentstroke}{rgb}{0.000000,0.000000,1.000000}%
\pgfsetstrokecolor{currentstroke}%
\pgfsetdash{{0.500000pt}{0.825000pt}}{0.000000pt}%
\pgfpathmoveto{\pgfqpoint{5.921402in}{1.739313in}}%
\pgfpathlineto{\pgfqpoint{7.457482in}{1.739313in}}%
\pgfusepath{stroke}%
\end{pgfscope}%
\begin{pgfscope}%
\pgfpathrectangle{\pgfqpoint{4.577333in}{0.150000in}}{\pgfqpoint{4.224218in}{2.565000in}}%
\pgfusepath{clip}%
\pgfsetbuttcap%
\pgfsetroundjoin%
\pgfsetlinewidth{0.501875pt}%
\definecolor{currentstroke}{rgb}{0.000000,0.000000,1.000000}%
\pgfsetstrokecolor{currentstroke}%
\pgfsetdash{{0.500000pt}{0.825000pt}}{0.000000pt}%
\pgfpathmoveto{\pgfqpoint{6.113412in}{1.961449in}}%
\pgfpathlineto{\pgfqpoint{7.265472in}{1.961449in}}%
\pgfusepath{stroke}%
\end{pgfscope}%
\begin{pgfscope}%
\pgfpathrectangle{\pgfqpoint{4.577333in}{0.150000in}}{\pgfqpoint{4.224218in}{2.565000in}}%
\pgfusepath{clip}%
\pgfsetbuttcap%
\pgfsetroundjoin%
\pgfsetlinewidth{0.501875pt}%
\definecolor{currentstroke}{rgb}{0.000000,0.000000,1.000000}%
\pgfsetstrokecolor{currentstroke}%
\pgfsetdash{{0.500000pt}{0.825000pt}}{0.000000pt}%
\pgfpathmoveto{\pgfqpoint{6.305422in}{2.183584in}}%
\pgfpathlineto{\pgfqpoint{7.073462in}{2.183584in}}%
\pgfusepath{stroke}%
\end{pgfscope}%
\begin{pgfscope}%
\pgfpathrectangle{\pgfqpoint{4.577333in}{0.150000in}}{\pgfqpoint{4.224218in}{2.565000in}}%
\pgfusepath{clip}%
\pgfsetbuttcap%
\pgfsetroundjoin%
\pgfsetlinewidth{0.501875pt}%
\definecolor{currentstroke}{rgb}{0.000000,0.000000,1.000000}%
\pgfsetstrokecolor{currentstroke}%
\pgfsetdash{{0.500000pt}{0.825000pt}}{0.000000pt}%
\pgfpathmoveto{\pgfqpoint{6.497432in}{2.405720in}}%
\pgfpathlineto{\pgfqpoint{6.881452in}{2.405720in}}%
\pgfusepath{stroke}%
\end{pgfscope}%
\begin{pgfscope}%
\pgfpathrectangle{\pgfqpoint{4.577333in}{0.150000in}}{\pgfqpoint{4.224218in}{2.565000in}}%
\pgfusepath{clip}%
\pgfsetbuttcap%
\pgfsetroundjoin%
\pgfsetlinewidth{0.501875pt}%
\definecolor{currentstroke}{rgb}{0.000000,0.000000,1.000000}%
\pgfsetstrokecolor{currentstroke}%
\pgfsetdash{{0.500000pt}{0.825000pt}}{0.000000pt}%
\pgfpathmoveto{\pgfqpoint{6.689442in}{2.627855in}}%
\pgfpathlineto{\pgfqpoint{4.769343in}{0.406500in}}%
\pgfusepath{stroke}%
\end{pgfscope}%
\begin{pgfscope}%
\pgfpathrectangle{\pgfqpoint{4.577333in}{0.150000in}}{\pgfqpoint{4.224218in}{2.565000in}}%
\pgfusepath{clip}%
\pgfsetbuttcap%
\pgfsetroundjoin%
\pgfsetlinewidth{0.501875pt}%
\definecolor{currentstroke}{rgb}{0.000000,0.000000,1.000000}%
\pgfsetstrokecolor{currentstroke}%
\pgfsetdash{{0.500000pt}{0.825000pt}}{0.000000pt}%
\pgfpathmoveto{\pgfqpoint{6.689442in}{2.627855in}}%
\pgfpathlineto{\pgfqpoint{8.609541in}{0.406500in}}%
\pgfusepath{stroke}%
\end{pgfscope}%
\begin{pgfscope}%
\pgfpathrectangle{\pgfqpoint{4.577333in}{0.150000in}}{\pgfqpoint{4.224218in}{2.565000in}}%
\pgfusepath{clip}%
\pgfsetbuttcap%
\pgfsetroundjoin%
\pgfsetlinewidth{0.501875pt}%
\definecolor{currentstroke}{rgb}{0.000000,0.000000,1.000000}%
\pgfsetstrokecolor{currentstroke}%
\pgfsetdash{{0.500000pt}{0.825000pt}}{0.000000pt}%
\pgfpathmoveto{\pgfqpoint{6.881452in}{2.405720in}}%
\pgfpathlineto{\pgfqpoint{5.153363in}{0.406500in}}%
\pgfusepath{stroke}%
\end{pgfscope}%
\begin{pgfscope}%
\pgfpathrectangle{\pgfqpoint{4.577333in}{0.150000in}}{\pgfqpoint{4.224218in}{2.565000in}}%
\pgfusepath{clip}%
\pgfsetbuttcap%
\pgfsetroundjoin%
\pgfsetlinewidth{0.501875pt}%
\definecolor{currentstroke}{rgb}{0.000000,0.000000,1.000000}%
\pgfsetstrokecolor{currentstroke}%
\pgfsetdash{{0.500000pt}{0.825000pt}}{0.000000pt}%
\pgfpathmoveto{\pgfqpoint{6.497432in}{2.405720in}}%
\pgfpathlineto{\pgfqpoint{8.225522in}{0.406500in}}%
\pgfusepath{stroke}%
\end{pgfscope}%
\begin{pgfscope}%
\pgfpathrectangle{\pgfqpoint{4.577333in}{0.150000in}}{\pgfqpoint{4.224218in}{2.565000in}}%
\pgfusepath{clip}%
\pgfsetbuttcap%
\pgfsetroundjoin%
\pgfsetlinewidth{0.501875pt}%
\definecolor{currentstroke}{rgb}{0.000000,0.000000,1.000000}%
\pgfsetstrokecolor{currentstroke}%
\pgfsetdash{{0.500000pt}{0.825000pt}}{0.000000pt}%
\pgfpathmoveto{\pgfqpoint{7.073462in}{2.183584in}}%
\pgfpathlineto{\pgfqpoint{5.537383in}{0.406500in}}%
\pgfusepath{stroke}%
\end{pgfscope}%
\begin{pgfscope}%
\pgfpathrectangle{\pgfqpoint{4.577333in}{0.150000in}}{\pgfqpoint{4.224218in}{2.565000in}}%
\pgfusepath{clip}%
\pgfsetbuttcap%
\pgfsetroundjoin%
\pgfsetlinewidth{0.501875pt}%
\definecolor{currentstroke}{rgb}{0.000000,0.000000,1.000000}%
\pgfsetstrokecolor{currentstroke}%
\pgfsetdash{{0.500000pt}{0.825000pt}}{0.000000pt}%
\pgfpathmoveto{\pgfqpoint{6.305422in}{2.183584in}}%
\pgfpathlineto{\pgfqpoint{7.841502in}{0.406500in}}%
\pgfusepath{stroke}%
\end{pgfscope}%
\begin{pgfscope}%
\pgfpathrectangle{\pgfqpoint{4.577333in}{0.150000in}}{\pgfqpoint{4.224218in}{2.565000in}}%
\pgfusepath{clip}%
\pgfsetbuttcap%
\pgfsetroundjoin%
\pgfsetlinewidth{0.501875pt}%
\definecolor{currentstroke}{rgb}{0.000000,0.000000,1.000000}%
\pgfsetstrokecolor{currentstroke}%
\pgfsetdash{{0.500000pt}{0.825000pt}}{0.000000pt}%
\pgfpathmoveto{\pgfqpoint{7.265472in}{1.961449in}}%
\pgfpathlineto{\pgfqpoint{5.921402in}{0.406500in}}%
\pgfusepath{stroke}%
\end{pgfscope}%
\begin{pgfscope}%
\pgfpathrectangle{\pgfqpoint{4.577333in}{0.150000in}}{\pgfqpoint{4.224218in}{2.565000in}}%
\pgfusepath{clip}%
\pgfsetbuttcap%
\pgfsetroundjoin%
\pgfsetlinewidth{0.501875pt}%
\definecolor{currentstroke}{rgb}{0.000000,0.000000,1.000000}%
\pgfsetstrokecolor{currentstroke}%
\pgfsetdash{{0.500000pt}{0.825000pt}}{0.000000pt}%
\pgfpathmoveto{\pgfqpoint{6.113412in}{1.961449in}}%
\pgfpathlineto{\pgfqpoint{7.457482in}{0.406500in}}%
\pgfusepath{stroke}%
\end{pgfscope}%
\begin{pgfscope}%
\pgfpathrectangle{\pgfqpoint{4.577333in}{0.150000in}}{\pgfqpoint{4.224218in}{2.565000in}}%
\pgfusepath{clip}%
\pgfsetbuttcap%
\pgfsetroundjoin%
\pgfsetlinewidth{0.501875pt}%
\definecolor{currentstroke}{rgb}{0.000000,0.000000,1.000000}%
\pgfsetstrokecolor{currentstroke}%
\pgfsetdash{{0.500000pt}{0.825000pt}}{0.000000pt}%
\pgfpathmoveto{\pgfqpoint{7.457482in}{1.739313in}}%
\pgfpathlineto{\pgfqpoint{6.305422in}{0.406500in}}%
\pgfusepath{stroke}%
\end{pgfscope}%
\begin{pgfscope}%
\pgfpathrectangle{\pgfqpoint{4.577333in}{0.150000in}}{\pgfqpoint{4.224218in}{2.565000in}}%
\pgfusepath{clip}%
\pgfsetbuttcap%
\pgfsetroundjoin%
\pgfsetlinewidth{0.501875pt}%
\definecolor{currentstroke}{rgb}{0.000000,0.000000,1.000000}%
\pgfsetstrokecolor{currentstroke}%
\pgfsetdash{{0.500000pt}{0.825000pt}}{0.000000pt}%
\pgfpathmoveto{\pgfqpoint{5.921402in}{1.739313in}}%
\pgfpathlineto{\pgfqpoint{7.073462in}{0.406500in}}%
\pgfusepath{stroke}%
\end{pgfscope}%
\begin{pgfscope}%
\pgfpathrectangle{\pgfqpoint{4.577333in}{0.150000in}}{\pgfqpoint{4.224218in}{2.565000in}}%
\pgfusepath{clip}%
\pgfsetbuttcap%
\pgfsetroundjoin%
\pgfsetlinewidth{0.501875pt}%
\definecolor{currentstroke}{rgb}{0.000000,0.000000,1.000000}%
\pgfsetstrokecolor{currentstroke}%
\pgfsetdash{{0.500000pt}{0.825000pt}}{0.000000pt}%
\pgfpathmoveto{\pgfqpoint{7.649492in}{1.517178in}}%
\pgfpathlineto{\pgfqpoint{6.689442in}{0.406500in}}%
\pgfusepath{stroke}%
\end{pgfscope}%
\begin{pgfscope}%
\pgfpathrectangle{\pgfqpoint{4.577333in}{0.150000in}}{\pgfqpoint{4.224218in}{2.565000in}}%
\pgfusepath{clip}%
\pgfsetbuttcap%
\pgfsetroundjoin%
\pgfsetlinewidth{0.501875pt}%
\definecolor{currentstroke}{rgb}{0.000000,0.000000,1.000000}%
\pgfsetstrokecolor{currentstroke}%
\pgfsetdash{{0.500000pt}{0.825000pt}}{0.000000pt}%
\pgfpathmoveto{\pgfqpoint{5.729393in}{1.517178in}}%
\pgfpathlineto{\pgfqpoint{6.689442in}{0.406500in}}%
\pgfusepath{stroke}%
\end{pgfscope}%
\begin{pgfscope}%
\pgfpathrectangle{\pgfqpoint{4.577333in}{0.150000in}}{\pgfqpoint{4.224218in}{2.565000in}}%
\pgfusepath{clip}%
\pgfsetbuttcap%
\pgfsetroundjoin%
\pgfsetlinewidth{0.501875pt}%
\definecolor{currentstroke}{rgb}{0.000000,0.000000,1.000000}%
\pgfsetstrokecolor{currentstroke}%
\pgfsetdash{{0.500000pt}{0.825000pt}}{0.000000pt}%
\pgfpathmoveto{\pgfqpoint{7.841502in}{1.295042in}}%
\pgfpathlineto{\pgfqpoint{7.073462in}{0.406500in}}%
\pgfusepath{stroke}%
\end{pgfscope}%
\begin{pgfscope}%
\pgfpathrectangle{\pgfqpoint{4.577333in}{0.150000in}}{\pgfqpoint{4.224218in}{2.565000in}}%
\pgfusepath{clip}%
\pgfsetbuttcap%
\pgfsetroundjoin%
\pgfsetlinewidth{0.501875pt}%
\definecolor{currentstroke}{rgb}{0.000000,0.000000,1.000000}%
\pgfsetstrokecolor{currentstroke}%
\pgfsetdash{{0.500000pt}{0.825000pt}}{0.000000pt}%
\pgfpathmoveto{\pgfqpoint{5.537383in}{1.295042in}}%
\pgfpathlineto{\pgfqpoint{6.305422in}{0.406500in}}%
\pgfusepath{stroke}%
\end{pgfscope}%
\begin{pgfscope}%
\pgfpathrectangle{\pgfqpoint{4.577333in}{0.150000in}}{\pgfqpoint{4.224218in}{2.565000in}}%
\pgfusepath{clip}%
\pgfsetbuttcap%
\pgfsetroundjoin%
\pgfsetlinewidth{0.501875pt}%
\definecolor{currentstroke}{rgb}{0.000000,0.000000,1.000000}%
\pgfsetstrokecolor{currentstroke}%
\pgfsetdash{{0.500000pt}{0.825000pt}}{0.000000pt}%
\pgfpathmoveto{\pgfqpoint{8.033512in}{1.072907in}}%
\pgfpathlineto{\pgfqpoint{7.457482in}{0.406500in}}%
\pgfusepath{stroke}%
\end{pgfscope}%
\begin{pgfscope}%
\pgfpathrectangle{\pgfqpoint{4.577333in}{0.150000in}}{\pgfqpoint{4.224218in}{2.565000in}}%
\pgfusepath{clip}%
\pgfsetbuttcap%
\pgfsetroundjoin%
\pgfsetlinewidth{0.501875pt}%
\definecolor{currentstroke}{rgb}{0.000000,0.000000,1.000000}%
\pgfsetstrokecolor{currentstroke}%
\pgfsetdash{{0.500000pt}{0.825000pt}}{0.000000pt}%
\pgfpathmoveto{\pgfqpoint{5.345373in}{1.072907in}}%
\pgfpathlineto{\pgfqpoint{5.921402in}{0.406500in}}%
\pgfusepath{stroke}%
\end{pgfscope}%
\begin{pgfscope}%
\pgfpathrectangle{\pgfqpoint{4.577333in}{0.150000in}}{\pgfqpoint{4.224218in}{2.565000in}}%
\pgfusepath{clip}%
\pgfsetbuttcap%
\pgfsetroundjoin%
\pgfsetlinewidth{0.501875pt}%
\definecolor{currentstroke}{rgb}{0.000000,0.000000,1.000000}%
\pgfsetstrokecolor{currentstroke}%
\pgfsetdash{{0.500000pt}{0.825000pt}}{0.000000pt}%
\pgfpathmoveto{\pgfqpoint{8.225522in}{0.850771in}}%
\pgfpathlineto{\pgfqpoint{7.841502in}{0.406500in}}%
\pgfusepath{stroke}%
\end{pgfscope}%
\begin{pgfscope}%
\pgfpathrectangle{\pgfqpoint{4.577333in}{0.150000in}}{\pgfqpoint{4.224218in}{2.565000in}}%
\pgfusepath{clip}%
\pgfsetbuttcap%
\pgfsetroundjoin%
\pgfsetlinewidth{0.501875pt}%
\definecolor{currentstroke}{rgb}{0.000000,0.000000,1.000000}%
\pgfsetstrokecolor{currentstroke}%
\pgfsetdash{{0.500000pt}{0.825000pt}}{0.000000pt}%
\pgfpathmoveto{\pgfqpoint{5.153363in}{0.850771in}}%
\pgfpathlineto{\pgfqpoint{5.537383in}{0.406500in}}%
\pgfusepath{stroke}%
\end{pgfscope}%
\begin{pgfscope}%
\pgfpathrectangle{\pgfqpoint{4.577333in}{0.150000in}}{\pgfqpoint{4.224218in}{2.565000in}}%
\pgfusepath{clip}%
\pgfsetbuttcap%
\pgfsetroundjoin%
\pgfsetlinewidth{0.501875pt}%
\definecolor{currentstroke}{rgb}{0.000000,0.000000,1.000000}%
\pgfsetstrokecolor{currentstroke}%
\pgfsetdash{{0.500000pt}{0.825000pt}}{0.000000pt}%
\pgfpathmoveto{\pgfqpoint{8.417531in}{0.628636in}}%
\pgfpathlineto{\pgfqpoint{8.225522in}{0.406500in}}%
\pgfusepath{stroke}%
\end{pgfscope}%
\begin{pgfscope}%
\pgfpathrectangle{\pgfqpoint{4.577333in}{0.150000in}}{\pgfqpoint{4.224218in}{2.565000in}}%
\pgfusepath{clip}%
\pgfsetbuttcap%
\pgfsetroundjoin%
\pgfsetlinewidth{0.501875pt}%
\definecolor{currentstroke}{rgb}{0.000000,0.000000,1.000000}%
\pgfsetstrokecolor{currentstroke}%
\pgfsetdash{{0.500000pt}{0.825000pt}}{0.000000pt}%
\pgfpathmoveto{\pgfqpoint{4.961353in}{0.628636in}}%
\pgfpathlineto{\pgfqpoint{5.153363in}{0.406500in}}%
\pgfusepath{stroke}%
\end{pgfscope}%
\begin{pgfscope}%
\pgfpathrectangle{\pgfqpoint{4.577333in}{0.150000in}}{\pgfqpoint{4.224218in}{2.565000in}}%
\pgfusepath{clip}%
\pgfsetbuttcap%
\pgfsetroundjoin%
\pgfsetlinewidth{0.501875pt}%
\definecolor{currentstroke}{rgb}{0.000000,0.000000,1.000000}%
\pgfsetstrokecolor{currentstroke}%
\pgfsetdash{{0.500000pt}{0.825000pt}}{0.000000pt}%
\pgfpathmoveto{\pgfqpoint{8.609541in}{0.406500in}}%
\pgfpathlineto{\pgfqpoint{8.609541in}{0.406500in}}%
\pgfusepath{stroke}%
\end{pgfscope}%
\begin{pgfscope}%
\pgfpathrectangle{\pgfqpoint{4.577333in}{0.150000in}}{\pgfqpoint{4.224218in}{2.565000in}}%
\pgfusepath{clip}%
\pgfsetbuttcap%
\pgfsetroundjoin%
\pgfsetlinewidth{0.501875pt}%
\definecolor{currentstroke}{rgb}{0.000000,0.000000,1.000000}%
\pgfsetstrokecolor{currentstroke}%
\pgfsetdash{{0.500000pt}{0.825000pt}}{0.000000pt}%
\pgfpathmoveto{\pgfqpoint{4.769343in}{0.406500in}}%
\pgfpathlineto{\pgfqpoint{4.769343in}{0.406500in}}%
\pgfusepath{stroke}%
\end{pgfscope}%
\begin{pgfscope}%
\pgfpathrectangle{\pgfqpoint{4.577333in}{0.150000in}}{\pgfqpoint{4.224218in}{2.565000in}}%
\pgfusepath{clip}%
\pgfsetrectcap%
\pgfsetroundjoin%
\pgfsetlinewidth{1.003750pt}%
\definecolor{currentstroke}{rgb}{0.000000,0.000000,0.000000}%
\pgfsetstrokecolor{currentstroke}%
\pgfsetdash{}{0pt}%
\pgfpathmoveto{\pgfqpoint{8.609541in}{0.406500in}}%
\pgfpathlineto{\pgfqpoint{8.647943in}{0.406500in}}%
\pgfusepath{stroke}%
\end{pgfscope}%
\begin{pgfscope}%
\pgfpathrectangle{\pgfqpoint{4.577333in}{0.150000in}}{\pgfqpoint{4.224218in}{2.565000in}}%
\pgfusepath{clip}%
\pgfsetrectcap%
\pgfsetroundjoin%
\pgfsetlinewidth{1.003750pt}%
\definecolor{currentstroke}{rgb}{0.000000,0.000000,0.000000}%
\pgfsetstrokecolor{currentstroke}%
\pgfsetdash{}{0pt}%
\pgfpathmoveto{\pgfqpoint{8.417531in}{0.628636in}}%
\pgfpathlineto{\pgfqpoint{8.455933in}{0.628636in}}%
\pgfusepath{stroke}%
\end{pgfscope}%
\begin{pgfscope}%
\pgfpathrectangle{\pgfqpoint{4.577333in}{0.150000in}}{\pgfqpoint{4.224218in}{2.565000in}}%
\pgfusepath{clip}%
\pgfsetrectcap%
\pgfsetroundjoin%
\pgfsetlinewidth{1.003750pt}%
\definecolor{currentstroke}{rgb}{0.000000,0.000000,0.000000}%
\pgfsetstrokecolor{currentstroke}%
\pgfsetdash{}{0pt}%
\pgfpathmoveto{\pgfqpoint{8.225522in}{0.850771in}}%
\pgfpathlineto{\pgfqpoint{8.263924in}{0.850771in}}%
\pgfusepath{stroke}%
\end{pgfscope}%
\begin{pgfscope}%
\pgfpathrectangle{\pgfqpoint{4.577333in}{0.150000in}}{\pgfqpoint{4.224218in}{2.565000in}}%
\pgfusepath{clip}%
\pgfsetrectcap%
\pgfsetroundjoin%
\pgfsetlinewidth{1.003750pt}%
\definecolor{currentstroke}{rgb}{0.000000,0.000000,0.000000}%
\pgfsetstrokecolor{currentstroke}%
\pgfsetdash{}{0pt}%
\pgfpathmoveto{\pgfqpoint{8.033512in}{1.072907in}}%
\pgfpathlineto{\pgfqpoint{8.071914in}{1.072907in}}%
\pgfusepath{stroke}%
\end{pgfscope}%
\begin{pgfscope}%
\pgfpathrectangle{\pgfqpoint{4.577333in}{0.150000in}}{\pgfqpoint{4.224218in}{2.565000in}}%
\pgfusepath{clip}%
\pgfsetrectcap%
\pgfsetroundjoin%
\pgfsetlinewidth{1.003750pt}%
\definecolor{currentstroke}{rgb}{0.000000,0.000000,0.000000}%
\pgfsetstrokecolor{currentstroke}%
\pgfsetdash{}{0pt}%
\pgfpathmoveto{\pgfqpoint{7.841502in}{1.295042in}}%
\pgfpathlineto{\pgfqpoint{7.879904in}{1.295042in}}%
\pgfusepath{stroke}%
\end{pgfscope}%
\begin{pgfscope}%
\pgfpathrectangle{\pgfqpoint{4.577333in}{0.150000in}}{\pgfqpoint{4.224218in}{2.565000in}}%
\pgfusepath{clip}%
\pgfsetrectcap%
\pgfsetroundjoin%
\pgfsetlinewidth{1.003750pt}%
\definecolor{currentstroke}{rgb}{0.000000,0.000000,0.000000}%
\pgfsetstrokecolor{currentstroke}%
\pgfsetdash{}{0pt}%
\pgfpathmoveto{\pgfqpoint{7.649492in}{1.517178in}}%
\pgfpathlineto{\pgfqpoint{7.687894in}{1.517178in}}%
\pgfusepath{stroke}%
\end{pgfscope}%
\begin{pgfscope}%
\pgfpathrectangle{\pgfqpoint{4.577333in}{0.150000in}}{\pgfqpoint{4.224218in}{2.565000in}}%
\pgfusepath{clip}%
\pgfsetrectcap%
\pgfsetroundjoin%
\pgfsetlinewidth{1.003750pt}%
\definecolor{currentstroke}{rgb}{0.000000,0.000000,0.000000}%
\pgfsetstrokecolor{currentstroke}%
\pgfsetdash{}{0pt}%
\pgfpathmoveto{\pgfqpoint{7.457482in}{1.739313in}}%
\pgfpathlineto{\pgfqpoint{7.495884in}{1.739313in}}%
\pgfusepath{stroke}%
\end{pgfscope}%
\begin{pgfscope}%
\pgfpathrectangle{\pgfqpoint{4.577333in}{0.150000in}}{\pgfqpoint{4.224218in}{2.565000in}}%
\pgfusepath{clip}%
\pgfsetrectcap%
\pgfsetroundjoin%
\pgfsetlinewidth{1.003750pt}%
\definecolor{currentstroke}{rgb}{0.000000,0.000000,0.000000}%
\pgfsetstrokecolor{currentstroke}%
\pgfsetdash{}{0pt}%
\pgfpathmoveto{\pgfqpoint{7.265472in}{1.961449in}}%
\pgfpathlineto{\pgfqpoint{7.303874in}{1.961449in}}%
\pgfusepath{stroke}%
\end{pgfscope}%
\begin{pgfscope}%
\pgfpathrectangle{\pgfqpoint{4.577333in}{0.150000in}}{\pgfqpoint{4.224218in}{2.565000in}}%
\pgfusepath{clip}%
\pgfsetrectcap%
\pgfsetroundjoin%
\pgfsetlinewidth{1.003750pt}%
\definecolor{currentstroke}{rgb}{0.000000,0.000000,0.000000}%
\pgfsetstrokecolor{currentstroke}%
\pgfsetdash{}{0pt}%
\pgfpathmoveto{\pgfqpoint{7.073462in}{2.183584in}}%
\pgfpathlineto{\pgfqpoint{7.111864in}{2.183584in}}%
\pgfusepath{stroke}%
\end{pgfscope}%
\begin{pgfscope}%
\pgfpathrectangle{\pgfqpoint{4.577333in}{0.150000in}}{\pgfqpoint{4.224218in}{2.565000in}}%
\pgfusepath{clip}%
\pgfsetrectcap%
\pgfsetroundjoin%
\pgfsetlinewidth{1.003750pt}%
\definecolor{currentstroke}{rgb}{0.000000,0.000000,0.000000}%
\pgfsetstrokecolor{currentstroke}%
\pgfsetdash{}{0pt}%
\pgfpathmoveto{\pgfqpoint{6.881452in}{2.405720in}}%
\pgfpathlineto{\pgfqpoint{6.919854in}{2.405720in}}%
\pgfusepath{stroke}%
\end{pgfscope}%
\begin{pgfscope}%
\pgfpathrectangle{\pgfqpoint{4.577333in}{0.150000in}}{\pgfqpoint{4.224218in}{2.565000in}}%
\pgfusepath{clip}%
\pgfsetrectcap%
\pgfsetroundjoin%
\pgfsetlinewidth{1.003750pt}%
\definecolor{currentstroke}{rgb}{0.000000,0.000000,0.000000}%
\pgfsetstrokecolor{currentstroke}%
\pgfsetdash{}{0pt}%
\pgfpathmoveto{\pgfqpoint{6.689442in}{2.627855in}}%
\pgfpathlineto{\pgfqpoint{6.727844in}{2.627855in}}%
\pgfusepath{stroke}%
\end{pgfscope}%
\begin{pgfscope}%
\pgfpathrectangle{\pgfqpoint{4.577333in}{0.150000in}}{\pgfqpoint{4.224218in}{2.565000in}}%
\pgfusepath{clip}%
\pgfsetrectcap%
\pgfsetroundjoin%
\pgfsetlinewidth{1.003750pt}%
\definecolor{currentstroke}{rgb}{0.000000,0.000000,0.000000}%
\pgfsetstrokecolor{currentstroke}%
\pgfsetdash{}{0pt}%
\pgfpathmoveto{\pgfqpoint{4.769343in}{0.406500in}}%
\pgfpathlineto{\pgfqpoint{4.750142in}{0.428714in}}%
\pgfusepath{stroke}%
\end{pgfscope}%
\begin{pgfscope}%
\pgfpathrectangle{\pgfqpoint{4.577333in}{0.150000in}}{\pgfqpoint{4.224218in}{2.565000in}}%
\pgfusepath{clip}%
\pgfsetrectcap%
\pgfsetroundjoin%
\pgfsetlinewidth{1.003750pt}%
\definecolor{currentstroke}{rgb}{0.000000,0.000000,0.000000}%
\pgfsetstrokecolor{currentstroke}%
\pgfsetdash{}{0pt}%
\pgfpathmoveto{\pgfqpoint{4.961353in}{0.628636in}}%
\pgfpathlineto{\pgfqpoint{4.942152in}{0.650849in}}%
\pgfusepath{stroke}%
\end{pgfscope}%
\begin{pgfscope}%
\pgfpathrectangle{\pgfqpoint{4.577333in}{0.150000in}}{\pgfqpoint{4.224218in}{2.565000in}}%
\pgfusepath{clip}%
\pgfsetrectcap%
\pgfsetroundjoin%
\pgfsetlinewidth{1.003750pt}%
\definecolor{currentstroke}{rgb}{0.000000,0.000000,0.000000}%
\pgfsetstrokecolor{currentstroke}%
\pgfsetdash{}{0pt}%
\pgfpathmoveto{\pgfqpoint{5.153363in}{0.850771in}}%
\pgfpathlineto{\pgfqpoint{5.134162in}{0.872985in}}%
\pgfusepath{stroke}%
\end{pgfscope}%
\begin{pgfscope}%
\pgfpathrectangle{\pgfqpoint{4.577333in}{0.150000in}}{\pgfqpoint{4.224218in}{2.565000in}}%
\pgfusepath{clip}%
\pgfsetrectcap%
\pgfsetroundjoin%
\pgfsetlinewidth{1.003750pt}%
\definecolor{currentstroke}{rgb}{0.000000,0.000000,0.000000}%
\pgfsetstrokecolor{currentstroke}%
\pgfsetdash{}{0pt}%
\pgfpathmoveto{\pgfqpoint{5.345373in}{1.072907in}}%
\pgfpathlineto{\pgfqpoint{5.326172in}{1.095120in}}%
\pgfusepath{stroke}%
\end{pgfscope}%
\begin{pgfscope}%
\pgfpathrectangle{\pgfqpoint{4.577333in}{0.150000in}}{\pgfqpoint{4.224218in}{2.565000in}}%
\pgfusepath{clip}%
\pgfsetrectcap%
\pgfsetroundjoin%
\pgfsetlinewidth{1.003750pt}%
\definecolor{currentstroke}{rgb}{0.000000,0.000000,0.000000}%
\pgfsetstrokecolor{currentstroke}%
\pgfsetdash{}{0pt}%
\pgfpathmoveto{\pgfqpoint{5.537383in}{1.295042in}}%
\pgfpathlineto{\pgfqpoint{5.518182in}{1.317256in}}%
\pgfusepath{stroke}%
\end{pgfscope}%
\begin{pgfscope}%
\pgfpathrectangle{\pgfqpoint{4.577333in}{0.150000in}}{\pgfqpoint{4.224218in}{2.565000in}}%
\pgfusepath{clip}%
\pgfsetrectcap%
\pgfsetroundjoin%
\pgfsetlinewidth{1.003750pt}%
\definecolor{currentstroke}{rgb}{0.000000,0.000000,0.000000}%
\pgfsetstrokecolor{currentstroke}%
\pgfsetdash{}{0pt}%
\pgfpathmoveto{\pgfqpoint{5.729393in}{1.517178in}}%
\pgfpathlineto{\pgfqpoint{5.710192in}{1.539391in}}%
\pgfusepath{stroke}%
\end{pgfscope}%
\begin{pgfscope}%
\pgfpathrectangle{\pgfqpoint{4.577333in}{0.150000in}}{\pgfqpoint{4.224218in}{2.565000in}}%
\pgfusepath{clip}%
\pgfsetrectcap%
\pgfsetroundjoin%
\pgfsetlinewidth{1.003750pt}%
\definecolor{currentstroke}{rgb}{0.000000,0.000000,0.000000}%
\pgfsetstrokecolor{currentstroke}%
\pgfsetdash{}{0pt}%
\pgfpathmoveto{\pgfqpoint{5.921402in}{1.739313in}}%
\pgfpathlineto{\pgfqpoint{5.902201in}{1.761527in}}%
\pgfusepath{stroke}%
\end{pgfscope}%
\begin{pgfscope}%
\pgfpathrectangle{\pgfqpoint{4.577333in}{0.150000in}}{\pgfqpoint{4.224218in}{2.565000in}}%
\pgfusepath{clip}%
\pgfsetrectcap%
\pgfsetroundjoin%
\pgfsetlinewidth{1.003750pt}%
\definecolor{currentstroke}{rgb}{0.000000,0.000000,0.000000}%
\pgfsetstrokecolor{currentstroke}%
\pgfsetdash{}{0pt}%
\pgfpathmoveto{\pgfqpoint{6.113412in}{1.961449in}}%
\pgfpathlineto{\pgfqpoint{6.094211in}{1.983662in}}%
\pgfusepath{stroke}%
\end{pgfscope}%
\begin{pgfscope}%
\pgfpathrectangle{\pgfqpoint{4.577333in}{0.150000in}}{\pgfqpoint{4.224218in}{2.565000in}}%
\pgfusepath{clip}%
\pgfsetrectcap%
\pgfsetroundjoin%
\pgfsetlinewidth{1.003750pt}%
\definecolor{currentstroke}{rgb}{0.000000,0.000000,0.000000}%
\pgfsetstrokecolor{currentstroke}%
\pgfsetdash{}{0pt}%
\pgfpathmoveto{\pgfqpoint{6.305422in}{2.183584in}}%
\pgfpathlineto{\pgfqpoint{6.286221in}{2.205798in}}%
\pgfusepath{stroke}%
\end{pgfscope}%
\begin{pgfscope}%
\pgfpathrectangle{\pgfqpoint{4.577333in}{0.150000in}}{\pgfqpoint{4.224218in}{2.565000in}}%
\pgfusepath{clip}%
\pgfsetrectcap%
\pgfsetroundjoin%
\pgfsetlinewidth{1.003750pt}%
\definecolor{currentstroke}{rgb}{0.000000,0.000000,0.000000}%
\pgfsetstrokecolor{currentstroke}%
\pgfsetdash{}{0pt}%
\pgfpathmoveto{\pgfqpoint{6.497432in}{2.405720in}}%
\pgfpathlineto{\pgfqpoint{6.478231in}{2.427933in}}%
\pgfusepath{stroke}%
\end{pgfscope}%
\begin{pgfscope}%
\pgfpathrectangle{\pgfqpoint{4.577333in}{0.150000in}}{\pgfqpoint{4.224218in}{2.565000in}}%
\pgfusepath{clip}%
\pgfsetrectcap%
\pgfsetroundjoin%
\pgfsetlinewidth{1.003750pt}%
\definecolor{currentstroke}{rgb}{0.000000,0.000000,0.000000}%
\pgfsetstrokecolor{currentstroke}%
\pgfsetdash{}{0pt}%
\pgfpathmoveto{\pgfqpoint{6.689442in}{2.627855in}}%
\pgfpathlineto{\pgfqpoint{6.670241in}{2.650069in}}%
\pgfusepath{stroke}%
\end{pgfscope}%
\begin{pgfscope}%
\pgfpathrectangle{\pgfqpoint{4.577333in}{0.150000in}}{\pgfqpoint{4.224218in}{2.565000in}}%
\pgfusepath{clip}%
\pgfsetrectcap%
\pgfsetroundjoin%
\pgfsetlinewidth{1.003750pt}%
\definecolor{currentstroke}{rgb}{0.000000,0.000000,0.000000}%
\pgfsetstrokecolor{currentstroke}%
\pgfsetdash{}{0pt}%
\pgfpathmoveto{\pgfqpoint{4.769343in}{0.406500in}}%
\pgfpathlineto{\pgfqpoint{4.750142in}{0.384286in}}%
\pgfusepath{stroke}%
\end{pgfscope}%
\begin{pgfscope}%
\pgfpathrectangle{\pgfqpoint{4.577333in}{0.150000in}}{\pgfqpoint{4.224218in}{2.565000in}}%
\pgfusepath{clip}%
\pgfsetrectcap%
\pgfsetroundjoin%
\pgfsetlinewidth{1.003750pt}%
\definecolor{currentstroke}{rgb}{0.000000,0.000000,0.000000}%
\pgfsetstrokecolor{currentstroke}%
\pgfsetdash{}{0pt}%
\pgfpathmoveto{\pgfqpoint{5.153363in}{0.406500in}}%
\pgfpathlineto{\pgfqpoint{5.134162in}{0.384286in}}%
\pgfusepath{stroke}%
\end{pgfscope}%
\begin{pgfscope}%
\pgfpathrectangle{\pgfqpoint{4.577333in}{0.150000in}}{\pgfqpoint{4.224218in}{2.565000in}}%
\pgfusepath{clip}%
\pgfsetrectcap%
\pgfsetroundjoin%
\pgfsetlinewidth{1.003750pt}%
\definecolor{currentstroke}{rgb}{0.000000,0.000000,0.000000}%
\pgfsetstrokecolor{currentstroke}%
\pgfsetdash{}{0pt}%
\pgfpathmoveto{\pgfqpoint{5.537383in}{0.406500in}}%
\pgfpathlineto{\pgfqpoint{5.518182in}{0.384286in}}%
\pgfusepath{stroke}%
\end{pgfscope}%
\begin{pgfscope}%
\pgfpathrectangle{\pgfqpoint{4.577333in}{0.150000in}}{\pgfqpoint{4.224218in}{2.565000in}}%
\pgfusepath{clip}%
\pgfsetrectcap%
\pgfsetroundjoin%
\pgfsetlinewidth{1.003750pt}%
\definecolor{currentstroke}{rgb}{0.000000,0.000000,0.000000}%
\pgfsetstrokecolor{currentstroke}%
\pgfsetdash{}{0pt}%
\pgfpathmoveto{\pgfqpoint{5.921402in}{0.406500in}}%
\pgfpathlineto{\pgfqpoint{5.902201in}{0.384286in}}%
\pgfusepath{stroke}%
\end{pgfscope}%
\begin{pgfscope}%
\pgfpathrectangle{\pgfqpoint{4.577333in}{0.150000in}}{\pgfqpoint{4.224218in}{2.565000in}}%
\pgfusepath{clip}%
\pgfsetrectcap%
\pgfsetroundjoin%
\pgfsetlinewidth{1.003750pt}%
\definecolor{currentstroke}{rgb}{0.000000,0.000000,0.000000}%
\pgfsetstrokecolor{currentstroke}%
\pgfsetdash{}{0pt}%
\pgfpathmoveto{\pgfqpoint{6.305422in}{0.406500in}}%
\pgfpathlineto{\pgfqpoint{6.286221in}{0.384286in}}%
\pgfusepath{stroke}%
\end{pgfscope}%
\begin{pgfscope}%
\pgfpathrectangle{\pgfqpoint{4.577333in}{0.150000in}}{\pgfqpoint{4.224218in}{2.565000in}}%
\pgfusepath{clip}%
\pgfsetrectcap%
\pgfsetroundjoin%
\pgfsetlinewidth{1.003750pt}%
\definecolor{currentstroke}{rgb}{0.000000,0.000000,0.000000}%
\pgfsetstrokecolor{currentstroke}%
\pgfsetdash{}{0pt}%
\pgfpathmoveto{\pgfqpoint{6.689442in}{0.406500in}}%
\pgfpathlineto{\pgfqpoint{6.670241in}{0.384286in}}%
\pgfusepath{stroke}%
\end{pgfscope}%
\begin{pgfscope}%
\pgfpathrectangle{\pgfqpoint{4.577333in}{0.150000in}}{\pgfqpoint{4.224218in}{2.565000in}}%
\pgfusepath{clip}%
\pgfsetrectcap%
\pgfsetroundjoin%
\pgfsetlinewidth{1.003750pt}%
\definecolor{currentstroke}{rgb}{0.000000,0.000000,0.000000}%
\pgfsetstrokecolor{currentstroke}%
\pgfsetdash{}{0pt}%
\pgfpathmoveto{\pgfqpoint{7.073462in}{0.406500in}}%
\pgfpathlineto{\pgfqpoint{7.054261in}{0.384286in}}%
\pgfusepath{stroke}%
\end{pgfscope}%
\begin{pgfscope}%
\pgfpathrectangle{\pgfqpoint{4.577333in}{0.150000in}}{\pgfqpoint{4.224218in}{2.565000in}}%
\pgfusepath{clip}%
\pgfsetrectcap%
\pgfsetroundjoin%
\pgfsetlinewidth{1.003750pt}%
\definecolor{currentstroke}{rgb}{0.000000,0.000000,0.000000}%
\pgfsetstrokecolor{currentstroke}%
\pgfsetdash{}{0pt}%
\pgfpathmoveto{\pgfqpoint{7.457482in}{0.406500in}}%
\pgfpathlineto{\pgfqpoint{7.438281in}{0.384286in}}%
\pgfusepath{stroke}%
\end{pgfscope}%
\begin{pgfscope}%
\pgfpathrectangle{\pgfqpoint{4.577333in}{0.150000in}}{\pgfqpoint{4.224218in}{2.565000in}}%
\pgfusepath{clip}%
\pgfsetrectcap%
\pgfsetroundjoin%
\pgfsetlinewidth{1.003750pt}%
\definecolor{currentstroke}{rgb}{0.000000,0.000000,0.000000}%
\pgfsetstrokecolor{currentstroke}%
\pgfsetdash{}{0pt}%
\pgfpathmoveto{\pgfqpoint{7.841502in}{0.406500in}}%
\pgfpathlineto{\pgfqpoint{7.822301in}{0.384286in}}%
\pgfusepath{stroke}%
\end{pgfscope}%
\begin{pgfscope}%
\pgfpathrectangle{\pgfqpoint{4.577333in}{0.150000in}}{\pgfqpoint{4.224218in}{2.565000in}}%
\pgfusepath{clip}%
\pgfsetrectcap%
\pgfsetroundjoin%
\pgfsetlinewidth{1.003750pt}%
\definecolor{currentstroke}{rgb}{0.000000,0.000000,0.000000}%
\pgfsetstrokecolor{currentstroke}%
\pgfsetdash{}{0pt}%
\pgfpathmoveto{\pgfqpoint{8.225522in}{0.406500in}}%
\pgfpathlineto{\pgfqpoint{8.206321in}{0.384286in}}%
\pgfusepath{stroke}%
\end{pgfscope}%
\begin{pgfscope}%
\pgfpathrectangle{\pgfqpoint{4.577333in}{0.150000in}}{\pgfqpoint{4.224218in}{2.565000in}}%
\pgfusepath{clip}%
\pgfsetrectcap%
\pgfsetroundjoin%
\pgfsetlinewidth{1.003750pt}%
\definecolor{currentstroke}{rgb}{0.000000,0.000000,0.000000}%
\pgfsetstrokecolor{currentstroke}%
\pgfsetdash{}{0pt}%
\pgfpathmoveto{\pgfqpoint{8.609541in}{0.406500in}}%
\pgfpathlineto{\pgfqpoint{8.590340in}{0.384286in}}%
\pgfusepath{stroke}%
\end{pgfscope}%
\begin{pgfscope}%
\pgfpathrectangle{\pgfqpoint{4.577333in}{0.150000in}}{\pgfqpoint{4.224218in}{2.565000in}}%
\pgfusepath{clip}%
\pgfsetrectcap%
\pgfsetroundjoin%
\pgfsetlinewidth{1.003750pt}%
\definecolor{currentstroke}{rgb}{0.000000,0.000000,1.000000}%
\pgfsetstrokecolor{currentstroke}%
\pgfsetdash{}{0pt}%
\pgfpathmoveto{\pgfqpoint{6.864334in}{1.402646in}}%
\pgfpathlineto{\pgfqpoint{6.736403in}{1.668565in}}%
\pgfpathlineto{\pgfqpoint{6.504726in}{1.818620in}}%
\pgfpathlineto{\pgfqpoint{6.277153in}{1.811868in}}%
\pgfpathlineto{\pgfqpoint{6.063177in}{1.687650in}}%
\pgfpathlineto{\pgfqpoint{5.860195in}{1.499266in}}%
\pgfpathlineto{\pgfqpoint{5.684373in}{1.293021in}}%
\pgfpathlineto{\pgfqpoint{5.570383in}{1.106762in}}%
\pgfpathlineto{\pgfqpoint{5.572260in}{0.969316in}}%
\pgfpathlineto{\pgfqpoint{5.765377in}{0.904850in}}%
\pgfpathlineto{\pgfqpoint{6.183054in}{0.943790in}}%
\pgfpathlineto{\pgfqpoint{6.636004in}{1.120439in}}%
\pgfpathlineto{\pgfqpoint{6.811135in}{1.414905in}}%
\pgfpathlineto{\pgfqpoint{6.702823in}{1.701698in}}%
\pgfpathlineto{\pgfqpoint{6.499753in}{1.851208in}}%
\pgfpathlineto{\pgfqpoint{6.293707in}{1.841191in}}%
\pgfpathlineto{\pgfqpoint{6.091288in}{1.715210in}}%
\pgfpathlineto{\pgfqpoint{5.893090in}{1.523299in}}%
\pgfpathlineto{\pgfqpoint{5.718759in}{1.309414in}}%
\pgfpathlineto{\pgfqpoint{5.606752in}{1.110924in}}%
\pgfpathlineto{\pgfqpoint{5.613542in}{0.956730in}}%
\pgfpathlineto{\pgfqpoint{5.806168in}{0.870585in}}%
\pgfpathlineto{\pgfqpoint{6.191935in}{0.884612in}}%
\pgfpathlineto{\pgfqpoint{6.587710in}{1.049222in}}%
\pgfpathlineto{\pgfqpoint{6.731814in}{1.369813in}}%
\pgfpathlineto{\pgfqpoint{6.627406in}{1.696607in}}%
\pgfpathlineto{\pgfqpoint{6.449235in}{1.865220in}}%
\pgfpathlineto{\pgfqpoint{6.270297in}{1.861987in}}%
\pgfpathlineto{\pgfqpoint{6.088303in}{1.741149in}}%
\pgfpathlineto{\pgfqpoint{5.903539in}{1.553378in}}%
\pgfpathlineto{\pgfqpoint{5.737539in}{1.340495in}}%
\pgfpathlineto{\pgfqpoint{5.632309in}{1.138467in}}%
\pgfpathlineto{\pgfqpoint{5.650282in}{0.975800in}}%
\pgfpathlineto{\pgfqpoint{5.860504in}{0.875993in}}%
\pgfpathlineto{\pgfqpoint{6.248558in}{0.870629in}}%
\pgfpathlineto{\pgfqpoint{6.604856in}{1.012569in}}%
\pgfpathlineto{\pgfqpoint{6.700925in}{1.322707in}}%
\pgfpathlineto{\pgfqpoint{6.575339in}{1.660593in}}%
\pgfpathlineto{\pgfqpoint{6.399164in}{1.843262in}}%
\pgfpathlineto{\pgfqpoint{6.231264in}{1.849652in}}%
\pgfpathlineto{\pgfqpoint{6.061004in}{1.737695in}}%
\pgfpathlineto{\pgfqpoint{5.886049in}{1.559822in}}%
\pgfpathlineto{\pgfqpoint{5.726589in}{1.357307in}}%
\pgfpathlineto{\pgfqpoint{5.623533in}{1.164929in}}%
\pgfpathlineto{\pgfqpoint{5.641505in}{1.010110in}}%
\pgfpathlineto{\pgfqpoint{5.862250in}{0.914756in}}%
\pgfpathlineto{\pgfqpoint{6.286949in}{0.907092in}}%
\pgfpathlineto{\pgfqpoint{6.678255in}{1.034997in}}%
\pgfpathlineto{\pgfqpoint{6.765501in}{1.320432in}}%
\pgfpathlineto{\pgfqpoint{6.603124in}{1.639483in}}%
\pgfpathlineto{\pgfqpoint{6.393849in}{1.814851in}}%
\pgfpathlineto{\pgfqpoint{6.206901in}{1.817778in}}%
\pgfpathlineto{\pgfqpoint{6.028215in}{1.704886in}}%
\pgfpathlineto{\pgfqpoint{5.851429in}{1.530639in}}%
\pgfpathlineto{\pgfqpoint{5.693700in}{1.337534in}}%
\pgfpathlineto{\pgfqpoint{5.591973in}{1.160290in}}%
\pgfpathlineto{\pgfqpoint{5.606500in}{1.025820in}}%
\pgfpathlineto{\pgfqpoint{5.823419in}{0.954937in}}%
\pgfpathlineto{\pgfqpoint{6.270888in}{0.969751in}}%
\pgfpathlineto{\pgfqpoint{6.719799in}{1.100916in}}%
\pgfpathlineto{\pgfqpoint{6.850384in}{1.358580in}}%
\pgfpathlineto{\pgfqpoint{6.686665in}{1.645213in}}%
\pgfpathlineto{\pgfqpoint{6.446344in}{1.805686in}}%
\pgfpathlineto{\pgfqpoint{6.227248in}{1.800375in}}%
\pgfpathlineto{\pgfqpoint{6.025530in}{1.678959in}}%
\pgfpathlineto{\pgfqpoint{5.834273in}{1.497594in}}%
\pgfpathlineto{\pgfqpoint{5.668786in}{1.301592in}}%
\pgfpathlineto{\pgfqpoint{5.563566in}{1.127257in}}%
\pgfpathlineto{\pgfqpoint{5.574688in}{1.002474in}}%
\pgfpathlineto{\pgfqpoint{5.784720in}{0.950087in}}%
\pgfpathlineto{\pgfqpoint{6.230264in}{0.993626in}}%
\pgfpathlineto{\pgfqpoint{6.698685in}{1.151681in}}%
\pgfpathlineto{\pgfqpoint{6.864592in}{1.408532in}}%
\pgfpathlineto{\pgfqpoint{6.731325in}{1.673106in}}%
\pgfpathlineto{\pgfqpoint{6.498600in}{1.819678in}}%
\pgfpathlineto{\pgfqpoint{6.271235in}{1.809638in}}%
\pgfpathlineto{\pgfqpoint{6.057488in}{1.683371in}}%
\pgfpathlineto{\pgfqpoint{5.854956in}{1.494183in}}%
\pgfpathlineto{\pgfqpoint{5.680227in}{1.288208in}}%
\pgfpathlineto{\pgfqpoint{5.568335in}{1.103071in}}%
\pgfpathlineto{\pgfqpoint{5.573849in}{0.967403in}}%
\pgfpathlineto{\pgfqpoint{5.772554in}{0.905380in}}%
\pgfpathlineto{\pgfqpoint{6.195085in}{0.947590in}}%
\pgfpathlineto{\pgfqpoint{6.645359in}{1.127542in}}%
\pgfpathlineto{\pgfqpoint{6.812601in}{1.422769in}}%
\pgfpathlineto{\pgfqpoint{6.699715in}{1.706723in}}%
\pgfpathlineto{\pgfqpoint{6.495470in}{1.852092in}}%
\pgfpathlineto{\pgfqpoint{6.289014in}{1.838731in}}%
\pgfpathlineto{\pgfqpoint{6.086248in}{1.710627in}}%
\pgfpathlineto{\pgfqpoint{5.888083in}{1.517724in}}%
\pgfpathlineto{\pgfqpoint{5.714577in}{1.303890in}}%
\pgfpathlineto{\pgfqpoint{5.604514in}{1.106333in}}%
\pgfpathlineto{\pgfqpoint{5.614706in}{0.953758in}}%
\pgfpathlineto{\pgfqpoint{5.812124in}{0.869928in}}%
\pgfpathlineto{\pgfqpoint{6.201584in}{0.887382in}}%
\pgfpathlineto{\pgfqpoint{6.595238in}{1.056512in}}%
\pgfpathlineto{\pgfqpoint{6.732961in}{1.379449in}}%
\pgfpathlineto{\pgfqpoint{6.624980in}{1.703283in}}%
\pgfpathlineto{\pgfqpoint{6.446103in}{1.867047in}}%
\pgfpathlineto{\pgfqpoint{6.266756in}{1.860144in}}%
\pgfpathlineto{\pgfqpoint{6.084253in}{1.737009in}}%
\pgfpathlineto{\pgfqpoint{5.899344in}{1.548021in}}%
\pgfpathlineto{\pgfqpoint{5.734061in}{1.334881in}}%
\pgfpathlineto{\pgfqpoint{5.630852in}{1.133431in}}%
\pgfpathlineto{\pgfqpoint{5.652494in}{0.972011in}}%
\pgfpathlineto{\pgfqpoint{5.867340in}{0.874128in}}%
\pgfpathlineto{\pgfqpoint{6.257294in}{0.871810in}}%
\pgfpathlineto{\pgfqpoint{6.609493in}{1.018276in}}%
\pgfpathlineto{\pgfqpoint{6.699484in}{1.331825in}}%
\pgfpathlineto{\pgfqpoint{6.571628in}{1.667804in}}%
\pgfpathlineto{\pgfqpoint{6.395800in}{1.845881in}}%
\pgfpathlineto{\pgfqpoint{6.228071in}{1.848673in}}%
\pgfpathlineto{\pgfqpoint{6.057574in}{1.734496in}}%
\pgfpathlineto{\pgfqpoint{5.882560in}{1.555414in}}%
\pgfpathlineto{\pgfqpoint{5.723781in}{1.352517in}}%
\pgfpathlineto{\pgfqpoint{5.622710in}{1.160445in}}%
\pgfpathlineto{\pgfqpoint{5.644622in}{1.006461in}}%
\pgfpathlineto{\pgfqpoint{5.870844in}{0.912481in}}%
\pgfpathlineto{\pgfqpoint{6.297737in}{0.907212in}}%
\pgfpathlineto{\pgfqpoint{6.682905in}{1.039088in}}%
\pgfpathlineto{\pgfqpoint{6.761824in}{1.328001in}}%
\pgfpathlineto{\pgfqpoint{6.596794in}{1.645865in}}%
\pgfpathlineto{\pgfqpoint{6.388668in}{1.817145in}}%
\pgfpathlineto{\pgfqpoint{6.202762in}{1.816786in}}%
\pgfpathlineto{\pgfqpoint{6.024465in}{1.702014in}}%
\pgfpathlineto{\pgfqpoint{5.848029in}{1.526916in}}%
\pgfpathlineto{\pgfqpoint{5.691173in}{1.333731in}}%
\pgfpathlineto{\pgfqpoint{5.591379in}{1.156985in}}%
\pgfpathlineto{\pgfqpoint{5.609670in}{1.023402in}}%
\pgfpathlineto{\pgfqpoint{5.832436in}{0.953709in}}%
\pgfpathlineto{\pgfqpoint{6.283631in}{0.970256in}}%
\pgfpathlineto{\pgfqpoint{6.727116in}{1.104234in}}%
\pgfpathlineto{\pgfqpoint{6.847939in}{1.364703in}}%
\pgfpathlineto{\pgfqpoint{6.679594in}{1.650603in}}%
\pgfpathlineto{\pgfqpoint{6.439570in}{1.807335in}}%
\pgfpathlineto{\pgfqpoint{6.221570in}{1.798696in}}%
\pgfpathlineto{\pgfqpoint{6.020525in}{1.675406in}}%
\pgfpathlineto{\pgfqpoint{5.829931in}{1.493389in}}%
\pgfpathlineto{\pgfqpoint{5.665613in}{1.297674in}}%
\pgfpathlineto{\pgfqpoint{5.562524in}{1.124325in}}%
\pgfpathlineto{\pgfqpoint{5.577427in}{1.001024in}}%
\pgfpathlineto{\pgfqpoint{5.793337in}{0.950522in}}%
\pgfpathlineto{\pgfqpoint{6.243384in}{0.996249in}}%
\pgfpathlineto{\pgfqpoint{6.707718in}{1.156406in}}%
\pgfpathlineto{\pgfqpoint{6.864621in}{1.414404in}}%
\pgfpathlineto{\pgfqpoint{6.726135in}{1.677592in}}%
\pgfpathlineto{\pgfqpoint{6.492440in}{1.820663in}}%
\pgfpathlineto{\pgfqpoint{6.265307in}{1.807356in}}%
\pgfpathlineto{\pgfqpoint{6.051803in}{1.679071in}}%
\pgfpathlineto{\pgfqpoint{5.849742in}{1.489108in}}%
\pgfpathlineto{\pgfqpoint{5.676130in}{1.283428in}}%
\pgfpathlineto{\pgfqpoint{5.566368in}{1.099432in}}%
\pgfpathlineto{\pgfqpoint{5.575570in}{0.965557in}}%
\pgfpathlineto{\pgfqpoint{5.779913in}{0.905992in}}%
\pgfpathlineto{\pgfqpoint{6.207205in}{0.951474in}}%
\pgfpathlineto{\pgfqpoint{6.654562in}{1.134669in}}%
\pgfpathlineto{\pgfqpoint{6.813888in}{1.430554in}}%
\pgfpathlineto{\pgfqpoint{6.696519in}{1.711628in}}%
\pgfpathlineto{\pgfqpoint{6.491136in}{1.852876in}}%
\pgfpathlineto{\pgfqpoint{6.284279in}{1.836203in}}%
\pgfpathlineto{\pgfqpoint{6.081180in}{1.706006in}}%
\pgfpathlineto{\pgfqpoint{5.883068in}{1.512142in}}%
\pgfpathlineto{\pgfqpoint{5.710415in}{1.298387in}}%
\pgfpathlineto{\pgfqpoint{5.602329in}{1.101787in}}%
\pgfpathlineto{\pgfqpoint{5.615965in}{0.950853in}}%
\pgfpathlineto{\pgfqpoint{5.818210in}{0.869365in}}%
\pgfpathlineto{\pgfqpoint{6.211319in}{0.890283in}}%
\pgfpathlineto{\pgfqpoint{6.602698in}{1.063927in}}%
\pgfpathlineto{\pgfqpoint{6.734002in}{1.389066in}}%
\pgfpathlineto{\pgfqpoint{6.622518in}{1.709820in}}%
\pgfpathlineto{\pgfqpoint{6.442948in}{1.868747in}}%
\pgfpathlineto{\pgfqpoint{6.263184in}{1.858212in}}%
\pgfpathlineto{\pgfqpoint{6.080173in}{1.732809in}}%
\pgfpathlineto{\pgfqpoint{5.895134in}{1.542632in}}%
\pgfpathlineto{\pgfqpoint{5.730597in}{1.329262in}}%
\pgfpathlineto{\pgfqpoint{5.629442in}{1.128415in}}%
\pgfpathlineto{\pgfqpoint{5.654786in}{0.968266in}}%
\pgfpathlineto{\pgfqpoint{5.874240in}{0.872340in}}%
\pgfpathlineto{\pgfqpoint{6.265996in}{0.873113in}}%
\pgfpathlineto{\pgfqpoint{6.614027in}{1.024135in}}%
\pgfpathlineto{\pgfqpoint{6.698000in}{1.340995in}}%
\pgfpathlineto{\pgfqpoint{6.567955in}{1.674926in}}%
\pgfpathlineto{\pgfqpoint{6.392467in}{1.848393in}}%
\pgfpathlineto{\pgfqpoint{6.224877in}{1.847614in}}%
\pgfpathlineto{\pgfqpoint{6.054129in}{1.731238in}}%
\pgfpathlineto{\pgfqpoint{5.879062in}{1.550963in}}%
\pgfpathlineto{\pgfqpoint{5.720988in}{1.347701in}}%
\pgfpathlineto{\pgfqpoint{5.621945in}{1.155952in}}%
\pgfpathlineto{\pgfqpoint{5.647851in}{1.002822in}}%
\pgfpathlineto{\pgfqpoint{5.879536in}{0.910244in}}%
\pgfpathlineto{\pgfqpoint{6.308431in}{0.907423in}}%
\pgfpathlineto{\pgfqpoint{6.687309in}{1.043318in}}%
\pgfpathlineto{\pgfqpoint{6.758023in}{1.335643in}}%
\pgfpathlineto{\pgfqpoint{6.590513in}{1.652189in}}%
\pgfpathlineto{\pgfqpoint{6.383563in}{1.819359in}}%
\pgfpathlineto{\pgfqpoint{6.198668in}{1.815744in}}%
\pgfpathlineto{\pgfqpoint{6.020739in}{1.699114in}}%
\pgfpathlineto{\pgfqpoint{5.844649in}{1.523177in}}%
\pgfpathlineto{\pgfqpoint{5.688676in}{1.329918in}}%
\pgfpathlineto{\pgfqpoint{5.590847in}{1.153673in}}%
\pgfpathlineto{\pgfqpoint{5.612962in}{1.020977in}}%
\pgfpathlineto{\pgfqpoint{5.841603in}{0.952478in}}%
\pgfpathlineto{\pgfqpoint{6.296349in}{0.970786in}}%
\pgfpathlineto{\pgfqpoint{6.734157in}{1.107633in}}%
\pgfpathlineto{\pgfqpoint{6.845254in}{1.370890in}}%
\pgfpathlineto{\pgfqpoint{6.672473in}{1.655949in}}%
\pgfpathlineto{\pgfqpoint{6.432834in}{1.808906in}}%
\pgfpathlineto{\pgfqpoint{6.215932in}{1.796969in}}%
\pgfpathlineto{\pgfqpoint{6.015555in}{1.671839in}}%
\pgfpathlineto{\pgfqpoint{5.825630in}{1.489195in}}%
\pgfpathlineto{\pgfqpoint{5.662496in}{1.293783in}}%
\pgfpathlineto{\pgfqpoint{5.561566in}{1.121425in}}%
\pgfpathlineto{\pgfqpoint{5.580300in}{0.999606in}}%
\pgfpathlineto{\pgfqpoint{5.802126in}{0.950980in}}%
\pgfpathlineto{\pgfqpoint{6.256537in}{0.998880in}}%
\pgfpathlineto{\pgfqpoint{6.716537in}{1.161128in}}%
\pgfpathlineto{\pgfqpoint{6.864421in}{1.420266in}}%
\pgfpathlineto{\pgfqpoint{6.720833in}{1.682026in}}%
\pgfpathlineto{\pgfqpoint{6.486245in}{1.821575in}}%
\pgfpathlineto{\pgfqpoint{6.259370in}{1.805020in}}%
\pgfpathlineto{\pgfqpoint{6.046122in}{1.674749in}}%
\pgfpathlineto{\pgfqpoint{5.844551in}{1.484042in}}%
\pgfpathlineto{\pgfqpoint{5.672080in}{1.278682in}}%
\pgfpathlineto{\pgfqpoint{5.564484in}{1.095845in}}%
\pgfpathlineto{\pgfqpoint{5.577427in}{0.963775in}}%
\pgfpathlineto{\pgfqpoint{5.787459in}{0.906684in}}%
\pgfpathlineto{\pgfqpoint{6.219412in}{0.955439in}}%
\pgfpathlineto{\pgfqpoint{6.663610in}{1.141818in}}%
\pgfpathlineto{\pgfqpoint{6.814997in}{1.438263in}}%
\pgfpathlineto{\pgfqpoint{6.693235in}{1.716417in}}%
\pgfpathlineto{\pgfqpoint{6.486750in}{1.853562in}}%
\pgfpathlineto{\pgfqpoint{6.279502in}{1.833608in}}%
\pgfpathlineto{\pgfqpoint{6.076082in}{1.701348in}}%
\pgfpathlineto{\pgfqpoint{5.878047in}{1.506553in}}%
\pgfpathlineto{\pgfqpoint{5.706274in}{1.292905in}}%
\pgfpathlineto{\pgfqpoint{5.600199in}{1.097286in}}%
\pgfpathlineto{\pgfqpoint{5.617322in}{0.948015in}}%
\pgfpathlineto{\pgfqpoint{5.824432in}{0.868896in}}%
\pgfpathlineto{\pgfqpoint{6.221142in}{0.893317in}}%
\pgfpathlineto{\pgfqpoint{6.610089in}{1.071466in}}%
\pgfpathlineto{\pgfqpoint{6.734936in}{1.398657in}}%
\pgfpathlineto{\pgfqpoint{6.620017in}{1.716219in}}%
\pgfpathlineto{\pgfqpoint{6.439768in}{1.870320in}}%
\pgfpathlineto{\pgfqpoint{6.259577in}{1.856192in}}%
\pgfpathlineto{\pgfqpoint{6.076062in}{1.728550in}}%
\pgfpathlineto{\pgfqpoint{5.890910in}{1.537209in}}%
\pgfpathlineto{\pgfqpoint{5.727145in}{1.323637in}}%
\pgfpathlineto{\pgfqpoint{5.628081in}{1.123420in}}%
\pgfpathlineto{\pgfqpoint{5.657158in}{0.964567in}}%
\pgfpathlineto{\pgfqpoint{5.881205in}{0.870628in}}%
\pgfpathlineto{\pgfqpoint{6.274666in}{0.874540in}}%
\pgfpathlineto{\pgfqpoint{6.618463in}{1.030149in}}%
\pgfpathlineto{\pgfqpoint{6.696476in}{1.350218in}}%
\pgfpathlineto{\pgfqpoint{6.564319in}{1.681958in}}%
\pgfpathlineto{\pgfqpoint{6.389159in}{1.850798in}}%
\pgfpathlineto{\pgfqpoint{6.221677in}{1.846474in}}%
\pgfpathlineto{\pgfqpoint{6.050666in}{1.727919in}}%
\pgfpathlineto{\pgfqpoint{5.875554in}{1.546468in}}%
\pgfpathlineto{\pgfqpoint{5.718209in}{1.342858in}}%
\pgfpathlineto{\pgfqpoint{5.621237in}{1.151450in}}%
\pgfpathlineto{\pgfqpoint{5.651192in}{0.999192in}}%
\pgfpathlineto{\pgfqpoint{5.888325in}{0.908047in}}%
\pgfpathlineto{\pgfqpoint{6.319028in}{0.907728in}}%
\pgfpathlineto{\pgfqpoint{6.691470in}{1.047691in}}%
\pgfpathlineto{\pgfqpoint{6.754106in}{1.343357in}}%
\pgfpathlineto{\pgfqpoint{6.584285in}{1.658455in}}%
\pgfpathlineto{\pgfqpoint{6.378532in}{1.821494in}}%
\pgfpathlineto{\pgfqpoint{6.194614in}{1.814651in}}%
\pgfpathlineto{\pgfqpoint{6.017032in}{1.696184in}}%
\pgfpathlineto{\pgfqpoint{5.841285in}{1.519419in}}%
\pgfpathlineto{\pgfqpoint{5.686209in}{1.326092in}}%
\pgfpathlineto{\pgfqpoint{5.590378in}{1.150351in}}%
\pgfpathlineto{\pgfqpoint{5.616378in}{1.018543in}}%
\pgfpathlineto{\pgfqpoint{5.850921in}{0.951244in}}%
\pgfpathlineto{\pgfqpoint{6.309037in}{0.971345in}}%
\pgfpathlineto{\pgfqpoint{6.740921in}{1.111117in}}%
\pgfpathlineto{\pgfqpoint{6.842334in}{1.377144in}}%
\pgfpathlineto{\pgfqpoint{6.665305in}{1.661252in}}%
\pgfpathlineto{\pgfqpoint{6.426136in}{1.810398in}}%
\pgfpathlineto{\pgfqpoint{6.210335in}{1.795195in}}%
\pgfpathlineto{\pgfqpoint{6.010620in}{1.668258in}}%
\pgfpathlineto{\pgfqpoint{5.821369in}{1.485010in}}%
\pgfpathlineto{\pgfqpoint{5.659433in}{1.289915in}}%
\pgfpathlineto{\pgfqpoint{5.560690in}{1.118555in}}%
\pgfpathlineto{\pgfqpoint{5.583308in}{0.998216in}}%
\pgfpathlineto{\pgfqpoint{5.811087in}{0.951460in}}%
\pgfpathlineto{\pgfqpoint{6.269720in}{1.001519in}}%
\pgfpathlineto{\pgfqpoint{6.725142in}{1.165852in}}%
\pgfpathlineto{\pgfqpoint{6.863994in}{1.426124in}}%
\pgfpathlineto{\pgfqpoint{6.715422in}{1.686410in}}%
\pgfpathlineto{\pgfqpoint{6.480016in}{1.822415in}}%
\pgfpathlineto{\pgfqpoint{6.253424in}{1.802632in}}%
\pgfpathlineto{\pgfqpoint{6.040447in}{1.670404in}}%
\pgfpathlineto{\pgfqpoint{5.839384in}{1.478983in}}%
\pgfpathlineto{\pgfqpoint{5.668080in}{1.273968in}}%
\pgfpathlineto{\pgfqpoint{5.562684in}{1.092308in}}%
\pgfpathlineto{\pgfqpoint{5.579424in}{0.962058in}}%
\pgfpathlineto{\pgfqpoint{5.795195in}{0.907454in}}%
\pgfpathlineto{\pgfqpoint{6.231702in}{0.959485in}}%
\pgfpathlineto{\pgfqpoint{6.672500in}{1.148987in}}%
\pgfpathlineto{\pgfqpoint{6.815927in}{1.445896in}}%
\pgfpathlineto{\pgfqpoint{6.689860in}{1.721093in}}%
\pgfpathlineto{\pgfqpoint{6.482311in}{1.854154in}}%
\pgfpathlineto{\pgfqpoint{6.274681in}{1.830948in}}%
\pgfpathlineto{\pgfqpoint{6.070954in}{1.696653in}}%
\pgfpathlineto{\pgfqpoint{5.873017in}{1.500954in}}%
\pgfpathlineto{\pgfqpoint{5.702154in}{1.287444in}}%
\pgfpathlineto{\pgfqpoint{5.598124in}{1.092831in}}%
\pgfpathlineto{\pgfqpoint{5.618780in}{0.945243in}}%
\pgfpathlineto{\pgfqpoint{5.830796in}{0.868522in}}%
\pgfpathlineto{\pgfqpoint{6.231056in}{0.896485in}}%
\pgfpathlineto{\pgfqpoint{6.617408in}{1.079125in}}%
\pgfpathlineto{\pgfqpoint{6.735759in}{1.408221in}}%
\pgfpathlineto{\pgfqpoint{6.617476in}{1.722481in}}%
\pgfpathlineto{\pgfqpoint{6.436560in}{1.871770in}}%
\pgfpathlineto{\pgfqpoint{6.255935in}{1.854083in}}%
\pgfpathlineto{\pgfqpoint{6.071917in}{1.724232in}}%
\pgfpathlineto{\pgfqpoint{5.886670in}{1.531754in}}%
\pgfpathlineto{\pgfqpoint{5.723707in}{1.318007in}}%
\pgfpathlineto{\pgfqpoint{5.626770in}{1.118446in}}%
\pgfpathlineto{\pgfqpoint{5.659614in}{0.960914in}}%
\pgfpathlineto{\pgfqpoint{5.888240in}{0.868994in}}%
\pgfpathlineto{\pgfqpoint{6.283310in}{0.876093in}}%
\pgfpathlineto{\pgfqpoint{6.622804in}{1.036318in}}%
\pgfpathlineto{\pgfqpoint{6.694911in}{1.359488in}}%
\pgfpathlineto{\pgfqpoint{6.560716in}{1.688897in}}%
\pgfpathlineto{\pgfqpoint{6.385873in}{1.853095in}}%
\pgfpathlineto{\pgfqpoint{6.218469in}{1.845253in}}%
\pgfpathlineto{\pgfqpoint{6.047183in}{1.724538in}}%
\pgfpathlineto{\pgfqpoint{5.872034in}{1.541927in}}%
\pgfpathlineto{\pgfqpoint{5.715444in}{1.337986in}}%
\pgfpathlineto{\pgfqpoint{5.620589in}{1.146938in}}%
\pgfpathlineto{\pgfqpoint{5.654646in}{0.995573in}}%
\pgfpathlineto{\pgfqpoint{5.897209in}{0.905893in}}%
\pgfpathlineto{\pgfqpoint{6.329526in}{0.908131in}}%
\pgfpathlineto{\pgfqpoint{6.695395in}{1.052208in}}%
\pgfpathlineto{\pgfqpoint{6.750082in}{1.351144in}}%
\pgfpathlineto{\pgfqpoint{6.578111in}{1.664661in}}%
\pgfpathlineto{\pgfqpoint{6.373573in}{1.823550in}}%
\pgfpathlineto{\pgfqpoint{6.190598in}{1.813506in}}%
\pgfpathlineto{\pgfqpoint{6.013342in}{1.693221in}}%
\pgfpathlineto{\pgfqpoint{5.837934in}{1.515639in}}%
\pgfpathlineto{\pgfqpoint{5.683770in}{1.322251in}}%
\pgfpathlineto{\pgfqpoint{5.589973in}{1.147018in}}%
\pgfpathlineto{\pgfqpoint{5.619919in}{1.016100in}}%
\pgfpathlineto{\pgfqpoint{5.860391in}{0.950010in}}%
\pgfpathlineto{\pgfqpoint{6.321690in}{0.971936in}}%
\pgfpathlineto{\pgfqpoint{6.747407in}{1.114691in}}%
\pgfpathlineto{\pgfqpoint{6.839185in}{1.383465in}}%
\pgfpathlineto{\pgfqpoint{6.658097in}{1.666511in}}%
\pgfpathlineto{\pgfqpoint{6.419478in}{1.811814in}}%
\pgfpathlineto{\pgfqpoint{6.204777in}{1.793374in}}%
\pgfpathlineto{\pgfqpoint{6.005719in}{1.664662in}}%
\pgfpathlineto{\pgfqpoint{5.817146in}{1.480833in}}%
\pgfpathlineto{\pgfqpoint{5.656423in}{1.286069in}}%
\pgfpathlineto{\pgfqpoint{5.559897in}{1.115713in}}%
\pgfpathlineto{\pgfqpoint{5.586454in}{0.996853in}}%
\pgfpathlineto{\pgfqpoint{5.820224in}{0.951960in}}%
\pgfpathlineto{\pgfqpoint{6.282928in}{1.004165in}}%
\pgfpathlineto{\pgfqpoint{6.733528in}{1.170580in}}%
\pgfpathlineto{\pgfqpoint{6.863340in}{1.431980in}}%
\pgfpathlineto{\pgfqpoint{6.709902in}{1.690746in}}%
\pgfpathlineto{\pgfqpoint{6.473756in}{1.823183in}}%
\pgfpathlineto{\pgfqpoint{6.247469in}{1.800191in}}%
\pgfpathlineto{\pgfqpoint{6.034776in}{1.666036in}}%
\pgfpathlineto{\pgfqpoint{5.834241in}{1.473930in}}%
\pgfpathlineto{\pgfqpoint{5.664130in}{1.269285in}}%
\pgfpathlineto{\pgfqpoint{5.560968in}{1.088819in}}%
\pgfpathlineto{\pgfqpoint{5.581562in}{0.960404in}}%
\pgfpathlineto{\pgfqpoint{5.803123in}{0.908301in}}%
\pgfpathlineto{\pgfqpoint{6.244073in}{0.963609in}}%
\pgfpathlineto{\pgfqpoint{6.681227in}{1.156173in}}%
\pgfpathlineto{\pgfqpoint{6.816679in}{1.453454in}}%
\pgfpathlineto{\pgfqpoint{6.686393in}{1.725658in}}%
\pgfpathlineto{\pgfqpoint{6.477816in}{1.854651in}}%
\pgfpathlineto{\pgfqpoint{6.269816in}{1.828222in}}%
\pgfpathlineto{\pgfqpoint{6.065796in}{1.691920in}}%
\pgfpathlineto{\pgfqpoint{5.867980in}{1.495348in}}%
\pgfpathlineto{\pgfqpoint{5.698056in}{1.282002in}}%
\pgfpathlineto{\pgfqpoint{5.596107in}{1.088419in}}%
\pgfpathlineto{\pgfqpoint{5.620343in}{0.942538in}}%
\pgfpathlineto{\pgfqpoint{5.837307in}{0.868243in}}%
\pgfpathlineto{\pgfqpoint{6.241062in}{0.899787in}}%
\pgfpathlineto{\pgfqpoint{6.624650in}{1.086903in}}%
\pgfpathlineto{\pgfqpoint{6.736470in}{1.417752in}}%
\pgfpathlineto{\pgfqpoint{6.614891in}{1.728604in}}%
\pgfpathlineto{\pgfqpoint{6.433321in}{1.873096in}}%
\pgfpathlineto{\pgfqpoint{6.252253in}{1.851887in}}%
\pgfpathlineto{\pgfqpoint{6.067738in}{1.719853in}}%
\pgfpathlineto{\pgfqpoint{5.882414in}{1.526265in}}%
\pgfpathlineto{\pgfqpoint{5.720282in}{1.312372in}}%
\pgfpathlineto{\pgfqpoint{5.625509in}{1.113494in}}%
\pgfpathlineto{\pgfqpoint{5.662153in}{0.957309in}}%
\pgfpathlineto{\pgfqpoint{5.895346in}{0.867440in}}%
\pgfpathlineto{\pgfqpoint{6.291931in}{0.877774in}}%
\pgfpathlineto{\pgfqpoint{6.627054in}{1.042645in}}%
\pgfpathlineto{\pgfqpoint{6.693307in}{1.368804in}}%
\pgfpathlineto{\pgfqpoint{6.557143in}{1.695743in}}%
\pgfpathlineto{\pgfqpoint{6.382606in}{1.855284in}}%
\pgfpathlineto{\pgfqpoint{6.215250in}{1.843948in}}%
\pgfpathlineto{\pgfqpoint{6.043678in}{1.721094in}}%
\pgfpathlineto{\pgfqpoint{5.868500in}{1.537339in}}%
\pgfpathlineto{\pgfqpoint{5.712693in}{1.333086in}}%
\pgfpathlineto{\pgfqpoint{5.620000in}{1.142416in}}%
\pgfpathlineto{\pgfqpoint{5.658216in}{0.991966in}}%
\pgfpathlineto{\pgfqpoint{5.906189in}{0.903783in}}%
\pgfpathlineto{\pgfqpoint{6.339924in}{0.908634in}}%
\pgfpathlineto{\pgfqpoint{6.699087in}{1.056874in}}%
\pgfpathlineto{\pgfqpoint{6.745958in}{1.359000in}}%
\pgfpathlineto{\pgfqpoint{6.571993in}{1.670808in}}%
\pgfpathlineto{\pgfqpoint{6.368683in}{1.825528in}}%
\pgfpathlineto{\pgfqpoint{6.186618in}{1.812309in}}%
\pgfpathlineto{\pgfqpoint{6.009666in}{1.690224in}}%
\pgfpathlineto{\pgfqpoint{5.834596in}{1.511835in}}%
\pgfpathlineto{\pgfqpoint{5.681357in}{1.318393in}}%
\pgfpathlineto{\pgfqpoint{5.589632in}{1.143672in}}%
\pgfpathlineto{\pgfqpoint{5.623588in}{1.013648in}}%
\pgfpathlineto{\pgfqpoint{5.870013in}{0.948775in}}%
\pgfpathlineto{\pgfqpoint{6.334303in}{0.972565in}}%
\pgfpathlineto{\pgfqpoint{6.753612in}{1.118362in}}%
\pgfpathlineto{\pgfqpoint{6.835812in}{1.389855in}}%
\pgfpathlineto{\pgfqpoint{6.650853in}{1.671724in}}%
\pgfpathlineto{\pgfqpoint{6.412861in}{1.813151in}}%
\pgfpathlineto{\pgfqpoint{6.199257in}{1.791505in}}%
\pgfpathlineto{\pgfqpoint{6.000849in}{1.661051in}}%
\pgfpathlineto{\pgfqpoint{5.812960in}{1.476662in}}%
\pgfpathlineto{\pgfqpoint{5.653466in}{1.282243in}}%
\pgfpathlineto{\pgfqpoint{5.559188in}{1.112896in}}%
\pgfpathlineto{\pgfqpoint{5.589739in}{0.995516in}}%
\pgfpathlineto{\pgfqpoint{5.829536in}{0.952479in}}%
\pgfpathlineto{\pgfqpoint{6.296157in}{1.006820in}}%
\pgfpathlineto{\pgfqpoint{6.741693in}{1.175318in}}%
\pgfpathlineto{\pgfqpoint{6.862460in}{1.437840in}}%
\pgfpathlineto{\pgfqpoint{6.704275in}{1.695034in}}%
\pgfpathlineto{\pgfqpoint{6.467463in}{1.823881in}}%
\pgfpathlineto{\pgfqpoint{6.241506in}{1.797697in}}%
\pgfpathlineto{\pgfqpoint{6.029110in}{1.661646in}}%
\pgfpathlineto{\pgfqpoint{5.829121in}{1.468883in}}%
\pgfpathlineto{\pgfqpoint{5.660228in}{1.264632in}}%
\pgfpathlineto{\pgfqpoint{5.559339in}{1.085379in}}%
\pgfpathlineto{\pgfqpoint{5.583844in}{0.958811in}}%
\pgfpathlineto{\pgfqpoint{5.811246in}{0.909225in}}%
\pgfpathlineto{\pgfqpoint{6.256521in}{0.967809in}}%
\pgfpathlineto{\pgfqpoint{6.689788in}{1.163374in}}%
\pgfpathlineto{\pgfqpoint{6.817251in}{1.460939in}}%
\pgfpathlineto{\pgfqpoint{6.682834in}{1.730117in}}%
\pgfpathlineto{\pgfqpoint{6.473265in}{1.855058in}}%
\pgfpathlineto{\pgfqpoint{6.264906in}{1.825432in}}%
\pgfpathlineto{\pgfqpoint{6.060607in}{1.687150in}}%
\pgfpathlineto{\pgfqpoint{5.862936in}{1.489733in}}%
\pgfpathlineto{\pgfqpoint{5.693980in}{1.276581in}}%
\pgfpathlineto{\pgfqpoint{5.594150in}{1.084052in}}%
\pgfpathlineto{\pgfqpoint{5.622016in}{0.939899in}}%
\pgfpathlineto{\pgfqpoint{5.843970in}{0.868059in}}%
\pgfpathlineto{\pgfqpoint{6.251160in}{0.903225in}}%
\pgfpathlineto{\pgfqpoint{6.631811in}{1.094797in}}%
\pgfpathlineto{\pgfqpoint{6.737068in}{1.427248in}}%
\pgfpathlineto{\pgfqpoint{6.612260in}{1.734591in}}%
\pgfpathlineto{\pgfqpoint{6.430048in}{1.874300in}}%
\pgfpathlineto{\pgfqpoint{6.248532in}{1.849604in}}%
\pgfpathlineto{\pgfqpoint{6.063523in}{1.715415in}}%
\pgfpathlineto{\pgfqpoint{5.878141in}{1.520744in}}%
\pgfpathlineto{\pgfqpoint{5.716870in}{1.306731in}}%
\pgfpathlineto{\pgfqpoint{5.624299in}{1.108562in}}%
\pgfpathlineto{\pgfqpoint{5.664779in}{0.953750in}}%
\pgfpathlineto{\pgfqpoint{5.902526in}{0.865967in}}%
\pgfpathlineto{\pgfqpoint{6.300534in}{0.879587in}}%
\pgfpathlineto{\pgfqpoint{6.631215in}{1.049129in}}%
\pgfpathlineto{\pgfqpoint{6.691665in}{1.378163in}}%
\pgfpathlineto{\pgfqpoint{6.553599in}{1.702493in}}%
\pgfpathlineto{\pgfqpoint{6.379352in}{1.857365in}}%
\pgfpathlineto{\pgfqpoint{6.212016in}{1.842561in}}%
\pgfpathlineto{\pgfqpoint{6.040149in}{1.717584in}}%
\pgfpathlineto{\pgfqpoint{5.864952in}{1.532704in}}%
\pgfpathlineto{\pgfqpoint{5.709955in}{1.328156in}}%
\pgfpathlineto{\pgfqpoint{5.619473in}{1.137885in}}%
\pgfpathlineto{\pgfqpoint{5.661902in}{0.988372in}}%
\pgfpathlineto{\pgfqpoint{5.915262in}{0.901720in}}%
\pgfpathlineto{\pgfqpoint{6.350220in}{0.909243in}}%
\pgfpathlineto{\pgfqpoint{6.702553in}{1.061691in}}%
\pgfpathlineto{\pgfqpoint{6.741740in}{1.366926in}}%
\pgfpathlineto{\pgfqpoint{6.565933in}{1.676894in}}%
\pgfpathlineto{\pgfqpoint{6.363861in}{1.827426in}}%
\pgfpathlineto{\pgfqpoint{6.182668in}{1.811058in}}%
\pgfpathlineto{\pgfqpoint{6.006002in}{1.687190in}}%
\pgfpathlineto{\pgfqpoint{5.831268in}{1.508004in}}%
\pgfpathlineto{\pgfqpoint{5.678970in}{1.314516in}}%
\pgfpathlineto{\pgfqpoint{5.589355in}{1.140312in}}%
\pgfpathlineto{\pgfqpoint{5.627387in}{1.011185in}}%
\pgfpathlineto{\pgfqpoint{5.879787in}{0.947543in}}%
\pgfpathlineto{\pgfqpoint{6.346872in}{0.973236in}}%
\pgfpathlineto{\pgfqpoint{6.759537in}{1.122134in}}%
\pgfpathlineto{\pgfqpoint{6.832220in}{1.396313in}}%
\pgfpathlineto{\pgfqpoint{6.643577in}{1.676891in}}%
\pgfpathlineto{\pgfqpoint{6.406285in}{1.814412in}}%
\pgfpathlineto{\pgfqpoint{6.193776in}{1.789589in}}%
\pgfpathlineto{\pgfqpoint{5.996011in}{1.657422in}}%
\pgfpathlineto{\pgfqpoint{5.808809in}{1.472496in}}%
\pgfpathlineto{\pgfqpoint{5.650559in}{1.278435in}}%
\pgfpathlineto{\pgfqpoint{5.558561in}{1.110103in}}%
\pgfpathlineto{\pgfqpoint{5.558561in}{1.110103in}}%
\pgfusepath{stroke}%
\end{pgfscope}%
\begin{pgfscope}%
\pgfpathrectangle{\pgfqpoint{4.577333in}{0.150000in}}{\pgfqpoint{4.224218in}{2.565000in}}%
\pgfusepath{clip}%
\pgfsetbuttcap%
\pgfsetroundjoin%
\pgfsetlinewidth{1.003750pt}%
\definecolor{currentstroke}{rgb}{0.501961,0.501961,0.501961}%
\pgfsetstrokecolor{currentstroke}%
\pgfsetdash{{3.700000pt}{1.600000pt}}{0.000000pt}%
\pgfpathmoveto{\pgfqpoint{6.864334in}{1.402646in}}%
\pgfpathlineto{\pgfqpoint{6.736403in}{1.668565in}}%
\pgfpathlineto{\pgfqpoint{6.620564in}{1.743593in}}%
\pgfpathlineto{\pgfqpoint{6.506094in}{1.766351in}}%
\pgfpathlineto{\pgfqpoint{6.395365in}{1.746676in}}%
\pgfpathlineto{\pgfqpoint{6.288331in}{1.697194in}}%
\pgfpathlineto{\pgfqpoint{6.187671in}{1.629832in}}%
\pgfpathlineto{\pgfqpoint{6.099487in}{1.555107in}}%
\pgfpathlineto{\pgfqpoint{6.033584in}{1.481884in}}%
\pgfpathlineto{\pgfqpoint{6.003783in}{1.417769in}}%
\pgfpathlineto{\pgfqpoint{6.021710in}{1.370371in}}%
\pgfpathlineto{\pgfqpoint{6.077555in}{1.347650in}}%
\pgfpathlineto{\pgfqpoint{6.138687in}{1.353254in}}%
\pgfpathlineto{\pgfqpoint{6.182082in}{1.380058in}}%
\pgfpathlineto{\pgfqpoint{6.204773in}{1.413711in}}%
\pgfpathlineto{\pgfqpoint{6.210702in}{1.442210in}}%
\pgfpathlineto{\pgfqpoint{6.203238in}{1.459272in}}%
\pgfpathlineto{\pgfqpoint{6.184994in}{1.463039in}}%
\pgfpathlineto{\pgfqpoint{6.159092in}{1.454504in}}%
\pgfpathlineto{\pgfqpoint{6.130022in}{1.436421in}}%
\pgfpathlineto{\pgfqpoint{6.104198in}{1.412436in}}%
\pgfpathlineto{\pgfqpoint{6.090006in}{1.386634in}}%
\pgfpathlineto{\pgfqpoint{6.094639in}{1.363815in}}%
\pgfpathlineto{\pgfqpoint{6.116077in}{1.350137in}}%
\pgfpathlineto{\pgfqpoint{6.141733in}{1.350957in}}%
\pgfpathlineto{\pgfqpoint{6.161159in}{1.364783in}}%
\pgfpathlineto{\pgfqpoint{6.172239in}{1.384030in}}%
\pgfpathlineto{\pgfqpoint{6.175871in}{1.401732in}}%
\pgfpathlineto{\pgfqpoint{6.172744in}{1.413854in}}%
\pgfpathlineto{\pgfqpoint{6.163461in}{1.418665in}}%
\pgfpathlineto{\pgfqpoint{6.149263in}{1.416060in}}%
\pgfpathlineto{\pgfqpoint{6.132587in}{1.407105in}}%
\pgfpathlineto{\pgfqpoint{6.117515in}{1.393627in}}%
\pgfpathlineto{\pgfqpoint{6.109727in}{1.377941in}}%
\pgfpathlineto{\pgfqpoint{6.113810in}{1.363020in}}%
\pgfpathlineto{\pgfqpoint{6.127840in}{1.353007in}}%
\pgfpathlineto{\pgfqpoint{6.143759in}{1.352165in}}%
\pgfpathlineto{\pgfqpoint{6.155424in}{1.360501in}}%
\pgfpathlineto{\pgfqpoint{6.161838in}{1.373206in}}%
\pgfpathlineto{\pgfqpoint{6.163618in}{1.385422in}}%
\pgfpathlineto{\pgfqpoint{6.161053in}{1.394229in}}%
\pgfpathlineto{\pgfqpoint{6.154345in}{1.398268in}}%
\pgfpathlineto{\pgfqpoint{6.144161in}{1.397293in}}%
\pgfpathlineto{\pgfqpoint{6.132053in}{1.391889in}}%
\pgfpathlineto{\pgfqpoint{6.120904in}{1.383212in}}%
\pgfpathlineto{\pgfqpoint{6.115156in}{1.372802in}}%
\pgfpathlineto{\pgfqpoint{6.118891in}{1.362678in}}%
\pgfpathlineto{\pgfqpoint{6.130792in}{1.355706in}}%
\pgfpathlineto{\pgfqpoint{6.144015in}{1.354971in}}%
\pgfpathlineto{\pgfqpoint{6.153385in}{1.360777in}}%
\pgfpathlineto{\pgfqpoint{6.158194in}{1.369859in}}%
\pgfpathlineto{\pgfqpoint{6.159149in}{1.378641in}}%
\pgfpathlineto{\pgfqpoint{6.156631in}{1.384915in}}%
\pgfpathlineto{\pgfqpoint{6.150873in}{1.387665in}}%
\pgfpathlineto{\pgfqpoint{6.142407in}{1.386737in}}%
\pgfpathlineto{\pgfqpoint{6.132399in}{1.382619in}}%
\pgfpathlineto{\pgfqpoint{6.123008in}{1.376248in}}%
\pgfpathlineto{\pgfqpoint{6.117752in}{1.368857in}}%
\pgfpathlineto{\pgfqpoint{6.120392in}{1.361975in}}%
\pgfpathlineto{\pgfqpoint{6.130551in}{1.357551in}}%
\pgfpathlineto{\pgfqpoint{6.142549in}{1.357568in}}%
\pgfpathlineto{\pgfqpoint{6.151469in}{1.362283in}}%
\pgfpathlineto{\pgfqpoint{6.156225in}{1.369435in}}%
\pgfpathlineto{\pgfqpoint{6.157352in}{1.376275in}}%
\pgfpathlineto{\pgfqpoint{6.155292in}{1.381005in}}%
\pgfpathlineto{\pgfqpoint{6.150354in}{1.382798in}}%
\pgfpathlineto{\pgfqpoint{6.143057in}{1.381568in}}%
\pgfpathlineto{\pgfqpoint{6.134408in}{1.377772in}}%
\pgfpathlineto{\pgfqpoint{6.126177in}{1.372253in}}%
\pgfpathlineto{\pgfqpoint{6.121228in}{1.366135in}}%
\pgfpathlineto{\pgfqpoint{6.122786in}{1.360813in}}%
\pgfpathlineto{\pgfqpoint{6.130897in}{1.357868in}}%
\pgfpathlineto{\pgfqpoint{6.141087in}{1.358571in}}%
\pgfpathlineto{\pgfqpoint{6.149173in}{1.362880in}}%
\pgfpathlineto{\pgfqpoint{6.153895in}{1.369053in}}%
\pgfpathlineto{\pgfqpoint{6.155459in}{1.374928in}}%
\pgfpathlineto{\pgfqpoint{6.154170in}{1.378986in}}%
\pgfpathlineto{\pgfqpoint{6.150284in}{1.380482in}}%
\pgfpathlineto{\pgfqpoint{6.144258in}{1.379299in}}%
\pgfpathlineto{\pgfqpoint{6.136968in}{1.375803in}}%
\pgfpathlineto{\pgfqpoint{6.129929in}{1.370698in}}%
\pgfpathlineto{\pgfqpoint{6.125517in}{1.364953in}}%
\pgfpathlineto{\pgfqpoint{6.126365in}{1.359863in}}%
\pgfpathlineto{\pgfqpoint{6.132618in}{1.357064in}}%
\pgfpathlineto{\pgfqpoint{6.140713in}{1.357846in}}%
\pgfpathlineto{\pgfqpoint{6.147289in}{1.361951in}}%
\pgfpathlineto{\pgfqpoint{6.151338in}{1.367650in}}%
\pgfpathlineto{\pgfqpoint{6.152921in}{1.373065in}}%
\pgfpathlineto{\pgfqpoint{6.152163in}{1.376901in}}%
\pgfpathlineto{\pgfqpoint{6.149196in}{1.378483in}}%
\pgfpathlineto{\pgfqpoint{6.144367in}{1.377654in}}%
\pgfpathlineto{\pgfqpoint{6.138434in}{1.374673in}}%
\pgfpathlineto{\pgfqpoint{6.132741in}{1.370097in}}%
\pgfpathlineto{\pgfqpoint{6.129294in}{1.364719in}}%
\pgfpathlineto{\pgfqpoint{6.130063in}{1.359641in}}%
\pgfpathlineto{\pgfqpoint{6.134960in}{1.356450in}}%
\pgfpathlineto{\pgfqpoint{6.141189in}{1.356690in}}%
\pgfpathlineto{\pgfqpoint{6.146176in}{1.360263in}}%
\pgfpathlineto{\pgfqpoint{6.149237in}{1.365434in}}%
\pgfpathlineto{\pgfqpoint{6.150424in}{1.370431in}}%
\pgfpathlineto{\pgfqpoint{6.149762in}{1.374097in}}%
\pgfpathlineto{\pgfqpoint{6.147283in}{1.375819in}}%
\pgfpathlineto{\pgfqpoint{6.143232in}{1.375418in}}%
\pgfpathlineto{\pgfqpoint{6.138257in}{1.373068in}}%
\pgfpathlineto{\pgfqpoint{6.133586in}{1.369212in}}%
\pgfpathlineto{\pgfqpoint{6.131051in}{1.364497in}}%
\pgfpathlineto{\pgfqpoint{6.132242in}{1.359849in}}%
\pgfpathlineto{\pgfqpoint{6.136702in}{1.356657in}}%
\pgfpathlineto{\pgfqpoint{6.141913in}{1.356427in}}%
\pgfpathlineto{\pgfqpoint{6.145855in}{1.359284in}}%
\pgfpathlineto{\pgfqpoint{6.148127in}{1.363707in}}%
\pgfpathlineto{\pgfqpoint{6.148848in}{1.368076in}}%
\pgfpathlineto{\pgfqpoint{6.148033in}{1.371348in}}%
\pgfpathlineto{\pgfqpoint{6.145683in}{1.372977in}}%
\pgfpathlineto{\pgfqpoint{6.141982in}{1.372797in}}%
\pgfpathlineto{\pgfqpoint{6.137467in}{1.370951in}}%
\pgfpathlineto{\pgfqpoint{6.133218in}{1.367809in}}%
\pgfpathlineto{\pgfqpoint{6.130976in}{1.363917in}}%
\pgfpathlineto{\pgfqpoint{6.132389in}{1.360046in}}%
\pgfpathlineto{\pgfqpoint{6.137015in}{1.357349in}}%
\pgfpathlineto{\pgfqpoint{6.142222in}{1.357105in}}%
\pgfpathlineto{\pgfqpoint{6.145979in}{1.359491in}}%
\pgfpathlineto{\pgfqpoint{6.147968in}{1.363242in}}%
\pgfpathlineto{\pgfqpoint{6.148413in}{1.366930in}}%
\pgfpathlineto{\pgfqpoint{6.147414in}{1.369632in}}%
\pgfpathlineto{\pgfqpoint{6.145019in}{1.370890in}}%
\pgfpathlineto{\pgfqpoint{6.141417in}{1.370595in}}%
\pgfpathlineto{\pgfqpoint{6.137086in}{1.368914in}}%
\pgfpathlineto{\pgfqpoint{6.132965in}{1.366214in}}%
\pgfpathlineto{\pgfqpoint{6.130636in}{1.363016in}}%
\pgfpathlineto{\pgfqpoint{6.131813in}{1.359995in}}%
\pgfpathlineto{\pgfqpoint{6.136357in}{1.358043in}}%
\pgfpathlineto{\pgfqpoint{6.141748in}{1.358093in}}%
\pgfpathlineto{\pgfqpoint{6.145792in}{1.360293in}}%
\pgfpathlineto{\pgfqpoint{6.147984in}{1.363629in}}%
\pgfpathlineto{\pgfqpoint{6.148529in}{1.366852in}}%
\pgfpathlineto{\pgfqpoint{6.147588in}{1.369120in}}%
\pgfpathlineto{\pgfqpoint{6.145269in}{1.370027in}}%
\pgfpathlineto{\pgfqpoint{6.141793in}{1.369503in}}%
\pgfpathlineto{\pgfqpoint{6.137626in}{1.367739in}}%
\pgfpathlineto{\pgfqpoint{6.133625in}{1.365120in}}%
\pgfpathlineto{\pgfqpoint{6.131211in}{1.362179in}}%
\pgfpathlineto{\pgfqpoint{6.132001in}{1.359602in}}%
\pgfpathlineto{\pgfqpoint{6.136027in}{1.358181in}}%
\pgfpathlineto{\pgfqpoint{6.141087in}{1.358572in}}%
\pgfpathlineto{\pgfqpoint{6.145122in}{1.360772in}}%
\pgfpathlineto{\pgfqpoint{6.147501in}{1.363922in}}%
\pgfpathlineto{\pgfqpoint{6.148302in}{1.366939in}}%
\pgfpathlineto{\pgfqpoint{6.147650in}{1.369048in}}%
\pgfpathlineto{\pgfqpoint{6.145651in}{1.369853in}}%
\pgfpathlineto{\pgfqpoint{6.142520in}{1.369277in}}%
\pgfpathlineto{\pgfqpoint{6.138705in}{1.367490in}}%
\pgfpathlineto{\pgfqpoint{6.135000in}{1.364846in}}%
\pgfpathlineto{\pgfqpoint{6.132679in}{1.361847in}}%
\pgfpathlineto{\pgfqpoint{6.133163in}{1.359182in}}%
\pgfpathlineto{\pgfqpoint{6.136527in}{1.357734in}}%
\pgfpathlineto{\pgfqpoint{6.140869in}{1.358200in}}%
\pgfpathlineto{\pgfqpoint{6.144408in}{1.360451in}}%
\pgfpathlineto{\pgfqpoint{6.146603in}{1.363568in}}%
\pgfpathlineto{\pgfqpoint{6.147469in}{1.366541in}}%
\pgfpathlineto{\pgfqpoint{6.147054in}{1.368662in}}%
\pgfpathlineto{\pgfqpoint{6.145415in}{1.369553in}}%
\pgfpathlineto{\pgfqpoint{6.142729in}{1.369114in}}%
\pgfpathlineto{\pgfqpoint{6.139414in}{1.367474in}}%
\pgfpathlineto{\pgfqpoint{6.136222in}{1.364934in}}%
\pgfpathlineto{\pgfqpoint{6.134295in}{1.361930in}}%
\pgfpathlineto{\pgfqpoint{6.134759in}{1.359089in}}%
\pgfpathlineto{\pgfqpoint{6.137561in}{1.357322in}}%
\pgfpathlineto{\pgfqpoint{6.141111in}{1.357511in}}%
\pgfpathlineto{\pgfqpoint{6.143960in}{1.359595in}}%
\pgfpathlineto{\pgfqpoint{6.145719in}{1.362590in}}%
\pgfpathlineto{\pgfqpoint{6.146406in}{1.365489in}}%
\pgfpathlineto{\pgfqpoint{6.146020in}{1.367624in}}%
\pgfpathlineto{\pgfqpoint{6.144570in}{1.368636in}}%
\pgfpathlineto{\pgfqpoint{6.142191in}{1.368410in}}%
\pgfpathlineto{\pgfqpoint{6.139261in}{1.367038in}}%
\pgfpathlineto{\pgfqpoint{6.136508in}{1.364772in}}%
\pgfpathlineto{\pgfqpoint{6.135027in}{1.361990in}}%
\pgfpathlineto{\pgfqpoint{6.135762in}{1.359244in}}%
\pgfpathlineto{\pgfqpoint{6.138434in}{1.357372in}}%
\pgfpathlineto{\pgfqpoint{6.141543in}{1.357281in}}%
\pgfpathlineto{\pgfqpoint{6.143899in}{1.359036in}}%
\pgfpathlineto{\pgfqpoint{6.145265in}{1.361724in}}%
\pgfpathlineto{\pgfqpoint{6.145700in}{1.364380in}}%
\pgfpathlineto{\pgfqpoint{6.145202in}{1.366373in}}%
\pgfpathlineto{\pgfqpoint{6.143763in}{1.367371in}}%
\pgfpathlineto{\pgfqpoint{6.141490in}{1.367265in}}%
\pgfpathlineto{\pgfqpoint{6.138712in}{1.366135in}}%
\pgfpathlineto{\pgfqpoint{6.136101in}{1.364203in}}%
\pgfpathlineto{\pgfqpoint{6.134744in}{1.361801in}}%
\pgfpathlineto{\pgfqpoint{6.135658in}{1.359409in}}%
\pgfpathlineto{\pgfqpoint{6.138546in}{1.357755in}}%
\pgfpathlineto{\pgfqpoint{6.141772in}{1.357639in}}%
\pgfpathlineto{\pgfqpoint{6.144098in}{1.359166in}}%
\pgfpathlineto{\pgfqpoint{6.145332in}{1.361538in}}%
\pgfpathlineto{\pgfqpoint{6.145605in}{1.363867in}}%
\pgfpathlineto{\pgfqpoint{6.144968in}{1.365577in}}%
\pgfpathlineto{\pgfqpoint{6.143444in}{1.366377in}}%
\pgfpathlineto{\pgfqpoint{6.141147in}{1.366193in}}%
\pgfpathlineto{\pgfqpoint{6.135755in}{1.363405in}}%
\pgfpathlineto{\pgfqpoint{6.134291in}{1.361360in}}%
\pgfpathlineto{\pgfqpoint{6.135093in}{1.359427in}}%
\pgfpathlineto{\pgfqpoint{6.138045in}{1.358186in}}%
\pgfpathlineto{\pgfqpoint{6.141511in}{1.358249in}}%
\pgfpathlineto{\pgfqpoint{6.144101in}{1.359701in}}%
\pgfpathlineto{\pgfqpoint{6.145503in}{1.361881in}}%
\pgfpathlineto{\pgfqpoint{6.145843in}{1.363983in}}%
\pgfpathlineto{\pgfqpoint{6.145217in}{1.365463in}}%
\pgfpathlineto{\pgfqpoint{6.143688in}{1.366055in}}%
\pgfpathlineto{\pgfqpoint{6.138648in}{1.364553in}}%
\pgfpathlineto{\pgfqpoint{6.136015in}{1.362832in}}%
\pgfpathlineto{\pgfqpoint{6.134447in}{1.360898in}}%
\pgfpathlineto{\pgfqpoint{6.135017in}{1.359207in}}%
\pgfpathlineto{\pgfqpoint{6.137722in}{1.358285in}}%
\pgfpathlineto{\pgfqpoint{6.141087in}{1.358572in}}%
\pgfpathlineto{\pgfqpoint{6.143758in}{1.360063in}}%
\pgfpathlineto{\pgfqpoint{6.145329in}{1.362180in}}%
\pgfpathlineto{\pgfqpoint{6.145850in}{1.364202in}}%
\pgfpathlineto{\pgfqpoint{6.145397in}{1.365614in}}%
\pgfpathlineto{\pgfqpoint{6.144035in}{1.366149in}}%
\pgfpathlineto{\pgfqpoint{6.139320in}{1.364545in}}%
\pgfpathlineto{\pgfqpoint{6.136812in}{1.362756in}}%
\pgfpathlineto{\pgfqpoint{6.135259in}{1.360729in}}%
\pgfpathlineto{\pgfqpoint{6.135631in}{1.358936in}}%
\pgfpathlineto{\pgfqpoint{6.137957in}{1.357979in}}%
\pgfpathlineto{\pgfqpoint{6.140927in}{1.358331in}}%
\pgfpathlineto{\pgfqpoint{6.143339in}{1.359895in}}%
\pgfpathlineto{\pgfqpoint{6.145415in}{1.364083in}}%
\pgfpathlineto{\pgfqpoint{6.145116in}{1.365537in}}%
\pgfpathlineto{\pgfqpoint{6.143970in}{1.366142in}}%
\pgfpathlineto{\pgfqpoint{6.142099in}{1.365829in}}%
\pgfpathlineto{\pgfqpoint{6.137579in}{1.362920in}}%
\pgfpathlineto{\pgfqpoint{6.136258in}{1.360836in}}%
\pgfpathlineto{\pgfqpoint{6.136615in}{1.358872in}}%
\pgfpathlineto{\pgfqpoint{6.138596in}{1.357669in}}%
\pgfpathlineto{\pgfqpoint{6.141080in}{1.357840in}}%
\pgfpathlineto{\pgfqpoint{6.143068in}{1.359327in}}%
\pgfpathlineto{\pgfqpoint{6.144768in}{1.363474in}}%
\pgfpathlineto{\pgfqpoint{6.144487in}{1.364971in}}%
\pgfpathlineto{\pgfqpoint{6.143452in}{1.365674in}}%
\pgfpathlineto{\pgfqpoint{6.141759in}{1.365503in}}%
\pgfpathlineto{\pgfqpoint{6.137734in}{1.362910in}}%
\pgfpathlineto{\pgfqpoint{6.136704in}{1.360933in}}%
\pgfpathlineto{\pgfqpoint{6.137256in}{1.358987in}}%
\pgfpathlineto{\pgfqpoint{6.139173in}{1.357677in}}%
\pgfpathlineto{\pgfqpoint{6.141384in}{1.357648in}}%
\pgfpathlineto{\pgfqpoint{6.143056in}{1.358930in}}%
\pgfpathlineto{\pgfqpoint{6.144329in}{1.362770in}}%
\pgfpathlineto{\pgfqpoint{6.143963in}{1.364197in}}%
\pgfpathlineto{\pgfqpoint{6.142919in}{1.364906in}}%
\pgfpathlineto{\pgfqpoint{6.139265in}{1.363997in}}%
\pgfpathlineto{\pgfqpoint{6.137388in}{1.362594in}}%
\pgfpathlineto{\pgfqpoint{6.136434in}{1.360852in}}%
\pgfpathlineto{\pgfqpoint{6.137131in}{1.359123in}}%
\pgfpathlineto{\pgfqpoint{6.139238in}{1.357938in}}%
\pgfpathlineto{\pgfqpoint{6.141567in}{1.357883in}}%
\pgfpathlineto{\pgfqpoint{6.143238in}{1.359017in}}%
\pgfpathlineto{\pgfqpoint{6.144312in}{1.362456in}}%
\pgfpathlineto{\pgfqpoint{6.143837in}{1.363701in}}%
\pgfpathlineto{\pgfqpoint{6.142712in}{1.364280in}}%
\pgfpathlineto{\pgfqpoint{6.138989in}{1.363350in}}%
\pgfpathlineto{\pgfqpoint{6.136020in}{1.360577in}}%
\pgfpathlineto{\pgfqpoint{6.136651in}{1.359157in}}%
\pgfpathlineto{\pgfqpoint{6.138849in}{1.358255in}}%
\pgfpathlineto{\pgfqpoint{6.141397in}{1.358323in}}%
\pgfpathlineto{\pgfqpoint{6.143289in}{1.359417in}}%
\pgfpathlineto{\pgfqpoint{6.144543in}{1.362595in}}%
\pgfpathlineto{\pgfqpoint{6.144065in}{1.363687in}}%
\pgfpathlineto{\pgfqpoint{6.142916in}{1.364119in}}%
\pgfpathlineto{\pgfqpoint{6.139150in}{1.362989in}}%
\pgfpathlineto{\pgfqpoint{6.136049in}{1.360265in}}%
\pgfpathlineto{\pgfqpoint{6.136516in}{1.359010in}}%
\pgfpathlineto{\pgfqpoint{6.138567in}{1.358337in}}%
\pgfpathlineto{\pgfqpoint{6.141086in}{1.358573in}}%
\pgfpathlineto{\pgfqpoint{6.143074in}{1.359707in}}%
\pgfpathlineto{\pgfqpoint{6.144611in}{1.362819in}}%
\pgfpathlineto{\pgfqpoint{6.144254in}{1.363873in}}%
\pgfpathlineto{\pgfqpoint{6.143213in}{1.364266in}}%
\pgfpathlineto{\pgfqpoint{6.139635in}{1.363038in}}%
\pgfpathlineto{\pgfqpoint{6.136589in}{1.360153in}}%
\pgfpathlineto{\pgfqpoint{6.136908in}{1.358809in}}%
\pgfpathlineto{\pgfqpoint{6.138699in}{1.358107in}}%
\pgfpathlineto{\pgfqpoint{6.140957in}{1.358400in}}%
\pgfpathlineto{\pgfqpoint{6.142780in}{1.359605in}}%
\pgfpathlineto{\pgfqpoint{6.144336in}{1.362793in}}%
\pgfpathlineto{\pgfqpoint{6.144095in}{1.363891in}}%
\pgfpathlineto{\pgfqpoint{6.143206in}{1.364340in}}%
\pgfpathlineto{\pgfqpoint{6.139994in}{1.363205in}}%
\pgfpathlineto{\pgfqpoint{6.137307in}{1.360252in}}%
\pgfpathlineto{\pgfqpoint{6.137609in}{1.358756in}}%
\pgfpathlineto{\pgfqpoint{6.139152in}{1.357857in}}%
\pgfpathlineto{\pgfqpoint{6.141064in}{1.358018in}}%
\pgfpathlineto{\pgfqpoint{6.143522in}{1.360815in}}%
\pgfpathlineto{\pgfqpoint{6.143651in}{1.363526in}}%
\pgfpathlineto{\pgfqpoint{6.142841in}{1.364057in}}%
\pgfpathlineto{\pgfqpoint{6.139909in}{1.363142in}}%
\pgfpathlineto{\pgfqpoint{6.137629in}{1.360350in}}%
\pgfpathlineto{\pgfqpoint{6.138081in}{1.358846in}}%
\pgfpathlineto{\pgfqpoint{6.139582in}{1.357847in}}%
\pgfpathlineto{\pgfqpoint{6.141296in}{1.357852in}}%
\pgfpathlineto{\pgfqpoint{6.143333in}{1.360387in}}%
\pgfpathlineto{\pgfqpoint{6.143269in}{1.362976in}}%
\pgfpathlineto{\pgfqpoint{6.142444in}{1.363520in}}%
\pgfpathlineto{\pgfqpoint{6.139578in}{1.362788in}}%
\pgfpathlineto{\pgfqpoint{6.137393in}{1.360313in}}%
\pgfpathlineto{\pgfqpoint{6.137969in}{1.358959in}}%
\pgfpathlineto{\pgfqpoint{6.139633in}{1.358044in}}%
\pgfpathlineto{\pgfqpoint{6.141449in}{1.358023in}}%
\pgfpathlineto{\pgfqpoint{6.143428in}{1.360308in}}%
\pgfpathlineto{\pgfqpoint{6.143184in}{1.362620in}}%
\pgfpathlineto{\pgfqpoint{6.140948in}{1.362949in}}%
\pgfpathlineto{\pgfqpoint{6.137832in}{1.361313in}}%
\pgfpathlineto{\pgfqpoint{6.137028in}{1.360121in}}%
\pgfpathlineto{\pgfqpoint{6.137562in}{1.358999in}}%
\pgfpathlineto{\pgfqpoint{6.139319in}{1.358295in}}%
\pgfpathlineto{\pgfqpoint{6.141330in}{1.358367in}}%
\pgfpathlineto{\pgfqpoint{6.143601in}{1.360543in}}%
\pgfpathlineto{\pgfqpoint{6.143383in}{1.362637in}}%
\pgfpathlineto{\pgfqpoint{6.141086in}{1.362755in}}%
\pgfpathlineto{\pgfqpoint{6.137896in}{1.361033in}}%
\pgfpathlineto{\pgfqpoint{6.137006in}{1.359887in}}%
\pgfpathlineto{\pgfqpoint{6.137414in}{1.358893in}}%
\pgfpathlineto{\pgfqpoint{6.139074in}{1.358369in}}%
\pgfpathlineto{\pgfqpoint{6.141086in}{1.358573in}}%
\pgfpathlineto{\pgfqpoint{6.143576in}{1.360774in}}%
\pgfpathlineto{\pgfqpoint{6.143563in}{1.362820in}}%
\pgfpathlineto{\pgfqpoint{6.141408in}{1.362868in}}%
\pgfpathlineto{\pgfqpoint{6.138309in}{1.361030in}}%
\pgfpathlineto{\pgfqpoint{6.137401in}{1.359803in}}%
\pgfpathlineto{\pgfqpoint{6.137689in}{1.358732in}}%
\pgfpathlineto{\pgfqpoint{6.139154in}{1.358186in}}%
\pgfpathlineto{\pgfqpoint{6.140975in}{1.358442in}}%
\pgfpathlineto{\pgfqpoint{6.143334in}{1.360751in}}%
\pgfpathlineto{\pgfqpoint{6.143465in}{1.362875in}}%
\pgfpathlineto{\pgfqpoint{6.141557in}{1.363011in}}%
\pgfpathlineto{\pgfqpoint{6.138750in}{1.361182in}}%
\pgfpathlineto{\pgfqpoint{6.137959in}{1.359888in}}%
\pgfpathlineto{\pgfqpoint{6.138229in}{1.358684in}}%
\pgfpathlineto{\pgfqpoint{6.139499in}{1.357974in}}%
\pgfpathlineto{\pgfqpoint{6.141054in}{1.358130in}}%
\pgfpathlineto{\pgfqpoint{6.143038in}{1.360424in}}%
\pgfpathlineto{\pgfqpoint{6.143126in}{1.362617in}}%
\pgfpathlineto{\pgfqpoint{6.141373in}{1.362908in}}%
\pgfpathlineto{\pgfqpoint{6.138835in}{1.361237in}}%
\pgfpathlineto{\pgfqpoint{6.138215in}{1.359980in}}%
\pgfpathlineto{\pgfqpoint{6.138605in}{1.358756in}}%
\pgfpathlineto{\pgfqpoint{6.139842in}{1.357956in}}%
\pgfpathlineto{\pgfqpoint{6.141239in}{1.357984in}}%
\pgfpathlineto{\pgfqpoint{6.142891in}{1.360082in}}%
\pgfpathlineto{\pgfqpoint{6.142824in}{1.362194in}}%
\pgfpathlineto{\pgfqpoint{6.141071in}{1.362557in}}%
\pgfpathlineto{\pgfqpoint{6.138586in}{1.361095in}}%
\pgfpathlineto{\pgfqpoint{6.138013in}{1.359966in}}%
\pgfpathlineto{\pgfqpoint{6.138510in}{1.358854in}}%
\pgfpathlineto{\pgfqpoint{6.141373in}{1.358114in}}%
\pgfpathlineto{\pgfqpoint{6.142977in}{1.360017in}}%
\pgfpathlineto{\pgfqpoint{6.142760in}{1.361916in}}%
\pgfpathlineto{\pgfqpoint{6.140901in}{1.362173in}}%
\pgfpathlineto{\pgfqpoint{6.138332in}{1.360810in}}%
\pgfpathlineto{\pgfqpoint{6.137689in}{1.359822in}}%
\pgfpathlineto{\pgfqpoint{6.138159in}{1.358896in}}%
\pgfpathlineto{\pgfqpoint{6.141286in}{1.358397in}}%
\pgfpathlineto{\pgfqpoint{6.143137in}{1.360217in}}%
\pgfpathlineto{\pgfqpoint{6.142933in}{1.361944in}}%
\pgfpathlineto{\pgfqpoint{6.141010in}{1.362026in}}%
\pgfpathlineto{\pgfqpoint{6.138362in}{1.360587in}}%
\pgfpathlineto{\pgfqpoint{6.137643in}{1.359636in}}%
\pgfpathlineto{\pgfqpoint{6.138011in}{1.358815in}}%
\pgfpathlineto{\pgfqpoint{6.141086in}{1.358574in}}%
\pgfpathlineto{\pgfqpoint{6.143135in}{1.360421in}}%
\pgfpathlineto{\pgfqpoint{6.143099in}{1.362115in}}%
\pgfpathlineto{\pgfqpoint{6.141280in}{1.362136in}}%
\pgfpathlineto{\pgfqpoint{6.138691in}{1.360590in}}%
\pgfpathlineto{\pgfqpoint{6.138216in}{1.358681in}}%
\pgfpathlineto{\pgfqpoint{6.140988in}{1.358471in}}%
\pgfpathlineto{\pgfqpoint{6.142948in}{1.360418in}}%
\pgfpathlineto{\pgfqpoint{6.143036in}{1.362185in}}%
\pgfpathlineto{\pgfqpoint{6.141416in}{1.362278in}}%
\pgfpathlineto{\pgfqpoint{6.139056in}{1.360728in}}%
\pgfpathlineto{\pgfqpoint{6.138653in}{1.358635in}}%
\pgfpathlineto{\pgfqpoint{6.139737in}{1.358055in}}%
\pgfpathlineto{\pgfqpoint{6.142078in}{1.359031in}}%
\pgfpathlineto{\pgfqpoint{6.142937in}{1.361222in}}%
\pgfpathlineto{\pgfqpoint{6.142191in}{1.362337in}}%
\pgfpathlineto{\pgfqpoint{6.140153in}{1.361668in}}%
\pgfpathlineto{\pgfqpoint{6.138620in}{1.359725in}}%
\pgfpathlineto{\pgfqpoint{6.138968in}{1.358695in}}%
\pgfpathlineto{\pgfqpoint{6.141200in}{1.358075in}}%
\pgfpathlineto{\pgfqpoint{6.142584in}{1.359870in}}%
\pgfpathlineto{\pgfqpoint{6.142515in}{1.361650in}}%
\pgfpathlineto{\pgfqpoint{6.141016in}{1.361941in}}%
\pgfpathlineto{\pgfqpoint{6.138914in}{1.360685in}}%
\pgfpathlineto{\pgfqpoint{6.138445in}{1.359723in}}%
\pgfpathlineto{\pgfqpoint{6.138888in}{1.358781in}}%
\pgfpathlineto{\pgfqpoint{6.141319in}{1.358179in}}%
\pgfpathlineto{\pgfqpoint{6.142661in}{1.359813in}}%
\pgfpathlineto{\pgfqpoint{6.142462in}{1.361422in}}%
\pgfpathlineto{\pgfqpoint{6.140868in}{1.361627in}}%
\pgfpathlineto{\pgfqpoint{6.138685in}{1.360456in}}%
\pgfpathlineto{\pgfqpoint{6.138156in}{1.359611in}}%
\pgfpathlineto{\pgfqpoint{6.138582in}{1.358823in}}%
\pgfpathlineto{\pgfqpoint{6.141254in}{1.358418in}}%
\pgfpathlineto{\pgfqpoint{6.142807in}{1.359986in}}%
\pgfpathlineto{\pgfqpoint{6.142614in}{1.361451in}}%
\pgfpathlineto{\pgfqpoint{6.140955in}{1.361508in}}%
\pgfpathlineto{\pgfqpoint{6.139788in}{1.361004in}}%
\pgfpathlineto{\pgfqpoint{6.139788in}{1.361004in}}%
\pgfusepath{stroke}%
\end{pgfscope}%
\begin{pgfscope}%
\pgfpathrectangle{\pgfqpoint{4.577333in}{0.150000in}}{\pgfqpoint{4.224218in}{2.565000in}}%
\pgfusepath{clip}%
\pgfsetbuttcap%
\pgfsetroundjoin%
\definecolor{currentfill}{rgb}{0.501961,0.000000,0.501961}%
\pgfsetfillcolor{currentfill}%
\pgfsetlinewidth{1.505625pt}%
\definecolor{currentstroke}{rgb}{0.501961,0.000000,0.501961}%
\pgfsetstrokecolor{currentstroke}%
\pgfsetdash{}{0pt}%
\pgfsys@defobject{currentmarker}{\pgfqpoint{-0.017010in}{-0.017010in}}{\pgfqpoint{0.017010in}{0.017010in}}{%
\pgfpathmoveto{\pgfqpoint{-0.017010in}{0.000000in}}%
\pgfpathlineto{\pgfqpoint{0.017010in}{0.000000in}}%
\pgfpathmoveto{\pgfqpoint{0.000000in}{-0.017010in}}%
\pgfpathlineto{\pgfqpoint{0.000000in}{0.017010in}}%
\pgfusepath{stroke,fill}%
}%
\begin{pgfscope}%
\pgfsys@transformshift{6.864334in}{1.402646in}%
\pgfsys@useobject{currentmarker}{}%
\end{pgfscope}%
\end{pgfscope}%
\begin{pgfscope}%
\pgfpathrectangle{\pgfqpoint{4.577333in}{0.150000in}}{\pgfqpoint{4.224218in}{2.565000in}}%
\pgfusepath{clip}%
\pgfsetbuttcap%
\pgfsetroundjoin%
\definecolor{currentfill}{rgb}{0.501961,0.000000,0.501961}%
\pgfsetfillcolor{currentfill}%
\pgfsetlinewidth{1.003750pt}%
\definecolor{currentstroke}{rgb}{0.501961,0.000000,0.501961}%
\pgfsetstrokecolor{currentstroke}%
\pgfsetdash{}{0pt}%
\pgfsys@defobject{currentmarker}{\pgfqpoint{-0.016178in}{-0.013762in}}{\pgfqpoint{0.016178in}{0.017010in}}{%
\pgfpathmoveto{\pgfqpoint{0.000000in}{0.017010in}}%
\pgfpathlineto{\pgfqpoint{-0.003819in}{0.005256in}}%
\pgfpathlineto{\pgfqpoint{-0.016178in}{0.005256in}}%
\pgfpathlineto{\pgfqpoint{-0.006179in}{-0.002008in}}%
\pgfpathlineto{\pgfqpoint{-0.009998in}{-0.013762in}}%
\pgfpathlineto{\pgfqpoint{-0.000000in}{-0.006497in}}%
\pgfpathlineto{\pgfqpoint{0.009998in}{-0.013762in}}%
\pgfpathlineto{\pgfqpoint{0.006179in}{-0.002008in}}%
\pgfpathlineto{\pgfqpoint{0.016178in}{0.005256in}}%
\pgfpathlineto{\pgfqpoint{0.003819in}{0.005256in}}%
\pgfpathlineto{\pgfqpoint{0.000000in}{0.017010in}}%
\pgfpathclose%
\pgfusepath{stroke,fill}%
}%
\begin{pgfscope}%
\pgfsys@transformshift{5.558561in}{1.110103in}%
\pgfsys@useobject{currentmarker}{}%
\end{pgfscope}%
\end{pgfscope}%
\begin{pgfscope}%
\pgfpathrectangle{\pgfqpoint{4.577333in}{0.150000in}}{\pgfqpoint{4.224218in}{2.565000in}}%
\pgfusepath{clip}%
\pgfsetbuttcap%
\pgfsetroundjoin%
\definecolor{currentfill}{rgb}{0.000000,0.000000,0.000000}%
\pgfsetfillcolor{currentfill}%
\pgfsetlinewidth{1.505625pt}%
\definecolor{currentstroke}{rgb}{0.000000,0.000000,0.000000}%
\pgfsetstrokecolor{currentstroke}%
\pgfsetdash{}{0pt}%
\pgfsys@defobject{currentmarker}{\pgfqpoint{-0.017010in}{-0.017010in}}{\pgfqpoint{0.017010in}{0.017010in}}{%
\pgfpathmoveto{\pgfqpoint{-0.017010in}{0.000000in}}%
\pgfpathlineto{\pgfqpoint{0.017010in}{0.000000in}}%
\pgfpathmoveto{\pgfqpoint{0.000000in}{-0.017010in}}%
\pgfpathlineto{\pgfqpoint{0.000000in}{0.017010in}}%
\pgfusepath{stroke,fill}%
}%
\begin{pgfscope}%
\pgfsys@transformshift{6.864334in}{1.402646in}%
\pgfsys@useobject{currentmarker}{}%
\end{pgfscope}%
\end{pgfscope}%
\begin{pgfscope}%
\pgfpathrectangle{\pgfqpoint{4.577333in}{0.150000in}}{\pgfqpoint{4.224218in}{2.565000in}}%
\pgfusepath{clip}%
\pgfsetbuttcap%
\pgfsetroundjoin%
\definecolor{currentfill}{rgb}{0.000000,0.000000,0.000000}%
\pgfsetfillcolor{currentfill}%
\pgfsetlinewidth{1.003750pt}%
\definecolor{currentstroke}{rgb}{0.000000,0.000000,0.000000}%
\pgfsetstrokecolor{currentstroke}%
\pgfsetdash{}{0pt}%
\pgfsys@defobject{currentmarker}{\pgfqpoint{-0.016178in}{-0.013762in}}{\pgfqpoint{0.016178in}{0.017010in}}{%
\pgfpathmoveto{\pgfqpoint{0.000000in}{0.017010in}}%
\pgfpathlineto{\pgfqpoint{-0.003819in}{0.005256in}}%
\pgfpathlineto{\pgfqpoint{-0.016178in}{0.005256in}}%
\pgfpathlineto{\pgfqpoint{-0.006179in}{-0.002008in}}%
\pgfpathlineto{\pgfqpoint{-0.009998in}{-0.013762in}}%
\pgfpathlineto{\pgfqpoint{-0.000000in}{-0.006497in}}%
\pgfpathlineto{\pgfqpoint{0.009998in}{-0.013762in}}%
\pgfpathlineto{\pgfqpoint{0.006179in}{-0.002008in}}%
\pgfpathlineto{\pgfqpoint{0.016178in}{0.005256in}}%
\pgfpathlineto{\pgfqpoint{0.003819in}{0.005256in}}%
\pgfpathlineto{\pgfqpoint{0.000000in}{0.017010in}}%
\pgfpathclose%
\pgfusepath{stroke,fill}%
}%
\begin{pgfscope}%
\pgfsys@transformshift{6.139788in}{1.361004in}%
\pgfsys@useobject{currentmarker}{}%
\end{pgfscope}%
\end{pgfscope}%
\begin{pgfscope}%
\pgfsetrectcap%
\pgfsetmiterjoin%
\pgfsetlinewidth{0.803000pt}%
\definecolor{currentstroke}{rgb}{0.000000,0.000000,0.000000}%
\pgfsetstrokecolor{currentstroke}%
\pgfsetdash{}{0pt}%
\pgfpathmoveto{\pgfqpoint{4.577333in}{0.150000in}}%
\pgfpathlineto{\pgfqpoint{4.577333in}{2.715000in}}%
\pgfusepath{stroke}%
\end{pgfscope}%
\begin{pgfscope}%
\pgfsetrectcap%
\pgfsetmiterjoin%
\pgfsetlinewidth{0.803000pt}%
\definecolor{currentstroke}{rgb}{0.000000,0.000000,0.000000}%
\pgfsetstrokecolor{currentstroke}%
\pgfsetdash{}{0pt}%
\pgfpathmoveto{\pgfqpoint{8.801551in}{0.150000in}}%
\pgfpathlineto{\pgfqpoint{8.801551in}{2.715000in}}%
\pgfusepath{stroke}%
\end{pgfscope}%
\begin{pgfscope}%
\pgfsetrectcap%
\pgfsetmiterjoin%
\pgfsetlinewidth{0.803000pt}%
\definecolor{currentstroke}{rgb}{0.000000,0.000000,0.000000}%
\pgfsetstrokecolor{currentstroke}%
\pgfsetdash{}{0pt}%
\pgfpathmoveto{\pgfqpoint{4.577333in}{0.150000in}}%
\pgfpathlineto{\pgfqpoint{8.801551in}{0.150000in}}%
\pgfusepath{stroke}%
\end{pgfscope}%
\begin{pgfscope}%
\pgfsetrectcap%
\pgfsetmiterjoin%
\pgfsetlinewidth{0.803000pt}%
\definecolor{currentstroke}{rgb}{0.000000,0.000000,0.000000}%
\pgfsetstrokecolor{currentstroke}%
\pgfsetdash{}{0pt}%
\pgfpathmoveto{\pgfqpoint{4.577333in}{2.715000in}}%
\pgfpathlineto{\pgfqpoint{8.801551in}{2.715000in}}%
\pgfusepath{stroke}%
\end{pgfscope}%
\begin{pgfscope}%
\definecolor{textcolor}{rgb}{0.000000,0.000000,0.000000}%
\pgfsetstrokecolor{textcolor}%
\pgfsetfillcolor{textcolor}%
\pgftext[x=8.718987in,y=0.395393in,,base]{\color{textcolor}\sffamily\fontsize{10.000000}{12.000000}\selectfont 0.0}%
\end{pgfscope}%
\begin{pgfscope}%
\definecolor{textcolor}{rgb}{0.000000,0.000000,0.000000}%
\pgfsetstrokecolor{textcolor}%
\pgfsetfillcolor{textcolor}%
\pgftext[x=8.526977in,y=0.617529in,,base]{\color{textcolor}\sffamily\fontsize{10.000000}{12.000000}\selectfont 0.1}%
\end{pgfscope}%
\begin{pgfscope}%
\definecolor{textcolor}{rgb}{0.000000,0.000000,0.000000}%
\pgfsetstrokecolor{textcolor}%
\pgfsetfillcolor{textcolor}%
\pgftext[x=8.334967in,y=0.839664in,,base]{\color{textcolor}\sffamily\fontsize{10.000000}{12.000000}\selectfont 0.2}%
\end{pgfscope}%
\begin{pgfscope}%
\definecolor{textcolor}{rgb}{0.000000,0.000000,0.000000}%
\pgfsetstrokecolor{textcolor}%
\pgfsetfillcolor{textcolor}%
\pgftext[x=8.142957in,y=1.061800in,,base]{\color{textcolor}\sffamily\fontsize{10.000000}{12.000000}\selectfont 0.3}%
\end{pgfscope}%
\begin{pgfscope}%
\definecolor{textcolor}{rgb}{0.000000,0.000000,0.000000}%
\pgfsetstrokecolor{textcolor}%
\pgfsetfillcolor{textcolor}%
\pgftext[x=7.950947in,y=1.283935in,,base]{\color{textcolor}\sffamily\fontsize{10.000000}{12.000000}\selectfont 0.4}%
\end{pgfscope}%
\begin{pgfscope}%
\definecolor{textcolor}{rgb}{0.000000,0.000000,0.000000}%
\pgfsetstrokecolor{textcolor}%
\pgfsetfillcolor{textcolor}%
\pgftext[x=7.758937in,y=1.506071in,,base]{\color{textcolor}\sffamily\fontsize{10.000000}{12.000000}\selectfont 0.5}%
\end{pgfscope}%
\begin{pgfscope}%
\definecolor{textcolor}{rgb}{0.000000,0.000000,0.000000}%
\pgfsetstrokecolor{textcolor}%
\pgfsetfillcolor{textcolor}%
\pgftext[x=7.566927in,y=1.728206in,,base]{\color{textcolor}\sffamily\fontsize{10.000000}{12.000000}\selectfont 0.6}%
\end{pgfscope}%
\begin{pgfscope}%
\definecolor{textcolor}{rgb}{0.000000,0.000000,0.000000}%
\pgfsetstrokecolor{textcolor}%
\pgfsetfillcolor{textcolor}%
\pgftext[x=7.374918in,y=1.950342in,,base]{\color{textcolor}\sffamily\fontsize{10.000000}{12.000000}\selectfont 0.7}%
\end{pgfscope}%
\begin{pgfscope}%
\definecolor{textcolor}{rgb}{0.000000,0.000000,0.000000}%
\pgfsetstrokecolor{textcolor}%
\pgfsetfillcolor{textcolor}%
\pgftext[x=7.182908in,y=2.172477in,,base]{\color{textcolor}\sffamily\fontsize{10.000000}{12.000000}\selectfont 0.8}%
\end{pgfscope}%
\begin{pgfscope}%
\definecolor{textcolor}{rgb}{0.000000,0.000000,0.000000}%
\pgfsetstrokecolor{textcolor}%
\pgfsetfillcolor{textcolor}%
\pgftext[x=6.990898in,y=2.394613in,,base]{\color{textcolor}\sffamily\fontsize{10.000000}{12.000000}\selectfont 0.9}%
\end{pgfscope}%
\begin{pgfscope}%
\definecolor{textcolor}{rgb}{0.000000,0.000000,0.000000}%
\pgfsetstrokecolor{textcolor}%
\pgfsetfillcolor{textcolor}%
\pgftext[x=6.798888in,y=2.616748in,,base]{\color{textcolor}\sffamily\fontsize{10.000000}{12.000000}\selectfont 1.0}%
\end{pgfscope}%
\begin{pgfscope}%
\definecolor{textcolor}{rgb}{0.000000,0.000000,0.000000}%
\pgfsetstrokecolor{textcolor}%
\pgfsetfillcolor{textcolor}%
\pgftext[x=4.721340in,y=0.439820in,,base]{\color{textcolor}\sffamily\fontsize{10.000000}{12.000000}\selectfont 1.0}%
\end{pgfscope}%
\begin{pgfscope}%
\definecolor{textcolor}{rgb}{0.000000,0.000000,0.000000}%
\pgfsetstrokecolor{textcolor}%
\pgfsetfillcolor{textcolor}%
\pgftext[x=4.913350in,y=0.661956in,,base]{\color{textcolor}\sffamily\fontsize{10.000000}{12.000000}\selectfont 0.9}%
\end{pgfscope}%
\begin{pgfscope}%
\definecolor{textcolor}{rgb}{0.000000,0.000000,0.000000}%
\pgfsetstrokecolor{textcolor}%
\pgfsetfillcolor{textcolor}%
\pgftext[x=5.105360in,y=0.884091in,,base]{\color{textcolor}\sffamily\fontsize{10.000000}{12.000000}\selectfont 0.8}%
\end{pgfscope}%
\begin{pgfscope}%
\definecolor{textcolor}{rgb}{0.000000,0.000000,0.000000}%
\pgfsetstrokecolor{textcolor}%
\pgfsetfillcolor{textcolor}%
\pgftext[x=5.297370in,y=1.106227in,,base]{\color{textcolor}\sffamily\fontsize{10.000000}{12.000000}\selectfont 0.7}%
\end{pgfscope}%
\begin{pgfscope}%
\definecolor{textcolor}{rgb}{0.000000,0.000000,0.000000}%
\pgfsetstrokecolor{textcolor}%
\pgfsetfillcolor{textcolor}%
\pgftext[x=5.489380in,y=1.328362in,,base]{\color{textcolor}\sffamily\fontsize{10.000000}{12.000000}\selectfont 0.6}%
\end{pgfscope}%
\begin{pgfscope}%
\definecolor{textcolor}{rgb}{0.000000,0.000000,0.000000}%
\pgfsetstrokecolor{textcolor}%
\pgfsetfillcolor{textcolor}%
\pgftext[x=5.681390in,y=1.550498in,,base]{\color{textcolor}\sffamily\fontsize{10.000000}{12.000000}\selectfont 0.5}%
\end{pgfscope}%
\begin{pgfscope}%
\definecolor{textcolor}{rgb}{0.000000,0.000000,0.000000}%
\pgfsetstrokecolor{textcolor}%
\pgfsetfillcolor{textcolor}%
\pgftext[x=5.873400in,y=1.772633in,,base]{\color{textcolor}\sffamily\fontsize{10.000000}{12.000000}\selectfont 0.4}%
\end{pgfscope}%
\begin{pgfscope}%
\definecolor{textcolor}{rgb}{0.000000,0.000000,0.000000}%
\pgfsetstrokecolor{textcolor}%
\pgfsetfillcolor{textcolor}%
\pgftext[x=6.065410in,y=1.994769in,,base]{\color{textcolor}\sffamily\fontsize{10.000000}{12.000000}\selectfont 0.3}%
\end{pgfscope}%
\begin{pgfscope}%
\definecolor{textcolor}{rgb}{0.000000,0.000000,0.000000}%
\pgfsetstrokecolor{textcolor}%
\pgfsetfillcolor{textcolor}%
\pgftext[x=6.257420in,y=2.216904in,,base]{\color{textcolor}\sffamily\fontsize{10.000000}{12.000000}\selectfont 0.2}%
\end{pgfscope}%
\begin{pgfscope}%
\definecolor{textcolor}{rgb}{0.000000,0.000000,0.000000}%
\pgfsetstrokecolor{textcolor}%
\pgfsetfillcolor{textcolor}%
\pgftext[x=6.449430in,y=2.439040in,,base]{\color{textcolor}\sffamily\fontsize{10.000000}{12.000000}\selectfont 0.1}%
\end{pgfscope}%
\begin{pgfscope}%
\definecolor{textcolor}{rgb}{0.000000,0.000000,0.000000}%
\pgfsetstrokecolor{textcolor}%
\pgfsetfillcolor{textcolor}%
\pgftext[x=6.641440in,y=2.661175in,,base]{\color{textcolor}\sffamily\fontsize{10.000000}{12.000000}\selectfont 0.0}%
\end{pgfscope}%
\begin{pgfscope}%
\definecolor{textcolor}{rgb}{0.000000,0.000000,0.000000}%
\pgfsetstrokecolor{textcolor}%
\pgfsetfillcolor{textcolor}%
\pgftext[x=4.721340in,y=0.328753in,,base]{\color{textcolor}\sffamily\fontsize{10.000000}{12.000000}\selectfont 0.0}%
\end{pgfscope}%
\begin{pgfscope}%
\definecolor{textcolor}{rgb}{0.000000,0.000000,0.000000}%
\pgfsetstrokecolor{textcolor}%
\pgfsetfillcolor{textcolor}%
\pgftext[x=5.105360in,y=0.328753in,,base]{\color{textcolor}\sffamily\fontsize{10.000000}{12.000000}\selectfont 0.1}%
\end{pgfscope}%
\begin{pgfscope}%
\definecolor{textcolor}{rgb}{0.000000,0.000000,0.000000}%
\pgfsetstrokecolor{textcolor}%
\pgfsetfillcolor{textcolor}%
\pgftext[x=5.489380in,y=0.328753in,,base]{\color{textcolor}\sffamily\fontsize{10.000000}{12.000000}\selectfont 0.2}%
\end{pgfscope}%
\begin{pgfscope}%
\definecolor{textcolor}{rgb}{0.000000,0.000000,0.000000}%
\pgfsetstrokecolor{textcolor}%
\pgfsetfillcolor{textcolor}%
\pgftext[x=5.873400in,y=0.328753in,,base]{\color{textcolor}\sffamily\fontsize{10.000000}{12.000000}\selectfont 0.3}%
\end{pgfscope}%
\begin{pgfscope}%
\definecolor{textcolor}{rgb}{0.000000,0.000000,0.000000}%
\pgfsetstrokecolor{textcolor}%
\pgfsetfillcolor{textcolor}%
\pgftext[x=6.257420in,y=0.328753in,,base]{\color{textcolor}\sffamily\fontsize{10.000000}{12.000000}\selectfont 0.4}%
\end{pgfscope}%
\begin{pgfscope}%
\definecolor{textcolor}{rgb}{0.000000,0.000000,0.000000}%
\pgfsetstrokecolor{textcolor}%
\pgfsetfillcolor{textcolor}%
\pgftext[x=6.641440in,y=0.328753in,,base]{\color{textcolor}\sffamily\fontsize{10.000000}{12.000000}\selectfont 0.5}%
\end{pgfscope}%
\begin{pgfscope}%
\definecolor{textcolor}{rgb}{0.000000,0.000000,0.000000}%
\pgfsetstrokecolor{textcolor}%
\pgfsetfillcolor{textcolor}%
\pgftext[x=7.025460in,y=0.328753in,,base]{\color{textcolor}\sffamily\fontsize{10.000000}{12.000000}\selectfont 0.6}%
\end{pgfscope}%
\begin{pgfscope}%
\definecolor{textcolor}{rgb}{0.000000,0.000000,0.000000}%
\pgfsetstrokecolor{textcolor}%
\pgfsetfillcolor{textcolor}%
\pgftext[x=7.409479in,y=0.328753in,,base]{\color{textcolor}\sffamily\fontsize{10.000000}{12.000000}\selectfont 0.7}%
\end{pgfscope}%
\begin{pgfscope}%
\definecolor{textcolor}{rgb}{0.000000,0.000000,0.000000}%
\pgfsetstrokecolor{textcolor}%
\pgfsetfillcolor{textcolor}%
\pgftext[x=7.793499in,y=0.328753in,,base]{\color{textcolor}\sffamily\fontsize{10.000000}{12.000000}\selectfont 0.8}%
\end{pgfscope}%
\begin{pgfscope}%
\definecolor{textcolor}{rgb}{0.000000,0.000000,0.000000}%
\pgfsetstrokecolor{textcolor}%
\pgfsetfillcolor{textcolor}%
\pgftext[x=8.177519in,y=0.328753in,,base]{\color{textcolor}\sffamily\fontsize{10.000000}{12.000000}\selectfont 0.9}%
\end{pgfscope}%
\begin{pgfscope}%
\definecolor{textcolor}{rgb}{0.000000,0.000000,0.000000}%
\pgfsetstrokecolor{textcolor}%
\pgfsetfillcolor{textcolor}%
\pgftext[x=8.561539in,y=0.328753in,,base]{\color{textcolor}\sffamily\fontsize{10.000000}{12.000000}\selectfont 1.0}%
\end{pgfscope}%
\begin{pgfscope}%
\pgfpathrectangle{\pgfqpoint{4.577333in}{0.150000in}}{\pgfqpoint{4.224218in}{2.565000in}}%
\pgfusepath{clip}%
\pgfsetbuttcap%
\pgfsetroundjoin%
\definecolor{currentfill}{rgb}{1.000000,0.000000,0.000000}%
\pgfsetfillcolor{currentfill}%
\pgfsetlinewidth{1.505625pt}%
\definecolor{currentstroke}{rgb}{1.000000,0.000000,0.000000}%
\pgfsetstrokecolor{currentstroke}%
\pgfsetdash{}{0pt}%
\pgfsys@defobject{currentmarker}{\pgfqpoint{-0.013608in}{-0.017010in}}{\pgfqpoint{0.013608in}{0.008505in}}{%
\pgfpathmoveto{\pgfqpoint{0.000000in}{0.000000in}}%
\pgfpathlineto{\pgfqpoint{0.000000in}{-0.017010in}}%
\pgfpathmoveto{\pgfqpoint{0.000000in}{0.000000in}}%
\pgfpathlineto{\pgfqpoint{0.013608in}{0.008505in}}%
\pgfpathmoveto{\pgfqpoint{0.000000in}{0.000000in}}%
\pgfpathlineto{\pgfqpoint{-0.013608in}{0.008505in}}%
\pgfusepath{stroke,fill}%
}%
\begin{pgfscope}%
\pgfsys@transformshift{6.140842in}{1.358509in}%
\pgfsys@useobject{currentmarker}{}%
\end{pgfscope}%
\end{pgfscope}%
\begin{pgfscope}%
\definecolor{textcolor}{rgb}{0.000000,0.000000,0.000000}%
\pgfsetstrokecolor{textcolor}%
\pgfsetfillcolor{textcolor}%
\pgftext[x=6.689442in,y=2.798333in,,base]{\color{textcolor}\sffamily\fontsize{12.000000}{14.400000}\selectfont NeuRD - Alternate}%
\end{pgfscope}%
\end{pgfpicture}%
\makeatother%
\endgroup%
}
	\caption[Simultaneous gradient updates vs Alternate gradient updates updates]{NeuRD converges with alternating updates, but not with simultaneous updates.}
\end{figure}
We note that all the experiments in~\cite{hennesNeural2020} also employed alternating updates to
evaluate NeuRD and SPG.
A similar behavior of simultaneous gradient updates diverging was also observed
in~\cite{gidelVariational2020} for unconstrained min-max problems.
They note that the sequence of iterates under alternating updates have a bounded error, and thus
the average of the iterates converges to the solution.
Based on these observations, we use alternating udpates for all our tabular experiments.
Both of these are symmetric update schemes in the sense that both the players perform equal number
of updates every iteration.
Asymmetric updates have also been studied in a similar context and have been shown to improve the
speed of convergence~\cite{daskalakisTraining2018}.

\section{Extragradient and Optimistic Gradient}
 (\revdone{Since you say EG builds on
	 this (Arrow-Hurwicz) a few lines later, you need to tell the reader at least a bit more about what it is})
 (\revdone{For opt updates: explain what
	 has changed})
 (\revdone{Make sure you have defined the various notions of convergence somewhere, probably in 2 or 3})

While the NeuRD fix modifies the loss function to better adjust to changing dynamics in multiagent
settings, there is no inherent change to the learning procedure as such.
Iterative methods such as fictitous play, are part of a broader set of learning dynamics studied to
identify a simple sequence of iterative updates that each of the player in a multiagent setting can
follow to converge to some solution set like the set of equilibrium points.
It is common for these methods to borrow from first-order optimization
algorithms~\cite{beckFirstOrder2017} that have been studied extensively in varying contexts
including solving variational inequalities, saddle-point problems, and convex optimization.
In particular, saddle-point problems are of much interest due to their application in modeling
various problems including learning in games, generative models, and robust optimization problems.

The classic Arrow-Hurwicz (inexact Uwaza) method~\cite{arrowStudies1958} is a popular
gradient-based for saddle point problems that can be expressed as a sequence of updates:
%\begin{noindent}
\begin{equation}
	\label{eqn:arrowhurwicz}
	\begin{split}
		x_{k+1} = P_{\calX} (x - \nabla f_x(x_k, y_k)) \\
		x_{k+1} = P_{\calY} (y + \nabla f_y(x_{k+1}, y_k)) \\
	\end{split}
\end{equation}
%\end{noindent}

While first-order gradient based iterative methods like
the above work very well for function minimization problems, convergence for saddle point problems
is only possible under strong-convexity assumptions even in the unconstrained
case~\cite{heConvergence2022}.
However, extensions to the above have been proposed and analyzed for solving of variational
inequalities, and saddle point problems, that have better convergence guarantees under restrictive
assumptions.

\textbf{Extragradient Updates (EG):}
The Extragradient method was introduced by G.~M.~Korpelevich~\cite{korpelevichextragradient1976} as
a modification to the Arrow-Hurwicz method for solving convex-concave saddle point problems, and
variational inequality problems with strongly monotone operators.
The EG algorithm follows the sequence of updates as given below: %\begin{noindent}
\begin{equation}
	\label{eqn:eg}
	\begin{split}
		\bar{x}_{k} = P_{\calX}[x_k - \eta f(x_k, y_k)] \\ 
		\bar{y}_{k} = P_{\calY}[y_k + \eta f(x_k, y_k)] \\ 
		x_{k+1} = P_{\calX}[x_k - \eta f(\bar{x}_k, \bar{y}_k)] \\ 
		y_{k+1} = P_{\calY}[y_k + \eta f(\bar{x}_k, \bar{y}_k)]
	\end{split}
\end{equation}
% \end{noindent}

\textbf{Optimistic updates (OPT):} Optimistic updates are another modification
to the Arrow-Hurwicz gradient method proposed by L.~D.~Popov~\cite{popovmodification1980} for
solving convex-concave saddle point problems.
The algorithm follows the below sequence of updates: % \begin{noindent}
\begin{equation}
	\label{eqn:opt}
	\begin{split}
		\bar{x}_{k} = P_{\calX}[x_k - \eta f(\bar{x}_{k-1}, \bar{y}_{k-1})] \\ 
		\bar{y}_{k} = P_{\calY}[y_k + \eta f(\bar{x}_{k-1}, \bar{y}_{k-1})] \\ 
		x_{k+1} = P_{\calX}[x_k - \eta f(\bar{x}_k, \bar{y}_k)] \\
		y_{k+1} = P_{\calY}[y_k + \eta f(\bar{x}_k, \bar{y}_k)]
	\end{split}
\end{equation}
% \end{noindent}

Contrast to the EG method, optimistic updates reuse the gradients of the leading points of the
previous iteration $f(\bar{x}_{k-1}, \bar{y}_{k-1})$ instead of computing an \textit{extragradient}
in each iteration $f(\bar{x_{k}}, \bar{y_{k}})$ as done in~\ref{eqn:eg}.
Optimistic and Extragradient updates are also interpreted as a variants extrapolation methods where
EG uses an additional gradient computation, and Optimistic updates approximates this extrapolation
by reusing previous iteration's gradient.
Optimistic updates were studied in the context of online learning under the name \textit{Optimistic
	Mirror Descent (OMD)}~\cite{rakhlinOptimization2013}.
OMD when applied to saddle-point problems leads to two popularly studied variants - Optimistic
Gradient Descent Ascent (OGDA) and Optimistic Multiplicative Weight Updates (OMWU), that use
euclidean and entropic mirror maps respectively.

Due to their broad applicability, and theoretical grounding, convergence rates for these algorithms
have been studied under various settings.
% If $f$ is $L$-smooth, and $0 < \eta < 1/L$ is the step size, EG has asymptotic last-iterate
% convergence.
For VIs with strongly monotone operators or composite structures, EG has a linear last-iterate
convergence rate~\cite{tsenglinear1995}.
A linear last-iterate convergence for OGDA in bilinear games and an optimistic version of the Adam
optimizer was also given in~\cite{daskalakisTraining2018}.
~\cite{mertikopoulosOptimistic2019} also study the EG method for training GANs by establishing a framework of
\textit{coherence}, and show that EG has last-iterate convergence for coherent VI problems.
A unified analysis of both these algorithms was also presented~\cite{mokhtariUnified2020} studied
both these algorithms in a unified way as an approximation of the Proximal point method, and showed
that both these algorithms have linear last-iterate convergence for the strongly-convex
strongly-convex and bilinear cases.
More recently, these results have been extended to monotone VIs, constrained saddle point problems,
markov games, and extensive form games under various
conditions~\cite{weiLINEAR2021,cenFaster2022,caiTight2022,gorbunovExtragradient2022,gorbunovLastIterate2022}.
In our work, we incorporate extragradient and optimistic updates into the algorithms discussed in
the previous chapter and experimentally evaluate the improvement gains when applied to two-player
zero-sum games.

